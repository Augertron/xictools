% -----------------------------------------------------------------------------
% MRouter Manual file: mrman1.tex
% Stephen R. Whiteley, Whiteley Research Inc., Sunnyvale CA
% No copyright, open-source
% $Id: mrman1.tex,v 1.4 2017/02/17 05:44:31 stevew Exp $
% -----------------------------------------------------------------------------

\chapter{Introduction to {\MRouter}}
\pagenumbering{arabic}

In this manual, text which is provided in {\vt typewriter} font
represents verbatim input to or output from the program.  Text
enclosed in square brackets ( [text] ) is optional in the given
context, as in optional command arguments, whereas other text should
be provided as indicated.  Text which is {\it italicized\/} should be
replaced with the necessary input, as described in the accompanying
text.

\section{{\MRouter} Overview}

The {\MRouter} is an open-source maze router available from the free
software archive on {\vt wrcad.com}.  It is derived from the {\it
Qrouter} from {\vt opencircuitdesign.com}.

The {\MRouter} is a separate, stand-alone open-source project. 
Presently, users must download and build the {\MRouter} tool. 
Instructions can be found in the README file in the distribution. 
Eventually, packages for the supported operating systems will be
provided.

The {\it Qrouter} source code was converted from C to C++, and
architected into two classes:

\begin{enumerate}
\item{A database class.  This caches technology and routing
information.  It can read and write technology and route information
in standard LEF and DEF file formats.  It makes use of the open-source
Cadence LEF/DEF toolkit to provide complete and correct handling of
these files.}

\item{The maze router itself.  This is logically the same as the
{\it Qrouter}, but the code has been refactored to make use of C++
encapsulation and other elements of the language.  This will
facilitate future development of new modes and features, and
maintenance.}
\end{enumerate}

The two classes are exported as a plug-in library, intended to be used
with {\Xic}, or another application.  {\Xic} is a commercial graphical
editor for integrated circuit design from Whiteley Research Inc.  More
information can be gound on the company's web site {\vt
http://wrcad.com}.  {\Xic} has a developing interface to the
{\MRouter} plug-in.  providing full control of the routing operations,
visualization, etc.

\section{Theory of Operation}

The {\MRouter} is a maze router, which is terminology that describes
a routing program that makes use of the Lee algorithm.  On
oversimplified description of the Lee algorithm is as follows.

\begin{enumerate}
\item{The routing space is logically divided into an x-y grid of
routing channels.  We will assume that each layer has the same
routing pitch for simplicity.  From each routing grid location, one
can logically move north, south, east, west, and up or down.}

\item{Obstructions and pins from placed gate instances are added to
the grid by setting flags in the overlapping grid locations.  These
indicate locations where it is not possible to "travel".}

\item{Suppose you have a route with two connections:  the source and
the target.  Both the source and target consist initially of a set of
one or more points, each marked accordinly.  One iteratively visits
all of the non-obstructed grid points adjacent to a previously
visited grid point, whth the initial points being the source. 
Iteration continues until a new point falls on a target location.  We
have then found a route through the maze.}

\item{Nets with more than two connections are handled analogously.}

\item{Once a complete route is found, we then "commit" the route by
marking all of the grid points that the route covers, which are then
treated like obstructions when finding routes on other nets.}
\end{enumerate}

Maze routers are most successful for small to medium designs, as they
are compute and memory intensive.  They do have the advantage that if
a layout is routable, the Lee algorithm will find a route, and the
route will be the lowest cost (perhaps shortest) of any alternatives.

{\MRouter} (and {\it Qrouter}) use extensions to the basic Lee
algorithm in order to reduce routing timer.  First of all, routing is
a two stage process.  In stage 1, we route as much as can be routed,
but typically some fraction of the nets will be painted into a corner
and not be routable.  We keep a "failed route" list of these routes. 
In stage 2, we use a different approach.  For each route in the failed
list, we will find a best route while ignoring collisions with
existing routes.  If this succeeds, we then find the routes that
collide, rip them up, and add them to the end of the failed route
list.  We continue processing the failed route list until all nets are
routed, or stage 2 can no longer make progress and stalls.

One important factor is the use of masking and multiple passes when
finding routes.  The mask identifies grid points that are candidates
for searching for a route.  We attempt to identify where a solution
is likely to be found, and enable this area, plus some "slop" space. 
This minimizes the number of grid locations that need processing,
speeding up the finding of nets, at least until congestion becomes
heavy.  As congestion becomes heavier, one has to increase the size
of the mask open areas to find a solution.  To find the last of the
nets, one may work with no mask at all.  This is resource consuming,
but may be the only way to find nets that need to wander a long way
off-course in order to finally make the connection.

When finding a route, we try an initial pass with a tight mask.  If
that fails, we perform subsequent passes, each time increasing the
open area of the mask, until we have success or reach a limit on
the number of passes.

\section{Use with {\Xic}}

Once installed in the standard location, the {\Xic} program from
Whiteley Research Inc.  will find and load the plug-in on start-up.  A
message ``{\vt Loading MRouter}'' will be printed in the console along
with the other startup messages.

When {\Xic} starts, it will look for the {\MRouter}
plug-in in the following locations, in order.
\begin{enumerate}
\index{MROUTER\_PATH environment variable}
\item{The value of the {\vt MROUTER\_PATH} environment variable,
if found.  This is the full path to the shared library.  If this
variable exists, no other locations are checked.}

\index{MROUTER\_HOME environment variable}
\item{The {\MRouter} installation location, which defaults to {\vt
/usr/local/mrouter}.  If installed in another location, the
environment variable {\vt MROUTER\_HOME} can be set to the equivalent
path.  The library will be found in the {\vt lib} subdirectory of the
installation location.}
\end{enumerate}

There is a version compatibility test applied before the plug-in can
be loaded.  The {\MRouter} version string consists of three integers
separated by periods, as

\begin{quote}
{\it major\/}{\vt .}{\it minor\/}{\vt .}{\it release}
\end{quote}

The {\it major} and {\it minor} numbers must match the version {\Xic}
was built for.  If not, the plug-in is not loaded.  The {\it release}
number need not match, consequently there may be differences in
behavior seen, but there should be no instability.

If the {\vt XIC\_PLUGIN\_DBG} variable is set in the environment,
messages are printed in the console tracing the plug-in search and
load.  This can be used to diagnosae a problem, as the process is
otherwise silent.

Presently, the router is controlled exclusively with the {\cb !mr}
text command.  The {\vt MRouter} script function can be used to
perform the operations from scripts.

