\appendix
\chapter{File Formats}
\index{file formats}

%S-----------------------------------------------------------------------------
\section{Rawfile Format}
\index{file formats!rawfile}
\label{rawfilefmt}

% spMain.hlp:rawfilefmt 012409

Rawfiles produced and read by {\WRspice} have either an ASCII or a
binary format.  ASCII format is the preferred format for general use,
as it is hardware independent and easy to modify, though the binary
format is the most economical in terms of space and speed of access.

\index{rawfile ASCII format}
The ASCII format consists of lines or sets of lines introduced by a
keyword.  The {\vt Title} and {\vt Date} lines should be the first in
the file and should occur only once.  There may be any number of plots
in the file, each one beginning with the {\vt Plotname}, {\vt Flags},
{\vt No.~Variables}, {\vt No.~Points}, {\vt Variables}, and {\vt
Values} lines.  The {\vt Command} and {\vt Option} lines are optional
and may occur anywhere between the {\vt Plotname} and {\vt Values}
lines.  Note that after the {\vt Variables} keyword there must be {\it
numvars} ``declarations'' of outputs, and after the {\vt Values}
keyword, there must be {\it numpoints} lines, each consisting of {\it
numvars} values.  To clarify this discussion, one should create an
ASCII rawfile with {\WRspice} and examine it.

\begin{quote}
\begin{tabular}{|l|l|}\hline
\bf Line Name     & \bf Description\\[0.5ex]\hline \hline
\vt Title         & An arbitrary string describing the circuit\\ \hline
\vt Date          & A free-format date string\\ \hline
\vt Plotname      & A string describing the analysis type\\ \hline
\vt Flags         & Either ``{\vt complex}'' or ``{\vt real}''\\ \hline
\vt No.~Variables & The number of variables (numvars)\\ \hline
\vt No.~Points    & The number of points (numpoints)\\ \hline
\vt Command       & An arbitrary {\WRspice} command\\ \hline
\vt Option        & {\WRspice} variables\\ \hline
\vt Variables     & A number of variable lines (see below)\\ \hline
\vt Values        & A number of data lines (see below)\\ \hline
\end{tabular}
\end{quote}
Any text on a {\vt Command} line is executed when the file is loaded
as if it were typed as a command.  By default, {\WRspice} puts a
{\cb version} command into every rawfile it creates.

\index{rawfile options}
Text on an {\vt Option} line is parsed as if it were the arguments to a
{\WRspice} {\cb set} command.  The variables set are then available
normally, except that they are read-only and are associated with the
plot.

\index{rawfile variables}
A {\vt Variable} line looks like
\begin{quote}
{\it number name typename} {\vt [ {\it parm=value} ] ....}
\end{quote}
The number field is ignored by {\WRspice}.  The {\it name} is the name
by which this quantity will be referenced in {\WRspice}.  The {\it
typename} may be either a pre-defined type from the table below, or
one defined with the {\cb deftype} command.

\begin{quote}
\begin{tabular}{|l|c|c|}\hline
Name         & Description          & SPICE2 Numeric Code\\ \hline\hline
{\vt notype}       & Dimensionless value  & 0\\ \hline
{\vt time}         & Time                 & 1\\ \hline
{\vt frequency}    & Frequency            & 2\\ \hline
{\vt voltage}      & Voltage              & 3\\ \hline
{\vt current}      & Current              & 4\\ \hline
{\vt output-noise} & SPICE2 .noise result & 5\\ \hline
{\vt input-noise}  & SPICE2 .noise result & 6\\ \hline
{\vt HD2}          & SPICE2 .disto result & 7\\ \hline
{\vt HD3}          & SPICE2 .disto result & 8\\ \hline
{\vt DIM2}         & SPICE2 .disto result & 9\\ \hline
{\vt SIM2}         & SPICE2 .disto result & 10\\ \hline
{\vt DIM3}         & SPICE2 .disto result & 11\\ \hline
{\vt pole}         & SPICE3 pz result     & 12\\ \hline
{\vt zero}         & SPICE3 pz result     & 13\\ \hline
\end{tabular}
\end{quote}

The (optional) {\it parm} keywords and values follow.  The known
parameter names are listed in the table below.

\begin{quote}
\begin{tabular}{|l|l|}\hline
Name      & Description\\ \hline\hline
\vt min   & Minimum significant value for this output\\ \hline
\vt max   & Maximum significant value for this output\\ \hline
\vt color & The name of a color to use for this value\\ \hline
\vt scale & The name of another output to use as the scale\\ \hline
\vt grid  & The type of grid to use -- numeric codes are:\\
\qquad\vt 0 &  Linear grid\\
\qquad\vt 1 &  Log-log grid\\
\qquad\vt 2 &  X-log/Y-linear grid\\
\qquad\vt 3 &  X-linear/Y-log grid\\
\qquad\vt 4 &  Polar grid\\
\qquad\vt 5 &  Smith grid\\ \hline
\vt plot  & The plotting style to use -- numeric codes are:\\
\qquad\vt 0 & Connected points\\
\qquad\vt 1 & ``Comb'' style\\
\qquad\vt 2 & Unconnected points\\ \hline
\vt dims  & The dimensions of this vector -- not fully supported\\ \hline
\end{tabular}
\end{quote}

If the flags value is {\vt complex}, the points look like {\it
r\/},{\it i} where {\it r} and {\it i} are exponential floating point
format.  Otherwise they are real values in exponential format.  Only
one of {\vt real} and {\vt complex} should appear.

The lines are guaranteed to be less than 80 columns wide, unless the
plot title or variable names are very long, or a large number of
variable options are given.

\index{rawfile binary format}
The binary format is similar to the ASCII format in organization,
except that it is not text-mode.  Strings are NULL terminated instead
of newline terminated, and the values are in the machine's double
precision floating point format instead of in ASCII.  This makes it
much easier to read and write and reduces file size, but the binary
format is not portable between machines with different floating point
formats.

The circuit title, date, and analysis type name in that order are at
the start of the plot, each terminated by a NULL byte.  Then the flags
field (a short, which is 1 for real data and 2 for complex data), the
number of outputs, and the number of points (both integers) are
present.  Following this is a list of NULL-terminated strings which
are command lines.  This list is terminated by an extra NULL byte. 
Then come the options, which consist of the name, followed by the type
and the value in binary.  The output ``declarations'' consist of the
name, type code, flags, color, grid type, plot type, and dimension
information in that order.  Next come the values, which are either
doubles or pairs of doubles in the case of complex data.

The ``old'' binary format, which is used by SPICE2, is not accepted by
{\WRspice}, however the format is given below should it be necessary
to write a translator.

\begin{quote}
\begin{tabular}{|l|l|}\hline
\multicolumn{2}{|l|}{\bf SPICE2 Binary Rawfile Format}\\ \hline
\bf Field       & \bf Size in Bytes\\ \hline\hline
title           & 80\\ \hline
date            & 8\\ \hline
time            & 8\\ \hline
numoutputs      & 2\\ \hline
the integer 4   & 2\\ \hline
output names    & 8 for each output\\ \hline
types of output & 2 for each output\\ \hline
node index      & 2 for each output\\ \hline
plot title      & 24\\ \hline
data            & {\vt numpoints * numoutputs * 8}\\ \hline
\end{tabular}
\end{quote}

The data are in the form of double precision numbers, or
pairs of single precision numbers if the data are complex.

The values recognized for the ``types of output'' fields are listed in
the data types (top) table above as the ``SPICE2 Numeric Code''.


%S-----------------------------------------------------------------------------
\section{Help Database Files}
\label{helpfiles}
\index{file formats!help files}
\index{help files}

% HelpSys.hlp:helpfiles

The help information is obtained from database files suffixed with
{\vt .hlp} found along the help search path.  These directories may
also contain other files referenced in the help text, such as image
files.  The help search path can be set in the environment with the
variable {\et SPICE\_HLP\_DIR}, and/or may be set with the {\vt
helppath} variable, which will override the environment.  These files
have a simple format, allowing users to create and modify them.  Each
help entry is associated with one or more keywords, which should be
unique in the database.  The help system has a debugging mode, which
can usually be switched on by the application, which will issue a
warning message on {\vt stderr} if a name clash is detected.  The
files are ASCII text, either in DOS or Unix format.  Fields are
separated by keywords which begin with ``{\vt !!}''.  Although the
help system provides rich-text presentation from HTML formatting,
entries can be in plain text.  A sample plain-text entry has the form:

\begin{quote}
\begin{verbatim}
!!KEYWORD
excmd
!!TITLE
Example Command
!!TEXT
    This command exists only in this example.  Note that the
    !!keywords only have effect if they start in the first
    column.  The blank line below is optional.

!!SUBTOPICS
akeyword
anotherkeyword
!!SEEALSO
yetanotherkeyword
\end{verbatim}
\end{quote}

In this example, the keyword ``{\vt excmd}'' is used to access the
topic, and should be unique among the database entries accessed by the
application.  The text which appears in the topic (following {\vt
!!TEXT}) is shown indented, which is recommended for clarity, but is
not required.

In ``{\vt .hlp}'' files, lines anywhere with `{\vt *}' or `{\vt \#}'
in the first column are ignored, as they are assumed to be comments.
Blank lines outside of the {\vt !!TEXT} field are ignored.  Leading
white space is stripped, which can be a problem for maintaining
indentation in formatted plain text.  To add a space which will not be
stripped, use the HTML escape ``{\vt \&\#32;}''.

The following `{\vt !!}' keywords can appear in ``{\vt .hlp}'' files.
These are recognized only in upper case, and must start in the first
text column.

\begin{description}
\item{{\vt !!}(space) {\it anything}}\\
A line beginning with two exclamation points followed by a space
character is ignored.

\item{\vt !!KEYWORD {\it keyword-list}}\\
This keyword signals the start of a new topic.  The {\it keyword-list}
consists of one or more tokens, each of which must be unique among all
topics in the database.  The words are used to identify the topic, and
if more than one is listed, the additional words are equivalent
aliases.  The {\it keyword-list} may follow {\vt !!KEYWORD} on the
same line, or may be listed in the following line, in which case {\vt
!!KEYWORD} should appear alone on the line.

Punctuation is allowed in keywords, only white space characters can
not be used.  The `{\vt \#}' character has special meaning and should
not be part of a keyword name.  Also, character sequences that could
be confused with a URL or directory path should be avoided.  The
latter basically prohibits the `{\vt /}' character (and also
`$\backslash$' under Windows) from being included in keywords.  There
are special names starting with `\$' which are expanded to
application-specific internal variables, as described below.  To avoid
any possibility of a clash, it is probably best to avoid `\$' in
general keywords.

It is often useful to include a meaningful prefix in keywords to
ensure uniqueness, for example in {\Xic}, all commands have keywords
prefixed with ``{\vt xic:}''.

\item{\vt !!TITLE {\it string}}\\
The {\vt !!TITLE} specifies the title of the topic, and should follow
the {\vt !!KEYWORD} specification.  The title text can appear on the
same line following {\vt !!TITLE}, or on the next line, in which case
{\vt !!TITLE} should appear alone in the line.  The title is printed
at the top of the topic display, and is used in menus of topics.

\item{\vt !!TEXT}\\
This line signals the beginning of the topic text, which is expected
to be plain text.  The keyword is mutually exclusive with the {\vt
!!HTML} keyword.  The lines following {\vt !!TEXT} up to the next {\vt
!!KEYWORD}, {\vt !!SEEALSO}, or {\vt !!SUBTOPICS} line or end of file
are read into the display window.  The plain text is converted to HTML
before being sent to the display in the following manner:

\begin{enumerate}\rr
\item The title text is enclosed in {\vt <H1>...</H1>}.
\item Each line of text has a {\vt <BR>} appended.
\item The subtopics and see-alsos are preceded with added
    {\vt <H3>Subtopics</H3>} and {\vt <H3>References</H3>} lines.
\item The subtopics and see-alsos are converted to links of the form
    {\vt <A HREF="{\it keyword\/}">{\it title\/}</A>} where the {\it
    keyword} is the database keyword, and the {\it title} is the title
    text for the entry.
\end{enumerate}

Note that the text area can contain HTML tags for various things, such
as images.  Also note that text formatting is taken from the help file
(the {\vt <BR>} breaks lines), and not reformatted at display time.
The {\vt !!HTML} line should be used rather than {\vt !!TEXT} if the
text requires full HTML formatting.

\item{\vt !!HTML}\\
This line signals the beginning of the topic text, which is expected
to be HTML-formatted.  The keyword is mutually exclusive with the {\vt
!!TEXT} keyword.  The parser understands all of the standard HTML 3.2
syntax, and a few 4.0 extensions.  References are to keywords found in
the database and general URLs.  Image ({\vt .gif}, etc.) files can be
referenced, and are expected to be found along with the {\vt .hlp}
files.

\item{\vt !!IFDEF {\it word}}\\
This line can appear in the block of text following {\vt !!TEXT} or
{\vt !!HTML}.  In conjunction with the {\vt !!ELSE} and {\vt !!ENDIF}
directives, it allows for the conditional inclusion of blocks of text
in the topic.  The {\it word} is one of the special words defined by
the application.  Presently, the following words are defined:

\begin{tabular}{ll}\\
in {\Xic} & {\vt Xic}\\
in {\WRspice} & {\vt WRspice}\\
in either, under Windows & {\vt Windows}\\
\end{tabular}

If {\it word} is defined, the text up to the next {\vt !!ELSE} or {\vt
!!ENDIF} will be included in the topic, and any text following an {\vt
!!ELSE} up to {\vt !!ENDIF} is discarded.  If {\it word} is not
defined, the text up to the next {\vt !!ELSE} or {\vt !!ENDIF} is
discarded, and any text following an {\vt !!ELSE} is included.  The
constructs can be nested.  A word that is not recognized or absent is
``not defined''.  Every {\vt !!IFDEF} should have a corresponding {\vt
!!ENDIF}.  The {\vt !!ELSE} is optional.  The {\vt !!SEEALSO} and {\vt
!!SUBTOPICS} lines can appear within the blocks.

Example:
\begin{verbatim}
!!HTML
   Here is some text.
!!IFDEF Xic
   You are reading this in Xic.
!!ELSE
!!IFDEF WRspice
   You are reading this in WRspice.
!!ELSE
   You are not reading this in Xic or WRspice.
!!ENDIF
!!ENDIF
\end{verbatim}

\item{\vt !!IFNDEF {\it word}}\\
This keyword can appear in the block of text following {\vt !!TEXT} or
{\vt !!HTML}.  It is similar to {\vt !!IFDEF} but has the reverse
logic.

\item{\vt !!ELSE}\\
This keyword can follow {\vt !!IFDEF} or {\vt !!IFNDEF} and defines
the start of a block of text to include in the topic if the condition
is not satisfied.

\item{\vt !!ENDIF}\\
This keyword terminates the text blocks to be conditionally included
in the topic, using {\vt !!IFDEF} or {\vt !!IFNDEF}.

\item{\vt !!INCLUDE {\it filename}}\\
The keyword may appear in the text following {\vt !!TEXT} or {\vt
!!HTML}.  When encountered in the text to be included in the topic,
the text of {\it filename}, which is searched for in the help search
path if not an absolute pathname, is added to the displayed text of
the current topic.  There is no modification of the text from {\it
filename}.

If the filename is a relative path to a subdirectory of one of the
directories of a directory in the help search path, the subdirectory
is added to the search list.  Thus, an HTML document and associated
gif files can be placed in a separate subdirectory in the help tree.
The HTML document can be referenced from the main help files with a
{\vt !!INCLUDE} directive, and there is no need to explicitly change
the help search path.

\item{\vt !!REDIRECT {\it keyword target}}\\
This will define {\it keyword} as an alias for {\it target}.  The {\it
target} can be any input token recognizable by the help system,
including URLs, named anchors, and local files.  For example:
\begin{quote}
{\vt !!REDIRECT nyt http://www.nytimes.com}
\end{quote}
Giving ``{\vt !help nyt}'' in {\Xic} or ``{\vt help nyt}'' in
{\WRspice} will bring up a help window containing the New York Times
web page.

\item{\vt !!HEADER}\\
The text that follows, up until the next {\vt !!KEYWORD} or {\vt
!!FOOTER}, is saved for inclusion in each page composed from the {\vt
!!HTML} lines for database keywords.  The header is inserted at the
top of the page.  There can be only one header defined, and if more
than one are found in the help files, the first one read will be used.

In the header text, the literal token {\vt \%TITLE\%} is replaced with
the {\vt !!TITLE} text of the current topic when displayed.

\item{\vt !!FOOTER}\\
The text that follows, up until the next {\vt !!KEYWORD} or {\vt
!!HEADER}, is saved for inclusion in each page composed from the {\vt
!!HTML} lines for database keywords.  The footer is inserted at the
bottom of the page.  There can be only one footer defined, and if more
than one are found in the help files, the first one read will be used.

\item{\vt !!SEEALSO {\vt keyword-list}}\\
The {\it keyword-list} consists of a list of keywords that are
expected to be defined by {\vt !!KEYWORD} lines elsewhere in the
database.  A menu of these items is displayed at the bottom of the
topic text, under the heading ``References''.  The keywords specified
after {\vt !!SEEALSO} can appear on the same line separated with
space, or on multiple lines.  If a keyword in these lists is not found
in the database, the normal action is to ignore the error.  The
application may provide a debugging mode, whereby unresolved
references will produce a warning message.

\item{\vt !!SUBTOPICS {\it keyword-list}}\\
This produces a menu of the topics found in the {\it keyword-list}
very similar to {\vt !!SEEALSO}, however under the heading
``Subtopics''.  This can be used in addition to {\vt !!SEEALSO}.
\end{description}

%SU-------------------------------------
\subsection{Anchor Text}

Clickable references in the HTML text have the usual form:
\begin{quote}
{\vt <a href="{\it something}">{\it highlighted text}</a>}
\end{quote}
Here, ``{\it something}'' can be a help database keyword or an
ordinary URL.

One can use named anchors in help keywords.  This means that the `{\vt
\#}' symbol is holy, and should not be used in help keywords.  The
named anchors can appear in the {\vt !!HTML} part of the help database
entries in the usual HTML way, e.g.

\begin{verbatim}
!!KEYWORD
somekeyword
...
!!HTML
    ...
    <a name="refname">some text</a>
\end{verbatim}

\begin{flushleft}
Then, referencing forms like ``{\vt !help somekeyword\#refname}'' and
{\vt <a href="somekeyword\#refname">blather</a>} will bring up the
``somekeyword'' topic, but with ``some text'' at the top of the help
window, rather than the start of the document.
\end{flushleft}

There is an additional capability:  `{\vt \$}' expansion.  Words found
in anchor text that begin with a dollar sign (`{\vt \$}') character
may be replaced by either a path related to the program, the value of
a variable saved in the program, or the value of an environment
variable.  The character that immediately follows the word can not be
alphanumeric.

This replacement is handled by a callback to the application, but both
{\Xic} (and its derivatives) and {\WRspice} support the following
keywords and behavior.

\begin{description}
\item{\vt \$PROGROOT}\\
This word is replaced by the full path to the program installation
directory, for example\\ ``{\vt /usr/local/xictools/xic}''.
\item{\vt \$HELP}\\
This word is replaced by {\vt \$PROGROOT/help}, meaning the same
directory as {\vt \$PROGROOT} suffixed with {\vt /help}.
\item{\vt \$EXAMPLES}\\
This word is replaced by {\vt \$PROGROOT/examples}, as above.
\item{\vt \$DOCS}\\
This word is replaced by {\vt \$PROGROOT/docs}, as above.
\item{\vt \$SCRIPTS}\\
This word is replaced by {\vt \$PROGROOT/scripts}, as above.
\end{description}

If there is no match to these words, the word, without the dollar
sign, is checked against the variable database.  If a variable is set
with the same name, the string value of the variable replaces the
word.  If there is no match, but the word without the dollar sign
matches tne name of an environment variable, the value of the
environment variable will replace the word.  If there is no match,
there is no substitution.  Substitutions are evaluated recursively.

If the first character of an anchor URL is `{\vt \symbol{126}}', the
path is tilde expanded.  This is done after `{\vt \$}' substitution. 
Tildes denote a user's home directory:  ``{\vt \symbol{126}/mydir}''
might expand to ``{\vt /home/yourhome/mydir}'', and ``{\vt
\symbol{126}joe/joesdir}'' might expand to ``{\vt
/home/joe/joesdir}'', etc.

In {\WRspice}, one can source files from anchor text in the HTML
viewer, if the anchor text consists of a file name with a ``{\vt
.cir}'' extension.  Thus, if one has a circuit file named ``{\vt
mycircuit.cir}'', and the HTML text in the help window contains a
reference like
\begin{quote}
{\vt <a html="mycircuit.cir">click here</a>}
\end{quote}
then clicking on the ``click here'' tag will source {\vt
mycircuit.cir} into {\WRspice}.  Similarly, anchor references to files
with a ``{\vt .raw}'' extension will be loaded into {\WRspice} (as a
{\it rawfile\/}, i.e., a plot data file) when the anchor is clicked.

%SU-------------------------------------
\subsection{{\vt .mozyrc} File}
\label{mozyrc}

% HelpSys.hlp:mozyrcfile

\index{.mozyrc file}
The help system looks for a file named ``{\vt .mozyrc}'' in the user's
home directory, which contains keywords which define the default
behavior of many of the commands and features of the help window. 
This is used only in UNIX/Linux releases.  It is necessary to install
this file if one wants alternate selections from the help window, for
example different fonts, to be persistent.

A sample {\vt .mozyrc} file listing is provided below.  The file can
be found in the {\vt startup} directory in the installation tree,
under the name ``{\vt mozyrc}''.  To install, edit the file if
necessary, then move it to your home directory under the name ``{\vt
.mozyrc}''.

\begin{verbatim}
# This is the startup file which sets defaults for the mozy web browser
# and the Xic/WRspice HTML viewer.  It should be installed as ".mozyrc"
# in the user's home directory, should the user wish to change the
# defaults.

# --- DISPLAY ATTRIBUTES -------------------------------------------------

# Background color used for pages that don't have a <body> tag,
# such as help text (default #e8e8f0)
DefaultBgColor #e8e8f0


# Background image URL to use for pages that don't have a <body> tag
# (no default)
#DefaultBgImage /some/dir/pretty_picture.jpg

# Text color to use for pages that don't have a <body> tag
# (default black)
DefaultFgText black

# Color to use for links in pages without a <body> tag
# (default blue)
DefaultFgLink blue

# How to handle images:
#  0 Don't display images that require downloading
#  1 Download images when encountered in document
#  2 Download images after document is displayed
#  3 Display images progressively after document is displayed (the default)
ImageLoadMode 3

# How to underline anchors when underlining is enabled
#  0 No underline
#  1 Single solid underline (default)
#  2 Double solid underline
#  3 Single dashed underline
#  4 Double dashed underline
AnchorUnderline 1

# If this is set to one (the default) anchors are shown as buttons.  If set
# to zero, anchors use the underlining style
AnchorButtons 0

# If set to one (the default) anchors will be highlighted when the pointer
# passes over them.  If zero, there will be no highlighting
AnchorHighlight 1

# The default font families.  This is the XLFD family name with "-size"
# appended.  Defaults:  adobe-times-normal-p-14   misc-fixed-normal-c-14
FontFamily adobe-times-normal-p-14
FixedFontFamily misc-fixed-normal-*-14

# If set to one, animations are frozen.  If zero (the default) animations
# will be shown normally
FreezeAnimations 0

# --- COMMUNICATIONS -----------------------------------------------------

# Time in seconds allowed for a response from a message (0 for no timeout,
# to 600, default 15)
Timeout 15

# Number of times to retry a message after a timeout (0 to 10, default 4)
Retries 4

# The port number used for HTTP communications (1 to 65536, default 80)
HTTP_Port 80

# The port number used for FTP communications (1 to 65536, default 21)
FTP_Port 21

# --- GENERAL ------------------------------------------------------------

# Number of cache files to save (2 to 4096, default 64)
CacheSize 64

# Set to one to disable disk cache, 0 (the default) enables cache
NoCache 0

# Set to one to disable sending and receiving cookies
NoCookies 0

# --- DEBUGGING ----------------------------------------------------------

# Set this to one to print extended status messages on terminal screen
# (default 0)
DebugMode 0

# Set this to one to print transaction headers to terminal screen
# (default 0)
PrintTransact 0

# Debugging mode for images
#  0 Disable debugging mode (the default)
#  1 Load local images after document is displayed
#  2 Display local images progressively after document is displayed
LocalImageTestMode 0

# Issue warnings about bad HTML syntax to terminal (1) or not (0, the default)
BadHTMLwarnings 0
\end{verbatim}


%S-----------------------------------------------------------------------------
\section{Example Data Files}
\label{examples}

% spExamples.hlp:examples 012309

The following circuits are examples.  There are a number of example
files available with the {\WRspice} distribution.  These are normally
found in {\vt /usr/local/xictools/wrspice/examples}.

%SU-------------------------------------
\subsection{Circuit 1:  Simple Differential Pair}

% spExamples.hlp:ex1 012309

The following file determines the dc operating point of a simple
differential pair.  In addition, the ac small-signal response is
computed over the frequency range 1Hz to 100MHz.

\begin{verbatim}
    Simple differential pair.
    vcc 7 0 12
    vee 8 0 -12
    vin 1 0 ac 1
    rs1 1 2 1k
    rs2 6 0 1k
    q1 3 2 4 mod1
    q2 5 6 4 mod1
    rc1 7 3 10k
    rc2 7 5 10k
    re 4 8 10k
    .model mod1 npn bf=50 vaf=50 is=1.E-12 rb=100 cjc=.5pf tf=.6ns
    .ac dec 10 1 100meg
    .end
\end{verbatim}

%SU-------------------------------------
\subsection{Circuit 2:  MOS Output Characteristics}

% spExamples.hlp:ex2 012309

The following file computes the output characteristics of a MOSFET
device over the range 0-10V for VDS and 0-5V for VGS.

\begin{verbatim}
    MOS output characteristics
    .options node nopage
    vds 3 0
    vgs 2 0
    m1 1 2 0 0 mod1 l=4u w=6u ad=10p as=10p
    .model mod1 nmos vto=-2 nsub=1.0e15 uo=550
    * vids measures Id, we could have used Vds, but Id would be negative
    vids 3 1
    .dc vds 0 10 .5 vgs 0 5 1
    .end
\end{verbatim}

%SU-------------------------------------
\subsection{Circuit 3:  Simple RTL Inverter}

% spExamples.hlp:ex3 012309

The following file determines the dc transfer curve and the transient
pulse response of a simple RTL inverter.  RTL was an early logic
family which died out in the early 1970's.  We could not think of
anything more archaic, as {\WRspice} does not contain a vacuum tube
model.

The input is a pulse from 0 to 5 Volts with delay, rise, and fall
times of 2ns and a pulse width of 30ns.  The transient interval is 0
to 100ns, with printing to be done every nanosecond.

\begin{verbatim}
    Simple Resistor-Transistor Logic (RTL) inverter
    vcc 4 0 5
    vin 1 0 pulse 0 5 2ns 2ns 2ns 30ns
    rb 1 2 10k
    q1 3 2 0 q1
    rc 3 4 1k
    .model q1 npn bf 20 rb 100 tf .1ns cjc 2pf
    .dc vin 0 5 0.1
    .tran 1ns 100ns
    .end
\end{verbatim}

%SU-------------------------------------
\subsection{Circuit 4:  Four-Bit Adder}

% spExamples.hlp:ex4 012309

The following file simulates a four-bit binary adder, using several
subcircuits to describe various pieces of the overall circuit.

\begin{verbatim}
    ADDER - 4 BIT ALL-NAND-GATE BINARY ADDER
    *
    *** SUBCIRCUIT DEFINITIONS
    .SUBCKT NAND 1 2 3 4
    *NODES: INPUT(2), OUTPUT, VCC
    Q1 9 5 1 QMOD
    D1CLAMP 0 1 DMOD
    Q2 9 5 2 QMOD
    D2CLAMP 0 2 DMOD
    RB 4 5 4K
    R1 4 6 1.6K
    Q3 6 9 8 QMOD
    R2 8 0 1K
    RC 4 7 130
    Q4 7 6 10 QMOD
    DVBEDROP 10 3 DMOD
    Q5 3 8 0 QMOD
    .ENDS NAND
    .SUBCKT ONEBIT 1 2 3 4 5 6
    *NODES: INPUT(2), CARRY-IN, OUTPUT, CARRY-OUT, VCC
    X1 1 2 7 6 NAND
    X2 1 7 8 6 NAND
    X3 2 7 9 6 NAND
    X4 8 9 10 6 NAND
    X5 3 10 11 6 NAND
    X6 3 11 12 6 NAND
    X7 10 11 13 6 NAND
    X8 12 13 4 6 NAND
    X9 11 7 5 6 NAND
    .ENDS ONEBIT
    .SUBCKT TWOBIT 1 2 3 4 5 6 7 8 9
    *NODES: INPUT - BIT0(2) / BIT1(2), OUTPUT - BIT0 / BIT1,
    *       CARRY-IN, CARRY-OUT, VCC
    X1 1 2 7 5 10 9 ONEBIT
    X2 3 4 10 6 8 9 ONEBIT
    .ENDS TWOBIT
    *
    .SUBCKT FOURBIT 1 2 3 4 5 6 7 8 9 10 11 12 13 14 15
    *NODES: INPUT - BIT0(2) / BIT1(2) / BIT2(2) / BIT3(2),
    *       OUTPUT - BIT0 / BIT1 / BIT2 / BIT3, CARRY-IN, CARRY-OUT, VCC
    X1 1 2 3 4 9 10 13 16 15 TWOBIT
    X2 5 6 7 8 11 12 16 14 15 TWOBIT
    .ENDS FOURBIT
    *
    *** DEFINE NOMINAL CIRCUIT
    *
    .MODEL DMOD D
    .MODEL QMOD NPN(BF=75 RB=100 CJE=1PF CJC=3PF)
    VCC 99 0 DC 5V
    VIN1A 1 0 PULSE(0 3 0 10NS 10NS   10NS   50NS)
    VIN1B 2 0 PULSE(0 3 0 10NS 10NS   20NS  100NS)
    VIN2A 3 0 PULSE(0 3 0 10NS 10NS   40NS  200NS)
    VIN2B 4 0 PULSE(0 3 0 10NS 10NS   80NS  400NS)
    VIN3A 5 0 PULSE(0 3 0 10NS 10NS  160NS  800NS)
    VIN3B 6 0 PULSE(0 3 0 10NS 10NS  320NS 1600NS)
    VIN4A 7 0 PULSE(0 3 0 10NS 10NS  640NS 3200NS)
    VIN4B 8 0 PULSE(0 3 0 10NS 10NS 1280NS 6400NS)
    X1 1 2 3 4 5 6 7 8 9 10 11 12 0 13 99 FOURBIT
    RBIT0 9 0 1K
    RBIT1 10 0 1K
    RBIT2 11 0 1K
    RBIT3 12 0 1K
    RCOUT 13 0 1K
    *
    *** (FOR THOSE WITH MONEY (AND MEMORY) TO BURN)
    * Ah, the good olde days...
    .TRAN 1NS 6400NS
    .END
\end{verbatim}

%SU-------------------------------------
\subsection{Circuit 5:  Transmission Line Inverter}

% spExamples.hlp:ex5 012309

The following file simulates a transmission line inverter.  Two
transmission line elements are required since two propagation modes
are excited.  In the case of a coaxial line, the first line (t1)
models the inner conductor with respect to the shield, and the second
line (t2) models the shield with respect to the outside world.

\begin{verbatim}
    transmission-line inverter
    v1 1 0 pulse(0 1 0 0.1N)
    r1 1 2 50
    x1 2 0 0 4 tline
    r2 4 0 50
    .subckt tline 1 2 3 4
    t1 1 2 3 4 z0=50 td=1.5ns
    t2 2 0 4 0 z0=100 td=1ns
    .ends tline
    .tran 0.1ns 20ns
    .end
\end{verbatim}

%SU-------------------------------------
\subsection{Circuit 6:  Function and Table Demo}

% spExamples.hlp:ex6 012309

Below is a file which illustrates some features exclusive to {\WRspice} 
for specifying the output of sources.

\begin{verbatim}
    WRspice function and table demo
    *
    * WRspice allows arbitrary functional dependence in sources.  This
    * file demonstrates some of the capability.
    *
    * v1 is numerically equal to the exponentiation of
    * 2 times the sine.  "x" is replaced by the time variable.
    v1 1 0 exp(2*sin(6.28e9*x))
    r1 1 0 1
    *
    * v2 obtains values from table t1
    v2 2 0 table(t1, time)
    r2 2 0 1
    .table t1 0 0 100p .1 500p 0 750p .2 1000p 0
    *
    * v3 is a 0-1 ramp
    v3 3 0 pwl(0 0 1n 1)
    r3 3 0 1
    *
    * e1 illustrates use of sub-tables.  x is the voltage from v3
    e1 4 0 3 0 table(t2, x)
    * below is an alternative equivalent form for e1
    *e1 4 0 t2(v(3))
    r4 4 0 1
    .table t2 0 table t3 .5 table t4 .75 .75 1 0
    .table t3 0 0 .25 1 .5 0
    .table t4 0 0 .5 0 .625 1 .75 .75, x)
    *
    * e2 produces the same output as e1, but uses a PWL statement.
    * when the controlling nodes are given, pwl uses the control source,
    * and not time when used in e,f,g,h sources
    e2 5 0 3 0 pwl(0 0 .25 1 .5 0 .625 1 .75 .75 1 0)
    r5 5 0 1
    *
    .tran 1p 1n
    * type "run", then "plot v(1) v(2) v(3) v(4)"
    .end
\end{verbatim}

%SU-------------------------------------
\subsection{Circuit 7:  MOS Convergence Test}

% spExamples.hlp:ex7 012309

Below is an example circuit that illustrates some of the new features,
and some older features perhaps not widely appreciated.  This also
served as a test for improving MOS convergence.  To run a convergence
test:
\begin{enumerate}
\item{Source this circuit.}
\item{Press the {\cb siminterface} button in the Debug tool.}
\item{Type ``{\vt set value1 = mult}''.}
\item{Type ``{\vt loop .5 2.5 .1 op}''.}
\end{enumerate}

This runs operating point analysis for M values from .5 to 2.5, and
displays a little plot of the convergence process.  In the display,
`{\vt +}' means an increasing gmin step, `.' is a source step, `{\vt
-}' is a decreasing gmin step.

Note that 100 nS is too coarse to get a decent looking plot with BSIM
devices.  Use ``{\vt tran 1n 10u}'' in that case, since the circuit
seems to be oscillating at very high frequency.

\begin{verbatim}
    mosamp2 - mos amplifier - transient
    .options abstol=10n vntol=10n noopiter mult=2 nqs=1 reltol=1e-4
    *.op
    .tran 0.1us 10us
    .plot  tran v(20) v(66)
    * set below to 0 for old MOS model
    .param bsim = 1
    .if bsim = 1
    m1  15 15  1 32 n1 w=88.9u  l=25.4u  m=$mult nqsmod=$nqs
    m2   1  1  2 32 n1 w=12.7u  l=266.7u m=$mult nqsmod=$nqs
    m3   2  2 30 32 n1 w=88.9u  l=25.4u  m=$mult nqsmod=$nqs
    m4  15  5  4 32 n1 w=12.7u  l=106.7u m=$mult nqsmod=$nqs
    m5   4  4 30 32 n1 w=88.9u  l=12.7u  m=$mult nqsmod=$nqs
    m6  15 15  5 32 n1 w=44.5u  l=25.4u  m=$mult nqsmod=$nqs
    m7   5 20  8 32 n1 w=482.6u l=12.7u  m=$mult nqsmod=$nqs
    m8   8  2 30 32 n1 w=88.9u  l=25.4u  m=$mult nqsmod=$nqs
    m9  15 15  6 32 n1 w=44.5u  l=25.4u  m=$mult nqsmod=$nqs
    m10  6 21  8 32 n1 w=482.6u l=12.7u  m=$mult nqsmod=$nqs
    m11 15  6  7 32 n1 w=12.7u  l=106.7u m=$mult nqsmod=$nqs
    m12  7  4 30 32 n1 w=88.9u  l=12.7u  m=$mult nqsmod=$nqs
    m13 15 10  9 32 n1 w=139.7u l=12.7u  m=$mult nqsmod=$nqs
    m14  9 11 30 32 n1 w=139.7u l=12.7u  m=$mult nqsmod=$nqs
    m15 15 15 12 32 n1 w=12.7u  l=207.8u m=$mult nqsmod=$nqs
    m16 12 12 11 32 n1 w=54.1u  l=12.7u  m=$mult nqsmod=$nqs
    m17 11 11 30 32 n1 w=54.1u  l=12.7u  m=$mult nqsmod=$nqs
    m18 15 15 10 32 n1 w=12.7u  l=45.2u  m=$mult nqsmod=$nqs
    m19 10 12 13 32 n1 w=270.5u l=12.7u  m=$mult nqsmod=$nqs
    m20 13  7 30 32 n1 w=270.5u l=12.7u  m=$mult nqsmod=$nqs
    m21 15 10 14 32 n1 w=254u   l=12.7u  m=$mult nqsmod=$nqs
    m22 14 11 30 32 n1 w=241.3u l=12.7u  m=$mult nqsmod=$nqs
    m23 15 20 16 32 n1 w=19u    l=38.1u  m=$mult nqsmod=$nqs
    m24 16 14 30 32 n1 w=406.4u l=12.7u  m=$mult nqsmod=$nqs
    m25 15 15 20 32 n1 w=38.1u  l=42.7u  m=$mult nqsmod=$nqs
    m26 20 16 30 32 n1 w=381u   l=25.4u  m=$mult nqsmod=$nqs
    m27 20 15 66 32 n1 w=22.9u  l=7.6u   m=$mult nqsmod=$nqs
    .else
    m1  15 15  1 32 n1 w=88.9u  l=25.4u  m=$mult
    m2   1  1  2 32 n1 w=12.7u  l=266.7u m=$mult
    m3   2  2 30 32 n1 w=88.9u  l=25.4u  m=$mult
    m4  15  5  4 32 n1 w=12.7u  l=106.7u m=$mult
    m5   4  4 30 32 n1 w=88.9u  l=12.7u  m=$mult
    m6  15 15  5 32 n1 w=44.5u  l=25.4u  m=$mult
    m7   5 20  8 32 n1 w=482.6u l=12.7u  m=$mult
    m8   8  2 30 32 n1 w=88.9u  l=25.4u  m=$mult
    m9  15 15  6 32 n1 w=44.5u  l=25.4u  m=$mult
    m10  6 21  8 32 n1 w=482.6u l=12.7u  m=$mult
    m11 15  6  7 32 n1 w=12.7u  l=106.7u m=$mult
    m12  7  4 30 32 n1 w=88.9u  l=12.7u  m=$mult
    m13 15 10  9 32 n1 w=139.7u l=12.7u  m=$mult
    m14  9 11 30 32 n1 w=139.7u l=12.7u  m=$mult
    m15 15 15 12 32 n1 w=12.7u  l=207.8u m=$mult
    m16 12 12 11 32 n1 w=54.1u  l=12.7u  m=$mult
    m17 11 11 30 32 n1 w=54.1u  l=12.7u  m=$mult
    m18 15 15 10 32 n1 w=12.7u  l=45.2u  m=$mult
    m19 10 12 13 32 n1 w=270.5u l=12.7u  m=$mult
    m20 13  7 30 32 n1 w=270.5u l=12.7u  m=$mult
    m21 15 10 14 32 n1 w=254u   l=12.7u  m=$mult
    m22 14 11 30 32 n1 w=241.3u l=12.7u  m=$mult
    m23 15 20 16 32 n1 w=19u    l=38.1u  m=$mult
    m24 16 14 30 32 n1 w=406.4u l=12.7u  m=$mult
    m25 15 15 20 32 n1 w=38.1u  l=42.7u  m=$mult
    m26 20 16 30 32 n1 w=381u   l=25.4u  m=$mult
    m27 20 15 66 32 n1 w=22.9u  l=7.6u   m=$mult
    .endif
    cc 7 9 40pf
    cl 66 0 70pf
    vin 21 0 AC pulse(0 5 1ns 1ns 1ns 5us 10us)
    vccp 15 0 dc +15
    vddn 30 0 dc -15
    vb 32 0 dc -20
    .if bsim
    .model n1 nmos(level=8 capmod=3)
    .else
    .model n1 nmos(nsub=2.2e15 uo=575 ucrit=49k uexp=0.1 tox=0.11u xj=2.95u
    +   level=2 cgso=1.5n cgdo=1.5n cbd=4.5f cbs=4.5f ld=2.4485u nss=3.2e10
    +   kp=2e-5 phi=0.6 )
    .endif
\end{verbatim}

%SU-------------------------------------
\subsection{Circuit 8:  Verilog Pseudo-Random Sequence}

% spExamples.hlp:ex8 012309

This example illustrates use of a {\vt .verilog} block to generate a
digital signal, that is then interfaced to and processed by a
conventional SPICE circuit.  The digital signal is a 511 step
pseudo-random sequence, which is converted to a voltage and filtered.

\begin{verbatim}
    * WRspice pseudo-random bit sequence demo
    v1 1 0 a/255-1
    r1  1 2 100
    c1 2 0 10p
    .tran 1p 10n
    .plot tran v(1) v(2)

    .verilog
    module  prbs;
    reg [8:0] a, b;
    reg clk;
    integer cnt;

    initial
        begin
        a = 9'hff;
        clk = 0;
        cnt = 0;
        $monitor("%d", cnt, "%b", a, a[0]);
        end

    always
        #5 clk = ~clk;

    always
        @(posedge clk)
        begin
        a = { a[4]^a[0], a[8:1] };
        if (a ==  9'hff)
            $stop;
        cnt = cnt + 1;
        end

    endmodule
    .endv
\end{verbatim}

%SU-------------------------------------
\subsection{Circuit 9:  Josephson Junction I-V Curve}

% spExamples.hlp:ex9 012309

\begin{verbatim}
    WRspice jj I-V curve demo
    *
    * One can plot a pretty decent iv curve using transient analysis.
    * This will show the differences between the various model options.
    *
    b1 1 0 jj1 control=v2
    v1 2 0 pwl(0 0 2n 70m 4n 0 5n 0)
    r1 2 1 100
    *
    * for rtype=4, vary v2 between 0 and 1 for no gap to full gap
    v2 3 0 .5
    *
    r2 3 0 1
    *
    * It is interesting to set rtype and delv to different values, and note
    * the changes.
    *
    *Nb 1000 A/cm2   area = 30 square microns
    .model jj1 jj(rtype=4,cct=1,icon=10m,vg=2.8m,delv=.1m,
    + icrit=0.3m,r0=100,rn=5.4902,cap=1.14195p)
    .tran 5p 5n
    * type "run", then "plot -b v(1) (-v1#branch)"
    .end
\end{verbatim}

%SU-------------------------------------
\subsection{Circuit 10: Josephson Gap Potential Modulation}

% spExamples.hlp:ex10 012309

\begin{verbatim}
    WRspice jj qp modulation demo
    *
    * The rtype=4 option of the Josephson model causes the gap potential
    * to scale with the external "control current" absolute value.  For
    * unit control current (1 Amp) or larger, the full potential is used,
    * otherwise it scales linearly to zero.  The transfer function is defined
    * externally with controlled sources, as below.  The approximation
    * Vg = Vg0*(1-t**4) is pretty good, except near t = 1 (T = Tc, t = T/Tc).
    * The actual transfer function is left to the user - in the example below,
    * the ambient temperature is 7K, Tc=9.2K, and 1mv of "input" causes 1K
    * temperature shift.
    *
    * For amusement, change cct=1 to cct=0 below.  This runs much more quickly
    * as critical current is set to zero.
    *
    b1 1 0 jj1 control=v2
    v1 2 0 pulse(0 35m 10p 10p)
    r1 2 1 100
    *
    v2 3 0
    g1 3 0 4 0 function 1 - (1000*x+7)/9.2)^4
    v4 4 0 pulse(-1m 1m 10p 10p 10p 20p 60p)
    *
    *Nb 1000 A/cm2   area = 30 square microns
    .model jj1 jj(rtype=4,cct=1,icon=10m,vg=2.8m,delv=.1m,
    + icrit=0.3m,r0=100,rn=5.4902,cap=1.14195p)
    .tran 1p 500p uic
    * type "run", then plot v(1) and v(4) to see the gap shift and input
    .end
\end{verbatim}



