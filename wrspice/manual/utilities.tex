
\chapter{Utility Programs}

% spMain.hlp:utilities 012609

The {\WRspice} distribution provides a few supplemental utility
and accessory programs.


%S-----------------------------------------------------------------------------
\section{The {\vt multidec} Utility: Coupled Lossy Transmission Lines}
\index{multidec program}

% spMain.hlp:multidec 012609

The standalone program {\vt multidec} produces a subcircuit for
multiconductor lossy transmission lines in terms of uncoupled (single)
simple lossy lines.  This decomposition is valid only if the following
hold:

\begin{enumerate}
\item{The electrical parameters (R, G, Cs, Cm, Ls, Lm) of all wires are
 identical and independent of frequency.}
\item{Each line is coupled only to its (maximum 2) nearest neighbors.}
\end{enumerate}

The subcircuit is sent to the standard output and is intended to be
included in an input file.

The command-line options for {\vt multidec} are as follows:
\begin{quote}
{\vt -l}$<$self-inductance Ls$>$\\
{\vt -c}$<$self-capacitance Cs$>$\\
{\vt -r}$<$series-resistance R$>$\\
{\vt -g}$<$parallel-conductance G$>$\\
{\vt -k}$<$coeff-of-inductive-coupling K$>$\\
{\vt -x}$<$mutual-capacitance Cm$>$\\
{\vt -o}$<$subckt-name$>$\\
{\vt -n}$<$number-of-conductors$>$\\
{\vt -L}$<$length$>$
\end{quote}

The inductive coupling coefficient K is the ratio of Lm
to Ls. Values for -l, -c, -o, -n and -L must be specified.

Example:
\begin{quote}
{\vt multidec -n4 -l9e-9 -c20e-12 -r5.3 -x5e-12 -k0.7 -otest -L5.4}
\end{quote}

This utility was written by J.S.  Roychowdhury for use with the lossy
transmission line model \cite{ltra}.


%S-----------------------------------------------------------------------------
\section{The {\vt printtoraw} Utility: Print to Rawfile Conversion}
\index{printtoraw program}

% spMain.hlp:printtoraw 012609

The {\vt printtoraw} program is a stand-alone utility provided with
the {\WRspice} distribution.  This converts the data in files produced
by the {\et print} command using output redirection into the rawfile
format, which can be plotted.  This works only for print files in the
standard columnar form.

\begin{quote}
Usage: {\vt printtoraw} [{\it printfile\/}]
\end{quote}

The argument, if given, is assumed to be a path to a file that was
produced by the {\WRspice} {\et print} command through redirection. 
If no argument is given, the standard input is read.  The data are
converted to rawfile format and dumped to the standard output.

Example:
\begin{quote}
{\vt wrspice> run}\\
{\vt wrspice> print v(1) v(2) v(3) > myfile}\\
{\vt wrspice> quit}\\
\\
{\vt bash> printtoraw myfile > myfile.raw}\\
\\
{\vt wrspice> load myfile.raw}\\
{\vt wrspice> plot all}
\end{quote}


%S-----------------------------------------------------------------------------
\section{The {\vt proc2mod} Utility: BSIM1 Model Generation}
\index{proc2mod program}

% spMain.hlp:proc2mod 012609

This utility, provided with SPICE3, produces a set of BSIM1 models
from process-dependent data provided in a ``process'' file.  An
example process ({\vt .pro}) file is provided with the {\WRspice}
examples.  This utility was written by J.  Pierret \cite{pierret},
and the reference presumably provides more information.


%S-----------------------------------------------------------------------------
\section{The {\vt wrspiced} Daemon: Remote SPICE Controller}
\label{wrspiced}
\index{wrspiced program}

% spMain.hlp:wrspiced 021811

{\WRspice} can be accessed and run from a remote system for
asynchronous simulation runs, for assistance in computationally
intensive tasks such as Monte Carlo analysis, and as a simulator for
the {\Xic} graphical editor.  This is made possible through a daemon
(background) process which controls {\WRspice} on the remote machine. 
The daemon has the executable name ``{\vt wrspiced}'', and should be
running on the remote machine.  This can be initiated in the system
startup procedure, or manually.  Generally, any user can start {\vt
wrspiced}, but only one daemon can be running on the host computer.

The {\vt wrspiced} program is part of the {\WRspice} distribution, and
is installed in the same directory as the {\vt wrspice} executable. 
The daemon manages the queue of submitted jobs and responses, and
maintains the communications port.  The {\vt wrspiced} daemon will
establish itself on a port, and wait for client messages.

%SU-------------------------------------
\subsection{SPICE Server Configuration}

There is little or no configuration required to run {\vt wrspiced},
but there are a few basic prerequisites.  Our assumption is that
{\WRspice} is installed on a network-reachable remote computer (the
``SPICE server''), and we wish to submit jobs to this {\WRspice} from
within {\Xic}, or from within {\WRspice} running on local computers
(the "clients").

The SPICE server must have {\WRspice} installed, and {\WRspice} must
be licensed to run on the server.
As a prerequisite, {\WRspice} should operate on the
SPICE server host in the normal way.

Historically, {\vt wrspiced} has used the service name ``{\vt spice}''
and port number 3004.  Releases 3.2.8 and later use the service name
``{\vt wrspice}'' instead of ``{\vt spice}'', and use port number 6114
by default.  The port 6114 is registered with IANA for this service.

The system services database is represented by the contents of the
file {\vt /etc/services} in simple installations.  If using NIS, then
the system will get its services information from elsewhere.  A system
administrator can add service names and port assignments to this
database.  The {\vt wrspiced} program does not require this.

%SU-------------------------------------
\subsection{Starting the Daemon}

The {\vt wrspiced} program command line has the following form:

\begin{quote}
{\vt wrspiced} [{\vt -fg}] [{\vt -l} {\it logfile\/}]
  [{\vt -p} {\it program\/}] [{\vt -m} {\it maxjobs\/}] [{\vt -t} {\it port\/}]
\end{quote}

There are five optional arguments.
\begin{description}
\item{\vt -fg}\\
If given, the {\vt wrspiced} program will remain in the foreground
(i.e., not become a ``daemon''), but will service requests normally. 
This may be useful for debugging purposes.

\item{{\vt -l} {\it logfile}}\\
The {\it logfile} is a path to a file that will receive status
messages from {\vt wrspiced}.  The default is the value of the {\et
SPICE\_DAEMONLOG} environment variable if set when the program is
started, or {\vt /tmp/wrspiced.log}.

\item{{\vt -p} {\it program}}\\
This specifies the {\WRspice} program to run, in case for some reason
the {\vt wrspice} binary has been renamed, or {\vt wrspice} is not in
the expected location.  This overrides the values of the {\et
SPICE\_PATH} and {\et SPICE\_EXEC\_DIR} environment variables, which
can also be used to set the path to the binary.  The default is ``{\vt
/usr/local/xictools/bin/wrspice}''.

\item{{\vt -m} {\it maxjobs}}\\
This sets the maximum number of jobs that the server will allow to be
running at the same time.  The default is 5.

\item{{\vt -t} {\it port}}\\
This sets the port to be used by the daemon, and overrides any port
set in the services database.  Clients must use the same port number
to connect to the SPICE server.
\end{description}

The daemon is started by simply typing the command.  If a machine is
to operate continuously as a SPICE server, it is recommended that the
{\vt wrspiced} daemon be started in the system initialization scripts. 
The daemon will run until explicitly killed by a signal, or the
machine is halted.  When the {\vt wrspiced} process terminates, any
{\WRspice} processes under management will also be killed.  The daemon
can be terminated, by the process owner, by giving the command ``{\vt
ps aux | grep wrspiced}'' and noting the process id (pid) number of
the running {\vt wrspiced} process, and then issuing ``{\vt kill} {\it
pid\/}'' using this pid number.

It may be necessary to become root before starting {\vt wrspiced}, as
on some systems connection to the port will otherwise be refused due
to permission requirements.  Starting by root is also required if the
log file is to be written to a directory such as {\vt /var/log} that
requires root permission for writing.

%SU-------------------------------------
\subsection{Client Configuration}

The port number used by the client must be the same as that used for
the server.  As for the server, if not supplied the port number will
be resolved if possible in the services database (e.g., the {\vt
/etc/services} file), and will revert to a default if not found.

In {\Xic} and {\WRspice}, the port number to use can be specified with
the host name, by appending the number following a colon, i.e.,
\begin{quote}
{\it hostname\/}[:{\it port\/}]
\end{quote}

A {\WRspice} server can receive jobs from {\Xic}, and from {\WRspice}
({\cb rspice} command).  Both programs have means by which the SPICE
server can be specified from within the program.  One means common to
both programs is through use of the {\et SPICE\_HOST} environment
variable.  The variable should be set to the host name of the SPICE
server, as resolvable by the client, followed by the optional colon
and port number.  When set, {\Xic} will by default use this server for
SPICE jobs initiated with the {\cb Run} button in the side menu, and
{\WRspice} will use this host in the {\et rspice} command.  In a
situation where the SPICE server provides the only SPICE available,
the {\et SPICE\_HOST} variable should be set in the user's shell
startup script.  In {\WRspice} the {\et rhost} shell variable and the
{\cb rhost} command can also be used to specify the remote host, and
these override any value set in the environment.

Note:  In {\Xic}, when {\WRspice} connects, a message is printed in
the terminal window similar to
\begin{quote}
{\vt Stream established to wrspice://chaucer, port 4573.}
\end{quote}
The ``port'' in this case is {\it not} the {\vt wrspiced} port
discussed above, but is a transient port created for the process.

