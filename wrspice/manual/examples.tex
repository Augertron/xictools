
%S-----------------------------------------------------------------------------
\section{Example Data Files}
\label{examples}

% spExamples.hlp:examples 012309

The following circuits are examples.  There are a number of example
files available with the {\WRspice} distribution.  These are normally
found in {\vt /usr/local/xictools/wrspice/examples}.

%SU-------------------------------------
\subsection{Circuit 1:  Simple Differential Pair}

% spExamples.hlp:ex1 012309

The following file determines the dc operating point of a simple
differential pair.  In addition, the ac small-signal response is
computed over the frequency range 1Hz to 100MHz.

\begin{verbatim}
    Simple differential pair.
    vcc 7 0 12
    vee 8 0 -12
    vin 1 0 ac 1
    rs1 1 2 1k
    rs2 6 0 1k
    q1 3 2 4 mod1
    q2 5 6 4 mod1
    rc1 7 3 10k
    rc2 7 5 10k
    re 4 8 10k
    .model mod1 npn bf=50 vaf=50 is=1.E-12 rb=100 cjc=.5pf tf=.6ns
    .ac dec 10 1 100meg
    .end
\end{verbatim}

%SU-------------------------------------
\subsection{Circuit 2:  MOS Output Characteristics}

% spExamples.hlp:ex2 012309

The following file computes the output characteristics of a MOSFET
device over the range 0-10V for VDS and 0-5V for VGS.

\begin{verbatim}
    MOS output characteristics
    .options node nopage
    vds 3 0
    vgs 2 0
    m1 1 2 0 0 mod1 l=4u w=6u ad=10p as=10p
    .model mod1 nmos vto=-2 nsub=1.0e15 uo=550
    * vids measures Id, we could have used Vds, but Id would be negative
    vids 3 1
    .dc vds 0 10 .5 vgs 0 5 1
    .end
\end{verbatim}

%SU-------------------------------------
\subsection{Circuit 3:  Simple RTL Inverter}

% spExamples.hlp:ex3 012309

The following file determines the dc transfer curve and the transient
pulse response of a simple RTL inverter.  RTL was an early logic
family which died out in the early 1970's.  We could not think of
anything more archaic, as {\WRspice} does not contain a vacuum tube
model.

The input is a pulse from 0 to 5 Volts with delay, rise, and fall
times of 2ns and a pulse width of 30ns.  The transient interval is 0
to 100ns, with printing to be done every nanosecond.

\begin{verbatim}
    Simple Resistor-Transistor Logic (RTL) inverter
    vcc 4 0 5
    vin 1 0 pulse 0 5 2ns 2ns 2ns 30ns
    rb 1 2 10k
    q1 3 2 0 q1
    rc 3 4 1k
    .model q1 npn bf 20 rb 100 tf .1ns cjc 2pf
    .dc vin 0 5 0.1
    .tran 1ns 100ns
    .end
\end{verbatim}

%SU-------------------------------------
\subsection{Circuit 4:  Four-Bit Adder}

% spExamples.hlp:ex4 012309

The following file simulates a four-bit binary adder, using several
subcircuits to describe various pieces of the overall circuit.

\begin{verbatim}
    ADDER - 4 BIT ALL-NAND-GATE BINARY ADDER
    *
    *** SUBCIRCUIT DEFINITIONS
    .SUBCKT NAND 1 2 3 4
    *NODES: INPUT(2), OUTPUT, VCC
    Q1 9 5 1 QMOD
    D1CLAMP 0 1 DMOD
    Q2 9 5 2 QMOD
    D2CLAMP 0 2 DMOD
    RB 4 5 4K
    R1 4 6 1.6K
    Q3 6 9 8 QMOD
    R2 8 0 1K
    RC 4 7 130
    Q4 7 6 10 QMOD
    DVBEDROP 10 3 DMOD
    Q5 3 8 0 QMOD
    .ENDS NAND
    .SUBCKT ONEBIT 1 2 3 4 5 6
    *NODES: INPUT(2), CARRY-IN, OUTPUT, CARRY-OUT, VCC
    X1 1 2 7 6 NAND
    X2 1 7 8 6 NAND
    X3 2 7 9 6 NAND
    X4 8 9 10 6 NAND
    X5 3 10 11 6 NAND
    X6 3 11 12 6 NAND
    X7 10 11 13 6 NAND
    X8 12 13 4 6 NAND
    X9 11 7 5 6 NAND
    .ENDS ONEBIT
    .SUBCKT TWOBIT 1 2 3 4 5 6 7 8 9
    *NODES: INPUT - BIT0(2) / BIT1(2), OUTPUT - BIT0 / BIT1,
    *       CARRY-IN, CARRY-OUT, VCC
    X1 1 2 7 5 10 9 ONEBIT
    X2 3 4 10 6 8 9 ONEBIT
    .ENDS TWOBIT
    *
    .SUBCKT FOURBIT 1 2 3 4 5 6 7 8 9 10 11 12 13 14 15
    *NODES: INPUT - BIT0(2) / BIT1(2) / BIT2(2) / BIT3(2),
    *       OUTPUT - BIT0 / BIT1 / BIT2 / BIT3, CARRY-IN, CARRY-OUT, VCC
    X1 1 2 3 4 9 10 13 16 15 TWOBIT
    X2 5 6 7 8 11 12 16 14 15 TWOBIT
    .ENDS FOURBIT
    *
    *** DEFINE NOMINAL CIRCUIT
    *
    .MODEL DMOD D
    .MODEL QMOD NPN(BF=75 RB=100 CJE=1PF CJC=3PF)
    VCC 99 0 DC 5V
    VIN1A 1 0 PULSE(0 3 0 10NS 10NS   10NS   50NS)
    VIN1B 2 0 PULSE(0 3 0 10NS 10NS   20NS  100NS)
    VIN2A 3 0 PULSE(0 3 0 10NS 10NS   40NS  200NS)
    VIN2B 4 0 PULSE(0 3 0 10NS 10NS   80NS  400NS)
    VIN3A 5 0 PULSE(0 3 0 10NS 10NS  160NS  800NS)
    VIN3B 6 0 PULSE(0 3 0 10NS 10NS  320NS 1600NS)
    VIN4A 7 0 PULSE(0 3 0 10NS 10NS  640NS 3200NS)
    VIN4B 8 0 PULSE(0 3 0 10NS 10NS 1280NS 6400NS)
    X1 1 2 3 4 5 6 7 8 9 10 11 12 0 13 99 FOURBIT
    RBIT0 9 0 1K
    RBIT1 10 0 1K
    RBIT2 11 0 1K
    RBIT3 12 0 1K
    RCOUT 13 0 1K
    *
    *** (FOR THOSE WITH MONEY (AND MEMORY) TO BURN)
    * Ah, the good olde days...
    .TRAN 1NS 6400NS
    .END
\end{verbatim}

%SU-------------------------------------
\subsection{Circuit 5:  Transmission Line Inverter}

% spExamples.hlp:ex5 012309

The following file simulates a transmission line inverter.  Two
transmission line elements are required since two propagation modes
are excited.  In the case of a coaxial line, the first line (t1)
models the inner conductor with respect to the shield, and the second
line (t2) models the shield with respect to the outside world.

\begin{verbatim}
    transmission-line inverter
    v1 1 0 pulse(0 1 0 0.1N)
    r1 1 2 50
    x1 2 0 0 4 tline
    r2 4 0 50
    .subckt tline 1 2 3 4
    t1 1 2 3 4 z0=50 td=1.5ns
    t2 2 0 4 0 z0=100 td=1ns
    .ends tline
    .tran 0.1ns 20ns
    .end
\end{verbatim}

%SU-------------------------------------
\subsection{Circuit 6:  Function and Table Demo}

% spExamples.hlp:ex6 012309

Below is a file which illustrates some features exclusive to {\WRspice} 
for specifying the output of sources.

\begin{verbatim}
    WRspice function and table demo
    *
    * WRspice allows arbitrary functional dependence in sources.  This
    * file demonstrates some of the capability.
    *
    * v1 is numerically equal to the exponentiation of
    * 2 times the sine.  "x" is replaced by the time variable.
    v1 1 0 exp(2*sin(6.28e9*x))
    r1 1 0 1
    *
    * v2 obtains values from table t1
    v2 2 0 table(t1, time)
    r2 2 0 1
    .table t1 0 0 100p .1 500p 0 750p .2 1000p 0
    *
    * v3 is a 0-1 ramp
    v3 3 0 pwl(0 0 1n 1)
    r3 3 0 1
    *
    * e1 illustrates use of sub-tables.  x is the voltage from v3
    e1 4 0 3 0 table(t2, x)
    * below is an alternative equivalent form for e1
    *e1 4 0 t2(v(3))
    r4 4 0 1
    .table t2 0 table t3 .5 table t4 .75 .75 1 0
    .table t3 0 0 .25 1 .5 0
    .table t4 0 0 .5 0 .625 1 .75 .75, x)
    *
    * e2 produces the same output as e1, but uses a PWL statement.
    * when the controlling nodes are given, pwl uses the control source,
    * and not time when used in e,f,g,h sources
    e2 5 0 3 0 pwl(0 0 .25 1 .5 0 .625 1 .75 .75 1 0)
    r5 5 0 1
    *
    .tran 1p 1n
    * type "run", then "plot v(1) v(2) v(3) v(4)"
    .end
\end{verbatim}

%SU-------------------------------------
\subsection{Circuit 7:  MOS Convergence Test}

% spExamples.hlp:ex7 012309

Below is an example circuit that illustrates some of the new features,
and some older features perhaps not widely appreciated.  This also
served as a test for improving MOS convergence.  To run a convergence
test:
\begin{enumerate}
\item{Source this circuit.}
\item{Press the {\cb siminterface} button in the Debug tool.}
\item{Type ``{\vt set value1 = mult}''.}
\item{Type ``{\vt loop .5 2.5 .1 op}''.}
\end{enumerate}

This runs operating point analysis for M values from .5 to 2.5, and
displays a little plot of the convergence process.  In the display,
`{\vt +}' means an increasing gmin step, `.' is a source step, `{\vt
-}' is a decreasing gmin step.

Note that 100 nS is too coarse to get a decent looking plot with BSIM
devices.  Use ``{\vt tran 1n 10u}'' in that case, since the circuit
seems to be oscillating at very high frequency.

\begin{verbatim}
    mosamp2 - mos amplifier - transient
    .options abstol=10n vntol=10n noopiter mult=2 nqs=1 reltol=1e-4
    *.op
    .tran 0.1us 10us
    .plot  tran v(20) v(66)
    * set below to 0 for old MOS model
    .param bsim = 1
    .if bsim = 1
    m1  15 15  1 32 n1 w=88.9u  l=25.4u  m=$mult nqsmod=$nqs
    m2   1  1  2 32 n1 w=12.7u  l=266.7u m=$mult nqsmod=$nqs
    m3   2  2 30 32 n1 w=88.9u  l=25.4u  m=$mult nqsmod=$nqs
    m4  15  5  4 32 n1 w=12.7u  l=106.7u m=$mult nqsmod=$nqs
    m5   4  4 30 32 n1 w=88.9u  l=12.7u  m=$mult nqsmod=$nqs
    m6  15 15  5 32 n1 w=44.5u  l=25.4u  m=$mult nqsmod=$nqs
    m7   5 20  8 32 n1 w=482.6u l=12.7u  m=$mult nqsmod=$nqs
    m8   8  2 30 32 n1 w=88.9u  l=25.4u  m=$mult nqsmod=$nqs
    m9  15 15  6 32 n1 w=44.5u  l=25.4u  m=$mult nqsmod=$nqs
    m10  6 21  8 32 n1 w=482.6u l=12.7u  m=$mult nqsmod=$nqs
    m11 15  6  7 32 n1 w=12.7u  l=106.7u m=$mult nqsmod=$nqs
    m12  7  4 30 32 n1 w=88.9u  l=12.7u  m=$mult nqsmod=$nqs
    m13 15 10  9 32 n1 w=139.7u l=12.7u  m=$mult nqsmod=$nqs
    m14  9 11 30 32 n1 w=139.7u l=12.7u  m=$mult nqsmod=$nqs
    m15 15 15 12 32 n1 w=12.7u  l=207.8u m=$mult nqsmod=$nqs
    m16 12 12 11 32 n1 w=54.1u  l=12.7u  m=$mult nqsmod=$nqs
    m17 11 11 30 32 n1 w=54.1u  l=12.7u  m=$mult nqsmod=$nqs
    m18 15 15 10 32 n1 w=12.7u  l=45.2u  m=$mult nqsmod=$nqs
    m19 10 12 13 32 n1 w=270.5u l=12.7u  m=$mult nqsmod=$nqs
    m20 13  7 30 32 n1 w=270.5u l=12.7u  m=$mult nqsmod=$nqs
    m21 15 10 14 32 n1 w=254u   l=12.7u  m=$mult nqsmod=$nqs
    m22 14 11 30 32 n1 w=241.3u l=12.7u  m=$mult nqsmod=$nqs
    m23 15 20 16 32 n1 w=19u    l=38.1u  m=$mult nqsmod=$nqs
    m24 16 14 30 32 n1 w=406.4u l=12.7u  m=$mult nqsmod=$nqs
    m25 15 15 20 32 n1 w=38.1u  l=42.7u  m=$mult nqsmod=$nqs
    m26 20 16 30 32 n1 w=381u   l=25.4u  m=$mult nqsmod=$nqs
    m27 20 15 66 32 n1 w=22.9u  l=7.6u   m=$mult nqsmod=$nqs
    .else
    m1  15 15  1 32 n1 w=88.9u  l=25.4u  m=$mult
    m2   1  1  2 32 n1 w=12.7u  l=266.7u m=$mult
    m3   2  2 30 32 n1 w=88.9u  l=25.4u  m=$mult
    m4  15  5  4 32 n1 w=12.7u  l=106.7u m=$mult
    m5   4  4 30 32 n1 w=88.9u  l=12.7u  m=$mult
    m6  15 15  5 32 n1 w=44.5u  l=25.4u  m=$mult
    m7   5 20  8 32 n1 w=482.6u l=12.7u  m=$mult
    m8   8  2 30 32 n1 w=88.9u  l=25.4u  m=$mult
    m9  15 15  6 32 n1 w=44.5u  l=25.4u  m=$mult
    m10  6 21  8 32 n1 w=482.6u l=12.7u  m=$mult
    m11 15  6  7 32 n1 w=12.7u  l=106.7u m=$mult
    m12  7  4 30 32 n1 w=88.9u  l=12.7u  m=$mult
    m13 15 10  9 32 n1 w=139.7u l=12.7u  m=$mult
    m14  9 11 30 32 n1 w=139.7u l=12.7u  m=$mult
    m15 15 15 12 32 n1 w=12.7u  l=207.8u m=$mult
    m16 12 12 11 32 n1 w=54.1u  l=12.7u  m=$mult
    m17 11 11 30 32 n1 w=54.1u  l=12.7u  m=$mult
    m18 15 15 10 32 n1 w=12.7u  l=45.2u  m=$mult
    m19 10 12 13 32 n1 w=270.5u l=12.7u  m=$mult
    m20 13  7 30 32 n1 w=270.5u l=12.7u  m=$mult
    m21 15 10 14 32 n1 w=254u   l=12.7u  m=$mult
    m22 14 11 30 32 n1 w=241.3u l=12.7u  m=$mult
    m23 15 20 16 32 n1 w=19u    l=38.1u  m=$mult
    m24 16 14 30 32 n1 w=406.4u l=12.7u  m=$mult
    m25 15 15 20 32 n1 w=38.1u  l=42.7u  m=$mult
    m26 20 16 30 32 n1 w=381u   l=25.4u  m=$mult
    m27 20 15 66 32 n1 w=22.9u  l=7.6u   m=$mult
    .endif
    cc 7 9 40pf
    cl 66 0 70pf
    vin 21 0 AC pulse(0 5 1ns 1ns 1ns 5us 10us)
    vccp 15 0 dc +15
    vddn 30 0 dc -15
    vb 32 0 dc -20
    .if bsim
    .model n1 nmos(level=8 capmod=3)
    .else
    .model n1 nmos(nsub=2.2e15 uo=575 ucrit=49k uexp=0.1 tox=0.11u xj=2.95u
    +   level=2 cgso=1.5n cgdo=1.5n cbd=4.5f cbs=4.5f ld=2.4485u nss=3.2e10
    +   kp=2e-5 phi=0.6 )
    .endif
\end{verbatim}

%SU-------------------------------------
\subsection{Circuit 8:  Verilog Pseudo-Random Sequence}

% spExamples.hlp:ex8 012309

This example illustrates use of a {\vt .verilog} block to generate a
digital signal, that is then interfaced to and processed by a
conventional SPICE circuit.  The digital signal is a 511 step
pseudo-random sequence, which is converted to a voltage and filtered.

\begin{verbatim}
    * WRspice pseudo-random bit sequence demo
    v1 1 0 a/255-1
    r1  1 2 100
    c1 2 0 10p
    .tran 1p 10n
    .plot tran v(1) v(2)

    .verilog
    module  prbs;
    reg [8:0] a, b;
    reg clk;
    integer cnt;

    initial
        begin
        a = 9'hff;
        clk = 0;
        cnt = 0;
        $monitor("%d", cnt, "%b", a, a[0]);
        end

    always
        #5 clk = ~clk;

    always
        @(posedge clk)
        begin
        a = { a[4]^a[0], a[8:1] };
        if (a ==  9'hff)
            $stop;
        cnt = cnt + 1;
        end

    endmodule
    .endv
\end{verbatim}

%SU-------------------------------------
\subsection{Circuit 9:  Josephson Junction I-V Curve}

% spExamples.hlp:ex9 012309

\begin{verbatim}
    WRspice jj I-V curve demo
    *
    * One can plot a pretty decent iv curve using transient analysis.
    * This will show the differences between the various model options.
    *
    b1 1 0 jj1 control=v2
    v1 2 0 pwl(0 0 2n 70m 4n 0 5n 0)
    r1 2 1 100
    *
    * for rtype=4, vary v2 between 0 and 1 for no gap to full gap
    v2 3 0 .5
    *
    r2 3 0 1
    *
    * It is interesting to set rtype and delv to different values, and note
    * the changes.
    *
    *Nb 1000 A/cm2   area = 30 square microns
    .model jj1 jj(rtype=4,cct=1,icon=10m,vg=2.8m,delv=.1m,
    + icrit=0.3m,r0=100,rn=5.4902,cap=1.14195p)
    .tran 5p 5n
    * type "run", then "plot -b v(1) (-v1#branch)"
    .end
\end{verbatim}

%SU-------------------------------------
\subsection{Circuit 10: Josephson Gap Potential Modulation}

% spExamples.hlp:ex10 012309

\begin{verbatim}
    WRspice jj qp modulation demo
    *
    * The rtype=4 option of the Josephson model causes the gap potential
    * to scale with the external "control current" absolute value.  For
    * unit control current (1 Amp) or larger, the full potential is used,
    * otherwise it scales linearly to zero.  The transfer function is defined
    * externally with controlled sources, as below.  The approximation
    * Vg = Vg0*(1-t**4) is pretty good, except near t = 1 (T = Tc, t = T/Tc).
    * The actual transfer function is left to the user - in the example below,
    * the ambient temperature is 7K, Tc=9.2K, and 1mv of "input" causes 1K
    * temperature shift.
    *
    * For amusement, change cct=1 to cct=0 below.  This runs much more quickly
    * as critical current is set to zero.
    *
    b1 1 0 jj1 control=v2
    v1 2 0 pulse(0 35m 10p 10p)
    r1 2 1 100
    *
    v2 3 0
    g1 3 0 4 0 function 1 - (1000*x+7)/9.2)^4
    v4 4 0 pulse(-1m 1m 10p 10p 10p 20p 60p)
    *
    *Nb 1000 A/cm2   area = 30 square microns
    .model jj1 jj(rtype=4,cct=1,icon=10m,vg=2.8m,delv=.1m,
    + icrit=0.3m,r0=100,rn=5.4902,cap=1.14195p)
    .tran 1p 500p uic
    * type "run", then plot v(1) and v(4) to see the gap shift and input
    .end
\end{verbatim}

