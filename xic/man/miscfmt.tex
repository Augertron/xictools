% -----------------------------------------------------------------------------
% Xic Manual
% (C) Copyright 2013, Whiteley Research Inc., Sunnyvale CA
% $Id: miscfmt.tex,v 1.4 2014/09/09 16:58:23 stevew Exp $
% -----------------------------------------------------------------------------

\chapter{Other File Formats}

% -----------------------------------------------------------------------------
% vectorfont 062908
\section{Label Font File Format}
\label{vectorfont}
\index{font file}
\index{vector font}
\index{label font}

The font used to render text labels in drawing windows is a vector
font for scalability.  The character maps have internal defaults,
which should be suitable in most cases, however these can be
overridden by external definitions from a file.  One can dump the
current set of character maps to a file with the {\cb Dump Vector
Font} button in the font setting panel available in the {\cb
Attributes Menu}.  Character maps from this file can be modified and
placed in a file named ``{\vt xic\_font}'' in the library search path,
in which case they will override the internal definitions when
producing label text.

The same default character maps are also used by default for the
vector font in the {\cb logo} command, for producing physical
characters with wire elements.  These too can be overridden by
definitions from a file.  The {\cb Dump Vector Font} button in the
setup panel of the {\cb logo} command can be used to dump the current
set of character maps to a file.  Character maps from this file can be
modified and placed in a file named ``{\vt xic\_logofont}'' in the
library search path, in which case they will override the internal
definitions when producing vector-based characters in the {\cb logo}
command.

The generated font file consists of vector specifications for the
characters `{\vt !}' through `{\vt \symbol{126}}' in the ASCII chart. 
The user's file need not contain all characters, missing characters
will use the internal default definitions.

The file consists of character specifications of the form described
below.  The first line of the specification defines the character. 
This is followed by one or more path vertex lists which define the
``strokes'' of the character.  These are followed by a couple of
numerical entries which affect placement.  For example, the entry for
the default exclamation point (!) appears as:

\begin{quote}\vt
character !\\
path 4,2 4,7\\
path 4,9 4,10\\
width 2\\
offset 4\\
\end{quote}

The coordinate system has its origin in the upper left corner.  The
size is limited to 256 X 256, but the basic cell size used by the
default set is 7 X 14.  The y values increase downward, and x values
increase to the right.  Negative values are not permitted.

Only the first character of the leading keyword is necessary, and this
is case insensitive.  The first line of the block defines the
character.  The order of the following lines is unimportant.  Each
path is a sequence of coordinates which render a part of the
character.  The {\vt width} is the horizontal space provided for the
character, which should include trailing space, typically one column. 
The {\vt offset} is the column which is placed at the end of the
preceding character.  Row and column numbering begin with 0.

 
% -----------------------------------------------------------------------------
% labelflags 022713
\section{Label Flags}
\label{labelflags}
\index{label flags}

Internally, every {\Xic} text label object has a set of flags which
control presentation and other attributes of the label.  The flags are
visible in the label specifications in native cell files and default
extended CIF files (see \ref{cifext}).  It is also used with script
functions ({\vt GetLabelFlags}, {\vt SetLabelFlags}, {\vt Label}, {\vt
LabelH}) and the {\et XprpXform} pseudo-property (see
\ref{pseudoprops}).

\begin{tabular}{lll}\\
\bf Bits & \bf Hex & \bf Effect\\
0-1  & 0003 & text rotation angle\\
     & 0000 & no rotation\\
     & 0001 & 90 degrees\\
     & 0002 & 180 degrees\\
     & 0003 & 270 degrees\\
2    & 0004 & mirror Y after rotation\\
3    & 0008 & mirror X after rotation and mirror Y\\
4    & 0010 & shift rotations to 45, 135, 225, 315 degrees\\
5-6  & 0060 & horizontal justification\\
     & 0000 & left justification\\
     & 0020 & center horizontal justification\\
     & 0040 & right justification\\
     & 0060 & right justification\\
7-8  & 0180 & vertical justification\\
     & 0000 & bottom justification\\
     & 0080 & center vertical justification\\
     & 0100 & top justification\\
     & 0180 & top justification\\
9-10 & 0600 & font selection (unused)\\
11   & 0800 & unused, reserved\\
12   & 1000 & show text\\
13   & 2000 & hide text\\
14   & 4000 & show only when container is current cell\\
15   & 8000 & limit number of lines displayed\\
\end{tabular}

See the discussion of the {\vt XprpXform} pseudo-property in
\ref{pseudoprops} for more information on the effects of these flags.


% -----------------------------------------------------------------------------
% helpfiles 090814
\section{Help Database Files}
\label{helpfiles}
\index{help mode!file format}

\index{XIC\_HLP\_PATH}
The help information is obtained from database files suffixed with
{\vt .hlp} found along the help search path.  These directories may
also contain other files referenced in the help text, such as image
files.  In {\Xic}, the help search path can be set in the environment
with the variable {\et XIC\_HLP\_PATH}, and/or may be set in the
technology file (the technology file overrides the environment). 
These files have a simple format allowing users to create and modify
them.  Each help entry is associated with one or more keywords, which
should be unique in the database.  A warning message will be issued on
{\vt stderr} if a name clash is detected.  The files are ASCII text,
either in DOS or Unix format.  Fields are separated by keywords which
begin with ``{\vt !!}''.  Although the help system provides rich-text
presentation from HTML formatting, entries can be in plain text.  A
sample plain-text entry has the form:

\begin{quote}
\begin{verbatim}
!!KEYWORD
excmd
!!TITLE
Example Command
!!TEXT
    This command exists only in this example.  Note that the
    !!keywords only have effect if they start in the first
    column.  The blank line below is optional.

!!SUBTOPICS
akeyword
anotherkeyword
!!SEEALSO
yetanotherkeyword
\end{verbatim}
\end{quote}

In this example, the keyword ``{\vt excmd}'' is used to access the
topic, and should be unique among the database entries accessed by the
application.  The text which appears in the topic (following {\vt
!!TEXT}) is shown indented, which is recommended for clarity, but is
not required.

In {\vt .hlp} files, outside of {\vt !!TEXT} and {\vt !!HTML} blocks
(described below), lines with `{\vt *}' or `{\vt \#}' in the first
column are ignored, as they are assumed to be comments.  Lines that
begin in the first column with ``{\vt !!}(space)'' (space character
following two exclamation points) anywhere are also ignored, as
comments.  Blank lines outside of the {\vt !!TEXT} and {\vt !!HTML}
fields are ignored.  Leading white space is stripped from all lines
read, which can be a problem for maintaining indentation in formatted
plain text.  To add a space which will not be stripped, one can use
the HTML escape ``{\vt \&\#32;}''.

The following `{\vt !!}' keywords can appear in {\vt .hlp} files. 
These are recognized only in upper case, and must start in the first
text column.

\begin{description}
\item{\vt !!}(space) {\it anything}\\
A line beginning with two exclamation points followed by a space
character is ignored.

\item{\vt !!KEYWORD {\it keyword-list}}\\
This keyword signals the start of a new topic.  The {\it keyword-list}
consists of one or more tokens, each of which must be unique among all
topics in the database.  The words are used to identify the topic, and
if more than one is listed, the additional words are equivalent
aliases.  The {\it keyword-list} may follow {\vt !!KEYWORD} on the
same line, or may be listed in the following line, in which case {\vt
!!KEYWORD} should appear alone on the line.

Punctuation is allowed in keywords, only white space characters can
not be used.  The `{\vt \#}' character has special meaning and should
not be part of a keyword name.  Also, character sequences that could
be confused with a URL or directory path should be avoided.  The
latter basically prohibits the `{\vt /}' character (and also
`$\backslash$' under Windows) from being included in keywords.  There
are special names starting with `{\vt \$}' which are expanded to
application-specific internal variables, as described below.  To avoid
any possibility of a clash, it is probably best to avoid `{\vt \$}' in
general keywords.

It is often useful to include a meaningful prefix in keywords to
ensure uniqueness, for example in {\Xic}, all commands have keywords
prefixed with ``{\vt xic:}''.

\item{\vt !!TITLE {\it string}}\\
The {\vt !!TITLE} specifies the title of the topic, and should follow
the {\vt !!KEYWORD} specification.  The title text can appear on the
same line following {\vt !!TITLE}, or on the next line, in which case
{\vt !!TITLE} should appear alone in the line.  The title is printed
at the top of the topic display, and is used in menus of topics.

\item{\vt !!TEXT}\\
This line signals the beginning of the topic text, which is expected
to be plain text.  The keyword is mutually exclusive with the {\vt
!!HTML} keyword.  The lines following {\vt !!TEXT} up to the next {\vt
!!KEYWORD}, {\vt !!SEEALSO}, or {\vt !!SUBTOPICS} line or end of file
are read into the display window.  The plain text is converted to HTML
before being sent to the display in the following manner:

\begin{enumerate}\rr
\item The title text is enclosed in {\vt <H1>...</H1>}.
\item Each line of text has a {\vt <BR>} appended.
\item The subtopics and see-alsos are preceded with added
    {\vt <H3>Subtopics</H3>} and {\vt <H3>References</H3>} lines.
\item The subtopics and see-alsos are converted to links of the form
    {\vt <A HREF="{\it keyword\/}">{\it title\/}</A>} where the {\it
    keyword} is the database keyword, and the {\it title} is the title
    text for the entry.
\end{enumerate}

Note that the text area can contain HTML tags for various things, such
as images.  Also note that text formatting is taken from the help file
(the {\vt <BR>} breaks lines), and not reformatted at display time. 
The {\vt !!HTML} line should be used rather than {\vt !!TEXT} if the
text requires full HTML formatting.

\item{\vt !!HTML}\\
This line signals the beginning of the topic text, which is expected
to be HTML-formatted.  The keyword is mutually exclusive with the {\vt
!!TEXT} keyword.  The parser understands all of the standard HTML 3.2
syntax, and a few 4.0 extensions.  References are to keywords found in
the database and general URLs.  Image ({\vt .gif}, etc.) files can be
referenced, and are expected to be found along with the {\vt .hlp}
files.

\item{\vt !!IFDEF {\it word}}\\
This line can appear in the block of text following {\vt !!TEXT} or
{\vt !!HTML}.  In conjunction with the {\vt !!ELSE} and {\vt !!ENDIF}
directives, it allows for the conditional inclusion of blocks of text
in the topic.  The {\it word} is one of the special words defined by
the application.  Presently, the following words are defined:

\begin{description}
\item{\vt Xic}\\
Defined when running the {\Xic} program with any feature set.
\item{\vt XicII}
Defined when running the {\Xic} program with the {\XicII} feature set.
\item{\vt Xiv}\\
Defined when running the {\Xic} program with the {\Xiv} feature set.
\item{\vt WRspice}\\
Defined when running the {\WRspice} program. 
\item{\vt Windows}\\
Defined when running under Microsoft Windows.
\end{description}

If {\it word} is defined, the text up to the next {\vt !!ELSE} or {\vt
!!ENDIF} will be included in the topic, and any text following an {\vt
!!ELSE} up to {\vt !!ENDIF} is discarded.  If {\it word} is not
defined, the text up to the next {\vt !!ELSE} or {\vt !!ENDIF} is
discarded, and any text following an {\vt !!ELSE} is included.  The
constructs can be nested.  A word that is not recognized or absent is
``not defined''.  Every {\vt !!IFDEF} should have a corresponding {\vt
!!ENDIF}.  The {\vt !!ELSE} is optional.  The {\vt !!SEEALSO} and {\vt
!!SUBTOPICS} lines can appear within the blocks.

Example:
\begin{verbatim}
!!HTML
   Here is some text.
!!IFDEF Xic
   You are reading this in Xic.
!!ELSE
!!IFDEF WRspice
   You are reading this in WRspice.
!!ELSE
   You are not reading this in Xic or WRspice.
!!ENDIF
!!ENDIF
\end{verbatim}

\item{\vt !!IFNDEF {\it word}}\\
This keyword can appear in the block of text following {\vt !!TEXT} or
{\vt !!HTML}.  It is similar to {\vt !!IFDEF} but has the reverse
logic.

\item{\vt !!ELSE}\\
This keyword can follow {\vt !!IFDEF} or {\vt !!IFNDEF} and defines
the start of a block of text to include in the topic if the condition
is not satisfied.

\item{\vt !!ENDIF}\\
This keyword terminates the text blocks to be conditionally included
in the topic, using {\vt !!IFDEF} or {\vt !!IFNDEF}.

\item{\vt !!INCLUDE {\it filename}}\\
The keyword may appear in the text following {\vt !!TEXT} or {\vt
!!HTML}.  When encountered in the text to be included in the topic,
the text of {\it filename}, which is searched for in the help search
path if not an absolute pathname, is added to the displayed text of
the current topic.  There is no modification of the text from {\it
filename}.

If the filename is a relative path to a subdirectory of one of the
directories of a directory in the help search path, the subdirectory
is added to the search list.  Thus, an HTML document and associated
gif files can be placed in a separate subdirectory in the help tree. 
The HTML document can be referenced from the main help files with a
{\vt !!INCLUDE} directive, and there is no need to explicitly change
the help search path.

\item{\vt !!REDIRECT {\vt keyword target}}\\
This will define {\it keyword} as an alias for {\it target}.  The {\it
target} can be any input token recognizable by the help system,
including URLs, named anchors, and local files.  For example:
\begin{quote}
{\vt !!REDIRECT nyt http://www.nytimes.com}
\end{quote}
giving ``{\vt !help nyt}'' in {\Xic} or ``{\vt help nyt}'' in
{\WRspice} will bring up a help window containing the New York Times
web page.

\item{\vt !!SEEALSO {\it keyword-list}}\\
This keyword, if used, is expected to be found at the end of the topic
text.  The {\it keyword-list} consists of a list of keywords that are
expected to be defined by {\vt !!KEYWORD} lines elsewhere in the
database.  A menu of these items is displayed at the bottom of the
topic text, under the heading ``References''.  The keywords specified
after {\vt !!SEEALSO} can appear on the same line separated with
space, or on multiple lines that follow.  If a keyword in the list is
not found in the database, it is silently ignored.  The keywords
listed {\it must} be given in a {\vt !!KEYWORD} line, and not contain
named anchor references (violating entries are silently ignored).

\item{\vt !!SUBTOPICS {\it keyword-list}}\\
This keyword, if used, is expected to be found at the end of the topic
text.  This produces a menu of the topics found in the {\it
keyword-list} very similar to {\vt !!SEEALSO}, however under the
heading ``Subtopics''.  This can be used in addition to {\vt
!!SEEALSO}, the order is unimportant.
\end{description}

The following definitions supply header and footer text which will be
applied to each page.  These should be defined at most once each in
the database.

\begin{description}
\item{\vt !!HEADER}\\
The text that follows, up until the next {\vt !!KEYWORD} or {\vt
!!FOOTER}, is saved for inclusion in each page composed from the {\vt
!!HTML} lines for database keywords.  The header is inserted at the
top of the page.  There can be only one header defined, and if more
than one are found in the help files, the first one read will be used.

In the header text, the literal token {\vt \%TITLE\%} is replaced with
the {\vt !!TITLE} text of the current topic when displayed.

\item{\vt !!FOOTER}\\
The text that follows, up until the next {\vt !!KEYWORD} or {\vt
!!HEADER}, is saved for inclusion in each page composed from the {\vt
!!HTML} lines for database keywords.  The footer is inserted at the
bottom of the page.  There can be only one footer defined, and if more
than one are found in the help files, the first one read will be used.
\end{description}

The following keywords inplement a means to mark topics that are from
imported or supplemental files.  For example, in {\Xic}, many of the
{\WRspice} help files are included for reference and to satisfy links
in the {\Xic} help files.  There is a need to mark these pages as
applying to the {\WRspice} program, otherwise the information could be
confusing.  In the {\Xic} help system, the pages from {\WRspice} have
a banner just below the header identifying the topic as applying to
{\WRspice}.

\begin{description}
\item{\vt !!MAINTAG} {\it tagname}\\
This keyword should appear once in the database, probably defined
along with the header/footer.  The {\it tagname} is an arbitrary short
keyword which identifies the database, such as ``{\vt Xic}''.

\item{\vt !!TAG} {\it tagname}\\
This should be given at the top of each help file in the database. 
Those files that are part if the main database should have the same
{\it tagname} as was given to {\vt !!MAINTAG}.  Files containing
supplemental information should have some other {\it tagname}, e.g.,
``{\vt WRspice}''

\item{\vt !!TAGTEXT} {\it tagname}\\
This should be given once only in the database, probably where the
{\vt !!MAINTAG} is defined.  It is followed by HTML text, in the
manner of the header and footer.  This text will be inserted just
below the header in topic pages that come from files with tags that
differ from the main tag.  For this to happen, both the tag and main
tag must have been defined.  In the text, the token ``{\vt \%TAG\%}''
will be replaced with the actual tag that applies to the topic.
\end{description}

\subsection{Anchor Text}

Clickable references in the HTML text have the usual form:
\begin{quote}
{\vt <a href="{\it something}">{\it highlighted text}</a>}
\end{quote}
Here, ``{\it something}'' can be a help database keyword or an
ordinary URL.

One can use named anchors in help keywords.  This means that the `{\vt
\#}' symbol is holy, and should not be used in help keywords.  The
named anchors can appear in the {\vt !!HTML} part of the help database
entries in the usual HTML way, e.g.

\begin{verbatim}
!!KEYWORD
somekeyword
...
!!HTML
    ...
    <a name="refname">some text</a>
\end{verbatim}

\begin{flushleft}
Then, referencing forms like ``{\vt !help somekeyword\#refname}'' and
{\vt <a href="somekeyword\#refname">blather</a>} will bring up the
``somekeyword'' topic, but with ``some text'' at the top of the help
window, rather than the start of the document.
\end{flushleft}

There is an additional capability:  `{\vt \$}' expansion.  Words found
in anchor text that begin with a dollar sign (`{\vt \$}') character
may be replaced by either a path related to the program, the value of
a variable saved in the program, or the value of an environment
variable.  The character that immediately follows the word can not be
alphanumeric.

This replacement is handled by a callback to the application, but both
{\Xic} (and its derivatives) and {\WRspice} support the following
keywords and behavior.

\begin{description}
\item{\vt \$PROGROOT}\\
This word is replaced by the full path to the program installation
directory, for example\\ ``{\vt /usr/local/xictools/xic}''.
\item{\vt \$HELP}\\
This word is replaced by {\vt \$PROGROOT/help}, meaning the same
directory as {\vt \$PROGROOT} suffixed with {\vt /help}.
\item{\vt \$EXAMPLES}\\
This word is replaced by {\vt \$PROGROOT/examples}, as above.
\item{\vt \$DOCS}\\
This word is replaced by {\vt \$PROGROOT/docs}, as above.
\item{\vt \$SCRIPTS}\\
This word is replaced by {\vt \$PROGROOT/scripts}, as above.
\end{description}

If there is no match to these words, the word, without the dollar
sign, is checked against the variable database.  If a variable is set
with the same name, the string value of the variable replaces the
word.  If there is no match, but the word without the dollar sign
matches the name of an environment variable, the value of the
environment variable will replace the word.  If there is no match,
there is no substitution.  Substitutions are evaluated recursively.

If the first character of an anchor URL is `{\vt \symbol{126}}', the
path is tilde expanded.  This is done after `{\vt \$}' substitution. 
Tildes denote a user's home directory:  ``{\vt \symbol{126}/mydir}''
might expand to ``{\vt /home/yourhome/mydir}'', and ``{\vt
\symbol{126}joe/joesdir}'' might expand to ``{\vt
/home/joe/joesdir}'', etc.

In {\Xic}, one can open input files from anchor text in the HTML
viewer.  The type of file is recognized by the suffix.  These are:

\begin{tabular}{ll}
CGX    & {\vt .cgx} (.gz may follow)\\
GDSII  & {\vt .gds, .str, .strm, .stream}  (.gz may follow)\\
OASIS  & {\vt .oas}\\
CIF    & {\vt .cif}\\
Xic    & {\vt .xic}\\
\end{tabular}

The anchor text to open a cell can actually have the following syntax. 
It can consist of up to three space-separated words.
\begin{quote}
[{\it sourcetype\/}] {\it sourcename} [{\it cellname\/}]
\end{quote}

The optional {\it sourcetype} can be one of the following literal
tokens.
\begin{description}
\item{\vt @XIC}\\
The {\it sourcename} is the name of a native cell existing either in
memory or in the search path for cell files, or the name may contain a
path to the file.  The {\it cellname} word is not used.

\item{\vt @CHD}\\
The {\it sourcename} will provide the database name of a cell
hierarchy digest.  The {\it cellname} if used provides the name of a
cell to open.  If not given, the CHD's default cell will be opened.

\item{\vt @LIB}\\
The {\it sourcename} is a path to an {\Xic} library file, and the {\it
cellname} is the name of a reference or cell in the library.

\item{\vt @OA}\\
The {\it sourcename} is the name of an OpenAccess library, and the
{\it cellname}, which is required, is the name of a cell in the
library.
\end{description}

If no {\it sourcetype} is given, the file type is determined by the
file extension, as listed above.  The optional {\it cellname} can
specify the name of a cell to open.

In addition, if the {\it sourcename} has a {\vt .scr} suffix, it is
taken to be a script file, and is executed.  Thus, one can execute
{\Xic} scripts by clicking on an anchor.  The referenced script is
expected to be found somewhere in the script path, or be defined in
the technology file, if a rooted file path is not provided.

{\bf Examples:}\\
One can actually load a layout from another machine.
\begin{quote}\vt
Click <a href="http://somewhere/lib/cell.gds">here</a> to view the
design.
\end{quote}

A second argument can specify the cell to open.  The quoting is
required in this case.
\begin{quote}\vt
Click <a href="/usr/joe/library/joeslayout.gds joescell">here</a> to
view Joe's cell.
\end{quote}

Unless the native cell happens to have a {\vt .xic} file name
extension, one should use the magic word.
\begin{quote}\vt
Click <a href="@XIC mynativecell">here</a> to view my native cell.
\end{quote}

If the OpenAccess plug-in is loaded, one can access cells from
OpenAccess libraries.
\begin{quote}\vt
Click <a href="@OA oalibrary oacell">here</a> to view my OpenAccess
cell.
\end{quote}

Finally, to execute a script when the user clicks on the link:
\begin{quote}\vt
Click <a href="myscript.scr">here</a> to execute myscript.
\end{quote}
The script {\vt myscript.scr} must exist somewhere in the script path,
or be defined in the technology file.  When the user clicks on
``here'', this script will be executed.

In {\WRspice}, a similar capability exists.  One can source files from
anchor text in the HTML viewer, if the anchor text consists of a file
name with a {\vt .cir} extension.  Thus, if one has a circuit file
named {\vt mycircuit.cir}, and the HTML text in the help window
contains a reference like
\begin{quote}\vt
<a html="mycircuit.cir">click here</a>
\end{quote}
then clicking on the ``click here'' tag will source {\vt
mycircuit.cir} into {\WRspice}.  Similarly, anchor references to files
with a {\vt .raw} extension will be loaded into {\WRspice} as a
{\it rawfile\/}, i.e., a plot data file, when the anchor is clicked.

