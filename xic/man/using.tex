% -----------------------------------------------------------------------------
% Xic Manual
% (C) Copyright 2009, Whiteley Research Inc., Sunnyvale CA
% $Id: using.tex,v 1.64 2017/03/22 07:29:56 stevew Exp $
% -----------------------------------------------------------------------------

% -----------------------------------------------------------------------------
% usingxic 022816
\chapter{Using {\Xic}}
\index{physical mode}
\index{electrical mode}
{\Xic} has two basic operating modes: physical and electrical.  In
physical mode, one is editing the geometry of the mask patterns on the
multiple layers used in the photomasks to manufacture the circuit.  In
electrical mode, one is editing an electrical schematic of the circuit
or subcircuit represented by the cell.  The schematic is used for
documentation, and also for performing simulation of the circuit to
verify performance.  The schematic and layout can be interlinked to
provide consistency verification.  This is the purpose of the functions
in the {\cb Extract Menu}, to be described in Chapter \ref{chpext}.

\index{hierarchy of cells}
A full design database typically consists of a hierarchy of cells. 
The top level or main cell usually depicts the entire chip.  Subcells
represent the bond pads, annotation, and major circuit blocks.  The
circuit blocks in turn have subcells representing more primitive
circuit blocks, down to the gate level and below.

In {\Xic}, one can edit any of these cells and their subcells at any
depth in the hierarchy, as both physical layout and electrical
schematic.  The use of a hierarchical database is far more efficient
and convenient than a flat database.  The designer is encouraged to
make liberal use of subcells rather than designing single, highly
complex cells.

When a design is complete, i.e., when all electrical simulations and
physical design rule checks have been performed, the physical part of
the database can be submitted for processing.  The exact mechanism
varies with organization, but the physical-only ({\cb Strip For
Export} button in the {\cb Export Control} panel from the {\cb
Convert Menu} active) GDSII, OASIS and CIF outputs provided by {\Xic}
are portable to any mask fabrication facility or foundry.

The user can switch between physical and electrical modes at any time,
by pressing the {\cb Electrical} or {\cb Physical} button (whichever
appears) in the {\cb View Menu}.  Sub-windows, brought up with the {\cb
Viewport} button in the {\cb View Menu}, are individually switchable
between schematic and physical views.  The side menus differ somewhat
between the two modes, and some menu commands operate a little
differently.

The next two sections of this chapter provide an introduction to
editing in physical and electrical modes.  The remaining sections
provide information on certain {\Xic} operation modes and features,
and are somewhat more advanced in nature.  The following chapters
provide detailed information on all of the menu command functions.

The new user should read the first two sections of this chapter, and
practice using {\Xic} while reading the help messages.


% -----------------------------------------------------------------------------
% physedit 020615
\section{Physical Layout Editing}

In physical mode, one arranges geometrical shapes on the various
layers to produce a working circuit.  One can also place subcells,
which have been previously created.  The knowledge of what shapes to
place, and where, is dependent on the technology in use, and
represents the essence of integrated circuit engineering.  The user
must be familiar with these fundamentals, as {\Xic} is only a tool for
application of this knowledge.

\index{box}
\index{box!merge}
\index{box button}
The basic primitive used by {\Xic} is the box.  Boxes are filled
rectangular structures representing an area of opacity on the
corresponding mask level.  The {\cb box} button in the side menu, with
the rectangular icon, is used to create boxes.  With the {\cb box}
button active, the user points to the two diagonal corners of the box
desired in the drawing window, and a colored box will appear.  The
color and fill pattern are set for each layer in the technology file,
and can be changed by the user with the {\cb Set Color} and {\cb Set
Fill} buttons in the {\cb Attributes Menu}.  The layer can be selected
by clicking on the desired layer in the layer table, which is arrayed
near the bottom of the main {\Xic} window.  Note that when boxes
created on the same level overlap, they are clipped or merged so as to
not actually overlap.  This increases the storage and retrieval
efficiency of the database.

If the created box is too small or otherwise causes a design rule
violation, a message will appear, if interactive rule checking is
active.  By default, all objects are checked for design rule
violations when they are added to the database, though this can be set
otherwise in the technology file or if the {\cb Set Interactive}
button in the {\cb DRC Menu} is not active.  Objects that ``fail'' are
actually in the database, and it is the responsibility of the user to
correct the error when it is flagged.

\index{wires}
\index{wire button}
Boxes can be used exclusively to create a working circuit, however
other structures are sometimes more convenient.  Wires are fixed-width
paths that are often used to make electrical connections.  The {\cb
wire} button in the side menu allows the creation of wires, and the
{\cb style} button can be used to change or set the wire width and end
style.  The {\cb wire} button has a sideways L-shaped icon.  Every
layer has a default wire width.  To construct a wire, simply click on
the points of the drawing window which correspond to wire vertices,
and click the last vertex twice to end the wire.  Note that the wire
can zigzag at any angle, however the angles can be fixed to multiples
of 45 degrees by setting the {\cb Constrain angles to 45 degree
multiples} check box in the {\cb Editing Setup} panel from the {\cb
Edit Menu}.  Also note that acute angles will most likely cause a
design rule violation message to appear.

\index{polygons}
\index{polyg button}
Polygons are constructed in a manner similar to wires, using the {\cb
polyg} button in the side menu.  This button has a triangle icon.  The
polygon is constructed by clicking at each desired vertex location,
and is terminated by clicking again on the first vertex.  Polygons can
have edges with arbitrary angles, which can be constrained to
multiples of 45 degrees with the {\cb Constrain angles to 45 degree
multiples} check box in the {\cb Editing Setup} panel.  Again, acute
angles are likely to cause design rule violations.  Polygons are most
useful for constructing rounded or off-angle shapes used in high
frequency circuits.  It is also slightly more efficient to use
polygons rather than a collection of boxes.

\index{selections}
With none of the geometry-creating buttons active, clicking on an
object can cause it to be ``selected''.  Only objects on layers that
are selectable, as shown in the layer table, can be selected.  A
selected object will be outlined with a flashing highlight.  Selected
objects are used by many of the other commands.  An object can be
deselected by clicking on it a second time.  The status window below
the layer table will indicate the number of objects selected. 
Multiple objects can be selected at once by pressing and holding
button 1, dragging the pointer, and releasing.  A ghost-drawn
rectangle will appear during this operation.  Objects which overlap
this rectangle will be selected (or deselected if already selected). 
All selected objects can be deselected with the {\cb desel} button in
the top button menu (above the main drawing window).

\index{object deletion}
\index{deleting objects}
Once selected, an object can be deleted, either by pressing the {\kb
Delete} key, or by pressing the {\cb Delete} button in the {\cb Modify
Menu}.  The objects will disappear from the screen, and the database.

\index{undo}
\index{redo}
\index{undo button}
\index{redo button}
Almost any operation which modifies the database can be undone with
the {\cb Undo} button in the {\cb Modify Menu}, which is equivalent to
pressing the {\kb Tab} key.  The last 25 operation are saved, and can
be undone.  The {\cb Redo} button, or equivalently {\kb Shift-Tab}
will redo the last undo.  All of the undone operations are saved in
the redo list, however the redo list is cleared after each new
operation that is not an undo.

\index{strch button}
The {\cb Stretch} button in the {\cb Modify Menu} is used to modify the
shapes or sizes of boxes, polygons, wires, and labels.  By pointing at
the edge or corner of a box, one can move that edge or corner to a new
location.  Similarly, polygon and wire vertices can be moved. 
Polygons and wires can also be modified with the vertex editor built
into the {\cb polyg} and {\cb wire} command buttons.  If a polygon or
wire is selected before pressing the corresponding command button, the
vertices of the selected object will be marked.  The selected vertices
can be deleted or moved, and new vertices added.

\index{erase button}
The {\cb erase} button in the side menu has an icon consisting of a
box with a corner missing.  This button is used to delete parts of
objects.  One clicks twice, or presses and drags, to define a
rectangle, which is ghost-drawn during the operation.  This
rectangular area will be cleared of fill from any box, polygon, or
wire.  Wires may not be entirely erased, as they are only cut at
points where the central path crosses the erase box boundary.

The user may have already designed one or more cells using {\Xic},
which are then available for use as subcells in the present layout. 
Subcells are called and placed with the {\cb place} command button in
the side menu.  After pressing the {\cb place} button, the {\cb Cell
Placement Control} pop-up will appear, which allows the user to select
a cell to place from cells that have been placed previously, or to
enter a new cell name to place.  The cell name can be dragged from the
{\cb File Selection} panel or from the list pop-ups in the {\cb File
Menu}.  In addition, the {\cb List} pop-ups contain a {\cb Place}
button which will also set the name of the current ``master'' cell to
be placed, and pop up the {\cb Cell Placement Control} pop-up if it is
not already visible.  When the {\cb Place} button in the {\cb Cell
Placement Control} pop-up is active, the current ``master'' will be
``attached'' to the mouse pointer, and instances will be placed at
locations where the user clicks with mouse button 1 in the drawing. 
There is provision in the {\cb Cell Placement Control} pop-up to
define array parameters, so that an array of instances will be created
rather than a single instance.  The placement mode can be exited by
pressing the {\kb Esc} key, or by unsetting the {\cb Place} button in
the {\cb Cell Placement Control} pop-up.

Once a physical layout is substantially complete, the layout is a
candidate for batch design rule checking and extraction.  These
capabilities are described in detail in later chapters.

This brief introduction should convey the flavor of using {\Xic} in
physical mode.  There are many more commands, and some of the commands
introduced have additional features not mentioned.  The best way to
learn {\Xic} is to use it, and read the on-line help available for the
commands.  After pressing the {\cb Help} button in the {\cb Help Menu},
pressing any command button will bring up a help screen describing the
command.  Reading the help and then trying the operation is the
fastest way to learn.  The help mode, and any command, can ge exited
by pressing the {\kb Esc} key.


% -----------------------------------------------------------------------------
% elecedit 040316
\section{Electrical Schematic Editing}
\index{electrical mode editing}

The eclectical mode of {\Xic} allows a schematic representation of the
cell to be entered.  This electrical representation is used to
generate a SPICE file for simulation purposes, by {\WRspice} or
another simulator.  The electrical representation can be generated or
updated from the physical layout, if extraction has been properly set
up, and can be compared with the physical representation to identify
wiring errors.

\index{cell hierarchy}
The electrical representation of a hierarchy of cells follows the same
hierarchy as the physical cells, for the most part.  Physical cells
that contain wire only, i.e., no devices or subcircuits, generally do
not have an electrical-mode counterpart.  Such cells are effectively
flattened into their parents in the electrical representation.  The
physical implementation of devices can include structure from
subcells.  In this case, the electrical implementation of the device
is in the electrical cell corresponding to the top-level physical cell
containing the device geometry.

One does not need a physical representation in order to use electrical
mode.  In this case, {\Xic} is used exclusively as a schematic capture
front-end for {\WRspice} or another SPICE-compatible simulator.

This section will focus on the mechanics of schematic entry and
simulation using {\WRspice}.  The chapter on extraction (\ref{chpext})
will provide detail on how the electrical and physical data can be
made to interact.

To produce a schematic cell, one follows this basic outline:
\begin{enumerate}
\item{Devices from the device menu or some other source are placed at
  various locations in the drawing.  Also, subcircuits from the user's
  library are similarly added to the drawing.}

\item{The devices and subcircuits are wired together.}

\item{Properties are given to the devices, which designate component
  values, models referenced, or other information.}

\item{If the cell is to be used as a subcircuit in another schematic,
  connection points are added, and possibly a symbolic representation
  defined.}

\item{A SPICE file representing the present hierarchy can be generated
  at this point, or, if the circuit is top-level (not used as a
  subcircuit) interactive simulation using {\WRspice} is possible.}
\end{enumerate}

The following sections will describe these steps in more detail.

A prerequisite for using electrical mode is basic knowledge of the
SPICE syntax and SPICE file format.  One does not need to be an
expert, but a certain proficiency is assumed for such steps as
property setting.  It is recommended that users unfamiliar with SPICE
skim the {\WRspice} manual or other reference.

% 040316
\subsection{Placement of Devices and Subcircuits}
\index{device placement}

{\Xic} is distributed with a representative device library, which is
contained in a file named {\vt device.lib} found in the installation
startup directory.  This contains most if not all of the devices
supported by {\WRspice}, however it may be necessary to customize this
file to the user's unique requirements.  The format of this file is
described in the appendix.  The devices found in the device library
file are those listed in the device menu, which is available while in
electrical mode.

\ifoa
Devices can also be supplied in cell files, or from an OpenAccess
database.  For example, it is feasible to use devices from the {\vt
analogLib} library from a Virtuoso installation, or from a foundry
design kit.
\fi

\index{device menu}
{\Xic} usually starts in physical mode, though if given the {\et -E}
option on the command line {\Xic} will start in electrical mode.  To
switch from physical to electrical mode, press the {\cb Electrical}
button in the {\cb View Menu}.  {\Xic} will reconfigure the side menu,
and display the schematic for the current cell (if any).  Pressing the
{\cb devs} button in the side menu will bring up a device menu which
extends across the top of the main {\Xic} window.  There are two
styles of device menu available.  The default menu consists of an
array of lettered buttons.  Pressing button 1 while the pointer is
over one of these buttons will cause a drop-down menu to appear, which
consists of more buttons containing device names.  The first letter of
these devices is that on the original button.  A device can be
selected by releasing button 1 while the pointer is over the desired
button.

A second device menu style consists of panels containing the names and
schematic symbols of the various devices with perhaps a button with a
right-pointing arrow, if the selections do not entirely fit on-screen. 
Clicking on the arrow button will show the devices which did not fit
in the initial menu.  This menu has the disadvantage of occupying a
lot of screen space, but it may be easier for new users.

Both menu styles contain a button that switches to the other style of
menu.  The present style will be used until changed by the user.  The
style used is completely arbitrary, and simply a user-preference.

Clicking on one of the device panels in the pictorial menu, or
releasing button 1 on a selection in the pull-down menu will attach
the schematic symbol to the mouse pointer.  Then clicking in the
drawing window will leave instances of that device at those locations. 
Press {\kb Esc} to exit this mode.  This is the means by which devices
are added to the circuit.  New devices can also be produced by using a
copy operation (a button 1 operation, or explicitly using the {\cb
Copy} command in the {\cb Modify Menu}) from an existing device in the
circuit.

\index{subcircuit placement}
\index{Place button}
The user may have already designed one or more circuits using {\Xic},
which are then available for use as subcircuits in the present
schematic.  The details of how to create a ``true'' subcircuit will be
presented shortly; for now, assume that such cells already exist. 
Subcircuits are called and placed with the {\cb place} command in the
side menu, in the same manner as subcells in physical mode. 
After pressing the {\cb place} button, the {\cb Cell Placement
Control} pop-up will appear, which allows the user to select a cell to
place from cells that have been placed previously, or to enter a new
cell name to place.  The cell name can be dragged from the {\cb File
Selection} panel or from the List pop-ups in the {\cb File Menu}.  In
addition, the {\cb List} pop-ups contain a {\cb Place} button which
will also set the name of the current ``master'' cell to be placed,
and pop up the {\cb Cell Placement Control} pop-up if it is not
already visible.  When the {\cb Place} button in the {\cb Cell
Placement Control} pop-up is active, the current ``master'' will be
``attached'' to the mouse pointer, and instances will be placed at
locations where the user clicks with mouse button 1 in the drawing. 
The placement mode can be exited by pressing the {\kb Esc} key, or by
unsetting the {\cb Place} button in the {\cb Cell Placement Control
pop-up}.

Once devices and subcircuits have been placed in the drawing, they can
be moved and copied as for physical cells.  Not all of the
transformations of physical mode are available, however, from the {\cb
xform} command in the side menu.  Specifically, rotations are limited
to multiples of 90 degrees, and there is no magnification capability.

% 062113
\subsection{Semiconductor Devices}
\index{devices}
\index{mos devices}

The device menu contains symbols for the semiconductor devices
supported by {\WRspice}.  These include diodes, bipolar and junction
field-effect transistors, MESFETs, and MOSFETs.

\begin{tabular}{ll}\\
\cb Device &     \cb Description\\
\et dio & junction diode\\
\et npn & npn bipolar transistor\\
\et pnp & pnp bipolar transistor\\
\et njf & n-channel junction field-effect transistor\\
\et pjf & p-channel junction field-effect transistor\\
\et nmes & n-MESFET\\
\et pmes & p-MESFET\\
\et nmos & n-MOSFET (3-terminal)\\
\et pmos & p-MOSFET (3-terminal)\\
\et nmos1 & n-MOSFET (4-terminal)\\
\et pmos1 & p-MOSFET (4-terminal)\\
\end{tabular}

Unlike simple devices such as resistors and capacitors, which are
fully specified by a value, these devices almost always require a
model.  The model is specified with a {\et model} property, which is
applied to the device in the same way that a {\et value} property is
applied to a simple device.

In order for {\Xic} to include the model in the SPICE file, the model
must be available to {\Xic}.  Device models are provided to {\Xic}
through a file read by {\Xic} when the program starts.  When {\Xic}
starts, it traverses the library search path, looking for model files. 
A model file is 1) a file usually named ``{\vt model.lib}'', in which
case the first such file is read, or 2) any file found in a
subdirectory usually named ``{\vt models}'' of a directory in the
search path.  The names assumed (``{\vt model.lib}'' and ``{\vt
models}'') can be changed in the technology file.

The files that contain the models consist of the {\vt .model} lines
for SPICE.  These blocks are placed one after another, with no order
assumed.

\index{model library file}
Perhaps the simplest way to add a model to {\Xic} is through the {\vt
model.lib} file.  A skeletal {\vt model.lib} file is provided with
{\Xic}, in the startup directory.  Models added to this file will be
available to all users.  If a copy of the {\vt model.lib} file is
placed in the current directory, (which is always searched first) then
that file will be used instead.  The first {\vt model.lib} file found
in the library search path will be used.  This allows users to access
their own custom {\vt model.lib} file.

If large numbers of models are to be added, it may be more convenient
to add a ``{\vt models}'' subdirectory to one of the directories in
the library search path.  One may add a directory to the search path
for this purpose.  In the models subdirectory, add the files
containing the SPICE models.  The file names are unimportant, and all
files found in the subdirectory will be searched.

Each model block starts with
\begin{quote}
{\vt .model} {\it modname modtype} ....
\end{quote}

The {\it modname} is an arbitrary word which designates the model, and
this should be unique among all of the models {\Xic} will find along
the library search path.  The {\it modtype} is the SPICE name for the
model for a given device, as specified in the {\WRspice}
documentation.  The remaining text consists of parameter value
assignments as per the documentation.  The {\it modname} should be
used in a {\et model} property of the devices that are to use the
model.

\index{mos substrate bias}
There are two different MOS device types:  the {\et nmos1}/{\et pmos1}
devices contain stubs for all four nodes (gate, drain, source, and
bulk).  The {\et nmos}/{\et pmos} devices automatically connect the
bulk node to global nodes named NSUB and PSUB, respectively.  Most of
the time, it is more convenient to use the {\et nmos}/{\et pmos}
devices to avoid having to make explicit contact to the substrate
nodes in the circuit, however one {\it must} remember to bias the NSUB
and PSUB nodes.  To do this:

\begin{description}
\item{If there is one or more {\et nmos} devices in the circuit:}\\
\begin{enumerate}
\item{Add a voltage source to the schematic.}
\item{Place a ground terminal on the negative terminal of the
    voltage source.}
\item{Place a {\et tbar} terminal device on the positive terminal of
    the voltage source.}
\item{Select the `tbar' label of this terminal device.}
\item{Press the {\cb label} button (side menu), and change the name from
    ``tbar'' to ``NSUB''.}
\item{Add a {\et value} property to the voltage source to set the
    substrate voltage.  This procedure is described below.}
\end{enumerate}
\item{If there is one or more {\et pmos} devices in the circuit:}\\
Follow the same procedure above, however use ``PSUB'' as the name
    for the {\et tbar} device.
\end{description}

This will provide a dc bias voltage to the common connection of all of
the {\et nmos} and {\et pmos} bulk nodes in the circuit.  The value of
NSUB is usually equal to the most negative supply voltage in the
circuit, and the value of PSUB is usually equal to the most positive
voltage in the circuit.

% 062113
\subsection{Wiring Devices and Subcircuits}
\index{connecting devices}

Once the devices and subcircuits have been placed, wires can be added
to make connections between them.  This is not typically a two-step
process, as most users build a schematic by mixing placement and wiring
operations.

First, it should be stressed that connections do not always require
wires, and in particular it is often most convenient to make
connections between devices by abutment.  Devices and subcircuits have
specific local coordinates where a connection is possible.  In a
device, these are typically at the end of the wire stubs shown as part
of the device symbol.  In subcircuits, these are the terminal
locations defined by the designer of the subcell, and can be made
visible with the {\cb terms} button in the side menu.  When moving or
placing a device, or creating a wire, visual feedback is provided when
the mouse pointer is over a possible connection point.  Connections
can only occur at the connection points.  The {\cb Connection Dots}
button in the {\cb Attributes Menu} can be used to draw a dot at all
connection locations.

\index{gnd device}
\index{tbar device}
The devices in the device menu should mostly be familiar to users of
SPICE.  There are special terminal ``devices'' that can be used
instead of wires to provide interconnections.  These are the ``{\et
gnd}'', ``{\et tbar}'' and equivalent terminals.  In the first case,
the symbol is of a ground connection, and it provides exactly that. 
At least one point of every circuit must be grounded, or the SPICE
simulation may fail.  The {\et tbar} terminal is more general purpose. 
As it is, this terminal will tie all locations attached to such
terminals together.  This is a convenient way of distributing a power
net, for example.  If the name label of the {\et tbar} device is
changed, then all locations attached to terminals with this name will
form a {\it different} network.  The easiest way to change the name is
to click on the ``tbar'' label of an existing {\et tbar} device (thus
selecting the label), then press the {\cb label} button in the side
menu.  The user will be prompted for a new string.  Once the new
string has been entered, the label will be updated, and the terminal
can be copied to other locations to from the network.

\index{SCED layer}
\index{wire connections}
Remaining connections are made with the {\cb wires} button in the side
menu, which has an icon that looks like a sideways L.  Before
generating wires for connections, the user should make sure that the
current layer is the ``SCED'' layer.  Wires on this layer are
electrically active.  Wires created on other layers are for decoration
purposes only, unless the {\et WireActive} flag is set for the layer.

Wires are used to connect the devices together by clicking on the
vertex locations of the wires.  The vertices must be on the contact
points of devices and subcircuits, i.e., the ends of the connecting
wire stubs of the devices, and the terminal locations of subcircuits. 
These vertices are created automatically in horizontal or vertical
wire segments which cross over contact points.

One of the problems that some new users encounter is that contact is
not made due to improper placement of wires in relation to device
contact points.  To reiterate the previous discussion, only the ends
of the wire stubs of devices are ``active'', and these must physically
coincide with a wire vertex.  Although a vertex will generally be
created if necessary in an intersecting wire, new users should form
the habit of explicitly creating a vertex, by clicking on the contact
point while creating the wire,

\index{chlyr button}
In electrical mode, the first layer in the layer table is a layer
named ``SCED''.  This is an active wiring layer, and by default only
this layer can be used for electrically significant wires.  The layer
named ``SPTX'' is also active, in that labels on this layer are
included in the SPICE text generated for the cell.  Other layers are
used for visual purposes only (such as color-coding the displayed
property labels), or for temporary ``storage'' of parts of the circuit
not in use.  The {\cb Chg Layer} button in the {\cb Edit Menu} is used
to change the layer of objects.

\index{electrical layers}
The additional layers can be used for anything, but serve the
following purposes:

\begin{quote}
\begin{tabular}{ll}
\sc SCED & active wiring layer\\
\sc SPTX & active label layer\\
\sc NAME & device/subcircuit {\et name} property labels\\
\sc MODL & device {\et model} property labels\\
\sc VALU & device {\et value} property labels\\
\sc PARM & device/subcircuit {\et param} property labels\\
\sc NODE & terminal label\\
\sc ETC1 & general purpose\\
\sc ETC2 & general purpose\\
\end{tabular}
\end{quote}

\index{Connection Dots button}
\index{vertex editor}
The {\cb Connection Dots} button can be used to show dots at
connection points.  New users often appreciate the feedback provided
by the {\cb Connection Dots} button that a connection has been made. 
One has a choice of whether dots appear at every connection, or only
at those likely to be ambiguous.  When a wire is created, if it runs
over a device terminal or a vertex of another wire while horizontal or
vertical, a vertex is generated, which implies a connection.  Two
wires crossing do not connect, unless a vertex existed in one of the
wires at the crossing point.  Sometimes, is is desirable to remove a
connection, or to enforce a connection of two crossing wires.  This
can be accomplished with the vertex editor available with the {\cb
wires} button.  First, select the wire by clicking on it.  After
pressing the {\cb wires} button, each vertex of the wire will be shown
with a small box.  Clicking on a vertex box will select that vertex,
and allow the vertex to be dragged to a new location or deleted.  In
either case, the connection to an underlying vertex or device terminal
will be broken.  To add a vertex, click on the selected wire at the
point where the vertex is to be added.  A new vertex box will appear. 
If there is an underlying device terminal or wire vertex, a connection
will have been established.  If two wires cross with neither wire
having a vertex at the crossing point, adding a vertex to one of the
wires will automatically add a corresponding vertex to the second wire
if the second wire is horizontal or vertical at the crossing point.

% 030715
\subsection{Adding Properties to Devices}
\index{device properties}

\index{properties!adding}
\index{prpty button}
\index{Property Editor}
Once the devices have been placed, device properties can be assigned. 
This is the method by which {\Xic} knows the values, models, and other
characteristics of the devices.  Device properties are initially added
with the {\cb Property Editor} brought up by the {\cb Properties}
button in the {\cb Edit Menu}.  The {\cb Property Editor} contains a
text window showing the properties of a selected device, if any.  The
features and capabilities of the {\cb Property Editor} are rather
complicated, and are described fully in the section of this manual
(\ref{prptybut}) describing the {\cb Properties} command in the {\cb
Edit Menu}.  This section will describe some of the basic operations.

At this point there are four properties of interest:  {\et devref},
{\et value}, {\et model}, and {\et param}.  The purpose of the {\et
devref} property is to hold the name of a device whose current is to
be referenced.  This is used by the current-controlled sources and
switch devices only.  The {\et value} and {\et model} are just
different names for the same underlying text field, thus a device
should not be assigned both a {\et value} and a {\et model} property. 
The {\et param} property will hold text for initial condition and
parameter assignment.

The string for a device, which will be generated in SPICE output, has
the generic form

\begin{quote}
  {\it device\_name node\_list} [{\it dev\_ref\/}] {\it model\_or\_value}
  [{\it parameters\/}]
\end{quote}

The current-controlled dependent sources and switch require a 
{\et devref} property.  This should not be used in other devices.
Every device should have a {\et model} or {\et value} assigned.  The
parameter ({\et param} property) is optional, but may be needed for
certain devices for particular types of simulation.  It is also used
to provide parameter values, such as the width or length of a MOSFET. 
This is where knowledge of the SPICE syntax is necessary, in order to
know what parameters are required for a given device.

For simple devices such as resistors, only a {\et value} property is
generally required.  To apply a {\et value} property, with the {\cb
Property Editor} visible, click on the device to receive the
property.  The editor will list any existing properties, and the
selected device will be highlighted.  From the {\cb Add} menu of the
{\cb Property Editor}, press the {\cb Value} button, and enter the
value on the prompt line, followed by {\kb Enter}.  A label showing
the new value will appear next to the selected device.

The ``value'' can be just about any string, so it is important that
this input have relevance to SPICE.  The format of the numerical
entries is as recognized by SPICE, in MKS units.  One common error is
to leave off the units, e.g., entering ``{\vt 50}'' for the value of a
capacitor when the correct entry should be ``{\vt 50fF}''.  Of course,
``{\vt 50e-15}'' would suffice as well in this case.

The {\cb Global} button on the {\cb Property Editor} can be used to
set the properties of several devices at once.  The {\cb Edit} button
can be used to edit an existing property.  Once a property has been
assigned to a device, copies of the device will contain the same
property, thus it may be preferable to assign properties in part early
in the placement step, and generate copies of similar devices rather
than placing new instances.

Once a property has been assigned, it can be changed with the label
editor, thus the {\cb Property Editor} needs to be invoked only for
the initial assignment.  To change the value of any editable property,
select the label displaying that value (you can select properties in
multiple devices).  Then, press the {\cb label} button in the side
menu.  This will prompt for a new value, and when given, all of the
selected labels will be updated with the new value, and the underlying
properties will have been changed.

% 010914
\subsection{Creating Subcircuits}
\index{subcircuit creation}

\index{subcircuit terminals}
In order for a cell to be a valid subcircuit, i.e., electrically
active when placed into another cell, one or more contact terminal
locations must be defined.  This is accomplished with the {\cb subct}
button in the side menu.  When this button is pressed, the user may
click on contact points within the circuit to define contact
locations.  Only valid contact points can be selected, i.e., the point
must fall on a wire vertex, or a contact point of a device or
subcircuit.  When a valid point is clicked on, a boxed digit will
appear at the location, and a pop-up window will appear allowing the
user to set the name and other properties of the terminal.  If no name
is given, {\Xic} will use a default name.

Clicking on an existing terminal will start a move operation on the
terminal, attaching its outline to the mouse pointer.  Pressing the
{\kb Delete} key at this point will delete the terminal.  Clicking on
a terminal with the {\kb Shift} key held, or double-clicking, will
bring up the terminal editing window for the terminal, allowing
modification of its properties.

The {\cb terms} button in the side menu, when on, will display the
terminal locations, as well as the terminal locations of subcells in
the drawing.

Subcells will most often have terminals defined, which are the
connections points to the cell.  It is possible, though, that a
subcell will have no terminals, if connection is made via global nets. 
Imagine a subcell containing only a capacitor, which is connected to
global nets {\vt vdd!} and ground.  Adding an instance of the cell is
equivalent to adding a decoupling capacitor.

It is possible, after an instance of a cell has been placed, to use
the {\cb Push} command to push into the new cell, and define
additional subcircuit contacts, and pop back to the parent cell.

\index{symbolic representation}
\index{symbl button}
In some cases, it is preferable that the subcell be displayed as a
symbol, rather than a schematic, when expanded.  For example, if the
subcell represents an AND gate, and there are many instances of the
subcell, the drawing of the parent cell will appear much neater if the
AND gate cell is represented by an AND symbol rather than  its full
schematic.  One can define such a representation with the {\cb symbl}
button in the side menu.

On pressing the {\cb symbl} button for a cell without a previous
symbolic representation defined, the schematic will disappear, and the
screen will be blank.  One is free to use the objects from the {\cb
shapes} menu, wires, and labels, on any of the layers, to construct a
symbol which will be displayed for that cell.  When the new drawing is
complete, the {\cb subct} button should be pressed again.  This will
make the contact point indicators visible, however they will be in
arbitrary locations.  The user should move the terminals to where they
belong in the symbolic representation, by dragging them with the left
mouse button.  Unlike in the normal schematic representation, the
terminals can be placed anywhere.  It is possible to copy terminals by
holding {\kb Shift} during the ``move'', so that the symbol may have
multiple connection points for the same terminal.

New terminals can be added, or terminals deleted, only by returning to
schematic mode, and similarly the schematic can be edited only by
returning to schematic mode.  The display status of the cell is set by
the status of the {\cb symbl} button when it was saved to disk, or
last edited if it is still in memory.

% 062113
\subsection{Node and Device Naming}
\index{node naming}

\index{name property}
{\Xic} will assign names and node numbers to the device, subcircuits
and nodes in the circuit, by default.  These will be unique numbers
for each type of device and for each node.  One problem, however, is
that these numbers will change when the circuit topology is changed. 
Often, the SPICE output may be used by another application, that may
need to access circuit node voltages, for example, in a predictable
way.  Thus, {\Xic} has provision for assigning an immutable name to
wire nets, and to devices and subcircuits.

By default, device names are assigned by {\Xic} as the device key
letter followed by an integer that {\Xic} generates.  This can be
overridden by assigning a {\et name} property to the device.  The
procedure is identical to assigning the properties that we have
discussed previously.  The {\cb Name} button in the {\cb Add} menu of
the {\cb Property Editor} is used.  Although the string that is
entered as the name property can be anything, there are some very
important constraints for correct SPICE output.

\begin{enumerate}
\item{The first letter of the name must be the same (case insensitive)
  as the default name.  This is the `key' that identifies the type of
  device in SPICE.}

\item{The name should be a single word containing alpha-numeric
  characters only.}

\item{The name should be unique in the circuit.}
\end{enumerate}

Although {\Xic} provides flexibility in assigning this property, SPICE
simulations will fail unless these constraints are observed.  Once the
name property is assigned to a device, that name, rather than the
default, will be used to reference the device.  The name will appear
in a label next to the device on-screen.  As we have previously seen,
the name can be modified subsequently with the label editor.

The procedure for assignment of names to subcircuits is identical. 
The `key' letter for subcircuits is `{\vt X}'.

\index{node mapping editor}
The node mapping editor, which appears when the {\cb nodmp} button in
the side menu is pressed, is used to assign names to nodes.  A
``node'' is SPICE terminology for a collection of one or more device
and subcircuit terminals that are connected together.  Each node is
given a unique number by {\Xic}, which is used as the node ``name'' in
SPICE output.  The node mapping editor allows the node to have an
assigned name, which will be used instead.

\index{nodmp button}
Full information on the node mapping editor can be found in the
section describing the {\cb nodmp} command (\ref{nodmp}).  Here, we
will briefly outline its use.  The left panel of the node mapping
editor contains a list of the circuit nodes, with the left column
containing the internal number, and the right column containing the
assigned name, if any.  Selecting an entry in this list will cause the
device terminals for that node to be listed in the right panel, and
these will be highlighted in the schematic.  Pressing the {\cb Rename}
button will prompt the user for a name for that node.  This can be any
word consisting of alpha-numeric characters.  This word will be used
in SPICE output to designate the node, rather than the number.

% 062113
\subsection{Connectivity Overview}
\label{connect}
\index{connections}

Thus far we have described the basic methodology for producing a
schematic.  Armed with this information, users can quickly produce
schematics of simple circuits.  However, a lot has been skipped over,
including the use of multi-conductor nets and vectorized instances. 
This section will review the basic connectivity concepts, and
introduce these new topics.

Devices and subcircuits generally have ``pins'' which are hot-spots in
the drawing where connection can occur.  These hot spots may or may
not be marked in the device or subcircuit symbol or schematic.  In any
case, pressing the {\cb terms} button in the electrical side menu will
cause the display of terminal symbols at these locations.

The current cell will have its own terminal locations, if any have
been defined with the {\cb subct} command in the side menu.  These
will be the connections points to instances of the current cell.

Establishing connectivity in the schematic involves logically grouping
the device, subcircuit, and cell terminals that should be connected. 
Each such group is termed a ``net''.  There are a number of ways to
define this grouping.

\begin{enumerate}
\item{Most commonly, a wire is placed by the user using the {\cb wire}
command in the side menu.  To establish connectivity, a vertex of the
wire must be at a connector hot-spot.  If the {\cb dots} display is
enabled, a dot may be shown at the connection points.}

\item{Connection points whose hot-spots are placed at the same
location will be connected.}
\end{enumerate}

These two methods illustrate connection by location.  It is also
possible to use connection by name.  For this, one must use named
nets.  Looking ahead just a bit, it is possible for a net to be scalar
(single conductor) or multi-conductor.  The type of net is described
by the name, which is interpreted as a ``net expression'', which is a
syntax which allows detailed definition of the conductors in the net.

There are several ways by which a net can acquire a name.
\begin{enumerate}
\item{Nets connected to named cell terminals will have the same name
as the cell terminal, but only if the terminal has an applied name. 
Names can be given to cell terminals with the {\cb subct} command in
the side menu.}

\item{A scalar (single conductor) net can be assigned a name with the
{\cb Node (Net) Name Mapping} panel brought up with the {\cb nodmp}
button in the side menu.  This name has priority over the ``candidate
names'' applied with wire labels or terminal devices.}

\item{A candidate net name will be supplied by associated labels of
wires in the net.  A label is given to a wire through the
    following procedure.

\begin{itemize}
\item{In electrical mode, select a single wire, which shall receive a
name.}

\item{Press the {\cb label} button in the side menu.}

\item{Type the label text in the prompt line, and press the {\kb
Enter} key.}

\item{The label is ghost-drawn and attached to the mouse pointer. 
Resize or rotate the label if desired, and click in the drawing near
the selected wire to place the label.  This completes the operation.}
\end{itemize} }

\item{A candidate net name can also be supplied by placing a terminal
device from the device library in contact with the net.  The device
library provides several terminal styles.  Each has a label that can
be edited to apply a net name.  Once placed, the label can be
selected, the {\cb label} button pressed, and new label text entered.}
\end{enumerate}

A scalar net may have multiple ``candidate names'', and each can be
used to establish connections by name.  However, the single name
chosen to represent the net in netlist output will be the name that
comes first in alphabetical order.

Nets that otherwise appear disjoint but have a common name are
actually connected.  This illustrates connect by name.  In fact, it is
possible to draw perfectly good schematics without using wires, by
using terminal devices only.  The schematics produced by {\Xic} from
SPICE files or physical extraction use this approach.

{\Xic} supports multi-conductor wire nets in schematics, using a
syntax and methodology that should be familiar to users of Cadence
Virtuoso.  The net name uses a syntax which describes the net. 
Unnamed nets will assume the characteristics of connected terminals. 

There are three types of net.
\begin{description}
\item{Scalar nets}\\
Single-conductor ``scalar'' nets provide the basic connectivity
description in a schematic, and are the only electrical nets that may
have a counterpart in the physical layout.

A scalar net name consists of a simple name, or an indexed vector
name, in a format to be described.

\index{vector nets}
\item{Vector nets}\\
A vector net contains multiple conductors, accessible as indices in a
range, with a common base name.  A name specifying a vector net may
have the form
\begin{quote}
{\it basename\/}{\vt [}{\it start\/}{\vt :}{\it end\/}{\vt ]}
\end{quote}

The {\it start} and {\it end} are non-negative integers.  The two
colon-separated numbers provides a range of subscripts which identify
the individual conductors, or ``bits'', of the net.

For example, the vector net ``{\vt foo[3:0]}'' consists of four
conductors, in order ``{\vt foo[3]}'', ``{\vt foo[2]}'', ``{\vt
foo[1]}'', and ``{\vt foo[0]}''.  Note that the range values can be
ascending or descending.

In {\Xic}, the square brackets can be replaced by {\vt <...>} or {\vt
\{...\}}.  That is, for subscripting in {\Xic}, square brackets, curly
brackets, and angle brackets are equivalent.  This documentation will
use square brackets.

Vector nets differ fundamentally from scalar nets in {\Xic} in that
they simply reference scalar nets.  The scalar nets actually provide
the electrical connections, and the correspondence between layout and
schematic.  The vector and multi conductor nets in general simply
provide an organizational framework for the scalar nets.

In particular, this requires that each ``bit'' of a vector net have an
existing scalar net of the same name.  In the example above, for the
vector net {\vt foo[3:0]} to be valid, the individual scalar nets {\vt
foo[3]} etc.  must exist.

\index{bundle nets}
\item{Bundle nets}\\
A bundle net is a net of nets.  Its name is a net expression
consisting of comma-separated names of scalar and vector nets.  Some
examples would be
\begin{quote}
{\vt a,data[0:7],addr[2]}\\
{\vt b0,b1,b2}
\end{quote}

These are simple cases of a net expression which describes the
conductor sequence of a general net.  Net expressions and vector
expressions may be familiar from Cadence Virtuoso, and in fact the
same operations and syntax are supported.
\end{description}

% 062313
\subsection{Net and Vector Expressions}
\label{netex}
\index{net expression}

The name of a net is parsed as an expression using a set of rules to
be described.  The result of this interpretation is that each
conductor (``bit'') of the net has a well-defined name, which is
associated by name with all other nets in the cell with bits of a
matching name.

We say ``matching'' rather than ``the same'' as {\Xic} will ignore the
different subscripting characters.  In {\Xic}, square, curly, and
angle brackets are accepted for subscripting, thus the forms {\vt
foo<2>}, {\vt foo[2]}, and {\vt foo\{2\}} are equivalent ane can be
freely intermixed in the design.

A net expression consists of one or more comma-separated
{\it terms\/}.
\begin{quote}
{\it net expression} = {\it term\/}[{\vt ,}{\it term\/}]...
\end{quote}

A {\it term} has the general form
\begin{quote}
{\it subterm} = {\it name\/}[{\it vector expression}]\\
{\it multiplier} = {\vt [*}{\it N\/}{\vt ]}, or\\
{\it multiplier} = {\it N\/}{\vt *}\\
{\it term} = [{\it multiplier\/}]{\it subterm\/}, or\\
{\it term} = [{\it multiplier\/}]{\vt (}{\it term\/}[{\vt ,}{\it term\/}]...{\vt )}
\end{quote}

The basic element of a {\it term} is a {\it subterm\/}, which consists
of a name optionally followed by a {\it vector expression\/}.  The
{\it name} is an alphanumeric text name.  The {\it vector expression}
represents subscripting to be described.

An optional {\it multiplier} can prefix the {\it term\/}.  This is an
integer {\it N\/}, and a literal asterisk, in one of the forms shown. 
Here, the literal square brackets can be replaced by curly brackets or
angle brackets equivalently.  Both forms of the multiplier prefix are
equivalent.  The effect of the multiplier is to repeat what follows
{\it N} times.

The second form of the {\it term} allows for a list of {\it terms\/},
separated by commas and enclosed in parentheses.  The commas and
parentheses are literal.  This allows the multiplier to cause
repetition of the group of terms.

The multiplier provides a shorthand way to express repetitions, but is
not required.  Below are some examples and equivalences.
\begin{quote}
{\vt 3*A} = {\vt A,A,A}\\
{\vt 2*(A,B)} = {\vt A,B,A,B}\\
{\vt 2*(A,2*B)} = {\vt A,B,B,A,B,B}
\end{quote}

In each case, the shorthand on the left is equivalent to the ordering
on the right.  The {\vt A} and {\vt B} are scalar conductor names. 
The third line above, for example, describes a six-conductor net with
the net bits connected to either net {\vt A} or {\vt B} in the order
shown.

\index{vector expression}
A {\it vector expression} represents a sequence on integers, each
representing a conductor index.
\begin{quote}
{\it bit} = {\it N}\\
{\it range} = {\it N\/}{\vt :}{\it M\/}[{\vt :}{\it S\/}]\\
{\it postmult} = {\vt *}{\it N}\\
{\it vector expression} = {\vt [}{\it bit\/}$|${\it range\/}[{\it postmult\/}][{\vt ,}...]{\vt ]}\\
{\it vector expression} = {\vt [}{\vt (}{\it vector expression\/}[{\vt ,}...]{\vt )}[{\it postmult\/}][{\vt ,}...]{\vt ]}
\end{quote}

Again, where literal square brackets are shown, curly brackets and
angle brackets are equivalent in {\Xic}.  The elemental decomposition
of a vector expression is a comma-separated list of non-negative
integers.  A {\it bit} constitutes one such integer.  A {\it range} is
specified by two or three colon-separated non-negative integers.  In
the simplest and most common form, the range consists of two integers
and represents the two integers and all intermediate integers, in
order.  If a third integer is given, this represents the increment. 
The number sequence consists of the start value, and multiples of the
increment, terminating at the final value that would not fall outside
of the range.  Note that the increment is always a positive value,
whether the range values are decreasing or increasing.  Below are some
examples.

\begin{quote}
{\vt [3:0]} = {\vt [3,2,1,0]}\\
{\vt [3:0:2]} = {\vt [3,1]}\\
{\vt [1:4]} = {\vt [1,2,3,4]}\\
{\vt [1:4:4]} = {\vt [1]}
\end{quote}

Either can be followed by a {\it postmult} multiplier, which causes
each element of the sequence to repeat.
\begin{quote}
{\vt [0*2]} = {\vt [0,0]}\\
{\vt [3:0*2]} = {\vt [3,3,2,2,1,1,0,0]}\\
{\vt [1:4:4*2]} = {\vt [1,1]}
\end{quote}

The final form illustrates use of literal parentheses and commas to
associate a list of vector expressions to a post-multiplier.  The
entire list will be repeated.  The parentheses can be nested to
arbitrary depth.

\begin{quote}
{\vt [(1,3:5)*3]} = {\vt [1,3,4,5,1,3,4,5,1,3,4,5]}\\
{\vt [(1,(2,3*2)*2,4:6)*2]} = {\vt [1,2,3,3,2,3,3,4,5,6,1,2,3,3,2,3,3,4,5,6]}
\end{quote}

% 062113
\subsection{Vectored Instances}
\label{vecinst}
\index{vectored instance}

Device and subcell instances can be scalar or vectorized.  By giving
an instance a {\et range} property with the {\cb Property Editor} from
the {\cb Edit Menu}, the instance will become vectored.  The single
schematic representation in the drawing of a vectored instance
actually corresponds to multiple ``bit'' instances.  This can greatly
clarify schematics with repeated circuit blocks.

The connections to a vectored instance are all multi-conductor nets
(assuming that the array range contains more than one element).

% 062113
\subsection{Connection Rules}
\index{connection rules}

The following rules are applied when connecting by location.

\begin{enumerate}
\item{Any named scalar net can connect to any other named (or unnamed)
scalar net.  A scalar net can have any number of associated names,
each of which is a valid target for connect by name.}

\item{If a scalar net connects to a non-scalar net, the scalar bit
will connect to each bit of the non-scalar net.}

\item{A net connecting to a vectored instance terminal must have a
width equal to one of the following:

\begin{itemize}
\item{The total connection width, given by the pin width multiplied by
the vector instance width.  For example, suppose that the instance is
arrayed {\vt [0:3]} and the pin is {\vt A[0:1]}.  Suppose that the
connecting net is {\vt net[7:0]}.  Then, all is well as the widths
match, and connections will be as shown.

\begin{quote}
{\vt net[7]} = {\vt X[0]A[0]}\\
{\vt net[6]} = {\vt X[0]A[1]}\\
{\vt net[5]} = {\vt X[1]A[0]}\\
{\vt net[4]} = {\vt X[1]A[1]}\\
{\vt net[3]} = {\vt X[2]A[0]}\\
{\vt net[2]} = {\vt X[2]A[1]}\\
{\vt net[1]} = {\vt X[3]A[0]}\\
{\vt net[0]} = {\vt X[3]A[1]}
\end{quote}

If the widths do not match, a warning will be issued.  {\Xic} will
connect what it can, in an order like that above, but some bits will
remain unconnected.}

\item{The pin width.  In this case, a virtual multiplier prefix is
applied to the net.  For the example above, but with {\vt net[1:0]}
that matches the width of {\vt A[0:1]}, the connections would be

\begin{quote}
{\vt net[1]} = {\vt X[0]A[0]}, {\vt X[1]A[0]}, {\vt X[2]A[0]}, {\vt X[3]A[0]}\\
{\vt net[0]} = {\vt X[0]A[1]}, {\vt X[1]A[1]}, {\vt X[2]A[1]}, {\vt X[3]A[1]}
\end{quote} }

\item{The width is one (scalar net).  In this case, all of the
instance pin bits would connect to the same scalar net.}
\end{itemize} }

\item{Named multi-contact nets cannot connect to incompatible nets. 
Two named nets are ``compatible'' if one is a ``tap'' of the other. 
This will be described in the next section.  Violations generate an
error message and no connection is made.}
\end{enumerate}

% 062113
\subsection{Tap Wires}
\index{tap wires}

The concept of tap wires may be familiar from Cadence Virtuoso. 
Tap wires are fully supported in {\Xic}.

A wire is considered to be a ``tap'' of another wire if every bit in
the first wire is included in the second.  Note that they may have
very different bit order.

If a wire is a tap for another, then the two wires are allowed to
connect.  Note, however, that the visual connection may serve no real
purpose, as the bits are already connected by name.  However, the
visible indication of connectivity may make the schematic more
readable.  The tap wire will allow connection to a subset of the
conductors in the wire being tapped.

An interesting special case is when the wire being tapped is a pure
vector.  In this case (only), the tap wire label need not include a
name, but only a vector expression.  Also in this case, a connection
is required.  Then, the tap wire will obtain the name from the wire
being tapped.

For example, suppose that we have a net {\vt data[0:3]}, and we want
to connect {\vt data[0]} to a scalar instance pin {\vt A}.  If we
connect the {\vt A} pin directly to the {\vt data[0:3]} wire, all four
bits of the wire would be connected to {\vt A}, which is not what we
want.  Instead, create a new wire, connected to the original wire and
to {\vt A}.  Give the new wire a label ``{\vt [0]}''.  This becomes a
tap wire, connecting {\vt data[0]} to {\vt A}.

% 062113
\subsection{Generating Output and Running Simulations}
\index{SPICE output}

\index{deck button}
\index{spice analysis}
\index{run button}
Once the device properties have been entered, the user can export the
circuit for further analysis.  The {\cb deck} command in the side menu
can be used to produce a SPICE file of the current hierarchy.  If the
{\WRspice} program is accessible, the {\cb run} command in the side
menu can be used to initiate analysis.  The user will be prompted for
a SPICE analysis string, and the simulation will run.  A small window
will appear that will inform the user when the analysis is complete.

\index{plot button}
After {\WRspice} analysis, circuit variables may be plotted.  The {\cb
plot} command in the side menu allows the user to click on circuit
nodes to plot.  After each click, the corresponding node is added to
the string shown on the prompt line.  This string can be edited
manually in the usual way, if necessary.  Pressing {\kb Enter} will
terminate the string, and the plot will be displayed on-screen.  The
{\cb iplot} button works similarly to the {\cb plot} button, though
the plot will be generated dynamically during simulation on subsequent
runs.  Plotting is available only through the {\WRspice} program.

\index{properties!changing}
\index{label button}
Once properties have been entered, they are easy to alter without the
use of the {\cb Properties} command.  The {\cb label} button in the
side menu is used primarily to add annotation to the drawing. 
However, if a label is selected before pressing the {\cb label}
button, the existing label can be edited, rather than a new label
created.  If the selected label is one of those created for a
property, then that property can be altered merely by editing the
label.  Thus, to change a property of a device, click on the label to
select it.  Then, after pressing the {\cb label} button, enter the new
text.  The circuit can then be re-simulated with the altered
parameters.

\index{hypertext}
One feature of {\Xic} is the use of hypertext.  This is most evident
when using the {\cb plot} command.  When the user clicks on a circuit
node, the name of that node is entered, in color, on the prompt line.
Note that when using the arrow keys to move the prompt text cursor
across a node name, the cursor widens to underline the name, and the
name otherwise behaves as a single character.  The name shown is a
link to the internal database, and has the property that if the node
number assigned to that contact point changes (it may, if the circuit is
modified, as it is by default randomly assigned) the string will
automatically be updated to the new node number.

When creating a label, clicking on a connection point in the drawing,
for example, will enter a hypertext link to the node into the label. 
The label will always display the correct node number or name for the
node.  This is the means by which node labels should be added to the
drawing.

\index{spicetext label}
The same feature can be used to set up specialized spice output. 
Suppose one wishes to use the {\cb save} command in SPICE.  A
``spicetext'' label can be created, where the nodes to be included in
the save are inserted in the label by clicking on the drawing.  When a
SPICE file is produced, the contents of the ``spicetext'' labels is
added to the deck.  The resulting save command will always save the
clicked-on nodes, whether of not the actual internally generated
number changes.

The ``spicetext'' label is simply a label where the first word is
``spicetext'' or ``spicetext{\it N}'' where {\it N} is an integer. 
These labels have the property that any text following the
``spicetext'' keyword is added to the SPICE output verbatim.  The
optional integer that follows ``spicetext'' determines the order of
appearance of the lines, where no integer is equivalent to 0.  This is
the mechanism for placing arbitrary text into the SPICE output.

This has been a brief introduction to the use of {\Xic} in electrical
mode.  There are numerous commands and features, and many of the
commands mentioned have not been fully described.  The easiest way to
learn {\Xic} is to use it.  After switching to electrical mode, press
the {\cb Help} button in the {\cb Help Menu}.  Pressing any button
will bring up a description of that command.  Press {\kb Esc} to exit
help mode.

If a cell has both a physical layout and electrical schematic, there
is provision for verifying consistency of the two representations by
performing layout vs. schematic (LVS) testing.  This is one of the
functions which can be found in the {\cb Extract Menu}, and the process
is described in Chapter \ref{chpext}.


% -----------------------------------------------------------------------------
% fileorg 020615
\section{Cell Organization and Libraries}
{\Xic} provides several methods by which collections of cells can be
organized.

\begin{itemize}
\item{{\Xic} makes use of a search path for file names given to {\Xic}
    which do not have a directory path prepended.  A search path is a
    list of directories where {\Xic} searches for a named file.  If
    the file name contains a full path, that path will be used to
    obtain the file.  If a file name has a relative path, {\Xic} will
    look for the file relative to each of the directories in the
    search path.  The search path can be set in the technology file,
    or by setting the {\et Path} variable with the {\et !set} command. 
    The current path can be examined by entering ``{\vt !set}'', which
    pops up a list of the currently defined variables, including {\et
    Path}.  The directories are searched in left-to-right order.
}

\item{{\Xic} accepts library files.  These are text-mode files which
    contain references to cells and other libraries, and may contain
    cell definitions.  If a library file is ``open'', cell names
    referenced or defined in the library will be resolved through the
    library, before resolving through the search path.  The name of a
    cell reference in a library is the name by which the cell will be
    added to {\Xic} memory, which can be different from the name by
    which the cell is stored on disk.  The fact that a library can
    reference other libraries allows a hierarchy to be established for
    accessing cells, independent of the search path.

    The {\cb Libraries List} button in the {\cb File Menu} brings up a
    panel which lists the currently open libraries, and provides
    command buttons for performing basic manipulations on libraries,
    including opening/closing, viewing content, and opening cells.
}

\item{Cells contained in archive files can be randomly accessed from
    the file, thus these files can be used for archival purposes.  The
    {\cb Contents} button in the panel brought up by the {\cb Files
    List} button in the {\cb Files Menu} will display the cells
    contained in these files.  The {\cb Contents} button will also
    list the contents of library files.  Individual cells (and their
    subcells) can be opened for editing or placement through this
    panel.  Also, when giving a name to the {\cb Open} command, or the
    {\cb place} command in the side menu, one can give two names:  the
    name of an archive file and a space-separated name of a cell in
    the archive.  That cell will be opened.  If the cell name is not
    given, the top-level cell in the archive is opened.  }
\end{itemize}

The strategy used to organize cells is highly dependent upon the
user's needs and preferences.  Below are some recommendations which
are probably suitable for most applications.

\begin{itemize}
\item{Keep the search path short.  This can usually consist of two
    directories:  the current directory (``.'') listed first, and a
    root directory for the user's design files.  The search path is
    most conveniently defined in the technology file, with the {\et
    Path} keyword.  The search path has the disadvantage that all
    components are visible at all times.  If a cell name appears more
    than once in the search path, only the first instance will be
    found, unless the full path is given.  Libraries, on the other
    hand, can be opened and closed easily, changing the accessibility
    of the contents.
}
\item{Use hierarchies of libraries rooted in the search path to
    organize cells.  One can open only the libraries in use,
    preventing loading of cells unexpectedly.
}
\item{Place collections of cells to be referenced through libraries
    in separate directories not in the search path.  Alternatively,
    the {\Xic} cell definitions can be incorporated directly into the
    library file.  The cells can otherwise be kept as individual cells
    of any compatible format, or combined into a single archive file.
}
\end{itemize}

Library files have a simple format which allows the user to easily
create and customize them with a text editor.  There is a {\et !mklib}
command in {\Xic} which can create a new library or append to an
existing library references to cells in the current editing hierarchy
or cells in a given archive file.

If one clicks on a reference in a library content listing which points
to another library, without a resolving ``{\it cellname\/}'', a second
content window appears providing a listing of the second library's
references.  Thus, when constructing library files, one should use an
easily recognizable name for browsable references to other libraries. 
This is natural, if the file name is used as the reference name, and
the filename has a ``{\vt .lib}'' extension as is recommended.


% -----------------------------------------------------------------------------
% xic:batch 022916
\section{Batch Mode}
\index{batch mode}
\label{batchmode}

{\Xic} has a batch mode of operation, where {\Xic} will start without
graphics, run commands, and exit.  Batch mode is signaled by giving
the {\vt -B} option in the command line, in one of the following
forms:
\begin{quote}
{\vt -B}{\it scriptfile}[,{\it param1\/}={\it value1\/}][,{\it param2\/}={\it value2\/}]...\\
{\vt -B-}{\it command}[{\vt @}{\it arguments}]
\end{quote}
 
In the first form, the path to a file containing {\Xic} script
statements immediately follows ``{\vt -B}'' with no space.  The
statements in the script file will be executed after the first input
file is loaded.  If no input file is given on the command line, the
script will be executed after the default ``noname'' cell is loaded.
 
It is possible to pass parameters to the batch-mode scripts from the
command line.  The comma is used as a delimiter.  Commas in the line
that remain in single or double quotes {\it after} the shell has
treated the line are not taken as separators.  The entire construct
should not have any embedded white space, except when single or double
quoted as part of the {\it values}.
 
The {\it param1}, {\it param2}, etc.  are the names of variables that
will be defined in the execution context of the script.  These
variables will be set to {\it value1}, {\it value2}, etc.  The
values are numbers, strings, or executable text.  Values that contain
white space must be quoted, but note that the shell will strip the
quote marks, so that a string constant should be single and double
quoted as shown below.
 
Example
\begin{quote}
{\vt xic -Bmyscript,p1=1.234,p2='"a string"',p3="p1 + 1"}
\end{quote}
 
This translates into the virtual addition of three lines to the
beginning of the script:
\begin{quote}\vt
p1 = 1.234\\
p2 = "a string"\\
p3 = p1 + 1
\end{quote}

In the second form, the ``{\vt -B}'' is immediately followed by
another `{\vt -}' and one of the command keywords listed below.  After the
first cell is loaded (or ``noname'' if no input file was named in the
command line) the command will be executed.  The recognized commands
are listed below.

The command name can be immediately followed by an argument string that
begins with the `{\vt @}' character.  The arguments are specific to the
command.  Multiple arguments can be separated by `{\vt @}' characters,
or by white space if quoted.

The {\vt .xicstart} file is read and executed (if it exists) before
the first cell is loaded, and all other initialization is performed in
the normal sequence.  The commands below are simple shortcuts to
common operations.  If unavailable options are required, then these
can either be set in a {\vt .xicinit} or {\vt .xicstart} file, or the
first form of the {\vt -B} option should be used.

\index{batch mode commands}
\begin{description}
\item{\vt tocgx}, {\vt tocif}, {\vt togds}, {\vt tooas}, {\vt toxic}\\
These write the hierarchy under the current cell to CGX, CIF, GDSII,
OASIS, and native cell files, respectively.  They perform file
conversion by reading a file into {\Xic}, then writing it out in the
specified format.  The {\FileTool} utility and {\vt -F} command line
option provide a far more powerful format translation capability.

The default name for the file written is the name of the current cell,
suffixed with ``{\vt .cgx}'', ``{\vt .cif}'', ``{\vt .gds}'', and
``{\vt .oas}'' for the four archive file formats.  Native cell files
always have the same name as the cell contained.

These commands can take the following options.  The options are
separated from the command name and from one another by `{\vt @}'
characters, and consist of a single character identifier, an optional
`=' character, and a value.

\begin{description}
\item{\vt o=}{\it outfile}\\
The {\it outfile} is the name of the file to be generated.  If not
provided, the file name will be the name of the top-level cell
suffixed with an extension appropriate for the format.  In the case of
{\vt toxic}, the {\it outfile} is a path to a directory where the
cell files will be created.

\item{\vt s=}{\it scale}\\
The {\it scale} is a floating point value from 0.001 to 1000.0 which
applied when the file is written.

\item{\vt l=+|-}{\it lname\/}[{\vt ,}{\it lname} ...]\\
This option specifies a list of layer names.  The first character in
the list is a {\vt +} or {\vt -} to indicate that only the listed
layers will be output, or that all layers except the listed layers
will be output, respectively.  Immediately following is a layer name,
optionally followed by additional layer names separated by commas.

\item{\vt e}[{\it N\/}]\\
The letter `{\vt e}' can be immediately followed by an integer
0--3.  This sets the empty cell filtering level, as described for the
{\cb Format Conversion} panel in \ref{ecfilt}.  The values are

\begin{tabular}{ll}
{\vt e} or {\vt e1} & Use both pre- and post-filtering.\\
{\vt e2} & Use pre-filtering only.\\
{\vt e3} & Use post-filtering only.\\
{\vt e0} & No empty cell filtering (no operation).\\
\end{tabular}

\item{\vt f}\\
This flag option indicates that the output will contain a flat
representation of the cell hierarchy.  If the {\vt w} option is given,
only objects that overlap the window area will be present in output. 
This option will not work with {\vt toxic}.

\item{\vt w=}{\it l\/},{\it b\/},{\it r\/},{\it t}\\
This specifies a rectangular area, in microns, for use when
flattening.

\item{\vt c}\\
This flag indicates that when flattening with a window (both {\vt f}
and {\vt w} options also given) objects will be clipped to the window
boundary in output.
\end{description}

Example:\\

{\vt xic -B-togds@o=file1.gds@w=100,200,200,300@fc@l=+0600 myfile.gds}\\

This will create {\vt file1.gds}, containing objects on layer 0600
within the window area, flattened and clipped.  Note that the {\vt @}
separation character is actually optional after flags, and other
options which are not lists or strings.

\item{\vt drc}\\
Design rule checking is performed, and results are written to a log
file.

There are optional arguments that can be provided, separated from the
command name and from each other with `{\vt @}' characters.

\begin{description}
\item{\vt w=}{\it l\/},{\it b\/},{\it r\/},{\it t}\\
This provides an area, given in microns, of the top-level cell where
checking will be performed.  The value consists of four
comma-separated floating-point numbers.  If not given, the entire cell
will be checked.

\item{\vt m=}{\it maxerrs}\\
This provides the maximum batch-mode error count, checking will
terminate when this count is reached.  The {\it maxerrs} is an
integer 0--100000, with 0 indicating no limit.  This will override
the maximum error count set in the technology file, if any.

\item{\vt r=}{\it level}\\
This sets the error recording level to use when checking.  The {\it
level} is an integer 0--2.  These correspond to recording one error
per object, one error of each type per object, or all errors.  This
will override the recording level set in the technology file.

\item{\vt d}\\
This a flag, not followed by an `=' sign or value.  If given, the file
which was the source of the current cell will be deleted from the disk
when DRC completes.  This facilitates cleaning up temporary files,
but obviously should be used with care.
\end{description}
\end{description}

In batch mode, the log files for reading and writing of files are
written to the current directory.  


% -----------------------------------------------------------------------------
% xic:server 012411
\section{Server Mode}
\index{server mode}
\label{servermode}

{\Xic} has the capability of operating as a daemon process, servicing
requests for processing of design data.  This allows {\Xic} to be used
as a back-end for automation systems designed by the user or third
parties.

To start {\Xic} in server mode, the {\vt -S} option is used, as
\begin{quote}
{\vt xic -S}[{\it port\/}]
\end{quote}

This causes {\Xic} to start without graphics, go to the background,
and listen to a system port for requests.  The port number used can be
provided on the command line immediately following the ``{\vt -S}''.  If
not given on the command line, the ``{\vt xic/tcp}'' service is
queried from the local host.  This will come up empty unless the
``{\vt xic/tcp}'' service has been added to the host database, usually
by adding a line like the following to the {\vt /etc/services} file:
\begin{quote}
\begin{verbatim}
xic           6115/tcp   #Whiteley Research Inc.
\end{verbatim}
\end{quote}
where the port number 6115 is replaced by the desired port number.  If
there is no port assigned for ``{\vt xic/tcp}'', port 6115 is used,
as this is the IANA registered port number for this service.

If the {\et XTNETDEBUG} environment variable is defined when {\Xic} is
started in server mode, a debugging mode is active.  {\Xic} will
remain in the foreground, but will service requests while printing
status messages to the standard output.  This may be useful for
debugging.  If the {\vt dumpmsg} command is given, {\Xic} will print
the text of messages received on the terminal screen, enclosed in `|'
symbols to delineate the text.  The command {\vt nodumpmsg} can be
given to turn off the message printing.  This can be a useful feature
for debugging a client-side program which is communicating with
{\Xic}.

The user's application should open a socket to this port for
communications.  Up to five channels can be open simultaneously.

All transmission to the server is in ASCII string format, however
replies are in a binary format, and are likely to be invisible or
gibberish in a text-mode connection such as {\vt telnet}.  However,
the {\vt telnet} program can be used to connect to the {\Xic} daemon,
and can be used to give simple commands, such as the {\vt kill}
command.  After starting the daemon, one types
\begin{quote}
{\vt telnet} {\it hostname port}
\end{quote}
where {\it hostname} is the name of the machine running the daemon
(one can use ``{\vt localhost}'' if running on the local machine). 
The {\it port} is the port number in use by the daemon.

An example file {\vt xclient.cc} is available which provides a
demonstration of how to interact with the {\Xic} daemon through a
C/C++ program.  This file can be found in the examples directory of
the {\Xic} installation.

Communication can also be established through use of the example {\vt
xclient.scr} script, which illustrates use of script functions to
implement a client within {\Xic}.

While the server is working on a task, the server is sensitive to
interrupts.  An interrupt will cause the server to abort the current
task and begin listening for new instructions.  The interrupt handling
works about the same as in graphical mode when the user types
{\kb Ctrl-c}, though there is no confirmation prompt --- the task is
always aborted.  There may be a short delay before the interrupt is
recognized.
    
Interrupts can be sent to the server by sending an interrupt (``{\vt
INT}'') to the process number of the server with the Unix {\vt kill}
command.  The server socket will also raise an interrupt if out of
band (OOB) data are received.  Thus, the client can send a single
arbitrary byte of OOB data to generate an interrupt.  The Unix manual
pages describe the concept of OOB data.

The text expected by the daemon is in the form of statements which can
be understood by the script interpreter, i.e., script lines.  In
addition, there are a number of special control commands.

As more than one connection can exist at the same time, commands from
one connection can dramatically alter the environment seen by the
other connections, including clearing of data and killing the server. 
Though the connections are separate, they should be considered as
multiple windows into a single processing environment rather than
separate processing environments.

Generally, when the last connection closes, all data within the server
will be cleared and its state reinitialized, though this can be
suppressed, allowing persistence of state and data.

The server may be used as a ``geometry server'', providing compressed
representations of the geometry in cells, by layer, as from a Cell
Geometry Digest (CGD).  A connection object can be linked to a Cell
Hierarchy Digest (CHD), allowing operations with the CHD to obtain
geometry through the server.  This would reduce memory use on the
local machine, assuming that the geometry is stored on a remote
server.

The built-in non-script commands are described below.  All other input
should be parsable by the script parser, except that lines that start
with `\#' are not allowed, so no comments or preprocessor directives
are allowed.

All transmissions to the server are readable ASCII text, using
standard network ``{\vt $\backslash$r$\backslash$n}'' line
termination.  Replies from the server are in a binary form described
below.

After each line of input is given, the server returns a message giving
the data type and possibly the data for each script command.  Most
script functions return some value.  Assignments return the value
assigned.  A variable name returns the value of that variable, if the
variable has a known type.  The default mode is to return only the
data type code, which minimizes the network overhead.  Optionally, the
{\vt longform} command can be applied, in which case the data are
returned.  Note that this can be arbitrarily large for some data
types.

\begin{description}
\item{\vt close}\\
This will close the connection to the daemon, and is the normal way to
end a session.  If no other connections are open, the daemon will
generally clear the database of all cells and otherwise initialize
itself to a clean state for the next connection (effectively calling
{\vt reset} and {\vt clear}, see below), though this can be suppressed
with {\vt keepall} (see below).  The daemon will continue listening
for new connections.

\item{\vt kill}\\
This will close the connection and cause the server to exit.

\item{\vt reset}\\
This command will reset the script parser to its initial state,
exiting from any control block in effect and deleting any script
variables that may have been defined previously.  This will affect all
open connections.

\item{\vt clear}\\
This will clear the server database of all cells, and delete any
layers that were not initially read from the technology file.  This is
equivalent to calling the {\vt ClearAll} script function.  This will
affect all open connections.

\item{\vt longform}\\
After each line of script input is given and the line processed, a
response message will be returned based on the computed result from
the line, if any.  The user has a choice of receiving a very brief
reply, giving only the response code - an integer which indicates
pass/fail and the type of computed data, if any.  The other choice is
to actually return the data along with the response code.  The data
can be arbitrarily large.

The default return is ``shortform'' which does not transmit the data
values.  Giving this command switches to the mode where values are
returned, for the present connection only.

\item{\vt shortform}\\
When given, subsequent replies fro the present connection will use the
short form for returned data, which consists of only the data type
code.  This is the default.

\item{\vt dumpmsg}\\
When given, the text of subsequently received messages from the
present connection is printed, surrounded by vertical bar (`{\vt |}')
symbols, on the standard output, meaning that the text will appear in
the {\vt daemon\_out.log} file in normal operation.  If the server is
running in debugging mode (the {\et XTNETDEBUG} environment variable
was found when the server started), this text will be printed on the
console window.

\item{\vt nodumpmsg}\\
This turns off the printing of received messages if {\vt dumpmsg} was
given.  It has no effect otherwise, and applies only to the current
connection.

\item{\vt dieonerror}\\
Ordinarily, if the client crashes or there is a connection failure,
the server will simply reset itself and continue waiting for new
connections and handling other existing connections.  If {\vt
dieonerror} was given, the server will instead exit on failure of the
current connection.

\item{\vt nodieonerror}\\
This will undo the effect of {\vt dieonerror}, if {\vt dieonerror} was
given, and has no effect otherwise.  It applies only to the current
connection.

\item{\vt keepall}\\
Ordinarily, when the server receives a {\vt close} command, and there
are no other connections open, the interpreter context is reset, the
cell database is cleared, and other steps are taken to provide a clean
environment for the next connection.  If this command is given, all of
this will be skipped, so that the same context and environment will be
available to the next connection.  This is a single flag which can be
set or reset from any connection, but applies to all connections.

\item{\vt nokeepall}\\
This will undo the effect of {\vt keepall}, if {\vt keepall} was
given, and has no effect otherwise.  This can be given from any
connection, and applies to all connections.

\item{\vt geom} [{\it chd\_name\/}] [{\it cellname\/}]\\
The {\vt geom} command implements the ``geometry server'', and unlike
the other built-in commands this is an actual function and does not
affect the interface state.

Information from Cell Geometry Digests saved in server memory is made
available through this interface.  The {\vt OpenCellGeomDigest} script
function can be used to create CGDs in the server, and of course the
target layout file must be accessible to the server.

All of the arguments that follow ``{\vt geom}'' are optional, though
arguments to the left of a given argument are required.  Below are the
accepted forms and returns.  In all cases, the actual data are
returned, as with {\vt longform}.

\begin{description}
\item{\vt geom}\\
If no arguments are given, the reply is a space-separated string
listing of CGD access names found in the server.  If an access
name contains white space, it will be quoted.

\item{\vt geom ?} {\it cgd\_name}\\
This form will return the string ``{\vt y}'' if {\it cgd\_name} is the
access name of a CGD in memory, ``{\vt n}'' if not found.

\item{\vt geom} {\it cgd\_name}\\
The argument is taken as an access name of a CGD in server memory. 
The return is a string containing space-separated cell names found in
the indicated CGD.

\item\parbox[b]{4in}{{\vt geom} {\it cgd\_name} {\vt -?}\\
  {\vt geom} {\it cgd\_name} {\vt ?-}\\
  {\vt geom} {\it cgd\_name} {\vt -}}\\
The argument is taken as an access name of a CGD in server memory.
The return is a string containing space-separated cell names that
have been removed from the CGD.

\item{\vt geom} {\it cgd\_name} {\vt ?} {\it cellname}\\
This form will return the string ``{\vt y}'' if {\it cgd\_name} is the
access name of a CGD in memory, and {\it cellname} is found in that
CGD.  The string ``{\vt n}'' is returned if the CHD access name
matches a CGD name, but the {\it cellname} is not found in that CGD. 
An empty string is returned otherwise.

\item{\vt geom} {\it cgd\_name} {\vt -} {\it cellname}\\
if the {\it cgd\_name} and {\it cellname} match a CGD and cell, that
cell will be removed from the CGD, and resources freed.  However, the
cell name and its status as having been removed is retained.  This
will return the string ``{\vt y}'' if {\it cgd\_name} is the access
name of a CGD in memory, and {\it cellname} is found in that CGD (and
removed).  The string ``{\vt n}'' is returned if the CHD access name
matches a CGD name, but the {\it cellname} is not found in that CGD. 
An empty string is returned otherwise.

\item\parbox[b]{4in}{{\vt geom} {\it cgd\_name} {\vt -?} {\it cellname}\\
  {\vt geom} {\it cgd\_name} {\vt ?-} {\it cellname}}\\
These forms will return the string ``{\vt y}'' if {\it cgd\_name} is
the access name of a CGD in memory, and {\it cellname} has been
removed from that CGD.  The string ``{\vt n}'' is returned if the CHD
access name matches a CGD name, but the {\it cellname} is not in the
removed list for CGD.  An empty string is returned otherwise.

\item{\vt geom} {\it cgd\_name} {\it cellname}\\
If two arguments, they are taken as the CGD access name and a cell
name in the indicated CGD.  The return is a string consisting of
space-separated layer names of layers in the cell that contain
geometry.

\item{\vt geom} {\it cgd\_name} {\it cellname} {\vt ?} {\vt layername}\\
This form will return the string ``{\vt y}'' if {\it cgd\_name} is the
access name of a CGD in memory, and {\it cellname} is found in that
CGD, and {\it layername} the name of a layer found in that cell.  The
string ``{\vt n}'' is returned if the CHD access name matches, but
either {\it cellname} or {\it layername} is not found.  An empty
string is returned otherwise.

\item{\vt geom} {\it cgd\_name} {\it cellname} {\vt layername}\\
With this form, the return value is the compressed string representing
the geometry.  These data have a unique return class, described in the
format documentation below.
\end{description}
\end{description}

The normal way to terminate a session with the server is to issue the
{\vt close} command.  Unless {\vt keepall} is in effect, if there are
no other open connections the server will be cleared and
reinitialized.  The clearing and reinitialization is equivalent to
giving the {\vt reset} and {\vt clear} commands, which can be given at
any time from any connection, and affects all connections.  If the
{\vt keepall} command was in effect, the server will not be reset and
cleared before the connection is closed, thus its state will be
retained for the next connection.  If there is a communications error,
the server will exit if {\vt dieonerror} was in effect for the
affected connection, otherwise the behavior will be the same as for a
{\vt close} operation.

There is quite a bit of internal server state that is not reset to any
preset value between connections.  Examples are the mode (physical or
electrical) and the status of variables set with the {\cb !set}
command or {\vt Set} function.  Thus, when writing scripts for
execution by the server, it is important to explicitly initialize any
such state or variable.

The {\vt ReadReply} and {\vt ConvertReply} script functions can be
used the to handle server responses when the client is implemented as
a script.  For other applications, the user will have to write a
parser, perhaps using the code from the {\vt xclient.cc} example. 
Whiteley Research can provide assistance to users who need to develop
this capability.

\subsection{The Response Message Format}
\index{server mode!protocol}

Numeric data are sent in ``network byte order'' which means that the
MSB arrives first.  Integers are always 32-bits, other numeric data
are 64-bit IEEE floating point values.  The raw bytes read for a
numeric value must be converted to the machine's byte order before
being processed in a program.  For integers, the {\vt ntohl} C library
function is usually available.  For floating values, an example
conversion function is provided in the {\vt xclient.cc} file.  The
byte order is the same as that used by Sun sparc systems, thus this
issue can be ignored on those systems, unless code portability is
desired.

All response messages begin with a 4-byte integer, which may
constitute the entire message in some circumstances.  This (and all
numeric values) is in network byte order, so must be converted to host
byte order before processing.  The first integer is the ``response
code'' possibly ORed with the ``longform'' flag.  The response code is
an integer 0-9, and the longform flag is hex value 80.

If the longform flag is not set, then no more data exists in the
message.  Otherwise, most response codes will be followed by
additional data.  The possible responses are described below.

\begin{description}
\item{0}\\
This is the server ``ok'' message.  There is no additional data.

\item{1}\\
This is the server ``more'' message.  There is no additional data.
This response is given when the server is waiting for input required
to complete a script conditional block, for example:

\begin{quote}
\begin{tabular}{ll}
\bf command  & \bf response\\
\vt keepall  & 0\\
\vt if (x)   & 1\\
...          & \\
\vt end      & 0\\
\end{tabular}
\end{quote}

\item{2}\\
This is the server ``error'' message.  There is no additional data.
This response is given if the command produces an error.

\item{3}\\
This is the server ``scalar'' message.  If the longform flag is set,
there are 8 bytes of following data, representing a
double-precision IEEE floating-point value.

\item{4}\\
This is the server ``string'' message.  If the longform flag is set,
a 4-byte size integer follows, in turn followed by the string
characters.  The size value is the number of characters in the
string and includes the null termination character of ASCII
strings.

\item{5}\\
This is the server ``array'' message.  If the longform flag is set,
a 4-byte integer follows, giving the number of elements in the
array.  This is followed by the array data, 8 bytes per element,
in IEEE double-precision floating-point form.

\item{6}\\
This is the server ``zlist'' message.  If the longform flag is set, a
4-byte integer follows, which gives the number of trapezoids in the
list.  This is followed by the trapezoid list data, with 24 bytes per
trapezoid (six 4-byte integers each).  The values are coordinates in
the internal units (usually 1000 units per micron), in the order {\it
xll}, {\it xlr}, {\it yl}, {\it xul}, {\it xur}, {\it yu}.

\item{7}\\
This is the server ``lexpr'' message, which is the return for the
layer expression type.  This is treated as a string.  If the longform
flag is set, a 4-byte size integer follows, followed by the text of
the layer expression.  The size includes the null termination
character of the string.

\item{8}\\
This is the server ``handle'' message, which is the return for all
handle types.  This is basically useless on the local machine, since
the underlying data resides on the server.  If the longform flag is
set, a 4-byte integer follows, which gives the handle identification
value.

\item{9}\\
This is the server geometry stream message.  This message always
returns data, the longform flag is ignored.  The type 9 return is
unique to the geometry stream response from the {\vt geom} command. 
The ASCII string responses from the {\vt geom} command use type 4 in
the normal way, though they are always in ``longform''.  The type 9
record is very similar to a string, however the first 8 bytes of the
string contains two integers:  the first integer is the compressed
size of the following data, and the second integer is the uncompressed
size.  The compressed size can be zero, in which case compression is
not used.  The actual string length is the compressed size if nonzero,
otherwise the uncompressed size.  The string contains OASIS geometry
records, as in a CBLOCK if compressed.

The user will have to supply an OASIS reader to interpret the stream. 
{\Xic} provides script functions for this purpose.
\end{description}

\subsection{Operation}

Internal script variables are defined and set in accord with
instructions received.  The variables and other context are cleared
when an initial connection to the server is made or or final
connection broken (and {\vt keepall} is not in effect), or when ``{\vt
reset}'' is given.

Other state, such as the current directory and cells in {\Xic} memory,
is persistent, thus users should initialize {\Xic} appropriately, and
clear the database before closing the connection.

While in server mode (also in batch mode) the {\Xic} functions that
query the user for some decision are not available.  If the prompt
line editor is invoked, it will return immediately as if the user hit
{\kb Enter}.  The return value is the default string, if any, or any
text that was previously supplied with the {\vt StuffText} function. 
The {\cb Merge Control} behavior is as if the {\et NoAskOverwrite}
variable was set, i.e., the overwriting behavior will be the default
as set with the {\et NoOverwritePhys} and {\et NoOverwriteElec}
variables.  If neither of these is set, the action will be to
overwrite the cell in memory.

The server produces a log file directory in the same manner as under
normal {\Xic} operation.  These files are removed when the server
exits normally, i.e., when a ``{\vt kill}'' command is received.  In
server mode, there are files used that are not used in normal mode:

\begin{description}
\item{\vt daemon.log}\\
  This records connection activity to the daemon.
\item{\vt daemon\_out.log}\\
  This records the ``stdout'' channel from the daemon, i.e., the text
  that would go to the console in normal mode.  Under Microsoft
  Windows, this file is not located with the other log files, but is
  created in the parent directory of the directory containing the log
  files.  This is due to a technical issue in Windows.
\item{\vt daemon\_err.log}\\
  This records the ``stderr'' channel from the daemon, i.e., the error
  text that would go to the console in normal mode.  Under Microsoft
  Windows, this file is not located with the other log files, but is
  created in the parent directory of the directory containing the log
  files.  This is due to a technical issue in Windows.
\end{description}

