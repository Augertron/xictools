% -----------------------------------------------------------------------------
% Xic Manual
% (C) Copyright 2009, Whiteley Research Inc., Sunnyvale CA
% $Id: editmenu.tex,v 1.77 2017/03/22 07:29:56 stevew Exp $
% -----------------------------------------------------------------------------

% -----------------------------------------------------------------------------
% xic:editmenu 031815
\chapter{The Edit Menu:  Edit Layout}
\index{Edit Menu}
The {\cb Edit Menu} contains commands which control aspects of layout
editing, such as transformations and other settings, and commands that
bring up panels that control cell placement and flattening, property
editing, and other functions.

The table below summarizes the commands that appear in the {\cb Edit Menu},
including the internal command name and the command function.

\begin{tabular}{|l|l|l|p{1.75in}|} \hline
\multicolumn{4}{|c|}{\kb Edit Menu}\\ \hline
\kb Label & \kb Name & \kb Pop-up & \kb Function\\ \hline\hline
\et Enable Editing & \vt cedit & none & Enable/disable editing mode
 for current cell\\ \hline
\et Setup & \vt edset & \cb Editing Setup & Show {\cb Editing Setup} panel\\
 \hline
\et PCell Control & \vt pcctl & \cb PCell Control & Set pcell options\\ \hline
\et Create Cell & \vt crcel & none & Create new cell\\ \hline
\et Create Via & \vt crvia & none & Create a standard via\\ \hline
\et Flatten & \vt flatn & \cb Flatten Hierarchy & Flatten hierarchy\\ \hline
\et Join/Split & \vt join & \cb Join or Split Objects & Control join/split
  operations\\ \hline
\et Layer Expression & \vt lexpr & \cb Evaluate Layer Expression & Control
  layer expression evaluation\\ \hline
\et Properties & \vt prpty & \cb Property Editor & Edit properties\\ \hline
\et Cell Properties & \vt cprop & \cb Cell Property Editor & Edit cell
  properties\\ \hline
\end{tabular}


% -----------------------------------------------------------------------------
% properties 021513
\section{Cell, Instance, and Object Properties}
\index{properties}

A property consists of an integer and a corresponding text string. 
Every database object, including cells, instances, and geometrical
objects, has the native ability to accept properties, though this is
enabled selectively.  Properties are saved in the design data file
along with the item to which it is attached.

The {\cb Property Editor}, which is brought up with the {\cb
Properties} button in the {\cb Edit Menu}, provides the primary means
of property manipulation of objects found in the current cell.  The
{\cb Cell Property Editor}, which is obtained with the {\cb Cell
Properties} button in the same menu, provides the primary means for
manipulating properties of the current cell itself.

Properties can be applied to physical objects and cells by the user,
using the user's property number and format, to suit tye user's
purposes.  This is fine, as long as the user's property numbers are
outside of the range reserved by {\Xic}.  Other properties are set by
{\Xic} for internal use such as to store the grid used for the layout,
or the GDSII end style for wires.  Still others may be set by the
user, but have significance to {\Xic}.  In schematic layouts, these
properties define electrical parameters for devices, and are used when
generating SPICE output.  In electrical mode, the user is restricted
to a small set of properties understood by {\Xic}, whereas in physical
mode any non-restricted number and string are allowed.  Finally,
{\Xic} supports ``pseudo-properties'' which are not actual properties,
i.e., they are not stored, but their application modifies or returns
some parameter related to the object.  These are listed in the {\cb
Property Editor} when in physical mode.


%-----------------------------------
%physprops 032015
\subsection{Physical Mode Properties}

In physical mode, the user has wide freedom to apply properties to
cells, subcells, and objects.  The only constraint is that the
following number ranges are restricted and must be avoided, for
arbitrary user-specified properties.  However, there are some
properties within the range that can be set by the user, that have
specific syntax requirements and meaning to {\Xic}.  These will be
described.

The property number ranges used by {\Xic} are:
\begin{description}
\item{1 -- 30}\\
These numbers are used or reserved in electrical mode only, they can
be used freely in physical mode cells, instances, and objects.
\item{7000 -- 7199}\\
Property values in this range are reserved for use by {\Xic},
and should (in general) not be assigned by the user.
\item{7200 -- 7299}\\
These values are reserved for the pseudo-properties (see
\ref{pseudoprops}) and can not be used for other purposes.
\end{description}

The {\cb Property Editor} from the {\cb Properties} button in the {\cb
Edit Menu} is used to assign properties to cell instances and objects
within the current cell.  The {\cb Add} button in the editor allows
addition of special properties used by {\Xic}, and arbitrary
user-specified properties.  See the {\cb Property Editor} description
in \ref{prptybut}) for a description of the properties that can be
added.

The object's existing properties, and available pseudo-properties are
listed in the {\cb Property Editor}, with color and syntax coding to
indicate the classification.  Pseudo-properties are not actual
properties, but when applied to an object will change or report some
parameter of the object.

The {\cb Cell Property Editor} from the {\cb Cell Properties} button
in the {\cb Edit Menu} allows editing of the properties of the current
cell.  See the {\cb Cell Property Editor} description for a list and
discussion of the cell properties that can be added with the {\cb Add}
menu.

A description of the physical properties used by {\Xic}, including the
property string syntax, is provided in \ref{physpropfmt}.


%--------------------------------------------
% pseudoprops 101016
\subsection{Pseudo-Properties}
\index{pseudo-properties}
\label{pseudoprops}

{\Xic} supports ``pseudo-properties'' which when applied are not saved
as properties, but rather change or return some parameter related to
the object.  This allows the property setting mechanism to be used to
alter the physical layout, which can be an important feature in design
automation.

In physical mode, when the {\cb Property Editor} is in use, the
listing will include the available pseudo-properties for the current
object.  The pseudo-properties can be ``added'' or ``edited'' to
modify the current object (or all objects if in global mode).  As
usual, such changes can be undone/redone with the standard operations. 

Internally, pseudo-properties can be applied to any object, electrical
or physical.  Many of the script functions that modify objects use the
pseudo-property mechanism internally.  These functions can take
electrical or physical input.  The graphical user interface, though,
allows pseudo-properties to be applied in physical mode only, through
the {\cb Property Editor}.

The pseudo-properties are listed below, giving the property number
and an internal name for the pseudo-property.

\begin{description}
\index{pseudo-properties!XprpType}
\index{XprpType pseudo-property}
\item{7200: {\et XprpType}}\\
This value can be read from all objects.  The returned property string
consists of a single character:  {\vt b}, {\vt p}, {\vt w}, {\vt l},
or {\vt c} for boxes, polygons, wires, labels, or subcells
respectively.  The returned value indicates the type of object.

\index{pseudo-properties!XprpBB}
\index{XprpBB pseudo-property}
\item{7201: {\et XprpBB}}\\
This value can be read from all objects, and can be applied to boxes,
polygons, wires, and labels.  The property string is in the form {\it
left\/},{\it bottom} {\it right\/},{\it top} where the {\it left\/},
etc.  are the coordinates of the object's bounding box in internal
units.  The {\et x} and {\et y} values are separated by commas.  When
this property is applied to an object other than a subcell, the
object's geometry is stretched to conform to the bounding box given.

\index{pseudo-properties!XprpLayer}
\index{XprpLayer pseudo-property}
\item{7202: {\et XprpLayer}}\\
This value can be read from all objects, and can be applied to boxes,
polygons, wires, and labels.  The property string is the name of the
layer on which the object is defined.  For subcells, the returned name
is ``{\vt \$\$}'', which is the internal name for the layer on which
subcells are defined.  When this property is given to an object (not a
subcell), and if the name is found in the layer table, the object will
be moved to the given layer.

\index{pseudo-properties!XprpFlags}
\index{XprpFlags pseudo-property}
\item{7203: {\et XprpFlags}}\\
This value can be read from all objects, and can be applied to all
objects.  The property string is a list of values and keywords
corresponding to special flags associated with the object.  These
flags are set internally, and should not be set by the general user.

\index{pseudo-properties!XprpState}
\index{XprpState pseudo-property}
\item{7204: {\et XprpState}}\\
This value can be read from all objects, and can be applied to all
objects.  The property string contains one of the keywords {\vt
normal}, {\vt selected}, {\vt deleted}, {\vt incomplete}, and {\vt
internal}.  This indicates a state value for the object which is used
internally.  These values should not be set by the general user.

\index{pseudo-properties!XprpGroup}
\index{XprpGroup pseudo-property}
\item{7205: {\et XprpGroup}}\\
This value can be read from all objects, and can be applied to all
objects.  The property string is an integer corresponding to the
conductor group assigned to the object by the extraction system. 
Though all objects have this data field, it has relevance to objects
that are defined on conducting layers only.  It is generally unwise
for the user to set this value.

\index{pseudo-properties!XprpCoords}
\index{XprpCoords pseudo-property}
\item{7206: {\et XprpCoords}}\\
This value can be read from all objects, and can be applied to boxes,
polygons, wires, and labels.  The property string is a list of
coordinates, one for each vertex, with the {\et x} and {\et y} values
separated by a comma.  Line feeds are included in returned strings to
keep the line length below 80 characters.  The values are in internal
units.  For boxes, labels, and subcells, the coordinates are those of
the bounding box.  For polygons and wires, the coordinates are the
actual vertices.  For all but wires, the first and last coordinates
are the same, i.e., the path is closed.  For boxes and polygons,
applying this property will change the object geometry.  If the new
geometry is a Manhattan rectangle, the new figure will be a box,
otherwise it will be a polygon.  When applied to wires, the new object
will always be a wire, but with the new path.  The coordinates given
to a label must describe a Manhattan rectangle, and the label will be
stretched to fill the given rectangle, as with applying {\et XprpBB}.

\index{pseudo-properties!XprpMagn}
\index{XprpMagn pseudo-property}
\item{7207: {\et XprpMagn}}\\
This value can be read from all objects, and can be applied to all
objects.  The return value is ``1.000000'' for objects other than cell
instances, and the magnification value for cell instances.  When
applied to an object or cell instance, the size of the object will
change, and a ``reference point'' of the object will remain in a fixed
location.

\begin{quote}
\begin{tabular}{ll}\\
\bf object & \bf reference point\\
box & lower-left corner\\
polygon & first vertex in internal list\\
wire & first vertex in internal list\\
label & label reference point\\
instance & transformed master origin\\
\end{tabular}
\end{quote}

\index{pseudo-properties!XprpWwidth}
\index{XprpWwidth pseudo-property}
\item{7208: {\et XprpWwidth}}\\
This value can be read from wires, and can be applied to wires.  The
property string is the width of the wire in internal units.  When
applied to a wire, the width will take the new value.  This has no
effect when applied to objects other than wires.

\index{pseudo-properties!XprpWstyle}
\index{XprpWstyle pseudo-property}
\item{7209: {\et XprpWstyle}}\\
This value can be read from and applied to wires.  When the property
string is read, only the first character is significant, the rest if
any are ignored.  This is used to set the end style of the wire to one
of three possible states:  {\bf flush}, {\bf rounded}, or {\bf
extended}.  In {\Xic}, both wire ends will have the same style.  If
{\bf flush}, the wire is truncated normal to the edges at the end
vertices.  If {\bf rounded}, the wire continues beyond the vertex by
half the wire width, but is given a rounded (ideally semicircular)
shape (this style is rarely used and is not recommended).  The {\bf
extended} style is similar in that the wire extends a half-width past
the vertices, but the end is square.  This is the {\Xic} default.

\begin{quote}
\begin{tabular}{ll}
\bf leading character & \bf style\\
{\vt f},{\vt F}, or {\vt 0} (zero) & flush\\
{\vt r},{\vt R}, or {\vt 1} (one) & rounded\\
{\vt e},{\vt E}, or {\vt 2} & extended\\
\end{tabular}
\end{quote}
 
Applying this property to a wire will cause that wire to be rendered
with the given end style.  The property has no effect if given to
objects other than wires.

\index{pseudo-properties!XprpText}
\index{XprpText pseudo-property}
\item{7210: {\et XprpText}}\\
This value can be read from labels, and can be applied to labels.  The
return value is the text of the label.  The full text including
encoded hypertext entries is provided.  When applied to a label, the
label takes the new text.  There is no effect if this property is
applied to objects other than labels.

\index{pseudo-properties!XprpXform}
\index{XprpXform pseudo-property}
\item{7211: {\et XprpXform}}\\
This value can be read from and applied to text labels.  It controls a
set of flags associated with the label which define the presentation
attributes.

The general syntax for the string value is
\begin{quote}
[{\vt +|-}] [{\vt 0x}]{\it hex\/}{\vt |}{\it word\/}[,...]
\end{quote}

Optionally, the string begins with a {\vt +} or {\vt -} character. 
If {\vt +} appears, it indicates that the flag bits that are specified
will be set, and those not specified will be unchanged.  If {\vt -}
is given, the flag bits specified will be unset, those not specified
will be unchanged.  If neither, the flags will be set to a new value
consisting of the bits specified which are set, other bits are not
set.

The remaining part of the string effectively specifies a set of flag
bits.  This consists of space or comma-separated keywords or hex
integers.  Hex integers can have an optional ``{\vt 0x}'' or ``{\vt
0X}'' prefix.  The overall value is the OR of all terms given.  The
table below lists the accepted keywords and the equivalent flag bits. 
Keyword recognition is case-insensitive.

\begin{quote}
\begin{tabular}{|l|l|l|} \hline
\bf Word & \bf Hex Bits & \bf Description\\ \hline\hline
\vt R0 & 0 & no rotation (dummy token)\\ \hline
\vt R45 & 10 & 45 degree rotation\\ \hline
\vt R90 & 1 & 90 degree rotation\\ \hline
\vt R135 & 11 & 135 degree rotation\\ \hline
\vt R180 & 2 & 180 degree rotation\\ \hline
\vt R225 & 12 & 225 degree rotation\\ \hline
\vt R270 & 3 & 270 degree rotation\\ \hline
\vt R315 & 13 & 315 degree rotation\\ \hline
\vt MY & 4 & mirror Y after rotation\\ \hline
\vt MX & 8 & mirror X after rotation and mirror Y\\ \hline
\vt HJL & 0 & left justify (dummy token)\\ \hline
\vt HJC & 20 & center X justify\\ \hline
\vt HJR & 40 & right justify\\ \hline
\vt VJB & 0 & bottom justify (dummy token)\\ \hline
\vt VJC & 80 & center Y justify\\ \hline
\vt VJT & 100 & top justify\\ \hline
\vt T0 & 0 & text font 0 (dummy)\\ \hline
\vt T1 & 200 & text font 1 (unused)\\ \hline
\vt T2 & 400 & text font 2 (unused)\\ \hline
\vt T3 & 600 & text font 3 (unused)\\ \hline
\vt SHOW & 1000 & show hidden label\\ \hline
\vt HIDE & 2000 & hide label\\ \hline
\vt TLEN & 4000 & show in top-level only\\ \hline
\vt LIML & 8000 & limit lines\\ \hline
\end{tabular}
\end{quote}

The {\vt HJR} will override {\vt HJC} if both are given, similarly
{\vt VJT} will override {\vt VJC}.

The {\vt SHOW}/{\vt HIDE} bits are for implementing a clickable text
display, where the label text can be shown or ``hidden'' by rendering
a small glyph instead.  At most one of these bits should be set. 
Either bit overrides the default which is in force when neither is
set.  These can be applied to any label, however the ``clickability''
of the label is set by the {\et LabelHiddenMode} variable.  All labels
are ``clickable'' by default, press {\kb Shift} and click on the label
to toggle the hidden/viewing status.

The {\vt TLEV} bit gives the label the property of being invisible in
instances of the containing cell, but visible when the cell is viewed
as the top-level (current cell).

The {\vt LIML} bit causes the label to limit the number of lines
displayed, when the label text has multiple lines.  The maximum line
count defaults to 5, and is otherwise given with the {\et
LabelMaxLines} variable.

\ifoa
The {\vt TLEV} and {\vt LIML} bits may be applied when reading
schematic cells through OpenAccess for Virtuoso compatibility, but are
not otherwise used in {\Xic}, except as controlled through this
pseudo-property.
\fi

When applied to a label, the label will be rendered using the new flags. 
This property has no effect when applied to objects other than labels.

\index{pseudo-properties!XprpArray}
\index{XprpArray pseudo-property}
\item{7212: {\et XprpArray}}\\
This value can be read from subcell instances, and can be applied to
subcell instances.  The property string is of the form ``{\it
nx\/},{\it ny} {\it dx\/},{\it dy}'' where {\it nx} and {\it ny} are
the number of columns and rows, and the {\it dx} and {\it dy} are the
center to center spacings in internal units, for an array of subcells. 
When applied to an instance, the array parameters of the instance are
correspondingly changed.  This property has no effect on objects other
than subcells.

\index{pseudo-properties!XprpTransf}
\index{XprpTransf pseudo-property}
\item{7213: {\et XprpTransf}}\\
This value can be read from subcell instances, and can be applied to
subcell instances.  The property string is the CIF transformation
string for the instance, with coordinates in internal units.  When
applied to an instance, the instance placement and orientation change
to reflect the new transformation.  This property has no effect on
objects other than subcells.

\index{pseudo-properties!XprpName}
\index{XprpName pseudo-property}
\item{7214: {\et XprpName}}\\
This value can be read from subcell instances, and can be applied to
subcell instances.  The property string is the name of the
instantiated cell.  If this property is set, the instance is replaced
by an instance of the given cell name.  The current transform is added
to the existing transform when the new instance is placed.  This
property has no effect on objects other than subcells.

\index{pseudo-properties!XprpXY}
\index{XprpXY pseudo-property}
\item{7215: {\et XprpXY}}\\
This pseudo-property has a value that is an x,y coordinate, and can be
read from or applied to any object or subcell.  The interpretation of
this coordinate depends on the type of object.  For boxes, it is the
lower-left corner.  For polygons and wires, it is the first vertex in
the vertex list.  For labels, it is the text anchor point, and for
subcells it is the placement coordinate.  Setting the property is
equivalent to moving the object.

\index{pseudo-properties!XprpWidth}
\index{XprpWidth pseudo-property}
\item{7216: {\et XprpWidth}}\\
This pseudo-property returns the width of any object or cell instance
in internal units.  It can be applied to objects but not cell
instances, and will scale the object to the specified width.

\index{pseudo-properties!XprpHeight}
\index{XprpHeight pseudo-property}
\item{7217: {\et XprpHeight}}\\
This pseudo-property returns the height of any object or cell instance
in internal units.  It can be applied to objects but not cell
instances, and will scale the object to the specified height.
\end{description}

The settable pseudo-properties for an object are listed in the {\cb
Property Editor}, along with the ``real'' properties.  These can be
changed in the same way, which will produce physical changes to the
object.

%-------------------------------------
% elecprops 110313
\subsection{Electrical Mode Properties}
\label{elprops}

In electrical mode, only properties with certain values and data can
be entered, and only to objects corresponding to library devices or
subcircuit instances.

These properties are generally applied with the {\cb Property Editor}
from the {\cb Properties} button in the {\cb Edit Menu}.  However, the
most frequently used properties, such as those that set device
parameters, have values that are shown on-screen as text labels.  If
one edits the label, the underlying property value is changed as well. 
This is generally more convenient than using the {\cb Property
Editor}.  Labels are edited by first selecting a label by clicking on
it, then entering the {\cb label} command by pressing the button in
the side menu.

A description of the electrical properties used by {\Xic}, including
the property string syntax, is provided in \ref{elecpropuser}.

See the description of the {\cb Property Editor} for a listing and
discussion of the properties that can be added with the {\cb Add}
menu.

Properties of the current call can be added or modified with the {\cb
Cell Property Editor} from the {\cb Cell Properties} button in the
{\cb Edit Menu}, and are listed in the {\cb Add} menu of the editor. 
See the {\cb Cell Property Editor} description for a list and
discussion of these properties.


% -----------------------------------------------------------------------------
% xic:cedit 111609
\section{The {\cb Enable Editing} Button: Enable Cell Editing}
\index{Enable Editing button}
This button tracks the state of the IMMUTABLE flag of the current
cell, and will alter the flag if editing mode is changed.  When the
current cell is IMMUTABLE, it can not be modified, and all editing
features are disabled.  The side menu is hidden, the {\cb Modify Menu}
is disabled, and all but this button and {\cb Create Cell} are
disabled in the {\cb Edit Menu}.

If there is no current cell, editing features are disabled.

Cells read into {\Xic} through the library mechanism have the
IMMUTABLE flag set.  This button can be used to allow modification of
these cells.

Setting the IMMUTABLE flag of the current cell from the {\cb Flags}
button in the {\cb Cells Listing} panel from the {\cb Cells Menu} or
with the {\cb !setflag} command will have the same effect as use of
this button.


% -----------------------------------------------------------------------------
% xic:edset 100416
\section{The {\cb Setup} Button: Show Editing Setup Panel}
\index{Setup button (Edit Menu)}

The {\cb Setup} button in the {\cb Edit Menu} brings up the {\cb
Editing Setup} panel.  The panel provides controls for setting various
parameters and options which apply during layout editing.

\begin{description}
\item{\cb Constrain angles to 45 degree multiples}\\
When this check box is active, wire and polygon vertices are
constrained to form angles of multiples of 45 degrees.  By default, a
``smart'' path generator is employed, which will construct a valid
path to the pointer location from the previous point during wire or
polygon construction.  This will often add two vertices:  a 45 degree
extension, followed by a Manhattan extension, in order to connect the
points.  If the {\kb Ctrl} key is held while the new point is defined,
the ``smart'' feature is disabled, and only one new vertex is added. 
If the {\kb Shift} key is held, then the 45 degree constraint is
removed entirely.

The {\et Constrain45} variable tracks the state of this check box, and
setting or clearing the variable will also set or clear the mode.

When active, rotation angles available in the {\cb spin} command, and
translation angles in the {\cb Stretch} command, and the vertex
editors for polygons and wires, are constrained to multiples of 45
degrees.  However, pressing the {\kb Shift} key will remove the
constraint in these commands while the key is held.  If the {\et
Constrain 45} variable is not defined, holding {\kb Shift} will impose
the 45 degree angle constraint.  Thus, the {\kb Shift} key inverts the
effective state of the {\et Constrain 45} variable (and this check
box) in these commands.

\item{\cb Merge new boxes and polys with existing boxes/polys}\\
When this check box is set, new boxes and polygons that are created
with the side menu commands are merged with existing boxes and
polygons to form a larger polygon in the database.  New wires will be
connected to existing wires on the same layer with the same width and
endpoint, but do not participate in the box/polygon merging of new
objects.

However, on layers with the {\vt NoMerge} technology file keyword set,
merging is always suppressed.

The state of this check box tracks the logically inverted state of the
boolean variable {\et NoMergeObjects}, which can (equivalently) be set
with the {\cb !set} command.

Existing objects can be similarly joined, or split into trapezoids,
with the buttons in the {\cb Join or Split Objects} panel brought up
with the {\cb Join/Split} button in the {\cb Edit Menu}.

Join (merging) operations are subject to the settings of several
variables, which have equivalent entries in the {\cb Join or Split
Objects} panel.  These limit the complexity of polygons created by
merging, mostly for optimizing for speed for merging large object
collections.

The {\cb Clip and merge new boxes only, not polys} button in this
panel modifies the merging behavior to clip and merge boxes only.

This object merging is separate and unrelated to the box merging
available when reading a layout file into the database, which has a
separate merging control in the {\cb Setup} page of the {\cb Import
Control} panel from the {\cb Convert Menu}.

\item{\cb Clip and merge new boxes only, not polys}\\
When merging is enabled (the {\cb Merge new boxes and polys with
existing boxes/polys} check box is active), and this check box is also
active, polygons are ignored when merging, and new boxes are
clipped/merged.  This was the merging behavior in releases prior to
3.1.7.

This setting has no effect if merging is not enabled.  It tracks the
state of the {\et NoMergePolys} variable.

\item{\cb Prompt to save modified native cells}\\
When the box is checked, the user will be prompted to save the current
cell if the cell is modified, and would be saved as a native cell
file, and a new current cell is about to be set.  This was standard
behavior in releases earlier than generation 4.  Although it is always
a good idea to save work periodically, the prompt can be annoying to
experienced users and is now disabled by default.  The user will be
given the chance to save modified cells when exiting {\Xic} in any
case.

This check box tracks the state of the {\et AskSaveNative} variable. 
The variable can be set as a boolean or cleared to change the mode,
which is equivalent to checking or un-checking this check box.

\item{\cb No wire width change in magnification}\\
When the box is checked, the width of wires does not change when the
wire undergoes magnification, in a {\cb Move}, {\cb Copy}, or {\cb
Flatten} operation.

This check box tracks the state of the {\et NoWireWidthMag} variable. 
The variable can be set as a boolean or cleared to change the mode,
which is equivalent to checking or un-checking this check box.

\item{\cb Allow Create Cell to overwrite existing cell}\\
When this check box is active, The {\cb Create Cell} operation in the
{\cb Edit Menu} and the {\vt CreateCell} script function can overwrite
cells already in memory.  This can be dangerous and is prevented by
default, and the user is advised to be careful if using this feature.

This check box tracks the state of the {\et CrCellOverwrite} variable. 
The variable can be set as a boolean or cleared to change the mode,
which is equivalent to checking or un-checking this check box.

\item{\cb Maximum undo list length}\\
This integer entry sets the number of operations remembered in the
{\cb Undo} command.  If not set, 25 operations are saved.  If set to
zero, the length is unlimited.

This entry tracks the value of the {\et UndoListLength} variable.  The
variable can be set as an integer or cleared to change the value,
which is equivalent to changing the integer entry in this panel.

\item{\cb Maximum number of ghost-drawn objects}\\
This integer entry sets the maximum number of objects to render
individually as ``ghosts'' attached to the mouse pointer during
operations such as move and copy.  This can be set to an unsigned
integer in the range 50--50000.  If there are more than this number,
some outlines won't be shown, the smaller-area objects will be
skipped.  The default is 4000 if this variable is not set.  If, when
moving a large number of objects, the pointer motion is too sluggish,
the user can set this variable to compensate.

This entry tracks the value of the {\et MaxGhostObjects} variable. 
The variable can be set as an integer or cleared to change the value,
which is equivalent to changing the integer entry in this panel.

\item{\cb Maximum subcell depth in ghosting}\\
This menu sets the maximum expansion depth for instance expansion in
ghosting.  If {\cb as expanded}, this is the same as the normal
expansion depth.  The actual expansion depth used in ghosting will not
be larger than the normal expansion depth, but can be smaller.  For
example, setting this to 0 (zero) will prevent expansion of ghosted
subcells entirely.

This entry tracks the value of the {\et MaxGhostDepth} variable. 
The variable can be set as an integer or cleared to change the value,
which is equivalent to changing the integer entry in this panel.
\end{description}


% -----------------------------------------------------------------------------
% xic:pcctl 091614
\section{The {\cb PCell Control} Button: PCell Control Panel}
\index{PCell Control button}
\index{PCell Control panel}
\index{pcell options}
The {\cb PCell Control} button in the {\cb Edit Menu} brings up the
{\cb PCell Control} panel.  From the panel, various options related to
parameterized cells (pcells, see \ref{pcells}) can be set.  The
following elements are available.

\begin{description}
\item{\cb Auto-abutment mode}\\
The drop-down menu provides three choices, {\cb Mode 1} through {\cb
Mode 3}.  This provides the three values for the {\et otherPinsOnNet}
parameter that is part of the Ciranova auto-abutment protocol as
implemented in {\Xic} (see \ref{pcabut}).  How the pcell uses this
parameter is up to the pcell author, there is really no {\it a-priori}
interpretation.

The Ciranova {\et Nmos2} example pcell interprets the value to
have the following meanings.  This is likely to be used in other
pcells as well.

\begin{tabular}{ll}\\
{\bf Mode 1} value 0 & Auto-abutment is disabled.\\
{\bf Mode 2} value 1 & Abutment takes place with no contact between
  the gates.\\
{\bf Mode 3} value 2 & Abutment takes place with a contact (to layer M1)
  between the gates.\\
\end{tabular}

This setting tracks the value of the {\et PCellAbutMode} variable. 
The default (when the variable is not set) is Mode 2.  The variable
can be set to the integers 0--2, which correspond to the three modes.

\item{\cb Hide and disable stretch handles}\\
{\Xic} implements the Ciranova stretch handle protocol (see
\ref{pcstretch}).  By default, the stretch handles are visible in
selected, expanded instances, and in the current cell if the cell is a
sub-master.  Setting this option will hide and disable all stretch
handles.

The {\et PCellHideGrips} variable tracks the state of this check box.

\item{\cb Instance min. pixel size for stretch handles}\\
The stretch handles of a selected instance are shown only if the
instance is displayed large enough to avoid inadvertently engaging the
stretch handles when using the mouse for other purposes.  By default,
the smaller of the instance width or height must be 100 pixels or
larger for stretch handles to appear.  This numerical entry will
modify this threshold.

The {\et PCellGripInstSize} variable tracks the value of this entry.

\item{\cb List sub-masters as modified cells}\\
When this check box is checked, the {\cb Modified Cells} panel, which
appears when exiting {\Xic} or from the {\cb !sa} command and {\cb
Save} button (in the {\cb Edit Menu}) when there are cells that are
modified and unsaved, will include sub-master cells.  Normally,
sub-masters live only in memory, to be re-created when needed, and are
excluded from the listing.  However, there may be times when it is
desirable to write these to disk.

The {\et PCellListSubMasters} variable tracks the state of this check
box.

\item{\cb Show all evaluation warnings}\\
By default, certain warning messages are suppressed while
evaluating a pcell script, since they can be annoying and are
probably only of interest to the pcell author.  At present, this
applies only to coincident (duplicate) object checking.  If this
check box is checked, these messages will pop up in a window when
the pcell sub-master is created.

The {\et PCellShowAllWarnings} variable tracks the state of this check
box.
\end{description}


% -----------------------------------------------------------------------------
% xic:crcel 062109
\section{The {\cb Create Cell} Button: Create New Cell}
\index{Create Cell button}
\index{cell creation}
The {\cb Create Cell} button in the {\cb Edit Menu} will create a new
cell from the currently selected objects.  The user is prompted for a
name for the new cell.  The new cell is created in memory, and should
be saved to disk if future use is intended.  It is marked as
``modified'' so the user will be given the chance to save it when
exiting {\Xic}.

The {\cb !sqdump} function is similar, but writes a native file to
disk and does not create a cell in memory.

In electrical mode, note that the new cell is not a subcircuit.  It
must be edited and connection points added (with the {\cb subct}
command) before it can be used in another circuit.

The user is given the option to replace the selected objects with
an instance of the new cell.

By default, an attempt to overwrite a cell already in memory will
fail.  If the {\et CrCellOverwrite} variable is set, existing cells in
memory can be overwritten (use this with care). 


% -----------------------------------------------------------------------------
% xic:crvia 031815
\section{The {\cb Create Via} Button: Create Standard Via Variant}
\index{Create Via button}
\index{standard via}

The {\cb Create Via} button in the {\cb Edit Menu} brings up the {\cb
Via Creation} panel, if standard vias are defined in the current
technology.  If no standard vias are defined, the menu entry will be
grayed, and the panel will not be available.  This will be the case
for the example {\vt scmos} and Hypres technology files provided.  The
{\vt xic\_tech.demo} technology file found with the memory chip
examples does provide standard via definitions, should the user wish
to try this feature.

The panel contans a number of entry areas, corresponding to the
standard via parameters as described below.  Of these, the numerical
parameters can be changed by the user to create variants.  The fields
that contain layer names can not be changed, except by creating a new
standard via definition.  Presently, this must be done by editing the
technology file.

Each row of the panel contains a description and two entry areas, as
most of the entries have separate values for X and Y directions. 
Dimensions are in microns.

\begin{description}
\item{\cb Via name, cut layer}\\
This row contains two menus, which together provide access to all of
the available standard via definitions.  The menu on the right
provides the process layer names that are used as ``cuts''.  These are
the layers that represent holes in an insulating layer, generally
called ``via layers''.  The menu on the left provides the names of
standard vias defined which use that via layer.  Most often, there is
only one such definition, for a metal to metal contact.  In other
cases, one of the ``conductors'' may be an implanted area, in which
case there may be several choices.  When the user selects a standard
via using these menus, the other fields in the panel will be set to
the default values for the various parameters.

\item{\cb Layer 1, Layer 2}\\
These are the layer names of the two layers to be connected by the
via.  These can not be changed by the user, except by selecting
another standard via.  In {\Xic}, {\cb Layer 1}, and the `1'
designation in general, corresponds to the bottom conductor.

\item{\cb Cut width, height}\\
The ``cut'' is the feature on the via layer that actually forms the
contact.  In a standard via, this is always rectangular.  In a
semiconductor process, this is almost always a square of a fixed size. 
Although the two dimensions can be changed by the user, one must be
aware of the relevant design rules before doing so.

\item{\cb Cut rows, columns}\\
In order to lower contact resistance and handle higher current, the
cut structure can be arrayed.  In some situations it may be possible
to simply increase the size of a single cut, but in more advanced
processing the cut size is fixed and arrays are used.  This is
probably the most common variant.

\item{\cb Cut spacing X,Y}\\
This is the space between cut edges (not center-to-center) in the X
and Y directions.  This is only useful if the cut is arrayed.  In
general, the two values are the same, and fixed at a minimum from a
design rule.  The values should be changed only with knowledge of the
appropriate design rules.

\item{\cb Enclosure 1 X,Y}\\
In addition to the cut, the via will also contain squares of the two
metal layers.  The ``Enclosure'' is the distance the metal layer
overhangs the cut.  The two numbers apply to the bottom layer, in the
X and Y directions.  The two numbers are typically set to a design
rule minimum, and should be changed only with knowledge of the design
rules involved.

\item{\cb Offset 1 X,Y}\\
If the offset parameters are zero, the metal rectangle is centered on
the cut.  One can set the offset to a nonzero value, which will move
the center of the metal rectangle relative to the center of the cut. 
This entry applies to the bottom conductor.  These entries are almost
always zero.

\item{\cb Enclosure 2 X,Y}\\
As for {\cb Enclosure 1}, but the values apply to the top conductor.

\item{\cb Offset 2 X,Y}\\
As for {\cb Offset 1}, but the values apply to the top conductor.

\item{\cb Origin offset X,Y}\\
This is the origin of the sub-master coordinate system, which if zero
is centered on the cut array.  This is the same as Virtuoso, but
appears to differ from the OpenAccess specification which seems to
indicate that the origin is centered on the lower-left cut element. 
All features of the via are drawn relative to this offset, so there
are no design rule implications.  These values are most often zero.

\item{\cb Implant 1, Implant 2}\\
Up to two additional rectangles can be drawn in the via, representing
implant areas.  These may apply when contacting activated substrate
areas, where additional spacing rules to an implant region edge may
apply.  If defined in the standard via definition, one or both of
these entries may contain an implant layer name.  If not, the entry
area is grayed.  The layer names can not be changed by the user,
except by selecting another standard via.

\item{\cb Implant 1 enc X,Y}\\
If an {\cb Implant 1} layer is present, these entries will contain the
enclosure values of the implant 1 layer rectangle relative to the
bottom conductor rectangle.  That is, the implant 1 rectangle will
overhang the bottom conductor rectangle in the X and Y directions by
the values given.

\item{\cb Implant 2 enc X,Y}\\
If an {\cb Implant 2} layer is present, these entries will contain the
enclosure values of the implant 2 layer rectangle relative to the top
conductor rectangle.  That is, the implant 2 rectangle will overhang
the top conductor rectangle in the X and Y directions by the values
given.
\end{description}

The easiest way to understand the effect of these parameters is to
create some vias.

Entering the parameters has no effect until the {\cb Apply} button is
pressed.  When {\cb Apply} is pressed, a new internal sub-master cell
for the variant is created if necessary, and the via structure is
ghost-drawn and attached to the mouse pointer.  Instances of the via
will be placed where the user clicks in a drawing window.  As for
normal subcells, the current transform will be applied to the via. 
Most process rules will not accept 45-degree rotations.  The placement
mode can be exited by pressing the {\kb Esc} key.


% -----------------------------------------------------------------------------
% xic:flatn 102913
\section{The {\cb Flatten} Button: Flatten Hierarchy}
\index{Flatten button}
\index{flatten hierarchy}
The {\cb Flatten} button in the {\cb Edit Menu} brings up a small {\cb
Flatten Hierarchy} pop-up which controls flattening of the hierarchy
by moving the contents of selected subcells into the current cell.  A
{\cb Depth} choice menu selects the depth into the hierarchy to
flatten.  If 0, geometry in the selected subcells is brought into the
current cell, and sub-subcells are placed in the current cell,
becoming subcells.  If ``{\vt all}'', the entire subcell hierarchy is
flattened, i.e., all geometry under a selected subcell is brought into
the current cell.

The {\cb Use fast mode} check box will select a processing mode that
will skip undo list processing and object merging operations for speed
and reduced memory use.  This may be desirable for large jobs
containing complex cells, which may take a long time to process.  In
this mode, there is no ``undo'' capability, however.

If the {\cb Use object merging when flattening} check box is checked,
the new object merging will be performed when objects from the
subcells are promoted to the current cell.  This is the same merging
as specified in the {\cb Editing Setup} panel from the {\cb Edit
Menu}.  Use of full polygon merging can greatly increase processing
time, simple box clipping/merging has much lower overhead.  Merging
will generally reduce the object count in the layout.

The {\cb Flatten} button on the pop-up initiates the operation.  The
subcells to be flattened must have been selected at this point.

Pressing {\kb Ctrl-c} will pause the process, and give the user the
option of terminating the job.  It is usually not desirable to stop in
the middle of a flatten operation, but invoking this prompt may
reassure the user that the operation is in progress and not ``hung''.

In electrical mode, symbolic instances and library devices are never
flattened, they are considered atomic.  If you must flatten an
instance that is displayed symbolically, the instance must first be
forced to display as a schematic, either by reverting its master to
non-symbolic temporarily, or by adding a {\et NoSymb} property to the
instance with the {\cb Property Editor}.


% -----------------------------------------------------------------------------
% xic:join 070616
\section{The {\cb Join/Split} Button: Join or Split Objects}
\index{Join/Split button}

The {\cb Join/Split} button in the {\cb Edit Menu} brings up the {\cb
Join or Split Objects} panel.  This panel contains controls for
setting defaults and initiating join and split operations.  These
operations are identical to those available from the {\cb !join} and
{\cb !split} text commands.

The panel contains the following controls:
\begin{description}
\item{\cb No limits in join operation}\\
This check box unsets the limits on the complexity of polygons that
are created during the merge, by setting the {\et JoinMaxPolyVerts},
{\et JoinMaxPolyGroup}, and {\et JoinMaxPolyQueue} variables to
``0'' (zero).

\item{\cb Maximum vertices in joined polygon}\\
This provides an entry area for setting the value of the {\et
JoinMaxPolyVerts} variable, which limits the number of vertices
allowed in a polygon created as the result of a join operation.

\item{\cb Maximum trapezoids per poly for join}\\
This provides an entry area for setting the value of the {\et
JoinMaxPolyGroup} variable, which places a limit on the number of
connected trapezoids that can be used to form a polygon.

\item{\cb Trapezoid queue size for join}\\
This provides an entry area for setting the value of the {\et
JoinMaxPolyQueue} variable, which provides a limit on the number of
trapezoids that can be considered for joining into polygons in a
single pass.

\item{\cb Clean break in join operation limiting}\\
When this check box is set, {\Xic} will attempt to break polygons
where the vertex limit is reached into pieces so that the boundaries
are more visually attractive.  This tracks the state of the {\et
JoinBreakClean} variable.

\item{\cb Include wires (as polygons) in join/split}\\
If this check box is set, wire objects will be included in join/split
operations, treated as polygons.  If not checked, wires are ignored in
these operations.  This tracks the state of the {\et JoinSplitWires}
variable.

\item{\cb Join}\\
This push button initiates a join operation on selected objects.  All
suitable selected objects will be joined on their respective layers,
as for the {\cb !join} command without an argument.  Unlike other
commands that join (merge) objects, this overrides the {\vt NoMerge}
technology attribute if set on an object's layer.

\item{\cb Join Lyr}\\
This will join objects on the current layer, whether selected or not. 
The layer must be visible, and not have the {\vt NoMerge} technology
attribute set.  This is equivalent to the {\cb !join} command with the
``layer'' argument.

\item{\cb Join All}\\
\index{NoMerge keyword}
This push button initiates a join operation on all objects in the
current cell, selected or not.  It applies to objects on visible,
selectable layers that do not have the {\vt NoMerge} technology
attribute applied.  This is the same as the {\cb !join} command with
the ``all'' argument given.

\item{\cb Split Horiz}\\
This will decompose complex polygons into a collection of trapezoids
(boxes and simple polygons in the database) that collectively cover
the same area and do not overlap.  The splitting is performed using
horizontal scan lines.  This is the same effect as the {\cb !split}
command.  Wire objects will also be split if the {\et JoinSplitWires}
variable or the corresponding check box is set.

\item{\cb Split Vert}\\
This will also decompose complex polygons into a collection of
trapezoids, however vertical scan lines are used.  This is the same
effect as using the ``{\vt v}'' argument to the {\cb !split} command. 
Wire objects will also be split if the {\et JoinSplitWires} variable
or the corresponding check box is set.
\end{description}


% -----------------------------------------------------------------------------
% xic:lexpr 100516
\section{The {\cb Layer Expression} Button: Evaluate Layer Expression}
\index{Layer Expression button}
The {\cb Layer Expression} button in the {\cb Edit Menu} brings up the
{\cb Evaluate Layer Expression} panel.  Layer expressions are logical
expressions referencing the geometry on existing layers, with various
operators and functions.  These evaluate to a pattern, which can be
applied to a new or existing layer.  Operations include polarity
inversion, intersection, union, and many more complex possibilities.

The {\cb Evaluate Layer Expression} panel allows layer expressions to
be applied to the current cell hierarchy, much the same as the
text-mode {\cb !layer} command.  The panel allows easy setting of
variables which control the expression evaluation, whether initiated
from the panel or the {\cb !layer} command.

Full layer expression evaluation is available in physical mode only,
though joining, splitting and copying are available in either mode.

The controls found in the {\cb Evaluate Layer Expression} panel are
described below.

\begin{description}
\item{\cb To layer}\\
This entry area requires the name of a layer on which new geometry
will be created while the layer expression is evaluated.  This is the
only control in the panel that requires an entry.  If the layer name
does not match an existing layer name (as a short or long name), a new
layer is created, with a name generated from the given name in the
same way as in the technology file layer definitions.

If no expression is given, the {\cb To layer} is created, if it does
not exist.  If the layer exists, and one of {\cb Joined}, {\cb Horiz
Split}, or {\cb Vert Split} is set, that operation will be performed
on the {\cb To layer}.  The result is similar to the corresponding
operations as initiated form the {\cb Join or Split Objects} panel
from the {\cb Join/Split} button in the {\cb Edit Menu}, or the {\cb !join}
and {\cb !split} commands.  If the {\cb To layer} did not previously
exist, or the {\cb Default} new object format is selected, layer
creation is the only operation performed.

\item{\cb Depth to process}\\
When the layer expression is evaluated, the layer geometry used in the
processing is obtained to this level in the hierarchy.  If 0, only the
geometry in the present cell is considered.  If ``{\vt all}'', the
geometry of the complete hierarchy is taken.

\item{\cb Recursively create in subcells}\\
This check box has effect only if the {\cb Depth} is not zero.  When
checked, the layer expression is evaluated in the current cell and
each subcell to depth, using only the objects from that cell.  If this
is unchecked, the operation is quite different.  In this case, there
is no recursion, and all new geometry is created in the current cell,
but geometry in cells to depth is considered when creating the new
geometry.

\item{\cb Partition Size}\\
To maximize computation speed, the expression evaluation is performed
step-wise over a logical grid in the target cell.  The grid origin is
the lower-left corner of the cell.  The partition size is the width of
an (assumed square) grid cell.  The calculations are performed for
each grid square that overlaps the cell area.  This can be more
efficient than calculating the whole cell in one shot (which might not
even be possible due to memory limitations).

The gridding is used only if an actual expression is given, and not
simply a layer name (or no expression at all).  If the expression
consists only of a layer name, processing requires only a simple copy
and there would be no reason to use partitioning.

If the {\cb None} button is pressed, no partitioning will be used.

The default partition size is 100 microns, which can be adjusted for
best performance.  The size should be large enough to minimize the
number of grid cells to evaluate, but small enough to limit the amount
of geometry to process on average in each grid, to avoid huge memory
consumption and other ill effects of taking too big of a ``bite''.

For simple cells, the grid size can be large, or partitioning can be
skipped entirely.  Partitioning can be skipped by pressing the {\cb
None} button, or by setting the size to a value larger than the cell
bounding box width and height.

This entry tracks the state of the {\cb Partition Size} variable,
which is also used by the {\cb !layer} command and elsewhere.

\item{\cb Number of helper threads}\\
{\bf PRELIMINARY. EXPERIMENTAL!}

Multiple threads can be used when evaluating a layer expression over a
grid.  Evaluation in each of the grid cells can be done in parallel,
so these jobs are submitted to the thread pool.  This allows processor
cores to work simultaneously on different parts of the grid.
  
Multi-threading will be used if this entry is set nonzero.  The value
is the number of helper threads that can be called upon to parallelize
the operation.  The speediest value is probably one less than twice
the number of available processor cores, as each Intel core provides
two hardware threads.  Your results may differ, so one should
experiment.  One can also experiment with the partition size to get
fastest results, larger partitions are more likely to overcome the
multi-threading overhead.

This should not be set to a value larger than the number of available
hardware threads minus one, but one might wish to try smaller values. 
If set to a larger value, software threads will be used, which will
increase computation time.  If set to 0, the operation is
single-threaded.

This entry tracks the value of the {\et Threads} variable.

\item{\cb Don't clear layer before evaluation}\\
By default, the layer given in the {\cb To layer} entry is cleared
before the expression is evaluated, so that the layer will contain
only the result of the operation.  It this check box is set, the {\cb
To layer} will not be cleared, new data will appear in addition to
existing data.

\item{\cb New object format}\\
This consists of four interlocking ``radio'' buttons which establish
the nature of the new objects created by evaluating the layer
expression.  If {\cb Joined} is selected, objects will be combined
into polygons before being added to the cell.  If {\cb Horiz Split} is
selected, objects are added as trapezoids, with a horizontal
orientation (maximal width) favored.  If {\cb Vert Split} is selected,
objects will also be added as trapezoids, however a vertical
orientation (maximal height) is favored.

The {\cb Default} choice has the same effect as {\cb Joined} in cases
where the layer expression contains more than a layer name, i.e., it
contains at least one operator, function, or numeric entry.  If the
expression consists of a layer name only, the {\cb Default} choice
will read the objects from that layer and add them to the {\cb To
layer}, without modification.  The other new object format choices
will cause the objects read from the layer to be joined or split
before being added to the {\cb To layer}.

When joining objects, there are several variables which fine-tune the
operation.  These are most conveniently set from the {\cb Join or
Split Objects} panel brought up by the {\cb Join/Split} button in the
{\cb Edit Menu}.

\item{\cb Expression}\\
This entry area contains the layer expression to evaluate.  This is an
expression consisting of existing layer names, operators, and function
calls, which will be evaluated.  Dark areas will be rendered on the
layer given in the {\cb To layer} entry.

Thus, this provides a means of creating a new layer from geometry on
existing layers.  Labels are ignored during processing, but all other
objects contribute.  The same layer name can appear in the {\cb To
layer} entry and in the expression, in which case the contents of that
layer is updated with the result of the expression.

There are eight registers which can be used to save and recall layer
expression strings, for convenience.  The {\cb Save} and {\cb Recall}
buttons provide access to these registers.  Selecting an item in the
{\cb Save} menu will save the current contents of the {\cb Expression}
entry in that register.  Selecting an item in the {\cb Recall} menu
will load that text into the {\cb Expression} entry area.

\item{\cb Use object merging while processing}\\
When this check box is set, new objects created during evaluation of
the layer expression will be merged with existing objects, using the
same object merging as specified in the {\cb Editing Setup} panel from
the {\cb Edit Menu}.

If there is no {\cb Expression} given, or the expression consists only
of the same layer name given in {\cb To layer}, then merging is not
performed.

In every other case, the merging enabled from the {\cb Edit Menu} will
be performed as new objects are added to the {\cb To Layer}.  This
merging will defeat the purpose of the join and split format choices,
so one must consider when merging makes sense.  Merging applies to
objects initially on the {\cb To layer}, if not clearing, plus the
accumulated objects added as the operation progresses.

Full polygon merging can greatly increase the time and memory required
to process a large job.  Box clipping has much less overhead.

\item{\cb Fast mode}\\
When set, undo list processing and object merging will be skipped,
which reduces memory use and computational overhead to a minimum. 
However, the operation can not be undone, so this mode should be used
with care.

\item{\cb Evaluate}\\
Pressing this button will create the {\cb To layer} if necessary,
evaluate the layer expression, and add the newly created geometry to
the current hierarchy.
\end{description}

\subsection{Examples}

\begin{description}
\item{Clear layer {\vt M0}}\\
{\cb To layer}: {\vt M0}\\
{\cb Expression}: {\vt 0}
\item{Copy layer {\vt M1} to layer {\vt NEW}}\\
{\cb To layer}: {\vt NEW}\\
{\cb Expression}: {\vt M1}
\item{Copy the inverse of layer {\vt M1} to {\vt NEW}}\\
{\cb To layer}: {\vt NEW}\\
{\cb Expression}: {\vt !M1}
\item{Copy the intersection area of {\vt I1} and {\vt I2} to {\vt NEW}}\\
{\cb To layer}: {\vt NEW}\\
{\cb Expression}: {\vt I1\&I2}
\item{Copy the {\vt R1} and {\vt R2} areas to {\vt New}}\\
{\cb To layer}: {\vt NEW}\\
{\cb Expression}: {\vt R1|R2}
\end{description}

\subsection{Extended Layer Names}

The layer names in the layer expression (but not the {\cb To layer}
entry) can actually be given in an extended form:

\begin{quote}
{\it lname\/}[{\vt .}{\it stname\/}][{\vt .}{\it cellname\/}]
\end{quote}

Most generally, the ``layer'' name consists of three tokens, two of
which are optional (indicated by square brackets above).  The tokens
are separated by a period (`{\vt .}') character.  The individual
tokens can be double-quoted (i.e., using the double-quote (`{\vt "}')
character), which must be used if the tokens contain non-alphanumeric
characters.  The period separators must appear outside the scope of
any quoting.

\begin{description}
\item{\it lname}\\
This is a short or long layer name, as found in the layer table.
\item{\it stname}\\
The name of a symbol table which contains the {\it cellname\/}.
\item{\it cellname}\\
The name of a cell.
\end{description}

If only one separator appears, the token that follows is taken as the
{\it cellname}, and the current symbol table is assumed.

The {\it cellname} is the name of a cell used as the source for
geometry.  If no {\it cellname} is given, the name of the current cell
is understood.  The odd case of an empty {\it stname} indicates the
``{\vt main}'' symbol table, e.g., {\vt layer..cell} is equivalent to
{\vt layer.main.cell}.

If the {\it cellname} starts with the `{\vt \@}' character, and no
symbol table name is given, then the rest of the {\it cellname} is
taken as the name of a ``special'' database, as created with script
functions like {\vt ChdOpenZdb}.  If found, geometry will be obtained
from the database rather than a cell.  Otherwise, when a {\it
cellname} is given, the geometry is obtained from the given cell, as
if it were overlaid on the current cell.  The {\it cellname} (or any
of the three tokens) can be double quoted, and must be quoted if the
name contains a `{\vt .}' character, for example {\vt
CPG."mycell.xic"}.

If a {\it stname} is given, and the name matches an existing symbol
table name, the cell is obtained from that symbol table.  If the
symbol table name is given, the {\it cellname} field must appear, but
can be empty (a trailing period) which indicates the name of the
current cell.

If the {\it stname} is given, and the cell is not in this table, it
will be opened from disk into the given table (not the current table)
if found as a native cell file in the search path.

The coordinate origin of the source cell is taken as the origin of the
current cell.  The source cell must be in memory, or be in a native
cell in the search path.

Objects read from a ``special'' database are clipped to the boundary
of the cell being added to.  No such clipping is done when objects are
read from another cell.

\subsection{Advanced Examples}

Suppose one has two versions of a cell, {\vt cell} and {\vt cell\_old},
and one needs to know if they differ on layer {\vt M1}.  Open a dummy
cell for editing, then supply the following and evaluate.
\begin{quote}
{\cb To layer}: {\vt ZZ}\\
{\cb Expression}: {\vt M1.cell\symbol{94}M1.cell\_old}
\end{quote}
Press the {\kb Home} key to view the entire cell space.  Any geometry
shown on the new dummy layer {\vt ZZ} is the exclusive-OR of the
geometry on {\vt M1} of the two cells, i.e., the difference.  If there
is no geometry on {\vt ZZ}, {\vt M1} is the same in {\vt cell} and
{\vt cell\_old}.

As a variation, suppose that the user has done the following:
\begin{quote}
{\it Set symbol table to} "{\vt old}''.\\
{\it open} {\vt oldstuff/mycell}\\
{\it return to previous symbol table}\\
{\it open} {\vt newstuff/mycell}
\end{quote}

There are now two versions of {\vt mycell} in memory.  To compare the
layer {\vt M1} in the two cells, one could then evaluate
\begin{quote}
{\cb To layer}: {\vt ZZ}\\
{\cb Expression}: {\vt M1\symbol{94}M1.old.}
\end{quote}
Then the {\vt ZZ} layer, which consists of the exclusive-OR of old and
new {\vt M1} in {\vt mycell}, would be added to the current {\vt
mycell}.  Pressing the {\kb Tab} key undoes the addition.

Suppose one wants to import the inverse of the geometry on layer {\vt
VIA} from {\vt cell} into the current cell, also on layer {\vt VIA}:
\begin{quote}
{\cb To layer}: {\vt VIA}\\
{\cb Expression}: {\vt !VIA.cell}
\end{quote}
The {\vt VIA} layer now consists of the inverse from {\vt cell}.  Any
geometry that existed on {\vt VIA} in the current cell before the
command was given is deleted (assuming that the {\cb Don't clear}
check box is unchecked).  The bounding box of the current cell may
have been expanded to include the bounding box of {\vt cell}.  The
area used to create an inversion is the rectangle bounding all cells
referenced in the expression, plus the current cell.

Suppose one simply wants to copy the geometry from layer {\vt M2} of
{\vt cell} into the current cell:
\begin{quote}
{\cb To layer}: {\vt M2}\\
{\cb Expression}: {\vt M2.cell}
\end{quote}
The {\vt M2} layer now consists of the geometry on {\vt M2} from {\vt
cell}.  The bounding box of the current cell may have been expanded,
in which case some of the {\vt M2} features may be off-screen (press
the {\kb Home} key to view the entire cell).  Any objects previously
existing on {\vt M2} in the current cell are deleted before the
operation, unless the {\cb Don't clear} check box is checked.


% -----------------------------------------------------------------------------
% xic:prpty 030415
\section{The {\cb Properties} Button: Property Editor Panel}
\label{prptybut}
\index{Properties button}
\index{properties}
\index{Property Editor window}
The {\cb Properties} button in the {\cb Edit Menu} brings up the {\cb
Property Editor} containing commands for adding and modifying
properties of objects.  For the most part {\Xic} does not use
properties in physical layouts, but they provide important electrical
information in schematic layouts, which is required when building a
netlist or SPICE deck.

Clicking on a selected non-pcell instance with button 1 and the {\kb
Ctrl} key held will also bring up the {\cb Property Editor}, if it is
not already present.

When the {\cb Property Editor} first appears, or upon pressing the
{\cb Activate} button in the panel, or if the {\cb Properties} menu
button is pressed with the {\cb Property Editor} already visible but
inactive, a command state begins where it is possible to list and edit
the properties of selected objects.  The command state is terminated
by pressing the {\cb Activate} button again, or pressing the {\cb
Properties} button in the {\cb Edit Menu}, or pressing the {\kb Esc}
key, or by starting a different command.  The {\cb Property Editor}
remains visible, but will go to an inactive state.  The {\cb Dismiss}
button in the {\cb Property Editor} will exit the command state if
active, and retire the panel.

Unless stated otherwise, the descriptions of operations below apply
only when the command state is active.  When inactive, the presence of
the {\cb Property Editor} window has no effect, and other commands can
be executed normally.

When the command mode becomes active, properties of one of the
selected objects (if any) are shown in the text window of the panel. 
The objects are not generally shown as selected, but an internal list
of objects that were selected before the command mode was started, or
were clicked on with the command state active, is maintained.  The
object for which the properties are displayed is marked with a dotted
outline around the object or a cross over the object.  Clicking on the
marked object will delete that object from the internal list, and
another object's properties (if any in the list) will be shown. 
Clicking on an unmarked object will mark that object, add it to the
list if it is not already there, and display its properties.

The {\cb desel} button in the top button menu and other methods of
deselection will clear the list of objects.

If the {\cb Global} button in the panel is active, all objects in the
list are shown as selected (blinking outline or symbol).  The {\cb
Global} button allows manipulation of the properties of all objects in
the list, not just the marked object.

When more than one device is in the list, the arrow keys can be used
to cycle the marked object through the list.

\index{Property Info window}
When the {\cb Info} button is active, clicking on an object will bring
up or update the {\cb Property Info} window, loaded with the
properties of the object.  This contains a listing identical to the
{\cb Property Editor}, however there are no buttons other than {\cb
Dismiss}.  The object whose properties are listed in the {\cb Property
Info} window is marked on-screen similarly to the current object in
the {\cb Property Editor}, but with a different color.

When the {\cb Property Editor} is active, clicking on an object with
the {\kb Shift} or {\kb Control} key pressed will also bring up or
update the {\cb Property Info} window, whether or not the {\cb Info}
button is active.

The {\cb Property Editor} and {\cb Property Info} windows are
drag/drop sources and receivers, meaning that one can drag properties
from one window to another.  This will apply the dragged property to
the object associated with the drop window (the source object is not
affected).  Properties that must be unique, such as most electrical
properties, will be replaced with the dropped property.  Properties
that are not unique will be added, without replacement.  Only
ordinary, user-modifiable properties can be copied in this manner. 
The prompt line, while in editing mode, is also a drop receiver for
these windows.

The listing in either window shows the property number, a descriptive
name in electrical mode, and the property string, for all properties
attached to the current object.  A property can be selected in the
list by clicking on the text --- it will be shown highlighted when
selected.  The current selection is used as input by many of the
command buttons in the panel.

In the properties listing, color is used to distinguish the types of
properties.  The colors can be modified by setting the Special GUI
Colors (see \ref{attrcolor}) listed below.  This can be done in the
technology file, or with the {\cb !setcolor} command.

\begin{tabular}{|l|l|l|} \hline
\bf variable & \bf default & \bf purpose\\ \hline
 & black & internal properties\\ \hline
\vt GUIcolorHl1 & red & user-set {\et name} property\\ \hline
\vt GUIcolorHl2 & dark blue & physical mode pseudo-properties\\ \hline
\vt GUIcolorHl4 & sienna & ordinary (user-modifiable) properties\\ \hline
\end{tabular}

The value of the electrical mode {\et name} property is shown in a
different color when the property is set.  This property always
exists, and it would not otherwise be obvious when viewing the listing
when the {\et name} property has been set by the user, or is simply
showing the name assigned by {\Xic}.

The command buttons in the {\cb Property Editor} allow addition,
modification, and deletion of properties both globally (on all
selected objects) or on the marked object.  Those properties in the
list marked as ``internal'' can not be modified.  The physical mode
pseudo-properties can not be edited, but can be added (with the {\cb
Add} button).  In this case, no property is added, but the operation
will cause some aspect of the object to change.

\index{editing properties}
\index{properties panel!Edit}
\subsection{The {\cb Edit} Button: Edit Property}
The {\cb Edit} button allows editing of the current property.  If no
property is selected in the text when the {\cb Edit} button is
pressed, the first user-modifiable property listed will become
selected, and the text of that property will appear on the prompt line
for editing.  If a user-modifiable property was selected before the
{\cb Edit} button was pressed, the text of that property will appear
on the prompt line.  The up/down arrow keys will cycle through the
editable properties listed in the window, selecting and placing the
text on the prompt line in sequence.  Also, clicking on an entry for a
modifiable property in the window will select it and load its text
into the prompt line. 

The text in the prompt line can be edited, and pressing the {\kb
Enter} key completes the edit.  The property listing will show the
changes, if any.  While editing, text from other windows can be
inserted using drag/drop (from the property windows or the {\cb File
Selection} pop-up only) or with the window system cut/paste method.

When inserting text from property windows, hypertext references (see
\ref{hypertext}) are preserved.  Hypertext entries can also be
inserted in electrical mode by clicking on a device contact point or
wire (node reference), on the `$+$' symbol of some devices (branch
reference), or elsewhere on a device (name reference).  Pressing {\kb
Shift} or {\kb Control} while clicking on a device or subcircuit will
bring up the {\cb Property Info} window, whether or not editing is
active.

For physical properties and {\et value}, {\et param}, and {\et other}
electrical properties, the ``long text'' feature (see \ref{longtext})
is available.  This is indicated by the presence of a small ``{\cb
L}'' button to the left of the prompt line, which appears when the
prompt line cursor is in the first column.  If this button is pressed,
or {\kb Ctrl-t} typed, a text editor window appears, loaded with the
text of the property (if any).  When a property is in long text
format, the display listings will show only ``{\vt [text]}'' as the
content, and the prompt line will show the same string as a hypertext
entry.  In this case, just pressing {\kb Enter} will bring up the text
editor leaded with the ``real'' property text.  This feature allows
long, multi-line text blocks to be associated with properties.

The description thus far applies whether or not the {\cb Global}
button is set.  With the {\cb Global} button not set, when the {\kb
Enter} key is pressed to complete the editing, the property will be
updated, and the text in the {\cb Property Editor} will display the
change.  The operation, as with all operations described in this
section, can be undone or redone with the {\cb Undo}/{\cb Redo}
commands or {\kb Tab}/{\kb Shift Tab} keys.

If the {\cb Global} button is set, the user will be prompted, in
sequence, for a new string for each of the devices in the internal
list.  After the first prompt, the arrow keys and click-selection are
disabled.  Each device will be assigned a new property or a matching
existing property will be replaced.  For properties that can have more
than one instance ({\et other} electrical properties and all physical
properties) if the number and string of the original property shown
highlighted in the {\cb Property Editor} window match those of an
existing property, that property will be replaced, otherwise a new
property will be added.

\subsection{The {\cb Add} Button: Add New property}
\index{properties!addition of}
In physical mode, the {\cb Add} button will produce a drop-down menu
containing the following items.

\begin{description}
\index{nomerge property}
\item{\cb nomerge}\\
The {\cb nomerge} choice will add a {\et nomerge} property (a property
used by the extraction system) to the selected object or objects.

\item{\cb flatten}\\
The {\et flatten} property applies to electrical and physical cells
and instances.  It is used during association and LVS to determine if
the contents of the instance master should be logically promoted into
the containing cell (see \ref{exthier}).  Although {\Xic} can handle
most hierarchy differences automatically and transparently, this
property may be used when needed to force proper behavior.

If a {\et flatten} property has been applied to the master cell, then
instances of the cell will be flattened, unless the instance also has
the {\et flatten} property applied, in which case the instance will
not be flattened.  If the master does not contain a {\et flatten}
property, then and instance will be flattened only if the instance has
a {\et flatten} property applied.  Thus, the {\et flatten} property of
an instance reverses the effect of a {\et flatten} property applied to
the master.

The {\et FlattenPrefix} variable, and equivalently the {\cb Cell
flattening name keys} entry area in the {\cb Net and Cell Config} page
of the {\cb Extraction Setup} panel from the {\cb Extract Menu},
provide another means of causing instances of cells to be flattened.

\item{\cb any}\\
The {\cb any} choice allows an arbitrary property to be added.  This will
initiate prompting for a property number and string to add.
\end{description}

In electrical mode, the {\cb Add} button brings up a menu of property
types that can be added.  Selecting an entry will initiate prompting
for the associated string.  Any selection in the listing will be
ignored.  Unlike the case of the {\cb Edit} button, the arrow keys and
subsequent selection in the listing will not affect the prompt line.

With the {\cb Global} button off, completion of editing by pressing
{\kb Enter} will ``add'' the new property to the current object.  In
electrical mode, properties other than the {\et other} property will
be replaced if they exist, since there can be at most one such
property.  There can be arbitrarily many {\et other} properties, or
properties of any number in physical mode.  Such properties are always
added and not replaced.

If the {\cb Global} button is active, an identical copy of the
property will be added to each of the devices in the internal list. 
This will be a replacement for electrical properties other than {\et
other}, and an addition otherwise.  Unlike the {\cb Edit} button case,
there is no individual prompting for a string for each device.  The
initial string (and number, in the case of physical mode) is added to
each object.

In electrical mode, the {\cb Add} menu contains buttons for the
modifiable device and subcircuit instance properties listed below. 
Unless stated otherwise, there can be at most one each of the
properties described below.  This is enforced by {\Xic}, i.e.,
attempts to add a second property of a given type will cause
replacement, not addition.

\begin{description}
\index{name property}
\item{\cb name}\\
The {\cb name} button allows the modification of a {\et name}
property.  The {\et name} property specifies the device or subcircuit
instance name to SPICE.  Unlike the other user-settable properties,
the {\et name} property always exists.  If not explicitly set by the
user, the device name will be generated internally.  However, if a
correspondence to an existing SPICE file is necessary, the name must
be specified.  {\Xic} allows any name, however for the device to be
recognized by SPICE, the name must start with the device's key letter
as expected by SPICE.  Deleting the {\et name} property simply reverts
back to the internally generated name.

If an assigned name property conflicts with an internally generated
name, the internally generated name will be updated so as to not
conflict by appending ``\_{\it N\/}'', where {\it N} is some integer.

\index{model property}
\index{value property}
\item{\cb model\newline\cb value}\\
The {\cb model} and {\cb value} buttons allow addition of a {\et
model} or {\et value} property, respectively.  Only one of the {\et
model} or {\et value} properties should be used per device, as this
really represents two different names for the same text field in SPICE
output.  One has the choice a supplying a device model or component
value to the device, but not both.  These properties generally apply
to devices only, not subcircuits.

\index{param property}
\item{\cb param}\\
The {\et param} property is a catch-all for additional parameters
found in the device and subcircuit instance lines in SPICE, such as
initial conditions or device geometrical factors.  The string will
generally contain a list of {\it name\/}{\vt =}{\it value} terms, each
separated by white space.  Only one {\et param} property is allowed.

\index{devref property}
\item{\cb devref}\\
The {\et devref} property provides the name of the controlling device
to current-controlled sources and the current-controlled switch.  At
most one {\et devref} property is allowed.

\index{other property}
\item{\cb other}\\
The {\cb other} button allows addition of an {\et other} property. 
These properties have no significance to {\Xic} and are not used in
SPICE output.  They can be used to store alternate values for the {\et
model}, {\et value}, or {\et param} properties, or to store any other
information desired by the user.  There can be arbitrarily many
{\et other} properties per device or subcircuit instance.

\index{nophys property}
\item{\cb nophys}\\
The {\cb nophys} button allows addition of a {\et nophys} property. 
This property does not affect SPICE output, but specifies that the
device or subcircuit instance has no physical implementation.  When
{\Xic} is associating physical and electrical objects for extraction
and LVS, a physical implementation will not be sought for objects with
this property.

When the property is created, the user is prompted as to whether the
device terminals should be shorted together during LVS.  Devices that
have the {\et nophys} property applied will be rendered using a
different color than ``normal'' devices.  See the description of LVS
in \ref{lvs} for a more complete discussion of the use of this
property.

\index{flatten property}
\item{\cb flatten}\\
This will add a {\et flatten} property, which applies to electrical
and physical cells and instances.  See the description in the listing
of physical properties above.

\index{nosymb property}
\item{\cb nosymb}\\
The {\cb nosymb} button is used to add a property to electrical
subcircuit instances which forces them to be displayed as expanded,
whether of not the master cell of the instance is symbolic. 
Instances with this property will behave in all respects as if the
master were non-symbolic.  Thus, instances of the same master can be
displayed symbolically or not, in the same design.  This property
uses the same property number as the {\et symbolic} property applied
to cells.

\index{range property}
\item{\cb range}\\
The {\cb range} button is used to add a property to electrical device
or subcell instances (other than terminal devices) that {\it
vectorizes} the instance.  The user is prompted for two non-negative
numbers which define the subscripting range.  Vectorized instances and
connection rules are discussed in \ref{vecinst}.
\end{description}

\subsection{The {\cb Delete} Button: Delete Property}
\index{property deletion}
\index{properties panel!Delete}
If a modifiable property is selected in the list, pressing the {\cb
Delete} button will delete the property.  If there is no selection,
the user will be prompted.  In physical mode, the user is requested to
provide the number for the property or properties to delete.  In
electrical mode, the user is requested to provide a code consisting of
any combination of the letters {\vt n}, {\vt m}, {\vt v}, {\vt p},
{\vt o}, {\vt y} to specify the properties to delete.  If the response
is ``{\vt vp}'', for example, the {\et value} and {\et param}
properties will be deleted.  In the case of physical properties, all
of the properties with the given number will be deleted (there can be
more than one).  Similarly, in electrical mode, if ``{\vt o}'' is
given, all {\et other} properties will be deleted.

If the {\cb Global} button is not active, the properties are deleted
from the current object.  If the {\cb Global} button is active,
properties will be removed from all objects in the internal list.  If
a property was selected in the listing before the {\cb Delete} button
was pressed, and this is a physical property or {\et other} electrical
property, only properties that match both the number and string
(physical mode), or {\et other} properties that match the string
(electrical mode) of the selected property will be deleted.  If no
selection is given, all properties that match the specification given
will be deleted.


% -----------------------------------------------------------------------------
% xic:cprop 012014
\section{The {\cb Cell Properties} Button: Edit Cell properties}
\index{Cell Properties button}
The {\cb Cell Properties} button in the {\cb Edit Menu} brings up the
{\cb Cell Property Editor}, which is used to view and manipulate
properties of the current cell.  It is a simplified version of the
{\cb Property Editor} which is used to manipulate the properties of
objects contained within the current cell.

The {\cb Cell Property Editor} contains buttons to add, edit, and
remove cell properties.  In general, cell properties are assigned
internally and can not be modified.  The exceptions are the
properties listed in the {\cb Add} menu and further discussed below. 
Pressing the {\cb Add} button brings up a pop-up menu containing
entries corresponding to properties that can be set or modified by
the user.  Only the properties that are applicable to the current
mode (physical or electrical) are active.

In physical mode, the entries listed below are available, allowing
modification of physical cell properties (see \ref{physpropfmt}).

\begin{description}
\item{\cb any}\\
The {\cb any} entry allows an arbitrary property to be assigned to
the cell.  The user will be prompted for a number and string for the
property.  These are arbitrary, however there are certain numbers
that are reserved by {\Xic} and will not be accepted.  {\Xic} will
not use these properties, but they may be important for interfacing
to third-party applications.

\index{flags property}
\item{\cb flags}\\
The {\cb flags} entry is used to set flags in the cell, notably the
{\et OPAQUE} flag which causes the cell contents to be ignored during
extraction.

\index{FlattenPrefix variable}
\index{flatten property}
\item{\cb flatten}\\
The {\et flatten} property applies to electrical and physical cells
and instances.  It is used during association and LVS to determine if
the contents of the instance master should be logically promoted into
the containing cell (see \ref{exthier}).  Although {\Xic} can handle
most hierarchy differences automatically and transparently, this
property may be used when needed to force proper behavior.

If a {\et flatten} property has been applied to the master cell, then
instances of the cell will be flattened, unless the instance also has
the {\cb flatten} property applied, in which case the instance will
not be flattened.  If the master does not contain a {\et flatten}
property, then and instance will be flattened only if the instance has
a {\et flatten} property applied.  Thus, the {\et flatten} property of
an instance reverses the effect of a {\et flatten} property applied to
the master.

The {\et FlattenPrefix} variable, and equivalently the {\cb Cell
flattening name keys} entry area in the {\cb Net and Cell Config} page
of the {\cb Extraction Setup} panel from the {\cb Extract Menu},
provide another means of causing instances of cells to be flattened.

\index{pc\_params property}
\item{\cb pc\_params}\\
The {\cb pc\_params} entry is used when defining parameterized cells
(pcells, see \ref{pcells}).  It is used to set or modify the parameter
list associated with the pcell.

\index{pc\_script property}
\item{\cb script}\\
The {\cb pc\_script} entry is used when defining parameterized cells. 
It is used to set or modify the script which implements the pcell
features.
\end{description}

In electrical mode, the following properties can be set from the
{\cb Add} menu.

\begin{description}
\index{param property}
\item{\et param}\\
Selecting the {\cb param} button allows a {\et param} property to be
added to the cell.  The {\et param} property provides support for the
subcircuit parameterization feature of {\WRspice} (see the description
of the {\vt .subckt} line).  The use of parameterization is briefly
described in \ref{spiceline}.

\index{other property}
\item{\et other}\\
Selecting {\cb other} allows an {\et other} property to be added to
the cell.  These have no meaning to {\Xic}, but might be of use to the
user.  Any number of {\et other} properties can be added.

\index{virtual property}
\item{\et virtual}\\
Adding a {\et virtual} property will prevent the cell from being
included in netlist output, most importantly SPICE output.  The cell
becomes a ``placeholder'', and the actual {\vt .subckt} text, which is
required to satisfy references, is included in the SPICE file by
another means.  For example, the cell might represent an opamp, and a
{\vt .include} line can be used to bring in the {\vt .subckt} block
representing the opamp, from a vendor model file.

\index{flatten property}
\item{\et flatten}\\
See the description of the {\et flatten} property in the physical {\cb
Add} menu properties list above.  The property has the same use in
electrical mode.

\index{macro property}
\item{\et macro}\\
This property can be applied to device masters.  The only effect is
that when {\Xic} is generating SPICE output, instances of devices with
this property will have an `{\vt X}' prepended to the device name,
thereby making the device appear as a subcircuit call in SPICE.  This
enables a component, such as a resistor, to be modeled by a subcircuit
that may contain dependent sources to model nonlinear behavior.  The
subcircuit definition is likely provided in the SPICE models from a
vendor or foundry.
\end{description}

For device cells, as would appear in the device library file, {\et
model}, {\et value}, and {\et param} properties can be applied.  When
a device instance is placed, the instance will inherit copies of these
properties.  Instances of non-devices do {\bf not} inherit a {\et
param} property from the master.  The {\et model} and {\et value}
properties can not be applied with the {\cb Cell Properties Editor},
but can be added to device masters with the {\cb Library Device
Parameters} panel (see \ref{devedit}).
 
