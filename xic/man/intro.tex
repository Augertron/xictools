% -----------------------------------------------------------------------------
% Xic Manual
% (C) Copyright 2013, Whiteley Research Inc., Sunnyvale CA
% $Id: intro.tex,v 1.109 2017/03/22 22:27:01 stevew Exp $
% -----------------------------------------------------------------------------

% -----------------------------------------------------------------------------
% not in help
\chapter{Introduction to {\Xic}}
\pagenumbering{arabic}

This chapter will provide an overview of the {\Xic} program, setup and
initialization information, and information for basic use.  Detailed
information on the various commands, features, and modes will be found
in the following chapters.  Information on file formats and other
rather technical topics can be found in the appendices.  New users
should read this chapter and the first two sections of the following
chapter thoroughly, and read the sections in the remaining chapters
describing the commands referred to in the usage sections in chapter
2.  The on-line help contains most of the information presented in
this manual, in a cross-referenced format.  Users will likely make
extensive use of the help system.  The information provided in the
help system is generally more up-to-date than can be provided in the
manual, and should be considered to be correct if there is ever a
conflict.

Whiteley Research is more than happy to assist users by answering
questions and providing information.  The ``{\cb WR}'' button in the
{\Xic} interface brings up a mail client which can be used to send
questions to Whiteley Research, which will be answered as soon as
possible.  However, in order for this service to operate efficiently,
it is requested that users make an effort to answer questions by
reading the provided documentation before contacting Whiteley
Research.

In this manual, text which is provided in {\vt typewriter} font
represents verbatim input to or output from the program.  Text
enclosed in square brackets ( [text] ) is optional in the given
context, as in optional command arguments, whereas other text should
be provided as indicated.  Text which is {\it italicized\/} should be
replaced with the necessary input, as described in the accompanying
text.

% -----------------------------------------------------------------------------
% xictop 040713
\section{{\Xic} Graphical Editor Overview}

{\Xic} is a dual-mode graphical editing tool.  In the physical editing
mode, {\Xic} is a hierarchical mask layout editor, with interactive
and batch mode design rule checking, arbitrary angle polygon and wire
support, netlist and parameter value extraction, and many more
advanced features.  In electrical layout mode, {\Xic} serves as a
hierarchical electrical schematic editor and schematic capture front
end for SPICE.  In the {\XicTools} environment, circuit simulation can
be performed and results analyzed from within {\Xic}, through an
interprocess communication channel established to the companion
{\WRspice} program.

Arrayed along the top of the main window is a toolbar containing
drop-down menu selectors.  Below the menu bar is a tool bar containing
buttons and other controls, including the coordinate readout area to
the right.  To the left of the main window is an array of additional
command buttons.  These menu commands control the operation of {\Xic}. 
The main drawing window occupies the largest section of the main
window.  The main drawing window supports drag and drop as a drop
receiver for files.  To the left of the main drawing window is the
layer menu, which displays a listing of the layers used in the
process.  The layers, and their attributes, are specified in a
technology file read by {\Xic} at program startup.

Just below the main drawing window is the prompt line, which provides
a channel for text-mode interaction with the program.  In the same
row, below the buttons in the side menu is the key press buffer area,
which records characters typed into the graphics window.  It is
invisible until characters are typed.  The typed characters are
interpreted as command accelerators.  Below the prompt line, at the
bottom of the main window, is a status line which provides information
about the current program operating state.

The WR button, in the upper left corner of the main window, brings
up a mail client which can be used to send messages and files via
internet mail.  It is preloaded with the address of the technical
support group at Whiteley Research.

Despite the array of features, {\Xic} is intended to be
straightforward and intuitive to use, {\Xic} has extensive on-line
documentation available through a context-sensitive help system.  This
help system can easily be augmented and customized by the user, so
that the user's design rules and tips, and other technical information
can be made available from within {\Xic}.

{\Xic} includes a native, script execution facility, with plug-in
support for Python and Tcl/Tk.  These languages will be available if
installed on the user's computer.  The native scripting language is a
straightforward but powerful C-like language with a rich library of
primitives for controlling the operation of {\Xic}.  Scripting can be
used for automation, for parameterized cells and executable labels,
and to implement user-defined commands.  These commands may appear as
buttons in the {\cb User Menu}.

One application of the user scripts is to provide simple, menu based
commands for creating geometrical objects, devices, or parameterized
device structures for use in circuit layout.  Further uses for this
capability are limited only by the user's imagination.

{\Xic} can execute scripts in batch and server modes, allowing
geometrical manipulations to be performed in a background or non-local
environment.  As a server, {\Xic} can serve as the workhorse back-end
for web-based or turn-key third-party products or services, or
in-house custom applications.

\ifoa
{\Xic} provides access to the OpenAccess database via a plug-in.  It
can utilize the OpenAccess database provided with Cadence Virtuoso, or
Synopsys/Ciranova PyCell Studio, and others.  {\Xic} has some limited
compatibility with Cadence Virtuoso:  {\Xic} can directly read Cadence
technology and display resource files, and can read layout and some
schematic and schematic symbol views.
\fi

Default schematic editing support is provided for a wide variety of
devices, even Josephson junctions.  Additional devices and subcircuits
can easily be added by the user, or changes can be made to existing
devices, by editing a single text file.  {\Xic} also provides a
high-powered model library search engine compatible with any SPICE
format model or subcircuit library files, such as those provided by
semiconductor manufacturers.

Hard copy support is available for a variety of printers and file
formats, including PostScript (mono and color), HPGL, and HP laser. 

{\Xic} has support for several archive layout file formats, plus
native input and output.  Data input in a given format will remain in
that format, unless explicitly converted.

{\Xic} produces data files which contain both electrical and physical
data, though one of these two data areas may be empty.  The file
format used can be one of:
\begin{itemize}
\item{The native format, in which each cell of a design is written to
an independent ASCII file.}
\item{An extension of GDSII, a binary format where the entire design    
can be written to a single file.}
\item{The newer and more compact OASIS format, which is a replacement
for GDSII.}
\item{An extension of CIF, a multi-cell format, somewhat archaic,
but an ASCII format so human-readable.}
\item{The CGX file format, developed by Whiteley Research.}
\ifoa
\item{OpenAccess, a third-party database used by Cadence and others.}
\fi
\end{itemize}
{\Xic} will read any of these file types automatically, and save any
editing changes in the same file type unless instructed otherwise.

Built-in converters can be used to convert between the file formats. 
It is possible to ``strip'' the output, providing a
physical-data file completely compatible with the industry standard
file formats, for portability of mask layout information.  It is also
possible to read and write a ``text-mode'' version of GDSII files,
which can be used to repair corrupted or misbehaving GDSII databases.

{\Xic} provides a powerful facility for translating between supported
layout file formats, while potentially modifying the data.  Possible
modifications include layer filtering and aliasing, cell name global
modification and aliasing, flattening, and spatial filtering to a
rectangular area with or without clipping, cell replacement, and more. 
These operations can be applied to very large files, as a unique
technique minimizes memory use.

In physical mode, design rule checking can be performed as each new
object is created of modified.  Batch mode checking is also available,
either in the foreground, or as a background child process.  The
philosophy of {\Xic} is that it is never in the user's best interest
to ``cheat'' in the enforcement of design rules, yet there may be
times when a given rule is not appropriate, and a modified rule should
be used.  Following this philosophy, the user is given complete
control over the design rules, which can be edited, disabled, or rules
added interactively.  The user can initiate batch mode design rule
checking over a given area or over a complete cell.  Design rule
checking is performed over a pseudo-flat internal representation of
the layout, so that physical rules are checked without any constraint
based upon which subcells contain the geometry.

{\Xic} has provision for netlist and parameter extraction.  The
netlist obtained from the physical layout, plus extracted physical
device parameters, can be used to generate a SPICE output file, and
even a schematic.  Automated layout vs.~schematic (LVS) testing is
provided.

% -----------------------------------------------------------------------------
% not in help
\section{{\Xic} Feature Sets}

Depending on the license purchased, the user may have access to only a
subset of features.  These feature sets correspond to "virtual"
products, that were historically separate programs.

There are three feature sets available.  The ``{\vt FULL}'' set enables
all {\Xic} features.  The ``{\vt EDITOR}'' feature set corresponds to the
{\XicII} program, which provides physical layout editing capability.
The ``{\vt VIEWER}'' feature set corresponds to the {\Xiv} program, which
allows physical layout viewing.  The subsections that follow describe these
feature sets in more detail.

%----------------------------------
% xicii 090814
\subsection{The EDITOR Feature Set}
\index{EDITOR feature set}
\index{XicII program}
This feature set corresponds to the {\XicII} virtual product.  This
was once a stand-alone layout editor product.  Currently, the same
functionality is provided via running {\Xic} with the EDITOR feature
set, which is set during license authentication.

If the full {\Xic} feature set is licensed, one can force running with
the EDITOR feature set by setting the environment variable {\et
FORCE\_XICII} before starting the {\Xic} program.

This feature set restricts the functionality to physical layout
editing.  This provides a low-cost alternative for users that do not
require the full functionality of {\Xic}.  We will continue to use
``{\XicII}'' to refer to {\Xic} running with this feature set.

In order to streamline support and maintenance, the documentation
tree, i.e., the manual, help database, and release notes, is common to
all feature sets.  This is a slight disadvantage to users of
restricted feature sets, as the documentation contains descriptions of
disabled features, which may lead to confusion.  However, this greatly
simplifies maintaining the documentation which translates into lower
cost to the customer, and also provides information to users about the
potential functionality available by upgrading to a non-restricted
license.

This section will list the differences and features that are
unavailable in the {\XicII} virtual product.

\begin{enumerate}
\item{\bf Technology File}\\
Parts of the technology file that relate to features that are not
available in {\XicII} are ignored, but will generate warning messages. 
In the example technology files, these features are enclosed in
macro-tested blocks to avoid the warnings.  The syntax is

\begin{quote}\vt
If FEATURESET == "FULL"\\
...\\
EndIf
\end{quote}

The right side of the conditional can take these values:

\begin{description}
\item{\vt "FULL"}\\
All features enabled.
\item{\vt "EDITOR"}\\
Layout editing feature set ({\XicII})
\item{\vt "VIEWER"}\\
Layout viewing feature set ({\Xiv})
\end{description}

\item{\bf No Design Rule Checking}\\
{\XicII} does not have DRC support, consequently there is no
{\cb DRC Menu} in {\XicII}.

\item{\bf No Electrical Mode}\\
{\XicII} is a physical layout tool only.  There is no schematic entry,
and no SPICE capability.  There is no {\cb Electrical} or {\cb
Physical} button in the {\cb View Menu}.

\item{\bf No Extraction}\\
{\XicII} has no extraction capability and no {\cb Extract Menu}.

\item{\bf No Batch or Server Modes}\\
The background processing capability is not available in {\XicII}.

\item{\bf `!' Commands}\\
The `!' commands in {\XicII} are identical to those in {\Xic}, however
`!' commands in {\XicII} which relate to unavailable features will not
be recognized.
\end{enumerate}

%----------------------------------
% xiv 090814
\subsection{The VIEWER Feature Set}
\index{VIEWER feature set}
\index{Xiv program}
This feature set corresponds to the {\Xiv} virtual product.  This was
once a stand-alone layout viewer product.  Currently, the same
functionality is provided via running {\Xic} with the VIEWER feature
set, which is set during license authentication.

If the full {\Xic} feature set or the EDITOR feature set is licensed,
one can force running with the VIEWER feature set by setting the
environment variable {\et FORCE\_XIV} before starting the {\Xic}
program.

This feature set restricts the functionality to physical layout
viewing.  This provides a low-cost alternative for users that do not
require the full functionality of {\Xic}.  We will continue to use
``{\Xiv}'' to refer to {\Xic} running with this feature set.

In order to streamline support and maintenance, the documentation
tree, i.e., the manual, help database, and release notes, is common to
all feature sets.  This is a slight disadvantage to users of
restricted feature sets, as the documentation contains descriptions of
disabled features, which may lead to confusion.  However, this greatly
simplifies maintaining the documentation which translates into lower
cost to the customer, and also provides information to users about the
potential functionality available by upgrading to a non-restricted
license.

This section will list the differences and features that are
unavailable in the {\Xiv} virtual product.

\begin{enumerate}
\item{\bf Technology File}\\
Parts of the technology file that relate to features that are not
available in {\Xiv} are ignored, but will generate warning messages. 
In the example technology files, these features are enclosed in
macro-tested blocks to avoid the warnings.  The syntax is

\begin{quote}\vt
If FEATURESET == "FULL"\\
...\\
EndIf
\end{quote}

The right side of the conditional can take these values:

\begin{description}
\item{\vt "FULL"}\\
All features enabled.
\item{\vt "EDITOR"}\\
Layout editing feature set ({\XicII})
\item{\vt "VIEWER"}\\
Layout viewing feature set ({\Xiv})
\end{description}

\item{\bf No Editing}\\
All cells are treated as read-only.  The menus that relate to changing
the layout ({\cb Edit} and {\cb Modify}) are absent.

\item{\bf No Design Rule Checking}\\
{\Xiv} does not have DRC support, consequently the {\cb DRC Menu} is
absent.

\item{\bf No Electrical Mode}\\
{\Xiv} is a physical layout viewing tool only.  There is no schematic
entry, and no SPICE capability.  There is no {\cb Electrical} or {\cb
Physical} button in the {\cb View Menu}.

\item{\bf No Extraction}\\
{\Xiv} has no extraction capability and no {\cb Extract Menu}.

\item{\bf No User Menu}\\
Scripting is not available.

\item{\bf No Batch or Server Modes}\\
The background processing capability is not available in {\Xiv}.

\item{\bf `!' Commands}\\
The `!' commands in {\Xiv} are identical to those in {\Xic}, however
`!' commands in {\Xiv} which relate to unavailable features will not
be recognized.
\end{enumerate}


% -----------------------------------------------------------------------------
% captour 010615
\section{A Quick Tour of {\Xic} Capabilities}

\subsection{History of {\Xic}}

The precursor to {\Xic} was the Kic layout editor, a very simple
no-frills layout editor developed at Berkeley in the 1980's.  In the
late 1980s, the author needed a layout editor to support contract
development and research efforts in superconductive electronics, and
adopted Kic, run under something called a ``DOS extender'' (to support
32-bit applications) on an early and very expensive i386 computer. 
This required extensive modification to Kic, mostly to support the PC
graphics.  Kic is still available as free software on the Whiteley
Research web site.

After Unix became available for 386/486 PCs in the form of the FreeBSD
operating system, DOS and direct-write graphics became history. 
{\Xic} became a separate program in late 1995, initially using the
X-window system (Xt) user interface toolkit.  Over the following
years, {\Xic} became a full-time development effort, and the
extraction, DRC, and other subsystems were added.  Although to this
day faint similarities to Kic exist, internally the code has been
replaced has been replaced by several iterations of more modern code,
and the database and other systems were replaced with improved
implementations.

Eventually, {\Xic} underwent a complete rewrite into C++ (from C) to
improve maintainability and organization.  The GTK toolkit was
adopted for the user interface.

Whiteley Research Inc.  was founded in 1996 to market {\Xic}, and the
companion {\WRspice} program.  Since then, {\Xic} has continued to
develop, as new users brought forward new ideas and requirements.

\subsection{General}

{\Xic} provides a menu of buttons along the side (the ``side menu''),
and s number of drop-down menus along the top of the main window. 
{\Xic} responds to key presses in various ways, and provides an
input/output text area just below the main window.  Key presses are
interpreted as macros, special commands, menu command accelerators, or
as `!' commands.  Several control sequences directly initiate certain
operations, for example {\kb Ctrl-r} will redraw the window and {\kb
Ctrl-g} will prompt for grid parameters.  Other control sequences will
trigger menu commands as accelerators, and typing the unique prefix of
the command name (as shown in the tool tip which appears as the mouse
pointer hovers over a menu entry) will trigger menu commands.  If `!'
is pressed, the rest of the sequence (until {\kb Enter} is pressed) is
taken as an internal or Unix shell command.  If `?' is pressed, the
rest of the sequence (until {\kb Enter} is pressed) is taken as a help
database keyword.

\subsection{The Help System}

{\Xic} contains a comprehensive HTML-based on-line help system.  The
help viewer can also function as a web browser, providing access to
internet resources.  The viewer can serve as an input device for
scripts, i.e., the window would contain a form which provides
parameters to a script.  The help database can be augmented by the
user, allowing local information to be easily accessed.

{\Xic} is internet aware, and can actually open design files
served by a remote HTTP or FTP host.  Files can also be opened in
response to clicking on links in the help viewer.

\subsection{Cadence Virtuoso and OpenAccess Compatibility}

\ifoa
{\Xic} can read and write design data to an OpenAccess database, but
OpenAccess is not required.  {\Xic} can read and use ASCII technology
and DRF files intended for Cadence Virtuoso and other similar tools,
as provided by chip foundries.  {\Xic} can read schematic, symbol, and
layout views produced by Cadence, and to varying degrees, obtain a
working, simulatable cell hierarchy.  Presently, it is not possible
to write back schematic information to Cadence without corruption.

{\Xic} supports Ciranova/Synopsys portable Python-based parameterized
cells, and provides support for abutment and stretch handles in native
parameterized cells.
\else
{\Xic} can read and use ASCII technology and DRF files intended for
Cadence Virtuoso and other similar tools, as provided by chip
foundries.

OpenAccess support is presently not available.  Such support would
allow access to an OpenAccess database, such as used with Virtuoso. 
This support, once available.  has been removed for legal/financial
reasons, but could be reinstated if there is sufficient demand. 
Contact Whiteley Research for more information.
\fi

\subsection{Layout Editing}

First and foremost, {\Xic} is an editor for integrated circuit mask
layouts.  Although, in large measure, the notion of mask layout from
manual polygon placement has disappeared in modern electronics, having
been replaced by automated cell place and route, there are still many
instances where layout viewing and editing are essential.  {\Xic} is
designed the make this task efficient and straightforward.

{\Xic} makes use of a proprietary database technology which provides
extremely fast access to spatially-keyed data.  The database
technology has changed several times over the life of the program, and
the current database, though invisible to users, is an important
achievement.

{\Xic} has a complete set of features for creating, moving,
transforming, and modifying geometrical features and subcells, with
complete undo/redo capability.  Most of these features are accessed
from the side menu, and from the {\cb Edit Menu} and {\cb Modify
Menu}.  Basic mouse operations allow selection, and moving, copying,
or stretching selected objects.  The ability to create physical text
or crude images (e.g., for company logos) is built in.

{\Xic} operates on a cell hierarchy, and has commands to push
and pop the editing context through the hierarchy, and to flatten
the hierarchy to arbitrary depth.

Some releases of {\Xic} are 32-bit applications, and as such have an
inherent memory limitation of about 3Gb.  {\Xic} has internal memory
management which is designed to use as much available virtual memory
as possible.  On a system with sufficient memory, 2-3 GB files can be
read in for editing directly.  In {\Xic} releases compiled for
64-bits, there is no such memory limitation.

\subsection{Input/Output}

The technology-specific information used by {\Xic} is maintained in a
single human-readable file.  Most of the parameters set by the
technology file can be set or reset from within {\Xic}, and an updated
technology file can be easily generated.

{\Xic} can read or write files in several formats.  These include

\begin{description}
\item{GDSII}\\
The industry-standard binary data format.

\item{OASIS}\\
A new standard intended to replace GDSII and is far more
compact.

\item{CIF}\\
An ancient ASCII data format, still in use occasionally.

\item{CGX}\\
A more compact replacement for GDSII developed by Whiteley Research
(and placed in the public domain).  It still uses fixed-sized
integers, so is not nearly as compact as OASIS, but is simple to
generate and parse.

\item{Native}\\
A CIF-like cell-per-file format.

\ifoa
\item{OpenAccess}\\
If present, {\Xic} can read and write to an OpenAccess database,
including the databases provided with Cadence Virtuoso and Ciranova
PyCell Studio.
\fi
\end{description}

Any files in these formats can be read directly into {\Xic}, whether
or not the current technology matches.  In fact, it is possible (and
sometimes desirable) to start {\Xic} with no technology file.  As the
file is read, {\Xic} will add layers as necessary to represent the
file.  After changing layer colors and fill patterns as desired, a new
technology file can be dumped.

Files can be read into the {\Xic} database, and later written to disk
in any of these formats.  The default is to write in the same format
as the original file.

In addition, format conversions can be applied directly, bypassing the
database load.  While converting, windowing operations (clipping),
scaling, or flattening can be applied.  Since {\Xic} uses 64-bit file
offsets, the direct conversions can be applied to huge files, even
with 32-bit {\Xic} binaries and modest memory.

\subsection{Design Rule Checking}

{\Xic} has a built-in design rule checking engine, based on rules
provided in the technology file or interactively in {\Xic}.  Both
interactive (performed after every geometry modification) and
batch-mode checking (foreground or background) is supported, in all or
a portion of the layout.

Errors are reported in a log file, and indicators added on-screen. 
Clicking on the indicator can provide a close-up view of the error and
explanatory text.

There is a rule editor that gives the user complete control over the
rules and parameters in use.  Although a fairly complete set of
built-in tests is provided, user-defined tests allow more specialized
tests to be performed.  Special layers and flags allow objects and
regions to be ignored during testing.

\subsection{Electrical Mode}

When {\Xic} is in electrical mode (selectable under the {\cb View
Menu}) the main window is set up for schematic editing.  A
user-configurable palette of devices is available for placement. 
Devices are placed, wired together, and properties added to provide
device parameters.  Once a schematic is complete, it can be dumped as
a SPICE file, or simulation can be performed interactively through the
companion {\WRspice} program.  Performing a simulation is as easy as
clicking the {\cb run} button in the side menu, then, when complete,
the {\cb plot} button can be pressed, then clicking on nodes in the
circuit diagram will display simulation plots.  Plots can also be
created while simulating, and are updated as the simulation
progresses.

There are provisions for providing arbitrary names for nodes and
devices in the circuit.  The default is for {\Xic} to define the names
in most cases.  There is a symbolic representation capability,
enabling a subcircuit to have a special symbol, instead of a
schematic, when used as a subcell.

{\Xic} provides vectorized instance placements, and a complete
net expression capability for multi-conductor wire net definition.

Electrical-mode data is ``tied'' to the physical mode data, and saved
in the same file.  This requires some extensions to be employed in the
files.  These extensions are 1) usually ignored by other programs, and
2) can be easily stripped out to ensure portability of physical data.

\subsection{Extraction}

The commands in the {\cb Extract Menu} deal with the
electrical/physical association defined for a cell, i.e., the
electrical schematic and the physical layout.

It is not always necessary to enter the schematic by hand.  A
schematic can be produced from a SPICE file, or from the physical
layout.  The resulting schematic is perhaps not too useful from a
human-readability standpoint, but is valid nonetheless.  The user of
course has the option to rearrange things and make other changes to
promote readability and aesthetics.

There are provisions to update the schematic from the physical layout,
either globally or per-device.  It is possible to dump a netlist file
or SPICE file created directly from the physical layout.

There is provision for LVS (layout vs. schematic) analysis.

The parameters that control extraction, and device definitions for
extraction, generally appear in the technology file.  These can be
created or modified from within {\Xic} through the technology parameter
editor window.

\subsection{Automation}

{\Xic} contains a just-in-time compiler for a powerful built-in
scripting language.  The native language is C-like, though a Lisp-like
variant is also supported.  There is also interoperability with the
popular tcl/tk scripting language.

A lengthly and expanding set of interface functions allow {\Xic} to be
controlled by the scripts, and a very efficient computational geometry
engine allows database manipulation.

{\Xic} even supports a server mode, whereby {\Xic} does not use
graphics, and instead becomes a ``daemon'', listening for job
requests.  Other applications can use the server for geometrical and
other manipulations.  A similar batch mode, where {\Xic} again does
not use graphics but instead executes a script and exits, is also
available.

The user's scripts can appear as command buttons in the {\cb User
Menu}, allowing custom operations to be easily accessible in normal
operation.

The script language is used elsewhere, for example in user-defined
design rule tests, and in executable labels.  An executable label is a
text object in a design that when clicked-on will perform some
operation.  Scripts are also used in template (parameterized) cells,
which enable on-the-fly generation of subcells based on an arbitrary
set of parameters.


% -----------------------------------------------------------------------------
% menutour 022916
\section{A Quick Tour of the {\Xic} Menus}

\subsection{Side Button Menu}

Buttons arrayed along the side of the main window control the
generation of objects - rectangles, polygons, wires (fixed-width
paths), arcs, and rounded objects.  Other buttons enable setting
related defaults, such as wire end style and width, and the number of
vertices used in ``round'' objects.  Additional buttons control
operations such as erase/yank/put, xor, clipping, and rotating.  In
electrical mode, this menu changes to provide buttons for adding
connection terminals, controlling the node-naming, and managing the
simulation interface to the companion {\WRspice} program.  These
buttons are described in chapter \ref{sidemenu}.

\subsection{Top Button Menu}

There are a few buttons arrayed horizontally above the main drawing
window, along with the coordinates display.  These are associated with
the layer table and selection control.  The controls in this menu are
described in \ref{topbuttons}.

The drop-down menus arrayed along the top of the main window control
additional features.

In addition, there are a number of special `!' commands that are
entered by typing the command name.  These control or enable
additional features that are not as frequently used.

Finally, there is a rather sophisticated scripting interface with a
large collection of built-in functions, which enables the user to
create automation scripts.  These scripts can be initiated from the
{\cb User Menu}.

\subsection{File Menu}

The {\cb File Menu} provides commands to open, save, and list files,
cells, and other things.  This menu also contains the printer
interface.

\begin{tabular}{|l|l|l|p{2in}|} \hline
\multicolumn{4}{|c|}{\kb File Menu}\\ \hline
\kb Label & \kb Name & \kb Pop-up & \kb Function\\ \hline\hline
\et File Select & \vt fsel & \cb File Selection & Open file\\ \hline
\et Open & \vt open & none & Open new cell or file\\ \hline
\et Save & \vt sv & {\cb Modified Cells} & Save modified cells\\ \hline
\et Save As & \vt save & none & Save file, rename\\ \hline
\et Save As Device & \vt sadev & {\cb Device Parameters} & Electrical
  mode only,{\newline}apply defaults and save device\\ \hline
\et Print & \vt hcopy & \cb Print Control Panel & Hard copy plot\\ \hline
\et Files List & \vt files & \cb Path Files Listing & List search path files\\
  \hline
\et Hierarchy Digests & \vt hier & \cb Cell Hierarchy Digests &
  List of Cell Hierarchy Digests\\ \hline
\et Geometry Digests & \vt geom & \cb Cell Geometry Digests &
  List of Cell Geometry Digests\\ \hline
\et Libraries List & \vt libs & \cb Libraries & List libraries\\ \hline
\ifoa
\et OpenAccess Libs & \vt oalib & \cb OpenAccess Libraries & List OA
  libraries (with OA only)\\ \hline
\fi
\et Quit & \vt quit & none & Exit {\Xic}\\ \hline
\end{tabular}

\subsection{Cell Menu}

The {\cb Cell Menu} contains command buttons to change the current
cell, and to get information about cells in memory.

\begin{tabular}{|l|l|l|l|} \hline
\multicolumn{4}{|c|}{\kb Cell Menu}\\ \hline
\kb Label & \kb Name & \kb Pop-up & \kb Function\\ \hline\hline
\et Push & \vt push & none & Edit subcell\\ \hline
\et Pop & \vt pop & none & Edit parent cell\\ \hline
\et Symbol Tables & \vt stabs & \cb Symbol Tables & List of cell symbol
  tables\\ \hline
\et Cells List & \vt cells & \cb Cells Listing & List cells in memory\\ \hline
\et Show Tree & \vt tree & \cb Cell Hierarchy Tree & Display cell hierarchy\\
  \hline
\end{tabular}

\subsection{Edit Menu}

The {\cb Edit Menu} contains commands which provide panels for cell
placement and property editing, and other features.

\begin{tabular}{|l|l|l|p{1.75in}|} \hline
\multicolumn{4}{|c|}{\kb Edit Menu}\\ \hline
\kb Label & \kb Name & \kb Pop-up & \kb Function\\ \hline\hline
\et Enable Editing & \vt cedit & none & Enable/disable editing mode
 for current cell\\ \hline
\et Setup & \vt edset & \cb Editing Setup & Show {\cb Editing Setup} panel\\
 \hline
\et Create Cell & \vt crcel & none & Create new cell\\ \hline
\et Create Via & \vt crvia & none & Create a standard via\\ \hline
\et Flatten & \vt flatn & \cb Flatten Hierarchy & Flatten hierarchy\\ \hline
\et Join/Split & \vt join & \cb Join or Split Objects & Control join/split
  operations\\ \hline
\et Layer Expression & \vt lexpr & \cb Evaluate Layer Expression & Control
  layer expression evaluation\\ \hline
\et Properties & \vt prpty & \cb Property Editor & Edit properties\\ \hline
\et Cell Properties & \vt cprop & \cb Cell Property Editor & Edit cell
  properties\\ \hline
\end{tabular}

\subsection{Modify Menu}

The {\cb Modify Menu} contains supplements the side menu with commands
to undo/redo operations, and move, copy, and delete objects.  Most of
these commands have a faster keyboard equivalent.

\begin{tabular}{|l|l|l|l|} \hline
\multicolumn{4}{|c|}{\kb Modify Menu}\\ \hline
\kb Label & \kb Name & \kb Pop-up & \kb Function\\ \hline\hline
\et Undo & \vt undo & none & Undo last operation\\ \hline
\et Redo & \vt redo & none & Redo last undo\\ \hline
\et Delete & \vt delet & none & Delete objects\\ \hline
\et Erase Under & \vt eundr & none & Erase under objects\\ \hline
\et Move & \vt move & none & Move objects\\ \hline
\et Copy & \vt copy & none & Copy objects\\ \hline
\et Stretch & \vt strch & none & Stretch objects\\ \hline
\et Chg Layer & \vt chlyr & none & Move object to new layer\\ \hline
\et Set Layer Chg Mode & \vt mclcg & {\cb Layer Change Mode} &
 Set layer change mode for move/copy\\ \hline
\end{tabular}

\subsection{View Menu}

The {\cb View Menu} contains commands which affect the presentation of
the current design, including the selection of physical and electrical
(schematic) modes.

\begin{tabular}{|l|l|l|l|} \hline
\multicolumn{4}{|c|}{\kb View Menu}\\ \hline
\kb Label & \kb Name & \kb Pop-up & \kb Function\\ \hline\hline
\et View & \vt view & none & Set view in window\\ \hline
{\et Physical} or {\et Electrical} & {\vt phys} or {\vt sced} & none &
 Switch mode\\ \hline
\et Expand & \vt expnd & \cb Expand & Show detail in window\\ \hline
\et Zoom & \vt zoom & dialog & Change window scale\\ \hline
\et Viewport & \vt vport & sub-window & New drawing window\\ \hline
\et Peek & \vt peek & none & Show layers in area\\ \hline
\et Cross Section & \vt csect & sub-window & Show layers in cross-section\\
 \hline
\et Rulers & \vt ruler & none & Add transient gradations\\ \hline
\et Info & \vt info & \cb Info & Show cell/object parameters\\ \hline
\et Allocation & \vt alloc & \cb Memory Monitor & Show memory statistics\\
 \hline
\end{tabular}

\subsection{Attributes Menu}

The {\cb Attributes Menu} provides commands which affect the
presentation of the design, such as the colors used.

\hspace*{-1.5em}
\begin{tabular}{|l|l|l|l|} \hline
\multicolumn{4}{|c|}{\kb Attributes Menu}\\ \hline
\kb Label & \kb Name & \kb Pop-up & \kb Function\\ \hline\hline
\et Save Tech & \vt updat & none & Save technology file\\ \hline
\et Key Map & \vt keymp & none & Create keyboard mapping file\\ \hline
\et Define Macro & \vt macro & none & Define a keyboard macro\\ \hline
\et Main Window & & Attributes sub-menu & Set main window attributes\\ \hline
\et Set Attributes & \vt attr & \cb Window Attributes &
    Set rendering attributes for main window\\ \hline
\et Connection Dots & \vt dots & \cb Connection Points &
    Show connection dots in schematics\\ \hline
\et Set Font & \vt font & \cb Font Selection & Set text fonts used\\ \hline
\et Set Color & \vt color & \cb Color Selection & Set layer and other colors\\
    \hline
\et Set Fill & \vt fill & \cb Fill Pattern Editor & Set layer fill patterns\\
    \hline
\et Edit Layers & \vt edlyr & \cb Layer Editor & Add or remove layers\\ \hline
\et Edit Tech Params & \vt lpedt & \cb Tech Parameter Editor &
    Edit technology parameters\\ \hline
\end{tabular}

\subsection{Convert Menu}

The {\cb Convert Menu} provides commands for importing and exporting
designs to various non-native file formats, and for converting between
file formats.

\hspace*{-1.5em}
\begin{tabular}{|l|l|l|l|} \hline
\multicolumn{4}{|c|}{\kb Convert Menu}\\ \hline
\kb Label & \kb Name & \kb Pop-up & \kb Function\\ \hline\hline
\et Export Cell Data & \vt exprt & \cb Export Control &
  Create a cell data file\\ \hline
\et Import Cell Data & \vt imprt & \cb Import Control &
  Read a cell data file\\ \hline
\et Format Conversion & \vt convt & \cb Format Conversion & Direct file-to-file
  format conversions\\ \hline
\et Assemble Layout & \vt assem & \cb Layout File Merge Tool &
  Merge layout data\\ \hline
\et Compare Layouts & \vt diff & \cb Compare Layouts &
  Find differences between layouts\\ \hline
\et Cut and Export & \vt cut & \cb Export Control &
  Write out part of a layout\\ \hline
\et Text Editor & \vt txted & \cb Text Editor & Text edit cell file\\ \hline
\end{tabular}

\subsection{DRC Menu}

The {\cb DRC Menu} contains commands associated with design rule
checking.

\begin{tabular}{|l|l|l|p{2.25in}|} \hline
\multicolumn{4}{|c|}{\kb DRC Menu}\\ \hline
\kb Label & \kb Name & \kb Pop-up & \kb Function\\ \hline\hline
\et Setup & \vt limit & \cb DRC Parameter Setup & Set limits and other
   parameters\\ \hline
\et Set Skip Flags & \vt sflag & none & Set skip flags\\ \hline
\et Enable Interactive & \vt intr & none & Set interactive DRC\\ \hline
\et No Pop Up Errors & \vt nopop & none & No interactive errors list\\ \hline
\et Batch Check & \vt check & \cb DRC Run Control & Initiate DRC run\\ \hline
\et Check In Region & \vt point & none & Test rules in region\\ \hline
\et Clear Errors & \vt clear & none & Erase error indicators\\ \hline
\et Query Errors & \vt query & none & Print error messages\\ \hline
\et Dump Error File & \vt erdmp & none & Dump errors to file\\ \hline
\et Update Highlighting & \vt erupd & none & Update highlighting from file\\
  \hline
\et Show Errors & \vt next & sub-window & Sequentially display errors from
  file\\ \hline
\et Create Layer & \vt erlyr & none & Write highlight error regions to
  objects on layer\\ \hline
\et Edit Rules & \vt dredt & \cb Design Rule Editor & Edit rules for layers\\
 \hline
\end{tabular}

\subsection{Extract Menu}

The {\cb Extract Menu} provides commands associated with the
extraction of electrical information and netlists from the physical
layout, and layout versus schematic checking.

\hspace*{-1.5em}
\begin{tabular}{|l|l|l|l|} \hline
\multicolumn{4}{|c|}{\kb Extract Menu}\\ \hline
\kb Label & \kb Name & \kb Pop-up & \kb Function\\ \hline\hline
\et Setup & \vt excfg & \cb Extraction Setup & Set up and control
  extraction\\ \hline
\et Net Selections & \vt exsel & \cb Path Selection Control & Select groups,
 nodes, paths\\ \hline
\et Device Selections & \vt dvsel & \cb Show/Select Devices & Select
 and highlight devices\\ \hline
\et Source SPICE & \vt sourc & \cb Source SPICE File & Update from SPICE
 file\\ \hline
\et Source Physical & \vt exset & \cb Source Physical & Update electrical from
 physical\\ \hline
\et Dump Phys Netlist & \vt pnet & \cb Dump Phys Netlist & Save physical
 netlist\\ \hline
\et Dump Elec Netlist & \vt enet & \cb Dump Elec Netlist & Save electrical
 netlist\\ \hline
\et Dump LVS & \vt lvs & \cb Dump LVS & Save physical/electrical comparison\\
 \hline
\et Extract C & \vt exc & \cb Cap Extraction & Extract capacitance
 using Fast[er]Cap\\ \hline
\et Extract LR & \vt exlr & \cb LR Extraction & Extract L/R using FastHenry\\
 \hline
\end{tabular}

\subsection{User Menu}

The {\cb User Menu} contains the script debugger, and the buttons that
correspond to user-generated scripts.

\begin{tabular}{|l|l|l|l|} \hline
\multicolumn{4}{|c|}{\kb User Menu}\\ \hline
\kb Label & \kb Name & \kb Pop-up & \kb Function\\ \hline\hline
\et Debugger & \vt debug & \cb Script Debugger & Debug scripts\\ \hline
\et Rehash & \vt hash & none & Rebuild {\cb User Menu}\\ \hline
others &  --- & --- & User scripts and menus\\ \hline
\end{tabular}


% -----------------------------------------------------------------------------
% xicdb 090814
\section{Database Overview}

The core of {\Xic} is the main database, which stores objects in a
format that can be rapidly accessed spatially.  The database, when
given a rectangular region, will efficiently provide a list of
contained objects whose bounding boxes overlap the given region.  For
example, when the user clicks or drags in a drawing window, the main
database will quickly provide a list of the objects which overlap this
area, so they may be shown as selected.

Each cell in memory has a database for each layer used by objects in
the cell, plus a database corresponding to a dummy layer which
contains the locations of subcell instances.  The cells themselves are
saved in one or more hash tables, the ``symbol tables''.  The symbol
tables allow cell data to be rapidly found by name.  Cell name strings
are saved in a common string table, so that address values can be used
for efficient string comparison.

Each symbol table represents a self-contained design space, which can
be rapidly switched between.  {\Xic} allows the user to define any
number of symbol tables.  Cells of the same name can not be saved in
the same symbol table, but can exist in different symbol tables. 
Thus, for example, different versions of the same cell hierarchy can
be kept in memory simultaneously, but the user can only view/edit
using one symbol table at a time.  This capability is used
transparently by the geometry comparison functions, for example, in
comparing two versions of the same cell.

The main database is organized as a tree, though the details are
proprietary.  This structure is self-balancing, unlike KD trees, thus
there is no need to ``rebuild'' the database when objects are added or
removed.  The structure is optimized for rapid access, at a cost of
time to build the structure.  It is also optimized for low memory
consumption, at a slight cost in speed.

When a file in loaded into the {\Xic} ``main'' database, cell
structures are created for each cell defined in the file.  The cell
structures contain trees for each layer used plus one for subcells if
any, and are linked into the current symbol table.

The main database, with spatial access features, is not particularly
efficient with regard to memory use.  Large designs may not fit into
available memory, depending on the machine.  The physical memory
limitation of the computer determines the maximum size of a file that
can be read into {\Xic} efficiently.  Very roughly, the memory
available should equal the size of the (uncompressed) GDSII file.  If
the file requires too much memory, {\Xic} performance can become very
sluggish due to page swapping, or in some cases the operating system
will halt the process if memory limits are exceeded.

Although the design must reside in the main database for efficient
cell editing, there are operations where this is not needed.  There
are provisions for handling extremely large files which can not be
normally loaded.

\subsection{Cell Hierarchy Digest}

The Cell Hierarchy Digest (CHD) is a data structure designed to solve
this problem.  A CHD is an in-memory database which contains
information about a hierarchy of cells, in a very compact manner.  It
holds no information about the geometry contained in the cells, but
does contain offsets into the original layout file, so that through
the CHD, the cell contents can be obtained reasonably quickly.  Since
the CHD uses a small fraction of the memory of the full design in the
main database, it allows operations to be performed on very large
designs with a modest computer.

The operations that can be performed with a CHD generally involve
translation of a layout file into another layout file.  For example,
cell sub-hierarchies can be extracted, scaled, layers filtered or
aliased, or cell names globally changed or aliased.  The hierarchy can
be flattened, filtered through a rectangular window and possibly
clipped to the window, and empty cells (possibly produced by layer
filtering) can be removed.

The CHD can also be used to view but not (directly) edit a large file. 
This is not as fast as viewing through the main database, but it is
possible to view much larger files with a CHD.

There are also some novel ways to use CHDs in {\Xic} to perform some
limited editing.  Reference cells in the main database are dummy cells
that contain no data, but reference a cell hierarchy through a CHD. 
These cells can be instantiated in other cells normally.  However,
when written to a layout file on disk, they are replaced in output
with the full referenced hierarchy obtained through the CHD.  Thus one
can use reference cells to assemble the top-level cell of a very large
design.  Each reference cell points to a sub-part of the design, kept
in a separate layout file.  When the top-level cell is written to
disk, all of the parts will be extracted and combined into this file.

There is a cell override table which contains the names of cells in
main memory.  When enabled, when reading cell data through a CHD,
cells in the override table will supersede cells in the original
layout file.  Thus, the cell override table provides a substitution
mechanism.  To perform minor editing in a hierarchy too large for main
memory, one can
\begin{enumerate}
\item{extract only the cells to be edited into main memory through a
CHD,}
\item{edit these cells, and place their names in the override
table, then}
\item{write a new layout file using the CHD, which will
contain the new versions of the cells.}
\end{enumerate}

There is a related Cell Geometry Digest (CGD) which contains highly
compact geometry collections on a per-cell/per-layer basis.  A CGD can
be linked to a CHD, with the total memory used still far smaller (by
approximately a factor of 10) than the same cell hierarchy in the main
database.  With a linked CGD, when reading cell data through the CHD,
the data are extracted from the CGD, avoiding accessing the original
file on disk.  This is usually faster.

\subsection{Database Resolution}

By default, {\Xic} uses an internal resolution of 1000 units per
micron.  In releases prior to 3.0.12, this was internally hard-coded. 
As the dimensions used in integrated circuits continue to shrink, an
option for higher resolution has been added.

The resolution can be set with the {\et DatabaseResolution} variable,
which can be set to ``{\vt 1000}'', ``{\vt 2000}'', ``{\vt 5000}'', or
``{\vt 10000}''.  If unset, 1000 units is used.  This resolution
applies only to physical data, electrical resolution is fixed at 1000.

This variable can be set only from the {\vt .xicinit} file, which is
read before the technology file, or the technology file.  It can not
be set or unset in a {\vt .xicstart} file (read after the technology
file) unless no technology file is read, or by any other means.  It is
important that the resolution be set before reading such things as DRC
rules, since the rules contain resolution-dependent numbers which
would be incorrect after a resolution change. 

Superficially, changing the internal resolution has only subtle
effects from the user's vantage point.  Some of these are:

\begin{enumerate}
\item{If not 1000, four digits following the decimal point are used
when printing coordinates in microns, in many places in {\Xic}. 
Otherwise, only three digits are used.}

\item{The ultimate zoom-in and grid spacing sizes are smaller for
higher resolutions.}

\item{The size of ``infinity'', the maximum accessible size for the
design, becomes smaller as resolution is increased, since
coordinates are stored internally as 32-bit integers.  For 1000
units, the field width is approximately 2 meters, which decreases
to 20 centimeters at 10000 units.  This should still be plenty for
most purposes.}

\item{Layout files produced by {\Xic} will use the internal
resolution, so that no accuracy is lost.}
\end{enumerate}

Unless there is a specific need, it is recommended that users employ
the default resolution.

