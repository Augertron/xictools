% -----------------------------------------------------------------------------
% Xic Manual
% (C) Copyright 2013, Whiteley Research Inc., Sunnyvale CA
% $Id: techfile.tex,v 1.32 2017/03/22 23:39:49 stevew Exp $
% -----------------------------------------------------------------------------

\appendix
\chapter{Technology File}
% -----------------------------------------------------------------------------
% techfile 032117
\label{techfile}
\index{technology file}
\index{xic\_tech file}
\index{-T option}
\index{technology file!extension}
The technology file tells {\Xic} all it knows about the layers and
display attributes, as well as being a general source of
initialization information.  The name of the file is ``{\vt
xic\_tech}'', and an extension {\it .xxx\/} can be added to the name,
so that if {\Xic} is started with the {\vt -T}{\it xxx\/} option, the
technology file with the extension will be used.  For example, ``{\vt
xic -Ttrw}'' would cause {\Xic} to read {\vt xic\_tech.trw}.

\index{starting {\Xic}!no technology file}
It is legitimate to start {\Xic} without reading a technology file, by
using ``{\vt xic -T}''.  In this case, new layers will be assigned as
needed as cells are read in.  This can be useful for examining an
undocumented GDSII file, for example.  Once the layout has been read
in, new colors and fill styles can be assigned, and the {\cb Save
Tech} command in the {\cb Attributes Menu} used to dump an appropriate
technology file for the next time.

\index{library path}
\index{Save Tech button}
The technology file is expected to be found along the library search
path, which can be set with the environment variable {\et
XIC\_LIB\_PATH}.  The default path is
\begin{quote}
 {\vt ( . /usr/local/xictools/xic/startup )}.
\end{quote}
The default technology file has been provided by your system
administrator.  A personalized version can be generated with the {\cb
Save Tech} command.

The technology file generally begins with comment lines explaining the
process that the file supports.  The order of the sections that follow
is rather flexible, though the printer driver blocks should appear
last.  It is recommended that one follow the ordering described here,
which is the order used by {\Xic} when generating a technology file,
to be on the safe side.  None of the sections is required to exist. 
Technology files for {\XicII} and {\Xiv} feature sets are simplified,
omitting the sections that apply to unavailable features.

At the top of the file are macro definitions using the {\vt Set} or
{\vt Define} keywords, and {\cb !set} lines for setting global
variables.  The introductory part of the file further consists of
optional path specifications.  The layer blocks follow, which is where
the core information about the particular technology resides.  The
electrical layers are defined first, followed by user-defined design
rules, followed by the physical layer definitions.

The physical layers are followed by the standard via definitions, then
the device blocks, where physical characteristics for device
extraction are given.  These are followed by script function
definitions.  Finally, there is a section containing display attribute
specifiers and other parameters, and the hard-copy driver parameter
blocks.

\index{technology file!backslash continuation}
Long lines can be continued in the technology file by using backslash
continuation.  For example, the following would be read as one line:
\begin{samepage}
\begin{quote}\vt
    This a line to be continued, the backslash $\backslash$ \\
    must be the last character in the line.
\end{quote}
\end{samepage}

The technology file has a macro facility which can be used to simplify
the constructs and to customize the file to a particular variation of
the technology.

The technology file may contain the following keyword/value pair
near the top of the file:
\begin{description}
\index{technology name}
\item{\vt Technology} {\it name}\\
The {\it name} can be any character token (no white space allowed)
and gives a name to the technology.  This is not directly used by
{\Xic}, but but the {\it name} is placed in the macro name space of
the macro preprocessor used when reading various types of input
files, including the device library.  The name is displayed in the
status line of the main window, and is part of the information
available for output in scripts and elsewhere.

\index{device library name}
\item{\vt DeviceLibrary} {\it libname}\\
The {\it libname} is the name of a device library file which provides
device outlines for use in schematics.  If not given, the name
defaults to ``{\vt device.lib}''.  The {\it libname} should be a file
name, without any directory path.  A file by that name should be found
in the library search path on program startup.

\index{model library name}
\item{\vt ModelLibrary} {\it libname}\\
The {\it libname} is the name of a model library file which provides
SPICE models for use in SPICE output.  If not given, the name defaults
to ``{\vt model.lib}''.  A file by that name should be found in the
library search path on program startup.

\index{model subdirectory name}
\item{\vt ModelSubdir} {\it dirname}\\
The {\it dirname} is the name of a subdirectory of the directories of
the library search path, in which are found SPICE model files.  All
directories of this name found in the library path will be searched
for SPICE models.  If not given, the name defaults to ``{\vt
models}''.

\item{\vt ReadDRF} {\it filename}\\
This is part of the Cadence$^{TM}$ compatibility package (see
\ref{cdscompat}).  The {\it filename} is the name of or path to a file
in the format of a Virtuoso display resource file (including those
from the Ciranova PyCell Studio).  The full path should be given
unless the file is in the library search path.  This provides display
attributes for layers.

\item{\vt ReadCdsTech} {\it filename}\\
This is part of the Cadence$^{TM}$ compatibility package (see
\ref{cdscompat}).  The {\it filename} is the name of or path to a file
in the format of a Virtuoso ASCII technology file.  The full path
should be given unless the file is in the library search path.  This
provides layer and purpose definitions, rules, constraints, and other
technology data.  Layers defined in this file will appear in addition
to those defined elsewhere.

%\ifoa
\item{\vt ReadOaTech} {\it library}\\
This will obtain Virtuoso technology information directly from
OpenAccess.  The {\it library} is an OpenAccess library, listed in
the {\vt lib.defs} or {\vt cds.lib} file.  This obtains technology
information by use of the OpenAccess plug-in.  There should be no
reason to use both this and {\vt ReadCdsTech}, as they should
retrieve the same information.
%\fi

\item{\vt ReadCdsLmap} {\it filename}\\
This is part of the Cadence$^{TM}$ compatibility package (see
\ref{cdscompat}).  The {\it filename} is a path to a Virtuoso
layer-mapping file, which provides GDSII layer/datatype numbers for
the layers.  This can be used in addition to, and must be called
after, {\vt ReadCdsTech}.  It is used to import the Stream mapping
for the layers.

\item{\vt ReadCniTech} {\it filename}\\
The {\it filename} is the name of or path to a file in the format
of a Ciranova (now Synopsys) ASCII technology file.  The full path
should be given unless the file is in the library
search path.  This file format is
superficially similar to a Virtuoso ASCII technology file, yet
sufficiently different that a separate reader is required.  The
format is documented in the PyCell Studio distribution from
%\ifoa
Synopsys, and example files are provided (see \ref{ciranova}).
%\else
%Synopsys.
%\fi

When setting up a technology file for the PyCell Studio or something
similar using Ciranova technology, it may be necessary to use this
keyword more than once, if the technology is described in more than
one file.  It is also necessary to use the {\vt ReadDRF} keyword to
read display resource files.

For example, here is a skeletal technology file for the Ciranova
130nm model process in the PyCell Studio, which is installed under
{\vt /usr/local/ciranova}.

\begin{quote} \vt
Set cni130 = /usr/local/ciranova/tech/cni130\\
ReadDRF \$(cni130)/santanaDisplay/SantanaDisplay.drf\\
ReadCniTech \$(cni130)/santanaTech/Santana.tech\\
ReadCniTech \$(cni130)/santanaDisplay/SantanaDisplay.tech
\end{quote}
\end{description}

The ability to read the Lisp/Skill file format used by Virtuoso is
provided by an internal Lisp parser.  The parser is available to run
general scripts through the {\cb !lisp} command, though this has
limited utility at present.

In the technology file, is is sometimes useful to enable debugging
output from the Lisp parser.  The following keyword enables this.

\begin{description}
\item{\vt LispLogging} [{\vt y/n}]\\
If this boolean keyword is set in the technology file, a log file will
be generated when the Lisp parser is used.  This can be used to track
down issues when parsing Virtuoso-style input files.  Asserting this
keyword is equivalent to setting the Lisp logging in the {\cb Logging
Options} panel from the {\cb Help Menu}, which otherwise can't be done
before the technology file is read on program startup.

The logging output is put into a file named {\it filename\/}{\vt
-lisp.log} in the logfiles directory.  The {\it filename} is the name
of the input file being parsed.
\end{description}


% -----------------------------------------------------------------------------
% techfile:comments 032117
\section{Technology File Comments}
\index{Comment keyword}

The technology file recognizes a {\vt Comment} keyword.  These lines have
no effect, but are saved and included when the file is written with
the {\cb Save Tech} command.  Thus, notes about the file can be preserved. 
An attempt is made to to place the comment in the same relative
position during an update.

Comments can also be included in the technology file after the `{\vt
\#}' character or `{\vt //}' sequence, however these comments will not
appear in a file written with the {\cb Save Tech} command.

Example:
\begin{quote} \vt
Comment  Technology file for the Ultra-MOS version 3.5 process\\
Comment  Version 1.3  March 24, 2002  George H. Frump
\end{quote}


% -----------------------------------------------------------------------------
% techfile:macros 032613
\section{Technology File Macros}
\index{technology file!macros}
\label{techmacros}

\index{Define keyword}
In order to facilitate customization of the technology file to
different variations, in particular to support scalable technology, a
macro facility is provided, along with an expression evaluator. 
Macros can be used to simplify or clarify the constructs used in the
technology file, and facilitate portability by effectively customizing
the technology file to different environments.

The macro capability makes use of the generic macro preprocessor
provided in {\Xic}, which is described in \ref{macro}.  The reader
should refer to this section for a full description of the
preprocessor capabilities.  The preprocessor provides a few predefined
macros used for testing (and customizing for) release number,
operating system, etc.  The keyword names, which correspond to the
generic names as described for the macro preprocessor, are
case-insensitive and listed in the following table.

\begin{tabular}{ll}
\bf Keyword & \bf Function\\
\vt Define & Define a macro.\\
\vt If & Conditional evaluated test.\\
\vt IfDef & Conditional definition test.\\
\vt IfnDef & Conditional non-definition test.\\
\vt Else & Conditional else clause.\\
\vt Endif & Conditional end clause.\\
\end{tabular}

A macro definition can appear anywhere in the technology file. 
Throughout the technology file, each line is macro expanded.  The
actual arguments replace the formal arguments (if any) in the
substitution text, which replaces the macro reference.  The macro is
recognized as a text token.

Example:

\begin{quote}\rr\vt
Define  mytext(x)  this is rule number x\\
...\\
MinWidth 2 \# mytext(1.2)\\
\end{quote}

The {\vt MinWidth} line expands to
\begin{quote}\vt
MinWidth 2 \# this is rule number 1.2
\end{quote}

The conditional keywords provide tests which can be used to select
which lines of the technology file are actually read, based of the
settings of existing macros and/or expression evaluation.  The logic
is explained in the description of the generic macro preprocessor.

Example:

\begin{quote}\rr\vt
Define TightRules\\
...\\
Layer M1\\
IfDef TightRules\\
MinWidth .4\\
Else\\
MinWidth .8\\
Endif\\
\end{quote}

In the example above, commenting out the {\vt Define} line
\begin{quote}\vt
\#Define TightRules
\end{quote}
reconfigures the technology file.

When the technology file is updated with the {\cb Save Tech} command,
only the lines that were actually processed are written, i.e., the
{\vt IfDef}, etc.  lines and unused blocks are stripped.

\subsection{The {\vt Set} Keyword: Variable Expansion}
\index{Set keyword}

A different type of macro is defined using the {\vt Set} keyword,
where the words following are parsed into three tokens
\begin{quote}
{\vt Set} {\it name\/} {\vt =} {\it value\/}
\end{quote}

This type of macro is referred to by
\begin{quote}
{\vt \$({\it name\/})}
\end{quote}

which is replaced by {\it value\/} as the file is read.  If the {\it
name} has not been assigned in a {\vt Set} line, but an environment
variable by that name is found, the substitution will be made from the
value of the environment variable.  Otherwise, the variable must be
set before being referenced, meaning that the {\vt Set} line must
appear before the first reference in the technology file.

Neither the {\it name\/} or {\it value\/} tokens can contain a
carriage return, though they can contain embedded white space.  In
either case, the beginning and end of the token is the first and last
non-white character.  Substitution is performed recursively.  The two
types of macro can be mixed, though the {\vt Set} line is not expanded
for {\vt Define}'ed macros.  Other lines are first expanded for {\vt
Define}'ed macros, then for {\vt Set} macros.

The {\vt Set} keyword should not be confused with the {\cb !set}
command, which can also appear in the technology file.

\subsection{The {\vt eval} Keyword: Expression Evaluation}

\index{lambda}
An expression involving integers or floating point numbers can be
evaluated as the file is read, with the result inserted into the line
at the place of evaluation.  This facilitates, for example, the use of
design rules based on the {\it lambda} concept.  In this type of rule
set, design rules are specified in terms of a minimum dimension {\it
lambda\/}.  The {\it lambda\/} may vary between different process
implementations.  In the technology file, {\it lambda\/} is defined as
a macro, and inputs to the design rule specifications is evaluated in
terms of {\it lambda\/}.

\index{eval keyword}
\index{technology file!eval}
The syntax for expression evaluation is {\vt eval({\it expression\/})}. 
This construct can occur anywhere in the text, although it makes sense
only where a number is expected.  The result of the evaluation is
substituted into the text replacing the {\vt eval} construct, before
that line of the technology file is interpreted.  The expression is
interpreted by the parser otherwise used for interpreting command
scripts, and the full complement of operations and functions is
available.  Macros are expanded before the expression is parsed.

Example:
\begin{quote}\vt
Set lambda = .6\\
...\\
PhysLayer BASE\\
MinWidth eval(2*\$(lambda))  \#Minimum width of the BASE layer is 2*lambda
\end{quote}

In this example, the parameter {\vt lambda} is defined to ``.6'' with
the {\vt Set} keyword.  Elsewhere in the file, design rules can be
specified as functions of {\vt lambda} using the {\vt eval} construct,
as shown.

Example:

\begin{quote}\rr\vt
Set lambda = .6\\
Define L(x) eval(\$(lambda)*x)\\
...\\

PhysLayer BASE\\
MinWidth L(2)  \#Min width of BASE layer is L(2)\\
\end{quote}

In this example, the macro {\vt L(x)} is used to hide the call to the
evaluation function, simplifying syntax.

If the technology file is updated to disk using the {\cb Save Tech}
command button, only the macros used in the design rule keywords will
be preserved in their original macro form in the new file.  Elsewhere,
the written lines will contain the expanded quantity.  All of the {\vt
Set} and {\vt Define} lines will be preserved.  Thus, the use of
macros should be restricted to the design rule keywords, unless the
user is willing to hand edit the new files produced with the {\cb Save
Tech} command.


% -----------------------------------------------------------------------------
% techfile:set 021615
\section{Technology File Global Variables}

\index{technology file!"!set}
Also typically appearing near the top of the technology file are
the {\vt !set} commands.
\begin{quote}
{\vt !set} {\it arguments}
\end{quote}
Unlike the {\vt Set} keyword, this directive assigns variables as if
the keyboard {\cb !set} command, as used interactively from the prompt
line, had been given.  The {\it arguments} are exactly as they would
appear on the prompt line.  Thus, the command attributes that are
controlled with the {\cb !set} command can be specified in the
technology file.  The technology file is read after the {\vt .xicinit}
file and before the {\vt .xicstart} initialization file, which are
other options for executing the {\cb !set} command at program startup.

This form is appropriate for variables that are defined by the user. 
Variables that are known to {\Xic} can presently be set as keywords
(see \ref{attrvars}), though the form described here can be used as
well.

When a new technology file is written with the {\cb Save Tech}
command, all {\cb !set} lines from the original technology file (if
any) are written as a block, but commented out.  This is followed by
another block containing all of the currently defined variables,
except for those known to {\Xic} that can be set as keywords.  These
include the path variables, and are written as keyword definitions
elsewhere in the file.  The present list will contain variables
defined by the user.  These lines are active.  The user can edit these
blocks as necessary.

The {\cb !attrvars} command generates a listing of the variables that
can be set as technology file keywords.


% -----------------------------------------------------------------------------
% techfile:paths 062109
\section{Technology File Path Definitions}

There are four search paths that may be specified.  In each case, the
path specification consists of a keyword, followed by the path.  The
format of the path is described in section \ref{searchpaths}
detailing the {\Xic} search paths.

In the path defaults below, if the {\et XT\_PREFIX} environment
variable is defined, its value will replace ``{\vt /usr/local}''.

\begin{description}
\index{Path keyword}
\item{\vt Path} {\it path}\\
The {\vt Path} keyword specifies the path to design data files: 
native cell, archive, and library files.  The current directory ``.''
should generally be listed first in this path.  The design data path can
also be set in the environment with the {\vt XIC\_SYM\_PATH} variable. 
A specification in the technology file will override a specification
in the environment.\\
Default:  {\vt ( ~.  )}

\index{LibPath keyword}
\item{\vt LibPath} {\it path}\\
The {\vt LibPath} keyword specifies the path to the startup files. 
The startup files include the device library (default name {\vt
device.lib}), and the model library (default name {\vt model.lib}). 
This path can also be set with the environment variable {\et
XIC\_LIB\_PATH}, and a specification in the technology file will
override an environment specification.  Unlike other search paths, the
current directory is always checked first when looking for files in
this path, as if '.' was the first component.\\
Default: {\vt ( .  /usr/local/xictools/xic/startup )}

\index{HlpPath keyword}
\item{\vt HlpPath} {\it path}\\
The {\vt HlpPath} lists directories containing database files for the
help system.  These files have names with suffix {\vt .hlp}, and it is
possible for users to create customized help files for their own
purposes (the format is described in \ref{helpfiles}).  The help path
can also be specified with the environment variable {\et
XIC\_HLP\_PATH}, which will be overridden by a specification in the
technology file.\\
Default: {\vt ( /usr/local/xictools/xic/help )}

\index{ScriptPath keyword}
\item{\vt ScriptPath} {\it path}\\
The {\vt ScriptPath} contains directories where {\Xic} searches for
user generated command scripts.  The script files have names with
suffix ``{\vt .scr}'', except for the library script which is named
``{\vt library}''.  This path can also be set with the environment
variable {\et XIC\_SCR\_PATH}, which will be overridden by a
specification in the technology file.\\
Default:{\vt ( /usr/local/xictools/xic/scripts )}.
\end{description}

\index{XIC\_LIB\_PATH}
Note that the {\vt XIC\_LIB\_PATH} variable can be used to define the
location of the technology file, and then redefined in the technology
file to provide alternate locations for the device and model library
files.

The path keywords, and all other keywords, are interpreted without
case sensitivity when the technology file is read.


% -----------------------------------------------------------------------------
% techfile:scripts 033112
\section{Technology File Scripts}
\index{technology file!scripts in}

Scripts can be included in the technology file.  These scripts can
appear as buttons in the {\cb User Menu}, as with other scripts, or
they can be ``run once'' scripts.  This feature is useful for
including simple technology-specific commands, such as those that
create special extraction layers or physical features.  Scripts
defined in the technology file, however, can not be loaded into the
debugger.

A script is included In the technology file as follows.  The {\vt
Script} keyword is followed by the text which will appear in the
command button.  If the button text contains white space, it must be
quoted, e.g.,

\begin{quote} \vt
Script "My Cell Counter"
\end{quote}

The lines of the script follow, and the script text must be
terminated with the keyword {\vt EndScript} on a separate line.

\begin{quote}
  {\vt Script} {\it menu\_label}\\
  script text\\
  ...\\
  {\vt EndScript}
\end{quote}

If the line
\begin{quote} \vt
RunScript
\end{quote}
appears anywhere after the {\vt Script} line and before {\vt
EndScript}, the script is taken as a ``run once'' script.  It will not
be added to the {\cb User Menu}.  Instead, it will be executed after
the technology file has been read, then discarded.  Any number of
scripts can be treated this way, they execute in order of appearance
in the technology file.

Scripts defined in the technology file have lower priority than other
scripts in the event of a menu label text clash.  Thus, technology
file scripts will be ``hidden'' by other scripts with the same menu
label, should any exist.


% -----------------------------------------------------------------------------
% techfile:layers 032117
\section{Technology File Layer Blocks}
\label{techlayers}

{\Xic} maintains a table of layer aliases, which can be used instead
of the actual layer name where a layer entry is required.  This
follows the Virtuoso ``{\vt techParams}'' definitions where the value
is a layer name.  The alias name is intended to be a generic name such
as ``{\vt active\_layer}'', or ``{\vt nwell\_layer}'', which can be
used in device blocks and elsewhere to provide a degree of process
independence.  Further, some of these names may be specific to
Virtuoso, and be handled in special ways.  The only example of this at
present is handling of ``{\vt active\_layer}''.

Each line of the layer alias list takes the following form.
\begin{quote}
{\vt MapLayer} {\it alias} {\it layer\_name}
\end{quote}
The {\it layer\_name} must be resolvable as an {\Xic} layer.

\begin{description}
\item{\vt active\_layer} alias handling\\
If a layer alias named ``{\vt active\_layer}'' is given, as is at
least one of the alias names ``{\vt ngate\_layer}'' and ``{\vt
pgate\_layer}'', and the {\vt active\_layer} does not have a {\vt
Conductor Exclude} directive, one will be created.  The excluded area
is logically {\vt ngate\_layer|pgate\_layer}.  This supports correct
MOS device recognition when technology data are obtained exclusively
from Virtuoso.
\end{description}

The component layer names and numbers, and purpose names and numbers,
are specified in optional tables.  These tables must appear before any
{\Xic} layer definitions.  A layer name or purpose name used by an
{\Xic} layer that is not found in a table will be created, and
assigned a number by {\Xic}.  The tables ensure a strict and
repeatable correspondence between names and numbers, which may be
necessary for compatibility with other tools.

\index{DefineLayer keyword}
\index{DefinePurpose keyword}
The tables consist of lines in the following form:
\begin{quote}
{\vt DefineLayer} {\it layer\_name} {\it layer\_number}\\
{\vt DefinePurpose} {\it purpose\_name} {\it purpose\_number}
\end{quote}

Name strings may contain alphanumeric characters plus the dollar sign
('{\vt \$}') and underscore ('{\vt \_}').  The numbers can be any
value representable with 32-bits, except that -1 is reserved.  Be
aware that other tools may define ranges of values that are reserved
for internal use.

\index{SCED layer}
Following the layer and purpose tables, if any, {\Xic} layers may be
defined.  There are separate definitions of layers used in electrical
(schematic) mode, and in physical mode (for layouts).  {\Xic}
maintains a standard set of electrical layers, in a standard order. 
These will be created if the definitions do not appear in the
technology file (or no technology file is read).  The SCED layer,
which is the electrically-active wiring layer, is always first.  The
user can modify the presentation attributes, and add layers as
desired.  For physical mode, there are no such layers, all layers must
be defined in some manner.

The separation of electrical and physical layers is a bit of an
anachronism, and in current {\Xic} releases a user-defined layer can
actually exist in both electrical and physical layer tables.  This
accommodates technologies imported from other tools, such as Cadence
Virtuoso, where no such distinction is made.

\index{PhysLayer keyword}
\index{ElecLayer keyword}
Each layer definition starts with the keyword {\vt PhysLayer} for
physical layers or {\vt ElecLayer} for electrical layers, followed by
a name.  Both of these keywords have synonyms (listed below) for
backwards compatibility.  The name should be a valid layer name,
though an attempt is made to use invalid names if possible by editing
out unacceptable characters.

Layer blocks appear in a contiguous section in the technology file,
and in physical mode will appear in the layer table in the order
given.  In electrical mode, reordering may be applied, as there are
some internal assumptions.

A layer block is terminated by the start of another layer block, or by
a keyword which would logically end per-layer parsing.

\begin{description}
\index{ElecLayer keyword}
\item{\vt ElecLayer} {\it name}\\
This keyword specifies the beginning of the layer block for the
electrical layer {\it name\/}.  The keyword {\vt ElecLayerName} is a
synonym.

\index{PhysLayer keyword}
\item{\vt PhysLayer} {\it name}\\
This keyword specifies the beginning of the layer block for the
physical layer {\it name\/}.  Layers will appear in the physical mode
layer table in the order given.  The keywords {\vt PhysLayerName},
{\vt Layer}, and {\vt LayerName} are all synonyms for this keyword.

\index{DerivedLayer keyword}
\item{\vt DerivedLayer} {\it name} [{\vt join|split|splitv}]
 {\it expression}\\
This line provides a definition of a derived layer.  Derived layers
represent an expression of other layers, derived and normal physical,
which can be referenced in layer expressions.  Derived layers were
introduced in support of the design rule checking system, but can be
accessed for other purposes through a script function interface.

This will add a derived layer to the database, under the name given in
the first token.  The remainder of the line is the layer expresion. 
The expression is not parsed until evaluation time.

When the derived layer is evaluated, the geometry can appear as an
assemblage of trapezoids if either of the {\vt split} or {\vt splitv}
keywords is given, or alternatively as a minimal number of complex
polygons if the {\vt join} keyword is given instead.  If {\vt splitv}
is given, a vertical orientation is favored for the decomposition,
whereas similarly {\vt split} will produce a decomposition favoring a
horizontal orientation.  The default is the joined form if none of
these optional keywords is given, except when simply copying from
another layer in which case the default is to copy objects without
change.  The keyword ``{\vt splith}'' is a synonym for ``{\vt
split}''.

These lines begin a layer block description, and any of the keywords
which can apply to physical layers can be used in the derived layer
blocks, though the definitions may be useless.  Layer block keywords
that are significant are listed below.

\begin{enumerate}
\item{Design rules.  These rules will be evaluated while doing design
rule checking.  As further described in the in the DRC description,
there are some types of tests that require use of derived layers.}

\item{When a new normal layer is created as a copy of a derived layer,
which can be done with the {\cb !layer} command, or with the {\cb
Layer Expression Evaluation} panel from the {\cb Edit Menu}, or with
the {\vt Layer} script function, the new layer will inherit the
attributes of the derived layer.  This includes color, fill pattern,
GDSII mpping, and other flags and properties.  This gives purpose to
the definitions provided in the derived layer block.}
\end{enumerate}
\end{description}

The sub-sections that follow categorize and describe the fairly
lengthly list of per-layer keywords.  All of the keywords are
optional, and can appear under an electrical or physical layer, unless
stated otherwise.  Many of these keywords can be programmed from
within {\Xic} with the {\cb Tech Parameter Editor} from the {\cb
Attributes Menu}.  Other panels from the {\cb Attributes Menu} allow
setting colors, fill patterns, etc.  which correspond to values from
keywords.

In the syntax descriptions, the italicized quantities represent data
the needs to be provided.  The ``{\vt y}$|${\vt n}'' symbol implies that
one of `{\vt y}' or `{\vt n}' should follow the keyword.  Actually,
`{\vt 0}' (zero), or any word that begins with the letters or sequence
(case insensitive) `{\vt n}', `{\vt f}', `{\vt of}' is taken as a
false value.  Anything else, including no following text, is taken as
true (`{\vt y}' is always redundant).


%-------------------------------
% techfile:layers:misc 070114
\subsection{Technology File Layer Block Keywords:  Misc. Attributes}
\label{misclattr}

These miscellaneous keywords apply bits of information to the layer,
which affects behavior in situations described.

\begin{description}
\index{LppName keyword}
\item{\vt LppName} {\it name}\\
This provides an optional alias for the layer/purpose pair that
represents the {\Xic} layer name.  The {\Xic} layer can be accessed by
this alias, in addition to the normal name.  If no non-space
characters are found after the keyword, the statement is ignored.  Any
character is allowed in the alias name, but leading and training white
space is removed, and inclusion of some characters, for example a
colon ('{\vt :}'), can definitely cause trouble.

\index{Description keyword}
\item{\vt Description} {\it description\_string}\\
This will set the description field of the current layer.  If no
non-space characters are found after the keyword, the statement is
ignored.  Leading and training white space is removed from the
description string.

\index{NoSelect keyword}
\item{\vt NoSelect} [{\vt y}$|${\vt n}]\\
If this keyword appears, and any following argument indicates true,
objects on the layer can not be selected.  The selectability status of
the layers can be changed from the layer table.

\index{NoMerge keyword}
\item{\vt NoMerge} [{\vt y}$|${\vt n}]\\
This keyword indicates that automatic merging of objects is suppressed
on the layer.  This overrides any merging enabled by the {\cb Merge
new boxes and polys with existing boxes/polys} and {\cb Clip and merge
new boxes only, not polys} check boxes in the {\cb Editing Setup}
panel from the {\cb Edit Menu}, and the {\cb Clip and merge
overlapping boxes} button in the {\cb Set Import Parameters} panel
from the {\cb Convert Menu}, and the corresponding variables.

\index{WireActive keyword}
\item{\vt WireActive} [{\vt y}$|${\vt n}]\\
If this keyword appears, and any following argument indicates true,
wires on the layer will be considered for wire connectivity in
schematics.  This flag is always set implicitly in the {\vt SCED}
layer.  The Cadence compatibility system may create a layer named {\vt
wire} with purpose {\vt drawing} which will have this flag set.

\index{Symbolic keyword}
\item{\vt Symbolic} [{\vt y}$|${\vt n}]\\
This keyword indicates that the layer will not be shown in the display
produced by the {\cb Cross Section} command (in the {\cb View Menu}). 
Otherwise, it doesn't have any purpose in {\Xic}, but might be useful
to the user as a flag to indicate a non-physical layer.

\index{Invalid keyword}
\item{\vt Invalid} [{\vt y}$|${\vt n}]\\
If this keyword appears, and any following argument indicates true,
the layer will not appear in the layer table, but will exist
internally and resolve any references to the layer in a design.  Such
layers are invisible, as the redisplay involves cycling through layers
in the layer table.

This is for compatibility with Cadence Virtuoso, whose layer
presentation attributes include a {\vt Valid} flag.  When reading a
Virtuoso technology file, if a layer is invisible, not selectable, and
is invalid, the {\Xic} {\vt Invalid} flag will be set.
\end{description}


%-------------------------------
% techfile:layers:pres 120314
\subsection{Technology File Layer Block Keywords:  Presentation}

These keywords impact the appearance of objects on the layer on-screen
and in prints.

\begin{description}
\index{RGB keyword}
\item{\vt RGB} {\it colorspec}\\
This keyword will set the color used to render objects on the layer
on-screen.  The {\it colorspec} string is the name of a color or an
RGB triple:
\begin{itemize}
\item{The name of a color.  The recognized names can be listed from
   the {\cb Set Color} pop-up in the {\cb Attributes} menu with the {\cb
   Colors} button.}
\item{Three space-separated numbers, each 0--255, representing the
   red, green, and blue intensity.  E.g., ``{\vt 196 240 235}''.}
\item{Other forms recognized by the {\vt XParseColor} C library
   function, including ``{\vt \#RRRRGGGGBBBB}'' and ``{\vt
   rgb:RRRR/GGGG/BBBB}''.  Here, {\vt R}, {\vt G}, and {\vt B} are
   single hexadecimal digits.}

\end{itemize}
If the color is given as a name, the color will be converted to its
RGB values if the file is updated.  If no {\vt RGB} keyword is given
for a layer, {\Xic} will assign a random color.  The {\vt RGB} keyword
is allowed in the mini-layer blocks found in the print driver
specifications.

\index{Filled keyword}
\item\parbox[b]{4in}{
{\vt Filled} [{\vt y}[...]]\\
{\vt Filled} {\vt n}[...] [{\vt o, f, c}]\\
{\vt Filled} {\it bit\_data\/} [{\vt o, f, c}]
}\\
This keyword sets the fill and outline style used to render objects on
the layer.  The tokens (other than {\it bit\_data}) can be words
starting with the indicated letters, or or just the letters
themselves, e.g., ``{\vt n}'', ``{\vt no}'', and ``{\vt none}'', are
equivalent.  This is case-insensitive.

If no tokens follow the keyword, or the first token starts with `{\vt
y}', solid fill will be used.  Additional tokens on the line will be
ignored.

If the first token starts with `{\vt n}', no fill pattern (empty fill)
will be used.  In this case, there are three outline styles available:
\begin{enumerate}
\item{A thin solid line boundary.}
\item{A thin dashed line boundary.}
\item{A thick solid line boundary for Manhattan boxes and polygons,    
and a thin solid line boundary for other objects.}
\end{enumerate}

There is also the ``cut'' attribute, where diagonal lines are drawn
over boxes, forming an X.  This applies to boxes only, not wires or
polygons, even though they may be rendered as four-sided rectangular
figures.

Any text that follows the word that started with `{\vt n}' is examined
for the presence of the characters `{\vt o}', `{\vt f}', and `{\vt
c}'.  These can be found as individual letters or parts of words, for
example ``{\vt outline cut}'' and ``{\vt oc}'' and ``{\vt o c}'' are
all equivalent.  In addition, this is all case-insensitive.

If neither `{\vt o}' or `{\vt f}' is found, a thin solid outline
(style 1) is used.  If `{\vt o}' is found but not `{\vt f}', a thin
dashed line (style 2) is used.  If `{\vt f}' is found, with or without
`{\vt o}', then a thick solid line is used for edge segments of
Manhattan objects, and a thin solid line is used for non-Manhattan
objects (style 3).

In any case, if `{\vt c}' is found, the ``cut'' attribute is applied. 
If `{\vt o}' is also found buf not `{\vt f}', the diagonals are shown
as dashed lines, the same as the boundary.  Otherwise, the diagonals
are always thin solid lines.

The form on the third line is used to specify a stipple pattern to
use for fill.  {\Xic} supports any stipple map size with the x and y
dimensions in the range of 2--32.  However, {\Xic} releases prior to
3.2.25 supported only 8x8, 8x16, 16x8, and 16x16 maps.  The format
described here is generally not backwards compatible with these
releases.

Maps can be read as hex numbers, or as ASCII tokens, but not in the
same line.  When {\Xic} writes a technology file, the default is to
use the ASCII token format, which actually renders the map in a crude
way.  This format is best illustrated by an example:

\begin{quote}
\begin{verbatim}
    Filled \
      |   ..   |  (0x18) \
      |  ....  |  (0x3c) \
      | ...... |  (0x7e) \
      |...  ...|  (0xe7) \
      |...  ...|  (0xe7) \
      | ...... |  (0x7e) \
      |  ....  |  (0x3c) \
      |   ..   |  (0x18) outline
\end{verbatim}
\end{quote}

The points to note here are the following.
\begin{enumerate}
\item{Line continuation is used so that the map is visible to a human
reader.  This is not required in general.}

\item{Each line of the map contains space and non-space characters,
surrounded by '{\vt |}' characters.  Although a period is used here,
any non-space printing character will work.}

\item{Each of these must contain the same number of characters, this
number being in the range 2--32.  This sets the width of the map.}

\item{The number of these constructs found in the line sets the height
of the map.  This must be in the range 2--32.}

\item{The map data parser ignores anything enclosed in parentheses. 
Above, the equivalent hex number for the data pattern is provided, but
is ignored by the parser.}
\end{enumerate}

An equivalent form using hex data is

\begin{quote}
    [{\vt x=}{\it width\/}] [{\vt y=}{\it height\/}] {\it hex\_number}
      {\it hex\_number} ...
\end{quote}

The {\it width} and {\it height} are decimal numbers in the range
2--32.  The number of hex digits that follow must match the {\it
height\/}.

The width and height specifications can be omitted, in which case the
format reverts to the pre-3.2.25 expectation.  The hex numbers must be
one of
\begin{itemize}
\item{8 2-digit hex numbers that specify an 8x8 map.}
\item{16 2-digit hex numbers that specify an 8x16 map.}
\item{8 4-digit hex numbers that specify a 16x8 map.}
\item{16 4-digit hex numbers that specify a 16x16 map.}
\end{itemize}

Additional text on the line is examined for the `{\vt o}', `{\vt f}',
and `{\vt c}' characters as described above for the no-fill case. 
With a fill pattern, the interpretation is slightly different, as
there is no dashed line outline available in this case.  If neither
`{\vt o}' or `{\vt f}' appear, the pattern will not be outlined.  If
`{\vt o}' appears without `{\vt f}', a thin solid outline will be
used.  If `{\vt f}' appears, edges of boxes and Manhattan polygons
will be thick.  The `{\vt c}' will draw diagonals on boxes.  For
historical reasons, the character `{\vt y}' is treated the same as
`{\vt o}'.

If the boolean variable {\et TechNoPrintPatMap} is set when {\Xic}
writes a technology file, then the hex form will be used to specify
fill patterns.  Otherwise, the ASCII form is used.

Here are a few more example fill specifications:\\
\begin{quote}
{\vt Filled y}\\
{\vt Filled no fat}\\
{\vt Filled cc aa cc aa cc aa cc aa outline}
\end{quote}

In electrical mode, the SCED layer defaults to solid fill, and other
layers default to empty fill with a thin outline.  All layers default
to empty fill with a thin outline in physical mode.  The {\vt Filled}
keyword is allowed in the mini-layer blocks found in the print driver
specifications.

\index{Invisible keyword}
\item{\vt Invisible} [{\vt y}$|${\vt n}]\\
If this keyword appears, and the following argument indicates true,
the layer will not be visible, though it will appear in the layer
table, where the visibility status can be changed.

The {\vt Invisible} keyword is allowed in the mini-layer blocks found
in the print driver specifications.  This is the only place where use
of the {\vt y}$|${\vt n} argument may be needed, in particular if {\vt
Invisible} is specified in the main layer block, {\vt Invisible n} may
be used in the driver block to make the layer visible in print driver
output.

\index{Blink keyword}
\item{\vt Blink} [{\vt y}$|${\vt n}]\\
If this keyword appears, the layer color will oscillate between two
shades with a 0.5 second period.  This is only supported in
pseudo-color (usually 256 colors) graphics mode.\\
Default: not blinking

\index{NoInstView keyword}
\item{\vt NoInstView} [{\vt y}$|${\vt n}]\\
If this keyword appears, and any following argument indicates true,
objects on the layer will not be shown in electrical instances of the
containing cell.  However they will appear when the cell is the
current cell.  This is ignored in physical node.

\index{WireWidth keyword}
\item{\vt WireWidth} {\it width}\\
This keyword can appear in physical layer fields.  The {\it width\/}
is a floating point number which sets the default wire width to that
value in microns.  This value will be used when wires are created in
{\Xic}.\\
Default: 0

\index{CrossThick keyword}
\item{\et CrossThick} {\it thickness}\\
This keyword, which can be applied to physical layers only, sets the
layer thickness as rendered in the {\cb Cross Section} command in the
{\cb View Menu}.  The {\it thickness} is given in microns.
\end{description}


%-------------------------------
% techfile:layers:conv 070114
\subsection{Technology File Layer Block Keywords:  Conversion}
\label{convlattr}

The following keywords set the layer mapping for GDSII and OASIS
format input and output.  These can be programmed from within {\Xic}
with the {\cb Tech Parameter Editor} in the {\cb Attributes Menu}. 

\begin{description}
\index{StreamData keyword}
\index{GDSII layer mapping}
\item{\vt StreamData} {\it layernum datatype}\\
This keyword is deprecated, and can be read but is not generated by
{\Xic}.  The {\it layernum\/} and {\it datatype\/} are the layer
mapping used when converting to and from GDSII format.  The layer must
be in the range 0 through 65535, and the datatype can take values -1
through 65535.  Values larger than 255 are outside of the GDSII
specification, but are sometimes used anyway although files containing
such data may not be generally portable.  If -1 is given as the
datatype, all GDSII datatypes will be mapped to the present {\Xic}
layer, and datatype 0 will be used for output.  Otherwise, the layer
and datatype in a GDSII file must match those given for successful
mapping to the {\Xic} layer.  Note that often the end of range values
are reserved in other CAD environments, and that some releases of the
GDSII format support only 64 layers and datatypes.  The datatype is
used by {\Xic} only in conjunction with the {\vt NoDrcDatatype}
keyword, and is otherwise typically set to 0.  This keyword has been
superseded by {\vt StreamIn} and {\vt StreamOut}.

\index{StreamIn keyword}
\item{\et StreamIn {\it layer\_list\/} [, {\it datatype\_list\/}]}\\
This keyword specifies a set of layer/datatype combinations that will
map to the present {\Xic} layer when reading GDSII and OASIS files. 
Any number of such lines can be present.  The {\it layer\_list\/} is a
space-separated list of tokens, each of which is either a GDSII layer
number (``32'') or a range of numbers (``35-41'').  The {\it
datatype\_list\/} is similarly constructed, and is optional.  The
numbers in either list can range from 0 to 65535, though numbers
larger than 255 are outside of the GDSII specification (but sometimes
used anyway).  If a {\it datatype\_list} appears, it is separated from
the {\it layer\_list} with a comma.  The line specifies that each of
the datatypes listed on each of the GDSII layers listed will be
converted to the present {\Xic} layer.  If the datatype list is
absent, it defaults to ``0-65535''.  For example,
\begin{quote}\vt
    StreamIn 5 7 8 21-30, 0 20-63
\end{quote}
specifies that datatypes 0 and 20-63 on GDSII layers 5, 7, 8, and
21-30 will be mapped to the present {\Xic} layer as a GDSII or OASIS
file is read.  Note that GDSII layers cam be mapped to more than one
{\Xic} layer.  In this case, the geometry will be created on each of
the {\Xic} layers mapped to.

\index{MultiMapOk variable}
It is possible for more than one {\Xic} layer to map from a given
GDSII layer/datatype.  If the {\et MultiMapOk} variable is set, then
multiple objects will be created when a GDSII or OASIS file is read,
one on each matching {\Xic} layer.  If this variable is not set, only
the first mapping will be used, which will be the lowest matching
layer found in the layer table.

\index{StreamOut keyword}
\item{\et StreamOut {\it out\_layer\/} [{\it out\_datatype\/}]}\\
This line specifies a layer/datatype combination to be used when
generating GDSII and OASIS files for the present {\Xic} layer.  One of
these should appear for each {\Xic} layer.  The {\it out\_layer} and
{\it out\_datatype} can be in the range 0--65535, though numbers
larger than 255 are outside of the GDSII specification but are
sometimes used anyway.  Be aware that use of numbers larger than 255
may render the file non-portable.  Note that often the end of range
values are reserved in other CAD environments, and that some releases
of the GDSII format support only 64 layers and datatypes.  The default
datatype, if not given, is 0.

If there are more than one {\et StreamOut} lines given for a layer,
and the {\et MultiMapOk} variable is set, the objects will be added to
the GDSII or OASIS file on each of the GDSII layers/datatypes
specified.  If the variable is not set, only the first {\et StreamOut}
specification will be used.

There is no default for this keyword.

\index{NoDrcDataType keyword}
\item{\et NoDrcDatatype} {\it datatype}\\
If this keyword is given, then any object that has the given datatype
will be ignored during DRC.  On output, objects that have their DRC
skip flags set will be written with this datatype, and not the default
datatype given in the {\et StreamOut} line.  The given datatype should
appear in the input mapping for the layer.
\end{description}


%-------------------------------
% techfile:layers:ext 062016
\subsection{Technology File Layer Block Keywords:  Extraction}
\label{exkwords}

This section describes the keyword entries which appear in layer
blocks which categorize the purpose of the layer for extraction. 
These define the conductor layers which are involved in grouping,
identify vias between conductors, etc.  These keywords can appear only
in physical layer fields.

All of these settings can be entered with the {\cb Edit Tech Params}
command in the {\cb Attributes Menu} and then written to disk with the
{\cb Save Tech} command in the {\cb Attributes Menu}, or be entered
with a text editor directly into the technology file.

Some of the keywords below use layer expressions, as were described in
\ref{layerexp}.  A layer expression in its simplest form is a layer
name.  More generally, it consists of an expression involving layer
names, the intersection operator ({\vt \&}), the union operator ({\vt
|}), and the inversion operator ({\vt !}).  Parentheses can be used to
enforce precedence.  These are the same type of expressions as used in
the DRC tests.  The expression is ``true'' at points where the
expression would return opacity.

\begin{description}
\item{\et Conductor} [{\et Exclude} {\it expression\/}]\\
\index{Conductor keyword}
This keyword indicates that the present layer is to be included in
conductor net grouping.  If the keyword {\et Exclude} and a following
layer expression are given, the regions of the current layer under
which the expression is true are clipped out for grouping purposes. 
For example, in CMOS technology a transistor is formed by a strip of
CAA (active area) bisected by a CPG (polysilicon) gate.  If ``{\vt
Conductor Exclude CPG}'' is given in the CAA layer block, the two
pieces of CAA will be given separate group numbers, which is necessary
to keep the transistor source and drain separate.

\item{\et Routing} [{it route params\/}]\\
This keyword implies that the layer is a conductor used for connecting
between cells.  The {\et Conductor} keyword is implied, so that the
{\et Conductor} keyword does not also have to be supplied, unless
there is an {\et Exclude} directive.  Only layers with the {\et
Routing} keyword given will be considered by the extraction system for
connecting between cells, and cell formal terminals will only be
assigned to {\et Routing} layers.  This is not absolute, however.  The
extraction system will place formal terminals on {\et Conductor}
layers under some circumstances, if necessary.

Optionally, routing parameter definitions may follow the keyword. 
These provide information to a third-party auto-route system, The
parameters are saved in the {\Xic} technology database, and are used
when writing a technology file, but are not otherwise used directly by
{\Xic}.  The recognized routing parameter definitions are listed
below.  These can appear in any order.  These parameters will be
parsed and set when reading the technology file, but can also be set
when reading Cadence ASCII technology files.

\begin{description}
\item{\vt dir=H|V|X|Y}[...]\\
This sets the preferred direction of routes on the layer.  The ``{\vt
dir=}'' is literal, and is followed by a letter or word, only the
first letter of which is significant.  If the first letter is {\vt H}
or {\vt X} (case insensitive), the route direction is horizontal.  If
the letter is {\vt V} or {\vt Y}, also case insensitive) the routing
direction is vertical.  Otherwise, an error ensues.

\item{\vt p}[{\vt itch}]{\vt =}{\it px\/}[{\vt ,}{\it py\/}]\\
This provides the values for the route pitch.  Only the first letter
of the ``{\vt pitch}'' keyword need be present.  This is followed by
an equal sign (``{\vt =}''), and one or two real numbers.  The numbers
are pitch values in microns.  If there are two numbers, the first is
the horizontal pitch, the second vertical, separated by a comma.  Two
numbers are required only if the horizontal and vertical pitch values
differ.

\item{\vt o}[{\vt ffset}]{\vt =}{\it ox\/}[{\vt ,}{\it oy\/}]\\
This provides values for the route offset, and is parsed the same way
as the pitch.  The values are real numbers giving the offset in
microns.  The second number can be omitted if it is the same as the
first.  The offset is the routing grid origin relative to the cell
origin.

\item{\vt w}[{\vt idth}]{\vt =}{\it w}\\
This specifies the line width, in microns, used for routing. 
Presently, only one number is accepted, implying that horizontal and
vertical routes have the same width.

\item{\vt maxd}[{\vt ist}]{\vt =}{\it d}\\
This provides a maximum route length, in microns.  A router may use
this value to limit route lengths.
\end{description}

\item\parbox[b]{4in}{\rr
{\et GroundPlane}\\
{\et GroundPlaneDark} (alias)\\
}\\
\index{GroundPlane keyword}
\index{GroundPlaneDark keyword}
\index{extraction!ground plane}
This keyword indicates that the present layer is to be treated as a
clear-field ground plane.  The layer is given the {\et Conductor}
attribute.  If the keyword ``{\vt Global}'' appears, then every object
on the layer will be assigned to the ground group 0.  This would be
appropriate if the layer represents a diffusion rather than a metallic
ground plane.  The default is to treat this level as a normal
conductor, except that when this layer is grouped in the top-level
cell, the group with the largest area is assigned to the ground group.

If ``{\vt Global}'' is given, the {\et GroundPlaneGlobal} variable,
which activates the mode, will be set.

Only one of the ground plane keywords can appear in the technology
file.  Conductor group 0 is used only if a ground plane has been
specified.  The ground plane layer can be referenced in {\et Via} and
{\et Contact} lines just as any {\et Conductor}.

\item\parbox[b]{4in}{\rr
{\et GroundPlaneClear} [{\et MultiNet} [0$|$1$|$2]]\\
{\et TermDefault} [{\et MultiNet} [0$|$1$|$2]] (alias)\\
}\\
\index{GroundPlaneClear keyword}
\index{TermDefault keyword}
\index{ MultiNet keyword}
This keyword indicates that the present layer is to be treated as a
dark-field ground plane.  These keywords imply {\et DarkField}. 
Giving {\et GroundPlane} (or {\et GroundPlaneDark}) and {\et
DarkField} is equivalent to {\et GroundPlaneClear} without {\et
MultiNet}.

Only one of the ground plane keywords can appear in the technology
file.  Conductor group 0 is used only if a ground plane has been
specified.

Without the {\et MultiNet} keyword, connections to this layer (as
specified with the {\et Via} and {\et Contact} keywords), where this
layer does {\it not} appear, are considered as connections to ground
(group 0).  Although this approach may work for simple cells, it can
lead to trouble.  Suppose that an island of ground plane metal is used
as part of the metalization for the chip pads.  This would appear as a
hole in the displayed representation of the ground plane layer.  Then
each pad will be extracted as shorted to ground!

\index{GroundPlaneMulti variable}
There is provision for more intelligent handling of the {\et
GroundPlaneClear} layer, allowing the layer to be included in paths
and groups.  If the {\et MultiNet} keyword appears, the inverse of the
layer is computed, and that (temporary) layer is used in the grouping. 
However, it can take quite a lot of behind-the-scenes computation if
the {\et GroundPlaneClear} layer has complex patterning.  Inversion is
also done if the {\cb !set} variable {\et GroundPlaneMulti} is given
(note:  this variable was formerly named {\et HandleTermDefault}). 
The temporary layer is treated as a clear-field ground plane, and all
references to the ground plane will be applied to the temporary layer
during grouping and extraction.

The name of the internal layer created is ``{\vt \$GPI}''.  By default,
this layer is invisible.  It should not be directly edited by the
user.  The inverse layer is an internal layer and is never written to
a file during conversion or a save.  During extraction the {\et
GroundPlaneClear} layer is ignored, and the inverse, which is a {\et
Conductor}, is used to establish connectivity.

\index{GroundPlaneMethod variable}
To establish connectivity for the commands in the {\cb Extract Menu},
the inverse layer is created according to one of the algorithms
described below.  An optional integer 0--2 may follow the {\et
MultiNet} keyword, which indicates the algorithm used for inversion. 
The algorithm can also be selected by setting the variable {\et
GroundPlaneMethod} to an integer in the same range, with the {\cb
!set} command.

\begin{description}
\item{\bf 0}
The inverted layer is created for each cell in the hierarchy by
computing
\begin{quote}
{\vt \$GPI = !{\it GP} \& !\$\$}
\end{quote}
i.e., for each cell the ground plane is inverted and the areas over
subcells are removed (recall that ``{\vt \$\$}'' is a pseudo-layer
representing subcell boundaries).  This is the default.
\item{\bf 1}
The inverted layer is created only in the top cell in the hierarchy,
and is the inverse of a flat representation of the ground plane layer
from all cells in the hierarchy.  The extraction algorithm will add
virtual contacts from this layer to the appropriate places in the
subcells.
\item{\bf 2}
The inverted layer is created in each cell of the hierarchy by
creating a flat inverse of all of the ground plane found in the cell
or lower in the hierarchy.
\end{description}

The default (0) method is the most efficient computationally, but the
method will probably fail if sibling subcells overlap.  In general, it
is good practice to avoid cell overlap.

Method 1 will work if subcells overlap.  However, since there is no
local ground plane in the subcells, generating a netlist while in a
{\cb Push} (subedit) will not yield correct results.

Method 2 is the least efficient computationally, but each cell has a
local ground plane.

\item{\et Via {\it layer1 layer2\/} [{\it expression\/}]}\\
\index{Via keyword}
\index{via expression}
This keyword indicates that the present layer may provide connection
points between conductor nets on {\it layer1\/} and {\it layer2\/}. 
The {\it layer1\/} and {\it layer2\/} are names of layers each of
which have the {\et Conductor}, {\et Routing}, or one of the
GroundPlane keywords specified.  In extraction, it is assumed that the
via is formed by dark area on the present layer, and vias are
completely covered by {\it layer1\/} and {\it layer2\/}.  A connection
is indicated if the {\it expression} (which is a layer expression) is
true at any point within the via.  The {\et Via} keyword implicitly
assigns {\et DarkField}.  The recognition logic is as follows:

\begin{quote}
{\bf for each} {\it region} of the {\et Via} layer \{\\
\hspace*{2mm}{\bf if} (there exists an object on {\it layer1} that overlaps {\it region})\\
\hspace*{4mm}{\bf if} (there exists an object on {\it layer2} that overlaps {\it region})\\
\hspace*{6mm}{\bf if} (there is no {\it expression}, {\bf or} the area where {\it expression} is true in {\it region} is nonzero)\\
\hspace*{8mm}{\bf then} the via indicates a connection between the two objects\\
\}
\end{quote}

If the {\it expression} is not given, it is always taken as ``true''.

Examples:
\begin{description}
\item{\vt Via M1 M2 !RES}\\
A via is indicated if part of the via object on the present layer
which is being evaluated is not covered by objects on {\vt RES}.

\item{\vt Via M1 M2 I2}\\
A via is indicated if the via object on the present layer is
partially or completely covered with {\vt I2}.

\item{\vt Via M1 M2 (!I2)\&(!RES)}\\
A via is indicated if part of the via object is not covered by {\vt I2}
or by {\vt RES}.
\end{description}

\item{\et Dielectric}\\
\index{Dielectric keyword}
This keyword is intended to specify an explicit capacitor dielectric,
which is different from a {\vt Via} layer.  A layer can not have both
keywords.  This is primarily to support the capacitance extraction
interface.  A {\vt Dielectric} layer is assumed to be clear-field,
unlike {\vt Via} layers, though the {\vt DarkField} keyword can also
be applied.  Also unlike {\vt Via} layers, {\vt Dielectric} layers are
not assumed to be planarizing by default.

\item{\et Contact} {\it layer} [{\it expression\/}]\\
\index{Contact keyword}
This keyword specifies that the present layer may be in contact with
{\it layer}, which has the {\et Conductor} attribute, and is to be
grouped accordingly in the wire net extraction.  The {\it expression}
(which is a layer expression), if given, must be true in the overlap
region between the object and the objects on {\it layer} for contact
to be established.

The purpose is to account for a contact metalization which is applied
over the normal wiring layers, which may itself be used for making
connections occasionally.  The {\et Contact} keyword implies {\et
Conductor}.  The {\et Contact} keyword should be given in the layer
block of the contact metal layer.  It is not necessary (or desirable)
to include a reciprocal {\et Contact} specification in the referenced
layer's block.

\item{\et DarkField}\\
\index{DarkField keyword}
This keyword indicates that the layer polarity on the chip is the
reverse of that shown on-screen.  This is usually the case for via
layers, for example, which are rendered as small squares to indicate
the contact location, which is actually a hole in an insulating layer. 
At present, the only command that uses this keyword is the {\cb Cross
Section} command in the {\cb View Menu}.  Layers with the keyword
applied will be shown as on-chip in the cross sectional view.
This keyword is implicitly assigned by both {\et Via} and
{\et GroundPlaneClear}.

The keyword has a secondary effect if used in conjunction with the
{\et GroundPlane} (or the equivalent {\et GroundPlaneDark}) keyword. 
The combination is equivalent to {\et GroundPlaneClear}.
\end{description}


%-------------------------------
% techfile:layers:phys 090714
\subsection{Technology File Layer Block Keywords:  Physical Properties}
\label{phkwords}

The following keywords can appear only in physical layer fields, and
they mostly specify physical material properties, or electrical
parameters, used in various ways by the extraction system.

Many of these parameters are redundant or incompatible with each
other.  Warning messages may be issued when incompatibilities are
detected, however unused information is usually simply ignored and
does no real harm.  In particular, there are two basic groups, those
keywords that apply to conductors, and those that apply to insulators. 
Mixing these parameters on the same layer will likely generate a
warning.

All of these settings can be entered with the {\cb Edit Tech Params}
command in the {\cb Attributes Menu} and then written to disk with the
{\cb Save Tech} command in the {\cb Attributes Menu}, or be entered
with a text editor directly into the technology file.

\begin{description}
\item{\et Planarize} [{\vt y}$|${\vt n}]\\
\index{Planarize keyword}
This specifies whether or not a layer is planarizing.  This is used by
the three-dimensional layer sequence generator when creating layer
sequences for the capacitance extraction interface.  The {\vt
Planarize} keyword can be applied to prevent planarization of layers
that are planarized by default, or to force planarizing of layers that
don't normally have this property.  See the description of the
sequence generator in \ref{ldb3d} for a description of planarization,
and which layers are planarized by default.

\item{\et Thickness {\it thickness}}\\
\index{Thickness keyword}
This keyword supplies the film thickness of the corresponding
deposited film.  The {\it thickness} is given in microns.  This can be
applied to any physical layer.
\end{description}

At most one of the following two keywords ({\et Rho} and {\et Sigma})
should be used.

\begin{description}
\item{\et Rho {\it resistivity}}\\
\index{Rho keyword}
This keyword supplies the resistivity, in MKS units (ohm-meters), of
the corresponding conducting film.  If {\et Rsh} (below) and {\et
Thickness} are both given, then the resistivity is already available
and this keyword is redundant.  Supplying this keyword overrides the
{\et Rsh}*{\et Thickness} value for the resistivity, when resistivity
is used explicitly in the extraction system (in the
inductance/resistance extraction interface).

\item{\et Sigma {\it conductivity}}\\
This keyword supplies the conductivity, in MKS units, of the
corresponding conducting film.  This is converted to resistivity
(1.0/{\it conductivity\/}) internally, i.e., it is equivalent to
giving {\et Rho}.

\item{\et Rsh {\it ohms\_per\_square}}\\
\index{Rho keyword}
The single parameter is a floating point number giving the ohms per
square value of the conducting material.  This is used in computation
of the resistance value of resistor devices.  If {\et Rho} or {\et
Sigma} is given, and also {\et Thickness}, then the sheet resistance
is already available and this keyword is redundant.  Supplying this
keyword overrides the {\et Rho}/{\et Thickness} value for sheet
resistance.

\item{\et EpsRel {\it diel\_constant}}\\
This keyword supplies the relative dielectric constant of insulating
layers.

\item{\et Capacitance {\it units\_per\_sq\_micron}
     [{\it units\_per\_micron}]}\\
\index{Capacitance keyword}
This enables computation of the capacitance of a conductor group on
the present conducting layer.  The first parameter is a floating point
number giving capacitance per square micron.  The optional second
parameter (default 0) is the edge capacitance, per micron.  The
extracted capacitance is the conductor group area multiplied by the
first parameter, plus the conductor group perimeter length multiplied
by the second parameter, if given.  The capacitance for each wire net
is computed during extraction, and will be printed (if enabled) in the
physical netlist output file.

The keyword ``{\et Cap}'' is accepted as an alias for ``{\et
Capacitance}''.

\item{\et Lambda {\it pene\_depth}}\\
This keyword specifies the London penetration depth of superconducting
conductors, in microns.  When {\et Lambda} is given, {\et Rho}/{\et
Sigma} (if given) represents the conductivity due to unpaired
electrons from the two-fluid model.

\item{\et Tline {\it grnd\_plane\_layer} [{\it diel\_thick diel\_const\/}]}\\
\index{Tline keyword}
This keyword will enable use of a microstrip model which computes
transmission line parameters.  A microstripline geometry is assumed,
with an object on the present layer forming a strip over an infinite
ground plane layer, separated by a homogeneous dielectric of constant
thickness.  No account is taken of ``real'' geometry, except for the
dimensions of the strip on the present layer.

The first argument is the name of a layer assumed as the ground plane. 
Both the present layer and the ground plane layer must be conductors
and have {\et Thickness} and, if superconductors, {\et Lambda}
defined.  Non-superconductors are treated as perfect conductors.

The second argument is the assumed height, in microns, of the
intervening dielectric.  The third argument is the relative dielectric
constant.  If either or both of these arguments is missing or given as
``0'' (zero), then {\Xic} will search for a layer with the {\et Via}
keyword set that contains the present and the ground plane layers, and
obtain the missing values from that layer.

\item{\et Antenna} {\it float\_value}\\
This keyword applies to the {\cb !antenna} command, and is meaningful
on conducting layers.  The {\it float\_value} is a threshold
antenna ratio, as explained for the {\cb !antenna} command.  The value
is effectively passed to that command as a default for the layer.
\end{description}


%-------------------------------
% techfile:layers:drc 090814
\subsection{Technology File Layer Block Keywords:  Design Rules}

The layer block may contain design rule specifications, which begin
with a keyword.  These keywords can appear only in physical layer
blocks.  See the description of the design rules in \ref{designrules}
for information regarding these keywords.  The rules can be programmed
from within {\Xic} with the {\cb Design Rule Editor}.  These keywords
are not recognized in the {\XicII} and {\Xiv} feature sets.


% -----------------------------------------------------------------------------
% techfile:stdvia 031815
\section{Technology File Standard Via Definitions}

{\Xic} provides support for OpenAccess/Virtuoso-style standard vias
(see \ref{stdvia}).  These definitions are imported from a Virtuoso
ASCII technology file when the {\vt ReadCdsTech} is used to source a
Cadence technology database, if any <tt>standardViaDefs</tt> nodes
exist.  They will be written to and read from the {\Xic} technology
file using syntax described in this section.

Standard via definitions will be written following the derived layers
when a new technology file is being created.  This is the recommended
location when hand editing a technology file.  The definitions are
required to follow the layer definition blocks of any layers used, but
otherwise location in the technology file is flexible.

The syntax for a standard via definition is as follows.

\begin{quote}
{\vt StandardVia} {\it viaName layer1 layer2 cutLayer cutWidth
cutHeight cutRows cutCols $\backslash$\\
\hspace*{2em}cutSpace\_x cutSpace\_y layer1Enc\_x layer1Enc\_y layer2Enc\_x
layer2Enc\_y $\backslash$\\
\hspace*{2em}layer1Off\_x layer1Off\_y layer2Off\_x layer2Off\_y
originOff\_x originOff\_y\/} $\backslash$\\
\hspace*{2em}$[${\it implant1 imp1Enc\_x imp1Enc\_y}
$[${\it implant2 imp2Enc\_x imp2Enc\_y\/}$]]$
\end{quote}

The terms correspond to the options shown in the {\cb Via Creation}
panel from the {\cb Edit Menu}, and their effects are described in
that section.  The definition must appear on a single logical line,
but backslash line continuation (as shown) can be employed to break
the line for improved human readability.

The line must begin with the {\vt StandardVia} keyword.  The remaining
tokens are as follows.  All of the numerical values can be altered by
the user before placement, the values provided in the definition are
the initial defaults.  The layer names, however, can not be changed
subsequently.  All dimensions are in microns.

\begin{description}
\item{\it viaName}\\
This is a unique name for the standard via, and can be any text word
that can be used as a cell name.  One convention is to use the layer
names of the twe conductors, top conductor first, separated by an
underscore (e.g., ``{\vt M2\_M1}'').

\item{\it layer1 layer2 cutlayer}\\
The three tokens that follow are the names of the bottom conductor,
the top conductor, and the via layer, in that order.  These layers
must have been already defined in the technology file.

\item{\it cutWidth cutHeight}\\
These are floating-point numbers giving the size of the cut in
microns.  The cut is always rectangular.

\item{\it cutRows cutCols}\\
These are integers not less than 1, which indicate that the cut should
be arrayed according to the numbers of rows and colums given.  These
are both usually 1 in a standard via definition, representing a
minimum via.  The user can array the cuts when necessary from the {\cb
Via Creation} panel.

\item{\it cutSpace\_x cutSpace\_y}\\
These apply when the cut is arrayed, and provide the edge-to-edge
space between cuts in the x and y direction.  This is usually a
minimum value given by a design rule.

\item{\it layer1Enc\_x layer1Enc\_y layer2Enc\_x layer2Enc\_y}\\
These four dimensions provide the enclosure distance for the bottom
(layer1) and top conductor rectangles relative to the cut.  The
enclosure is the distance that the metal rectangle extends outside of
the cut area.  This is usually a minimum value given by a design rule.

\item{\it layer1Off\_x layer1Off\_y layer2Off\_x layer2Off\_y}\\
These four dimensions provide offsets for the center of the two
conductor rectangles relative to the center of the cut.  These values
are unlikely to be other than zero.

\item{\it originOff\_x originOff\_y}\\
These coordinates provide the origin of the sub-master cell relative
to the center of the cut array.  It is the location that corresponds
to the mouse pointer when a new via instance is placed.  These are
unlikely to be other than zero.
\end{description}

All of the terms mentioned thus far are required.  The remaining terms
are optional.

\begin{description}
\item{\it implant1 imp1Enc\_x imp1Enc\_y}\\
This is the name of a layer followed by two dimensions.  If found, an
additional rectangle of {\it implant1} is centered over the {\it
layer1} (bottom conductor) rectangle.  The enclosure values specify
the distance the implant extends outside of the conductor, in the x
and y directions.

\item{\it implant2 imp2Enc\_x imp2Enc\_y}\\
These may follow an {\it implant1} group only.  This is the name of a
layer followed by two dimensions.  If found, an additional rectangle
of {\it implant2}> is centered over the {\it layer2} (top conductor)
rectangle.  The enclosure values specify the distance the implant
extends outside of the conductor, in the x and y directions.
\end{description}

Standard via definitions successfully read from the technology file
will be saved internally, and the definitions can be accessed from the
{\cb Via Creation} panel.  The panel allows the default values to be
overridden, and new vias to be created and placed.  If no stantard via
definitions were successfully read, the panel is unavailable and the
{\cb Create Via} button in the {\cb Edit Menu} is grayed.


% -----------------------------------------------------------------------------
% techfile:attr 021615
\section{Technology File Attributes}
\label{techattr}

The keywords described below appear (by convention) after the layer
specifications, and control various global attributes of {\Xic}. 
These are broken down into categories, which are presented in the
order in which they will be written to a new technology file created
by {\Xic}.  Actual order in the file is unimportant.  The categories
are:

\begin{description}
\item{\bf Grid Presentation}\\
Display options for the grid, which can be adjusted from within
{\Xic} in the {\cb Style} page of the {\cb Grid Setup} panel.

\item{\bf Misc. Presentation}\\
Other general display attributes that correspond to the entries in the
{\cb Main Window} sub-menu of the {\cb Attributes Menu}.

\item{\bf Attribute Colors}\\
Colors used for background, highlighting, etc.

\item{\bf Grid and Edge Snapping}\\
Parameters for grid spacing and edge snapping, which can be adjusted
from within {\Xic} in the {\cb Snapping} page of the {\cb Grid Setup}
panel.

\item{\bf Function Key Assignments}\\
Command mapping to keyboard function keys.

\item{\bf Grid Registers}\\
Saved grid register contents.

\item{\bf Layer Palette Registers}\\
Saved palette register contents.

\item{\bf Font Assignments}\\
Fonts used by the graphical user interface.

\item{\bf Keyword Variables}\\
Variable initialization as keywords.
\end{description}

Keywords listed in the first three categories ({\bf Grid
Presentation}, {\bf Misc.  Presentation}, and {\bf Attribute Colors}
can also appear in print driver blocks, in which case they are in
effect when printing with that driver.

In the syntax descriptions, the italicized quantities represent data
the needs to be provided.  The ``{\vt y}$|${\vt n}'' symbol implies that one
of `{\vt y}' or `{\vt n}' should follow the keyword.  Actually, `{\vt
0}' (zero), or any word that begins with the letters or sequence
(case insensitive) `{\vt n}', `{\vt f}', `{\vt of}' is taken as a
false value.  Anything else, including no following text, is taken as
true (`{\vt y}' is always redundant).


%-------------------------------
% techfile:attr:grid 021415
\subsection{Grid Presentation}

These keywords define the appearance of the axes and grid shown in the
drawing windows on program startup.  Within {\Xic}, the presentation
can be modified from the {\cb Style} page of the {\cb Grid Setup}
panels associated with the drawing windows.  The parameters given in
the technology file apply to the main window, which are inherited by
sub-windows when created.  The parameters can subsequently be changed
with the panel on a per-window basis.

For the main drawing window, the {\cb Main Window} sub-menu of the
{\cb Attributes Menu} provides the {\cb Set Grid} button, which brings
up the {\cb Grid Setup} panel.  Sub-windows have the {\cb Grid Setup}
panel available from the {\cb Attributes} menu in the sub-window. 
Pressing {\kb Ctrl-g} while a drawing window has focus will also bring
up the panel.

The keywords described in this section can also appear within print
driver blocks.  If they appear in a print driver block, the attribute
will apply on-screen when that driver is active in printing mode, and
in the printer output.

\begin{description}
\index{Axes keyword}
\item{\vt Axes} {\vt [Plain | Mark | None]}\\
This determines the presentation style for the axes in physical mode. 
The default is {\vt Mark}, where the origin is marked with a small
box.  If {\vt Plain} is given, the axes are simple lines.  If {\vt
None} is given, the axes will not be drawn.

\index{ShowGrid keyword}
\item{\vt ShowGrid} [{\vt y}$|${\vt n}]\\
This determines whether or not the grid will be shown by default, and
applies to both physical and electrical modes.\\
Default: {\vt y}

\index{PhysShowGrid keyword}
\index{ElecShowGrid keyword}
\item\parbox[b]{4in}{
{\vt ElecShowGrid} [{\vt y}$|${\vt n}]\\
{\vt PhysShowGrid} [{\vt y}$|${\vt n}]}\\
These keywords allow the grid display to be set independently for the
two modes.  The last {\vt ShowGrid} directive will have precedence for a
given mode.

\index{GridOnBottom keyword}
\item{\vt GridOnBottom} [{\vt y}$|${\vt n}]\\
This keyword determines whether the grid is shown on top of or below
the rendered objects.\\
Default: {\vt y}

\index{PhysGridOnBottom keyword}
\index{ElecGridOnBottom keyword}
\item\parbox[b]{4in}{
{\vt ElecGridOnBottom} [{\vt y}$|${\vt n}]\\
{\vt PhysGridOnBottom} [{\vt y}$|${\vt n}]}\\
These keywords allow the grid to be displayed above or below the
rendered objects independently for the two modes.  The last {\vt
GridOnBottom} directive will have precedence for a given mode.

\index{GridStyle keyword}
\item{\vt GridStyle} {\it style} [{\it xsize\/}]\\
This sets the style of grid to use in both electrical and physical
modes.  The style is a decimal of hex (with ``{\vt 0x}'' prefix)
integer whose binary pattern is used to replicate the grid lines.  A
value of 0 indicates a point grid, and -1 indicates solid grid lines. 
Other values are taken as a line pattern that is periodically
reproduced.  From the MSB, the pattern starts with the first set bit,
and continues through the LSB.

If the {\it style} value is 0, for a ``dots'' grid, a second integer
will be read if present.  This value can be 0--6, and represents the
number of pixels to light up around the central pixel in the four
compass directions.  The ``dots'' can appear as brighter dots or small
crosses, as set by this integer.  This integer is ignored if {\it
style} is nonzero, and is taken as 0 if absent.\\
Default: 0xcc (hex)

\index{PhysGridStyle keyword}
\index{ElecGridStyle keyword}
\item\parbox[b]{4in}{\vt
ElecGridStyle {\it style}\\
PhysGridStyle {\it style}}\\
These keywords allow the grid style to be set independently for
electrical and physical modes.  The last {\vt GridStyle} directive has
precedence for a given mode.

\index{CoarseGridMult keyword}
\item{\vt CoarseGridMult} {\it num}\\
This can be set to an integer 1--50, and specifies that coarse grid
lines will appear every {\it num} fine grid lines.  With value 1, the
grid will use the coarse grid color only.  This applies in both
electrical and physical modes.\\
Default: 5

\index{ElecCoarseGridMult keyword}
\index{PhysCoarseGridMult keyword}
\item\parbox[b]{4in}{{\vt ElecCoarseGridMult} {\it num}\\
{\vt PhysCoarseGridMult} {\it num}}\\
These provide the coarse grid multiplier independently for the two
modes.  The last {\vt CoarseGridMult} directive seen for a given mode
has precedence.
\end{description}


%-------------------------------
% techfile:attr:misc 021415
\subsection{Misc. Presentation}

These keywords set initial values for a number of display attributes. 
These generally apply to all drawing windows, but the values can be
reset on a per-window basis within {\Xic}.  For the main window, most
have corresponding toggle buttons in the {\cb Main Window} sub-menu of
the {\cb Attributes Menu}.  In sub-windows, the buttons are located
within the {\cb Attributes} menu itself.

The keywords described in this section can also appear within
print driver blocks.  If they appear in a print driver block, the
setting will apply on-screen when that driver is active in printing
mode, and in the printer output.

\begin{description}
\index{Expand keyword}
\item{\vt Expand} {\it num}\\
This keyword sets the initial expansion level for subcells, for both
electrical and physical modes.  If zero, no subcells are expanded.  If
-1, all subcells will be shown expanded.  A positive integer indicates
that subcells up to that depth will be shown expanded.\\
Default: 0

In {\Xic}, the {\cb Expand} pop-up controls expansion level, on a
per-window basis.  This panel is available from the {\cb Expand}
button in the main and sub-window {\cb View} menus.

\index{ElecExpand keyword}
\index{PhysExpand keyword}
\item\parbox[b]{4in}{\vt
ElecExpand {\it num}\\
PhysExpand {\it num}}\\
These forms allow the expansion level for electrical and physical
modes to be set separately.

\index{DisplayAllText keyword}
\item{\vt DisplayAllText} {\it num}\\
This keyword sets whether label text is displayed or not, for both
electrical and physical modes.  If {\it num} is 0, labels will not be
displayed.  If 1 (actually, any number not 0 or 2), labels will be
displayed in ``legible'' orientation.  If 2, labels will be shown in
true orientation, i.e., rotated and mirrored as placed and transformed
along with the containing instance.\\
Default: {\vt 1}

The {\cb Show Labels} and {\cb Label True Orient} buttons in the {\cb
Main Window} sub-menu of the {\cb Attributes Menu} and in the {\cb
Attributes} menu of sub-windows control these settings.

\index{ElecDisplayAllText keyword}
\index{PhysDisplayAllText keyword}
\item\parbox[b]{4in}{\vt
ElecDisplayAllText {\it num}\\
PhysDisplayAllText {\it num}}\\
These forms allow the display of label text for electrical and
physical modes to be set separately.

\index{ShowPhysProps keyword}
\item{\vt ShowPhysProps} [{\vt y}$|${\vt n}]\\
This keyword sets whether physical property strings are displayed
in physical mode.\\
Default: {\vt n}

The {\cb Show Phys Properties} button in the {\cb Main Window} sub-menu
of the {\cb Attributes Menu} and in the {\cb Attributes} menu of
sub-windows controls this setting.

\index{LabelAllInstances keyword}
\item{\vt LabelAllInstances} {\it num}\\
This keyword sets whether unexpanded instances are labeled or not, for
both electrical and physical modes.  If {\it num\/} is 0, instances
will not be labeled.  If 1, instances will be labeled, with the label
appearing either in horizontal or vertical orientation, whichever
provides the best fit into the cell bounding box.  If 2, the cell name
is rotated and mirrored along with the cell.\\
Default: 1

The {\cb Show Cell Names} and {\cb Cell Name True Orient} buttons in
the {\cb Main Window} sub-menu of the {\cb Attributes Menu} and in the
{\cb Attributes} menu of sub-windows control these settings.

\index{ElecLabelAllInstances keyword}
\index{PhysLabelAllInstances keyword}
\item\parbox[b]{4in}{\vt
ElecLabelAllInstances {\it num}\\
PhysLabelAllInstances {\it num}}\\
These forms allow the display of unexpanded instance text for
electrical and physical modes to be set separately.

\index{ShowContext keyword}
\item{\vt ShowContext} [{\vt y}$|${\vt n}]\\
When given `{\vt y}', the context surrounding a subcell is shown during
a sub-edit initiated with the {\cb Push} command in the {\cb Cell Menu}.
This applies to both electrical and physical modes.\\
Default: {\vt y}

The {\cb Show Context in Push} button in the {\cb Main Window}
sub-menu of the {\cb Attributes Menu} and in the {\cb Attributes} menu
of sub-windows controls this setting.

\index{ElecShowContext keyword}
\index{PhysShowContext keyword}
\item\parbox[b]{4in}{\vt
ElecShowContext {\it num}\\
PhysShowContext {\it num}}\\
These forms allow the display of editing context for electrical and
physical modes to be set separately.

\index{ShowTinyBB keyword}
\item{\vt ShowTinyBB} [{\vt y}$|${\vt n}]\\
If `y' is given, tiny subcells will be represented by their bounding
box.  Otherwise, these subcells will not be shown.  The size threshold
is given by the {\et CellThreshold} variable, set with the {\cb !set}
command.  This applies to both electrical and physical modes.\\
Default: {\vt y}

The {\cb Subthreshold Boxes} button in the {\cb Main Window} sub-menu
of the {\cb Attributes Menu} and in the {\cb Attributes} menu of
sub-windows controls this setting.

\index{ElecShowTinyBB keyword}
\index{PhysShowTinyBB keyword}
\item\parbox[b]{4in}{\vt
ElecShowTinyBB {\it num}\\
PhysShowTinyBB {\it num}}\\
These forms allow the tiny subcell rendering for electrical and
physical modes to be set separately.
\end{description}


%-------------------------------
% techfile:attr:color 070214
\subsection{Attribute Colors}
\label{attrcolor}

The following keywords set colors used on-screen and in printer
output.  All of these keywords take a {\it colorspec} string as the
argument list.  This is the name of a color or an RGB triple:

\begin{itemize}
\item{The name of a color.  The recognized names can be listed from
   the {\cb Set Color} pop-up in the {\cb Attributes} menu with the
   {\cb Colors} button.}
\item{Three space-separated numbers, each 0--255, representing the
   red, green, and blue intensity.  E.g., ``{\vt 196 240 235}''.}
\item{Other forms recognized by the {\vt XParseColor} C library
   function, including\\ ``{\vt \#RRRRGGGGBBBB}'' and ``{\vt
   rgb:RRRR/GGGG/BBBB}''.  Here, {\vt R}, {\vt G}, and {\vt B} are
   single hexadecimal digits.}
\end{itemize}

Following the general pattern for the technology file keywords, the
keyword form without the ``Phys'' or ``Elec'' prefix sets the color
for both modes.  The mode-specific keywords set the color only for
that mode.

A single internal data structure maintains all other attribute
(non-layer) colors.  All attribute colors can be set from the {\cb
Color Selection} panel provided by the {\cb Set Color} button in the
{\cb Attributes Menu}.  Attribute colors can also be changed with the
{\cb !setcolor} command.  In Unix/Linux, colors can be initialized
from a resource file (see \ref{resourcefile}), as well as from the
technology file.

When {\Xic} starts, the colors are set to default values.  Then, any
colors found in a resource file are updated.  Then, some of the colors
may be modified in the technology file.  Finally, the colors may be
changed in a {\vt .xicstart} file.

Below is the list of attribute colors, the defaults, and techfile
keywords and aliases.  The {\vt SelectColor}1/2 set the blinking
highlighting used for selected objects.  Setting both to the same
color stops the blinking.  The {\vt MarkerColor} is used for
electrical-mode terminal marks.  The {\cb Plot Mark} colors are used
only for the plot point indicators, and match the colors defined for
plots in {\WRspice}.

The {\bf Prompt Line Colors} apply tho the prompt line, status area,
coordinate readout, and main window keys-pressed area.  The {\vt
PromptBackgroundColor} controls the common background color, except
when the prompt line is in editing mode.  The other colors are
self-explanatory, with the \newline{\vt PromptHighlightingColor} being the
color used for hypertext entries (mostly for electrical mode).

The {\cb Special GUI Colors} are miscellaneous colors used for
highlighting and other purposes in the graphical user interface.

\begin{tabular}{|l|p{4in}|} \hline
\kb Variable & \kb Use\\ \hline
\vt GUIcolorDel & {\cb Cell Hierarchy Digests}, {\cb File Selection}, etc.\\
  \hline
\vt GUIcolorNo  & {\cb Empty Cells}, {\cb Modified Cells},
 {\cb Set Cell Flags}\\ \hline
\vt GUIcolorYes & {\cb Empty Cells}, {\cb Modified Cells},
 {\cb Set Cell Flags}\\ \hline
\vt GUIcolorHl1 & {\cb Script Debugger}, {\cb Design Rule Editor},
 {\cb Property Editor}\\ \hline
\vt GUIcolorHl2 & {\cb Modified Cells}, {\cb Property Editor},
 {\cb Cell Property Editor}\\ \hline
\vt GUIcolorHl3 & {\cb Modified Cells}\\ \hline
\vt GUIcolorHl4 & {\cb Design Rule Editor}, {\cb Tech Parameter Editor},
 {\cb Property Editor}, {\cb Cell Property Editor}\\ \hline
\vt GUIcolorDvBg & Pictorial device menu background\\ \hline
\vt GUIcolorDvFg & Pictorial device menu foreground\\ \hline
\vt GUIcolorDvHl & Pictorial device menu highlight\\ \hline
\vt GUIcolorDvSl & Pictorial device menu selection\\ \hline
\end{tabular}

The {\cb Attribute Colors} listed in the first block in the table
below can also be specified in printer driver blocks.  In this case,
the color will apply when that driver is selected in print mode, both
on-screen and in the hard-copy output generated by the driver.

\label{attrcolors}
\hspace*{-1.5em}\parbox[t]{3.0in} {
\begin{tabular}{|p{3.8cm}|l|} \hline
\kb Keyword{\newline}Alias & \kb Default\\ \hline
\multicolumn{2}{|l|}{\kb Attribute Colors}\\ \hline
\vt GhostColor & white\\ \hline
\vt ElecGhostColor & \vt GhostColor\\ \hline
\vt PhysGhostColor & \vt GhostColor\\ \hline
\vt HighlightingColor{\newline}Highlighting & white\\ \hline
\vt ElecHighlightingColor{\newline}ElecHighlighting &
  \vt HighlightingColor\\ \hline
\vt PhysHighlightingColor{\newline}PhysHighlighting &
  \vt HighlightingColor\\ \hline
\vt SelectColor1 & white\\ \hline
\vt ElecSelectColor1 & \vt SelectColor1\\ \hline
\vt PhysSelectColor1 & \vt SelectColor1\\ \hline
\vt SelectColor2 & pink\\ \hline
\vt ElecSelectColor2 & \vt SelectColor2\\ \hline
\vt PhysSelectColor2 & \vt SelectColor2\\ \hline
\vt MarkerColor & yellow\\ \hline
\vt ElecMarkerColor & \vt MarkerColor\\ \hline
\vt PhysMarkerColor & \vt MarkerColor\\ \hline
\vt InstanceBBColor{\newline}InstanceBB{\newline}InstanceBox & turquoise\\
  \hline
\vt ElecInstanceBBColor{\newline}ElecInstanceBB{\newline}ElecInstanceBox &
  \vt InstanceBBColor\\ \hline
\vt PhysInstanceBBColor{\newline}PhysInstanceBB{\newline}PhysInstanceBox &
  \vt InstanceBBColor\\ \hline
\vt InstanceNameColor{\newline}InstanceName & pink\\ \hline
\vt ElecInstanceNameColor{\newline}ElecInstanceName &
  \vt InstanceNameColor\\ \hline
\vt PhysInstanceNameColor{\newline}PhysInstanceName &
  \vt InstanceNameColor\\ \hline
\vt InstanceSizeColor{\newline}InstanceSize & salmon\\ \hline
\vt CoarseGridColor{\newline}CoarseGrid & sky blue\\ \hline
\vt ElecCoarseGridColor{\newline}ElecCoarseGrid & \vt CoarseGridColor\\ \hline
\vt PhysCoarseGridColor{\newline}PhysCoarseGrid & \vt CoarseGridColor\\ \hline
\vt FineGridColor{\newline}FineGrid & royal blue\\ \hline
\vt ElecFineGridColor{\newline}ElecFineGrid & \vt FineGridColor\\ \hline
\vt PhysFineGridColor{\newline}PhysFineGrid & \vt FineGridColor\\ \hline
\end{tabular}
}
\hspace{.5cm}
\parbox[t]{2.5in} {
\begin{tabular}{|p{4.0cm}|l|} \hline
\kb Keyword{\newline}Alias & \kb Default\\ \hline
\multicolumn{2}{|l|}{\kb Prompt Line Colors}\\ \hline
\vt PromptTextColor{\newline}PromptText & sienna\\ \hline
\vt PromptEditTextColor{\newline}PromptEditText & black\\ \hline
\vt PromptHighlightColor{\newline}PromptHighlight & red\\ \hline
\vt PromptCursorColor{\newline}PromptCursor & blue\\ \hline
\vt PromptBackgroundColor{\newline}PromptBackground & gray92\\ \hline
\vt PromptEditBackgColor{\newline}PromptEditBackg
  {\newline}PromptEditBackground & gray96\\ \hline
\vt PromptEditFocusBackgColor{\newline}PromptEditFocusBackg
  {\newline}PromptEditFocusBackground & gray100\\ \hline
\multicolumn{2}{|l|}{\kb Plot Mark Colors}\\ \hline
\vt Color2 & red\\ \hline
\vt Color3 & lime green\\ \hline
\vt Color4 & blue\\ \hline
\vt Color5 & orange\\ \hline
\vt Color6 & magenta\\ \hline
\vt Color7 & turquoise\\ \hline
\vt Color8 & sienna\\ \hline
\vt Color9 & gray\\ \hline
\vt Color10 & hot pink\\ \hline
\vt Color11 & slate blue\\ \hline
\vt Color12 & spring green\\ \hline
\vt Color13 & cadet blue\\ \hline
\vt Color14 & pink\\ \hline
\vt Color15 & indian red\\ \hline
\vt Color16 & chartreuse\\ \hline
\vt Color17 & khaki\\ \hline
\vt Color18 & dark salmon\\ \hline
\vt Color19 & rosy brown\\ \hline
\multicolumn{2}{|l|}{\kb Special GUI Colors}\\ \hline
\vt GUIcolorSel & \#e1e1ff\\ \hline
\vt GUIcolorNo & red\\ \hline
\vt GUIcolorYes & green3\\ \hline
\vt GUIcolorHl1 & red\\ \hline
\vt GUIcolorHl2 & darkblue\\ \hline
\vt GUIcolorHl3 & darkviolet\\ \hline
\vt GUIcolorHl4 & sienna\\ \hline
\vt GUIcolorDvBg & gray90\\ \hline
\vt GUIcolorDvFg & black\\ \hline
\vt GUIcolorDvHl & blue\\ \hline
\vt GUIcolorDvSl & gray80\\ \hline
\end{tabular}
}


%-------------------------------
% techfile:attr:snap 100916
\subsection{Grid and Edge Snapping}

These keywords define the grid and edge snapping parameters.  These
can be reset from within {\Xic} from the {\cb Snapping} page of the
{\cb Grid Setup} panels associated with the drawing windows.  The
parameters given in the technology file apply to the main window,
which are inherited by sub-windows when created.  The parameters can
subsequently be changed with the panel on a per-window basis.

For the main drawing window, the {\cb Main Window} sub-menu of the
{\cb Attributes Menu} provides the {\cb Set Grid} button, which brings
up the {\cb Grid Setup} panel.  Sub-windows have the {\cb Grid Setup}
panel available from the {\cb Attributes} menu in the sub-window. 
Pressing {\kb Ctrl-g} while a drawing window has focus will also bring
up the panel.

\begin{description}
\index{MfgGrid keyword}
\item{\vt MfgGrid} {\it delta}\\
If set nonzero, the actual {\vt SnapGridSpacing} used will be
constrained to be a multiple of this value.  This applies in physical
mode only.

This can be considered to be the ``pixel'' size of the mask.  The {\et
SpotSize} variable is related, see this topic in \ref{spotsize} for
more information.

\index{SnapGridSpacing keyword}
\item{\vt SnapGridSpacing} {\it spacing}\\
The {\it spacing} is a floating point number which represents the
spacing, in microns, between snap points.  This applies to physical
mode only.\\
Default: 1.0 microns    

The electrical grid is set to spacing value 1.0 with unit snap per
grid on program startup, which can't be changed from the technology
file.  The electrical grid can be changed within {\Xic} from the {\cb
Grid Parameters} pop-up, in the unusual circumstance that non-default
values are needed.

\index{SnapPerGrid keyword}
\index{GridPerSnap keyword}
\item\parbox[b]{4in}{{\vt SnapPerGrid} {\it num}\\
{\vt GridPerSnap} {\it num}}\\
At most one of these keywords should be given.  The {\it num} is an
integer 1--10.  These apply to physical mode only.

If {\vt SnapPerGrid} is given, then the fine grid lines will be spaced
{\it num\/}{\vt *SnapGridSpacing} apart.  If {\vt GridPerSnap} is
given, fine grid lines will appear at {\vt SnapGridSpacing/}{\it num}
intervals.\\
Default: 1

\index{EdgeSnapping keyword}
\item\parbox[b]{6in}{{\vt EdgeSnapping} [{\vt none}$|${\vt some}$|${\vt all}]
  [{\vt +}$|${\vt -off\_grid}]      [{\vt +}$|${\vt -non\_manh}]
  [{\vt +}$|${\vt -edge\_of\_wire}] [{\vt +}$|${\vt -path\_of\_wire}]}\\
This keyword sets the initial state of the controls of the {\cb Edge
Snapping} group in the {\cb Snapping} page of the main window {\cb
Grid Setup} panel.  All fields are optional, with the effective
default being

\begin{quote} \vt
 EdgeSnapping some -off\_grid -non\_manh +edge\_of\_wires -path\_of\_wires
\end{quote}

Only the first letter of the keywords is needed, and recognition is
case-insensitive.

The first word specifies when edge snapping is enabled:
\begin{quote}
\begin{tabular}{ll}
\vt none & not enabled\\
\vt some & enabled in some commands (the default)\\
\vt all  & always enabled\\
\end{tabular}
\end{quote}

The remaining items are flags that must start with a {\vt +} or {\vt
-} character.  The {\vt +} turns the option on, the {\vt -} turns the
option off.  These have obvious correspondence to the check boxes in
the {\cb Edge Snapping} control group in the {\cb Grid Setup} panel,
and set the initial state of the check boxes for the main window.

\index{RulerEdgeSnapping keyword}
\item\parbox[b]{6.5in}{{\vt RulerEdgeSnapping}
  [{\vt none}$|${\vt some}$|${\vt all}]
  [{\vt +}$|${\vt -off\_grid}]      [{\vt +}$|${\vt -non\_manh}]
  [{\vt +}$|${\vt -edge\_of\_wire}] [{\vt +}$|${\vt -path\_of\_wire}]}\\
This keyword uses the same syntax as the {\vt EdgeSnapping} keyword,
and species the initial edge snapping mode when the {\cb Rulers}
command in the {\cb View Menu} is in effect.  This command has its own
settings, with the default being that all flags are enabled.

\index{RulerSnapToGrid keyword}
\item{\vt RulerSnapToGrid} [{\vt y}$|${\vt n}]\\
This boolean keyword specifies the initial state of grid snapping in
the {\cb Rulers} command in the {\cb View Menu} In the command, the
mode can be toggled by presing the period (`{\vt .}') key.  By
default, grid snapping is asserted.
\end{description}


%-------------------------------
% techfile:attr:fkey 070214
\subsection{Function Key Assignments}
\index{function keys}

It is possible to map the keyboard function keys to {\Xic} operations. 
The function key assignments are sensitive to the {\kb Shift}, {\kb
Control}, and {\kb Alt} keys.  This means that a function key ({\kb
F1} - {\kb F12}) press can have different effects depending on the
state of these keys.

\begin{description}
\index{function keys assignment}
\index{F{\it N\/} keyword}
\item{\vt F{\it N\/}Key {\it text}}\\
The {\it N} is an integer in the range 1--12, to correspond to the
{\kb F1} -- {\kb F12} function keys found on most keyboards.

The {\it text} has the form
\begin{quote}
[{\vt <}{\it tok\/}{\vt >}] {\it cmd}
[{\vt <}{\vt tok\/}{\vt >} {\it cmd\/}] ...
\end{quote}

Each {\it tok} is a combination of the letters {\vt s}, {\vt c}, and
{\vt a}.  The presence of the letters indicates that {\kb Shift}, {\kb
Control} and {\kb Alt} are pressed, respectively.  The {\it tok} is
surrounded by angle brackets.

Examples:
\begin{description}
\item{\vt <s>}\\
The {\kb Shift} key is pressed, {\kb Control} and {\kb Alt} are not
pressed.
\item{\vt <ca>}\\
The {\kb Control} and {\kb Alt} keys are pressed, the {\kb Shift} key
is not pressed.
\end{description}

These tokens are followed by a {\it cmd}, which is a command.  If the
command starts with `{\vt !}', the remainder is treated as a ``bang''
command (see \ref{bangcmds}).  Otherwise, the text is the
five-character (or fewer) command keyword associated with GUI command
buttons.  If the {\it cmd} contains white space, it must be quoted.

The command keywords are displayed in the pop-up ``tooltip'' which
appears when the mouse pointer is positioned over a command button,
after a short delay.  This is the internal name for the command, which
is generally a short mnemonic of five characters or fewer.  The
keywords are also generally provided in the help system topic
describing the command.  In the {\cb User Menu}, for user scripts, the
name which appears on the menu button is the appropriate name to use.

The first {\it tok} is generally absent, and the {\it cmd} applies to
the function key with no modifiers pressed.

Example:
\begin{quote}
\vt F1Key box <s> "!exec /path/to/myscript.scr" <c> !!Clear(0) <sca> polyg
\end{quote}

The terms are:
\begin{description}
\item{\vt F1Key}\\
We're setting the {\kb F1} key in this example.

\item{\vt box}\\
This indicates that when {\kb F1} is pressed without pressing a
modifier key, the ``{\vt box}'' command from the side menu wil be
started.

\item{\vt <s> "!exec /path/to/myscript.scr"}\\
This specifies that when {\kb Shift-F1} is pressed, the script in the
{\vt myscript.scr} file will be executed, using the {\cb !exec} bang
command.  Since the command contains a space character, it is quoted.

\item{\vt <c> !!Clear(0)}\\
This specifies that when {\kb Control-F1} is pressed, the {\vt Clear()}
script function is called with argument 0.  This will clear the
database.  Note that the single/double exclamation point syntax is the
same as is accepted on the command line.

\item{\vt <sca> polyg}\\
When {\kb Shift}, {\kb Control}, and {\kb Alt} are pressed along with
{\kb F1}, the ``{\vt polyg}'' polygon creation command from the side
menu is started.
\end{description}

The menu containing the named button must be active (not grayed) for
the function key to have effect.  The mappings are completely defined
by the user --- there are no defaults.  Pressing an unmapped function
key has no effect on {\Xic}.  Be aware that the window manager in use,
and the GTK toolkit, may map functions keys, and this may have higher
priority than the mapping assigned here.  The use of the {\kb Alt} key
is generally not a good idea, as it is commonly assigned for other
purposes.  Sometimes, an assignment will simply be ignored for some
reason.  For example, on one system {\kb Control-F1} is never
returned, but {\kb Control} works fine with other function keys. 
\end{description}


%-------------------------------
% techfile:attr:gridreg 070214
\subsection{Grid Registers}
\index{grid registers}

The grid registers from the {\cb Grid Setup} pop-up are saved in
the technology file if they contain a non-default grid.

\begin{description}
\index{ElecGridReg keyword}
\index{PhysGridReg keyword}
\item\parbox[b]{4in}{{\vt ElecGridReg}{\it N} {\it spec}\\
 {\vt PhysGridReg}{\it N} {\it spec}}\\
The {\it N} is an integer value in the range 1--7.  Each register
index can store both a physical and electrical grid specification. 
The specifications define the contents of the grid registers,
available in the {\cb Grid Setup} panel and elsewhere.

The {\it spec} string has the form:
\begin{quote}
{\it snapspace} {\it snapval} {\it linestyle} [{\it xsize\/}]
[{\vt -a} {\it axes\/}] [{\vt -d} {\it dsp\/}]
[{\vt -t} {\it ontop\/}] [{\vt -m} {\it cmult\/}]
\end{quote}

The first three tokens are mandatory, and must appear in the order
shown.

\begin{description}
\item{\it snapspace} (real number)\\
The spacing between snap points, in microns.

\item{\it snapval} (integer, -10 through 10 excluding 0)\\
If the value is positive, it sets the number of snap points per fine
grid line.  For example, a value of 3 would indicate that a fine grid
line is drawn at every third snap point.  If negative, this sets the
number of fine grid lines per snap interval.  In this case, a value of
three indicates that fine grid lines appear at snap points and at the
1/3 and 2/3 proportional distances within the snap interval.

\item{\it linestyle} (integer)\\
This is the line style code.  The value is 0 for a dot grid, otherwise
the bit pattern represents the line dashes, as for the {\vt GridStyle}
keyword.
\end{description}

The remaining tokens are optional, and can follow the first three in
any order.

\begin{description}
\item{\it xsize} (integer 0--6)\\
If the {\it linestyle} code is 0 (for a dot grid), then a fourth
number can appear.  This is an integer 0--6 which indicates the number
of pixels in the four orthogonal directions to extend the dot into a
cross.

\item{\vt -a} {\it axes} (integer 0--2)\\
This sets the axes presentation mode in physical mode.  If 0, axes
aren't shown.  If 1, plain axes are shown, and if 2, axes are shown
and the origin decorated.

\item{\vt -d} {\it dsp} (boolean)\\
This sets whether the grid is displayed or not.  The {\it dsp} token
can be about any alphanumeric token that by convention indicates true
or false.

\item{\vt -t} {\it ontop} (boolean)\\
This sets whether the grid is displayed after all geometry (``on
top'') or before geometry.  The {\it ontop} token can be about any
alphanumeric token that by convention indicates true or false.

\item{\vt -m} {\it cmult} (integer 1--50)\\
This sets the number of fine grid lines per coarse grid line.
\end{description}
For backward compatibility, ``{\vt GridReg}'' is accepted as ``{\vt
PhysGridReg}''.
\end{description}


%-------------------------------
% techfile:attr:lpal
\subsection{Layer Palette Registers}
\index{palette registers}

The palette registers from the {\cb Layer Palette} are saved
to and assigned from the technology file.

\begin{description}
\index{ElecLayerPalette keyword}
\index{PhysLayerPalette keyword}
\item\parbox[b]{4in}{{\vt ElecLayerPalette}{\it N} {\it layer\_list}\\
 {\vt PhysLayerPalette}{\it N} {\it layer\_list}}\\
The {\it N} is an integer value in the range 1--7 that specifies a
register number.  The {\it layer\_list} is a list of layer names
separated by white space.  There are separate entries for electrical
and physical mode for each register number.  The list provides the
layer names and ordering of the layers in the ``user'' part of the
layer palette.
\end{description}


%-------------------------------
% techfile:attr:font 070214
\subsection{Font Assignments}
\index{Font keywords}

The keywords described below set the fonts used in various places in
{\Xic}.  These correspond to the fonts settable from the {\cb Font
Selection} pop-up from the {\cb Set Font} button in the {\cb
Attributes Menu}.

Since the font string format varies between the operating systems and
graphical interfaces supported by {\Xic}, provision is made for
separate font specifications for each supported variation, thus making
the technology file more portable between different versions of
{\Xic}.

There are six fonts that may be set, and four sets of corresponding
keywords, specific to different systems.  The four sets correspond to
a suffix character added to the font keyword.

\begin{description}
\item{\vt Font1 --- Font6} {\it name\_of\_font}\\
These keywords will be read and (if possible) applied by any version
of {\Xic}.  Although there is an attempt at portability, the {\it
name\_of\_font} should apply to the release of {\Xic} in use.  A
mismatch will not cause errors, but the font may not be as expected,
or a default may be used.  These keywords are mostly for backwards
compatibility, and are never written to a new technology file created
with the {\cb Save Tech} button in the {\cb Attributes Menu}.  Rather,
the system-specific keywords below will be written.

\item{\vt Font1P --- Font6P} {\it name\_of\_font}\\
These fonts apply to the releases that use the GTK-2 (Pango) font
system.  At the 3.3 release level, all {\XicTools} programs use this
graphical toolkit, and will use these keysords.

\item{\vt Font1X --- Font6X} {\it name\_of\_font}\\
These keywords apply to non-current releases (FreeBSD7, Linux2, OS X)
that use the GTK-1 X-windows font system.  The {\it name\_of\_font} is
the X Logical Font Descriptor for a font available on the user's
system, or an alias.  These font specifications are ignored in GTK-2
(all current) releases.

\item{\vt Font1W --- Font6W} {\it name\_of\_font}\\
These keywords apply only to the non-current Microsoft Windows
release, which used native Win32 for the graphical interface.  There
is really no syntactical difference between these and Pango ({\vt P})
specifications, and (current) GTK-2 releases will accept (but not
write) these.
\end{description}

If a font is specified more than once in the technology file, such as
with duplicate or equivalent keywords, the last specification read
will take precedence.

When a new technology file is written, only the keywords for
non-default fonts in use will actually be written in the file.

The index number of the keyword indicates the following fonts:
\begin{description}
\item{\vt 1} (Fixed Pitch Text Window Font)\\
This sets the font used in pop-up multi-line text windows other than
the text editor/file browser, such as the {\cb Files Listing} and {\cb
Cells Listing}, where the names are formatted into columns.\\
Defaults:\\
Unix/Linux: {\vt Monospace 9}\\
Windows: {\vt Lucida Console 9}

\item{\vt 2} (Proportional Text Window Font)\\
This sets the font used in pop-up multi-line text windows other than
the text editor/file browser, where text is not formatted, such as the
{\cb Info} and error message pop-ups.\\
Defaults:\\
Unix/Linux: {\vt Sans 9}\\
Windows: {\vt Sans 9}

\item{\vt 3} (Fixed Pitch Drawing Window Font)\\
This is the font used in the coordinate readout, the status line,
layer table, and the prompt line.  It is not the font used to render
label text in the drawing windows, which is a vector font generated by
other means.\\
Defaults:\\
Unix/Linux: {\vt Monospace 9}\\
Windows: {\vt Lucida Console 9}

\item{\vt 4} (Text Editor Font)\\
This is the font used in the {\cb Text Editor} and {\cb File Browser}
pop-ups.\\
Defaults:\\
Unix/Linux: {\vt Monospace 9}\\
Windows: {\vt Lucida Console 9}

\item{\vt 5} (HTML Viewer Proportional Font)\\
This is the base font used for proportional text in the HTML viewer
(help windows).  If set, this will override the font set in the {\vt
.mozyrc} file, if any.\\
Defaults:\\
Unix/Linux: {\vt Sans 9}\\
Windows: {\vt Sans 9}

\item{\vt 6} (HTML Viewer Fixed Pitch Font)\\
This is the base fixed-pitch font used by the HTML viewer.  If set,
this will override the font set in the {\vt .mozyrc} file, if any.\\
Defaults:\\
Unix/Linux: {\vt Monospace 9}\\
Windows: {\vt Lucida Console 9}
\end{description}

The platform-specific font keywords were added in release 3.1.6. 
Older technology files will use only the {\vt Font1 --- Font6}
keywords.  It may be be best to comment these out when importing a
technology file developed for another platform, or to modify the {\vt
Font} keywords to the appropriate flavor with a text editor.

Fonts can be set within {\Xic} with the {\cb Set Font} command in the
{\cb Attributes Menu}.


%-------------------------------
% techfile:attr:vars 021615
\subsection{Variable Setting as Keywords}
\label{attrvars}
In addition to the keywords described in the previous sections, most
of the variables (see \ref{variables}) that are known to {\Xic} can be
set as keywords.  These variables control various aspects of {\Xic},
including the states of most of the controls in the various pop-up
panels.  When a technology file is written, variables that participate
in this protocol and are set will contribute a corresponding line to
the attributes section of the new technology file.

Most variables participate in the protocol.  A few do not, for one
reason or another, and it is unlikely that these will be missed.  The
{\cb !attrvars} command will produce a list of the variables that
participate, the user can check this if necessary.

When a new technology file is being written, variables that are set
will generate content.  There is no ``default'', and the options in
the {\cb Write Tech File} panel that alter the treatment of ``default
definitions'' have no effect on these lines.

The same variables can also be set with the {\vt !set} lines.  If a
variable is set multiple times by any means, the last one seen will
have precedence.  The variables that participate in the protocol but
are set with the {\vt !set} line will not be remembered as having been
set.  When a technology file is written, the remembered variables are
given {\vt !set} lines in the new file.  This is not necessary for
variables that participate in the protocol.

Variables are logically divided into classes.  Boolean variables are
switches that are either set (usually to an empty string) or not set. 
Other variables we refer to as ``string'' variables.  They are set to
an arbitrary text string, when set at all.

In the technology file, booleans take the form
\begin{quote}
{\it VariableName} [{\vt y}$|${\vt n}]
\end{quote}
which is the same syntax as for boolean keywords.  The ``{\vt
y}$|${\vt n}'' symbol implies that one of `{\vt y}' or `{\vt n}'
should follow the keyword.  Actually, `{\vt 0}' (zero), or any word
that begins with the letters or sequence (case insensitive) `{\vt n}',
`{\vt f}', `{\vt of}' is taken as a false value.  Anything else,
including no following text, is taken as true (`{\vt y}' is always
redundant).  If the second token indicates affirmative, then the
variable will be set.  If the second token is negative, no action is
taken.

String variables take the form
\begin{quote}
{\it VariableName} {\it arbitrary text}
\end{quote}
where the variable will be assigned the {\it arbitrary text\/}, with
leading and training white space stripped.

The {\cb !attrvars} command lists the variables that are boolean and
string separately, so the user can check this list if unsure of the
variable type.

The {\it VariableName} in this context is recognized as a known
variable name without case sensitivity.  In every other context,
variable names are case-sensitive.  Since this syntax applies only to
internal variable names, there is no conflict as there are no such
variables that differ only in case.


% -----------------------------------------------------------------------------
% hcopytech 071010
\section{Hardcopy Driver Parameters}
By default, all hardcopy drivers available within the program are made
available to the user through the {\cb Format} menu in the {\cb Print
Control Panel}.  Drivers can be disabled, so they don't appear in the
{\cb Format} menu, by adding the ``{\vt off}'' keyword to the ``{\vt
HardCopyDevice}'' line, which begins the block of lines describing the
driver defaults.  The driver blocks are found near the end of the
technology file, and are written in their entirety when the {\cb Save
Tech} command is used to generate a technology file.  It is not an
error for a driver block to be absent; internal defaults will be used.

The following keyword(s) may be used outside of the driver blocks to
set the default print driver.

\begin{description}
\index{DefaultDriver keyword}
\index{AltDriver keyword}
\item{\vt DefaultDriver} {\it driver\_name}\\
This keyword sets the default print driver to use in both electrical
and physical modes.  When the {\cb Print Control Panel} initially
appears, the {\cb Format} menu will have this driver selected.  The
{\it driver\_name} is one of the driver names as listed in the {\vt
HardCopyDevice} keyword description below.  The keyword {\vt
AltDriver} is recognized as a synonym for this keyword.

\index{ElecDefaultDriver keyword}
\index{ElecAltDriver keyword}
\item{\vt ElecDefaultDriver} {\it driver\_name}\\
Similar to {\vt DefaultDriver}, but sets the default to use in
electrical mode only.  The keyword {\vt AltElecDriver} is a synonym.

\index{PhysDefaultDriver keyword}
\index{PhysAltDriver keyword}
\item{\vt PhysDefaultDriver} {\it driver\_name}\\
Similar to {\vt DefaultDriver}, but sets the default to use in
physical mode only.  The keyword {\vt AltPhysDriver} is a synonym.
\end{description}

A driver block begins with a {\vt HardCopyDevice} line naming the
driver, and ends with the next {\vt HardCopyDevice} line or end of
file.  In addition to the {\vt HardCopy...} keywords that specify
driver defaults, any of the keywords described in the {\bf
Presentation Attributes} and {\bf Attribute Colors} categories of the
{\bf Technology File Attributes} section \ref{techattr} can be used. 
The attribute or color will then apply while in print mode and the
driver is selected, both on-screen and in the driver output.  The
keyword formats are exactly as described in these subsections.  If not
given in a driver block, the driver will use the attribute or color
values set in the main part of the technology file, or the program
defaults if no value is specified.

Layer colors, fill, and visibility can be set on a per-layer basis for
the driver, by including a ``mini-layer block''.  This is a truncated
version of the layer blocks described in {\bf Technology File Layer
Blocks}, section \ref{techlayers}.  The only keywords which are
accepted in a mini-layer block are {\vt RGB} (to set the color), {\vt
Filled} (to set the fill pattern or outline style, and {\vt Invisible}
(to set visibility).  However, there are two additional special
keywords that may be included in specific drivers:

\begin{description}
\index{HPGLfilled keyword}
\item{\vt HPGLfilled} {\it filltype} [ {\it option1 option2\/} ]\\
This keyword is recognized and used only by the HP-GL hard copy driver
(``hpgl\_line\_draw\_color''), and is used to specify a fill pattern
for the layer (electrical or physical).  The parameters are those
appropriate for the {\vt FT} HPGL directive, as documented in
\begin{quote}
 ``{\sc The HP-GL2 and HP RTL Reference Guide:  A
Handbook for Program Developers}''
\end{quote}
from Hewlett-Packard, (ISBN 0-201-63325-6) pages 127-129.  This is
summarized below:

\begin{tabular}{|l|c|c|c|}\hline
\bf filltype & \bf description & \bf option1 & \bf option2\\ \hline
1  & solid, bidirectional & ignored & ignored\\ \hline
2  & solid, unidirectional & ignored & ignored\\ \hline
3  & hatched, parallel lines & line spacing & line angle\\ \hline
4  & crosshatched & line spacing & line angle\\ \hline
10 & shadings & shading level & ignored\\ \hline
11 & not supported & ignored & ignored\\ \hline
\end{tabular}

There are 1016 dots per inch and angles are in degrees.  Shading level
is 0--100.  If the {\vt HPGLfilled} keyword is supplied for a layer
and the {\it filltype} and options (if given) are valid, that fill
will be used with the layer in HPGL output.  There is presently no way
to assign the layer color.

This parameter must be added to the technology file with a text
editor.  The default is no fill.  Note that the fill patterns set on
the screen in hard copy mode are not used by the HP-GL driver.

\index{XfigFilled keyword}
\item{\vt XfigFilled} {\it filltype}\\
This keyword is recognized and used only by the {\et xfig} hard copy
driver (``xfig\_line\_draw\_color''), and allows setting the fill
patterns for the layer (electrical or physical).  The {\it filltype}
is an integer 1--56, which selects one of {\et xfig}'s internal fill
patterns.

\begin{tabular}{|l|l|}\hline
0 & No fill\\ \hline
... & shades\\ \hline
20 & Full saturation of the color\\ \hline
... & tints\\ \hline
40 & White\\ \hline
41 & 30 degree left diagonal pattern\\ \hline
42 & 30 degree right diagonal pattern\\ \hline
43 & 30 degree crosshatch\\ \hline
44 & 45 degree left diagonal pattern\\ \hline
45 & 45 degree right diagonal pattern\\ \hline
46 & 45 degree crosshatch\\ \hline
47 & Bricks\\ \hline
48 & Circles\\ \hline
49 & Horizontal lines\\ \hline
50 & Vertical lines\\ \hline
51 & Crosshatch\\ \hline
52 & Fish scales\\ \hline
53 & Small fish scales\\ \hline
54 & Octagons\\ \hline
55 & Horizontal "tire treads"\\ \hline
56 & Vertical "tire treads"\\ \hline
\end{tabular}

Values 1 to 19 are ``shades'' of the color, from darker to lighter, a
shade is defined as the color mixed with black.  Values from 21 to 39
are ``tints'' of the color from the color to white, a tint is defined
as the color mixed with white.  The {\vt XfigFilled} parameter must be
added to the technology file with a text editor.  The default is no
fill.  Note that the fill patterns set on the screen in hard copy mode
are not used by the {\et xfig} driver.
\end{description}

As for regular layer blocks, a mini-layer block starts with a {\vt
PhysLayer} or {\vt ElecLayer} keyword, or one of the aliases.  The
layer name given must be the name of a layer supplied in one of the
regular layer blocks.  A mini-layer block terminates when a new
mini-layer block starts, or at the end of the driver block.  The block
order, and order with respect to other keywords, is arbitrary.

The other keywords of the driver block are described below.

\index{hard copy driver parameters}
\begin{description}
\index{HardCopyDevice keyword}
\index{hard copy driver names}
\item{{\vt HardCopyDevice} {\it device\_name} [off]}\\
This line begins the driver block, and the keywords that follow apply
to the {\it device\_name\/} driver.  The names are internally
recognized strings:
\begin{quote}\rr\vt
hp\_laser\_pcl\\
hpgl\_line\_draw\_color\\
postscript\_bitmap\\
postscript\_bitmap\_encoded\\
postscript\_bitmap\_color\\
postscript\_bitmap\_color\_encoded\\
postscript\_line\_draw\\
postscript\_line\_draw\_color\\
windows\_native\\
xfig\_line\_draw\_color\\
image\\
\end{quote}
If the ``{\vt off}'' keyword is given (``{\vt disable}'' and ``{\vt
n}'' are synonyms), the driver is disabled, and will not appear in the
{\cb Format} menu of the {\cb Print Control Panel}.

See the description of the {\cb Print} button in the {\cb File Menu}
(\ref{hcformat}) for more information on these drivers.

\index{HardCopyLegend keyword}
\item{\vt HardCopyLegend} {\it n}\\
This keyword sets the default status of the {\cb Legend} button in the
{\cb Print Control Panel} when the driver is active.  Values can be 0,
1, or 2:

\begin{tabular}{ll}
0 & {\cb Legend} button is off\\
1 & {\cb Legend} button is on\\
2 & {\cb Legend} button is grayed and inactive\\
\end{tabular}

\index{HardCopyOrient keyword}
\item{\vt HardCopyOrient} {\it n}\\
This keyword sets the default status of the {\cb Portrait}, {\cb
Landscape}, and {\cb Best Fit} buttons in the {\cb Print Control
Panel} while the driver is active.  Values are 0--3:

\begin{tabular}{ll}
bit 0 set & {\cb Landscape} on, {\cb Portrait} off\\
bit 0 unset & {\cb Landscape} off, {\cb Portrait} on\\
bit 1 set & {\cb Best Fit} button on\\
bit 1 unset & {\cb Best Fit} button off\\
\end{tabular}

\index{HardCopyCommand keyword}
\item{{\vt HardCopyCommand} {\it command\_string}}\\
Specifies the command to use to queue the plot.  This will be shown in
the command text box of the {\cb Print Control Panel}.  The characters
``{\vt \%s}'' will be replaced with the name of the temporary file,
all other characters are passed verbatim.  If ``{\vt \%s}'' does not
appear in the string, the file name will be appended to the string,
separated by a space character.  This keyword is ignored under
Microsoft Windows.

\index{HardCopyResol keyword}
\item{{\vt HardCopyResol} {\it list\_of\_integers}}\\
This sets the resolutions supported by the driver, in dots per
inch.

\index{HardCopyDefResol keyword}
\item{{\vt HardCopyDefResol} {\it integer}}\\
This has meaning only to drivers that have selectable resolutions. 
The value following this keyword is a zero-based index into the list
of resolutions as given with the {\vt HardCopyResol} keyword, and
indicates the default resolution which will be selected in the {\cb
Print Control Panel} for the driver.

Example:
\begin{quote} \vt
HardCopyDevice postscript\_line\_draw\\
HardCopyResol 72 75 100 150 200 300 400\\
HardCopyDefResol 2\\
\end{quote}

This will select 100 as the resolution for the {\vt
postscript\_line\_draw} driver when the {\cb Print Control Panel}
first appears.  The resolution can be changed with the menu.

\index{HardCopyDefHeight keyword}
\index{HardCopyDefWidth keyword}
\index{HardCopyDefXoff keyword}
\index{HardCopyDefYoff keyword}
\item\parbox[b]{4in}{\vt
HardCopyDefHeight {\it float\_format\_number}\\
HardCopyDefWidth {\it float\_format\_number}\\
HardCopyDefXoff {\it float\_format\_number}\\
HardCopyDefYoff {\it float\_format\_number}}\\
These set the default image size and location, and are in inches,
unless followed by the letter `c' which denotes centimeters.  The {\vt
Yoff} number may be interpreted as a top or bottom margin, depending
upon the driver.  The dimensions are in all cases relative to the
portrait orientation of the page.  If the width or height is set to
zero (but not both) the driver will assume auto-width or auto-hight
mode, where the width or height is set to the minimum necessary to
render the object.

\index{HardCopyMinHeight keyword}
\index{HardCopyMinWidth keyword}
\index{HardCopyMinXoff keyword}
\index{HardCopyMinYoff keyword}
\item\parbox[b]{4in}{\vt
HardCopyMinHeight {\it float\_format\_number}\\
HardCopyMinWidth {\it float\_format\_number}\\
HardCopyMinXoff {\it float\_format\_number}\\
HardCopyMinYoff {\it float\_format\_number}}\\
These set the minimum acceptable values for the parameters.

\index{HardCopyMaxHeight keyword}
\index{HardCopyMaxWidth keyword}
\index{HardCopyMaxXoff keyword}
\index{HardCopyMaxYoff keyword}
\item\parbox[b]{4in}{\vt
HardCopyMaxHeight {\it float\_format\_number}\\
HardCopyMaxWidth {\it float\_format\_number}\\
HardCopyMaxXoff {\it float\_format\_number}\\
HardCopyMaxYoff {\it float\_format\_number}}\\
These set the maximum acceptable values for the parameters.
\end{description}


% -----------------------------------------------------------------------------
% resourcefile 090814
\section{Resource File}
\label{resourcefile}
One can use the resource-setting capability of the X-Windows system to
set attribute colors.  This applies when running under the X-Windows
system, which is presently true for all releases except those for
Microsoft Windows.  However, this is archaic and not really recommended.

One can create an X resource file for {\Xic}.  This is a file that
should be created in the user's home directory, with a name that is
the executable program name with the first letter capitalized, i.e.,
{\vt Xic}.  The file contains lines in the following form:

\begin{quote}\vt
xic.HighlightingColor: green\\
xic.MarkerColor: blue
\end{quote}

or generally
\begin{quote}\vt
{\vt xic.}{\it resourcename}: {\it colorspec}
\end{quote}

The {\it resourcename} is a keyword from the list of attribute colors
as listed in \ref{attrcolors}.  Note that the keyword must be used,
not an alias.  The aliases are recognized in the technology file and
{\cb !setcolor} command.  The {\it colorspec} string is the name of a
color or an RGB triple:

\begin{itemize}
\item{The name of a color.  The recognized names can be listed from
   the {\cb Set Color} pop-up in the {\cb Attributes} menu with the
   {\cb Colors} button.}
\item{Three space-separated numbers, each 0--255, representing the
   red, green, and blue intensity.  E.g., ``{\vt 196 240 235}''.}
\item{Other forms recognized by the {\vt XParseColor} C library
   function, including\\ ``{\vt \#RRRRGGGGBBBB}'' and ``{\vt
   rgb:RRRR/GGGG/BBBB}''.  Here, {\vt R}, {\vt G}, and {\vt B} are
   single hexadecimal digits.}
\end{itemize}

