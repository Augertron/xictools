% -----------------------------------------------------------------------------
% Xic Manual
% (C) Copyright 2009, Whiteley Research Inc., Sunnyvale CA
% $Id: properties.tex,v 1.43 2017/03/22 07:29:56 stevew Exp $
% -----------------------------------------------------------------------------

% -----------------------------------------------------------------------------
% prptyspec 021513
\index{properties}
\chapter{Property Specifications}
\label{prptyspec}

In {\Xic}, cells and database objects contain a list of number-string
associations called ``properties''.  These are used to store various
pieces of information about the object.  Some properties a used only
by the internals of {\Xic} and are not accessible to the user, while
other properties can be set by the user to assign certain attributes
to an object.  The user will encounter properties primarily in electrical
mode, as this is the means by which devices are assigned values, models,
and other parameters.

The properties that are assigned by {\Xic}, and/or have meaning to
{\Xic} are described if the following sections.  Generally, the
property numbers 7000 -- 7199 are reserved by {\Xic}, and property
numbers in this range should not be assigned by the user.  Also,
property numbers in the range 7200 -- 7299 correspond to
``pseudo-properties'' which are used to query or change the parameters
of a physical object (see \ref{pseudoprops}).  These values should not
be used for assigned properties.


% -----------------------------------------------------------------------------
% physpropfmt 101016
\section{Physical Mode Property Specifications}
\label{physpropfmt}
\index{properties!physical}

This section lists the properties known to {\Xic} that may be found in
physical cells, instances, or objects.  This lists only properties
likely to be encountered by the user, there may be additional
properties that are not specifically documented used internally by
{\Xic}.  All physical properties known to {\Xic} use numbers in the
reserved range 7000 -- 7199.

\begin{description}
\index{properties!text}
\index{text property}
\item{\et text} property, number 7012\\
This property saves GDSII label parameters ANGLE, MAG, WIDTH, and
PTYPE, which are unused by {\Xic}.  The string consists of a
concatenation of keyword/value pairs, using the keywords above
(not all need be present).  These attributes will be reassigned to
the label when a GDSII file is written.

\index{pathtype property}
\index{properties!pathtype}
\item{\et pathtype} property, number 7033\\
This property is used for physical mode wires of nonzero width
which have a non-default path type.  The string has the form
\begin{quote}
{\vt PATHTYPE} {\it pathtype}
\end{quote}
where {\it pathtype} is 0 for flush ends, 1 for rounded ends.  The
default pathtype is 2 (extended ends).
This property is added to wires in native and CIF output if the {\et
CifOutExtensions} variable has the {\cb wire extension} flag set, and
the {\cb wire extension new} flag is {\bf not} set.  The wire end
style is included in the wire specification in the present default
syntax, so {\cb wire extension new} is set by default. 

\index{properties!grid}
\index{grid property}
\item{\et grid} property, number 7100\\
This property is applied to the top-level physical cell when the cell
is saved, preserving the current grid setting.  This property is used
in physical mode only.  The property string has the format
\begin{quote}
{\vt grid} {\it resol} {\it snap}
\end{quote}
where {\it resol} is the number of internal units per snap point, and
{\it snap} is the number of snap points per grid line if positive, or
grid lines per snap point if negative.

\index{properties!flags}
\index{flags property}
\item{\et flags} property, number 7105\\
This property can be be applied to physical cells.  The property
string can take one of two forms:  a hex number, or a space-separated
list of string tokens.  The tokens and corresponding bits are

\begin{tabular}{|l|l|l|} \hline
\bf Bit & \bf Keyword & \bf Description\\ \hline\hline
  &            & When set, the cell is ``opaque'' with regard to\\
0 & \vt OPAQUE & extraction.  The cell will look like a black box\\
  &            & with terminals.\\ \hline
1 & \vt CONNECTOR & Not implemented, don't use.\\ \hline
2 & \vt USER0 & User flags, not used by {\Xic}.  These flags
  may be\\ \cline{1-2}
3 & \vt USER1 & useful to the user.\\ \hline
\end{tabular}

When the {\et ExtractOpaque} variable is set, the {\vt OPAQUE} flag is
ignored.

\index{properties!refcell}
\index{refcell property}
\item{\et refcell} property, number 7150\\
A reference cell is an empty cell with a {\et refcell} property,
which references a cell hierarchy in another layout file.  Reference
cells can exist in memory or as a native cell file on disk.

The string for this property consists of space-separated {\it
keyword\/}{\vt =}{\it value} pairs.  The known keywords are as
follows:

\begin{description}
\item{\vt cellname}\\
The top-level cell to extract from the referenced hierarchy.
\item{\vt dbname}\\
The CHD name in memory.  This is never written to a file, it is only
used when the cell is in memory.
\item{\vt filename}\\
The full path to the referenced layout file.
\item{\vt bound}\\
The bounding box, may be used for area filtering, in the form
{\it L\/},{\it B\/},{\it R\/},{\it T\/} where the values are
floating-point in microns.
\item{\vt aflags}\\
Alias flags integer, these set name aliasing modes.
\item{\vt aprefix}\\
Cell name change prefix.
\item{\vt asuffix}\\
Cell name change suffix.
\end{description}

\index{properties!flatten}
\index{flatten property}
\item{\et flatten} property, number 7151\\
During extraction, simple cells that contain only geometry or perhaps
all or part of a device can be logically flattened (see \ref{exthier})
into their parent cells for extraction purposes.  If this property is
set in a cell, that cell will always be considered as part of its
containing cell by the extraction system.

This is identical to the effect of listing the cell name in the {\et
FlattenPrefix} variable.

The string for this property is ignored, but is set to ``{\vt
flatten}'' by convention.

\index{properties!nomerge}
\index{nomerge property}
\item{\et nomerge} property, number 7152\\
The {\et nomerge} property applies to physical boxes, polygons, and
wires, and is used by the extraction system.  If this property is
found on any object used to recognize a device body, that device will
never be merged with similar devices.  This is relevant when merging
is enabled for the device during extraction, and one wants to suppress
this in individual cases.  It prevents both parallel and series
merging.

\index{properties!stdvia}
\index{stdvia property}
\item{\et stdvia} property, number 7160\\
This property is given to standard via sub-masters and instances.
%\ifoa
The property is recognized by the OpenAccess plug-in providing
transparent conversion between OpenAccess and {\Xic} standard vias.
%\fi
The syntax is described in \ref{stdviaprp}.

\index{properties!termorder}
\index{termorder property}
\item{\et termorder} property, number 7168\\
This is set to a space-separated list of group names, and can be
applied to physical cells.  It will provide the cell connection
terminal names and ordering when electrical data are absent.  The
names must match net name labels (see \ref{netname}) placed in the
layout.  Names not found are silently ignored.

\index{properties!skipdrc}
\index{skipdrc property}
\item{\et skipdrc} property, number 7178\\
This property is applied in output to boxes, polygons, or wires which
have the {\cb skip DRC} flag set.  It is used to set the {\cb skip
DRC} flag in boxes, polygons, and wires as an input file is being
read.

\index{properties!labelsize}
\index{labelsize property}
\item{\et labelsize} property, number 7180\\
This property is added to labels when writing to GDSII, and saves the
label width, height and visibility status.  The string has the format
\begin{quote}
{\vt width} {\it width} {\vt height} {\it height} [{\vt show}|{\vt hide}]
 [{\vt tlev}] [{\vt liml}]
\end{quote}
where {\it width} and {\it height} are in internal units.

The keywords ``{\vt show}'' or ``{\vt hide}'' appended to the string
store the display state of the label, which can be visible or
``hidden'', toggled by clicking with button 1 with the {\kb hift} key
held.  The {\et LabelHiddenMode} variable controls the scope of this
feature.

The {\vt tlev} keyword gives the label the property of being invisible
in instances of the containing cell, but visible when the cell is
viewed as the top-level (current cell).

The {\vt liml} keyword causes the label to limit the number of lines
displayed, when the label text has multiple lines.  The maximum line
count defaults to 5, and is otherwise given with the {\vt
LabelMaxLines} variable.

The four flags are the same as those accessible with the {\et
XprpXform} pseudo-property.
\end{description}

%\ifoa
This group of properties applies to the OpenAccess interface.

\begin{description}
\index{properties!oa\_cstmvia}
\index{oa\_cstmvia property}
\item{\et oa\_cstmvia} property, number 7161\\
This property is applied by the translator to {\Xic} cells that
represent a custom via object from OpenAccess.  The format is
described in \ref{oaplugin}.

\index{properties!oa\_orig}
\index{oa\_orig property}
\item{\et oa\_orig} property, number 7183\\
This property is applied transiently when reading cell data into
{\Xic}.  The format is described in \ref{oaplugin}.
\end{description}
%\fi

The following group of properties implements the Ciranova abutment
protocol for parameterized cells.  Parameterized cells may use this
protocol to automatically merge abutted instances so as to share
common features.

\begin{description}
\index{properties!ab\_class}
\index{ab\_class property}
\item{\et ab\_class} property, number 7185\\
This is similar to the Ciranova {\vt pycAbutClass} property.
The format of the {\et ab\_class} property is described in \ref{pcabut}.

\index{properties!ab\_rules}
\index{ab\_rules property}
\item{\et ab\_rules} property, number 7186\\
This is similar to the Ciranova {\vt pycAbutRules} property.
The format of the {\et ab\_rules} property is described in \ref{pcabut}.

\index{properties!ab\_directs}
\index{ab\_directs property}
\item{\et ab\_directs} property, number 7187\\
This is similar to the Ciranova {\vt pycAbutDirections} property.
The format of the {\et ab\_directs} property is described in \ref{pcabut}.

\index{properties!ab\_shapename}
\index{ab\_shapename property}
\item{\et ab\_shapename} property, number 7188\\
This is similar to the Ciranova {\vt pycShapeName} property.
The format of the {\et ab\_shapename} property is described in \ref{pcabut}.

\index{properties!ab\_pinsize}
\index{ab\_pinsize property}
\item{\et ab\_pinsize} property, number 7189\\
This is similar to the Ciranova {\vt pycPinSize} property.
The format of the {\et ab\_pinsize} property is described in \ref{pcabut}.

\index{properties!ab\_inst}
\index{ab\_inst property}
\item{\et ab\_inst} property, number 7190\\
The format of the {\et ab\_inst} property is described in \ref{pcabut}.

\index{properties!ab\_prior}
\index{ab\_prior property}
\item{\et ab\_prior} property, number 7191\\
The format of the {\et ab\_prior} property is described in \ref{pcabut}.

\index{properties!ab\_copy}
\index{ab\_copy property}
\item{\et ab\_copy} property, number 7192
The format of the {\et ab\_copy} property is described in \ref{pcabut}.
\end{description}

The following property is required to implement the Ciranova protocol
for stretch handles in parameterized cells.  A stretch handle is a
display element that can be dragged with the mouse, which initiates a
change of a cell property and appropriate remastering.

\begin{description}
\index{properties!grip}
\index{grip property}
\item{\et grip} property, number 7195\\
The format of the {\et grip} property is described in \ref{pcstretch}.
\end{description}

The remaining properties support parameterized cells (pcells).  The
super-master pcell contains a script reference, default parameter
values, and (optionally) parameter constraint strings.  When the
super-master is instantiated, the script is executed producing a
sub-master under a modified name, plus an instance of the sub-master. 
The instance contains the name of the super-master and a copy of the
instantiation parameters.

\begin{description}
\index{properties!pc\_name}
\index{pc\_name property}
\item{\et pc\_name} property, number 7197\\
This property is assigned by {\Xic} to pcell sub-masters and their
instances.  It provides the name of the pcell from which the
sub-master or instance was derived.

\index{properties!pc\_params}
\index{pc\_params property}
\item{\et pc\_params} property, number 7198\\
This property is assigned by the user to pcells, and contains the
default parameter set.  It will be assigned by {\Xic} to sub-masters
and instances, and contains the parameter set that was used to create
the sub-master.  See \ref{pcnative} for a complete description.

\index{properties!pc\_script}
\index{pc\_script property}
\item{\et pc\_script} property, number 7199\\
This property is assigned by the user to a pcell, and appears only in
the super-master.  It contains the script, or a path to a script,
which is executed when the pcell is instantiated.  See \ref{pcnative}
for a complete description.
\end{description}


% -----------------------------------------------------------------------------
% elecpropuser 081318
\section{User-Specified Electrical Property Specifications}
\label{elecpropuser}
\index{properties!electrical}

The properties described in this section provide user-specified
information to device and subcircuit instances, and to device and cell
definitions.  In many cases, the property applied to a device
definition will supply a default for a similar property created in the
new instance when the device is instantiated.  The instance property
can be subsequently modified by the user. 

The {\et name} property described in the next section, plus the {\et
devref}, {\et model}, {\et value} and {\et param} properties discussed
below, translate into fields of device definition lines when
generating SPICE output, and in order to set these properties
proficiently, the user must have familiarity with the SPICE syntax.

The strings for these properties may contain special escape sequences
indicating hypertext references or other characteristics.  These are
described in \ref{prpescapes}.

\begin{description}
\index{properties!model}
\index{model property}
\item{\et model} property, number 1\\
The {\et model} property appears in device instances and defines a
device model to be included in the SPICE line for the device.  This
property is normally assigned to the device instance with the {\cb
Property Editor} from the {\cb Edit Menu}, but a default model can be
supplied by including this property in the device definition in the
device library file.
\begin{quote}\vt
5 1 {\it model\_name\/};
\end{quote}
The {\it model\_name\/} is arbitrary, but a corresponding entry should
exist in a model library file.

\index{value property}
\index{properties!value}
\item{\et value} property, number 2\\
The {\et value} property supplies a string to be used in the device
line in SPICE output for the device ``value''.  It should not appear
if the device has a model property, and if it does, it will be
ignored.  The property is normally applied to device instances with
the {\cb Property Editor}, but can appear in the device definition
in the device library file to assign a default value for the device.
\begin{quote}\vt
5 2 {\it value\/};
\end{quote}
The {\it value\/} is a string which may, for example, represent a
floating point number specifying the component value, e.g., in ohms
for a resistor.  In general, any string can appear, and it may include
hypertext references.  A complex string would be necessary for a
voltage source with functional dependence, for example.

The {\et model} and {\et value} properties are mutually exclusive,
either can be supplied, but not both.

\index{initc property}
\index{param property}
\index{properties!param}
\item{\et param} property, number 3\\
The {\et param} property specifies the part of the device SPICE line
which provides an initial condition or other data not included in a
model or value string.  The property is normally applied to device and
subcircuit instances with the {\cb Property Editor}, or to cells with
the {\cb Cell Property Editor} command.  When applied to cells or
subcircuit instances, the property is used to provide parameter
definitions for SPICE (see the description of the {\vt .subckt} line
in the {\WRspice} manual).  This can also appear in the device
definition in the device library file to provide a default.  If given
to a cell, instances of the cell will inherit the property, which can
then be changed from within {\Xic} on a per-instance basis.  For
device instances, this property specifies any parameter, such as
device area, which is provided in the device line after the model. 
This manifestation was referred to as the initial condition (``{\et
initc}'') property in previous documentation.
\begin{quote}\vt
5 3 {\it string\/};
\end{quote}
The {\it string\/} will be appended to the device line when a SPICE file
is created.  It can contain initial condition data or other parameters
significant to the device, which are syntactically expected to the
right of the model or value.

The parameter definitions in a {\et param} property string have the
form
\begin{quote}
{\it name1\/}{\vt =}{\it value1} {\it name2\/}{\vt =}{\it value2} ...
\end{quote}

There may be white space around the `{\vt =}' character.  The {\it
name} tokens are parameter names, which are alphanumeric words
starting with an alpha character.  The {\it value} token can not be
empty, and must be a single text token.  This means that if the {\it
value} string contains white space, it must be single or double
quoted.  Be aware that the interpretation of single quoted ({\vt
'word'}) and double quoted ({\vt "word"}) differs fundamentally. 
Double quoting implies a manifest string type.  The string will be
assigned verbatim to the parameter, which will be of string type.  No
further processing will be done.  Single quoting implies an expression
which reduces to a number when evaluated.  If a {\it value} is not
quoted, it will be evaluated as an expression if necessary, otherwise
it will be taken as a numeric value.  Generally, parameter assignment
failures are silently ignored.

\index{other property}
\index{properties!other}
\item{\et other} property, number 4\\
The {\et other} property is a catch-all device property that is not
used by {\Xic} and does not appear in SPICE output.  There can be
arbitrarily many {\et other} properties specified for a device, unlike
the {\et model}, {\et value}, and {\et param} properties which can
appear at most once.  The {\et other} property can be used for storage
of alternate values for the {\et model}, {\et value}, and {\et param}
properties.  It is applied to device instances with the {\cb Property
Editor}.  Although it can be used in device definitions in the device
library file, there seems to be no reason for doing so.
\begin{quote}
{\vt 5 4 {\it string\/};}
\end{quote}

\index{nophys property}
\index{properties!nophys}
\item{\et nophys} property, number 5\\
When the {\et nophys} property is applied to an electrical device or
subcircuit, that device or subcircuit is assumed to have no physical
implementation and is ignored in the algorithm that associates
electrical and physical devices and subcircuits.  A device or
subcircuit with this property has no dual in the physical layout, and
its terminals will never be placed in the physical layout, where they
would otherwise be visible with the {\cb Show Terms} command.  Devices
and subcircuits with this property will be ignored in LVS testing.

In order to actually simulate a circuit that has been extracted from
the physical layout, it is necessary to add sources and perhaps other
devices, which have no counterparts in the physical layout.  In
general, this will cause LVS errors in subsequent LVS runs.  The {\et
nophys} property can be added to the additional devices to avoid these
errors.

By ``ignoring'' these devices, the device terminals are considered as
open circuits.  However, there are times when it would be useful to
consider these devices as shorted.  For example, suppose that one
wishes to include parasitic series inductance in a resistor during
simulation.  However, this inductance would cause LVS to fail, since
the series inductor added to the schematic has no explicit physical
counterpart.

It is possible to configure the {\et nophys} property to indicate that
when the electrical netlist is generated for use by the extraction
system, the corresponding devices will be forced such that all
terminals connect to the same net, i.e., the terminals are effectively
shorted together.  Thus, the inductor in the example above, if given
this property, would disappear properly during LVS.

The numerical value of the property is 5.  The property string is
either ``{\vt nophys}'' or ``{\vt shorted}''.  The latter indicates
that the shorting feature is to be used.  {\Xic} will always set the
property string to one of these values.  Devices inherit this property
from cell definitions in the device library file.  The format is
\begin{quote}
{\vt 5 5 nophys;} or\\
{\vt 5 5 shorted;}
\end{quote}

Devices with the {\et nophys} property applied will be rendered using
a different color than other devices.

\index{virtual property}
\index{properties!virtual}
\item{\vt virtual} property, number 6\\
When the {\et virtual} property is applied to an electrical
subcircuit, the subcircuit will not be included in netlist output. 
This means that in SPICE output, the corresponding ``{\vt .subckt}''
block of lines will be absent.  However, calls to this subcircuit, if
any, will be included, and must be resolved through text from a {\vt
.include} line or by some other means.

This is a method for including ``foreign'' subcircuits within the
{\Xic}/{\WRspice} framework.

The numerical value of the property is 6.  The property string is
``{\vt virtual}''.  {\Xic} will always set the property string to this
value.  This property applies only to electrical cell definitions
(subcircuits).  The format is
\begin{quote}
\vt 5 5 virtual;
\end{quote}

\index{flatten property}
\index{properties!flatten}
\item{\vt flatten} property, number 7\\
This can be applied to electrical masters and instances.  The state is
active if the instance has the property and the master does not, or
the instance does not have the property and the master does.  If
active, the schematic will be logically flattened into its parent
before association in LVS.

\index{range property}
\index{properties!range}
\item{\vt range} property, number 8\\
The {\et range} property can be applied to device (other than terminal
devices) and subcircuit instances.  The property contains two
non-negative integers, which define a range of values between the
start and end integers inclusive, stepping by one.  When applied to an
instance, the instance becomes {\it vectorized}, with the range
providing the subscripts for the individual scalar instances.  Scalar
contact terminals become vectors, and vectors become bundles.  Use of
vector instances can simplify some schematics with repeated circuit
blocks.  More information about vector instances and the rules for
connecting to them can be found in \ref{vecinst}.

The property number is 8, and the property string consists of two
non-negative integers, the starting and ending values of the
subscripting range.  The property applies only to non-terminal device
instances and subcell instances.

\index{macro property}
\index{properties!macro}
\item{\vt macro} property, number 20\\
The {\et macro} property is {\bf no longer in use}, having been
replaced by the {\vt macro} flag which is associated with the {\et
name} property.  However, it is still recognized and performs its
intended function when encountered.  By default, it will not be
generated in output, thus there is a potential compatibility issue
with {\Xic} release 4.3.5 and earlier.  The new variable {\et
WriteMacroProps} can be set before generating output to include {\et
macro} properties, thus providing backwards compatibility.

While reading input, if a {\et macro} property is read, a window
appears reminding the user to set {\et WriteMacroProps} if backwards
compatibility is needed.  The message can be avoided by either of the
following:
\begin{enumerate}
\item{Save the design to a new file, it will not be backwards
compatible, and will have no {\et macro} properties.}

\item{Set the {\et WriteMacroProps} variable in a startup script. 
This suppresses the message, and backwards-compatible files will be
produced.}
\end{enumerate}

It is no longer possible to (conveniently) create {\et macro}
properties in {\Xic}, for example with the {\cb Cell Property Editor}.

The {\vt macro} flag (or property) applies to device master cells. 
When present, its only effect is that in SPICE output, an `{\vt X}' is
prepended to the device name in instantiation lines of the device. 
Thus, SPICE will treat the device instance as a subcircuit call. 
These instances must have a {\et model} property giving a name that
will match a {\vt .subckt} definition somewhere, likely from a PDK
device model file.

This accounts for devices that are likely the electrical part of
parameterized cells, that implement nonlinear behavior through a
network of controlled sources expressed as a subcircuit in the SPICE
model definitions file.  MOS capacitors and poly resistors are devices
that are frequently modeled this way.

If the {\vt macro} flag is set and the name {\it prefix} already
begins with {\vt X} or {\vt x}, the device is taken as a macro,
meaning that {\Xic} will not output a subcircuit definition for the
cell, and a {\et model} property will provide the name of a subcircuit
definition expected to be found in the model library or elsewhere.

\index{devref property}
\index{properties!devref}
\item{\vt devref} property, number 21\\
This property maps text that appears in a SPICE device call after the
node list but ahead of the model or value.  The purpose is to provide
the name of a reference device for current-controlled sources (CCCS
and CCVS), and the current-controlled switch (CSW).  This property
can be applied to device instances only, and is supplied by the user
typically with the {\cb Property Editor}.

The property supports hypertext, and the reference name should be
added as a hypertext reference, so that the correct device is
referenced if the name should change.  That is, when editing the
property string on the prompt line, click on the device to reference. 
The device name will be entered in the line using colored text,
indicating a hypertext entry.  Unlike plain test, the hypertext entry
will still be correct if the referenced device name changes.
  
There is no default, and a missing property will produce a syntax
error in a generated SPICE file.
\end{description}


%------------------------------------------------------------------------------
% elecpropxic 081318
\section{{\Xic}-Managed Electrical Property Specifications}
\index{properties!electrical}
\label{branchprop}

The properties that are set by {\Xic} in electrical mode are described
below.  The electrical property values use the integers 1--21.  The
values 22--30 are reserved for future use.

\begin{description}
\index{bnode property}
\index{properties!bnode}
\item{\et bnode} property, number 9\\
The {\et bnode} property identifies the location of a ``bus
connector'' which is used to specify multiple connections to a device
or subcircuit.  It may appear in subcircuit cell definitions and
instance references.

\begin{quote}{\vt
5 9} {\it index beg\_range end\_range x y};
\end{quote}

The {\it index} is a non-negative integer index which serves to link
the bus node to an existing node with the same index.  The {\it
beg\_range} and {\it end\_range} are non-negative integers which set
the indexing of the bus.  The bus bit indices range from {\it
beg\_range} through {\it end\_range}.  Note that the numbers can be
ascending or descending.  The ``bit'' for {\it beg\_range} is
connected to the node with the given {\it index}.  The ``bit'' for
{\it end\_range} is connected to the node with index equal to {\it
index} + {\vt abs(}{\it beg\_range} - {\it end\_range\/}{\vt )}.  If
no node property has an equal index value, then that ``bit'' is simply
open.

For cells, the {\it elecX} and {\it elecY} each have the general form
\begin{quote}
{\it schemX\/}[{\vt :}{\it symbX\/}[{\vt ,}{\it symbX} ...]]
\end{quote}
and similar for the Y values.  This represents a single X,Y contact
location in the schematic, and an arbitrary number of contact
locations in the schematic symbol.  The schematic value is separated
from the symbolic values by a colon.  The symbolic values are
separated from each other by commas.  If the 3.2 format is being
written due to the {\et Out32nodes} variable being set, at most one
number will appear following the colon, the first that would otherwise
be listed if there are multiple contact points.

If there is no symbolic representation, or the terminal location has
not been set in the symbolic view, the {\it elecX} and {\it elecY}
each consist of a single number.  In the more general case, both terms
should supply the same number of integers.

In cell instances, there is no colon delimiter, and the general form
is simply a comma-separated list of numbers.  This is all identical to
the coordinate specification for a {\et node} property.

\index{node property}
\index{properties!node}
\item{\et node} property, number 10\\
The {\et node} property defines a circuit connection point.  It
appears as a property of wires, device and subcircuit instances, and
cells.  Its string is a bit different in the three cases.

Wire property\\
\begin{quote}{\vt
5 10} {\it circuit\_node}
\end{quote}

Any text that follows the form shown above is ignored.  The {\it
circuit\_node} is the node number in the current cell of the net
containing the wire.  All wires that participate in connectivity,
i.e., on the {\vt SCED} layer and any layer with the {\vt WireActive}
technology file keyword applied, should have a {\et node} property.

Instance property\\
\begin{quote}{\vt
5 10} {\it circuit\_node index elecX  elecY\/} [{\it name\/}]
\end{quote}

A subcircuit or device instance will have one {\et node} property per
circuit connection.  The {\it circuit\_node} is the node number of the
connection in the current cell.  The {\it index} is the terminal
ordering parameter.  Each {\et node} property of an instance will have
a unique {\it index\/}.  The indices form a compact run starting with
0.

The {\it elecX} and {\it elecY}, are integers, or comma-separated
lists of integers.  Both terms specify the same number of integers. 
Taken as ordered X,Y values, these provide the ``hot spots'' where
connection to the terminal can be made.  {\Xic} allows an arbitrary
number of hot spots per node.  If the {\et Out32nodes} variable is
set, which forces output compatible with earlier {\Xic} releases, the
{\it elecX} and {\it elecY} will each consist of a single value, the
first in the list that would otherwise be output if there are multiple
contact points.

Internally, there are 1000 integer counts per ``micron'', and hot
spots must appear on a 1 ``micron'' (1000 unit) grid.  In a cell file,
scaling may be applied.  In particular, the default for CIF and native
cell files is 100 units per micron, but this is usually changed to
1000 units with the {\vt RESOLUTION 1000} directive.  Anyway, if a
user is for some reason writing a node property string by hand, the
hot spot locations must be chosen appropriately, arbitrary locations
do not work.

Anything that follows is actually ignored by the reader.  The
terminal's name is printed in output since it might be of interest to
humans.

Cell property\\
\begin{quote}{\vt
5 10} {\it circuit\_node index elecX elecY\/} [{\vt 0x}{\it flagstype
 name phyX physY layer\_name\/}];
\end{quote}

Cell property, old 3.2 syntax\\
\begin{quote}{\vt
5 10} {\it circuit\_node index elecX elecY\/} [{\it name phyX physY
 flags layer\_name type\_name\/}];
\end{quote}

The {\et node} properties applied to a cell make it possible for the
cell to be instantiated and used as a subcircuit (or device) in
another cell.

The property string parser can recognize and read the old release 3.2
string format for compatibility.  When writing output in any file
format, if the variable {\et Out32nodes} is set, the old string format
will be generated.  This will, however, strip out multiple contact
points if any have been defined, as this is not supported by the older
format, which allows exactly one contact per node.  The variable
tracks the {\cb Use back-compatible format (warning!  data loss)}
check box in the {\cb Export Control} panel from the {\cb Export
Cell Data} button in the {\cb Convert Menu}.

The {\it circuit\_node} is the node number in the current cell where
contact is to be made.  In a device cell that has no internal nodes,
this will be -1.  The {\it index} is an ordering parameter as
discussed above.  Index zero is the reference node.  When the device
or subcircuit is placed in a schematic, the location of the reference
node corresponds to where the user clicks.

The {\it elecX} and {\it elecY} each have the general form
\begin{quote}
{\it schemX\/}[{\vt :}{\it symbX\/}[{\vt ,}{\it symbX} ...]]
\end{quote}
and similar for the Y values.  This represents a single X,Y contact
location in the schematic, and an arbitrary number of contact
locations in the schematic symbol.  The schematic value is separated
from the symbolic values by a colon.  The symbolic values are
separated from each other by commas.  If the 3.2 format is being
written due to the {\et Out32nodes} variable being set, at most one
number will appear following the colon, the first that would otherwise
be listed if there are multiple contact points.

If there is no symbolic representation, or the terminal location has
not been set in the symbolic view, the {\it elecX} and {\it elecY}
each consist of a single number.  In the more general case, both terms
should supply the same number of integers.

The remaining tokens are optional.  The {\it flagstype} is a hex
integer that if present {\bf must} be prefixed with ``{\vt 0x}'' or
``{\vt 0X}''.  The least significant byte contains a value that
specifies a terminal type.  This is a numerical equivalent of the
optional {\it type\_name} which appears in the old 3.2 format
syntax.  The values and keywords are listed below.  {\Xic} does not
presently use this.

\begin{quote}
\begin{tabular}{ll}\\
\bf value & \bf keyword\\
0 & {\vt input} (default)\\
1 & \vt output\\
2 & \vt inout\\
3 & \vt tristate\\
4 & \vt clock\\
5 & \vt outclock\\
6 & \vt supply\\
7 & \vt outsupply\\
8 & \vt ground\\
\end{tabular}
\end{quote}

The remainder of the word may contain any of the following flag bits.

\begin{description}
\item{\vt 0x100} (BYNAME)\\
The terminal will associate to a wire net by name, there will be no
connectivity due to placement location in the schematic.

\item{\vt 0x200} (VIRTUAL)\\
The terminal is ``virtual'' meaning that there is no wire vertex or
subcircuit or device contact at the terminal's location in the
schematic.  This is irrelevant if the BYNAME flag is set.

\item{\vt 0x400} (FIXED)\\
It set, {\Xic} will not move the corresponding physical terminal
location in the layout.  This indicates that the location has been
``locked'' by the user.

\item{\vt 0x800} (INVIS)\\
The terminal will be invisible in the schematic, except when the {\cb
subct} command from the side menu, used for terminal editing, is
active.  The corresponding terminal in the layout will show normally.
\end{description}

The {\it name} is the terminal name.  This is either a short name
provided by the user, or if not provided a default name will be
created by {\Xic}.  The name is unique among the cell's terminals.

If the cell has a physical counterpart, the remaining arguments have
significance.  In particular, if the cell has no physical counterpart,
or the node has no physical counterpart, the remaining parameters
should not appear.  The lack of physical coordinates informs the
reader that this terminal has no physical counterpart.  The
coordinates should be set to zero in device cells that have a physical
implementation.

If the node has a corresponding physical implementation in the layout,
the {\it physX} and {\it physY} will be given.  When the property is
being read, the presence of these numbers indicates that internal
setup to link to the layout is required.  If both numbers don't
appear, the node will exist in the schematic only.  This is
appropriate for devices that don't have a physical implementation,
such as voltage sources, or for cases like the phase node of a
Josephson junction, or in the case where there is no layout.  In
output, if the physical association exists.  the two numbers give the
corresponding point in the layout.  These will be nonzero if the
corresponding location has been identified, either by running
extraction, or if the location was set by hand.

It is not necessarily true that all nodes of the device either have or
don't have the optional parameters.  The phase node of a Josephson
junction device, for example, does not have a physical counterpart. 
The other two nodes do have physical implementations, since these are
the physical connection points.  A side-effect is that in SPICE files
extracted from physical data only the two nodes will appear in the
device instantiation lines.  This is acceptable to {\WRspice},
since the phase node is optional.  If a device has no nodes with the
optional parameters given, then it can never have a physical
counterpart.  The {\et nophys} property should also be given in that
case.  This is true for devices like voltage sources that have no
physical implementation.

If the physical location is valid, a {\it layer\_name} will be provided. 
This is the name of a physical layer which has the {\et Conductor}
attribute, and an object on this layer touches or is under the
physical location.

In the old 3.2 format, the flags values are the following:
\begin{description}
\item{\vt 0x2} (VIRTUAL)\\
The terminal is ``virtual'' meaning that there is no wire vertex or
subcircuit or device contact at the terminal's location in the
schematic.  This is irrelevant if the BYNAME flag is set.

\item{\vt 0x4} (FIXED)\\
It set, {\Xic} will not move the corresponding physical terminal
location in the layout.  This indicates that the location has been
``locked'' by the user.
\end{description}

In the old format, instead of a numerical type code, an optional {\it
type\_name} can be given.  This is one of the keywords from the table
shown earlier.

\index{name property}
\index{properties!name}
\item{\et name} property, number 11\\
The {\et name} property gives the device an identifying prefix or
name.  If a name has been assigned to the device with the {\cb
Property Editor} panel or equivalent, that name will be used in SPICE
output.  Otherwise, the name prefix is suffixed with a unique integer
generated by {\Xic} to form the name.  SPICE expects that the first
character of the name match the convention for the device, for
example, resistors use R, capacitors C, etc.~(see the SPICE
documentation).

Cell property, {\vt macro} flag set:
\begin{quote}
{\vt 5 11} {\it prefix} {\vt macro}
\end{quote}
Cell property, {\it prefix} starts with {\vt X} or {\vt x}:
\begin{quote}
{\vt 5 11} {\it prefix} {\vt 0 subckt}
\end{quote}
Cell property, otherwise:
\begin{quote}
{\vt 5 11} {\it prefix}
\end{quote}

Cell instance property:
\begin{quote}
{\vt 5 11} {\it prefix\/}.{\it assigned\_name\/} {\it devnum}
[{\vt subckt} [{\it physX physY\/}]];
\end{quote}

The {\it prefix} is the default name prefix, and should conform to the
SPICE conventions.  The {\it assigned\_name\/}, if present, will be
used in actual spice output.  The {\it assigned\_name} should {\bf
not} be present in device definitions, it is used in cell files for
device instances to which a name has been assigned with the {\cb
Property Editor}.  The {\it prefix} can start with any character, but
is intended to have significance to SPICE.  The character `{\vt @}' is
reserved for the terminal device.  The {\it assigned\_name\/} can be
any contiguous string.  The {\it devnum} is an index assigned by
{\Xic} to the device, and is used when forming the default device
name.  When reading, this value is ignored.

The {\et name} property for cells contains an internal {\vt macro}
flag, which replaces the {\et macro} property in 4.3.6 and later. 
This flag will be set if any of the following apply:
\begin{enumerate}
\item{Exactly two words are given, i.e., a single word follows the
{\it prefix\/}, which can be anything.}

\item{The word ``{\vt macro}'' appears in a third word, following an
integer.}

\item{A {\et macro} property is found.}
\end{enumerate}
This syntax is backwards compatible with release 4.3.5 and earlier.

When the {\vt macro} flag is set, its only effect is that in SPICE
output, an `{\vt X}' is prepended to the device name in device
instantiation lines.  Thus, SPICE will treat the device instance as a
subcircuit call.  These instances must have a {\et model} property
that will match a {\vt .subckt} definition somewhere, likely from a
PDK device model file.

This accounts for devices that are likely the electrical part of
parameterized cells, that implement nonlinear behavior through a
network of controlled sources expressed as a subcircuit in the SPICE
model definitions file.  MOS capacitors and poly resistors are devices
that are frequently modeled this way.

If the {\vt macro} flag is set and the {\it prefix} already begins
with {\vt X} or {\vt x}, the device is taken as a macro, meaning that
{\Xic} will not output a subcircuit definition for the cell, and a
{\et model} property will provide the name of a subcircuit definition
expected to be found in the model library or elsewhere.

The name property for instances is printed as shown.  The ``{\vt
subckt}'' appears if {\it prefix} starts with ``{\vt X}'' or ``{\vt
x}'' and {\et macro} is {\bf not} set in the master.  Coordinates
additionally appear if a physical label was placed (in extraction). 
This is where the physical subcircuit instance label is located.  All
but the {\it prefix} and {\it assigned\_name} (if any) are ignored by
the reader, but can be seen printed in native cell files (for example).

{\Xic} generates the internal device or subcircuit index, used as part
of the default device or subcircuit name, according to the position of
the upper-left corner of the bounding box of the object.  The
numbering starts with zero, and increases for positions with smaller Y
value, or with larger X value for devices with the same Y coordinate. 
Each device and subcircuit type has its own numbering.

\index{labloc property}
\index{properties!labloc}
\item{\et labloc} property, number 12\\
The {\et name}, {\et model}, {\et value}, {\et param}, and {\et
devref} property values are normally displayed on-screen near the
device body.  This is a device property for setting the default
locations of the property labels when shown on-screen.  If this
property does not appear, the internal default locations are used. 
This property allows more control over label placement, on a
per-device basis.  This property should only be used in devices in the
device library file.  Presently, the property can only be added with a
text editor by editing the property strings in the device library
file.

\begin{quote}\vt
5 12 {\it pname code} [ {\it pname code\/} ] ... ;
\end{quote}

The {\it pname} is one of the literal tokens ``{\vt name}'', ``{\vt
model}'', ``{\vt value}'', ``{\vt param}'', and ``{\vt devref}''.  For
backward compatibility, ``{\vt initc}'' is accepted as an alias for
``{\vt param}''.  The {\it code} is an integer, -1 -- 23.  If the {\it
code} is -1, the default placement is used.  If code is 0 -- 23, the
placement and justification are as shown in the figure:  The '.'
position implies the justification.  Horizontally, all are left or
right justified except for 16 and 19 which are centered.  Similarly,
vertical justification is bottom or top except for 17 and 18 which are
center justified.

\begin{figure}
\caption{\label{cpos}Locations and justification for character position 
codes around the device bounding box.}
\vspace{1.5ex}
\begin{center}
\input{images/cpos33.latex}
\end{center}
\end{figure}

The default locations are shwon in the table below.  The locations
differ when the height of the placement bounding box is less than or
greater than the width.
 
\begin{tabular}{|l|l|l|}
\bf Property & \bf height $>$ width & \bf height $<=$ width\\ \hline
\et name & 5 & 2\\ \hline
\et model & 8 & 13\\ \hline
\et value & 8 & 13\\ \hline
\et param & 11 & 14\\ \hline
\et devref & 18 & 12\\ \hline
\end{tabular}

\index{oldmut property}
\index{properties!oldmut}
\item{\et oldmut} property, number 13\\
This property is used for compatibility with the mutual inductors used
in the schematic files produced by the {\et Jspice3} program.  The
format should not be used, and is not documented.

\index{mut property}
\index{properties!mut}
\item{\et mut} property, number 14\\
This property appears with the properties of cells containing mutual
inductors, and is not copied to instantiations of the cell.  Mutual
inductors do not appear as devices in the device library file, rather,
they are implemented with this property.  Mutual properties are
generated by selecting the ``{\et mut}'' device from the device
menu, with one property assigned for each mutual inductor pair in
the circuit.
\begin{quote}\vt
5 14 {\it num name1 num1 name2 num2 coeff} [{\it name\/}];
\end{quote}

This property appears only in the list for cell definitions, and not
for instances.  It defines a mutual inductor pair within the cell. 
The {\it num} is the index of the mutual inductor pair, used in
forming the default specification to SPICE: ``{\vt K}{\it num\/}''. 
However, if the {\it name} appears (supplied in {\Xic} by using the
label editor on a mutual inductor label), the SPICE specification will
use {\it name\/} (without {\it num\/}).  The {\it name1, num1, name2,
num2} are the prefixes and assigned numbers of the inductors in the
mutual inductor pair.  The {\it coeff} is a string which represents
the coupling factor as given to SPICE.

\index{branch property}
\index{properties!branch}
\item{\et branch} property, number 15\\
The {\et branch} property is used to define a ``hot spot'' that when
clicked on yields a device parameter, such as device current, which
can be used in plots.  In SPICE, voltage sources and inductors have
internal storage for current values present by default.  Other device
parameters may require additional computational or storage overhead. 
If the {\et branch} property is given in the device definition in the
device library file, it is added to instantiated devices by {\Xic}.
\begin{quote}\vt
5 15 {\it x y dx dy} [{\it string\/}];
\end{quote}

The {\it x} and {\it y} values specify the hot spot where the branch
current can be accessed by clicking.  The next two numbers specify the
assumed direction of current flow.  They are interpreted as a unit
vector directed outward from the origin along the $+/-$ x or y axes. 
Thus,
\begin{quote}
\begin{tabular}{rrr}
direction & \it dx & \it dy\\
$+$y &  0 & 1\\
$-$y &  0 & $-$1\\
$+$x &  1 & 0\\
$-$x & $-$1 & 0
\end{tabular}
\end{quote}
are the options.  The {\it string} will be expanded and added to
the token list in the prompt line when the branch is selected for
plotting.

When the hot spot is clicked on, an expression will be produced which
after expansion is added to the input line in the {\cb plot} command. 
The {\it string} token can contain the following literal tokens, which
will be replaced with the appropriate values during expansion:

\begin{tabular}{ll}\\
\vt <v>      & Voltage across the device\\
\vt <value>  & The ``value'' property\\
\vt <name>   & The device name
\end{tabular}\\

Anything else in the {\it string} will be copied literally.  If the
string is absent, the expression will be ``{\vt <name>\#branch}''.

Here are some examples.
for a resistor, the string is
\begin{quote}
 {\vt <v>/<value>}
\end{quote}
to return the current.  Similarly for a capacitor,
\begin{quote}
 {\vt <value>*deriv(<v>)}.
\end{quote}
Thus the current will be computed using the {\WRspice} {\vt deriv}
function.  For an inductor or voltage source, no string is required,
as the default
\begin{quote}
 {\vt <name>\#branch}
\end{quote}
is appropriate.  For a current source, one can use
\begin{quote}
 {\vt @<name>[c]}.
\end{quote}
This works through the {\WRspice} {\vt @{\it device\/}[{\it param\/}]}
mechanism, however the vector must be saved, most conveniently by
setting the {\et LibSave} global property for the device (see
\ref{devdotlib}).

\index{labrf property}
\index{properties!labrf}
\item{\et labref} property, number 16\\
The {\et labrf} property is applied by {\Xic} to labels that are
associated with device properties or wire nodes.  The property is not
used in the device library file.

\begin{quote}\vt
5 16 {\it name num property\/};
5 16 {\it x y} {\vt 10};
\end{quote}
This property applies only to labels, and indicates that the label is
to be bound to a given property of a certain device or mutual
inductor, or wire.

The first form applies to a label for an instance or mutual inductor
property.  Bound labels automatically reflect changes in the
underlying property string, and may be used to set the string using
the label editing function in {\Xic}.  The {\it name} and {\it num}
are the device prefix and assigned number of the device to which the
label is bound.  The {\it property} is the property number of the
bound property.  If the label is assigned to a mutual inductor pair,
the {\it name} is `{\vt K}'.

The second form applies to labels that have been attached to a wire,
and are used to contribute a name for the net containing the wire. 
The {\it x} and {\it y} specify the coordinates of a vertex of the
wire.  The {\vt 10} is the value of the {\et node} property.

\index{mutlrf property}
\index{properties!mutlrf}
\item{\et mutlrf} property, number 17\\
This property is assigned to inductor instances which are referenced
for use in mutual inductor pairs.  One such property exists per
reference.  It is not used in device library files.
\begin{quote}\vt
5 17 mutual;
\end{quote}
This property applies only to inductors that are referenced as one of
a mutual inductor pair.  There can be several such properties if the
inductor is associated with multiple mutual inductor pairs.

\index{symbolic property}
\index{properties!symbolic}
\index{nosymb property}
\index{properties!nosymb}
\item{\et symbolic} property, number 18\\
The {\et symbolic} property is a property applied to cells which have
a symbolic view associated.  It does not appear in device library
files, as all devices are essentially symbolic.  It is not inherited by
instances.
\begin{quote}\vt
5 18 0/1 {\it geometry\_spec};
\end{quote}
The third field is nonzero if the symbolic mode is active, and 0
if symbolic mode is inactive.  The {\it geometry\_spec} is a string of
separated CIF primitives for the symbolic representation, which can
include L, B, P, W, and 94 (label) directives.  Each record (CIF
primitive) is terminated by a colon ({\bf not} a semicolon!) which
must be immediately followed by an end-of-line character.  Colons that
are not at the end of a text line will {\bf not} terminate a record.

The {\et symbolic} property may be applied to subcircuit instances, in
which case it will negate the effect of a {\et symbolic} property
found in the instance master cell.  In this utilization, the property
is named {\cb nosymb}.  The flag and {\it geometry\_spec} are ignored
and need not be provided.  A subcircuit instance with this property
would (in all cases) be displayed as expanded.  Thus, it is possible
in {\Xic} to have different instances of the same subcircuit cell
master display symbolically and expanded within the same containing
cell.

\index{nodemap property}
\index{properties!nodemap}
\item{\et nodemap} property, number 19\\
The {\et nodemap} property is applied to the electrical cell
definition and is not inherited by instances.  The {\et nodemap}
property provides a mapping between internally generated node numbers
and assigned textual names. 

\begin{quote}\vt
5 19 0/1 {\it name x y name x y ...\/};
\end{quote}

The third token can be 0 or 1, but is unused.  In releases prior to
3.1.5, a 0 value would disable the node map.  In current releases,
node mapping is always enabled.

The remainder of the line consists of triples containing an assigned
name and a coordinate pair.  The coordinates correspond to a device or
subcircuit terminal connected to the assigned node, and serve as the
reference to that node.  See \ref{nodmp} for more information on node
mapping.
\end{description}

The ``global'' properties are added to the electrical top level cell
of a hierarchy when being saved.  They save plot points and other
information in the file, to use as defaults when the file is
subsequently loaded for editing.

\begin{description}
\index{properties!run}
\index{run property}
\item{\et run} property, number 7101\\
The {\et run} property string specifies the default analysis command
entered when the {\cb run} button is pressed which initiates a
simulation.

\index{plot property}
\index{properties!plot}
\item{\et plot} property, number 7102\\
The {\et Plot} property is used only in electrical mode.  The string
represents the plot points used in the {\cb plot} command, in the
format of arguments to the {\WRspice} {\vt plot} command.

\index{iplot property}
\index{properties!iplot}
\item{\et iplot} property, number 7103\\
The {\et Iplot} property is used in electrical mode.  The string
specifies the points to plot when using the {\cb iplot} button, in the
format of arguments to the {\WRspice} {\vt plot} command.
\end{description}

The strings for the {\et plot} and {\et iplot} properties may contain
special escape sequences indicating hypertext references or other
characteristics.  These are described in \ref{prpescapes}.


% -----------------------------------------------------------------------------
% spiceline 030415
\label {spiceline}
\index{spice device line}
In SPICE, Each line of a given device type begins with the device
name, set by the {\et name} property.  This is followed by the device
nodes, corresponding to the order of enumeration in the {\it
device\_node\/} of the {\et node} properties.  This is followed by
text from the {\et devref} property, which is intended to provide a
reference device name for current-controlled sources and the
current-controlled switch.  It is not used generally.  This is
followed by the {\et value} or {\et model} property (these are really
just two different names for the same text field).  This is followed
by the text of the {\et param} property.

The device name, if not assigned by the user with the {\cb Property
Editor} command, and nodes are assigned by {\Xic} so as to be unique.

The line looks like:
\begin{quote}
{\it name n1 ... nLast devref value\/}/{\it model parameter\_string}
\end{quote}

The {\it name} is either the user assigned name, or the device prefix
with a unique numerical suffix created by {\Xic} if no name was
assigned.  The nodes can be numbers or text tokens, in accordance with
the current node name mapping (see \ref{nodmp}).  The remaining
properties are read verbatim from the specifications, with hypertext
references expanded.

Hypertext references are generated when assigning properties by
clicking on devices or other features in the drawing.  Since {\Xic}
assigns device names and nodes, if one needs to reference a specific
device or node, a hypertext reference provides a link which is
independent of the assigned values, which can change.

When applied to subcircuit cells and instances, the {\cb param}
property provides support for subcircuit parameterization, which is
available in {\WRspice} and some other simulators.

Here is a brief description of how to use parameterization.  Suppose
that you are editing a cell that contains a resistor, and you wish to
parameterize the resistance value.  Give the resistor a {\et value}
property consisting of some word, say ``{\vt rshunt}''.  Using the
{\cb Cell Properties Editor}, give the cell a {\et param} property
something like ``{\vt rshunt=2.5}''.  This will give the resistor a
default value of 2.5 ohms.  Editing another cell, place two instances
of the previous cell.  Using the {\cb Property Editor} give one of the
instances a {\et param} property of ``{\vt rshunt=5}''.  A label will
appear containing this text.  The other instance will not have a
similar label.  The resistor in the labeled subcircuit will have value
5, set by the {\et param} property applied to the instance.  The
other instance will have resistance value 2.5, as set by the {\et
param} property applied to the master, which serves as the default
value.


% -----------------------------------------------------------------------------
% prpescapes 062908
\section{Special Escapes}
\label{prpescapes}
\index{hypertext reference format}

In property and label strings, there is a special encoding used to
indicate certain attributes, such as hypertext references.  These are
in the form:
\begin{quote}
{\vt (||}{\it something\/}{\vt ||)}
\end{quote}

The following forms are recognized by {\Xic}
\begin{description}
\item{\vt (||sc||)}\\
This sequence is simply converted to a semicolon (`{\vt ;}')
when the string is internalized.  In CIF, semicolons can not be
included in label or property strings, as the character is
reserved for line termination.  {\Xic} will convert
semicolons in property strings and labels to this form when
creating a CIF or native file.

\item{\vt (||text||)}\\
This token may appear at the beginning of a label string, and
indicates that the string is in long text format (see \ref{longtext}). 
These labels do not appear on-screen (the characters ``{\vt [text]}''
appear instead), but the full string can be accessed with the label
editor.  Thus, large blocks of text can be saved as properties or {\vt
spicetext} labels without crowding the screen.

\item{\vt (||}{\it x\/}:{\it y type\/}{\vt ||)}\\
This sequence indicates a hypertext reference, and can appear anywhere
in a property or label string, in electrical data only.  Hypertext
references are generated when assigning properties by clicking on
other devices in the drawing.  Since {\Xic} by default internally
assigns device names and nodes, if one needs to reference a specific
device or node, a hypertext reference provides a link which is
independent of the assigned values, which can change.  The {\it
x},{\it y} is a coordinate, in internal units, giving a location for
the reference.  This is generally the point where the user clicked to
create the reference.  The space-separated integer that follows gives
the type of the reference, and is one of:

\begin{tabular}{ll}
1 & node reference\\
2 & branch reference\\
4 & device name reference\\
8 & subcircuit name reference\\
\end{tabular}

In the case of a node reference, the coordinate must be over a
connection point, or along a wire.  For a branch, the coordinate must
be over a branch reference point of a device.  For a device name
reference, the coordinate must be in or on the bounding box of a
device.  For a subcircuit name reference, the coordinate must be in or
on the bounding box of a subcircuit.

When the string is used, the hypertext reference is resolved, and
the actual text replaces the hypertext reference in the string.
\end{description}

