----------------------------------------------------------------------------
% Xic Manual
% (C) Copyright 2009, Whiteley Research Inc., Sunnyvale CA
% $Id: sidemenu.tex,v 1.83 2016/10/10 18:29:59 stevew Exp $
% -----------------------------------------------------------------------------

% -----------------------------------------------------------------------------
% sidemenu 020715
\chapter{The Side Menu:  Geometry Creation}
\label{sidemenu}
\index{side menu}

{\Xic} has a ``side'' menu of buttons, typically displayed along the
left edge of the application main window, next to the layer table. 
This menu contains buttons specific to editing, and is shown only when
editing is enabled (meaning that it never appears in the {\Xiv}
feature set).  The content of the menu differs between electrical and
physical modes.

If the environment variable {\et XIC\_MENU\_RIGHT} is set when {\Xic}
starts, the menu and layer table will be placed along the right edge
of the main application window.  This might be more convenient for
left-handed users.  If the {\et XIC\_HORIZ\_BUTTONS} environment
variable is set, the ``side'' menu buttons will instead be arrayed
horizontally across the top of the main application window, above the
top button menu.

This section describes in detail the commands available in the side
menu in physical and electrical modes.  These include commands for
geometry creation and other frequently used operations.

Again, the side menu is only visible when cell editing is possible.

Side menu commands are executed by clicking with button 1 on the
buttons.  Typing the first few letters of the command name while
pointing in a drawing window will also initiate a side menu command. 
The characters typed are displayed in the key press buffer area to the
left of the prompt line in the main window, or in the upper-right
corner of sub-window pop-ups.  Commands can be exited by selecting the
same or another command in most cases, or by pressing the {\kb Esc}
key.

\index{current transform}
In the command descriptions, reference if often made to the ``current
transform''.  This is a rotation, reflection, and magnification
specification for moved or copied objects, and for newly created
subcells.  The current transform is set with the pop-up produced by
the {\cb xform} button in the side.

Reference is also made to ``selected'' objects.  Objects are selected
by clicking the left mouse button (button 1) while pointing at the
object, or by pressing and holding button 1 so that the object is
enclosed in the rectangle formed with the press and release locations. 
Selecting a second time will deselect the objects, and all selected
objects can be deselected with the {\cb desel} button in the top
button menu.  Selected objects are displayed with a blinking
highlighted border.  Objects can also be selected with the {\cb
!select} command typed in the prompt area.

Reference is made to various commands that start with an exclamation
point ``{\cb !}'' such as ``{\cb !set}''.  These commands can be
entered from the keyboard.  Since most of these commands are used
infrequently, they are not assigned command buttons.  The most
important of these commands is probably {\cb !set}, since this allows
certain variables to be set which control the behavior of some side
menu commands.  These ``{\cb !}'' commands are described in chapter
\ref{bangcmds}.

The tables below summarize the command buttons provided in the side
menus in physical and electrical mode.  Note that the side menu is
different between physical and electrical modes, and that the
operation of some commands which appear in both may differ slightly. 
These differences are noted in the descriptions.  In the text, side
menu commands are referenced by their internal names, since the
command buttons contain an icon and not a label.

The side menu is not available in the {\Xiv} feature set, and is
invisible when certain modes are in effect, such as in CHD display
mode, where editing is not allowed.

\begin{table}
\begin{center}
\begin{tabular}{|l|l|l|} \hline
\multicolumn{3}{|c|}{\kb Physical Side Menu} \\[0.5ex] \hline
\et Icon & \et Name & \et Function\\ \hline
\fbox{\epsfbox{images/xform.eps}} & \rb{\vt xform} &
  \rb{Set current transform}\\ \hline
\fbox{\epsfbox{images/place.eps}} & \rb{\vt place} &
  \rb{Place subcells}\\ \hline
\fbox{\epsfbox{images/label.eps}} & \rb{\vt label} &
  \rb{Create/edit labels}\\ \hline
\fbox{\epsfbox{images/logo.eps}}  & \rb{\vt logo}  &
  \rb{Create text object}\\ \hline
\fbox{\epsfbox{images/box.eps}}   & \rb{\vt box}   &
  \rb{Create rectangles}\\ \hline
\fbox{\epsfbox{images/polyg.eps}} & \rb{\vt polyg} &
  \rb{Create polygons}\\ \hline
\fbox{\epsfbox{images/wire.eps}}  & \rb{\vt wire}  &
  \rb{Create wires}\\ \hline
\fbox{\epsfbox{images/style.eps}} & \rb{\parbox{1cm}{\vt 
 style\linebreak{\rm menu}}} & \rb{Set wire style}\\ \hline
\fbox{\epsfbox{images/round.eps}} & \rb{\vt round} &
  \rb{Create disk objects}\\ \hline
\fbox{\epsfbox{images/donut.eps}} & \rb{\vt donut} &
  \rb{Create disk with hole}\\ \hline
\fbox{\epsfbox{images/arc.eps}}   & \rb{\vt arc}   &
  \rb{Create arcs}\\ \hline
\fbox{\epsfbox{images/sides.eps}} & \rb{\vt sides} &
  \rb{Set rounded granularity}\\ \hline
\fbox{\epsfbox{images/xor.eps}}   & \rb{\vt xor}   &
  \rb{Exclusive-OR objects}\\ \hline
\fbox{\epsfbox{images/break.eps}} & \rb{\vt break} &
  \rb{Cut objects}\\ \hline
\fbox{\epsfbox{images/erase.eps}} & \rb{\vt erase} &
  \rb{Erase geometry}\\ \hline
\fbox{\epsfbox{images/put.eps}}   & \rb{\vt put}   &
  \rb{Paste from yank buffer}\\ \hline
\fbox{\epsfbox{images/spin.eps}}  & \rb{\vt spin}  &
  \rb{Rotate objects}\\ \hline
\end{tabular}
\begin{tabular}{|l|l|l|} \hline
\multicolumn{3}{|c|}{\kb Electrical Side Menu} \\[0.5ex] \hline
\et Icon & \et Name & \et Function\\ \hline
\fbox{\epsfbox{images/xform.eps}} & \rb{\vt xform} &
  \rb{Set current transform}\\ \hline
\fbox{\epsfbox{images/place.eps}} & \rb{\vt place} &
  \rb{Place subcells}\\ \hline
\fbox{\epsfbox{images/devs.eps}}  & \rb{\vt devs}  &
  \rb{Show device menu}\\ \hline
\fbox{\epsfbox{images/shapes.eps}} & \rb{\parbox{1cm}{\vt
  shapes\linebreak{\rm menu}}} & \rb{Create outline object}\\ \hline
\fbox{\epsfbox{images/wire.eps}}  & \rb{\vt wire}  &
 \rb{Create wires}\\ \hline
\fbox{\epsfbox{images/label.eps}} & \rb{\vt label} &
 \rb{Create/edit labels}\\ \hline
\fbox{\epsfbox{images/erase.eps}} & \rb{\vt erase} &
 \rb{Erase geometry}\\ \hline
\fbox{\epsfbox{images/break.eps}} & \rb{\vt break} &
 \rb{Cut objects}\\ \hline
\fbox{\epsfbox{images/symbl.eps}} & \rb{\vt symbl} &
 \rb{Set symbolic mode}\\ \hline
\fbox{\epsfbox{images/nodmp.eps}} & \rb{\vt nodmp} &
 \rb{Name wire nets}\\ \hline
\fbox{\epsfbox{images/subct.eps}} & \rb{\vt subct} &
 \rb{Set subcircuit contacts}\\ \hline
\fbox{\epsfbox{images/terms.eps}} & \rb{\vt terms} &
  \rb{Show terminals}\\ \hline
\fbox{\epsfbox{images/spcmd.eps}} & \rb{\vt spcmd}  &
  \rb{Execute {\WRspice} command}\\ \hline
\fbox{\epsfbox{images/run.eps}}   & \rb{\vt run}   &
  \rb{Run {\WRspice}}\\ \hline
\fbox{\epsfbox{images/deck.eps}}  & \rb{\vt deck}  &
  \rb{Save SPICE file}\\ \hline
\fbox{\epsfbox{images/plot.eps}}  & \rb{\vt plot}  &
  \rb{Plot SPICE results}\\ \hline
\fbox{\epsfbox{images/iplot.eps}} & \rb{\vt iplot} &
  \rb{Set dynamic plotting}\\ \hline
\end{tabular}
\end{center}
\caption{\label{sidetab}Commands found in the side menu in physical
and electrical modes.}
\end{table}

\newpage

% -----------------------------------------------------------------------------
% xic:arc 120615
\section{The {\cb arc} Button: Create Arcs}
\index{arc button}
\index{object creation!arcs}
\epsfbox{images/arc.eps}

The {\cb arc} command button allows the user to create arcs on the
current layer.  The {\cb sides} button, or the {\cb Sides} entry in
the {\cb shapes} menu in electrical mode, can be used to reset the
number of segments used to represent the circle containing the arc. 
Press button 1 first to define the center.  Subsequent presses, (or
drag releases) define the inner and outer radii, the arc start angle,
and the arc terminal angle.  In physical mode, if the arc path width
is set to zero, a round disk is created, as with the {\cb round}
button.  If the angle given is 360 degrees, then the created figure is
identical to that produced by the {\cb donut} button.  In electrical
mode, the arc function is entered through the {\cb arc} entry in the
menu brought up with the {\cb shapes} button.  In this case, the arc
path has no width, so that the inner and outer radii are equal and not
separately definable.  Arcs have no electrical significance, but can
be used for illustrative purposes.

While the command is active in physical mode, the cursor will snap to
horizontal or vertical edges of existing objects in the layout if the
edge is on-grid, when within a small distance.  When snapped, a small
dotted highlight box is displayed.  This makes it much easier to
create abutting objects when the grid snap spacing is very fine
compared with the display scaling.  This feature can be controlled
from the {\cb Edge Snapping} group in the {\cb Snapping} page of the
{\cb Grid Setup} panel.

In electrical mode, an arc is actually a wire, and as such should not
be used on the SCED layer.  If the current layer is the SCED layer,
the arc will be created using the ETC2 layer, otherwise the arc will
be created on the current layer.  Although there is no error, arc
vertices on the SCED layer are considered in the connectivity
establishment, leading to inefficiency.  If the user insists on the
arc being on the SCED layer, the {\cb Change Layer} command in the
{\cb Modify Menu} can be used to move it to that layer.

If the user presses and holds the {\kb Shift} key after the center
location is defined, and before the perimeter is defined by either
lifting button 1 or pressing a second time, the current radius is held
for x or y.  The pointer location of the {\kb Shift} press defines
whether x is held (pointer closer to the center y) or y is held
(pointer closer to the center x).  This allows elliptical arcs to be
generated.  This similarly applies when defining the outer radii, so
that the inner and outer surfaces can have different elliptical aspect
ratios, though the outer radius must be larger than the inner radius
at all angles.

The {\kb Ctrl} key also provides useful constraints.  Pressing and
holding the {\kb Ctrl} key when defining the radii produces a radius
defined by the pointer position projected on to the x or y axis
(whichever is closer) defined from the center.  Otherwise, off-axis
snap points are allowed, which may lead to an unexpected radius on a
fine grid.  When defining the angles of arcs with the {\kb Ctrl} key
pressed, the angle is constrained to multiples of 45 degrees. 
Ordinarily, the arc angle snaps to the nearest snap point.

When the command is expecting a mouse button press to define a radius,
the value as defined by the mouse pointer (in microns) is printed in
the lower left corner of the drawing window, or the X and Y values are
printed if different.  Pressing {\kb Enter} will cause prompting for
the value(s), in microns.  If one number is given, a circular radius
is accepted, however one can enter two numbers separated by space to
set the X and Y radii separately.

Similarly, the angles are displayed, and can be entered in this
manner.  Prompts can be obtained for the start and end angles
separately.  The angle should be entered in degrees.  Zero degress
points along the X axis, and positive angles advance clockwise.


% -----------------------------------------------------------------------------
% xic:box 012715
\section{The {\cb box} Button: Create Rectangles}
\index{box button}
\index{object creation!boxes}
\epsfbox{images/box.eps}

The {\cb box} command button allows creation of boxes (rectangles) on
the currently selected layer.  The box can be defined by either
clicking button 1 on two diagonal corners, or by pressing button 1 to
define the first corner, dragging, then releasing button 1 to define
the second corner.  The outline of the box is ghost-drawn during
creation.  The new box will be merged with or clipped to existing
boxes on the same layer, unless this feature has been suppressed.

While the command is active in physical mode, the cursor will snap to
horizontal or vertical edges of existing objects in the layout if the
edge is on-grid, when within two pixels.  When snapped, a small dotted
highlight box is displayed.  This makes it much easier to create
abutting objects when the grid snap spacing is very fine compared with
the display scaling.  This feature can be controlled from the {\cb
Edge Snapping} group in the {\cb Snapping} page of the {\cb Grid
Setup} panel.

In physical mode, boxes can also be created from the {\cb Show/Select
Devices} panel from the {\cb Device Selections} button in the {\cb
Extract Menu}.  The {\cb Enable Measure Box} button provides a means
of creating boxes of a specific size to match electrical requirements,
for example to create rectangular resistor bodies for a given
resistance.  Boxes can be created whether or not the electrical layer
parameters are used or present.

In physical mode while the {\cb box} command is active, holding down
the {\kb Ctrl} key while clicking on a subcell will paint the area of
the subcell with the current layer.

In electrical mode, the box command is available by selecting the {\cb
box} function in the {\cb shapes} menu.  If the current layer is the
SCED layer, the box will be created using the ETC2 layer, otherwise
the box will be created on the current layer.  It is best to avoid use
of the SCED layer for other than active wires, for efficiency reasons,
though it is not an error.  The {\cb Change Layer} command in the {\cb
Modify Menu} can be used to change the layer of existing objects to
the SCED layer, if necessary.  The outline style and fill will be
those of the rendering layer.  Boxes have no electrical significance,
but can be used for illustrative purposes.

The {\cb box}, {\cb erase}, and {\cb xor} commands participate in a
protocol that is handy on occasion.

Suppose that you want to erase an area, and you have zoomed in and
clicked to define the anchor, then zoomed out or panned and clicked to
finish the operation.  Oops, the {\cb box} command was active, not
{\cb erase}.  One can press {\kb Tab} to undo the unwanted new box,
then press the {\cb erase} button, and the {\cb erase} command will
have the same anchor point and will be showing the ghost box, so
clicking once will finish the erase operation.

The anchor point is remembered, when switching directly between these
three commands, and the command being exited is in the state where the
anchor point is defined, and the ghost box is being displayed.  One
needs to press the command button in the side menu to switch commands. 
If {\kb Esc} is pressed, or a non-participating command is entered,
the anchor point will be lost.


% -----------------------------------------------------------------------------
% xic:break 012815
\section{The {\cb break} Button: Cut Objects}
\index{break button}
\index{object breaking}
\epsfbox{images/break.eps}

The {\cb break} button is used to divide objects along a horizontal or
vertical line.  The command operates on boxes, polygons, and wires. 
If one or more of those objects was previously selected, the break
command will operate on those selections.  Otherwise, the user is
asked to select objects to break.  The user is then asked to click to
divide the selected objects along the break line, which is attached to
the pointer and ghost-drawn.  The orientation of the break line is
either horizontal or vertical, which can be toggled by pressing either
the {\cb /} (forward slash) or {\kb $\backslash$} (backslash) keys
when the break line is visible.  The {\cb break} command is useful
when one wants to relocate or create a subcell from pieces of an
existing design.

While the command is active in physical mode, the cursor will snap to
horizontal or vertical edges of existing objects in the layout if the
edge is on-grid, when within two pixels.  When snapped, a small dotted
highlight box is displayed.  This makes it much easier to create
abutting objects when the grid snap spacing is very fine compared with
the display scaling.  This feature can be controlled from the {\cb
Edge Snapping} group in the {\cb Snapping} page of the {\cb Grid
Setup} panel.

When the {\cb break} command is at the state where objects are
selected, and the next button press would initiate the break
operation, if either of the {\kb Backspace} or {\kb Delete} keys is
pressed, the command will revert the state back to selecting objects. 
Then, other objects can be selected or selected objects deselected,
and the command is ready to go again.  This can be repeated, to build
up the set of selections needed.

At any time, pressing the {\cb Deselect} button to the left of the
coordinate readout will revert the command state to the level where
objects may be selected to break.

The undo and redo operations (the {\kb Tab} and {\kb Shift-Tab}
keypreses and {\cb Undo}/{\cb Redo} in the {\cb Modify Menu}) will
cycle the command state forward and backward when the command is
active.  Thus, the last command operation, such as initiating the
break by clicking, can be undone and restarted, or redone if
necessary.  If all command operations are undone, additional undo
operations will undo previous commands, as when the undo operation is
performed outside of a command.  The redo operation will reverse the
effect, however when any new modifying operation is started, the redo
list is cleared.  Thus, for example, if one undoes a box creation,
then starts a break operation, the ``redo'' capability of the box
creation will be lost.


% -----------------------------------------------------------------------------
% xic:deck 062313
\section{The {\cb deck} Button:  Save SPICE File}
\index{deck button}
\index{SPICE deck creation}
\epsfbox{images/deck.eps}

The {\cb deck} command, available only in electrical mode, creates a
SPICE file of the current circuit hierarchy.  The file name is
prompted for, as is an analysis string.  If an analysis string is
given, it will be included in the SPICE file after prepending a `.',
unless it happens to start with ``run'', in which case it is ignored. 
If a plot string has been created with the {\cb plot} command, it will
also be included as a {\vt .plot} line. 

Unless the variable {\et SpiceListAll} is set (with the {\cb !set}
command), only devices and subcircuits that are ``connected'' will be
included in the SPICE file.  A device or subcircuit is connected if
any of the following is true:
\begin{itemize}
\item{The subcircuit has a global node.}
\item{The device or subcircuit has two or more non-ground
connections.}
\item{The device or subcircuit has one non-ground connection and one
or more grounds.}
\item{The device or subcircuit has one non-ground connection and no
opens.}
\item{The subcircuit has a non-ground connection.}
\end{itemize}

Note that it is possible for a subcircuit to have no connections on
the {\vt .subckt} line, if it contains a global node.  For example,
the subcircuit might consist of a decoupling capacitor to ground, from
a global power supply node (e.g., ``{\vt vdd!}'').

Node names will be assigned according to the node name mapping (see
 \ref{nodmp} currently in force.

After the new file is created, the user is given the option of viewing
it in a {\cb File Browser} window.

\index{CheckSolitary variable}
If the variable {\et CheckSolitary} is set (with the {\cb !set}
command) then a warning will be issued if nodes are found with only
one connection.


% -----------------------------------------------------------------------------
% xic:devs 020615
\section{The {\cb devs} Button: Device Menu}
\label{devmenu}
\index{devs button}
\index{device menu}
\index{PictorialDevs variable}
\epsfbox{images/devs.eps}

The {\cb devs} button appears only in electrical mode, and pressing
this button will toggle the display of the device selection menu.

There are three styles of the device menu.  The default style contains
a menu bar with four entries:  {\cb Devices}, {\cb Sources}, {\cb
Macros}, and {\cb Terminals}.  Each brings up a sub-menu containing
names of library ``devices'', that fall into each category.

The second menu style is similar, but the menu bar contains the first
letter of the device name (not the SPICE key).

In either of these styles, pressing and holding button 1 while the
pointer is over one of the menu bar buttons will pop up a menu of
device names.  Moving the pointer down the menu will highlight the
entry under the pointer.  A selection can be made by releasing the
button.

The third style is the pictorial menu, which displays the schematic
symbol of each available device, in alphabetical order.  Clicking on
one of the device images will establish the selection.

Each menu style contains a button from which the style can be cycled.

After a selection is made, the device symbol will be ghost-drawn and
attached to the pointer, and the device will be placed at positions
where the user clicks in the drawing windows.  The device is
positioned such that the reference terminal is located at the point
where the user clicked.  Devices are placed according to the current
transform, which is defined from the pop-up produced by the {\cb
xform} button in the side menu.

The devices available and other details depend upon the definitions in
the device library file.  By default, this file is named ``{\vt
device.lib}'', and is located in the installation startup directory,
but this can be superseded by a custom file of the same name which is
found in the library search path ahead of the default file.

The present device menu style tracks, and is tracked by, the {\et
DevMenuStyle} variable.  This variable can be set (with the {\cb !set}
command) to an integer 0--2.  If 0 or unset, the categorized layout is
used.  If 1, the alphabetized variation is used, and 2 specifies the
pictorial menu.  This variable tracks the style of the menu, and
resets the style when set.

The following table lists the devices found in the device library file
supplied with {\Xic}.

\begin{tabular}{|l|l|}\hline
\bf Name  & \bf Description\\ \hline
\multicolumn{2}{|c|}{Contact Devices}\\ \hline
\et gnd & Ground Contact\\ \hline
\et gnde & Alternative Ground Contact\\ \hline
\et tbar & Contact Terminal\\ \hline\hline
\et tblk & Alternative Contact Terminal\\ \hline\hline
\et tbus & Bus Contact Terminal\\ \hline\hline
\multicolumn{2}{|c|}{SPICE Devices}\\ \hline
\et res & Resistor\\ \hline
\et cap & Capacitor\\ \hline
\et ind & Inductor\\ \hline
\et mut & Mutual Inductor\\ \hline
\et isrc & Current Source\\ \hline
\et vsrc & Voltage Source\\ \hline
\et dio & Junction Diode\\ \hline
\et jj & Josephson Junction\\ \hline
\et npn & NPN Bipolar Transistor\\ \hline
\et pnp & PNP Bipolar Transistor\\ \hline
\et njf & N-Channel Junction FET\\ \hline
\et pjf & P-Channel Junction FET\\ \hline
\et nmos1 & N-Channel MOSFET, 4 Nodes\\ \hline
\et pmos1 & P-Channel MOSFET, 4 Nodes\\ \hline
\et nmos & N-Channel MOSFET, 3 Nodes\\ \hline
\et pmos & P-Channel MOSFET, 3 Nodes\\ \hline
\et nmes & N-Channel MESFET\\ \hline
\et pmes & P-Channel MESFET\\ \hline
\et tra & Transmission Line\\ \hline
\et ltra & Transmission Line (LTRA Compatible)\\ \hline
\et urc & Uniform RC Line\\ \hline
\et vccs & Voltage-Controlled Current Source\\ \hline
\et vcvs & Voltage-Controlled Voltage Source\\ \hline
\et cccs & Current-Controlled Current Source\\ \hline
\et ccvs & Current-Controlled Voltage Source\\ \hline
\et sw & Voltage-Controlled Switch\\ \hline
\et csw & Current-Controlled Switch\\ \hline
\multicolumn{2}{|c|}{Misc.}\\ \hline
\et opamp & Example Macro\\ \hline
\et vp & Current Meter\\ \hline
\end{tabular}

The colors used in the pictorial device menu can be changed by setting
the Special GUI Colors (see \ref{attrcolor}) listed below.  This can
be done in the technology file, or with the {\cb !setcolor} command.

\begin{tabular}{|l|l|l|} \hline
\bf variable & \bf purpose  & \bf default\\ \hline
\vt GUIcolorDvBg & background & \vt gray90\\ \hline
\vt GUIcolorDvFg & foreground & \vt black\\ \hline
\vt GUIcolorDvHl & highlight & \vt blue\\ \hline
\vt GUIcolorDvSl & selection & \vt gray80\\ \hline
\end{tabular}

% -----------------------------------------------------------------------------
% not in help
\subsection{Terminal Devices}

The following are not ``real'' devices, though they appear in the
device menu and can be placed in a drawing.  Their purpose is to
establish connectivity.

% -----------------------------------------------------------------------------
% dev:gnd 042611
\subsubsection{Ground Device}
\index{gnd device}

The {\et gnd} device is used to connect to node 0, which is always
taken as the reference (ground) node in SPICE.  This can be placed in
the main circuit and subcircuits.

The device library may contain multiple, functionally identical
``ground'' devices, that differ only visually.  In the library, any
device that has no {\et name} property and exactly one {\et node}
property is taken as a ground device.

% -----------------------------------------------------------------------------
% dev:gnde 042611
\subsubsection{Alternative Ground Device}
\index{gnde device}

The {\et gnde} device is used to connect to node 0, which is always
taken as the reference (ground) node in SPICE.  This can be placed in
the main circuit and subcircuits.  This is functionally identical to
the {\et gnd} device, but differs visually.

% -----------------------------------------------------------------------------
% dev:tbar 062013
\subsubsection{Terminal Device}
\label{devtbar}
\index{tbar device}
The {\et tbar}, {\et tblk}, {\et ttri}, and {\et txbox} are ``terminal
devices'' from the default device library.  These devices behave
identically, and differ only in appearance.  Each device has an
associated label (with text defaulting to the device name) which can
be changed by the user by selecting the label and pressing the {\cb
label} button in the side menu.  The label will supply a name, which
will be applied to a connected net.  All nets connected to a terminal
device with the same name are taken as being connected together.

This will not tie nets between the main circuit and subcircuits, or
between subcircuits, unless the terminal name is also a global net
name.  If not global, the scope is within the cell only.  See
\ref{nodmp} for more information about net name assignments.

Internally, the device will reconfigure itself as a scalar or
multi-contact device according to the label.  Older {\Xic} releases
provided a {\et tbus} terminal, which is no longer compatible.

The name applied to a net via a terminal device is handled
identically to a name obtained from a wire label.

% -----------------------------------------------------------------------------
% dev:tbus 062013
\subsubsection{Bus Terminal Device}
\index{tbus device}
The {\et tbus} terminal device was provided as a bus terminal in older
{\Xic} releases.  It is no longer compatible or supported, and must be
replaced by a current terminal device in legacy schematics.

% -----------------------------------------------------------------------------
% not in help
\subsection{SPICE Devices}

These devices correspond to element lines in SPICE output.  In general,
they reflect the generic SPICE syntax.

% -----------------------------------------------------------------------------
% dev:res 062908
\subsubsection{Resistor Device}
\index{res device}

The {\et res} device is a two-terminal resistor.  Typically, a {\et
value} property is added to specify resistance.  Alternatively, a {\et
model} property can be added to specify a resistor model.  If a {\et
model} property is assigned, then a {\et param} property can be used
to supply the geometric or other parameters.

The `$+$' symbol in the representation accesses a {\et branch}
property that returns a hypertext expression consisting of the voltage
across the resistor divided by the resistance in ohms, yielding the
current through the resistor.  The `O' that follows the resistance is
the `ohms' unit specifier, and {\it not} an extra zero.

% -----------------------------------------------------------------------------
% dev:cap 062908
\subsubsection{Capacitor Device}
\index{cap device}

The {\et cap} device is a two-terminal capacitor.  Typically, a {\et
value} property is added to specify capacitance.  Alternatively, a
{\et model} property can be added to specify a capacitor model.  If a
{\et model} property is assigned, then a {\et param} property can be
used to supply the geometric parameters.  In either case, the {\et
param} property can be used to provide initial conditions.

The `$+$' symbol in the representation accesses a {\et branch}
property that returns a hypertext expression consisting of the
capacitance value times the time-derivative of the voltage across the
capacitor, yielding the capacitor current.

% -----------------------------------------------------------------------------
% dev:ind 062908
\subsubsection{Inductor Device}
\index{ind device}

The {\et ind} device is a two-terminal inductor.  A {\et value}
property should be added to specify inductance.  A {\et param}
property can be used to provide initial conditions.

The `$+$' symbol in the representation accesses a {\et branch}
property that returns a hypertext link to the inductor current vector.

% -----------------------------------------------------------------------------
% dev:mut 062908
\subsubsection{Mutual Inductor}
\index{mut device}
\index{mutual inductors}

The {\et mut} device provides support for mutual inductors.  The {\et
mut} device is never placed.  When the {\et mut} device is selected in
the device menu, rather than selecting a device for placement as do
the other selections, a command mode is entered which allows existing
inductors to be selected into mutual inductor pairs.

When the {\et mut} device is selected, an existing pair of coupled
inductors (if any have been defined) is shown highlighted, and the
SPICE coupling factor printed.  The arrow keys cycle through the
internal list of coupled inductor pairs, or a pair may be selected by
clicking on one of the inductors or the coefficient label with button
1.  At any time, pressing the `{\kb a}' key will allow addition of a
mutual inductor pair.  The same effect is obtained by clicking on a
non-mutual inductor with button 1.  The user is asked to click on the
two coupled inductors (if `{\kb a}' entered or there are no existing
mutual inductors), or the second inductor (if the user clicked on an
inductor), and then to enter the coupling factor.  The coupling factor
can be any string, so as to allow shell variable expansion in
{\WRspice}, but if it parses as a number it must be in the range
between -1 and 1.

Pressing the `{\kb d}' key will delete the mutual inductance
specification for the two inductors currently shown. 

Pressing the `{\kb k}' key will prompt for a new value of the coupling
factor for the mutual inductors shown, as will clicking on the
coefficient label in a drawing window.  When entering the coefficient
string, one can enter either the form {\it name\/}={\it
coef\_string\/}, or simply the coefficient string.  In the first case,
the {\it name} will provide an alternate fixed name for the mutual
inductor in SPICE output.  This can be any alphanumeric name, but
should start with `k' or `K' for SPICE.  If no name is given, {\Xic}
will assign a name consisting of {\vt K} followed by a unique index
integer.

One can also change the coefficient string and/or name with the {\cb
label} button in the side menu.  Again, the label text can have either
of the forms described above.

Pressing the {\kb Esc} key terminates this (and every) command.  One
can back out of the operation if necessary with {\kb Tab} (undo), as
usual. 

% -----------------------------------------------------------------------------
% dev:isrc 062908
\subsubsection{Current Source}
\index{isrc device}

The {\et isrc} device is a general current source.  A {\et value}
and/or {\et param} property can be added to specify the value,
function, or other parameters required by the source.

The arrow head in the representation accesses a {\et branch} property
that returns a hypertext link to the current in the form ``{\vt @}{\it
name}{\vt [c]}''.  A {\vt .save} line for this vector is automatically
added to the SPICE output.

% -----------------------------------------------------------------------------
% dev:vsrc 042611
\subsubsection{Voltage Source}
\index{vsrc device}

The {\et vsrc} device is a general voltage source.  A {\et value}
and/or {\et param} property can be added to specify the value,
function, or other parameters required by the source.

The `$+$' symbol in the representation accesses a {\et branch}
property that returns a hypertext link to the current vector when
clicked on.

% -----------------------------------------------------------------------------
% dev:vp 042611
\subsubsection{Current Meter}
\index{vp device}

In SPICE, voltage sources are often used as ``current meters'', as the
current through a voltage source is saved with the simulation result
vectors, and can be plotted or printed.  The {\et vp} device is
actually a voltage source (identical to a {\vt vsrc} device) however
the symbol size is tiny, so that it can be more easily added to an
existing schematic for use as a current meter.  The symbol contains a
hot spot in the representation that accesses a {\et branch} property
that returns a hypertext link to the current vector when clicked on.

% -----------------------------------------------------------------------------
% dev:dio 062908
\subsubsection{Junction Diode}
\index{dio device}

The {\et dio} device is a junction diode.  A {\et model} property
should be added to specify a diode model.  A {\et param} property can
be added to specify additional parameters.

The diode contains no hidden targets.

% -----------------------------------------------------------------------------
% dev:jj 062908
\subsubsection{Josephson Junction}
\index{jj device}

The {\et jj} device is a Josephson junction.  A {\et model} property
should be added to specify a Josephson junction model.  A {\et param}
property can be added to specify additional parameters.

The `$+$' symbol in the representation accesses the phase node of the
Josephson junction.  The ``voltage'' on this node is equal to the
junction phase, in radians.

% -----------------------------------------------------------------------------
% dev:npn 062908
\subsubsection{NPN Bipolar Transistor}
\index{npn device}

The {\et npn} device is an npn bipolar transistor.  A {\et model}
property should be added to specify a bipolar transistor model.  A
{\et param} property can be added to specify additional parameters.

The bipolar transistor contains no hidden targets.

% -----------------------------------------------------------------------------
% dev:pnp 062908
\subsubsection{PNP Bipolar Transistor}
\index{pnp device}

The {\et pnp} device is a pnp bipolar transistor.  A {\et model}
property should be added to specify a bipolar transistor model.  A
{\et param} property can be added to specify additional parameters.

The bipolar transistor contains no hidden targets.

% -----------------------------------------------------------------------------
% dev:njf 062908
\subsubsection{N-Channel Junction FET}
\index{njf device}

The {\et njf} device is an n-channel junction field-effect transistor. 
A {\et model} property should be added to specify a JFET model.  A
{\et param} property can be added to specify additional parameters.

The JFET contains no hidden targets.

% -----------------------------------------------------------------------------
% dev:pjf 062908
\subsubsection{P-Channel Junction FET}
\index{pjf device}

The {\et pjf} device is a p-channel junction field-effect transistor. 
A {\et model} property should be added to specify a JFET model.  A
{\et param} property can be added to specify additional parameters.

The JFET contains no hidden targets.

% -----------------------------------------------------------------------------
% dev:nmos1 062908
\subsubsection{N-Channel MOSFET, 4 Nodes}
\index{nmos1 device}

The {\et nmos1} device is a 4-terminal n-channel MOSFET (drain, gate,
source, bulk).  A {\et model} property should be added to specify a
MOS model, suitable for 4-terminal devices.  Some of the MOS models
provided in {\WRspice}, for SOI devices, use more than four terminals
and will not work with this device.  It is left as an exercise for the
user to create a modified device suitable for use with these models. 
A {\et param} property can be added to specify additional parameters.

This device contains no hidden targets.

% -----------------------------------------------------------------------------
% dev:pmos1 062908
\subsubsection{P-Channel MOSFET, 4 Nodes}
\index{pmos1 device}

The {\et pmos1} device is a 4-terminal p-channel MOSFET (drain, gate,
source, bulk).  A {\et model} property should be added to specify a
MOS model, suitable for 4-terminal devices.  Some of the MOS models
provided in {\WRspice}, for SOI devices, use more than four terminals
and will not work with this device.  It is left as an exercise for the
user to create a modified device suitable for use with these models. 
A {\et param} property can be added to specify additional parameters.

This device contains no hidden targets.

% -----------------------------------------------------------------------------
% dev:nmos 042611
\subsubsection{N-Channel MOSFET, 3 Nodes}
\index{nmos device}

The {\et nmos} device is an n-channel MOSFET variation that contains
three visible nodes (drain, gate, source).  The bulk node is connected
to an internal global node named ``NSUB''.  To use this device, the
circuit should contain a voltage source tied to a terminal device with
label ``NSUB'' to provide substrate bias to all devices of this type. 
This simplifies the schematic by hiding the substrate connection to
each transistor.

A {\et model} property should be added to specify a MOS model,
suitable for 4-terminal devices.  Some of the MOS models provided in
{\WRspice}, for SOI devices, use more than four terminals and will not
work with this device.  It is left as an exercise for the user to
create a modified device suitable for use with these models.  A {\et
param} property can be added to specify additional parameters.

This device contains no hidden targets.

% -----------------------------------------------------------------------------
% dev:pmos 042611
\subsubsection{P-Channel MOSFET, 3 Nodes}
\index{pmos device}

The {\et pmos} device is a p-channel MOSFET variation that contains
three visible nodes (drain, gate, source).  The bulk node is connected
to an internal global node named ``PSUB''.  To use this device, the
circuit should contain a voltage source tied to a terminal device with
label ``PSUB'' to provide substrate bias to all devices of this type. 
This simplifies the schematic by hiding the substrate connection to
each transistor.

A {\et model} property should be added to specify a MOS model,
suitable for 4-terminal devices.  Some of the MOS models provided in
{\WRspice}, for SOI devices, use more than four terminals and will not
work with this device.  It is left as an exercise for the user to
create a modified device suitable for use with these models.  A {\et
param} property can be added to specify additional parameters.

This device contains no hidden targets.

% -----------------------------------------------------------------------------
% dev:nmes 062908
\subsubsection{N-Channel MESFET}
\index{nmes device}

The {\et nmes} device is an n-channel MESFET.  A {\et model} property
should be added to specify a MESFET model.  A {\et param} property can
be added to specify additional parameters.

The MESFET contains no hidden targets.

% -----------------------------------------------------------------------------
% dev:pmes 062908
\subsubsection{P-Channel MESFET}
\index{pmes device}

The {\et pmes} device is a p-channel MESFET.  A {\et model} property
should be added to specify a MESFET model.  A {\et param} property can
be added to specify additional parameters.

The MESFET contains no hidden targets.

% -----------------------------------------------------------------------------
% dev:tra 062908
\subsubsection{Transmission Line}
\index{tra device}

The {\et tra} device is a general transmission line.  In {\WRspice},
this can be lossy or lossless, and may access a model.  In other
versions of SPICE, this is a lossless line with no model.  A {\et
model} property can be added to specify a transmission line model.  A
{\et param} property can be added to specify additional parameters.

The transmission line contains no hidden targets.

% -----------------------------------------------------------------------------
% dev:ltra 062908
\subsubsection{Transmission Line (LTRA compatibility)}
\index{ltra device}

The {\et ltra} device is a general transmission line.  In {\WRspice},
this can be lossy or lossless, and is basically the same as the {\et
tra} device, but defaults to a convolution approach if lossy.  In
other versions of SPICE, this is a lossy line that requires a model. 
A {\et model} property can be added to specify a transmission line
model.  A {\et param} property can be added to specify additional
parameters.

The transmission line contains no hidden targets.

% -----------------------------------------------------------------------------
% dev:urc 062908
\subsubsection{Uniform RC Line}
\index{urc device}

The {\et urc} device is a lumped-approximation RC line.  A {\et model}
property should be added to specify a urc model.  A {\et param}
property can be added to specify additional parameters.

The {\et urc} line contains no hidden targets.

% -----------------------------------------------------------------------------
% dev:vccs 062908
\subsubsection{Voltage-Controlled Current Source}
\index{vccs device}

The {\et vccs} device is a voltage-controlled dependent current
source.  A {\et value} and/or {\et param} property can be added to
specify the gain, or other parameters required by the dependent
source.  Since all four nodes are specified, the two-node variants
supported by {\WRspice} are not supported by this device.

The VCCS contains no hidden targets.

% -----------------------------------------------------------------------------
% dev:vcvs 062908
\subsubsection{Voltage-Controlled Voltage Source}
\index{vcvs device}

The {\et vcvs} device is a voltage-controlled dependent voltage
source.  A {\et value} and/or {\et param} property can be added to
specify the gain, or other parameters required by the dependent
source.  Since all four nodes are specified, the two-node variants
supported by {\WRspice} are not supported by this device.

The VCVS contains no hidden targets.

% -----------------------------------------------------------------------------
% dev:cccs 030415
\subsubsection{Current-Controlled Current Source}
\index{cccs device}

The {\et cccs} device is a current-controlled dependent current
source.  A {\et devref} property can be used to specify the name of
the controlling voltage source or inductor in the common case.  A {\et
value} and/or {\et param} property should be added to specify gain, or
other parameters required by the dependent source.  This device
supports all of the variants supported in {\WRspice}.

The CCCS contains no hidden targets.

% -----------------------------------------------------------------------------
% dev:ccvs 030415
\subsubsection{Current-Controlled Voltage Source}
\index{ccvs device}

The {\et ccvs} is a current-controlled dependent voltage source.  A
{\et devref} property can be used to specify the name of the
controlling voltage source or inductor in the common case.  A {\et
value} and/or {\et param} property should be added to specify the
gain, or other parameters required by the dependent source.  This
device supports all of the variants supported in {\WRspice}.

The CCVS contains no hidden targets.

% -----------------------------------------------------------------------------
% dev:sw 062908
\subsubsection{Voltage-Controlled Switch}
\index{sw device}

The {\et sw} device is a voltage-controlled switch.  A {\et model}
property should be added to specify a switch model.  A {\et param}
property can be added to specify additional parameters.

This device contains no hidden targets.

% -----------------------------------------------------------------------------
% dev:csw 030415
\subsubsection{Current-Controlled Switch}
\index{csw device}

The {\et csw} device is a current-controlled switch.  A {\et devref}
property must be used to specify the name of the controlling voltage
source or inductor.  A {\et model} property should be added to specify
the switch model.  A {\et param} property can be added to specify
additional parameters.

This device contains no hidden targets.

% -----------------------------------------------------------------------------
% dev:opamp 062908
\subsubsection{Example Opamp Macro}
\index{opamp device}

The {\et opamp} device is an example ``black box'' device that expands
into a subcircuit.  It has a predefined {\et model} parameter which
gives the subcircuit name (which is resolved in the model library). 
No properties are required.

This device contains no hidden targets.


% -----------------------------------------------------------------------------
% xic:donut 100916
\section{The {\cb donut} Button: Create Donut Object}
\index{donut button}
\index{object creation!donut}
\epsfbox{images/donut.eps}

The {\cb donut} button appears only in physical mode.  It is used to
create a ring-like polygon.  The number of segments used to
approximate a circle can be altered with the {\cb sides} command.

If the user presses and holds the {\kb Shift} key after the center
location is defined, and before the perimeter is defined by either
lifting button 1 or pressing a second time, the current radius is held
for x or y.  The location of the {\kb Shift} press defines whether x
is held (pointer closer to the center y) or y is held (pointer closer
to the center x).  This allows elliptical donuts to be generated. 
This similarly applies when defining the outer radii, so that the
inner and outer surfaces can have different elliptical aspect ratios,
though the outer radius must be larger than the inner radius at all
angles. 

The {\kb Ctrl} key also provides useful constraints.  Pressing and
holding the {\kb Ctrl} key when defining the radii produces a radius
defined by the pointer position projected on to the x or y axis
(whichever is closer) defined from the center.  Otherwise, off-axis
snap points are allowed, which may lead to an unexpected radius on a
fine grid.

When the command is expecting a mouse button press to define a radius,
the value as defined by the mouse pointer (in microns) is printed in
the lower left corner of the drawing window, or the X and Y values are
printed if different.  Pressing {\kb Enter} will cause prompting for
the value(s), in microns.  If one number is given, a circular radius
is accepted, however one can enter two numbers separated by space to
set the X and Y radii separately.

While the command is active in physical mode, the cursor will snap to
horizontal or vertical edges of existing objects in the layout if the
edge is on-grid, when within two pixels.  When snapped, a small dotted
highlight box is displayed.  This makes it much easier to create
abutting objects when the grid snap spacing is very fine compared with
the display scaling.  This feature can be controlled from the {\cb
Edge Snapping} group in the {\cb Snapping} page of the {\cb Grid
Setup} panel.

\index{SpotSize variable}
If the {\et SpotSize} variable is set to a positive value, or the {\vt
MfgGrid} has been given a positive value in the technology file, tiny
round and donut figures are constructed somewhat differently.  the
figure is constructed somewhat differently.  Objects created with the
{\cb round} and {\cb donut} buttons will be constructed so that all
vertices are placed at the center of a spot, and a minimum number of
vertices will be used.  The {\cb sides} number is ignored.  This
applies only to figures with minimum radius 50 spots or smaller; the
regular algorithm is used otherwise.  An object with this
preconditioning applied should translate exactly to the e-beam grid. 
See \ref{spotsize} for more information.

% -----------------------------------------------------------------------------
% xic:erase 012715
\section{The {\cb erase} Button: Erase or Yank Geometry}
\index{erase button}
\index{object erasing}
\epsfbox{images/erase.eps}

Rectangular regions of polygons, boxes, and wires can be erased or
``yanked'' with the {\cb erase} button.  The user clicks twice or
presses and drags to define the diagonal of the region to be erased. 
Selected objects are not erased.  Wires maintain a constant width, and
are cut at the points where the midpoint crosses the boundary of the
erased area.

In physical mode, if the {\kb Shift} key is held during the operation
termination (click or button release), there is no erasure, however
the pieces that would have been erased are ``yanked'', i.e., added to
the yank buffer.  The pieces are also added to the yank buffer when
actually erased.  The yank buffer chain has a depth of five, meaning
that the contents of the last five yanks/erasures are available for
placement with the {\cb put} command.

Geometry in ``foreign'' windows can be yanked.  These are
physical-mode sub-windows showing a different cell than the current
cell being edited (as showing in the main window).  The foreign window
is never erased (i.e., holding {\kb Shift} is not necessary), but the
structure that would be erased is added to the yank buffer.  Thus, one
can quickly copy a rectangular area of geometry from another cell into
the current cell, by yanking with {\cb erase} and placing with the
{\cb put} command (below {\cb erase} in the side menu).

The {\cb SpaceBar} toggles ``clip mode''.  When clip mode is active,
for objects that overlap the rectangle defined with the mouse, instead
of erasing the interior of the rectangle as in the normal case, the
material outside of the rectangle will be erased instead.  The
overlapping objects will be clipped to the rectangle.  This applies
whether erasing or yanking, again the yank buffer will acquire the
pieces that would (or actually do) disappear in an erase operation.

When the {\kb Ctrl} key is held before the box is defined, clicking on
a subcell will cause the subcell's bounding box to be used as the
rectangle.  Thus, objects can be easily clipped to or around the
subcell boundary.  This applies when yanking as well.  The standard
erase is the inverse of the subcell paint operation in the {\cb box}
command.

While the command is active in physical mode, the cursor will snap to
horizontal or vertical edges of existing objects in the layout if the
edge is on-grid, when within two pixels.  When snapped, a small dotted
highlight box is displayed.  This makes it much easier to create
abutting objects when the grid snap spacing is very fine compared with
the display scaling.  This feature can be controlled from the {\cb
Edge Snapping} group in the {\cb Snapping} page of the {\cb Grid
Setup} panel.

The {\cb box}, {\cb erase}, and {\cb xor} commands participate in a
protocol that is handy on occasion.

Suppose that you want to erase an area, and you have zoomed in and
clicked to define the anchor, then zoomed out or panned and clicked to
finish the operation.  Oops, the {\cb box} command was active, not
{\cb erase}.  One can press {\kb Tab} to undo the unwanted new box,
then press the {\cb erase} button, and the {\cb erase} command will
have the same anchor point and will be showing the ghost box, so
clicking once will finish the erase operation.

The anchor point is remembered, when switching directly between these
three commands, and the command being exited is in the state where the
anchor point is defined, and the ghost box is being displayed.  One
needs to press the command button in the side menu to switch commands. 
If {\kb Esc} is pressed, or a non-participating command is entered,
the anchor point will be lost.


% -----------------------------------------------------------------------------
% xic:iplot 101810
\section{The {\cb iplot} Button: Interactive Analysis Plotting}
\index{iplot button}
\index{interactive plotting}
\epsfbox{images/iplot.eps}

The {\cb iplot} command, available in electrical mode, is useful only
if the {\WRspice} program is available.  Operation is similar to the
{\cb plot} button, whereby a command string is generated through
selection of nodes and branches with the pointer.  The command line
can be edited in the usual way to generate, for example, functions of
the plot variables.  Pressing the {\kb Enter} key saves the command. 
When the {\cb iplot} button is active and a command has been saved,
the plot is generated dynamically while a simulation, initiated with
the {\cb run} command, is in progress.

The {\cb S} and {\cb R} buttons, to the left of the prompt area, can be
used to save and restore prompt line text in a set of internal registers.

Pressing the {\cb iplot} button a second time will turn off the
interactive plotting.  Pressing {\cb iplot} and then {\kb Enter} will
turn the interactive plotting back on.  Of course, the trace points
and plotting command can be modified before pressing {\kb Enter}.  In
particular, if all prompt line text is deleted, pressing {\kb Enter}
will delete the internally saved command string, and turn interactive
plotting off.  Pressing the {\cb iplot} button again will take as
default text the string from the {\cb plot} command, if any.

The command text and mark locations are saved with the cell data
when written to disk, thus the {\cb iplot} command is persistent.


% -----------------------------------------------------------------------------
% xic:label 101016
\section{The {\cb label} Button: Create/Edit Labels}
\label{labelbut}
\index{label button}
\index{object creation!labels}
\index{keyboard!arrow keys}
\epsfbox{images/label.eps}

The {\cb label} button is used to create or modify a text label. 
Labels are abstract annotation objects which do not appear in physical
output.  For physical text, use the {\cb logo} command button.

\index{label editing}
If a label is selected before pressing the {\cb label} button, then
the selected label can be edited.  Multiple labels can be selected,
and each will receive the new label text.  If more than one label is
being changed, the command exits after the new text is entered on the
prompt line, i.e., after {\kb Enter} is pressed to terminate text
entry.

If only one label is being changed, on pressing {\cb Enter} the new
text is ``attached'' to the mouse pointer, as for a new label.  In
this state, the text size, orientation, and justification can be
changed as will be described below.  The user can either click in a
drawing window to place the label at the click location (effectively
moving the selected label), or press {\kb Enter} to update the
selected label at the existing location.

This is the recommended way to change the size of a label:  select it,
press the {\cb label} button, press {\kb Enter} to keep the same text,
adjust the size with the arrow keys, then press {\kb Enter} again to
update the label.  This keeps the label in a standard size and aspect
ratio which will match other labels.  This would not be the case if
the {\cb Stretch} command or operation was used instead.

If no label was initially selected, after the label text has been
entered, the label will appear ghost-drawn, attached to the mouse
pointer.  The text will be rotated or mirrored according to the
current transform, as set from the pop-up provided by the {\cb xform}
button in the side menu.  Instances of the label are placed where the
user clicks in a drawing window.

Label text in entered in the prompt line.  While editing, if the user
clicks on an existing label in a drawing widow which is contained in
the current cell, the text of that label will be inserted at the
prompt line cursor.  Hypertext entries (see ref{hypertext}) in the
label will be preserved.  If the existing label is a ``long text''
label (described below), the long text attribute will be lost, unless
the prompt line is empty before clicking on the label.  Particularly
in electrical mode, clicking on other objects in a drawing window will
insert text at the cursor position, as will be described.  Pressing
{\kb Enter} terminates the label text and will allow placement of
copies of the new label.

The size and justification of the label can be adjusted with the arrow
keys, before it is placed.  The arrow keys have the following effect: 

\begin{tabular}{ll}
\kb Up & enlarge by 2\\
\kb Right & enlarge by 10\%\\
\kb Down & reduce by 2\\
\kb Left & reduce by 10\%\\
\end{tabular}

The initial size of a label is determined by the present default label
height, and the magnification of the current drawing window.  The
default label height is 1.0 microns, which can be reset by setting the
{\et LabelDefHeight} variable to a different value.  The default
height is the smallest size available through scaling with the arrow
keys.  Generally, {\Xic} functions that create new labels will use the
default label height.  The default height of one micron is too large
for modern semiconductor processes, so one should redefine {\et
LabelDefHeight} in the technology file to a more suitable value,
typically the minimum feature size.

By default, the label is anchored at the lower left corner, though
this justification can be changed by holding the {\kb Shift} key while
pressing the arrow keys.  The {\kb Left} and {\kb Right} arrows cycle
through left, center, and right justification.  The {\kb Up} and {\kb
Down} arrow keys cycle through bottom, center, and top justification. 
Finally, holding the {\kb Ctrl} key while pressing the arrow keys will
change the current rotation angle.  The arrow keys implicitly cycle
through the angle choices, with {\kb Up} and {\kb Right} cycling in
the opposite sense from {\kb Down} and {\kb Left}.

Labels are scalable, and can be stretched with the {\cb Stretch}
button in the {\cb Edit Menu} or with button 1 operations.

Newlines can be embedded in the label text by pressing {\kb
Shift-Enter}.  The displayed label will contain line breaks at those
points.  The justification applies to the block, and line-by-line
within the block.

Labels are shown in legible orientation (i.e., left to right or down
to up) by default, independent of the actual transformation.  If the
{\cb Label True Orient} button in the {\cb Main Window} sub-menu of
the {\cb Attributes Menu} or the sub-window {\cb Attributes} menu
is set accordingly, labels will be shown in their actual orientation.

Pressing the {\kb Delete} key after the label text has been entered
will repeat prompting for new label text.  Labels have fixed size as
compared with layout geometry.

\subsection{Device Property Labels}

Labels are created internally for device properties in electrical
mode.  These labels can be moved, deleted, and edited just as
user-supplied labels.  Once deleted, though, such labels can not be
recreated except by recreating the device, or by using the {\cb
!regen} command.  The underlying property is not deleted, it simply
is not displayed in a label.

These labels can be ``hidden'' by clicking on the label text with
button 1 with the {\kb Shift} key held.  This replaces the label text
with a small box icon.  Shift-clicking the icon will redisplay the
text.  This can be useful when long labels obscure other features. 
See \ref{labelsize} for more information.

Labels can be edited by selecting the label before pressing the {\cb
label} button.  If the label was generated for a property in
electrical mode, the underlying property is also changed.  This is a
quick way to modify device properties, without invoking the {\cb
Properties} command button in the {\cb Edit Menu}.

\index{wire label}
\index{net name label}
\subsection{Wire Net Name Labels}

Similar to the property labels, electrical wires that participate in
schematic connectivity can have a bound label that provides a name for
the net containing the wire.

Unlike the device labels, wire net labels are created by the user.  If
the {\cb label} command is started with a single selected wire on an
electrically-active layer, the label created will be bound to that
wire.  Thus, to create a label for a wire, select the wire, press the
{\cb label} button in the side menu, and create the label.  These
labels can exist on any layer.

Once created, these labels can be edited in the same manner as
property labels, i.e., select the label and enter the label command by
pressing the side menu {\cb label} button.

\subsection{Ctrl-a and Ctrl-p}

In electrical mode, outside of any command, pressing {\kb Ctrl-a} will
cause the associated labels of any selected device or wire to also
become selected.  If labels are selected, then pressing {\kb Ctrl-a}
will cause their associated device or wire to also become selected. 
The associated labels can be deselected by pressing {\kb Ctrl-p}. 
This is useful for determining which labels are associated with a
given device or wire, and {\it vice-versa}.

\subsection{Spicetext Labels}
\label{spicetext}
\index{spicetext label}

In electrical mode, for efficiency reasons it is best not to use the
SCED layer for labels.  If the current layer is the SCED layer, a new
label will instead be created using the ETC1 layer.  If for some
reason a label is required on the SCED layer, the {\cb Change Layer}
command in the {\cb Modify Menu} can be used to move an existing label
to the SCED layer.

In electrical mode, labels can be used to enter arbitrary text into
the SPICE output.  There are two methods to achieve this.

If an electrical layer named ``SPTX'' exists, labels on this layer
will be included, verbatim, as separate lines in SPICE output, unless
the label is a ``spicetext'' label (below).  These labels are sorted
by position, top-to-bottom and left-to-right in output, and are placed
ahead of the spicetext labels.  A label on the SPTX layer in the
format of a spicetext label will be output as a spicetext label.

If the first word of the label is of the form
\begin{quote}\vt
spicetext{\it N}
\end{quote}
the label is a ``spicetext'' label, and the text which follows will be
entered verbatim as a separate line in the SPICE output.  The
spicetext labels can appear on any layer.  The integer {\it N\/},
which is optional, is a sorting parameter.  If there are multiple
labels containing SPICE text, they will be sorted by {\it N\/} before
being added to the SPICE output.  Smaller {\it N\/} will appear
earlier in the listing, with omitted {\it N\/} corresponding to a
value of zero.  The {\vt spicetext\/} lines are written as a
contiguous block in the listing.

Any text which can be interpreted by the SPICE simulator in use can be
added using these methods, but erroneous syntax will of course cause
errors as the SPICE text is sourced.

\subsection{``Long Text'' Capability}
\label{longtext}
\index{long text labels}

When editing or creating unbound labels, or labels for physical or
certain electrical properties ({\et value}, {\et param}, and {\et
other}), there is provision for entering a block of text that will not
be visible in the layout or schematic.  This avoids cluttering the
screen with labels containing large blocks of text.  Rather, a
symbolic form will be shown instead of the full text.

This same capability applies when adding or editing properties from
the {\cb Property Editor} provided by the {\cb Properties} button in
the {\cb Edit Menu}.

This capability is useful for properties which require a large block
of text, such as a long PWL statement in a {\et value} property for
SPICE.  It is not possible to edit a large text block in the prompt
area, and if displayed would cause the screen to be obscured or
cluttered.  The full text is added to SPICE output, however, and is
available as the property value in functions that query the value.

It is also useful for the {\vt spicetext} labels, so that a block of
text can be inserted into SPICE output, rather than one line. 
Remember that the text entered into the window must begin with ``{\vt
spicetext}'' and an optional integer, for the text to appear in SPICE
output.

When entering a label where this ``long text'' capability applies, a
small ``{\cb L}'' button will appear to the left of the prompt line,
and this will be active when the text cursor is in the leftmost
column.  Pressing this button will set the internal flag for ``long
text'', and open the text editor window for the text.  Any text that
was previously entered in the prompt line will be added to the text
editor window, or, if the label was already in long text mode, the
existing text will be shown in the editor.

If preexisting text was present on the prompt line when the {\cb L}
button was pressed, that text will be loaded into the text editor, but
any hypertext entries will be converted to plain text.  The long text
blocks do not support the hypertext feature.

Pressing {\kb Ctrl-t} has the same effect as pressing the {\cb L}
button when the button is visible and active.

From the editor window, one can edit the block of text, then press
{\cb Save} in the editor's {\cb File} menu to complete the operation,
or {\cb Quit} to abort.  The on-screen label will simply say ``{\vt
[text]}'' for a normal ``long text'' property or non-associated label,
or have the standard form for a script label (described below);

The long text labels can be edited with the label editor, as can
normal labels, by selecting the label before pressing the {\cb label}
button.  The prompt line will display ``{\vt [text]}'' as a hypertext
entry.  Pressing {\kb Enter} or the {\cb L} button will bring up the
text editor loaded with the text associated with the label, allowing
editing.

To convert a long text label to a normal label, instead of bringing up
the text editor, the hypertext ``{\vt [text]}'' entry can be deleted
in the prompt line.  Deleting the entry will place as much of the text
block as possible on the prompt line, and delete the text block and
the association of the label or property as a long text object.

\subsection{Script Labels}
\index{script labels}

{\Xic} provides the ability to embed a script or script reference in a
label, which is executed when the user clicks on the label.  These are
created like any other label, but have the form

\begin{quote}\vt
!!script [name={\it word\/}] [path={\it path\/}] [{\it script text}...]
\end{quote}

The leading token in the label must be ``{\vt !!script}'' to indicate
that the label text is executable.  This is followed by zero or more
keyword/value pairs as shown, followed by the script text that will be
executed.  The keywords and values must be separated by `=' with no
space.  The value is a single token, which should be double-quoted if
it contains white space.  These are optional.

The keywords have the following interpretations.
\begin{description}
\item{\vt name={\it some\_word}}\\
The script label is rendered on-screen as {\it some\_word} surrounded
by a box.  If no name is given, the word ``{\vt script}''" is shown.

\item{\vt path={\it some\_path}}\\
If this is given, then the script to be executed is given by {\it
some\_path} and any executable statements in the label are ignored. 
The {\it some\_path} can be an absolute path to a script file, or can
be the name of a script file expected to be found in the script search
path.
\end{description}

Any remaining text is executed as script commands, if {\vt path} is
not given.  For short scripts, semicolons can be used as command
terminators in a single line.  Otherwise, a text editor can be invoked
on the label string by pressing the ``L'' (long text) button when
creating the label.

Clicking on a script label will execute the script, and not select the
label as with normal labels.  To select a script label, hold {\kb
Shift} while clicking on the label, or drag over the label (area
select).  If a script label is selected, it will not execute when
clicked on, but rather be deselected.

For example, suppose that a user has a large layout, with a small
section that the user often needs to zoom into.  The user can create a
script label to perform the zoom operation.  After zooming in, one can
note the position and estimate the width of the drawing window.  Then,
one would create a label such as
\begin{quote} \vt
  !!script name=zoom Window({\it x\/}, {\it y\/}, {\it width\/}, GetWindow())
\end{quote}
and place it somewhere convenient.  The {\it x\/}, {\it y\/}, and {\it
width} above of course represent the actual values (in microns). 
Clicking on the label will always zoom to this area.

\subsection{Label Size Issues}
\label{labelsize}

In electrical mode, property text labels can be displayed or
``hidden''.  If a label is hidden, the text is not displayed, only a
small box at the text reference point is shown.

Labels with text size longer than a certain length will be shown as
hidden by default.  Hidden labels can be made visible, and {\it
vice-versa} by clicking on the label or small box with the {\kb Shift}
key held.  The label text can also be shrunk (with the {\cb Stretch}
command in the {\cb Modify Menu} or with button 1 operations) to make
it visible.

The label hidden status is persistent when the cell is saved in any
format, however changing the display status does {\bf not} change the
modified state of a cell, thus this can be done in IMMUTABLE cells.

It should be noted that the ``real'' bounding box of the label, which
is used to set the cell bounding box, is always the bounding box of
the actual text.  The hidden display mode is only available for the
labels that contain property strings in electrical mode.  Hidden
labels can be selected only over the small box, and only the small box
is highlighted.  However, when moving or stretching, the entire
``real'' bounding box is highlighted.

\index{LabelMaxLen variable}
The size threshold can be changed with the {\cb Maximum displayed
property label length} entry in the {\cb Window Attributes} panel from
the {\cb Set Attributes} button in the {\cb Attributes Menu}. 
Equivalently, the variable {\et LabelMaxLen} can be set to an
integer greater than 6 with the {\cb !set} command.  The units are the
width of a default-size character cell.  In releases prior to 2.5.66,
the default length was 32 default character size cells.  In this and
later releases, the value is 256 character cells.  The larger
threshold makes the nondisplay of label text much less probable, as
this feature has been confusing to users.

Another way to obscure a long label is to convert it to a ``long
text'' label.

To ``hide'' a label using the ``long text'' feature:
\begin{enumerate}
\item{Select the label.}
\item{Press the side menu {\cb label} button (with the black `T' icon).}
\item{Press the gray {\cb L} button that appears to the left of
 the prompt line.  This will cause the text editor to appear,
 loaded with the label text.  If there is no {\cb L} button, then
 the property can't use long text, which is true for properties
 that are ``always'' short, such as for device and model names.}
\item{In the text editor, press {\cb Save} in the {\cb File} menu.
 The editor will disappear, and the label displayed on-screen will
 have changed to ``{\vt [text]}''.}
\end{enumerate}

To convert back to a normal label:
\begin{enumerate}
\item{Select the long-text label (``{\vt [text]}'').}
\item{Press the side menu {\cb label} button (with the black 'T' icon).}
\item{With the cursor under ``{\vt [text]}'' on the prompt line, press
 the {\kb Delete} key.  The full label text will appear on the prompt
 line.}
\item{Press {\kb Enter}.  The label will be shown normally.}
\end{enumerate}

Long property text labels can also be broken into multiple lines by
adding embedded returns.  These are added with {\kb Shift-Enter} while
the string is being edited.  Note that this generates newlines in the
SPICE output, so that in most cases the extra lines should begin with
the ``+'' continuation character.


% -----------------------------------------------------------------------------
% xic:logo 052311
\section{The {\cb logo} Button: Create Physical Text}
\index{logo button}
\index{physical text}
\index{MinWidth and logo text}
\epsfbox{images/logo.eps}
The {\cb logo} command allows the creation of physical text and images
for labeling, identification, etc.  Operation is similar to the {\cb
label} command, where the arrow keys alter text or image size, {\kb
Shift}-arrow cycles through the justification choices, and {\kb
Ctrl}-arrow cycles through the rotation angles.  By default, the text
is implemented with rounded-end wires in the current layer, using a
vector font.

For rendering text, there are three font possibilities.  The default
font is a vector font which constructs the characters using wire
objects.  The Manhattan font is a built-in bitmap font from which the
characters are constructed using Manhattan polygons.  The Manhattan
font is fixed-pitch with an 8X16 map.  The ``pretty'' font is one of
the system fonts, which similarly creates characters constructed as
Manhattan polygons.  Logic is applied to extract the ``best''
rendition from anti-aliased fonts, which do not have a precisely
defined shape.  Some fonts may look better than others in this
application.

While in the {\cb logo} command and using the vector font, pressing
the {\kb Ctrl-Shift}-arrow key combinations will adjust the path
width; the {\kb Up} and {\kb Right} arrow keys increase the width,
{\kb Down} and {\kb Left} arrows decrease the path width.

The {\et LogoPathWidth} variable tracks the current path width
setting.  The {\et LogoEndStyle} variable tracks the current end style
setting.

Instead of a text label, the {\cb logo} command can be used to place
an image.  The image must be provided by a file in the XPM format. 
This is a simple ASCII bitmap format, commonly used in conjunction
with the X-windows system on Unix machines.  Other types of bitmap
files can be converted to XPM format with widely available free
software, such as the ImageMagick package.  Several XPM files are
supplied in the help directory for {\Xic} (located by default in {\vt
/usr/local/xictools/xic/help}), which illustrate the format.

This feature is enabled in the {\cb logo} command by giving the path
of an XPM file, which must have a ``{\vt .xpm}'' suffix, as the text
string.  This will cause the image to be imported such that it can be
scaled, transformed, and placed, just like a normal label.  The
background color (the first color listed in the XPM file) is taken as
transparent.  All other layers found in the XPM file are mapped to the
current layer.  The image is rendered as a collection of Manhattan
polygons.

Unlike in releases 3.0.11 and earlier, there is no attempt to limit
feature sizes according to design rules.  The minimum size of a
character is set by the internal resolution, while the maximum size is
about .4 X .7 cm.  Once the text is entered, the size and other
attributes can be changed with the arrow keys, and the text is placed
where the user clicks in the drawing with button 1.  The text can be
reentered, i.e., a new label or image file defined, if the {\kb
Delete} key is pressed.

Alternatively, a fixed "pixel" size can be specified.  In this case,
the arrow keys will pan the display window, and have no effect on the
label or image size.

The default operation is to apply the text or image feature directly
in the current cell, where the user clicks.  It is also possible to
create a subcell containing the text, which is instantiated at the
clicked-on locations.  This may be more efficient if there are many
copies of the same label.

\index{NoDRC flag}
Note that use of the vector font may produce design rule violations,
which are pretty much inevitable due to the presence of acute angles
in some characters.  Use of the other fonts, which are rendered using
Manhattan polygons, can avoid design rule violations, if the ``pixel''
size is larger than the MinWidth and MinSpace design rules for the
layer.  When physical text (or an image) is placed with the {\cb logo}
command, interactive design rule checking is suppressed.  The {\et
NoDRC} flag can be set on the new label, or the NDRC layer can be
used, to permanently suppress DRC.

It is possible to change the font used for the {\cb logo} command. 
The default font is set internally by {\Xic}, however individual
characters or the whole font will be updated upon startup if a file
named ``{\vt xic\_logofont}'' is found along the library search path,
which contains alternative character specifications.

\subsection{The Logo Font Setup Panel}
\index{Logo FontSetup panel}

While the {\cb logo} command is active, the {\cb Logo Font Setup}
panel is visible, though this can be dismissed without leaving the
{\cb logo} command.  The top of the panel provides three ``radio''
buttons for selecting the font:  {\cb Vector}, {\cb Manhattan}, and
{\cb Pretty}.  The {\et LogoAltFont} variable tracks the choice in
these buttons.

Below the {\cb Font} choice buttons is the {\cb Define ``pixel'' size}
check box and numeric entry window.  When checked, the numeric entry
area is enabled, and the value represents the size of a ``pixel'' used
for rendering the label or image, in microns.  When checked, the arrow
keys have no effect on label or image size, instead they revert to
their normal function of panning the display window.  This feature is
tied to the {\et LogoPixelSize} variable, which when set to a real
number larger than 0 and less than or equal to 100.0 will define the
``pixel'' size used in the {\cb logo} command.

There are two option menus in the {\cb Logo Font Setup} panel which
set the end style and path width assumed in the wires used for
constructing characters with the vector font.  The user can set these
according to personal preference.  Although rounded end paths may look
better, they are somewhat less efficient in terms of storage and
processing, and are not handled uniformly (or at all) in some CAD
environments.  For example, rounded-end wires may be converted to
square ends when written as OASIS data.

The {\cb Select Pretty Font} button brings up the {\cb Font Selection}
panel, allowing the user to select a system font for use as the
``pretty'' font.  In the {\cb Font Selection} panel, the user can
select a font, then press the {\cb Set Pretty Font} button to actually
export the choice.  This will set the {\et LogoPrettyFont} variable.

The {\cb Create cell for text} check box, when checked, sets a mode
where new labels and images are instantiated as subcells rather than
directly as geometrical objects.  In addition to generating a master
cell in memory, a native cell file with the same name is written in
the current directory.  The boolean variable {\et LogoToFile} tracks
this state of this check box.

The name of the file used for the label is internally generated, and
is guaranteed to be unique in the current search path.  The name
consists of the first 8 characters of the label, followed by an
encoding of the various parameters related to the label.  For a given
label, the uniqueness of the file name prevents recreating the same
label file in a subsequent session.

The {\cb Dump Vector Font} button will create a file containing the
vector font (see \ref{vectorfont}) currently being used by the {\cb
logo} command.  By default, the vector font uses the same character
maps as the vector font used to render label text on-screen.  However,
these maps can be overridden by definitions from a file.  The {\cb
Dump Vector Font} button can be used to dump the current set of
character maps to a file.  Character maps from this file can be
modified and placed in a file named ``{\vt xic\_logofont}'' in the
library search path, in which case they will override the internal
definitions when producing vector-based characters in the {\cb logo}
command.


% -----------------------------------------------------------------------------
% xic:nodmp 061913
\section{The {\cb nodmp} Button: Node (Net) Name Assignments}
\label{nodmp}
\index{node mapping}
\index{nodmp button}
\epsfbox{images/nodmp.eps}

The {\cb nodmp} button, which is available in the electrical mode side
menu, will bring up the {\cb Node (Net) Name Mapping} panel which is
used to display and alter the names used for ``nodes''
(single-conductor wire nets) in the schematic, and in SPICE and other
output.  This name may be internally generated, or may be derived from
a terminal name, or may be assigned by the user.  This panel is also
brought up by the {\cb Find Terminal} button in the {\cb Setup} page
of the {\cb Extraction Setup} panel, which is obtained from the {\cb
Setup} button in the {\cb Extract Menu}.

First, to facilitate the discussion that follows, some terminology
will be introduced.  See also the section on wire net naming in
\ref{netex}.

\begin{description}
\item{scalar}\\
Single-conductor wire nets, or ``nodes'' (from SPICE terminology) are
referred to as ``scalar'' nets.  These are the actual circuit
connections, which are compared in layout vs.  schematic (LVS)
testing.  {\Xic} also allows multi-conductor (including
single-conductor) ``vector'' and ``bundle'' nets.  These actually
reference and organize the nodes, but do not provide actually
connectivity, except through name matching.  The present {\cb Node
(Net) Name Mapping} panel applies only to the scalar nets.

\item{associated name}\\
A scalar wire net, or ``node'' may have ``associated names''.  These
names derive from named terminal devices which may be connected to the
net, or from labeled wires which are connected to the net.  Both the
terminal device and the labeled wire derive the net name from the text
of an associated label.  The labels can be edited, which will change
the text of the associated name.  A net may have any number of
associated names.

\item{cell terminal name}\\
Every electrical contact point of a cell has a name.  This name was
assigned when the cell terminal was created with the {\cb subct}
command button in the side menu, or if no name was given a default
name is used.

It is also possible to name cell contact terminals from the {\cb Edit
Terminals} command button in the {\cb Setup} page of the {\cb
Extraction Setup} panel.  This panel is brought up with the {\cb
Setup} button in the {\cb Extract Menu}.

\item{global names}\\
Certain names are registered within {\Xic} as ``global names'',
and are kept in an internal string table.  These names are known
at every level of the cell hierarchy.  There is always at least
one global name defined, the ground node with name ``{\vt 0}''.

Global names are easily created by the user, as any node name ending
with an exclamation point (`{\vt !}') is taken as a global name.
For example, ``{\vt vdd!}'' would be taken as a global name.

Global node names are also set with the {\et DefaultNode} global
properties, in the device library file.  They may be used as default
nodes in some devices.  In particular, the ``three terminal'' {\cb
nmos} and {\cb pmos} devices included in the default library make use
of this feature, defining global node names ``{\vt NSUB}'' and ``{\vt
PSUB}'' that connect to the device substrate.

\item{assigned name}\\
Names that are specified from the {\cb Node (Net) Name Mapping} panel
using the {\cb Map Name} button will be referred to as ``assigned
names''.
\end{description}

A wire net can clearly have a number of names associated with it.  The
actual name for the node will be chosen according to the priorities
listed below.

\begin{enumerate}
\item{If a net has an associated name that matches a global name, that
global name is used, and this can not be overridden by the user.

If two or more global names match associated names in the net, the
name chosen will be the one earliest in ASCII lexicographic order. 
This situation is unlikely and probably represents a topology error.}

\item{If a net is given an assigned name, that name will be used.}

\item{If a net contains a cell terminal, the cell terminal name will
be used.  It is possible that more than one cell terminal is connected
to the node, in which case the name chosen will be the one earliest in
ASCII lexicographic order.}

\item{If the net has an associated name, that name will be used.  It
is possible that more than one associated name will be found, in which
case the name chosen will be the one earliest in ASCII lexicographic
order.}

\item{The net will be given a name based on the internally-generated
node number.}
\end{enumerate}

For names other than the internally generated node numbers, the name
is predictable.  The internally generated numbers will change if the
circuit is modified, or possibly for other reasons.  Thus, if netlist
or SPICE output is to be used in another application, it may be
important to assign names to nodes to be referenced by name.

The {\cb Node (Net) Name Mapping} panel contains two text listing
windows.  The left (node listing) window lists all of the nets in the
current cell schematic.  An entry in the list can be selected by
clicking on the text with the mouse.  When a net is selected in this
list, the terminals to which the net connects are listed by name in
the right (terminal listing) window.  Entries in the terminal listing
can be selected as well by clicking on the text with the mouse.  In
either window, the selected entry, if any, is highlighted. 

There is a ``grip'' in the region separating the two text listings,
which can be dragged horizontally to change the relative widths of the
windows.

The left column in the node listing contains the internal node
numbers, which can change arbitrarily if the circuit is modified. 
Entries in the second column are the mapped names, i.e., the names
used in SPICE and netlist files.  If the second column entry is blank,
no name could be found for the net, and {\Xic} will create a name from
the node number for use in output.  The third column will contain the
letter ``{\vt Y}'' if the node has a name assigned by the user, and/or
a ``{\vt G}'' if the node name is that of a global node (including
ground).  Both letters will appear if the user assigns a name that
matches a global name, which includes any name that ends with an
exclamation point.  The ``{\vt G}'' nodes without {\vt Y} can not be
renamed by the user.

When a node is selected in the left text window, the right text
window lists terminals and other features that are found in the
selected net.  This includes
\begin{itemize}
\item{Device and subcircuit instance terminals.}
\item{Named terminal devices.  These start with a `{\vt T}' character,
followed by a space, followed by the name from the terminal
label.}
\item{Named wires.  These start with `{\vt W}' followed by space and
the name from the wire label.}
\item{Cell contact terminal names.}
\end{itemize}

The names used for device terminals are a concatenation of the device
name and the terminal names as supplied in the node properties in the
device library file, if a name was given.  If no name was given, a
default name is constructed as {\it devicename\/}\_{\it contactnum\/}. 
That is, the device name, followed by an underscore, followed by an
internal index number for contacts of that device.  The device name
starts with a letter which is the SPICE key for that device type. 
Subcircuits are similar, and the terminal names begin with `{\vt X}'.

In the electrical schematic drawing, when a net is selected in the
node listing window, wire objects that are included in the selected
net are highlighted.  Each of the device and subcircuit instance
terminals listed in the terminal listing area will have a small
highlighting box drawn around its location.  If one of the terminals
in the terminal listing is selected, that terminal will be displayed
using highlighting.

The panel will cooperate closely with the physical extraction system
when the {\cb Use Extract} check box is checked.  This means that
extraction/association will be performed as needed so that terminal
locations are correctly defined in the physical layout as well.  In
this case, a terminal selected in the terminal list will be shown in
physical layout windows, as well as the schematic.  If the check box
is not checked, extraction data will be used if present when showing
the terminal in layouts, but there is no attempt to maintain currency. 
The {\cb Node (Net) Name Mapping} panel is also available from the
{\cb Find Terminal} button in the {\cb Extraction Setup} panel in both
physical and electrical modes, in addition to the side-menu button in
electrical mode.

When an entry in the terminal listing window is selected, the {\cb
Find} button, below the listing, is un-grayed.  Pressing the {\cb
Find} button will bring up a sub-window displaying the current cell,
with the selected terminal at the exact center of the display.  One
can press the numeric keypad {\kb +} key repeatedly to zoom in to the
terminal location, and the terminal will remain centered.  Further, if
{\cb Use Extract} is set or the extraction state is current, the
terminal will also be displayed and centered if the sub-window is
changed to physical mode.

When the {\cb Click-Select Mode} button is pressed, a command state is
entered whereby clicking on a wire or contact point in a drawing
window will select that net.  This works a bit differently depending
on the state of the {\cb Use Extract} check box.  If the box is
checked, the button will bring up the {\cb Path Selection Control}
panel from the extraction system.  This allows selection of conducting
paths in the layout windows by clicking on objects.  The corresponding
net will be selected in the node listing window, with corresponding
highlighting shown in schematic windows.  One can also click on wires
and terminal locations in the schematic, and the clicked-on net will
become selected.  The corresponding conductor group will be displayed
highlighted in layout windows.

With {\cb Use Extract} not checked, the {\cb Path Selection Control}
panel will not appear, but clicking in schematic windows will have
similar effect.  The system will once again use extraction data if
available to map button presses in layout windows to a conductor group
and back to the coresponding electrical net to be highlighted. 
However, there is no highlighting of the physical conductor group.
  
In either case, the clicked-on node will be shown selected in the node
listing window, and scrolled into view if necessary.  The terminal
listing window will show the selected net details as usual.  {\cb
Click-Select Mode} is exited if another command is started, or {\kb
Esc} is pressed, or the {\cb Click-Select Mode} button is pressed
again, or, with {\cb Use Extract} checked, the {\cb Path Selection
Control} panel is retired by any means.

The {\cb Deselect} button will deselect selections in the node listing
window, and the corresponding highlighting in the drawing windows. 
The terminal listing window becomes empty.

It is also possible to search for nets and terminals by name using the
controls just above the two listing windows.  The two ``radio''
buttons select whether to search for node or terminal names.  One
enters a ``regular expression'' into the text area.  The button to the
left of the text entry initiates the search.  A matching net is
selected as is the matching terminal if searching for terminals.  One
can press the button again to move to the next and subsequent matches. 
If there is no initial selection, perhaps because {\cb Deselect} was
pressed, the search area starts at the top and extends toward the end
of the listing.  If a net is selected, the search starts with the next
item (terminal or net) after the selection end extends toward the end.

The regular expression conforms to POSIX.1-2001 as an extended,
case-ignored regular expression.  On a Linux system, ``{\vt man
grep}'' provides a good overview of regular expression syntax and
capability.  However, one probably doesn't need to know much more than

\begin{enumerate}
\item{A given string will match any name containing the string, case 
insensitive.}
\item{The carat (`{\vt \symbol{94}}') character matches the beginning of a 
name.}
\item{The dollar sign (`{\vt \$}') character matches the end of a 
name.}
\end{enumerate}

If the third column in the node listing window is not `{\vt G}', then
an overriding name for the selected node can be assigned with the {\cb
Map Name} button, but only while in electrical mode.  To apply a name,
select a node in the node listing area, then press the {\cb Map Name}
button.  A new name will be prompted for in a pop-up window.  The name
can be any text token (white space is not allowed), however it is up
to the user to ensure that the name makes sense in the context of the
output.  For example, for SPICE output, the node names must adhere to
the rules for valid node names in SPICE.  After pressing {\cb Apply},
the second column in the listing will be updated to show the new name,
and the third column will show ``{\vt Y}''.  Again, this can only be
done while in electrical mode, in physical mode the button remains
grayed.

The node naming can actually modify circuit topology, which can be a
powerful feature or a curse.  If two nets share a name, they will be
merged, and the left window will reflect this.  Thus, it is easy to
make connections using node name mapping that are not obvious when
looking at the schematic.  For this reason, if the user is about to
apply a duplicate name, a confirmation pop-up will appear.  The user
is given the choice to back out of the operation, or continue.

The node name assignment works by association with a connection point
in the net, equivalent to a hypertext reference.  This association
persists if the object is moved, and is transferred to another device
or wire if the object is deleted, if possible.  In some cases it may
get lost, however, so an assigned name may have to be reentered after
the circuit is edited.

In electrical mode, an assigned name can be deleted by first selecting
the node in the node listing area, then pressing the {\cb Unmap}
button.  The {\cb Unmap} button is un-grayed only if the third column
of the selected node shows ``{\vt Y}'' indicating that it has an
assigned name.  On pressing the button, the name will revert to the
default name.  This may effectively change circuit topology by undoing
the net merging brought about through net name assignments.  Again,
this operation is available only in electrical mode.

The internal data structure representing node name mapping, and the
listings, will be in one of two states.  Either devices and
subcircuits with the {\et nophys} property will be included as normal
devices and subcircuits, or these will be ignored.  In the latter
case, if the {\et nophys} property has the ``shorted'' option, the
terminals will be effectively shorted together, which will obviously
change the node numbering.

The current state is as set by the last function to generate the
connectivity map.  Functions in the extraction system will always
recognize the {\cb nophys} properties, and build the map excluding
these devices but taking the ``shorted'' {\et nophys} devices as
shorted.  Then, the schematic will correspond to the actual physical
layout.  Functions in the side menu which generate a SPICE listing
will ignore {\et nophys} properties and include all such devices in
the net list.  This produces a schematic appropriate for SPICE
simulation.

The {\cb Use nophys} button is used to switch between these two
representations, and the state of the button will be reset if another
function changes the state.

When the {\cb Use nophys} button is pressed, devices and subcircuits
with the {\et nophys} property set will be {\it included} in the
listings, just as ``normal'' devices.  Their terminals will be listed
in the terminals listing window.  The {\et nophys} property is ignored
in this case, as will be true when a listing is being prepared for
SPICE output from functions in the side menu.

When the {\cb Use nophys} button is not pressed, devices and
subcircuits with the {\et nophys} property will be ignored in the
listings, and the node numbering will respect the ``shorted'' {\et
nophys} properties.  Terminals from these devices and subcircuits will
not be listed in the terminal listing window.  This mode is consistent
with the usage by the extraction system.


% -----------------------------------------------------------------------------
% xic:place 100416
\section{The {\cb Place} Button: Cell Placement Control Panel}
\index{Place button}
\index{Cell Placement Control panel}
\index{cell placement}
\index{place cells}
\epsfbox{images/place.eps}

The {\cb place} button in the side menu brings up the {\cb Cell
Placement Control} panel which allows instances of cells (subcells) to
be added to the current editing cell.

When the {\cb Place} button in the panel or the {\cb place} button in
the side menu is active (the two buttons show the same status), the
current master can be instantiated at locations where the user clicks
(``place mode'').  The bounding box of the cell is ghost-drawn and
attached to the pointer.  The orientation and size of the instance are
set by the current transform.  If the {\cb Cell Placement Control}
panel is dismissed the place mode, if active, is exited.  The place
mode can be exited with the {\kb Esc} key, or by pressing the {\cb
Place} button (either one) a second time.  The panel is not popped
down when place mode is exited.

The substructure of cell instances being placed is highlighted to the
depth shown in the main window.  This facilitates alignment with other
objects.  One can change the display depth to reveal more or less of
the substructure.

From the {\cb Open} command in the {\cb File Menu}, if one holds down
{\kb Shift} while selecting one of cells from the history list, the
{\cb Cell Placement Control} panel will appear with that cell added as
the current master.  This applies to cell names and not the ``{\cb
new}'' entry.  This is a quick backdoor for instantiating cells
recently edited.

In electrical mode, when a connection point of a device or subcell is
near another connection point, it will snap to that location and a
small dotted box will be drawn around the point.  This facilitates
placement of devices and subcircuits in the schematic.  While the {\kb
Shift} or {\kb Ctrl} keys are held, this feature is disabled.

\index{place panel!Use Array}
\index{cell arrays}
Cells can be placed individually, or as arrays in physical mode.  When
the {\cb Use Array} button is active, cells will be placed as arrays,
governed by the currently set array parameters.  The array parameters
can be entered into the four text fields below, only when the {\cb Use
Array} button is active.  Arrays are allowed in physical mode only. 
If this button is not active, single cells are placed.

The array replication factors Nx and Ny can be set to any value in the
range of 1 through 32767.  The upper limit is set by the GDSII file
format, and is enforced by {\Xic} to avoid portability problems.

The spacing values Dx and Dy are edge to adjacent edge spacing, i.e.,
when zero the elements will abut.  If Dx or Dy is given the negative
cell width or height, so that all elements appear at the same
location, the corresponding Nx or Ny is taken as 1.  Otherwise, there
is no restriction on Dx or Dy except that very large (unphysical)
values can cause integer overflow internally.

The {\cb !array} command can be used to convert existing instances
into arrays, and to modify the array parameters of existing arrays.

In physical mode, the reference point of the cell, which is the point
in the cell located at the pointer, can be set to either the cell's
origin, or to one of the cell's corners.  A drop-down menu in the
{\cb Cell Placement Control} panel indicates the present selection,
and allows the user to make a new choice.  The nomenclature ``Upper
Left'', etc., refers to the corner of the untransformed cell array
bounding box.  When place mode is active, pressing the {\kb Enter} key
repeatedly will cycle the reference point around the corners and back
to the origin.

In electrical mode, the cell reference point is always set to the
location of the reference terminal, which is usually the first
terminal defined.  If the cell has no terminals, the reference point
can be cycled around the corners, as in physical mode, however for
corners the reference point is snapped to the nearest grid location. 
This should prevent device terminals from being located off-grid.  An
electrical cell should always have terminals (assigned with the {\cb
subct} command in the electrical side menu) if it is to be part of the
circuit, and not some kind of decoration or annotation.

When the {\cb Smash} button is active, is active, instances will be
smashed into the parent where the user clicks, meaning that the cell
content will be merged into the parent cell, rather than creating a
new instance.  The flattening is one-level, so that any subcells of
the cell being placed become subcells in the parent.

\index{place panel!Replace}
\index{replace cells}
When the {\cb Replace} button is active, existing cells are replaced
with the new master when clicked on.  and no cells are placed if the
user clicks in the area outside of any subcells.  When a cell is
replaced, the placement of the new cell is determined in physical mode
by the setting of the reference selection
drop-down menu.  For example, if this setting is ``Upper Right'', the new cell
untransformed upper-right corner will be placed at the existing cell
untransformed upper right corner.

In electrical mode, the reference terminal (the first connection
point) is always placed at the same location as the reference terminal
of the replaced cell.  In either case, any currently active
transformations are performed in addition to the transformations of
the replaced cell on the new cell.

Cells can be placed or replaced only when place mode is active, i.e.,
when the {\cb Place} button in the {\cb Cell Placement Control} pop-up
or the {\cb place} button in the side menu is active.

\index{place panel!editing array parameters}
If the {\cb Use Array} button is active when cells are being replaced,
the replaced cell will take the array parameters from the {\cb Cell
Placement Control} panel.  Otherwise, the array parameters are
unchanged during replacement.  Note that it is possible to replace an
instance with another instance of the same cell, but with different
array parameters.  This is one way that array parameters can be
``edited''.

The {\cb Dismiss} button will retire the {\cb Cell Placement Control}
panel, and exit place mode.

The cell currently being placed, the ``master'', can be selected in
several ways.  A list of masters is kept, and can be viewed with the
menu button in the {\cb Cell Placement Control} panel.  Pressing and
holding button 1 with the pointer on the menu button issues a
drop-down menu, whose entries are highlighted as the pointer passes
over them.  A selection is made by releasing button 1 over one of the
selections.  Pressing the {\cb New} button in this menu brings up a
dialog box which allows the user to enter a new master name.

The pop-up list of cells will grow with each addition until a limit is
reached, at which point new entries will replace the oldest one.  The
{\cb New} item is always at the top of the list.  The list consists of
the most recent masters specified, either with the {\cb New} button,
or through the {\cb Place} button in the {\cb Cells Listing} or {\cb
Files Listing} panels.

\index{MasterMenuLength variable}
The maximum number of masters saved in the menu can be specified with
the {\cb Maximum menu length} entry area just below the menu.  The
default is 25, which may not be suitable for some screen resolutions
or window systems.  It is not very useful if the pull-down menu
extends off-screen.  This entry tracks the value of the {\et
MasterMenuLength} variable.  The variable can be set as an integer or
cleared to change the value, which is equivalent to changing the
integer entry in this panel.

When a new entry is selected, a dialog pop-up appears for the new cell
name.  If a selection can be found in the various panels that provide
file or cell selection, that selection is pre-loaded into the dialog
as a default.  Each of these sources is tested in order, and the first
one that is visible and has a selection will yield the default cell
name.

\begin{itemize}
\item{A selection in the {\cb File Selection} pop-up from the {\cb
 File Select} button in the {\cb File Menu}.}
\item{A selection in the {\cb Cells Listing} pop-up from the {\cb
 Cells List} button in the {\cb Cell Menu}.}
\item{A selection in the {\cb Files Listing} pop-up from the {\cb Files
 List} button in the {\cb File Menu}, or its {\cb Content List}.}
\item{A selection in the {\cb Content List} of the
 {\cb Libraries} pop-up from the {\cb Libraries List} button
 in the {\cb File Menu}.}
\item{A selection in the {\cb Cell Hierarchy} pop-up from the {\cb
 Show Tree} button in the {\cb Cell Menu} or from the {\cb Tree}
 button in the {\cb Cells Listing} pop-up.}
\item{A cell name that is selected in the {\cb Info} pop-up, from
 the {\cb Info} button in either the {\cb View Menu} or the {\cb
 Cells Listing} pop-up.}
\item{The name of a selected subcell in the drawing window, the most
 recently selected if there is more than one.}
\end{itemize}

The first time the {\cb Cell Placement Control} panel comes up, the
user is prompted for the name of a cell, just as if the {\cb New} menu
button was pressed.

The name provided can be a file containing data in one of the
supported archive formats, the name of an {\Xic} cell, or a library
file.  If the name of an archive file is given, the name of the cell
to open can follow the file name separated by space.  If no cell name
is given, the top level cell (the one not used as a subcell by any
other cells in the file) is the one opened for placement.  If there is
more than one top level cell, the user is presented with a pop-up
choice menu and asked to make a selection.  If the file is a library
file, the second argument can be given, and it should be one of the
reference names from the library, or the name of a cell defined in the
library.  If no second name is given, a pop-up listing the library
contents will appear, allowing the user to select a reference or cell.

The given given string can also consist of the name of a Cell
Hierarchy Digest (CHD) in memory, optionally followed by the name of a
cell known within the CHD hierarchy.  If no cell name is provided, the
cell name configured into the CHD is understood.  The string can also
contain the name of a saved CHD file, with an optional following cell
name.

The {\cb Cell Placement Control} panel is sensitive as a drop
receiver.  If a file name is dragged over the panel and the mouse
button released, the behavior is as if the {\cb New} button in the
masters menu was pressed, and the file name will be loaded into the
dialog window.


% -----------------------------------------------------------------------------
% xic:plot 022316
\section{The {\cb plot} Button: Generate SPICE Plot}
\index{plot button}
\index{SPICE plots}
\index{hypertext}
\epsfbox{images/plot.eps}

The {\cb plot} button, available only in electrical mode, gathers
input for plotting via {\WRspice}.

The prompt area displays the command string as it is being built. 
Clicking on nodes or device ``hidden'' targets (usually indicated by a
`$+$' symbol in the device schematic representation) will add
hypertext entries to the command string, and will add a marker to the
screen at the clicked-on location.  One can click anywhere on a wire,
or on subcircuit and device connection points to select nodes. 
Clicking on a marker will delete the marker, and the corresponding
entry from the string.

Some devices have ``hidden'' nodes for accessing internal variables
for plotting, such as current through a voltage source or the phase of
a Josephson junction.  By convention, these are indicated with a `$+$'
mark in the symbol.  For many devices, this will access the current
through the device.  The marker for a current has an orientation in
the direction of positive current flow.  Ordinary node markers have no
orientation, and access the node voltage.

One can click on reference points to any depth in the hierarchy,
though selection requires that the cell be showing as a schematic, and
as expanded.  To make selections inside a subcircuit that is shown as
a symbol, one can use proxy windows (see \ref{hyproxy}).  Holding down
both the {\kb Shift} and {\kb Ctrl} keys, and clicking on a subcircuit
instance, will bring up a sub-window displaying the master of the
clicked-on instance in schematic form.  One can click on objects in
this window in the normal way, and plot points will be added to the
prompt line.

Holding the {\kb Shift} key while clicking on a device of subcircuit
not over any node or ``hidden'' location will enter the hypertext
device or subcircuit name.  These are not often needed in plot command
strings, and the requirement of holding down {\kb Shift} prevents
unwanted returns.

Markers can be deleted by clicking on them, or by deleting the
corresponding hypertext in the prompt line.  Multiple markers can
reference the same node, as long as they are spatially distinct.

Existing marks can be dragged and dropped to a new location, which
must also reference a node or reference point, as for the initial
placement.  If dropped on an existing plot mark, the two marks will
exchange locations, both as marks in the drawing window, and hypertext
entries in the prompt line.

The prompt line text is equivalent to the string given to the {\cb
plot} command in {\WRspice}.  The string can be edited in the usual
way.  The user can add text to combine the hypertext references into
expressions involving other syntax elements known to {\WRspice}.  The
registers available through the {\cb S} and {\cb R} buttons to the
left of the prompt line can be used to save and restore plot command
strings.

The {\WRspice} parser can distinguish the expressions, however in some
cases the user must intervene to force an expected result.  For
example,
\begin{quote}
\vt v(1) -v(2)
\end{quote}
will be interpreted as {\vt (v(1)-v(2))}, i.e., the difference.  To
force a unary minus interpretation, one can enclose the second token
in double quotes or parentheses, i.e. {\vt v(1) "-v(2)"} will plot
{\vt v(1)} and negative {\vt v(2)}.  Note that white space is not
considered when interpreting the expression, and is required only to
separate identifier names.

One should refer to the {\WRspice} documentation for a complete
description of the syntax accepted by the {\cb plot} command.  The
command line can also contain keyword assignments which override
defaults used when composing the plot.  It is also possible to produce
X-Y plots by using the ``{\vt vs}'' keyword.  The expression following
``{\vt vs}'' will be used as the X scale for the other expressions.

The color used to render a plot reference mark in the schematic will
be the same color used for the plot trace of the result to which the
corresponding hypertext contributes (however, if the user has changed
the plotting colors in {\WRspice} or {\Xic}, this may not be true). 
The number (or letter) enclosed by the plot mark represents a count of
the hypertext entries found in the prompt line, left to right,
starting with 1.

If {\Xic} detects a syntax error, one or more plot marks may be
rendered with ``no'' color (actually the highlighting color is used). 
This is also true for the marks used in the X-scale of an X-Y plot.

The {\kb Enter} key terminates entry, and creates the plot if
{\WRspice} is available, and the circuit has been simulated with the
{\cb run} command.  In the {\cb deck} command, the string will be
added to the SPICE listing, when generated, as a {\vt .plot} line.

While the {\cb plot} command is active, it is possible to select text
labels in the normal way, other selections are inhibited.  This
allows labels to be selected and modified without having to
explicitly terminate the {\cb plot} command first.

The command text and mark locations are saved with the cell data when
written to disk, thus the {\cb plot} command string is persistent.


% -----------------------------------------------------------------------------
% xic:polyg 012715
\section{The {\cb polyg} Button: Create/Edit Polygons}
\index{polyg button}
\index{object creation!polygons}
\epsfbox{images/polyg.eps}

The {\cb polyg} button is used to create and modify polygons.  In
electrical mode, this functionality is available from the {\cb poly}
menu selection brought up by the {\cb shapes} button.  When A polygon
is created by clicking on each vertex in sequence, and is terminated
by clicking again on the initial vertex, or clicking on the final
vertex twice.  The vertices can be undone and redone with the {\kb
Tab} key and {\kb Shift-Tab} combination, which are equivalent to the
{\cb Undo} and {\cb Redo} commands.

While the command is active in physical mode, the cursor will snap to
horizontal or vertical edges of existing objects in the layout if the
edge is on-grid, when within two pixels.  When snapped, a small dotted
highlight box is displayed.  This makes it much easier to create
abutting objects when the grid snap spacing is very fine compared with
the display scaling.  This is also applied to the first vertex of
polygons being created, facilitating point list termination.  This
feature can be controlled from the {\cb Edge Snapping} group in the
{\cb Snapping} page of the {\cb Grid Setup} panel.
 
When adding vertices during polygon creation, the angle of each
segment can be constrained to a multiple of 45 degrees with the {\cb
Constrain angles to 45 degree multiples} check box in the {\cb Editing
Setup} panel from the {\cb Edit Menu}, in conjunction with the {\kb
Shift} and {\kb Ctrl} keys.  There are three modes:  call them
``no45'' for no constraint, ``reg45'' for constraint to multiples of
45 degrees with automatic generation of the segment from the end of
the 45 section to the actual point, and ``simp45'' that does no
automatic segment generation.  The ``reg45'' algorithm adds a 45
degree segment plus possibly an additional Manhattan segment to
connect the given point.  The ``simp45'' adds only the 45 degree
segment.  The mode employed at a given time is given by the table
below.  The {\et Constrain45} boolean variable tracks the state (set
or not set) of the check box.

\begin{tabular}{|l|l|l|} \hline
\multicolumn{3}{|c|}{\kb Constrain45 not set}\\ \hline
& {\kb Shift} up & {\kb Shift} pressed\\ \hline
{\kb Ctrl} up & no45 & reg45\\ \hline
{\kb Ctrl} pressed & simp45 & simp45\\ \hline\hline
\multicolumn{3}{|c|}{\kb Constrain45 set}\\ \hline
& {\kb Shift} up & {\kb Shift} pressed\\ \hline
{\kb Ctrl} up & reg45 & no45\\ \hline
{\kb Ctrl} pressed & simp45 & no45\\ \hline
\end{tabular}

In physical mode, a new polygon is tested for reentrancy and other
problems, and a warning message is issued if a pathology is detected. 
The new polygon is {\it not} removed from the database if such an
error is detected.  It is up to the user to make appropriate changes.

In electrical mode, if the current layer is the SCED layer, the
polygon will be created using the ETC2 layer, otherwise the polygon
will be created on the current layer.  It is best to avoid use of the
SCED layer for other than active wires, for efficiency reasons, though
it is not an error.  The {\cb Change Layer} command in the {\cb Modify
Menu} can be used to change the layer of existing objects to the SCED
layer, if necessary.  The outline style and fill will be those of the
rendering layer.  Polygons have no electrical significance, but can be
used for illustrative purposes. 

\subsection{Polygon Vertex Editing}
\index{polygon vertex editor}

On entering the {\cb polyg} command, if a polygon is selected, a
vertex editing mode is active on all selected polygons.  Each vertex
of the selected object is marked with a small highlighting box. 
Clicking on the edge of a selected polygon away from an existing
vertex will create a new vertex, which can subsequently be moved.

In order to operate on a vertex, it must be selected.  A vertex can be
selected by clicking on it, or by dragging over it.  Any number of
vertices can be selected.  After the selection operation, selected
vertices are shown marked with a larger box, and unselected vertices
are not marked.  Additional vertices can be selected, and existing
selected vertices unselected, by holding the {\kb Shift} key while
clicking or dragging over vertex locations.  Selecting a vertex a
second time will deselect it.

Selected vertices can be deleted by pressing the {\kb Delete} key. 
This will succeed only if after vertex removal the object does not
become degenerate.  In particular, one can not delete the object in
this manner.

The selected vertices can be moved by dragging or clicking twice.  The
selected vertices will be translated according to the button-down
location and the button up location, or the next button-down location
if the pointer did not move.  While the translation is in progress,
the new borders are ghost-drawn.

All vertex operations can be undone and redone through use of the {\cb
Undo} and {\cb Redo} commands.

With vertices selected, pressing the {\kb Esc} or {\kb Backspace} keys
will deselect the vertices and return to the state with all vertices
marked.

While in the {\cb polyg} command, with no object in the process of
being created, it is possible to change the selected state of polygon
objects, thus displaying the vertices and allowing vertex editing. 
Pressing the {\kb Enter} key will cause the next button 1 operation to
select (or deselect) polygon objects.  This can be repeated
arbitrarily.  When one of these objects is selected, the vertices are
marked, and vertex editing is possible.

If the vertex editor is active, i.e., a selected polygon is shown with
the vertices marked, clicking with the {\cb Ctrl} button pressed will
start a new polygon, overriding the vertex editor.  This can be used
to start a new polygon at a marked vertex location, for example. 
Without {\cb Ctrl} pressed, the vertex editor would have precedence
and would select the marked vertex instead of starting a new polygon.

While moving vertices, holding the {\kb Shift} key will enable or
disable constraining the translation angle to multiples of 45 degrees. 
If the {\cb Constrain angles to 45 degree multiples} check box in the
{\cb Editing Setup} panel from the {\cb Edit Menu} is checked, {\kb
Shift} will disable the constraint, otherwise the constraint will be
enabled.  The {\kb Shift} key must be up when the button-down occurs
which starts the translation operation, and can be pressed before the
operation is completed to alter the constraint.  These operations are
similar to operations in the {\cb Stretch} command.

\subsection{Wire to Polygon Conversion}
\index{wires!convert to polygons}

If any non-zero width wires are selected before the {\cb polyg}
command is entered, the user is given the option of changing the
database representation of these objects to polygons.  Is is not
possible to convert polygons to wires (except via the {\cb Undo}
command if the polygon was originally a wire).


% -----------------------------------------------------------------------------
% xic:put 062908
\section{The {\cb put} Button: Extract From Yank Buffer}
\index{put button}
\index{unerase}
\index{yank/put}
\epsfbox{images/put.eps}

The {\cb put} command allows the contents of the yank buffers to be
added to the current cell.  This command is available in physical
mode.  When parts of objects are erased with the {\cb erase} command,
the erased pieces are added to a five-deep yank buffer queue.  When
the {\cb put} button becomes active, the most recent deletion is ghost
drawn and attached to the pointer.  Clicking will place the contents
of the buffer at the location of the pointer.  The yank buffers can be
cycled through with the arrow keys.


% -----------------------------------------------------------------------------
% xic:round 100916
\section{The {\cb round} Button: Create Disk Object}
\index{round button}
\index{object creation!disks and ellipses}
\epsfbox{images/round.eps}

The {\cb round} button, only available in physical mode, will create a
disk polygon object.  The number of sides can be altered with the {\cb
sides} command.  If the user presses and holds the {\kb Shift} key
after the center location is defined, and before the perimeter is
defined by either lifting button 1 or pressing a second time, the
current radius is held for x or y.  The location of the shift press
defines whether x is held (pointer closer to the center y) or y is
held (pointer closer to the center x).  This allows elliptical objects
to be generated.

The {\kb Ctrl} key also provides useful constraints.  Pressing and
holding the {\kb Ctrl} key when defining the radius produces a radius
defined by the pointer position projected on to the {\et x} or {\et y}
axis (whichever is closer) defined from the center.  Otherwise,
off-axis snap points are allowed, which may lead to an unexpected
radius on a fine grid.

When the command is expecting a mouse button press to define a radius,
the value as defined by the mouse pointer (in microns) is printed in
the lower left corner of the drawing window, or the X and Y values are
printed if different.  Pressing {\kb Enter} will cause prompting for
the value(s), in microns.  If one number is given, a circular radius
is accepted, however one can enter two numbers separated by space to
set the X and Y radii separately.

While the command is active in physical mode, the cursor will snap to
horizontal or vertical edges of existing objects in the layout if the
edge is on-grid, when within two pixels.  When snapped, a small dotted
highlight box is displayed.  This makes it much easier to create
abutting objects when the grid snap spacing is very fine compared with
the display scaling.  This feature can be controlled from the {\cb
Edge Snapping} group in the {\cb Snapping} page of the {\cb Grid
Setup} panel.

\index{SpotSize variable}
If the {\et SpotSize} variable is set to a positive value, or the {\vt
MfgGrid} has been given a positive value in the technology file, tiny
round and donut figures are constructed somewhat differently.  the
figure is constructed somewhat differently.  Objects created with the
{\cb round} and {\cb donut} buttons will be constructed so that all
vertices are placed at the center of a spot, and a minimum number of
vertices will be used.  The {\cb sides} number is ignored.  This
applies only to figures with minimum radius 50 spots or smaller; the
regular algorithm is used otherwise.  An object with this
preconditioning applied should translate exactly to the e-beam grid. 
See \ref{spotsize} for more information.


% -----------------------------------------------------------------------------
% xic:run 042411
\section{The {\cb run} Button: Run SPICE Analysis}
\index{run button}
\index{running SPICE}
\index{loading rawfiles}
\index{running SPICE!output to file}
\epsfbox{images/run.eps}

The {\cb run} button, available only in electrical mode, will
establish interprocess communication with the {\WRspice} program.  If
a link can not be established, the {\cb run} command terminates with a
message.  If connection is established, then a SPICE run of the
circuit is performed.

The user is first prompted for the {\WRspice} analysis command string
to run.  This should be in a format understandable to {\WRspice} as an
interactive-mode command.  During prompting, the last six unique
analysis commands entered are available and can be cycled through with
the up and down arrow keys.
 
The first word in the analysis string is checked, and only words from
the following list will be accepted:
 
\begin{tabular}{llll}
\vt ac    & \vt loop  & \vt run  & \vt tran\\
\vt check & \vt noise & \vt send & \\
\vt dc    & \vt op    & \vt sens & \\
\vt disto & \vt pz    & \vt tf   & \\
\end{tabular}

The ``{\vt send}'' keyword is not a {\WRspice} command.  If given, the
circuit will be sent to {\WRspice} and sourced, but no analysis is
run.  Other commands can be sent to {\WRspice} with the {\cb spcmd}
button.

The link is established to the SPICE server ({\vt wrspiced} daemon)
named in the {\et SPICE\_HOST} environment variable, or the {\et
SpiceHost} ``!set'' variable (which overrides the environment).  If
neither is set, {\Xic} will attempt to attach to {\WRspice} on the
local machine.

By default, the {\WRspice} toolbar is visible when a connection has
been established.  This gives the user more control over {\WRspice} by
providing access to the visual tools.  If the {\et NoSpiceTools}
variable is set (with the {\cb !set} command), the toolbar will not be
visible.

During a simulation run, a small pop-up window appears, which contains
a status message, and a {\cb Pause} button.  Pressing {\cb Pause} will
pause the analysis.  It can be resumed by pressing the {\cb run}
button again.  The analysis can also be paused by pressing {\kb
Ctrl-c} in the controlling terminal (xterm) window.  A {\kb Ctrl-c}
press over a drawing window goes to {\Xic}, where it stops redraws and
other functions as usual.

{\Xic} is notified when a run is paused from {\WRspice} (using the red
X button in the toolbar), and will change state accordingly.  However,
{\Xic} is {\it not} notified when a run is restarted from {\WRspice}
(with the green triangle button in the toolbar), and will continue to
assume that {\WRspice} is inactive.  In this case, commands from
{\Xic} that communicate with {\WRspice} will pause any analysis in
progress before executing.  The user will have to resume the analysis
manually after the operation completes, either with the {\cb run}
button or from the {\WRspice} toolbar.
 
This affects the {\cb plot}, {\cb iplot}, and {\cb run} buttons, and
the {\cb !spcmd} command.  When a run is started or resumed with the
{\cb run} button in {\Xic}, these features are locked out, producing a
``WRspice busy'' message, and the run in progress is not affected.

The node connectivity is recomputed, if necessary, before the run.  If
the variable {\et CheckSolitary} is set with the {\cb !set} command,
then warnings are issued if nodes with only one connection are
encountered.  A SPICE file is generated internally, and transmitted to
{\WRspice} for evaluation.  Only devices and subcircuits that are
``connected'' will be included in the SPICE file.  A device or
subcircuit is connected if one of the following is true:
\begin{itemize}
\item{There are two or more non-ground connections.}
\item{There is one non-ground connection and one or more grounds.}
\item{There is one non-ground connection and no opens.}
\item{There is one non-ground connection and the object is a subcircuit.}
\end{itemize}

\index{.spinclude directive}
\index{.splib directive}
As a final step before sending the circuit text to SPICE, {\Xic} will
recursively expand all {\vt .include} and {\vt .lib} lines, replacing
the {\vt .include} lines with the actual file text, and the {\vt .lib}
lines with the indicated text block from the library.  This is to
handle the case where {\WRspice} is located on a remote machine, and
the user's files are on the local machine.  As in {\WRspice}, {\vt
.inc} is a synonym for {\vt .include}, and the `{\vt h}' option (strip
`\$' comments for HSPICE compatibility) is recognized.
 
The {\vt .include} and {\vt .lib} lines are generally inserted into
the SPICE text using the {\vt spicetext} label mechanism.  There may
be occasions where the expansion of these lines by {\Xic} is
undesirable, such as when the included file resides on the SPICE host,
or one wishes to use the {\WRspice} {\vt sourcepath} variable to
resolve the file.  To this end, the user can use the {\vt .spinclude}
keyword rather than {\vt .include}, and {\vt .splib} rather than {\vt
.lib}.  The {\vt .sp} directives use the same syntax as the normal
keywords, however {\Xic} will not attempt to expand these directives,
rather it changes the keyword to the normal directive (``{\vt
.include}'' or ``{\vt .lib}'').  Thus, {\WRspice} will see and handle
these inclusions.

{\WRspice} release 2.2.60 and later recognize {\vt .spinclude} as a
synonym for {\vt .include}.  This allows {\WRspice} to be able to
directly source top-level cell files, where the SPICE listing may
contain {\vt .spinclude} lines, without syntax errors.  {\WRspice}
release 2.2.62-2 and later recognize {\vt .splib} as a synonym for
{\vt .lib}, and is able to handle {\vt .lib} constructs sent from
{\Xic}.

Sometimes it may be desirable to place the output of a SPICE run
initiated from {\Xic} into a rawfile, rather than saving the output
internally.  To do this, use the {\vt spicetext} labels to add an
analysis string, such as ``{\vt spicetext .tran 1p 1000p}'' (note that
the `.' ahead of ``tran'' is necessary).  One can also add a save
command using ``{\vt spicetext *\#save v(1) ...}'' to save only a
subset of the circuit variables.  The ``{\vt *\#}'' means that the
save is executed as a shell command, ``{\vt .save}'' is ignored since
{\WRspice} is in interactive mode.  Then, for the analysis string from
{\Xic}, use ``{\vt run} {\it filename\/}'',
where {\it filename\/} is the name for the rawfile.  The run will be
performed, but the output data will go to the file, so don't expect to
see it with the {\cb plot} command.  If the {\it filename\/} is
omitted, the run will be performed with internal storage as usual.

The {\cb !spcmd} command can be used to give arbitrary commands to
{\WRspice}.


% -----------------------------------------------------------------------------
% xic:shapes 111908
\section{The {\cb shapes} Button: Add Predefined Features}
\index{shapes button}
\index{shapes templates}
\index{object creation!shapes templates}
\epsfbox{images/shapes.eps}

The {\cb shapes} button appears in the electrical mode side menu. 
Pressing this button provides a pull-down menu of different outlines
that can be applied to drawings.  These outlines have no electrical
significance, but can be used for illustrative purposes.  In
particular, in symbolic mode, this facilitates creating symbol
representations.  After a selection is made from the pull-down menu,
the shape outline is ghost-drawn and attached to the pointer.  The
object is placed at locations where the user clicks.

\index{shape templates!box}
\index{shape templates!poly}
\index{shape templates!arc}
\index{shape templates!dot}
\index{shape templates!tri}
\index{shape templates!ttri}
\index{shape templates!and}
\index{shape templates!or}
The current choices in the pull-down menu are:

\begin{tabular}{|l|p{4.5in}|} \hline
{\cb box} & Create a box, like the physical mode {\cb box} command.\\ \hline
{\cb poly} & Create a polygon, like the physical mode {\cb polyg} command.\\
 \hline
{\cb arc} & Create an arc, similar to the physical mode {\cb arc} command.\\
 \hline
{\cb dot} & Place a dot (an octagonal polygon).\\ \hline
{\cb tri} & Place a triangle (buffer symbol).\\ \hline
{\cb ttri} & Place a truncated triangle symbol.\\ \hline
{\cb and} & Place an AND gate symbol.\\ \hline
{\cb or} & Place an OR gate symbol.\\ \hline
{\cb Sides} & Set the number of sides used to approximate rounded geometry,
 similar to the {\cb sides} command in physical mode.\\ \hline
\end{tabular}

None of these shapes have significance electrically, and for
efficiency is is best to avoid using the SCED layer for these objects. 
In particular, arcs are actually wires, and arc vertices on the SCED
layer are considered in the connectivity establishment.  If the
current layer is SCED when one of these objects is created, the object
is instead created on the ETC2 layer.  If the object must be on the
SCED layer, the {\cb Change Layer} command in the {\cb Modify Menu}
can be used to move it to that layer.

The {\et dot}, {\et tri}, {\et ttri}, {\et and}, and {\et or} choices
work a little differently from the others.  After selection, a ghost
rendering of the shape is attached to the pointer, and the objects are
placed where the user clicks.  The object can be modified with the
arrow keys:

\begin{tabular}{ll}
\kb    Up          & expand by 2\\
\kb    Right       & expand by 10\%\\
\kb    Down        & shrink by 2\\
\kb    Left        & shrink by 10\%\\
\kb    Shift-Up    & stretch vertically 10\%\\
\kb    Shift-Right & stretch horizontally 10\%\\
\kb    Shift-Down  & shrink vertically 10\%\\
\kb    Shift-Left  & shrink horizontally 10\%\\
\kb    Ctrl-Arrows & cycle through 90 degree rotations\\
\end{tabular}


% -----------------------------------------------------------------------------
% xic:sides 021615
\section{The {\cb sides} Button: Set Rounded Granularity}
\index{sides button}
\index{round figure sides}
\epsfbox{images/sides.eps}
The {\cb sides} button, available in physical mode, allows the user to
set the number of sides used to approximate rounded geometries. 
Larger numbers give better resolution, but decrease efficiency.  The
number provided is the sides for a full 360 degrees, arcs will use
proportionally fewer. 

The setting tracks the {\et RoundFlashSides} variable.  If the
variable is not set, 32 sides will be used.  The acceptable range is
8--256.

The setting applies when new round objects are created with the {\cb
round}, {\cb donut}, and {\cb arc} buttons in the physical side menu,
or the equivalent script functions. 

In electrical mode, the number of sides used has a separate setting
using the {\et ElecRoundFlashSides} variable, which can be set from
the {\cb sides} entry in the menu presented by the {\cb shapes} button
in the electrical side menu.


% -----------------------------------------------------------------------------
% xic:spcmd 110213
\section{The {\cb spcmd} Button: Execute {\WRspice} Command}
\index{spcmd button}
\index{SPICE command}
\index{WRspice command}
\epsfbox{images/spcmd.eps}

This will prompt the user, in the prompt area, for a command that will
be sent to {\WRspice} for execution.  If the user simply presses {\kb
Enter} without entering a command, or enters the command ``{\vt
setup}'', the {\cb WRspice Interface Control Panel} will appear, from
which the interface to {\WRspice} can be set up.  This panel is
described in the next section.

Otherwise, a stream to {\WRspice} will be established, if one is not
already active, providing a means for running arbitrary {\WRspice}
commands.  However, commands that cause {\WRspice} to prompt the user
for additional input (such as {\vt setplot}) will not work properly,
as the communication is one-way only and not interactive.  Text output
goes to the console window.

In addition to the {\WRspice} commands, the client-side directive
\begin{quote}
{\vt send} {\it filename}
\end{quote}
is available.  The {\it filename} is that of a local SPICE input file. 
The file will have {\vt .include} and {\vt .lib} lines expanded
locally, and {\vt .spinclude}, {\vt .splib} lines will be converted to
``{\vt .include}'', ``{\vt .lib}'', as is done for decks created within
{\Xic}.  The result will be sent to {\WRspice} and sourced.

This operation is basically identical to the {\cb !spcmd} command.

% -----------------------------------------------------------------------------
% xic:spif 110213
\subsection{The {\WRspice} Interface Control Panel}
\index{WRspice interface control panel}

This panel appears when the user presses the {\cb spcmd} button in the
electrical side menu, and either gives no command at the prompt, or
enters ``{\vt setup}''.  It provides entry areas for setting the
variables which control the interprocess communication channel to the
{\WRspice} circuit simulator, and other simulation settings.  Most
users will probably never need to use this panel or set the associated
variables as the defaults suffice in most installations.

The {\cb WRspice Interface Control Panel} contains the following entry
objects.

\begin{description}
\item{\cb List all devices and subcircuits}\\
This check box corresponds to the {\et SpiceListAll} variable.  When
checked, all devices and subcircuits in the schematic will be included
in SPICE output.  Otherwise, only devices and subcircuits that are
``connected'' will be included, as explained in the {\cb deck} and
{\cb run} command descriptions.

\item{\cb Check and report solitary connections}\\
This check box corresponds to the {\et CheckSolitary} variable.  If
checked, warning messages will be issued when electrical netlists are
generated for nodes having only one connection.  This affects the {\cb
run} and {\cb deck} commands, and the {\cb Dump Elec Netlist} command
in the {\cb Extract Menu}.

\item{\cb Don't show WRspice Tool Control panel}\\
This check box corresponds to the {\et NoSpiceTools} variable.  When
running {\WRspice} from {\Xic}, by default the {\WRspice} toolbar is
shown, if {\WRspice} is running on the local machine.  This gives the
user much greater flexibility and control over {\WRspice}.  If this
check box is checked, {\it before} the connection to {\WRspice} is
established, the toolbar will not be visible.

This check box will also control toolbar visibility if the {\vt
wrspiced} daemon is used.  However, this requires {\vt wrspiced}
distributed with wrspice-3.0.7 or later.  {\bf If this variable is set
with an earlier {\vt wrspiced} release, the {\WRspice} connection will
not work!}

\item{\cb Spice device prefix aliases}\\
This group consists of a check box and a text entry area.  When the
box is checked, the text in the entry area will be used to set the
{\et SpiceAlias} variable.  This can be set to a string which will
modify the printing of device names in SPICE output.  The aliasing
operates on the first token of device lines.  The format of the string
is
\begin{quote}
{\it prefix1\/}{\vt =}{\it newprefix1} {\it prefix2\/}{\vt =}{\it
newprefix2} ...
\end{quote}

This will cause lines beginning with {\it prefix} to have {\it prefix}
replaced with {\it newprefix}.  If the ``{\vt =}{\it newprefix\/}'' is
omitted, that line will not be printed.  For example, to map all
devices that begin with `B' to `J', and to suppress all `G' devices,
the string is

\begin{quote} \vt
B=J G
\end{quote}

Note that there can be no space around the `{\vt =}'.  With the text
entered and the box checked, the indicated mappings will be performed
as SPICE text is produced.

\item{\cb Remote WRspice server host name}\\
This group consists of a check box and a text entry area.  When the
box is checked, the {\et SpiceHost} variable is set to the text in the
text area.

The text should be the name of the host which maintains a server for
remote {\WRspice} runs.  If set, this will override the value of the
{\et SPICE\_HOST} environment variable.  The host name specified in
the {\et SPICE\_HOST} environment variable and the {\et SpiceHost}
{\cb !set} variable can have a suffix ``{\vt :}{\it portnum\/}'',
i.e., a colon followed by a port number.  The port number is the port
used by the {\vt wrspiced} program on the specified server, which
defaults to 6114, the IANA registered port for this service.  If the
server uses a non-standard port, and the {\vt wrspice/tcp} service has
not been registered (usually in the {\vt /etc/services} file) on this
port, the port number must be provided.

\item{\cb Remote WRspice server host display name}\\
This group consists of a check box and a text entry area.  When the
box is checked, the {\et SpiceHostDisplay} variable is set to the text
in the text area.

This text can be set to the X display string to use on a remote host
for running {\WRspice} through a {\vt wrspiced} daemon, from {\Xic} in
electrical mode.  This is intended to facilitate use of {\vt ssh} X
forwarding to take care of setting up permission for the remote host
to draw on the local display.  See the description of the {\et
piceHostDisplay} variable for complete details.

\item{\cb Path to local WRspice executable}\\
This group consists of a check box and a text entry area.  When the
box is checked, the {\et SpiceProg} variable is set to the text in the
text area.

The text is the full path name of the {\WRspice} executable.  This is
useful if there are multiple versions of {\WRspice} available, or the
binary has been renamed, or is not located in the standard location. 
If given, the value supersedes the values from environment variables
or other variables (and corresponding entries) which also set a path
to the SPICE executable.

\item{\cb Path to local directory containing WRspice executable}\\
This group consists of a check box and a text entry area.  When the
box is checked, the {\et SpiceExecDir} variable is set to the text in
the text area.

The text is a path to the directory to search for the {\WRspice}
executable.  If given, the value overrides the {\et SPICE\_EXEC\_DIR}
environment variable.  The default search location is ``{\vt
/usr/local/xictools/bin}'', or, if the {\et XT\_PREFIX} environment
variable has been set, its value will replace ``{\vt /usr/local}''.

\item{\cb Assumed WRspice program executable name}\\
This group consists of a check box and a text entry area.  When the
box is checked, the {\et SpiceExecName} variable is set to the text in
the text area.

The text will give the expected name of the {\WRspice} binary.  If
given, the value overrides the {\et SPICE\_EXEC\_NAME} environment
variable.  The default name is ``{\vt wrspice}''.

\item{\cb Assumed WRspice subcircuit concatenation character}\\
This group consists of a check box and a text entry area.  When the
box is checked, the {\et SpiceSubcCatchar} variable is set to the text
in the text area.  See the description of the variable for information
about this setting.

\item{\cb Assumed WRspice subcircuit expansion mode}\\
This group consists of a check box and a menu.  When the box is
checked, the {\et SpiceSubcCatmode} variable is set to the current
menu selection.  See the description of the variable for information
about this setting.
\end{description}


% -----------------------------------------------------------------------------
% xic:spin 012815
\section{The {\cb spin} Button: Rotate Objects}
\index{spin button}
\index{object rotation}
\epsfbox{images/spin.eps}

The {\cb spin} button, available in physical mode, allows rotation of
boxes, polygons, and wires by an arbitrary angle, and subcells and
labels by multiples of 45 degrees.  If no objects are selected, the
user is requested to select an object.  With the object selected, the
user is asked to click on the origin of rotation.  The selected
objects are ghost-drawn, and rotated about the reference point as the
pointer moves.

If the {\cb Constrain angles to 45 degree multiples} check box in the
{\cb Editing Setup} panel from the {\cb Edit Menu} is checked, the
angle will be constrained to multiples of 45 degrees.  Pressing the
{\kb Shift} key will remove the constraint.  If the check box is not
checked, holding the {\kb Shift} key will impose the constraint.  Thus
the {\kb Shift} key inverts the effect of the check box.  However, if
the selected objects include a subcell or label, the angle will always
be constrained to multiples of 45 degrees.  The {\et Constrain45}
variable tracks the state (set or unset) of the check box.

During rotation, the angle is displayed in the lower left corner of
the drawing window.  The readout defaults to degrees, pressing the
`{\vt r}' key will switch to radians, and pressing the `{\vt d}' key
will switch back to degrees.  Pressing the spacebar will toggle
between radians and degrees.

At this point, one can click to define the rotation angle, or an
absolute angle can be entered on the prompt line.  To enter an angle,
either press {\kb Enter} or click on the origin marker, then respond
to the prompt with an angle in degrees.  In either case, the rotated
boundaries of the selected objects are attached to the pointer, and
new objects can be placed by clicking.  Ordinarily, the original
objects will be deleted, however if the {\kb Shift} key is held while
clicking, the original objects are retained.  Instead of clicking, one
can press the {\kb Enter} key, which will simply rotate the selected
objects around the reference point.

When the {\cb spin} command is at the state where objects are
selected, and the next button press would establish the rotation
origin, if either of the {\kb Backspace} or {\kb Delete} keys is
pressed, the command will revert the state back to selecting objects. 
Then, other objects can be selected or selected objects deselected,
and the command is ready to go again.  This can be repeated, to build
up the set of selections needed.

At any time, pressing the {\cb Deselect} button to the left of the
coordinate readout will revert the command state to the level where
objects may be selected to rotate.

The undo and redo operations (the {\kb Tab} and {\kb Shift-Tab}
keypreses and {\cb Undo}/{\cb Redo} in the {\cb Modify Menu}) will
cycle the command state forward and backward when the command is
active.  Thus, the last command operation, such as setting the angle
by clicking, can be undone and restarted, or redone if necessary.  If
all command operations are undone, additional undo operations will
undo previous commands, as when the undo operation is performed
outside of a command.  The redo operation will reverse the effect,
however when any new modifying operation is started, the redo list is
cleared.  Thus, for example, if one undoes a box creation, then starts
a rotation operation, the ``redo'' capability of the box creation will
be lost.

It is possible to change the layer of rotated objects.  During the
time that newly-rotated objects are ghost drawn and attached to the
mouse pointer, if the current layer is changed, the objects that are
attached can be placed on the new layer.  Subcells are not affected. 

How this is applied depends on the setting of the {\et
LayerChangeMode} variable, or equivalently the settings of the {\cb
Layer Change Mode} pop-up from the {\cb Set Layer Chg Mode} button
in the {\cb Modify Menu}.  The three possible modes are to ignore the
layer change, to map objects on the old current layer to the new
current layer, or to place all objects on the new current layer.  If
the current layer is set back to the previous layer before clicking to
locate the new objects, no layers will change.

Note that this operation can change boxes to polygons and vice-versa. 
The rotation can be performed by clicking or dragging, however an
angle can only be entered textually if the clicking mode is used. 


% -----------------------------------------------------------------------------
% xic:style 100412
\section{The {\cb style} Button: Set/Change Wire Style}
\index{style button}
\index{wire width setting}
\index{wire end style setting}
\epsfbox{images/style.eps}

The {\cb style} button, available in physical mode, pops up a menu of
options for the presentation style of wires.  The {\cb Wire Width}
choice sets the default width if no wires are selected, or changes the
width of selected wires.  If there are wires selected, {\Xic} prompts
for a new wire width for the selected wires, and the selected wires
will have their widths altered.  The new width should not be less than
the minimum width ({\et MinWidth} design rule) for the layers.

If there are no applicable wires selected, the default wire width for
the current layer is set, which is constrained to be greater or equal
to the minimum width.  Wires subsequently created on the present layer
will have the new width.

The other choices set the default end style if no applicable wire is
selected, or changes selected wires to the chosen end style if wires
are selected.  All selections depend on layer-specific mode.  In
layer-specific mode, only selected wires on the current layer are
changed.  Otherwise, all selected wires are changed.

The possible end styles are flush ends, extended rounded ends, and
extended square ends.  The extended styles project the length of the
wire by half of the width beyond the terminating vertex.  The button
icon changes to indicate the present wire end style with a small dot.


% -----------------------------------------------------------------------------
% xic:subct 110713
\section{The {\cb subct} Button: Set Subcircuit Connections}
\label{subct}
\index{subct button}
\index{subcircuit terminals}
\epsfbox{images/subct.eps}

The {\cb subct} button, available in the electrical side menu, allows
electrical connection terminals to be added to a circuit.  The
terminals are points at which electrical connections are defined, as
in the SPICE subcircuit definition.  Terminal definition is necessary
if the circuit is to be used as a subcircuit in another circuit with
connections to the instance (it is possible for a subcircuit to
connect to global nets only (see \ref{nodmp}), in which case the
master and instances would have no terminals).  The terminals are also
used by the extraction system and can provide an initial association
of a particular schematic net and physical conductor group. 

Terminals can only be created in electrical mode.  Once created, a
terminal's flags may be edited so as to enable a corresponding
terminal location in the physical layout.  The extraction system will
most often find suitable physical terminal locations automatically,
however there are times when the user may need to place terminals
manually, which can be done with the {\cb Edit Terminals} button in
the {\cb Views and Operations} page of the {\cb Extraction Setup}
panel from the {\cb Setup} button in the {\cb Extract Menu}, while in
physical mode.  In electrical mode, this same button is equivalent to
the {\cb subct} button in the side menu.

Subsequent to creation with the present command, terminals can be made
visible with the {\cb terms} button in the electrical side menu. 
While in physical mode, the terminals will be visible in electrical
windows when either the {\cb All Terminals} or {\cb Cell Terminals
Only} check boxes in the {\cb Show} group in the {\cb Views and
Operations} page of the {\cb Extraction Setup} panel is checked. 

The terminals must be defined in the schematic representation of the
cell, whether or not the cell will ultimately be symbolic (see
\ref{symbolic}).  The terminals can be created and deleted only in the
schematic.  Once created, they will be visible in the symbol view, but
must be moved to the desired location by hand.  In the symbol view
(only) each terminal can have arbitrarily many copies or itself at
different locations, each one of which is an equivalent connection
point for the subcircuit.  This facilitates, for example, tiling.  If
an equivalent connection point appears on either side of the instance,
then placing a row of these instances side-by-side will automatically
connect this node to all of the instances.  This applies only to the
symbolic representation.  In the schematic, each cell terminal has a
single connection point.

In {\Xic}, there are two types of cell contact terminals.

\begin{description}
\item{Scalar terminals}\\
These are the ``normal'', single-contact terminals.  These terminals
actually convey the connectivity information between the parent and
subcell schematics, and are the only terminals that may have
corresponding terminals in the physical layout.  A scalar terminal is
associated with a {\et node} property, of a cell or cell instance.

\item{Multi-contact ``bus'' terminals}\\
These terminals reference the scalar terminals and provide a means for
connecting a number of these terminals to a multi-conductor net in the
schematic.  The use of multi-conductor nets and multi-contact
terminals can greatly simplify a schematic visually.  Be advised that
a multi-conductor terminal only references existing scalar terminals,
which must exist.  These terminals are associated with a {\et bnode}
property, of a cell or cell instance.
\end{description}

In the schematic, by default ordinary scalar terminals can only be
located at connection points of the underlying geometry.  These are
the vertices of electrically-active wires, and device or subcell
connection points.  Clicking on such a point, if no terminal already
exists at the point, will create a new scalar terminal at the
location.  The {\cb Terminal Edit} panel will appear, which can be
used to apply a name for the terminal and edit other terminal
properties.  The new terminal will be shown highlighted to indicate
that it is the target of the {\cb Terminal Edit} panel.

\subsection{Virtual Terminals}
\index{virtual terminals}

If one holds the {\kb Ctrl} key while clicking anywhere except over
another terminal, a scalar terminal will be placed, whether or not it
is over a circuit connection point.  This is useful if the {\et
BYNAME} flag is to be set for the terminal, which indicates that it
will not connect by location, but by name matching only.  It is also
useful for implementing ``virtual'' terminals which connect to
nothing, but satisfy connectivity references in layout vs.  schematic
testing, and for other purposes.

Suppose one has a subcell with physical layout only that one wishes to
include in a full design hierarchy.  It may not be convenient to
create a schematic for the subcell, but it is desired that the
connections to the subcell be included in the LVS checking of the
overall design.  It is possible to assign ``virtual terminals'' to the
subcell.  Virtual terminals are treated like ordinary terminals in
connecting to instances of the subcell, but are ignored when creating
netlists for the subcell itself.

A virtual terminal is created in the {\cb subct} command by holding
the {\kb Ctrl} key while clicking on locations in the electrical
schematic (even if the schematic is empty).  They can be placed
anywhere except on top of another terminal; location is not important. 
Once created, they can be moved or deleted like ordinary terminals.

Once placed, they will be considered in establishing the connectivity
to instances of the cell, but will be ignored when establishing
connections within the cell.  Thus the cell looks like a ``black box''
with terminals.  Virtual terminals can be used along with ordinary
terminals if only part of the internal circuit is to be visible from
the outside.

In SPICE netlists, virtual terminals will appear in the subcircuit
connection list in {\vt .subckt} and call lines, but will not be
connected in the {\vt .subckt} definition.  One can use a {\vt
spicetext} label to add a {\vt .include} line to bring in a circuit
definition from a file, for example, to satisfy the references. 

In the graphical display, virtual terminals of the current cell are
shown with a beer-barrel outline for differentiation from the standard
terminals which are square.  The cell bounding box is expanded to
contain all virtual terminal locations.  The center of a virtual
terminal is a ``hot spot'' for hypertext node references, i.e.,
clicking on the terminal center will add the associated node to the
prompt line edit string in the {\cb plot} and {\cb iplot} commands and
when creating labels or properties.

\subsection{Multi-Contact Connectors}
\index{bus connectors}
\index{multi-contact connectors}

If the {\kb Shift} key is held while clicking in the schematic,
a new multi-contact terminal will be created.  A different version
of the {\cb Terminal Edit} panel will appear, allowing the new terminal
to be configured.

Multi-contact terminals reference scalar terminals, and every
referenced scalar terminal should exist.  The pop-up provides
convenience functions for creating the ``bit'' terminals.  In some
cases, these will be made invisible and not shown in either the
schematic or symbol, yet they must exist as they provide a crucial
data structure required for actual connectivity.

Named and unnamed multi-conductor terminals identify their constituent
bits quite differently.  If a terminal is named, the name is a net
expression (see \ref{netex}) that unambiguously specifies the names of
the scalar terminals.  These terminals are referenced by name, so
ordering is unimportant.

If a multi-conductor terminal is unnamed, it will at least have a
default range of {\vt [0:0]}.  The terminal also has an index number
that defaults to 0.  The bits are the scalar terminals with indices
starting with the multi-conductor terminal index value, through the
width of the multi-conductor range, contiguously and increasing.  In
this case, terminal ordering is obviously quite important.

See the {\cb Terminal Edit} panel description in \ref{termedit} for a
complete discussion of the conrfiguration options for multi-contact
terminals (and scalar terminals, too).

\subsection{Terminal Ordering}
\index{terminal order}

By default, a newly-created scalar terminal will be given the largest
index number, meaning that it will be the last terminal listed when
the subcircuit is represented in SPICE or other netlisting output. 
However, it is possible to insert new terminals at any point in the
sequence.
 
If the user types a number while the command is active, the number
will appear in the keypress buffer area for the drawing window that
has the keyboard focus.  If this number is within the range of
existing terminal indices, then new terminals created from this window
will be given this index, and existing terminals with this index or
larger will have their indices incremented.
 
Suppose for example that the cell contains five terminals, and one
needs to add a sixth, and further the new terminal should be the
fourth terminal in the sequence (index number 3).  While in the {\cb
subct} command, one can type ``3'' and note that ``3'' appears in the
keypress buffer area.  One can now click on a circuit location to
create the new terminal, and note that the new terminal is given index
3, the previous 3 is now 4, etc.  The backspace key can be used to
clear the keypress buffer, or the next new terminal added will also be
inserted as number 3.  Note that one can type ``0'' and leave this in
place, so that all new terminals will be added to the front of the
list rather than the back.

The indexing and order can also be changed with the {\cb Terminal Edit}
panel.

For multi-contact terminals, the index parameter provides ordering
information.  The terminal order assumed by {\Xic} is that a
multi-contact terminal is ordered by its index, ahead of a scalar
terminal with the same index.  If the multi-contact terminal is named,
then the index number is arbitrary, however by convention {\Xic} will
set the index to the index of the first (leftmost) bit.  If the
terminal is unnamed, the index is also the index of the first bit, and
in fact this identifies the first bit.

\subsection{Terminal Naming and Editing}
\index{terminal naming}

If no name is given to a scalar terminal, {\Xic} will use a default
name, which is an underscore followed by the internal index (the
number shown in the marker).  Otherwise, a short descriptive name can
be entered.  The name must follow the rules for a scalar net
expression (see \ref{netex}), that is, it must be a simple text name,
with or without a single index subscript.  A non-default name will be
displayed next to the terminal marker (the default name is assumed if
the entry is an underscore followed by one or two digits).

Clicking on an existing terminal will select it, and begin a move
operation.  A box will be ghost-drawn and attached to the mouse
pointer.  If the terminal is scaler, it can be moved to a new location
by clicking on a connection point not occupied by another terminal. 
It can be moved to a non-contact point by holding {\kb Ctrl} while
clicking, and the terminal becomes ``virtual''.  Multi-contact
terminals can be moved to any location not already occupied by a
terminal.

While a terminal is selected, pressing the {\kb Delete} key will
delete the terminal.  Pressing {\kb Backspace} or {\kb Esc} will
deselect the terminal, aborting the move operation.

If an existing terminal is clicked on with the {\kb Shift} key held
down, or double-clicked on (including being ``moved'' to the same
location), the {\cb Terminal Edit} panel will appear, allowing the
user to edit the parameters for the terminal.

From the {\cb Terminal Edit} panel, it is possible to make the
terminal invisible.  This can be applied to terminals that do not
participate in the visual connections, so clutter the display
needlessly.  The {\kb PageUp} and {\kb PageDown} toggle the display of
(otherwise) invisible terminals while the {\cb subct} command is
active.  Invisible terminals can also be selected for editing with the
{\cb Next} and {\cb Prev} buttons in the panel, which cycle through
the terminals to edit by the index value.

In symbolic mode, terminals can not be added or deleted, however they
can be moved to new locations consistent with the symbolic
representation.  Terminals can be moved by dragging, or by clicking on
a terminal then clicking on the new location.  Terminals can be placed
anywhere in the symbolic representation.  Further, if the {\kb Shift}
key is held during the terminal placement, the original terminal mark
is retained, i.e., a copy is made.  Any number of copies can be
placed.  Copies can be deleted by clicking to select, then pressing
the {\kb Delete} key.  The last remaining instance of a terminal can
not be deleted in this way, one must go to the schematic to delete the
terminal.


% -----------------------------------------------------------------------------
% xic:edtrm 110713
\section{The {\cb Terminal Edit} Pop-Up:  Editing Terminals}
\label{termedit}
\index{Terminal Edit panel}

The {\cb Terminal Edit} pop-up appears when using the {\cb subct}
button in the electrical side menu.  It also appears while in physical
mode and using the {\cb Edit Terminals} button from the {\cb Setup}
page of the {\cb Extraction Setup} panel, which is brought up with the
{\cb Setup} button in the {\cb Extract Menu}.  In either case, it
provides a means for editing various properties of a terminal,
including its name.

When the panel is visible, one of the terminals in the display is
highlighted, and the controls in the panel represent the state for the
highlighted terminal.  This is the ``target terminal'' which will be
modified by the panel.

The panel configures itself for either scalar or multi-contact
terminals in electrical mode, depending on the target terminal.  In
physical mode, only scalar terminals exist and not all parameters are
editable, and the panel configures itself accordingly.  The panel will
appear quite different in these three cases.

The target terminal can be changed by {\kb Shift}-clicking or
double-clicking over a different terminal.  It can also be changed
with the {\cb Prev}, {\cb Next}, and {\cb To Index} buttons found in
the panel.

Every scalar terminal has a unique index number.  This is the number
that is shown in the box which represents the terminal in the
schematic.  This represents the order of the terminals in calls to
instances of the current cell.  Bus terminals have an index number as
well, which must be one of the scalar terminal indices.  The ordering
of the multi-contact terminal is at the index, but {\it before} the
scalar terminal with the same index.

The {\cb Prev} button will cycle the target terminal to the one with
index value one less than the current index, wrapping at zero.  The
{\cb Next} button will cycle the target terminal in the opposite
direction.  The {\cb To Index} button and numeric entry area can be
used to change the target terminal to one with the specified index, of
the same type (scalar or multi-contact terminal) as the present
terminal. 

No actual change is made unless or until the {\cb Apply} button is
pressed.  Pressing {\cb Apply} will update the target terminal
according to the entries in the panel.  Changes made can be undone and
redone with the standard {\Xic} undo/redo operations.

Pressing the {\cb Dismiss} button will retire the panel.

\subsection{Electrical Scalar Terminal Editing}
\index{editing terminals}

At the top of the panel is a {\cb Terminal Index} numeric entry area. 
This can be used to change the terminals index number, and therefor
order in subcircuit references.  The renumbering is a two step
process:

\begin{enumerate}
\item{The present terminal is removed, and the remaining terminals are
renumbered, using unique and contiguous new index values (zero based).}

\item{The terminal is reinserted at the given index.  The terminal
that had that index and those larger will have their index values
incremented.}
\end{enumerate}

Changing the index of a scalar terminal does {\cb not} update the
multi-contact terminals!  The index values used in the bus terminals
may require compensating changes.

Just below is the {\cb Terminal Name} text entry area.  This will
contain the name of the terminal, which can be edited by the user. 
The entry can be empty, in which case {\Xic} will generate a name.

The {\cb Has physical terminal} check box should be checked if the
terminal will have a corresponding contact point in the physical
layout.  Setting this check box will allocate the internal data
structure needed to maintain the association.  In most cases, this
will be required.  It is not required if, for example, the user at
this point is only concerned with a schematic for simulations.  The
terminal can be edited and this box checked at a later time, when the
user is ready to add a layout.  The box is never checked for terminals
used in the schematic for special purposes that are perhaps related to
simulation, that have no ``real'' implementation in the layout.

When the {\cb Has physical terminal} check box is checked, the {\cb
Physical} group is un-grayed.  There are two controls in this group. 

\begin{description}
\item{\cb Layer Binding}\\
The {\cb Layer Binding} menu provides a layer name that is a hint used
by the extraction system when placing the physical terminal in the
layout.  This is set by {\Xic} after extraction, and if correct should
not be changed.  It is set by the user when a terminal is manually
placed, to resolve ambiguity about which layer the terminal connects
to.

\index{FIXED terminal flag}
\item{\cb Location locked by user placement}\\
When a terminal is manually placed, the {\cb Location locked by user
placement} check box will become checked.  This indicates that the
{\et FIXED} flag is set in the terminal.  Terminals with this flag set
will never be moved by {\Xic} during extraction/association. 
\end{description}

The location and layer must be correct or association will fail. 
Although {\Xic} will automatically place terminals, at times this will
fail and the user will have to place some terminals manually to obtain
correct or complete association.

Below the {\cb Physical} group are check boxes for setting some binary
options.

\begin{description}
\index{BYNAME terminal flag}
\item{\cb Set contact by name only}\\
This check box, when checked, sets the {\et BYNAME} flag in the
terminal which changes its interpretation in the schematic (it has no
effect in physical mode).  Ordinarily, a terminal is placed on a
``connection point'' of a wire net in the schematic (i.e., a vertex),
or a device or subcircuit contact point.  Association of the terminal
to that wire net is by location.  If there is no underlying connection
point, and the terminal has an assigned name, {\Xic} will then attempt
to add the terminal to an existing net with a matching name.  If this
flag is set, then the initial attempt to connect the terminal by
location will be skipped.  This is useful if the terminal is to be
made invisible, to avoid accidental connections.  The scalar wire nets
can be named with the {\cb Node (Net) Name Mapping} panel from the
side menu (see \ref{nodmp}).

\index{SCINVIS terminal flag}
\item{\cb Set terminal invisible in schematic}\\
This check box, when checked, sets the {\et SCINVIS} flag in the
terminal which prevents the terminal from being displayed in
schematics.  This is for terminals that are used only as bit
connections for a multi-contact connector.  Recall that every bit in a
multi-contact connector is a scalar connector, that must exist if a
connection is to be established.  If connectivity is to be provided
only via the multi-contact connector, the individual bits are visually
superfluous and clutter the display.  However, they can be made
invisible in the schematic with this flag.  They should probably also
have the {\et BYNAME} flag set as well, so that they don't make an
unintended connection by location.  The setting has no effect in
physical mode.

\index{SYINVIS terminal flag}
\item{\cb Set terminal invisible in symbol}\\
This check box controls the analogous {\et SYINVIS} flag, which when
set causes the terminal to be invisible in the symbolic
representation, if any.  This flag will almost always track the state
of the {\et SCINVIS} flag, but this is not an absolute requirement. 
It is possible for a schematic to use individual bits for connections,
whereas the symbol uses a multi-contact terminal, or vice-versa.
\end{description}

\subsection{Physical Terminal Editing}

In physical mode, the panel allows changes only within the {\cb
Physical} group described above.  That is, the {\cb Layer Binding}
choice and the {\cb Location locked by user placement} check box are
the only editable entries.  These have the purpose and functionality
as described above.  One must return to electrical mode to change
other parameters.

\subsection{Multi-Contact Connector Editing}

When the target terminal is a multi-contact connector, the
panel reconfigures itself to provide the appropriate entry areas. 

At the top of the panel is a numeric {\cb Term Index} entry area. 
Just below this are two text entry areas with labels {\cb Terminal
Name} and {\cb Net Expression}.  A ``bundle'' terminal may have a
separate simple text name, as well as its net expression.  If given,
the simple text name will be used as a name for the terminal in
instance placements of the cell.  The terminal in the instance will
look like a pure vector terminal with the given name, and a range
starting with zero and extending to the width of the bundle minus one.

If the terminal does not represent a bundle, then internally there is
only one name, which is the net expression.  This is obtained from the
two entry areas, which should not conflict or an error will result. 
Probably the best approach is to use the {\cb Net Expression} entry
for the complete expression, and leave the {\cb Terminal Name} entry
blank.  Alternatively, one could put a text name in the name entry,
and the subscripting, without a name or with the same name, in the
expression entry.

It is legitimate to not provide a name, but to provide subscripting
only.  In this case:

\begin{enumerate}
\item{The subscripting is ignored, except to determine the implied
width (number of conductors).}

\item{The connector maps the scalar terminal with index value equal to
the {\cb Term Index} entry and terminals with successive indices, the
total number of which will be equal to the connector width.  Thus,
scalar terminal order and the {\cb Term Index} value are critical in
this case.  It is up to the user to maintain consistency while
editing, as indices may change.  Probably, though, there is no reason
to use this approach, and not supply a terminal name.}
\end{enumerate}

If the terminal has a name, or has a bundle net expression, then the
name of every scalar terminal ``bit'' is well defined.  These are
found by name, so there is no order requirement, only an existence
requirement.  Furthermore, the {\cb Term Index} entry has much less
significance.  It is only used to assign an order for the terminal
relative to other terminals.  Specifically, the terminal order is just
ahead of the scalar terminal with the same index (multi-conductor
terminal index values are required to be unique).  {\Xic} will
initially assign the index as the index of the first scalar terminal
referenced.  This can be changed if necessary.

Below the three entry areas is a {\cb Delete} button, which will
delete the terminal if pressed.  This, and all other operations, can
be undone/redone with the standard {\Xic} {\kb Tab}/{\kb Shift-Tab}
keys and equivalent operations in the {\cb Modify Menu}.

There are two check boxes for terminal visibility in the schematic and
symbol, as we saw for scalar terminals.  It is unlikely that the user
would go to the trouble of implementing a multi-contact terminal and
not have it visible, but it is possible.

The {\cb Bus Term Bits} group provides some specialized functions for
working with the scalar terminals referenced.  These can be applied
only if the terminal has a name or is a bundle terminal.

\begin{description}
\item{\cb Check/Create Bits}\\
This will create, at the end of the scalar terminal list, any scalar
terminal referenced by the present terminal and not found.  Newly
created scalar terminals whill have {\et BYNAME}, {\et SCINVIS}, and
{\et SYINVIS} set, meaning that the terminals will be invisible and
make contact by name only.  The new terminals are placed at the same
location as the present terminal.  As they are invisible and they do
not connect by location, there is no problem with this.  In one way or
another, the scalar terminals referenced by a multi-conductor terminal
must exist for connectivity to be established, even if they are
invisible and never dealt with again after creation.  The {\cb
Check/Create Bits} button makes the scalar terminal creation quick and
easy.  Be aware, though, that it will probably still be necessary to
edit these terminals to set the physical data.

\item{\cb Reorder to Index}\\
This will create missing scalar terminals as above, but in addition it
will reorder the scalar terminals list so that the index values of the
referenced terminals are contiguous and start with the {\cb Term
Index} value.  All other considerations aside, this may be a ``nice''
way to organize the terminals.  It is also potentially more efficient. 
If the net expression does not duplicate any connection bits, an
internal mapping step can be skipped as it becomes an identity, saving
a little memory and time.  This is the same ordering used with
``unnamed'' terminals.
\end{description}

The four buttons below allow setting of the visibility flags of all of
the referenced scalar terminals.  It is unlikely that the flag states
would vary between the bits.

The remaining buttons operate as described for scalar terminal editing.


% -----------------------------------------------------------------------------
% xic:symbl 021912
\section{The {\cb symbl} Button: Symbolic Representation}
\label{symbolic}
\index{symbl button}
\index{symbolic mode}
\epsfbox{images/symbl.eps}

The {\cb symbl} button, available in electrical mode, allows instances
of a cell to be shown as a symbol, rather than as a schematic.  In the
symbolic representation, the substructure of the cell is never shown,
instead a simple figure representing the cell is displayed.  This can
simplify complex schematics.

When this button is active, the current cell is in symbolic mode.  It
is not possible to add subcircuits or devices in this mode, but any
geometry added will show as the symbolic representation.  If the cell
is saved with this button active, then the cell and its instances will
use the symbolic representation.

However, it is possible to apply a property to individual instances
of the cell to force the display of that instance non-symbolically
(as a schematic).  This property can be applied with the {\cb
Property Editor}.

If the {\cb No Top Symbolic} button in the {\cb Main Window} sub-menu
of the {\cb Attributes Menu}, or in the sub-window {\cb Attributes}
menu, is set, the top cell will always display as a schematic in the
window, whether or not the {\cb symbl} button is pressed.

When a new cell is opened for editing, the {\cb symbl} button will become
active and the symbolic representation shown if the cell was
previously saved in symbolic mode.  Pressing the button a second time
will revert to normal presentation.

While in symbolic mode, subcircuit terminals can not be added, however
existing terminals can be moved to new locations by dragging.  One
should first place the terminals, with the {\cb subct} command, in
normal mode.  After switching to symbolic mode, the terminals can be
moved to new locations, in the generally more compact symbolic
representation.  The actual locations of subcircuit connections is
dependent upon the mode.


% -----------------------------------------------------------------------------
% xic:terms 013113
\section{The {\cb terms} Button: Show Subcircuit Connections}
\index{terms button}
\index{show terminals}
\epsfbox{images/terms.eps}

When the {\cb terms} button is active, the electrical connection
points of the subcircuits are shown.  These points are placed with the
{\cb subct} command.  The {\cb terms} button is available in
electrical mode only.  When active, the physical terminals will be
shown in physical mode windows, as if the {\cb All Terminals} check
box in the {\cb Setup} page of the {\cb Extraction Setup} panel was
checked.  This panel is obtained from the {\cb Setup} button in the
{\cb Extract Menu}.  Similarly, in physical mode, when physical
terminals are visible, electrical terminals will also be visible in
electrical windows, as if the {\cb terms} button was active.


% -----------------------------------------------------------------------------
% xic:wire 102913
\section{The {\cb wire} Button: Create/Edit Wires}
\index{wire button}
\index{object creation!wires}
\epsfbox{images/wire.eps}

The {\cb wire} button is used to create or modify wires.  A wire is
created by clicking the left mouse button on each vertex location in
sequence, and is terminated by clicking on the final vertex twice.  In
electrical mode, wires are used to connect devices into circuits. 
Vertices are recognized as connecting points, and are created where
the wire crosses a device or subcircuit terminal or a vertex of
another wire.  The {\cb Connection Dots} button in the {\cb Attributes
Menu} can be used to display connections.  The vertices can be edited
to remove or reestablish connections.

In electrical mode, entering the {\cb wire} command will switch the
current layer to the SCED (active) layer.  The current layer can be
changed if necessary, but without the reverting it was too easy to
create wires on another layer, sometimes difficult to visually
differentiate, that will not be electrically active in the schematic
causing the circuit to not work.

While the command is active in physical mode, the cursor will snap to
horizontal or vertical edges of existing objects in the layout if the
edge is on-grid, when within two pixels.  When snapped, a small dotted
highlight box is displayed.  This makes it much easier to create
abutting objects when the grid snap spacing is very fine compared with
the display scaling.  This is also applied to the last vertex of wires
being created, facilitating point list termination.  This feature can
be controlled from the {\cb Edge Snapping} group in the {\cb Snapping}
page of the {\cb Grid Setup} panel.

When adding vertices during wire creation, the angle of each segment
can be constrained to a multiple of 45 degrees with the {\cb Constrain
angles to 45 degree multiples} check box in the {\cb Editing Setup}
panel from the {\cb Edit Menu}, in conjunction with the {\kb Shift}
and {\kb Ctrl} keys.  There are three modes:  call them ``no45'' for
no constraint, ``reg45'' for constraint to multiples of 45 degrees
with automatic generation of the segment from the end of the 45
section to the actual point, and ``simp45'' that does no automatic
segment generation.  The ``reg45'' algorithm adds a 45 degree segment
plus possibly an additional Manhattan segment to connect the given
point.  The ``simp45'' adds only the 45 degree segment.  The mode
employed at a given time is given by the table below.  The {\et
Constrain45} variable tracks the state (set or not set) of the check
box.

\begin{tabular}{|l|l|l|} \hline
\multicolumn{3}{|c|}{\kb Constrain45 not set}\\ \hline
& {\kb Shift} up & {\kb Shift} pressed\\ \hline
{\kb Ctrl} up & no45 & reg45\\ \hline
{\kb Ctrl} pressed & simp45 & simp45\\ \hline\hline
\multicolumn{3}{|c|}{\kb Constrain45 set}\\ \hline
& {\kb Shift} up & {\kb Shift} pressed\\ \hline
{\kb Ctrl} up & reg45 & no45\\ \hline
{\kb Ctrl} pressed & simp45 & no45\\ \hline
\end{tabular}

In physical mode, three end styles are available for nonzero width
wires:  {\et Flush}, {\et Rounded}, and {\et Extended}.  The end style
and the default width are set from the menu provided by the {\cb
style} button.  The end style of selected wires can be changed from
this menu, from within the {\cb wire} command or without.

The width of wires on a particular layer, or the widths of existing
wires, can be set of changed with the {\cb Wire Width} button in the
menu brought up with the {\cb style} button.  Zero-width wires are
accepted into the database if they contain more than one point.  In
physical mode, they probably should not be used, and they will, of
course, fail DRC tests.  They are allowed in the off chance that the
user uses them for annotation purposes.  Such lines will be invisible,
however, unless the layer pattern is outlined or solid.  In electrical
cells, zero-width wires are commonly used for the connecting lines,
and there is no question of their legality in electrical cells.  The
width of selected wires can be changed with this menu command, from
within the {\cb wire} command or without.

If the first vertex of a wire being created falls on an end vertex of
an existing wire on the same layer, the new wire will use the same
width and end style as the existing wire, overriding the defaults. 
The completed new wire will be merged with the existing wire, unless
merging is disabled.  Merging can be controlled from the {\cb Editing
Setup} panel from the {\cb Edit Menu}, and note also that the {\et
NoMerge} layer attribute will prevent merging.

Wires with a single vertex are acceptable if the width is nonzero and
the end style is rounded or extended.  These are rendered as an
octagon or box, respectively, centered on the vertex.

Existing wires can be converted to polygons through selection and
execution of the {\cb polyg} command.

\subsection{Wire Vertex Editor}
\index{wire vertex editor}

On entering the {\cb wire} command, if a wire is selected, a vertex
editing mode is active on all selected wires.  Each vertex of the
selected object is marked with a small highlighting box.  Clicking on
a selected wire away from an existing vertex will create a new vertex,
which can subsequently be moved.

In order to operate on a vertex, it must be selected.  A vertex can be
selected by clicking on it, or by dragging over it.  Any number of
vertices can be selected.  After the selection operation, selected
vertices are shown marked with a larger box, and unselected vertices
are not marked.  Additional vertices can be selected, and existing
selected vertices unselected, by holding the {\kb Shift} key while
clicking or dragging over vertex locations.  Selecting a vertex a
second time will deselect it.

Selected vertices can be deleted by pressing the {\kb Delete} key. 
This will succeed only if after vertex removal the object does not
become degenerate.  In particular, one can not delete the object in
this manner.

The selected vertices can be moved by dragging or clicking twice.  The
selected vertices will be translated according to the button-down
location and the button up location, or the next button-down location
if the pointer did not move.  While the translation is in progress,
the new borders are ghost-drawn.

All vertex operations can be undone and redone through use of the {\cb
Undo} and {\cb Redo} commands.

With vertices selected, pressing the {\kb Esc} or {\kb Backspace} keys
will deselect the vertices and return to the state with all vertices
marked.

While in the {\cb wire} command, with no object in the process of
being created, it is possible to change the selected state of wire
objects, thus displaying the vertices and allowing vertex editing. 
Pressing the {\kb Enter} key will cause the next button 1 operation to
select (or deselect) wire objects.  This can be repeated arbitrarily. 
When one of these objects is selected, the vertices are marked, and
vertex editing is possible.

If the vertex editor is active, i.e., a selected wire is shown with
the vertices marked, clicking with the {\cb Ctrl} button pressed will
start a new wire, overriding the vertex editor.  This can be used to
start a new wire at a marked vertex location, for example.  Without
{\cb Ctrl} pressed, the vertex editor would have precedence and would
select the marked vertex instead of starting a new wire.

While moving vertices, holding the {\kb Shift} key will enable or
disable constraining the translation angle to multiples of 45 degrees. 
If the {\cb Constrain angles to 45 degree multiples} check box in the
{\cb Editing Setup} panel from the {\cb Edit Menu} is checked, {\kb
Shift} will disable the constraint, otherwise the constraint will be
enabled.  The {\kb Shift} key must be up when the button-down occurs
which starts the translation operation, and can be pressed before the
operation is completed to alter the constraint.  These operations are
similar to operations in the {\cb Stretch} command.

\index{wire label}
\index{net name label}
\subsection{Associated Net Name Label}

In electrical mode, wires that participate in schematic connectivity
can have an associated text label.  The text provides a name for the
net (node) that contains the wire, and is equivalent to the placement
of a named terminal device (see \ref{devtbar}) at a vertex of the
wire.

To create and bind a label to a wire, first select the wire.  Then,
press the {\cb label} button in the side menu.  Enter the text, and
place the label in the normal way.  The text in the label will be
taken as a candidate net name (see \ref{nodmp}) for the net containing
the wire.

Unlike unlabeled wires, a wire with a label will never be merged with
adjacent wires.  Labeled wires play an important role in the
connectivity of some schematics, by defining multi-conductor wire
nets, and providing the ``taps'' to access the net.  Complete
information is provided in the Connectivity Overview in \ref{connect}
and the sections that follow.


% -----------------------------------------------------------------------------
% xic:xform 020815
\section{The {\cb xform} Button: Current Transform Panel}
\index{xform button}
\index{current transform}
\label{curxform}
\epsfbox{images/xform.eps}

The {\cb xform} button in the side menu brings up the {\cb Current
Transform} panel, which allows the current transform to be set.  The
current transform is applied to newly-placed subcells, and to objects
which are moved or copied.

The transform that is applied to an instance of a cell is saved in an
irreducible form in the database representation of the instance.  The
irreducible form is an optional reflect-y ($y \rightarrow -y$),
followed by an optional rotation, followed by the translation.  This
maps directly to the format used in GDSII files.  However, the
``current transform'' applies rotation {\it before} the reflection, so
that on screen, a reflect-x, for example, will flip the object's x
coordinates independent of any rotation angle, which is what users
tend to expect.  The transform string printed on unexpanded instances
and on the status line reflects this, i.e., forms like ``{\vt R45M}''
imply a 45 degree rotation followed by a reflect-y (``{\vt M}'' always
denotes reflect-y, reflect-x is equivalent to some other rotation and
reflect-y combination).  However, the transformation shown in an {\cb
Info} window will be reflect-y followed by a 315 degree rotation (in
this example), since the internal representation performs the
reflection before the rotation.

If the current transform is set to something other than the default
identity transform, the transform code is printed on the status line.

The following buttons and input fields are available in the
{\cb Current Transform} panel.

\begin{description}
\item{\cb Angle}\\
This choice menu allows setting the rotation component of the current
transform.  The menu allows a choice of rotations in increments of 90
degrees in electrical mode or 45 degrees in physical mode.

Pressing and holding the {\kb Ctrl} key and then pressing the left or
right arrow keys cycles through the transformation angles, whether or
not the {\cb Current Transform} panel is visible.  The right arrow
increases the angle, the left arrow decreases the angle.

\item{\cb Reflect X}\\
Add a reflection of the x-axis to the current transform.  The
X-reflection is toggled by the {\kb Ctrl-Down Arrow} key sequence,
whether or not the {\cb Current Transform} panel is visible.

\item{\cb Reflect Y}\\
Add a reflection of the y-axis to the current transform.  The
Y-reflection is toggled by the {\kb Ctrl-Up Arrow} key sequence,
whether or not the {\cb Current Transform} panel is visible.

\item{\cb Magnification}\\
This entry field allows setting of the magnification component of the
current transform.  Any number from 0.001 through 1000.0 can be
entered.

\item{\cb Identity Transform}\\
This button will save the current parameters to internal storage, and
reset these values to the default state (no transformation).  The
saved state can be restored with the {\cb Last Transform} button.

When the panel first appears, this button will have the keyboard focus
if the current transform is not the identity transform.  The user can
press {\kb Enter} to ``press'' the button.  This will cause the focus
to switch to the {\cb Dismiss} button, so that another {\kb Enter}
press will retire the panel.  Thus, one can very quickly restore the
identity transform using the {\cb xform} button accelerator (``{\vt
xf}'') followed by pressing the {\kb Enter} key twice.

\item{\cb Last Transform}\\
This button will restore the state of the current transform last saved
with the {\cb Identity Transform} button, or one of the recall
buttons.  If no state has been saved, the identity transform (the
default) is set.  Note that there is separate storage for the current
transform in electrical and physical modes.

When the panel first appears, if the current transform is the identity
transform, this button will have the keyboard focus.  In this case,
the same key sequence as described above can be used to quickly restore
the last transform.

\item{Store and Recall}\\
There are five internal registers for storage of transformation
parameters.  Separate registers are used in electrical and physical
modes.  Pressing these buttons will either save the current parameters
to a register, or set the parameters from a register.  After a recall,
the original parameters can be restored with the {\cb Last Transform}
button.
\end{description}


% -----------------------------------------------------------------------------
% xic:xor 012715
\section{The {\cb xor} Button: Exclusive-OR Objects}
\index{xor button}
\index{inverting polarity}
\index{object polarity inversion}
\epsfbox{images/xor.eps}

The {\cb xor} button facilitates inverting the polarity of layers, and
is available only in physical mode.  The operation is similar to the
{\cb box} command, however all previously existing boxes, polygons,
and wires on the same layer which overlap the created box become holes
in the new box.  Only boxes, polygons, and wires are inverted, other
structures are covered.  When a wire is partially xor'ed, the part of
the wire outside of the xor region becomes a polygon.  The {\cb
!layer} command can also be used to invert layer polarity, and is
recommended when an entire cell is to be inverted.

While the command is active in physical mode, the cursor will snap to
horizontal or vertical edges of existing objects in the layout if the
edge is on-grid, when within two pixels.  When snapped, a small dotted
highlight box is displayed.  This makes it much easier to create
abutting objects when the grid snap spacing is very fine compared with
the display scaling.  This feature can be controlled from the {\cb
Edge Snapping} group in the {\cb Snapping} page of the {\cb Grid
Setup} panel.

The {\cb box}, {\cb erase}, and {\cb xor} commands participate in a
protocol that is handy on occasion.

Suppose that you want to erase an area, and you have zoomed in and
clicked to define the anchor, then zoomed out or panned and clicked to
finish the operation.  Oops, the {\cb box} command was active, not
{\cb erase}.  One can press {\kb Tab} to undo the unwanted new box,
then press the {\cb erase} button, and the {\cb erase} command will
have the same anchor point and will be showing the ghost box, so
clicking once will finish the erase operation.

The anchor point is remembered, when switching directly between these
three commands, and the command being exited is in the state where the
anchor point is defined, and the ghost box is being displayed.  One
needs to press the command button in the side menu to switch commands. 
If {\kb Esc} is pressed, or a non-participating command is entered,
the anchor point will be lost.

