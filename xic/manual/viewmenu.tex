% -----------------------------------------------------------------------------
% Xic Manual
% (C) Copyright 2009, Whiteley Research Inc., Sunnyvale CA
% $Id: viewmenu.tex,v 1.39 2015/03/07 21:18:34 stevew Exp $
% -----------------------------------------------------------------------------

% -----------------------------------------------------------------------------
% xic:viewmenu 061608
\chapter{The View Menu: Alter Presentation}
\index{View Menu}
The {\cb View Menu} contains commands which alter the view shown
in the drawing windows.

The table below lists the commands found in the {\cb View Menu}.  The
internal command name is listed, as is the command function.

\begin{tabular}{|l|l|l|l|} \hline
\multicolumn{4}{|c|}{\kb View Menu}\\ \hline
\kb Label & \kb Name & \kb Pop-up & \kb Function\\ \hline\hline
\et View & \vt view & none & Set view in window\\ \hline
{\et Physical} or {\et Electrical} & {\vt phys} or {\vt sced} & none &
 Switch mode\\ \hline
\et Expand & \vt expnd & \cb Expand & Show detail in window\\ \hline
\et Zoom & \vt zoom & dialog & Change window scale\\ \hline
\et Viewport & \vt vport & sub-window & New drawing window\\ \hline
\et Peek & \vt peek & none & Show layers in area\\ \hline
\et Cross Section & \vt csect & sub-window & Show layers in cross-section\\
 \hline
\et Rulers & \vt ruler & none & Add transient gradations\\ \hline
\et Info & \vt info & \cb Info & Show cell/object parameters\\ \hline
\et Allocation & \vt alloc & \cb Memory Monitor & Show memory statistics\\
 \hline
\end{tabular}


% -----------------------------------------------------------------------------
% xic:view 061608
\section{The {\cb View} Button: Select Cell View}
\index{View button}
\index{view}
The {\cb View} button in the {\cb View Menu}, and the {\cb View} menu
of sub-windows, produces a drop-down menu of view choices for the
associated window.  For each window, the last five views are saved in
a list.  In addition, up to five views can be saved by pressing {\kb
Ctrl-n}.  These are assigned names consisting of the letters A--E. 
The drop-down menu entries are:

\begin{tabular}{ll}
\cb full  & center full view of cell\\
\cb prev  & cycle view backwards\\
\cb next  & cycle view forwards\\
\cb A-E   & if view saved with {\kb Ctrl-n}, set selected view\\
\end{tabular}

If the {\cb View} command is ``pressed'' by a key sequence, the center
full view is shown (same as for the {\kb Home} key).

The view list is cleared whenever a new cell is displayed, or whenever
the mode is changed for the window.

The {\kb Ctrl-Shift-Right} arrow and {\kb Ctrl-Shift-Up} arrow are
accelerators for {\cb prev}, {\kb Ctrl-Shift-Left} arrow and {\kb
Ctrl-Shift-Down} arrow are accelerators for {\cb next}, and {\kb
Ctrl-Shift-a} through {\kb Ctrl-Shift-e} are accelerators for {\cb A}
through {\cb E}.


% -----------------------------------------------------------------------------
% xic:phys 062109
\section{The {\cb Physical} Button: Show Physical Mode}
\index{Physical button}
\index{physical mode!switch to}
The {\cb Physical} button in the {\cb View Menu}, and the {\cb View}
menu of sub-windows, available only when the window is displaying
electrical mode, changes the display from electrical (schematic) mode
to physical mode.  In the main menu, this places {\Xic} into an
editing mode appropriate for physical representation.  In the
sub-windows, the {\cb Physical} button changes the view only.  Editing
can not be performed in a sub-window whose mode is not that of the main
window.

While in a {\cb Push} (see \ref{pushcmd}), the cell currently being
edited remains the current cell, but becomes top-level (i.e., not in
a {\cb Push}) in the new mode.  If the original mode is returned to
without editing a different cell, the {\cb Push} stack is retained. 
If a new cell is edited in the new mode, through a {\cb Push} or
otherwise, the original {\cb Push} context is lost.  This context is
also lost if the {\cb Clear} function in the {\cb Cells Listing} is
invoked.

The present display mode can be made immutable, with certain
side-effects, by setting the variable {\et LockMode}.


% -----------------------------------------------------------------------------
% xic:sced 062109
\section{The {\cb Electrical} Button: Show Electrical Mode}
\index{Electrical button}
\index{mode switch}
The {\cb Electrical} button in the {\cb View Menu}, and the {\cb View}
menu of sub-windows, available only when the window is displaying
physical mode, changes the display from physical to electrical
(schematic) mode.  In the main menu, this places {\Xic} into an
editing mode appropriate for electrical representation.  In the
sub-windows, the {\cb Electrical} button changes the view only. 
Editing can not be performed in a sub-window whose mode is not that of
the main window.

While in a {\cb Push}, the cell currently being edited
remains the current cell, but becomes top-level (i.e., not in a {\cb Push})
in the new mode.  If the original mode is returned to without editing
a different cell, the {\cb Push} stack is retained.  If a new cell is
edited in the new mode, through a {\cb Push} or otherwise, the
original {\cb Push} context is lost.  This context is also lost if the
{\cb Clear} function in the {\cb Cells Listing} is invoked.

The present display mode can be made immutable, with certain
side-effects, by setting the variable {\et LockMode}.


% -----------------------------------------------------------------------------
% xic:expnd 011118
\section{The {\cb Expand} Button: Expand Subcells}
\index{Expand button}
\index{expanded view}
\label{expansion}
The {\cb Expand} button in the {\cb View Menu}, and the {\cb View}
menu of sub-windows, brings up the {\cb Expand} pop-up, which controls
the expansion of subcells in the window.  All geometry is shown in an
expanded cell, whereas only the bounding box and possibly a name label
are shown in the unexpanded state.  If the cell happens to be an
array, the bounding box in the unexpanded state is shown as a dashed
line.  Ordinary instances have a solid line bounding box.

The label shown in unexpanded physical instances is by default the
instance name, which consists of the master name followed by a colon
separator and an index number.  The index is a 0--based sequence for
instances with a particular master.  The index count advances by the
size of the array for arrayed instances, leaving room in the sequence
for individual elements.  The index is in database order (top to
bottom then left to right of the upper left corner of the instance
bounding box), and is stable and reproducible as long as instance
sizes and placement locations remain the same.  If the boolean {\et
NoInstnameLabels} variable is set, the label will display the master
name only, which was the behavior in {\Xic} releases prior to 4.3.3.

After pressing {\cb Expand}, the pop-up appears.  The pop-up contains
a text entry area, a number of buttons which push specific text into
the entry area, an {\cb Apply} button, a {\cb Dismiss} button, and a
{\cb Help} button.  When the pop-up first appears, it is given the
keyboard focus.  Under most (if not all) window managers, one should
be able to type into the text entry area immediately.  Pressing the
{\kb Enter} key is equivalent to pressing the {\cb Apply} button. 
Thus, one can quickly change the expansion status entirely with the
keyboard accelerators (the change will apply to the window containing
the pointer).

For example, the default keypress mapping applies {\kb Ctrl-x} to the
{\cb Expand} button, so typing
\begin{quote}
{\kb Ctrl-x} {\kb 0} {\kb Enter}
\end{quote}
will set the expansion level to 0, and
\begin{quote}
{\kb Ctrl-x} {\kb a} {\kb Enter}
\end{quote}
will set the expansion level to ``all''.

The functions of the symbols which are recognized in the text string
will be described below.  The buttons which push text into the entry
area avoid the need for typing.  These are:

\begin{tabular}{ll}
{\kb $+$} & set to `$+$' (there can be multiple $+$'s added)\\
{\kb $-$} & set to `$-$' (there can be multiple $-$'s added)\\
{\cb All} & set to `{\vt all}'\\
{\cb 0-5} & set to `{\vt 0}' -- `{\vt 5}'\\
{\cb Peek Mode} & set to `{\vt p}' (available from main window only)\\
\end{tabular}

Pressing the {\cb Apply} button will pass the expansion string to the
internal expansion control function.

The characters which are recognized in the string are the letters {\vt
a}, {\vt n} for ``all'' and ``none'', one or more $+$ or $-$ symbols
(not mixed) which will increment or decrement the hierarchy depth of
expansion, a $+$ or $-$ followed by an integer, which will increase or
decrease the level by that integer, or simply an integer, which will
set the hierarchy depth to that integer.  Setting the hierarchy depth
to zero is the same as ``none''.  All subcells up to the hierarchy
depth are shown expanded.

Each drawing window has its own expansion parameters and {\cb Expand}
button.  When a sub-window is created, it inherits the expansion status
of the main window.  The expansion depth entered applies only to that
window. 

\subsection{Peek Mode}
\index{peek mode}
If the {\cb Expand} button from the {\cb View Menu} is selected, there
is an additional feature available:  peek mode, which is entered by
returning {\et p}.  This should not be confused with the {\cb Peek}
command in the {\cb View Menu}.  In peek mode, the expanded status of
individual cells and subcells can be set by clicking or dragging with
button 1.  Only cells below the current expansion depth are affected,
i.e., those that are normally displayed as unexpanded.  Thus, peek
mode has no effect if the expansion depth is set to ``all''.  Clicking
on an unexpanded cell, or dragging such that the cell is enclosed
within the drag rectangle, will cause that cell to be shown as
expanded (to one level) in the window where the button down event
occurred.  The process can be repeated to expose cells arbitrarily
deep in the hierarchy.  If the {\kb Shift} key is held during the
pointing operation, previously expanded cells are unexpanded.  This
applies only to cells below the expansion depth.  Note that unlike a
standard selection operation, in peek mode one can address subcells
below the first hierarchy level, so long as the parent cell is shown
expanded.

Peek mode works by setting a flag in the instance descriptor of a
subcell.  Instance descriptors are stored in the parent cell.  Suppose
that a design contains multiple instances of cell B, each of which
contains a left and right instance of cell A.  In peek mode, for
example, if the left instance of A is made to be expanded in an
instance of B (which of course is also expanded), this expansion of A
will appear in all instances of B which are expanded, not just the one
clicked on.  This is a consequence of the hierarchical nature of the
database, where each instance of B represents the contents of B, which
includes the instance of A with the flag set.

In peek mode, the operation applies to any window in which the pointer
was located when button one was pressed.  The result will be
consistent with the expansion depth of the particular window.  While
in peek mode, certain keyboard commands can be applied, which will
affect the window where the pointer was located when the key press
occurred.  The {\kb $+$} and {\kb $-$} keys increment and decrement
the expansion depth, a number key will set the expansion depth, {\kb
a} will set the depth to ``all'', and {\kb n} will set the depth to
``none'', as will {\kb 0} (zero).  Each window has independent
expansion parameters.  Setting the expansion depth to zero by pressing
{\kb 0} or {\kb n} will clear the peek mode display flags.  Otherwise,
the expansion depth and the peek mode display of cells are
independent.

\index{symbolic cells!expanded view}
In electrical mode, symbolic cells can be shown as expanded, with a
miniature rendition of the actual circuit inside the symbolic bounding
box area, in peek mode.  Click on the symbolic cell to expand, {\kb
Shift}-click to unexpand.  Wires connecting the circuit connections to
the symbol terminals are added.  This rendition is for visual purposes
only.  If a subcell placed in symbolic mode is later changed to
non-symbolic mode, the view of the parent cell is likely to look
horrid, since the subcircuits will probably overlap.  The peek mode
feature allows viewing of the underlying circuit without this problem.

The expansion status of a given subcell in a window is retained after
exiting peek mode, and after canceling a sub-window.  Setting the
expansion to ``none'' clears all expansion in peek mode.


% -----------------------------------------------------------------------------
% xic:zoom 032213
\section{The {\cb Zoom} Button: Zoom In/Out}
\index{Zoom button}
\index{zooming}
Pressing the {\cb Zoom} button in the {\cb View Menu} or the {\cb
View} menu of sub-windows brings up the {\cb Set Display Window}
pop-up.  This pop-up can change the scale (zoom) the window, or set a
new display region.

To change the scale, enter a factor into the {\cb Zoom Factor} entry
area, and press the associated {\cb Apply} button.  Factors greater
that 1.0 will zoom out.

Alternatively, one can enter the center x and y values and width (all
in microns) of a new region to display.  The coordinates are relative
to the origin of the displayed cell.  The width is the displayed width
of the region to be displayed in the window, the displayed height will
depend on the drawing window's aspect ratio.  Pressing the associated
{\cb Apply} button will redisplay the new location.

The windowing parameter entries are pre-loaded with the current window
parameters, and track any changes made when the pop-up is visible.

The right mouse button (button 3) can also be used to zoom, as can the
numeric keypad {\kb $+$} and {\kb $-$} keys which zoom in and out by a
factor of two, or by ten percent if {\kb Shift} is also held.

In the case where the panel is brought by from a sub-window that is
displaying a cross-section, there are additional controls that allow
adjustment of the vertical scaling used in the cross section display.

This control group has its own {\cb Apply} button.  The {\cb Auto
Y-Scale} check box disables the automatic vertical scaling when set. 
The automatic scaling maintains full view of the entire layer stack
independent of the zoom factor.  The {\cb Y-Scale} entry allows a
vertical scale factor to be entered, for use when displaying the cross
section.  When the {\cb Apply} button is pressed, the cross-section is
redrawn with the settings given.  The {\et XSectNoAutoY} variable will
be set or unset to reflect the state of the check box.  The {\et
XSectYScale} variable will be set to the scale factor if it is other
than unity.  See \ref{csect} more more information.


% -----------------------------------------------------------------------------
% xic:vport 053110
\section{The {\cb Viewport} Button: Create Sub-Window}
\label{subwin}
\index{Viewport button}
\index{viewports}
\index{sub-windows}
The {\cb Viewport} button in the {\cb View Menu} brings up a
sub-window, which is a display window similar to the main drawing
window.  The user is requested to point at the diagonal endpoints of
the region to be displayed in the sub-window.  Each viewport contains a
menu of attribute buttons which apply to that window only.  In
particular, the sub-window can display cells in either electrical or
physical mode, however editing operations are only possible if the
sub-window mode and cell match those of the main window.

The sub-window has a set of menus which control attributes which can be
set on a per-window basis.  When the cursor is in a sub-window,
characters entered are delivered to that sub-window, and an unambiguous
sequence match will select a sub-window button.  Matches are looked for
in the sub-window menu, the main menu, and any pop-up menus, in that
order.

The {\cb View} menu button commands are mostly analogous to the
commands found in the main {\cb View Menu}, however there are a few
entries in the {\cb View} menu that have no analogs in the main menu.

\begin{tabular}{|l|l|l|l|} \hline
\multicolumn{4}{|c|}{\kb Sub-Window View Menu} (additional)\\ \hline
\kb Label & \kb Name & \kb Pop-up & \kb Function\\ \hline\hline
\et Dump To File & \vt wdump & text entry & Dump window to image file\\ \hline
\et Show Location & \vt lshow & none & Show position in main window\\ \hline
\et Swap With Main & \vt swap & none & Swap contents with main window\\ \hline
\et Load New & \vt load & none & Load cell or file for viewing\\ \hline
\end{tabular}

The {\cb Attributes} menu is identical to the {\cb Main Window} sub-menu
found in the main {\cb Attributes Menu}.  The functions are the same (see
\ref{attrsubmenu} but apply to the sub-window only.  When a new sub-window
appears, it inherits the current attribute settings of the main window.

\begin{tabular}{|l|l|l|l|} \hline
\multicolumn{4}{|c|}{\kb Sub-Window Attributes Menu}\\ \hline
\kb Label & \kb Name & \kb Pop-up & \kb Function\\ \hline\hline
\et Freeze Display & \vt freez & none & Suppress redisplay\\ \hline
\et Show Context in Push & \vt cntxt & none & Show context in subedit\\ \hline
\et Show Phys Properties & \vt props & none & Show physical properties\\ \hline
\et Show Labels & \vt labls & none & Show labels\\ \hline
\et Label True Orient & \vt larot & none & Show labels transformed\\ \hline
\et Show Cell Names & \vt cnams & none & Show cell names\\ \hline
\et Cell Name true Orient & \vt cnrot & none & Show cell names transformed\\
  \hline
\et Don't Show Unexpanded & \vt nouxp & none & Don't show unexpanded
  subcells\\ \hline
\et Objects Shown & \vt objs & none & Object display control\\ \hline
\et Subthreshold Boxes & \vt tinyb & none & Show outline of subthreshold
  cells\\ \hline
\et Set Grid & \vt grid & {\cb Grid Parameters} & Set grid parameters\\ \hline
\end{tabular}

If a cell shown in a sub-window is the cell shown in the main window,
with the same mode (physical or electrical), then all editing
operations will work in the sub-window as well as the main window.  The
sub-window will display all highlighting, terminals, and other special
markings.  If the sub-window shows a different cell, then in that
window selections and editing are not possible, and no highlighting or
special markings are shown.

When the {\cb Viewport} command is used to create a new sub-window, the
sub-window will initially show the same cell as the main window.

The sub-windows are sensitive as drop-receivers from the file manager
and other listing pop-ups.  File or cell names can be dragged from the
listing pop-up and dropped in a sub-window, which will cause that
cell/file to be opened and displayed in the window.

% -----------------------------------------------------------------------------
% xic:wdump 021410
\index{Dump to File button}
The {\cb Dump To File} button in the {\cb iew} menu of sub-windows will
dump the contents of the window to a disk file.  When pressed, a file
name will be solicited, and the contents of the sub-window will be
dumped to the file.  The filename extension determines the file type: 
jpg, tiff, png, etc.  This provides a mechanism for obtaining
printable output from the {\cb Cross Section} views.  The dumped
bitmap will be the same size as the window.

This feature makes use of the {\vt imsave} system, which is also used
in the Image print driver (see \ref{hcformat}).

% -----------------------------------------------------------------------------
% xic:lshow 061608
\index{Show Location button}
When the {\cb Show Location} button in the sub-window {\cb View} menu
is active, and both the sub-window and the main window are in physical
mode and displaying the same cell, an outline box is drawn in the main
window around the area displayed in the sub-window.  This indicates the
position of the sub-window display, assuming that the sub-window is
showing a zoomed-in part of the display in the main window.

% -----------------------------------------------------------------------------
% xic:swap 061608
\index{Swap With Main button}
The {\cb Swap With Main} button in the sub-window {\cb View} menu will
swap the cells, display modes, and views between the sub-window and the
main window.  This has the effect of making the cell displayed in the
sub-window the current cell, allowing it to be modified.

% -----------------------------------------------------------------------------
% xic:load 102208
\index{Load New button}
The {\cb Load New} button in the sub-window {\cb View} menu will prompt
for a new cell or file to display in the window.  The command will
prompt the user for a file/cell name, in the manner of the {\cb Open}
command.  The given file/cell will be opened for display in the
sub-window.


% -----------------------------------------------------------------------------
% xic:peek 061608
\section{The {\cb Peek} Button: Show Layer Composition}
\index{Peek button}
\index{PeekSleepMsec variable}
The {\cb Peek} button in the {\cb View Menu} asks the user to define a
rectangular area with pointer clicks ar drag, and then redisplays the
area slowly so that underlying layers can be seen.  It also prints the
names of layers found (only physical objects are considered, not
labels).  The delay, which defaults to .4 second per existing layer,
can be reset with the {\et PeekSleepMsec} variable, which can be set
to the delay time in milliseconds. 

% -----------------------------------------------------------------------------
% xic:ldb3d 102114
\section{Three-Dimensional Layer Sequence Generator}
\label{ldb3d}
\index{layer sequence generator}

{\Xic} contains functionality to generate sequenced three-dimensional
layer stack representations of layout geometry.  This capability is
employed by the {\cb Cross Section} command in the {\cb View Menu}
which displays the layout in cross-section, and in the interface to
external capacitance and inductance extraction programs.

In order to use this functionality, the appropriate keywords must have
been applied to the layers.  Most of the setup parallels that required
for the extraction system.

Layers that participate must satisfy the following constraints.
\begin{itemize}
\item{The layer must {\bf not} have the {\et Symbolic} technology file
keyword given.}

\item{The layer must be visible in the layer table.}

\item{The layer must have a nonzero thickness applied with the
{\et Thickness} technology file keyword.}

\item{The layer must be a conductor or an insulator.  A conductor
layer has one of the following:
\begin{enumerate}
\item{The {\et Conductor} keyword is given, explicitly or implicitly
if one of the {\et Routing}, {\et GroundPlane}, {\et
GroundPlaneClear}, {\et Contact}, or their aliases is given.}

\item{Any of the {\et Rsh}, {\et Rho}, {\et Sigma}, or {\et Lambda}
keywords is given with a positive value.}
\end{enumerate}

An insulating layer has either the {\et Via} or {\et Dielectric}
keyword given, and also the {\et EpsRel} keyword given with a value of
1.0 or larger.}
\end{itemize}

The technology file keywords are normally applied in layer blocks of
the technology file, but can be applied from within {\Xic} as well. 
The {\cb Tech Parameter Editor} from the {\cb Edit Tech Params} button
in the {\cb Attributes Menu} allows editing of the associated
keywords.

The polarity of the layer will be obtained from the keywords applied
to the layer.  The layer will be considered dark field if any of the
{\et DarkField}, {\et Via}, or {\et GroundPlaneClear} keywords are
given to the layer.  In this database, each layer represents the true
presence of physical material which is the inverse of the normal
presentation if the layer is dark field.  For example, a via showing
as a small colored square in a normal layout window is actually a hole
in an otherwise continuous insulating film.  The layer represented in
this database will contain shapes representing the areas where the
physical film is present.  This can be seen in the {\cb Cross Section}
displays, where the dark field layers will be displayed with the
inverse polarity from normal drawing windows.

All layers can be set to planarize, or not.  This is an important
difference from the original interface, which always assumed
planarization.  If a layer is not planarizing, it conforms to the
underlying topology, which is translated exactly to the top surface of
the layer.  If the layer is planarizing, representing the layer
three-dimensionally is accomplished as follows.  Initially, the shapes
on each layer are decomposed into a non-overlapping collection of
trapezoids.

\begin{enumerate}
\item{Consider the non-planarized representation of the layer, which
is composed of trapezoids in the X-Y plane (parallel to the
substrate), each with an elevation (distance from the substrate in the
Z direction) and constant thickness.  Each trapezoid represents an
area of the layer material at constant elevation.  Sort through the
trapezoids, and find the bottom surface value with the highest
elevation.}

\item{From the maximum lower surface elevation, set the ``plane''
value for the layer.  Set the top elevation of all trapezoids to the
plane value plus the film thickness value.  All trapezoids will have
the same top elevation, but in general will have differing bottom
elevation.  The actual film thickness varies, but is equal to or
greater than the thickness value of the layer.}
\end{enumerate}

Thus, for a planarizing layer, the top surface of the layer will be a
plane at the highest point on the top surface of the layer below plus
the layer thickness.  The bottom surface of the layer will conform to
the top surface of the layer below by filling in where material of the
layer below is not present.

By default, all {\et Conductor} (and the rest of the keywords that
imply {\et Conductor}) and {\et Via} layers are planarizing.  This is
normal for a modern metal stack in a semiconductor process.  The {\et
Dielectric} layers, and conductor layers that don't have {\et
Conductor} applied implicitly or explicitly, but therefor must have
{\et Rho} or {\et Rsh} applied, are not planarizing by default.

However, if the {\et NoPlanarize} variable is set, or the global
technology file attribute of the same name is given, by default no
layers will be planarizing.

There is a technology file layer block keyword which provides
planarization control for each layer.  It overrides defaults and state
of the {\et NoPlanarize} variable.

\begin{description}
\item{{\vt Planarize} [{\vt y}|{\vt n}]}\\
This specifies whether or not a layer is ``planarizing''.  The {\et
Planarize} keyword can be applied to prevent planarization of layers
that are planarized by default, or to force planarizing of layers that
don't normally have this property.
\end{description}

The database is not intended for large collections of objects, and
processing time is near quadratic in the number of database
trapezoids.  By default, the total trapezoid count is limited to
10000.  Attempts to exceed the limit will fail, causing the command
using the database to also fail, with an appropriate error message.

The limit can be modified with the {\et Db3ZoidLimit} variable.  This
variable can be set to an integer 1000 or larger to reset the limit. 
If not set, the limit of 10000 applies.

%-----------------------------------
\subsection{Layer Sequencing}
     
By default, all conducting and insulating layers are assumed to stack
in layer table order, starting at the substrate.  We therefor ignore
the upper and lower conductor references in {\et Via} layers.  Any
layer order is acceptable, there is no constraint regarding adjacency
of layer types, but of course the layer table order must match the
physical order.

However, we allow for the case that {\et Via} layers are out of
sequence in the layer table.  They will be moved to the correct
position in the stacking order used by the interface.  Since layers
are rendered bottom-up in drawing windows, having a {\et Via} layer
positioned above the referenced top conductor in the layer table would
cause the via to be drawn on top of the metal, probably enhancing
visibility.  The correct sequence, of course, would place the {\et
Via} layer below the top referenced conductor.

The {\et LayerReorderMode} variable can be set to allow {\et Via}
layer repositioning.  The variable is set to an integer in the range
0--2.

\begin{description}
\item{0}\\
No repositioning is done, the same as if the variable is not set.

\item{1}\\
Consider the via references.  A layer can have multiple {\et Via}
keywords, each specifying a pair of conducting layers which are to be
connected through a hole in this layer.  For each pair, we identify
the bottom conductor by its position in the layer table relative to
the other referenced conductor.  For each of the bottom consuctors, we
find the one that is highest in layer table order, and move the {\et
Via} layer to just above this layer.

\item{2}\\
Similarly, we identify the lowest of the upper conductors, and move
the {\et Via} layer to just below this layer.
\end{description}

Once the layers are recognized and sequenced, a three-dimensional
representation of the layers found within a specified area is
constructed.  This representation is then available for such useful
things as displaying a cross section, or building up an input file for
a parameter extraction program.


% -----------------------------------------------------------------------------
% xic:csect 070314
\section{The {\cb Cross Section} Button: Show Cross Section}
\label{csect}
\index{Cross Section button}
The {\cb Cross Section} button in the {\cb View Menu} brings up a
special sub-window which displays a cross sectional (side) view of the
layers under an arbitrary line.  After pressing the command button,
the user is asked to define a line, which can be done by clicking
twice or dragging.  If the line covers any geometry (which may be
implied by dark field layers), a sub-window showing the cross
sectional view will appear.  The process can be repeated.  Pressing
the {\kb Esc} key will exit the command.

All geometry under the line will be shown, without regard to cell
hierarchy.

If the {\cb Constrain angles to 45 degree multiples} check box in the
{\cb Editing Setup} panel from the {\cb Edit Menu} is checked, the
angle is constrained to multiples of 45 degrees.  If not checked, the
angle is unconstrained, but snaps to multiples of 45 degrees when the
angle is close.  In either case, pressing the {\kb Ctrl} key removes
the constraint.

The endpoints initially do not snap to grid points.  The period (`.')
key toggles snapping to grid of the endpoints, when defined with the
mouse or other pointing device.

The endpoints are saved in persistent storage, and the previous cross
section can be repeated by pressing {\kb Enter} while the command is
active, even if the command terminated after the last cross section
was displayed.  One can experiment with different thicknesses or
planarizing behavior of the layers, and easily compare cross sections
from the same line, using this feature.  If the current cell has
changed, the stored endpoints will have no effect.

The display makes use of the three-dimensional layer sequencing
database described in \ref{ldb3d} to build up a three-dimensional
representation of the geometry along the line.  This requires that the
layers are appropriately set up in the technology file.  There are few
defaults, and this command will not work without proper setup.

The layer thickness shown can be set with the {\et CrossThick}
technology file keyword.  This can be applied in the physical layer
blocks of the technology file, or can be set or edited from the {\cb
Tech Parameter Editor}.

If {\et CrossThick} is not set, the displayed thickness will be taken
from the {\et Thickness} parameter, which must be set to a nonzero
value in any case.  This is the physical film thickness.  In cases
where this is too thick or thin for convient viewing in cross-section,
the {\et CrossThick} keyword can be set to provide an overriding
thickness used in the display.

The layer shown in the cross section is always true polarity, showing
where the material will exist on the substrate.  This is the inverse
of the normal drawing windows when the layer is dark field.  A via,
for example, which appears as a colored square in the main window,
should appear as a hole in cross section, since the painted area
actually represents lack of insulating material.  This is a "dark
field" layer.

By default, the display uses {\cb Auto Y-Scale} mode, where the total
displayed thickness of the layer stack is a little less than the
display window height.  This scaling is used for any magnification,
which of course has the usual effect in the horizontal direction.

The Y-scale can be manipulated by the panel brought up from the {\cb
Zoom} button in the {\cb View} menu of the cross-section display
window.  In this configuration, the {\cb Set Display Window} panel has
a set of controls for adjusting the Y-scale.  The {\cb Auto Y-Scale}
check box sets whether of not the automatic scaling is used.  This
corresponds to the {\et XSectNoAutoY} variable.

The {\cb Y-Scale} entry area allows a scale factor to be entered.  In
{\cb Auto Y-Scale} mode, this will change the displayed layer stack
height relative to the window height.  Otherwise, this will be a
constant linear scale factor applied in the Y direction.

Note that the grid lines will be shown taking into account the current
Y scale, and the cells will therefor not, in general, be square.  The
scaling is also accounted for in the gradations computed for rulers
that may be applied to the cross-section window.


% -----------------------------------------------------------------------------
% xic:ruler 012815
\section{The {\cb Rulers} Button: Create Rulers}
\index{Rulers button}
\index{show rulers}
The {\cb Rulers} button in the {\cb View Menu} provides a facility for
creating rulers.  Rulers, available in physical mode, are visible
calibrated gradations which indicate physical distance in microns. 
Rulers are often convenient for measuring distances, and in hard
copies to indicate size scale.

When the {\cb Rulers} button is on, rulers can be created by clicking
twice at the endpoints of the ruler, or by pressing and dragging,
where the ruler will extend from the press and release points.  The
ruler will only be visible in the window where the first button press
occurred, and only rulers in the main window will be visible in hard
copies.  Rulers can be created in any physical-mode window, including
cross-section and cell hierarchy digest displays.

The computed distance between endpoints of the ruler is printed in the
prompt area after a ruler is created.  This can be used to accurately
measure the distance between two points.

Rulers remain visible until another cell is edited, or until deleted. 
The rulers in effect for a certain cell are remembered, so that upon
returning to a previously edited cell, the rulers previously in effect
will be visible.

Presently, rulers exist only in memory, and are not saved to disk with
cell data.

Rulers can be deleted, while the {\cb Rulers} button is on, by
pressing the {\kb Delete} or {\kb Tab} (undo) keys, and are deleted in
reverse order of creation, but only in the window that has keyboard
focus..  Rulers associated with the current cell can be deleted at any
time with the {\cb !dr} command.

When a ruler is being created, the ghost-drawn vector which appears
when creating a ruler indicates the side which will have the
gradations.  The side with the gradations can be toggled by pressing
the `/' or `$\backslash$' keys.

If {\kb Shift} is pressed during completion of a ruler, the endpoint
will be the start point of a new ruler, and the calibration in the new
ruler will be an extension of that in the current ruler.  Thus rulers
can be ``chained'' around an object to measure the periphery.

If the {\cb Constrain angles to 45 degree multiples} check box in the
{\cb Editing Setup} panel from the {\cb Edit Menu} is checked, the
angle is constrained to multiples of 45 degrees.  If not checked, the
angle is unconstrained, but snaps to multiples of 45 degrees when the
angle is close.  In either case, pressing the {\kb Ctrl} key removes
the constraint.

By default, the tiny snap-box near the mouse pointer, where ruler
endpoints can be placed, is constrained to the current snap grid, as
is true elsewhere in {\Xic}.  The period ('.') key toggles
snapping to grid points while in the {\cb Rulers} command.  When not
snapping, the snap box follows the mouse pointer, allowing per-pixel
resolution.

Additionally, if the window shows a normal layout (not CHD or cross
section) the endpoints will by default snap to nearby edges and
vertices.  This is the same edge snapping as is controlled from the
{\cb Edge Snapping} group in the {\cb Snapping} page of the {\cb Grid
Setup} panel (pressing {\kb Ctrl-g} will produce this panel). 
However, the {\cb Rulers} command has its own defaults, which are
active while the {\cb Rulers} command is active.  The default is to
allow off-grid locations, and non-Manhattan edges, and to snap to both
edges and the path (central spine) of wires.  Any of these settings
can be changed, and edge snapping disabled, from the {\cb Grid Setup}
panel.  Changes will apply only in the {\cb Rulers} command.  Edge
snapping applies whether or not grid snapping is enabled.


% -----------------------------------------------------------------------------
% xic:info 100312
\section{The {\cb Info} Button: Display Information About Objects}
\index{Info button}
\index{object info}
\index{InfoInternal variable}
The {\cb Info} button in the {\cb View Menu} brings up a an {\cb Info}
window, which can display information about the current cell or any
visible object.  This command can also facilitate pushing the editing
context to specific locations within the hierarchy.

Alternatively, if any objects are selected when the command is given,
information about all selected objects can be dumped to a file.  When
the command is entered, if there are selected objects, the user is
prompted whether to dump the info to a file.  If `{\vt y}' is given to
the prompt, the user is asked for a file name, and is given the option
of viewing the file when the dump is complete.

If there is no prompt, or `{\vt n}' is given at the prompt, any
selected objects become deselected, and ``info mode'' becomes active. 
Objects and unexpanded subcells can be clicked on, and information
about an object will be displayed in the text window that will appear. 
The chosen object will be highlighted in the display.

Although the clicking/highlighting operation is superficially similar
to normal selection, in fact there are important differences. 
However, the layer selectability and object type selectability flags
for normal selections are observed.

\begin{enumerate}
\item{Any object or unexpanded subcell visible in the drawing window
can be chosen, at any depth in the cell hierarchy.  In normal
selection, only objects and subcells of the current cell can be
selected.  The ``selection'' mechanism tracks the expansion depth of
the display window, including peek mode if active (see
\ref{expansion}).  Any object that is visible, or any subcell that is
shown as unexpanded, can be chosen by clicking on the object in the
window.}

\item{Only one object or unexpanded subcell can be highlighted at a
time.  This is the item whose information is shown in the window.  To
highlight a different object at the same location, click multiple
times.  A different object will be highlighted on each click.}
\end{enumerate}

The information shown in the window will include the name of the cell
that contains the chosen object, and a ``back trace'' of containing
subcells in the hierarchy up to the current cell.  Thus, one can
easily determine the cell which ``owns'' an object in the display.

While an object is selected, pressing {\kb Enter} will toggle
selection of all or other objects in the cell containing the original
selected object.  The objects selected will respect the settings of
layer-specific selection mode, and object type selectability, as set
in the {\cb Selection Control Panel}.  In this state, the text in the
{\cb Info} window will describe the instance containing the selected
objects.  Pressing {\kb Enter} again will revert to the previous
state.  This is useful for determining which objects belong to a
particular cell instance.

When an object is highlighted, initiating the {\cb Push} command in
the {\cb Cell Menu} will push the editing context to the cell
containing the chosen object.  That is, the current cell becomes the
cell containing the object, in the context of the instance in the
chosen location.  The objects in this cell can then be edited.  The
{\cb Pop} command in the {\cb Cell Menu} can be used to pop the
editing context back up the hierarchy to the original current cell. 
This can be a useful way to navigate through a complex hierarchy,
while editing a layout.

If the {\kb Shift} key is held while the user clicks anywhere in a
drawing window showing the current cell, or if the click occurs
outside of the current cell bounding box, information about the
current cell is shown, rather than information about a chosen object.

If the user clicks in a sub-window that is displaying a cell that is
not the current cell, information about the cell will be shown.  It is
not possible to select objects in this case.

The information shown in the text window contains items such as the
object type and bounding box, as well as details specific to the type
of object.  For objects not in the current cell, coordinates are
usually shown relative to the object's containing cell, as well as
those reflected to the current cell.

By default, dimensions are given in microns.  If the variable {\et
InfoInternal} is set (with the {\cb !set} command) then dimensions are
given in internal units (usually 1000 units per micron).

The text window contains two buttons.  The {\cb Dismiss} button
removes the pop-up and exits info mode.  The {\cb Activate} button,
which is initially active, can be used to exit and reenter info mode,
while the pop-up remains visible.

When applied to polygons, the {\cb Info} command performs reentrancy
tests, and a message is added if the condition is found, i.e., if the
polygon can not be rendered unambiguously.

Similarly, when applied to wires, the wire is checked for certain
properties that might cause trouble.  See the description of the {\cb
!wirecheck} command in \ref{wirecheck} for a description of the flag
keywords that might appear in the info text.

The {\cb Info} pop-up is also made visible by the {\cb Info} button in
the {\cb Cells Listing} panel brought up by the {\cb Cells List}
button in the {\cb Cell Menu}.  When this button is used, previously
selected objects are ignored, and there is no provision for dumping to
a file.  The {\cb Info} window will provide information on cells
selected in the cells listing, or on objects selected in the drawing
windows.


% -----------------------------------------------------------------------------
% xic:alloc 061608
\section{The {\cb Allocation} Button: Show Memory Allocation}
\index{Allocation button}
\index{memory}
Pressing the {\cb Allocation} button in the {\cb View Menu} brings up
the {\cb Memory Monitor} pop-up.  This displays the number of cells in
memory, the total dynamic memory in use by the program, and system
limits on dynamic memory.  While visible, the pop-up is refreshed
every few seconds.

The maximum memory that can be used by the program before a fault
occurs is not well defined, and may be much less that the limits,
depending on what other programs are running, the actual size of the
swap space, and other factors.  The limits are either system defaults,
or values set with the {\vt limits}(1) shell command (Unix/Linux). 
The ``hard'' and ``soft'' values are those returned by the system
call, and have different interpretations under different Unix
versions.

