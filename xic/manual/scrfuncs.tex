% -----------------------------------------------------------------------------
% Xic Manual
% (C) Copyright 2009, Whiteley Research Inc., Sunnyvale CA
% $Id: scrfuncs.tex,v 1.157 2017/03/22 22:27:02 stevew Exp $
% -----------------------------------------------------------------------------

% -----------------------------------------------------------------------------
% scr:iffuncs 082408
\chapter{Interface Functions}

There is a growing library of user interface functions which control
various aspects of {\Xic} for use in scripts.

Functions that manipulate objects in the database use a coordinate
system based in microns (1 micron usually equals 1000 database units). 
All coordinates are real values.

There are two levels of run-time error reporting.  For serious errors,
a message is emitted to the controlling terminal, and the script
terminates.  Most interface functions will generate this type of error
only in response to bad arguments, meaning usually arguments of the
wrong type.  Less serious errors simply cause the function to return,
returning a value that indicates that the operation was unsuccessful. 
Many of the functions return 1 if successful, or 0 if not successful. 
In some cases where a string is normally returned, a null string
return indicates an error occurred.  It is up to the user to test the
return values for success or failure.

When the documentation specifies that a null string value is
acceptable as a function argument, the value zero can be passed
instead of a string variable.  The token {\vt NULL}, which is predefined
as 0, can be used equivalently.

The tables below list the collections of interface functions presently
available, by category and sub-category.  Most of these functions
return a value.  In the descriptions, if a value is returned, the
type, in parentheses, is indicated ahead of the function name.

\newcommand{\vr}{\tt\raggedright}

The first group of main module functions:

\begin{longtable}{|p{3.0in}|p{2.875in}|} \hline
\multicolumn{2}{|l|}{\kb Main Functions 1}\\ \hline

% 030115
\multicolumn{2}{|c|}{\kb Current Cell}\\ \hline
\vr Edit({\it name\/}, {\it symname\/}) & Edit cell\\ \hline
\vr OpenCell({\it name\/}, {\it symname\/}, {\it curcell}) & Read file into
  memory\\ \hline
\vr TouchCell({\it cellname\/}, {\it curcell}) & Create cell in memory\\ \hline
\vr Push({\it object\_handle\/}) & Make a subcell the current cell\\ \hline
\vr PushElement({\it object\_handle\/}, {\it xind\/}, {\it yind\/}) & Make an
  arrayed subcell element the current cell\\ \hline
\vr Pop() & Make parent cell the current cell\\ \hline
\vr NewCellName() & Return empty new cell name\\ \hline
\vr CurCellName() & Return current cell name\\ \hline
\vr TopCellName() & Return cell name at top of editing hierarchy\\ \hline
\vr FileName() & Return file name for current cell\\ \hline
\vr CurCellBB({\it array\/}) & Return current cell bounding box\\ \hline
\vr SetCellFlag({\it cellname\/}, {\it flagname\/}, {\it set\/}) & Set the
  state of a cell flag\\ \hline
\vr GetCellFlag({\it cellname\/}, {\it flagname\/}) & Get cell flag
  state\\ \hline
\vr Save({\it newname\/}) & Save to disk\\ \hline
\vr UpdateNative({\it dir\/}) & Save modified hierarchy cells as native\\
  \hline

% 022012
\multicolumn{2}{|c|}{\kb Cell Info}\\ \hline
\vr CellBB({\it cellname\/}, {\it array\/} [, {\it symbolic\/}]) &
  Obtain cell bounding box\\ \hline
\vr ListSubcells({\it cellname\/}, {\it depth\/}, {\it array\/},
  {\it incl\_top\/}) & List subcells in area to depth\\ \hline
\vr ListParents({\it cellname\/}) & List instantiating cells\\ \hline
\vr InitGen() & Return handle to subcell name list\\ \hline
\vr CellsHandle({\it cellname\/}, {\it depth\/}) & Return handle to subcell
  name list\\ \hline
\vr GenCells({\it handle\/}) & Return name from name list\\ \hline

% 101208
\multicolumn{2}{|c|}{\kb Database}\\ \hline
\vr Clear({\it cellname\/}) & Delete cells from memory\\ \hline
\vr ClearAll({\it clear\_tech\/}) & Delete all cells and reinitialize\\ \hline
\vr IsCellInMem({\it cellname\/}) & Check if cell is in memory\\ \hline
\vr IsFileInMem({\it filename\/}) & Check if cell from file is in memory\\
 \hline
\vr NumCellsInMem() & Count cells in memory\\ \hline
\vr ListCellsInMem({\it options\_str\/}) & List names of cells in
  memory\\ \hline
\vr ListTopCellsInMem() & List names of top-level cells in memory\\ \hline
\vr ListModCellsInMem() & List names of modified cells in memory\\ \hline
\vr ListTopFilesInMem() & List source files of top-level cells in memory\\
 \hline

% 100408
\multicolumn{2}{|c|}{\kb Symbol Tables}\\ \hline
\vr SetSymbolTable({\it tabname\/}) & Switch to new or existing symbol
  table\\ \hline
\vr ClearSymbolTable({\it destroy\/}) & Clear or destroy current symbol
  table\\ \hline
\vr CurSymbolTable() & Return the name of the current symbol table\\ \hline

% 100408
\multicolumn{2}{|c|}{\kb Display}\\ \hline
\vr Window({\it x\/}, {\it y\/}, {\it width\/}, {\it win\/}) & Set display
  window view\\ \hline
\vr GetWindow() & Return window containing pointer\\ \hline
\vr GetWindowView({\it win\/}, {\it array\/}) & Return window view area
  coordinates\\ \hline
\vr GetWindowMode({\it win\/}) & Return window display mode\\ \hline
\vr Expand({\it win\/}, {\it string\/}) & Set expansion status\\ \hline
\vr Display({\it display\_string\/}, {\it win\_id\/}, {\it l\/}, {\it b\/},
  {\it r\/}, {\it t\/}) & Exportable rendering service\\ \hline
\vr FreezeDisplay({\it freeze\/}) & Turn off/on graphics screen updates\\
  \hline
\vr Redraw({\it win\/}) & Redraw the window\\ \hline

% 100408
\multicolumn{2}{|c|}{\kb Exit}\\ \hline
\vr Exit() & Exit script\\ \hline
\vr Halt() & Exit script\\ \hline

% 120909
\multicolumn{2}{|c|}{\kb Annotation}\\ \hline
\vr AddMark({\it type\/}, {\it arguments\/} ...) & Show a user-specified mark\\
  \hline
\vr EraseMark({\it id\/}) & Erase a mark\\ \hline
\vr DumpMarks({\it filename\/}) & Dump current cell marks to file\\ \hline
\vr ReadMarks({\it filename\/}) & Read marks from file\\ \hline

% 092915
\multicolumn{2}{|c|}{\kb Ghost Rendering}\\ \hline
\vr PushGhost({\it array\/}, {\it numpts\/}) & Register ghost-drawn polygon\\
  \hline
\vr PushGhostBox({\it left\/}, {\it bottom\/}, {\it right\/}, {\it top\/}) &
  Register ghost-drawn box\\ \hline
\vr PushGhostH({\it object\_handle\/}, {\it all\/}) & Register ghost-drawn
   outlines\\ \hline
\vr PopGhost() & Unregister ghost-drawn figure\\ \hline
\vr ShowGhost({\it type\/}) & Show ghost-drawn figures\\ \hline

% 022713
\multicolumn{2}{|c|}{\kb Graphics}\\ \hline
\vr GRopen({\it display\/}, {\it window\/}) & Open a graphics context\\ \hline
\vr GRcheckError() & Return graphics error status\\ \hline
\vr GRcreatePixmap({\it handle\/}, {\it width\/}, {\it height\/}) & Return a
  new pixmap id\\ \hline
\vr GRdestroyPixmap({\it handle\/}, {\it pixmap\/}) & Free pixmap\\ \hline
\vr GRcopyDrawable({\it handle\/}, {\it dst\/}, {\it src\/}, {\it xs\/},
  {\it ys\/}, {\it ws\/}, {\it hs\/}, {\it x\/}, {\it y\/}) & Copy area
  between drawables\\ \hline
\vr GRdraw({\it handle\/}, {\it l\/}, {\it b\/}, {\it r\/}, {\it t\/}) &
  Render cell\\ \hline
\vr GRgetDrawableSize({\it handle\/}, {\it drawable\/}, {\it array\/}) & Return
  size of drawable\\ \hline
\vr GRresetDrawable({\it handle\/}, {\it drawable\/}) & Switch drawable in
  context\\ \hline
\vr GRclear({\it handle\/}) & Clear window\\ \hline
\vr GRpixel({\it handle\/}, {\it x\/}, {\it y\/}) & Draw pixel\\ \hline
\vr GRpixels({\it handle\/}, {\it array}, {\it num\/}) & Draw pixels\\ \hline
\vr GRline({\it handle\/}, {\it x1\/}, {\it y1\/}, {\it x2\/}, {\it y2\/}) &
  Draw line\\ \hline
\vr GRpolyLine({\it handle\/}, {\it array\/}, {\it num\/}) & Draw path\\ \hline
\vr GRlines({\it handle\/}, {\it array\/}, {\it num\/}) & Draw lines\\ \hline
\vr GRbox({\it handle\/}, {\it l\/}, {\it b\/}, {\it r\/}, {\it t\/}) & Draw
  box\\ \hline
\vr GRboxes({\it handle\/}, {\it array\/}, {\it num\/}) & Draw boxes\\ \hline
\vr GRarc({\it handle\/}, {\it x0\/}, {\it y0\/}, {\it rx\/}, {\it ry\/},
  {\it theta1\/}, {\it theta2\/}) & Draw arc\\ \hline
\vr GRpolygon({\it handle\/}, {\it array\/}, {\it num\/}) & Draw polygon\\
  \hline
\vr GRtext({\it handle\/}, {\it text\/}, {\it x\/}, {\it y\/},
  {\it flags\/}) & Draw text\\ \hline
\vr GRtextExtent({\it handle\/}, {\it text\/}, {\it array\/}) & Return text
  size\\ \hline
\vr GRdefineColor({\it handle\/}, {\it red\/}, {\it green\/}, {\it blue\/}) &
  Return color code\\ \hline
\vr GRsetBackground({\it handle\/}, {\it pixel\/}) & Set default background
  color\\ \hline
\vr GRsetWindowBackground({\it handle\/}, {\it pixel\/}) & Set window background
  color\\ \hline
\vr GRsetColor({\it handle\/}, {\it pixel\/}) & Set foreground color\\ \hline
\vr GRdefineLinestyle({\it handle\/}, {\it index\/}, {\it mask\/}) & Define a
  line style\\ \hline
\vr GRsetLinestyle({\it handle\/}, {\it index\/}) & Set current line style\\
  \hline
\vr GRdefineFillpattern({\it handle\/}, {\it index\/}, {\it nx\/}, {\it ny\/},
  {\it array\_string\/}) & Define a fill pattern\\ \hline
\vr GRsetFillpattern({\it handle\/}, {\it index\/}) & Set current fill pattern\\
  \hline
\vr GRupdate({\it handle\/}) & Update rendering\\ \hline
\vr GRsetMode({\it handle\/}, {\it mode\/}) & Set drawing mode\\ \hline

% 071110
\multicolumn{2}{|c|}{\kb Hard Copy}\\ \hline
\vr HClistDrivers() & Return list of available drivers\\ \hline
\vr HCsetDriver({\it driver\/}) & Set current driver\\ \hline
\vr HCgetDriver() & Return current driver name\\ \hline
\vr HCsetResol({\it resol\/}) & Set current driver resolution\\ \hline
\vr HCgetResol() & Return current driver resolution\\ \hline
\vr HCgetResols({\it array\/}) & Return available driver resolutions\\ \hline
\vr HCsetBestFit({\it best\_fit\/}) & Set ``best fit'' mode\\ \hline
\vr HCgetBestFit() & Return ``best fit'' mode\\ \hline
\vr HCsetLegend({\it legend\/}) & Set ``legend'' mode\\ \hline
\vr HCgetLegend() & Return ``legend'' mode\\ \hline
\vr HCsetLandscape({\it landscape\/}) & Set ``landscape'' mode\\ \hline
\vr HCgetLandscape() & Return ``landscape'' mode\\ \hline
\vr HCsetMetric({\it metric\/}) & Set ``metric'' mode\\ \hline
\vr HCgetMetric() & Return ``metric'' mode\\ \hline
\vr HCsetSize({\it x\/}, {\it y\/}, {\it w\/}, {\it h\/}) & Set rendering
  area\\ \hline
\vr HCgetSize({\it array\/}) & Return rendering area\\ \hline
\vr HCshowAxes({\it style\/}) & Set axes display style\\ \hline
\vr HCshowGrid({\it show\/}, {\it mode\/}) & Set grid displayed or not\\ \hline
\vr HCsetGridInterval({\it spacing\/}, {\it mode\/}) & Set grid spacing\\
  \hline
\vr HCsetGridStyle({\it linemod\/}, {\it mode\/}) & Set grid line style\\
  \hline
\vr HCsetGridCrossSize({\it xsize\/}, {\it mode\/}) & Set grid ``dot'' cross
  size\\ \hline
\vr HCsetGridOnTop({\it on\_top\/}, {\it mode\/}) & Draw grid above or below
 geometry\\ \hline
\vr HCdump({\it l\/}, {\it b\/}, {\it r\/}, {\it t\/}, {\it filename\/},
  {\it command\/}) & Generate output\\ \hline
\vr HCerrorString() & Retrun error message\\ \hline
\vr HClistPrinters() & List MS Windows printers\\ \hline
\vr HCmedia({\it index\/}) & Set MS Windows page size\\ \hline

% 011114
\multicolumn{2}{|c|}{\kb Keyboard}\\ \hline
\vr ReadMapfile({\it mapfile\/}) & Read a keyboard mapping file\\ \hline

% 100408
\multicolumn{2}{|c|}{\kb Libraries}\\ \hline
\vr OpenLibrary({\it path\_name\/}) & Open a library file\\ \hline
\vr CloseLibrary({\it path\_name\/}) & Close an open library\\ \hline

\ifoa
% 030416
\multicolumn{2}{|c|}{\kb OpenAccess}\\ \hline
\vr OaVersion() & Get OpenAccess version string\\ \hline
\vr OaIsLibrary({\it libname\/}) & Check if argument is a library\\ \hline
\vr OaListLibraries() & Return list of libraries\\ \hline
\vr OaListLibCells({\it libname\/}) & Return list of cells in library\\ \hline
\vr OaListCellViews({\it libname\/}, {\it cellname\/}) & Return list of
  views in cell\\ \hline
\vr OaIsLibOpen({\it libname\/}) & Check if library is open\\ \hline
\vr OaOpenLibrary({\it libname\/}) & Open an OpenAccess library\\ \hline
\vr OaCloseLibrary({\it libname\/}) & Close an open OpenAccess library\\ \hline
\vr OaIsOaCell({\it libname\/}, {\it open\_only\/}) & Check if cell can be
  resolved\\ \hline
\vr OaIsCellInLib({\it libname\/}, {\it cellname\/}) & Check if cell exists
  in library\\ \hline
\vr OaIsCellView({\it cellname\/}, {\it viewname\/}, {\it open\_only\/}) &
  Check if view exists in cell\\ \hline
\vr OaIsCellViewInLib({\it libname\/}, {\it cellname\/}, {\it viewname\/}) &
  Check if view of cell exists in cell\\ \hline
\vr OaCreateLibrary({\it libname\/}, {\it techlibname\/}) & Create new
  library\\ \hline
\vr OaBrandLibrary({\it libname\/}, {\it branded\/}) & Set or unset
  writability from {\Xic}\\ \hline
\vr OaIsLibBranded({\it libname\/}) & Check if library writable from
  {\Xic}\\ \hline
\vr OaDestroy({\it libname\/}, {\it cellname\/}, {\it viewname\/}) & Destroy
  library, cell, or view\\ \hline
\vr OaLoad({\it libname\/}, {\it cellname\/}) & Load cell into {\Xic}\\ \hline
\vr OaReset() & Clear table of cells already loaded\\ \hline
\vr OaSave({\it libname\/}, {\it allhier\/}) & Save current cell to
  OpenAccess\\ \hline
\vr OaAttachTech({\it libname\/}, {\it techlibname\/}) & Attach the
  technology from another library\\ \hline
\vr OaGetAttachedTech({\it libname\/}) & Return the name of attached
  library\\ \hline
\vr OaHasLocalTech({\it libname\/}) & Check if library has local tech
  database\\ \hline
\vr OaCreateLocalTech({\it libname\/}) & Create a local tech database in
  library\\ \hline
\vr OaDestroyTech({\it libname\/}, {\it unattach\_only\/}) & Destroy/remove
  technology object\\ \hline
\fi

% 100408
\multicolumn{2}{|c|}{\kb Mode}\\ \hline
\vr Mode({\it window\/}, {\it mode\/}) & Set physical or electrical mode\\ \hline
\vr CurMode({\it window\/}) & Return current mode\\ \hline

% 100408
\multicolumn{2}{|c|}{\kb Prompt Line}\\ \hline
\vr StuffText({\it string\/}) & Register text for future access\\ \hline
\vr TextCmd({\it string\/}) & Execute a prompt line command\\ \hline
\vr GetLastPrompt() & Return most recent prompt line message\\ \hline

% 021913
\multicolumn{2}{|c|}{\kb Scripts}\\ \hline
\vr ListFunctions() & Return list of library file functions\\ \hline
\vr Exec({\it script\/}) & Execute a script\\ \hline
\vr SetKey({\it password\/}) & Set the current password for script
  decryption\\ \hline
\vr HasPython() & Return true if Python is available\\ \hline
\vr RunPython({\it command\/}) & Run a Python script\\ \hline
\vr RunPythonModFunc({\it module\/}, {\it function} [, {\it arg} ...]) &
  Execute a Python module function\\ \hline
\vr ResetPython() & Reset the Python interpreter\\ \hline
\vr HasTcl() & Return true if Tcl is available\\ \hline
\vr HasTk() & Return true if Tcl and Tk are available\\ \hline
\vr RunTcl({\it command\/} [, {\it arg} ...]) & Run a Tcl/Tk script\\ \hline
\vr ResetTcl() & Reset the Tcl/Tk interpreter\\ \hline
\vr HasGlobalVariable({\it globvar\/}) & Test if global variable\\ \hline
\vr GetGlobalVariable({\it globvar\/}) & Return value of global variable\\
  \hline
\vr GetGlobalVariable({\it globvar\/}, {\it value\/}) & Set value of global
  variable\\ \hline

% 021913
\multicolumn{2}{|c|}{\kb Technology File}\\ \hline
\vr GetTechName() & Return technology name\\ \hline
\vr GetTechExt() & Return technology file extension\\ \hline
\vr SetTechExt({\it extension\/}) & Define effective technology file
  extension\\ \hline
\vr TechParseLine({\it line\/}) & Parse text in technology file format\\ \hline
\vr TechGetFkeyString({\it fkeynum\/}) & Return function key encoding
  string\\ \hline
\vr TechSetFkeyString({\it fkeynum\/}, {\it string\/}) & Set function key
  encoding\\ \hline

% 113009
\multicolumn{2}{|c|}{\kb Variables}\\ \hline
\vr Set({\it name\/}, {\it string\/}) & Set a variable\\ \hline
\vr Unset({\it name\/}) & Unset a variable\\ \hline
\vr PushSet({\it name\/}, {\it string\/}) & Set a variable, allow revert\\
  \hline
\vr PopSet({\it name\/}) & Revert {\vt PushSet}\\ \hline
\vr SetExpand({\it string\/}, {\it use\_env\/}) & Perform variable
  substitution\\ \hline
\vr Get({\it name\/}) & Return variable contents\\ \hline
\vr JoinLimits({\it flags\/}) & Set or remove join operation limits\\ \hline

% 100408
\multicolumn{2}{|c|}{\kb {\Xic} Version}\\ \hline
\vr VersionString() & Return current {\it Xic} version\\ \hline
\end{longtable}

The second group of main module functions:

\begin{longtable}{|p{3.0in}|p{2.875in}|} \hline
\multicolumn{2}{|l|}{\kb Main Functions 2}\\ \hline

% 100408
\multicolumn{2}{|c|}{\kb Arrays}\\ \hline
\vr ArrayDims({\it out\_array\/}, {\it array\/}) & Get array dimensions\\
  \hline
\vr ArrayDimension({\it out\_array\/}, {\it array\/}) & Get array dimensions\\
 \hline
\vr GetDims({\it array\/}, {\it out\_array\/}) & Get array dimensions\\ \hline
\vr DupArray({\it dest\_array\/}, {\it src\_array\/}) & Copy an array\\ \hline
\vr SortArray({\it array\/}, {\it size\/}, {\it descend\/}, {\it indices\/}) &
  Sort array elements\\ \hline

% 100408
\multicolumn{2}{|c|}{\kb Bitwise Logic}\\ \hline
\vr ShiftBits({\it bits\/}, {\it val\/}) & Shift bit field\\ \hline
\vr AndBits({\it bits1\/}, {\it bits2\/}) & AND operation\\ \hline
\vr OrBits({\it bits1\/}, {\it bits2\/}) & OR operation\\ \hline
\vr XorBits({\it bits1\/}, {\it bits2\/}) & XOR operation\\ \hline
\vr NotBits({\it bits\/}) & NOT operation\\ \hline

% 101609
\multicolumn{2}{|c|}{\kb Error Reporting}\\ \hline
\vr GetError() & Return error message\\ \hline
\vr AddError({\it string\/}) & Save error string\\ \hline
\vr GetLogNumber() & Return current message index\\ \hline
\vr GetLogMessage({\it message\_num\/}) & Return string for message index\\
 \hline
\vr AddLogMessage({\it string\/}, {\it error\/}) & Add message to log\\ \hline

% 100408
\multicolumn{2}{|c|}{\kb Generic Handle Functions}\\ \hline
\vr NumHandles() & Returns the number of active handles\\ \hline
\vr HandleContent({\it handle\/}) & Returns count of list items\\ \hline
\vr HandleTruncate({\it handle\/}, {\it count\/}) & Truncate a list of items\\
  \hline
\vr HandleNext({\it handle\/}) & Advance list to next item\\ \hline
\vr HandleDup({\it handle\/}) & Duplicate a handle and list\\ \hline
\vr HandleDupNitems({\it handle\/}, {\it count\/}) & Duplicate a handle and
  list, truncating list\\ \hline
\vr H({\it scalar\/}) & Create temporary handle from scalar\\ \hline
\vr HandleArray({\it handle\/}, {\it array\/}) & Write an array of handles to
  list elements\\ \hline
\vr HandleCat({\it handle1\/}, {\it handle2\/}) & Add {\it handle2} list to end
  of {\it handle1} list\\ \hline
\vr HandleReverse({\it handle\/}) & Reverse list order\\ \hline
\vr HandlePurgeList({\it handle1\/}, {\it handle2\/}) & Remove from second list
  items in first\\ \hline
\vr Close({\it handle\/}) & Close a handle\\ \hline
\vr CloseArray({\it array\/}, {\it size\/}) & Close an array of handles\\ \hline

% 100408
\multicolumn{2}{|c|}{\kb Memory Management}\\ \hline
\vr FreeArray({\it array\/}) & Free memory used by array\\ \hline
\vr CoreSize() & Return kilobytes used by program\\ \hline

% 100408
\multicolumn{2}{|c|}{\kb Script Variables}\\ \hline
\vr Defined({\it variable\/}) & Check if variable is defined\\ \hline
\vr TypeOf({\it variable\/}) & Return variable type\\ \hline

% 100408
\multicolumn{2}{|c|}{\kb Path Manipulation and Query}\\ \hline
\vr PathToEnd({\it path\_name\/}, {\it dir\/}) & Modify search path\\ \hline
\vr PathToFront({\it path\_name\/}, {\it dir\/}) & Modify search path\\ \hline
\vr InPath({\it path\_name\/}, {\it dir\/}) & Check if directory is in search
  path\\ \hline
\vr RemovePath({\it path\_name\/}, {\it dir\/}) & Remove directory from the
  search path\\ \hline

% 100408
\multicolumn{2}{|c|}{\kb Regular Expressions}\\ \hline
\vr RegCompile({\it regex\/}, {\it case\_insens\/}) & Compile regular
  expression\\ \hline
\vr RegCompare({\it regex\_handle\/}, {\it string\/}, {\it array\/}) & Regular
  expression evaluation\\ \hline
\vr RegError({\it regex\_handle\/}) & Return error string\\ \hline

% 100408
\multicolumn{2}{|c|}{\kb String List Handles}\\ \hline
\vr StringHandle({\it string\/}, {\it sepchars\/}) & Return handle to string
  tokens\\ \hline
\vr ListHandle({\it arglist\/}) & Return handle to string arguments\\ \hline
\vr ListContent({\it stringlist\_handle\/}) & Return referenced string\\ \hline
\vr ListReverse({\it stringlist\_handle\/}) & Reverse order of strings in
  list\\ \hline
\vr ListNext({\it stringlist\_handle\/}) & Return referenced string and advance
  to next\\ \hline
\vr ListAddFront({\it stringlist\_handle\/}, {\it string\/}) & Add string to
  list\\ \hline
\vr ListAddBack({\it stringlist\_handle\/}, {\it string\/}) & Add string to
  list\\ \hline
\vr ListAlphaSort({\it stringlist\_handle\/}) & Sort string list\\ \hline
\vr ListUnique({\it stringlist\_handle\/}) & Remove duplicates from list\\
  \hline
\vr ListFormatCols({\it stringlist\_handle\/}, {\it columns\/}) & Format
  strings into columns\\ \hline
\vr ListConcat({\it stringlist\_handle\/}, {\it sepchars\/}) & Create single
  string from list\\ \hline
\vr ListIncluded({\it stringlist\_handle\/}, {\it string\/}) & Check if
  string is in list\\ \hline

% 102114
\multicolumn{2}{|c|}{\kb String Manipulation and Conversion}\\ \hline
\vr Strcat({\it string1\/}, {\it string2\/}) & String concatenation\\ \hline
\vr Strcmp({\it string1\/}, {\it string2\/}) & String comparison\\ \hline
\vr Strncmp({\it string1\/}, {\it string2}, {\it n\/}) & String comparison,
  fixed length\\ \hline
\vr Strcasecmp({\it string1\/}, {\it string2\/}) & String comparison, case
  insensitive\\ \hline
\vr Strncasecmp({\it string1\/}, {\it string2}, {\it n\/}) & String comparison,
  case insensitive, fixed length\\ \hline
\vr Strdup({\it string\/}) & String copy\\ \hline
\vr Strtok({\it str\/}, {\it sep\/}) & String tokenization\\ \hline
\vr Strchr({\it string\/}, {\it char\/}) & Return pointer to first instance
  of character\\ \hline
\vr Strrchr({\it string\/}, {\it char\/}) & Return pointer to last instance
  of character\\ \hline
\vr Strstr({\it string\/}, {\it substring\/}) & Return pointer to first
  instance of substring\\ \hline
\vr Strpath({\it string\/}) & Return pointer to filename in path\\ \hline
\vr Strlen({\it string\/}) & Return length of string\\ \hline
\vr Sizeof({\it arg\/}) & Return string length or array size\\ \hline
\vr ToReal({\it string\/}) & Convert string to number\\ \hline
\vr ToString({\it real\/}) & Convert number to string\\ \hline
\vr ToStringA({\it real\/}, {\it digits\/}) & Convert number to string
  using SPICE notation\\ \hline
\vr ToFormat({\it format\/}, {\it arg\_list\/}) & Print variables according to
  format string\\ \hline
\vr ToChar({\it integer\/}) & Convert character constant to string
  representation\\ \hline

% 100408
\multicolumn{2}{|c|}{\kb Current Directory}\\ \hline
\vr Cwd({\it path\/}) & Set current directory\\ \hline
\vr Pwd() & Return current directory\\ \hline

% 020411
\multicolumn{2}{|c|}{\kb Date and Time}\\ \hline
\vr DateString() & Return the date/time\\ \hline
\vr Time() & Return system-encoded time\\ \hline
\vr MakeTime({\it array\/}, {\it gmt\/}) & Create system-encoded time
  from values\\ \hline
\vr TimeToString({\it time\/}, {\it gmt\/}) & Return string from
  system-encoded time\\ \hline
\vr TimeToVals({\it time\/}, {\it gmt\/}, {\it array\/}) & Parse
  system-encoded time\\ \hline
\vr MilliSec() & Return elapsed time in milliseconds\\ \hline
\vr StartTiming({\it array\/}) & Initialize resource timing\\ \hline
\vr StopTiming({\it array\/}) & Obtain resource times\\ \hline

% 102214
\multicolumn{2}{|c|}{\kb File System Interface}\\ \hline
\vr Glob({\it pattern\/}) & Perform global expansion\\ \hline
\vr Open({\it file\/}, {\it mode\/}) & Open a file for read/write\\ \hline
\vr Popen({\it command\/}, {\it mode\/}) & Open a process for read/write\\
  \hline
\vr Sopen({\it host\/}, {\it port\/}) & Open a socket for read/write\\ \hline
\vr ReadLine({\it maxlen}, {\it file\_handle\/}) & Read a line of text from a
  file\\ \hline
\vr ReadChar({\it file\_handle\/}) & Read a character from a file\\ \hline
\vr WriteLine({\it string\/}, {\it file\_handle\/}) & Write a line of text to a
  file\\ \hline
\vr WriteChar({\it c\/}, {\it file\_handle\/}) & Write a character to a file\\
  \hline
\vr TempFile({\it prefix\/}) & Create a temporary file name\\ \hline
\vr ListDirectory({\it path\/}, {\it filter\/}) & Return handle to list of
  file names\\ \hline
\vr MakeDir({\it path\/}) & Create directory tree\\ \hline
\vr FileStat({\it path\/}, {\it array\/}) & Get file/directory
  statistics\\ \hline
\vr DeleteFile({\it path\/}) & Destroy file or empty directory\\ \hline
\vr MoveFile({\it from\_path\/}, {\it to\_path\/}) & Move (rename) file\\
  \hline
\vr CopyFile({\it from\_path\/}, {\it to\_path\/}) & Copy file\\ \hline
\vr CreateBak({\it path\/}) & Move file to backup\\ \hline
\vr Md5Digest({\it path\/}) & Return file digest string\\ \hline

% 100408
\multicolumn{2}{|c|}{\kb Socket and {\Xic} Client/Server Interface}\\ \hline
\vr ReadData({\it size\/}, {\it skt\_handle\/}) & Read data from a
  socket\\ \hline
\vr ReadReply({\it retcode\/}, {\it skt\_handle\/}) & Read a message from the
  {\Xic} server\\ \hline
\vr ConvertReply({\it message\/}, {\it retcode\/}) & Parse {\Xic} server
  response\\ \hline
\vr WriteMsg({\it string\/}, {\it skt\_handle\/}) & Write a message to a
  socket\\ \hline

% 100408
\multicolumn{2}{|c|}{\kb System Command Interface}\\ \hline
\vr Shell({\it command\/}) & Execute a shell command\\ \hline
\vr System({\it command\/}) & Execute a shell command\\ \hline
\vr GetPID({\it parent\/}) & Return process ID\\ \hline

% 100408
\multicolumn{2}{|c|}{\kb Menu Buttons}\\ \hline
\vr SetButtonStatus({\it menu\/}, {\it button\/}, {\it set\/}) & Set button
  toggle status\\ \hline
\vr GetButtonStatus({\it menu\/}, {\it button\/}) & Return button toggle
  status\\ \hline
\vr PressButton({\it menu\/}, {\it button\/}) & Synthesize a button press\\
  \hline
\vr BtnDown({\it num\/}, {\it state\/}, {\it x\/}, {\it y\/}, {\it widget\/}) &
  Synthesize a button press\\ \hline
\vr BtnUp({\it num\/}, {\it state\/}, {\it x\/}, {\it y\/}, {\it widget\/}) &
  Synthesize a button release\\ \hline
\vr KeyDown({\it keysym\/}, {\it state\/}, {\it widget\/}) & Synthesize a key
  press\\ \hline
\vr KeyUp({\it keysym\/}, {\it state\/}, {\it widget\/}) & Synthesize a key
  release\\ \hline

% 110115
\multicolumn{2}{|c|}{\kb Mouse Input}\\ \hline
\vr Point({\it array\/}) & Wait for a mouse button press\\ \hline
\vr Selection() & Wait for key press, allow selections\\ \hline

% 020109
\multicolumn{2}{|c|}{\kb Graphical Input}\\ \hline
\vr PopUpInput({\it message\/}, {\it default\/}, {\it buttontext\/},
  {\it multiline\/}) & Pop up text input dialog\\ \hline
\vr PopUpAffirm({\it message\/}) & Pop up yes/no dialog\\ \hline
\vr PopUpNumeric({\it message\/}, {\it initval\/}, {\it minval\/},
  {\it maxval\/}, {\it delta\/}, {\it numdgt\/}) & Pop up numeric entry
  dialog\\ \hline

% 100408
\multicolumn{2}{|c|}{\kb Text Input}\\ \hline
\vr AskReal({\it prompt\/}, {\it default\/}) & Prompt for a number from prompt
  line\\ \hline
\vr AskString({\it prompt\/}, {\it default\/}) & Prompt for a string from prompt
  line\\ \hline
\vr AskConsoleReal({\it prompt\/}, {\it default\/}) & Prompt for a number from
  console\\ \hline
\vr AskConsoleString({\it prompt\/}, {\it default\/}) & Prompt for a string from
  console\\ \hline
\vr GetKey() & Wait for key press\\ \hline

% 100408
\multicolumn{2}{|c|}{\kb Text Output}\\ \hline
\vr SepString({\it string\/}, {\it repeat\/}) & Create separation or
  indentation string\\ \hline
\vr ShowPrompt({\it arg\_list\/}) & Show arguments on prompt line\\ \hline
\vr SetIndent({\it level\/}) & Set indentation level for printing\\ \hline
\vr SetPrintLimits({\it num\_array\_elts\/}, {\it num\_zoids\/}) & Limit number
  of array values and trapezoids printed\\ \hline
\vr Print({\it arg\_list\/}) & Print arguments to console window\\ \hline
\vr PrintLog({\it file\_handle\/}, {\it arg\_list\/}) & Print arguments to
  file\\ \hline
\vr PrintString({\it arg\_list\/}) & Print arguments to a string\\ \hline
\vr PrintStringEsc({\it arg\_list\/}) & Print arguments to a string\\ \hline
\vr Message({\it arg\_list\/}) & Print arguments to pop-up window\\ \hline
\vr ErrorMsg({\it arg\_list\/}) & Print arguments to pop-up error window\\
  \hline
\vr TextWindow({\it fname\/}, {\it readonly\/}) & Show file in text editor\\
  \hline
\end{longtable}

The third group of main module functions:

\begin{longtable}{|p{3.0in}|p{2.875in}|} \hline
\multicolumn{2}{|l|}{\kb Main Functions 3}\\ \hline

% 012815
\multicolumn{2}{|c|}{\kb Grid and Edge Snapping}\\ \hline
\vr SetMfgGrid({\it mfg\_grid\/}) & Set the manufacturing grid\\ \hline
\vr GetMfgGrid() & Return the manufacturing grid\\ \hline
\vr SetGrid({\it interval\/}, {\it snap\/}, {\it win\/}) & Set grid parameters
  for window\\ \hline
\vr GetGridInterval({\it win\/}) & Return fine grid spacing\\ \hline
\vr GetSnapInterval({\it win\/}) & Return the snap grid spacing\\ \hline
\vr GetGridSnap({\it win\/}) & Return grid snap number\\ \hline
\vr ClipToGrid({\it coord\/}, {\it win\/}) & Move coord to grid\\ \hline
\vr SetEdgeSnappingMode({\it win\/}, {\it mode\/}) & Set edge snapping
  scope for window\\ \hline
\vr SetEdgeOffGrid({\it win\/}, {\it off\_grid\/}) & Enable off-grid edge
  snapping in window\\ \hline
\vr SetEdgeNonManh({\it win\/}, {\it non\_manh\/}) & Enable non-Manhattan
  edge snapping in window\\ \hline
\vr SetEdgeWireEdge({\it win\/}, {\it wire\_edge\/}) & Snap to wire edges
  in window\\ \hline
\vr SetEdgeWirePath({\it win\/}, {\it wire\_path\/}) & Snap to wire path
  in window\\ \hline
\vr GetEdgeSnappingMode({\it win\/}) & Return edge snapping mode for
  windoiw\\ \hline
\vr GetEdgeOffGrid({\it win\/}) & Return off-grid edge snapping flag
  for window\\ \hline
\vr GetEdgeNonManh({\it win\/}) & Return non-Manhattan edge snapping
  flag for window\\ \hline
\vr GetEdgeWireEdge({\it win\/}) & Return wire edge snapping flag for
  window\\ \hline
\vr GetEdgeWirePath({\it win\/}) & Return wire path snapping flag for
  window\\ \hline
\vr SetRulerSnapToGrid({\it snap\/}) & Set ruler command grid snapping
  state\\ \hline
\vr SetRulerEdgeSnappingMode({\it mode\/}) & Set ruler command edge
  snapping mode\\ \hline
\vr SetRulerEdgeOffGrid({\it off\_grid\/}) & Set ruler command edge
  snapping off-grid state\\ \hline
\vr SetRulerEdgeNonManh({\it non\_manh\/}) & Set ruler command edge
  snapping non-Manhattan state\\ \hline
\vr SetRulerEdgeWireEdge({\it wire\_edge\/}) & Set ruler command edge
  snapping wire-edge state\\ \hline
\vr SetRulerEdgeWirePath({\it wire\_path\/}) & Set ruler command edge
  snapping wire-path state\\ \hline
\vr GetRulerSnapToGrid() & Return ruler command grid snapping state\\ \hline
\vr GetRulerEdgeSnappingMode() & Return ruler command edge snapping
  mode\\ \hline
\vr GetRulerEdgeOffGrid() & Return ruler command edge snapping off-grid
  state\\ \hline
\vr GetRulerEdgeNonManh() & Return ruler command edge snapping
  non-Manhattan state\\ \hline
\vr GetRulerEdgeWireEdge() & Return ruler command edge snapping wire-edge
  state\\ \hline
\vr GetRulerEdgeWirePath() & Return ruler command edge snapping wire-path
  state\\ \hline

% 012815
\multicolumn{2}{|c|}{\kb Grid Style}\\ \hline
\vr ShowGrid({\it on\/}, {\it win\/}) & Set grid visibility in window\\ \hline
\vr ShowAxes({\it style\/}, {\it win\/}) & Set axes style in window\\ \hline
\vr SetGridStyle({\it style\/}, {\it win\/}) & Set grid line style\\ \hline
\vr GetGridStyle({\it win\/}) & Return grid line style\\ \hline
\vr SetGridCrossSize({\it xsize\/}, {\it win\/}) & Set grid ``dot'' cross
  size\\ \hline
\vr GetGridCrossSize({\it win\/}) & Return grid ``dot'' cross size\\ \hline
\vr SetGridOnTop({\it ontop\/}, {\it win\/}) & Set grid on top of geometry\\
  \hline
\vr GetGridOnTop({\it win\/}) & Return grid top/bottom status\\ \hline
\vr SetGridCoarseMult({\it mult\/}, {\it win\/}) & Set coarse grid spacing
  multiple\\ \hline
\vr GetGridCoarseMult({\it win\/}) & Return coarse grid spacing
  multiple\\ \hline
\vr SaveGrid({\it regnum\/}, {\it win\/}) & Save grid parameters in
  register\\ \hline
\vr RecallGrid({\it regnum\/}, {\it win\/}) & Recall grid parameters from
  register\\ \hline

% 030115
\multicolumn{2}{|c|}{\kb Current Layer}\\ \hline
\vr GetCurLayer() & Return name of current layer\\ \hline
\vr GetCurLayerIndex() & Return index of current layer\\ \hline
\vr SetCurLayer({\it name\/}) & Set current layer, layer must exist\\ \hline
\vr SetCurLayerFast({\it name\/}) & As SetCurLayer, but no screen update\\
  \hline
\vr NewCurLayer({\it name\/}) & Set current layer, create if necessary\\ \hline
\vr GetCurLayerAlias() & Return alias name of current layer\\ \hline
\vr SetCurLayerAlias({\it alias\/}) & Set alias name of current layer\\ \hline
\vr GetCurLayerDescr() & Return description of current layer\\ \hline
\vr SetCurLayerDescr({\it descr\/}) & Set description of current layer\\ \hline

% 101412
\multicolumn{2}{|c|}{\kb Layer Table}\\ \hline

\vr LayersUsed() & Return number of layers in table\\ \hline
\vr AddLayer({\it name\/}, {\it index\/}) & Add a new layer\\ \hline
\vr RemoveLayer({\it stdlyr\/}) & Remove a layer\\ \hline
\vr RenameLayer({\it oldname\/}, {\it newname\/}) & Give a new name to a
  layer\\ \hline
\vr LayerHandle({\it down\/}) & Return a handle to a list of layer
  names\\ \hline
\vr GenLayers({\it stringlist\_handle\/}) & Return a layer name and
  advance list to next\\ \hline
\vr GetLayerPalette({\it regnum}) & Return list of palette layers\\ \hline
\vr SetLayerPalette({\it list\/}, {\it regnum\/}) & Save list of palette
  layers\\ \hline

% 032017
\multicolumn{2}{|c|}{\kb Layer Database}\\ \hline

\vr GetLayerNum({\it name\/}) & Return component layer number for
  name\\ \hline
\vr GetLayerName({\it num\/}) & Return component layer name for
  number\\ \hline
\vr IsPurposeDefined({\it name\/}) & Return true if name matches a
  purpose\\ \hline
\vr GetPurposeNum({\it name\/}) & Return purpose number for name\\ \hline
\vr GetPurposeName({\it num\/}) & Return purpose name for number\\ \hline

% 032017
\multicolumn{2}{|c|}{\kb Layers}\\ \hline

\vr GetLayerLayerNum({\it stdlyr\/}) & Return the component layer number
  for layer\\ \hline
\vr GetLayerPurposeNum({\it stdlyr\/}) & Return the purpose
  number for layer\\ \hline
\vr GetLayerAlias({\it stdlyr\/}) & Return the alias for layer\\ \hline
\vr SetLayerAlias({\it stdlyr\/}, {\it alias\/}) & Set the alias for
  layer\\ \hline
\vr GetLayerDescr({\it stdlyr\/}) & Return the description for layer\\ \hline
\vr SetLayerDescr({\it stdlyr\/}, {\it descr\/}) & Set the description for
  layer\\ \hline
\vr IsLayerDefined({\it lname\/}) & Return nonzero if layer exists with
  given name\\ \hline
\vr IsLayerVisible({\it stdlyr\/}) & Return true if layer is visible\\ \hline
\vr SetLayerVisible({\it stdlyr\/}, {\it visible\/}) & Set layer visibility
  flag\\ \hline
\vr IsLayerSelectable({\it stdlyr\/}) & Return true if layer is
  selectable\\ \hline
\vr SetLayerSelectable({\it stdlyr\/}, {\it selectable\/}) & Set layer
  selectability flag\\ \hline
\vr IsLayerSymbolic({\it stdlyr\/}) & Return true if layer is symbolic\\ \hline
\vr SetLayerSymbolic({\it stdlyr\/}, {\it symbolic\/}) & Set layer
  symbolic flag\\ \hline
\vr IsLayerNoMerge({\it stdlyr\/}) & Return true if layer has no\_merge
  set\\ \hline
\vr SetLayerNoMerge({\it stdlyr\/}, {\it nomerge\/}) & Set layer no\_merge
  flag\\ \hline
\vr GetLayerMinDimension({\it stdlyr\/}) & Return minimum dimension\\ \hline
\vr GetLayerWireWidth({\it stdlyr\/}) & Return default wire width\\ \hline
\vr AddLayerGdsOutMap({\it stdlyr\/}, {\it layer\_num\/}, {\it datatype\/}) &
  Add GDSII output layer mapping\\ \hline
\vr RemoveLayerGdsOutMap({\it stdlyr\/}, {\it layer\_num\/}, {\it datatype\/}) &
  Remove GDSII output layer mapping\\ \hline
\vr AddLayerGdsInMap({\it stdlyr\/}, {\it string\/}) & Add GDSII input layer
  mapping\\ \hline
\vr ClearLayerGdsInMap({\it stdlyr\/}) & Clear GDSII input layer
  mapping\\ \hline
\vr SetLayerNoDRCdatatype({\it stdlyr\/}, {\it datatype\/}) & Set GDSII
  NoDRC datatype\\ \hline

% 120114
\multicolumn{2}{|c|}{\kb Layers -- Extraction Support}\\ \hline
\vr SetLayerExKeyword({\it stdlyr\/}, {\it string\/}) & Set extraction
  keyword/value of layer\\ \hline
\vr SetCurLayerExKeyword({\it string\/}) & Set extraction keyword/value of
  current layer\\ \hline
\vr RemoveLayerExKeyword({\it stdlyr\/}, {\it keyword\/}) & Remove
  extraction keyword spec from layer\\ \hline
\vr RemoveCurLayerExKeyword({\it keyword\/}) & Remove extraction keyword spec
  from current layer\\ \hline
\vr IsLayerConductor({\it stdlyr\/}) & Return nonzero for Conductor\\ \hline
\vr IsLayerRouting({\it stdlyr\/}) & Return nonzero for Routing\\ \hline
\vr IsLayerGround({\it stdlyr\/}) & Return nonzero for GroundPlane\\ \hline
\vr IsLayerContact({\it stdlyr\/}) & Return nonzero for Contact\\ \hline
\vr IsLayerVia({\it stdlyr\/}) & Return nonzero for Via\\ \hline
\vr IsLayerDielectric({\it stdlyr\/}) & Return nonzero for Dielectric\\ \hline
\vr IsLayerDarkField({\it stdlyr\/}) & Return nonzero for DarkField\\ \hline
\vr GetLayerThickness({\it stdlyr\/}) & Return Thickness\\ \hline
\vr GetLayerRho({\it stdlyr\/}) & Return resistivity\\ \hline
\vr GetLayerResis({\it stdlyr\/}) & Return resistance per square\\ \hline
\vr GetLayerEps({\it stdlyr\/}) & Return dielectric constant\\ \hline
\vr GetLayerCap({\it stdlyr\/}) & Return capacitance per area\\ \hline
\vr GetLayerCapPerim({\it stdlyr\/}) & Return capacitance per length\\ \hline
\vr GetLayerLambda({\it stdlyr\/}) & Return penetration depth\\ \hline

% 100412
\multicolumn{2}{|c|}{\kb Selections}\\ \hline
\vr SetLayerSpecific({\it state\/}) & Restrict selectability to current
  layer\\ \hline
\vr SetLayerSearchUp({\it state\/}) & Set layer traversal direction\\ \hline
\vr SetSelectMode({\it ptr\_mode\/}, {\it area\_mode\/}, {\it sel\_mode\/}) &
  Set selection modes\\ \hline
\vr SetSelectTypes({\it string\/}) & Set selectable object types\\ \hline
\vr Select({\it left\/}, {\it bottom\/}, {\it right\/}, {\it top\/},
  {\it types\/}) & Select objects\\ \hline
\vr Deselect() & Deselect objects\\ \hline

% 100408
\multicolumn{2}{|c|}{\kb Pseudo-Flat Generator}\\ \hline
\vr FlatObjList({\it l\/}, {\it b\/}, {\it r\/}, {\it t\/}, {\it depth\/}) &
  Return list of object copies\\ \hline
\vr FlatObjGen({\it l\/}, {\it b\/}, {\it r\/}, {\it t\/}, {\it depth\/}) &
  Return handle to object generator\\ \hline
\vr FlatObjGenLayers({\it l\/}, {\it b\/}, {\it r\/}, {\it t\/}, {\it depth\/},
  {\it layers\/}) & Return handle to object generator\\ \hline
\vr FlatGenNext({\it handle\/}) & Return handle to next object copy\\ \hline
\vr FlatGenCount({\it handle\/}) & Count objects accessible by handle\\ \hline
\vr FlatOverlapList({\it object\_handle\/}, {\it touch\_ok\/}, {\it depth\/},
  {\it layers\/}) & Return handle to next object copy\\ \hline

% 100408
\multicolumn{2}{|c|}{\kb Geometry Measurement}\\ \hline
\vr Distance({\it x\/}, {\it y\/}, {\it x1\/}, {\it y1\/}) & Measure distance
  between points\\ \hline
\vr MinDistPointToSeg({\it x\/}, {\it y\/}, {\it x1\/}, {\it y1\/}, {\it x2\/},
  {\it y2\/}, {\it aret\/}) & Measure minimum distance between point and line
  segment\\ \hline
\vr MinDistPointToObj({\it x\/}, {\it y\/}, {\it object\_handle\/},
  {\it aret\/}) & Measure minimum distance between point and object\\ \hline
\vr MinDistSegToObj({\it x1\/}, {\it y1\/}, {\it x2\/}, {\it y2\/},
  {\it object\_handle\/}, {\it aret\/}) & Measure minimum distance between line
  segment and object\\ \hline
\vr MinDistObjToObj({\it object\_handle1\/}, {\it object\_handle2\/},
  {\it aret\/}) & Measure minimum distance between objects\\ \hline
\vr MaxDistPointToObj({\it x\/}, {\it y\/}, {\it object\_handle\/},
  {\it aret\/}) & Measure maximum distance from point to object\\ \hline
\vr MaxDistObjToObj({\it object\_handle1\/}, {\it object\_handle2\/},
  {\it aret\/}) & Measure maximum distance between objects\\ \hline
\vr Intersect({\it object\_handle1\/}, {\it object\_handle2\/},
  {\it touchok\/}) & Check if objects touch or overlap\\ \hline
\end{longtable}

Functions related to reading and writing of layout data:

\begin{longtable}{|p{3.0in}|p{2.875in}|} \hline
\multicolumn{2}{|l|}{\kb Layout File Input/Output Functions}\\ \hline

% 101412
\multicolumn{2}{|c|}{\kb Layer Conversion Aliasing}\\ \hline
\vr ReadLayerCvAliases({\it handle\_or\_filename\/}) & Read file containing
  layer conversion aliases\\ \hline
\vr DumpLayerCvAliases({\it handle\_or\_filename\/}) & Dump file containing
  layer conversion aliases\\ \hline
\vr ClearLayerCvAliases() & Delete all layer conversion aliases\\ \hline
\vr AddLayerCvAlias({\it lname\/}, {\it new\_lname\/}) & Add layer conversion
  alias to table\\ \hline
\vr RemoveLayerCvAlias({\it lname\/}) & Remove layer conversion alias from
  table\\ \hline
\vr GetLayerCvAlias({\it lname\/}) & Return conversion alias for layer
  name\\ \hline

% 100408
\multicolumn{2}{|c|}{\kb Cell Name Mapping}\\ \hline
\vr SetMapToLower({\it state\/}, {\it rw}) & Set cell name case conversion
  \\ \hline
\vr SetMapToUpper({\it state\/}, {\it rw}) & Set cell name case conversion
  \\ \hline

% 100408
\multicolumn{2}{|c|}{\kb Cell Table}\\ \hline
\vr CellTabAdd({\it cellname\/}, {\it expand}) & Add cell(s) to cell
  table\\ \hline
\vr CellTabCheck({\it cellname\/}) & Return true if name is in cell
  table\\ \hline
\vr CellTabRemove({\it cellname\/}) & Remove name from cell table\\ \hline
\vr CellTabList({\it cellname\/}) & List names in cell table\\ \hline
\vr CellTabClear({\it cellname\/}) & Clear all names from cell table\\ \hline

% 120110
\multicolumn{2}{|c|}{\kb Windowing and Flattening}\\ \hline
\vr SetConvertFlags({\it use\_window\/}, {\it clip\/}, {\it flatten\/},
  {\it ecf\_level\/}, {\it rw\/}) & Set modes for format translation or
  output\\ \hline
\vr SetConvertArea({\it l\/}, {\it b\/}, {\it r\/}, {\it t\/}, {\it rw}) & Set
  filter/clipping area for translation or output\\ \hline

% 100408
\multicolumn{2}{|c|}{\kb Scale Factor}\\ \hline
\vr SetConvertScale({\it scale\/}, {\it which}) & Set scale factor for
  import/export\\ \hline

% 100408
\multicolumn{2}{|c|}{\kb Export Flags}\\ \hline
\vr SetStripForExport({\it state\/}) & Set flag to write physical data only\\
  \hline
\vr SetSkipInvisLayers({\it code\/}) & Set code to skip invisible layers in
  output\\ \hline

% 100408
\multicolumn{2}{|c|}{\kb Import Flags}\\ \hline
\vr SetMergeInRead({\it state\/}) & Enable box and wire merging in input\\
  \hline

% 072710
\multicolumn{2}{|c|}{\kb Layout File Format Conversion}\\ \hline
\vr FromArchive({\it file\_or\_chd\/}, {\it destination\/}) &
  Translate archive file to another format\\ \hline
\vr FromTxt({\it text\_file\/}, {\it gds\_file\/}) & Create GDSII file from GDSII
  text\\ \hline
\vr FromNative({\it dir\_path\/}, {\it archive\_file\/}) &
  Translate native cell files to archive\\ \hline

% 071915
\multicolumn{2}{|c|}{\kb Export Layout File}\\ \hline
\vr SaveCellAsNative({\it cellname\/}, {\it directory\/}) & Write a native
  cell file in the directory\\ \hline
\vr Export({\it filepath\/}, {\it allcells\/}) & Write data to disk\\ \hline
\vr ToXIC({\it destination\_dir\/}) & Write {\it Xic} files\\ \hline
\vr ToCGX({\it cgx\_name\/}) & Write CGX file\\ \hline
\vr ToCIF({\it cif\_name\/}) & Write CIF file\\ \hline
\vr ToGDS({\it gds\_name\/}) & Write GDSII file\\ \hline
\vr ToGdsLibrary({\it gds\_name\/}, {\it cellname\_list\/}) & Write GDSII
  library file\\ \hline
\vr ToOASIS({\it oas\_name\/}) & Write OASIS file\\ \hline
\vr ToTxt({\it archive\_file\/}, {\it text\_file\/}, {\it cmdargs\/}) &
  Write text-mode GDSII/CGX/OASIS file\\ \hline

% 030113
\multicolumn{2}{|c|}{\kb Cell Hierarchy Digest}\\ \hline
\vr FileInfo({\it filename\/}, {\it handle\_or\_filename\/}, {\it flags\/}) &
  Obtain info about archive file\\ \hline
\vr OpebCellHierDigest({\it filename\/}, {\it info\_saved\/}) &
  Create new CHD\\ \hline
\vr WriteCellHierDigest({\it chd\_name\/}, {\it filename\/},
  {\it incl\_geom\/}, {\it no\_compr\/}) & Write CHD to
  file\\ \hline
\vr ReadCellHierDigest({\it filename\/}, {\it cgd\_type\/}) & Obtain CHD
  from file\\ \hline
\vr ChdList() & Return a list of CHD access names\\ \hline
\vr ChdChangeName({\it old\_chd\_name\/}, {\it new\_chd\_name\/})) & Change
  the access name of a CHD\\ \hline
\vr ChdIsValid({\it chd\_name\/}) & Return true if named CHD exists\\ \hline
\vr ChdDestroy({\it chd\_name\/}) & Destroy the CHD\\ \hline
\vr ChdInfo({\it chd\_name\/}, {\it handle\_or\_filename\/}, {\it flags\/}) &
  Obtain CHD information\\ \hline
\vr ChdFileName({\it chd\_name\/}) & Obtain archive file name\\ \hline
\vr ChdFileType({\it chd\_name\/}) & Obtain archive file format\\ \hline
\vr ChdTopCells({\it chd\_name\/}) & Obtain archive top-level cell
  names\\ \hline
\vr ChdListCells({\it chd\_name\/}, {\it cellname\/}, {\it mode\/},
  {\it all}) & Obtain list of cell names\\ \hline
\vr ChdLayers({\it chd\_name\/}) & Obtain layers used in archive\\ \hline
\vr ChdInfoMode({\it chd\_name\/}) & Return saved info mode\\ \hline
\vr ChdInfoLayers({\it chd\_name\/}, {\it cellname\/}) & Return saved layer
  info\\ \hline
\vr ChdInfoCells({\it chd\_name\/}) & Return saved cell names\\ \hline
\vr ChdInfoCounts({\it chd\_name\/}) & Return saved statistics\\ \hline
\vr ChdCellBB({\it chd\_name\/}, {\it cellname\/}, {\it array\/}) & Obtain
  cell bounding box\\ \hline
\vr ChdSetDefCellName({\it chd\_name\/}, {\it cellname\/}) &
  Configure default cell name\\ \hline
\vr ChdDefCellName({\it chd\_name\/}) & Obtain default cell name\\ \hline
\vr ChdLoadGeometry({\it chd\_name\/}) & Create and link to a new Cell
  Geometry Digest\\ \hline
\vr ChdLinkCgd({\it chd\_name\/}, {\it cgd\_name\/}) & Link or unlink a
  CGD to the CHD\\ \hline
\vr ChdGetGeomName({\it chd\_name\/}) & Return name of attached Cell
  Geometry Digest\\ \hline
\vr ChdClearGeometry({\it chd\_name\/}) & Unlink attached Cell Geometry
  Digest\\ \hline
\vr ChdSetSkipFlag({\it chd\_name\/}, {\it cellname\/}, {\it skip\/}) & Set
  or clear skip flag\\ \hline
\vr ChdClearSkipFlags({\it chd\_name\/}) & Clear all skip flags\\ \hline
\vr ChdCompare({\it chd\_name1\/}, {\it cname1\/}, {\it chd\_name2\/},
  {\it cname2\/}, {\it layer\_list\/}, {\it skip\_layers\/}, {\it maxdiffs\/},
  {\it obj\_types\/}, {\it geometric\/}, {\it array\/}) & Compare objects in
  cells\\ \hline
\vr ChdCompareFlat({\it chd\_name1\/}, {\it cname1\/}, {\it chd\_name2\/},
  {\it cname2\/}, {\it layer\_list\/}, {\it skip\_layers\/}, {\it maxdiffs\/},
  {\it area\/}, {\it coarse\_mult\/}, {\it find\_grid\/}, {\it array\/}) &
  Compare objects in flat cell hierarchies\\ \hline
\vr ChdEdit({\it chd\_name\/}, {\it scale\/}, {\it cellname\/}) & Open cell
  for editing\\ \hline
\vr ChdOpenFlat({\it chd\_name\/}, {\it scale\/}, {\it cellname\/},
  {\it array\/}, {\it clip\/}) &
  Read a flattened hierarchy into memory\\ \hline
\vr ChdSetFlatReadTransform({\it tfstring\/}, {\it x\/}, {\it y\/}) & Set
  a transform for flat reading\\ \hline
\vr ChdEstFlatMemoryUse({\it chd\_name\/}, {\it cellname\/}, {\it array\/},
  {\it counts\_array\/}) & Estimate memory required for flat read\\ \hline
\vr ChdWrite({\it chd\_name\/}, {\it scale\/}, {\it cellname\/}, {\it array\/},
  {\it clip\/}, {\it all\/}, {\it flatten\/}, {\it ecf\_level\/},
  {\it outfile\/}) & Write cells to file\\ \hline
\vr ChdWriteSplit({\it chd\_name\/}, {\it cellname\/}, {\it basename\/},
  {\it array\/}, {\it regions\_or\_gridsize\/},
  {\it numregions\_or\_bloatval\/}, {\it maxdepth\/},
  {\it scale\/}, {\it flags}) &
  Write to flat files\\ \hline
\vr ChdCreateReferenceCell({\it chd\_name\/}, {\it cellname\/}) &
  Create a reference cell in memory\\ \hline
\vr ChdLoadCell({\it chd\_name\/}, {\it cellname\/}) &
  Load cell in memory, reference subcells\\ \hline
\vr ChdIterateOverRegion({\it chd\_name\/}, {\it cellname\/}, {\it funcname\/},
  {\it array\/}, {\it coarse\_mult\/}, {\it fine\_grid\/},
  {\it bloat\_val\/}) & Iterate over grid, call callback function\\ \hline
\vr ChdWriteDensityMaps({\it chd\_name\/}, {\it cellname\/}, {\it array\/},
  {\it coarse\_mult\/}, {\it fine\_grid\/}, {\it bloat}, {\it save\/}) &
  Iterate over grid, compute density\\ \hline

% 012111
\multicolumn{2}{|c|}{\kb Cell Geometry Digest}\\ \hline
\vr OpenCellGeomDigest({\it idname\/}, {\it string\/}, {\it type\/}) &
  Create a new CGD\\ \hline
\vr NewCellGeomDigest() & Create a new empty CGD\\ \hline
\vr WriteCellGeomDigest({\it cgd\_name\/}, {\it filename\/}) & Write CGD
  to file\\ \hline
\vr CgdList() & Return a list of CGD access names\\ \hline
\vr CgdChangeName({\it old\_cgd\_name\/}, {\it new\_cgd\_name\/}) & Change
  the access name of a CGD\\ \hline
\vr CgdIsValid({\it cgd\_name\/}) & Return true if named CGD exists\\ \hline
\vr CgdDestroy({\it cgd\_name\/}) & Destroy the CGD\\ \hline
\vr CgdIsValidCell({\it cgd\_name\/}, {\it cellname\/}) &
  Return true if cell is found in CGD\\ \hline
\vr CgdIsValidLayer({\it cgd\_name\/}, {\it cellname\/}, {\it layername\/}) &
  Return true if cell containing layer is found in CGD\\ \hline
\vr CgdRemoveCell({\it cgd\_name\/}, {\it cellname\/}) &
  Remove a cell from the CGD\\ \hline
\vr CgdIsCellRemoved({\it cgd\_name\/}, {\it cellname\/}) &
  Return true if the cell was removed from the CGD\\ \hline
\vr CgdRemoveLayer({\it cgd\_name\/}, {\it cellname\/}, {\it layername\/}) &
  Remove layer data from a cell in the CGD\\ \hline
\vr CgdAddCells({\it cgd\_name\/}, {\it chd\_name\/}, {\it cells\_list\/}) &
  Add cells to the CGD\\ \hline
\vr CgdContents({\it cgd\_name\/}, {\it cellname\/}, {\it layername\/}) &
  List contents of CGD\\ \hline
\vr CgdOpenGeomStream({\it cgd\_name\/}, {\it cellname\/}, {\it layername\/}) &
  Open geometry stream from CGD\\ \hline
\vr GsReadObject({\it gs\_handle\/}) & Read geometry from a geometry
  stream\\ \hline
\vr GsDumpOasisText({\it gs\_handle\/}) & Dump OASIS ASCII text representation
  to console\\ \hline

% 120110
\multicolumn{2}{|l|}{\kb Assembly Stream}\\ \hline
\vr StreamOpen({\it outfile\/}) & Open an assembly stream\\ \hline
\vr StreamTopCell({\it stream\_handle\/}, {\it cellname\/}) & Define a
  top-level cell in the stream\\ \hline
\vr StreamSource({\it stream\_handle\/}, {\it file\_or\_chd\/},
  {\it scale\/}, {\it layer\_filter\/}, {\it name\_change\/}) &
  Register a source archive for streaming\\ \hline
\vr StreamInstance({\it stream\_handle\/}, {\it cellname\/}, {\it x\/},
  {\it y\/}, {\it my\/}, {\it rot\/}, {\it magn\/}, {\it scale\/},
  {\it no\_hier\/}, {\it ecf\_level\/}, {\it flatten\/}, {\it array\/},
  {\it clip\/}) & Add an instance conversion spec to a source\\ \hline
\vr StreamRun({\it stream\_handle\/}) & Initiate streaming to output\\ \hline
\end{longtable}

First group of functions for geometry editing

\begin{longtable}{|p{3.0in}|p{2.875in}|} \hline
\multicolumn{2}{|l|}{\kb Geometry Editing Functions 1}\\ \hline

% 032015
\multicolumn{2}{|c|}{\kb General Editing}\\ \hline
\vr ClearCell({\it undoable\/}, {\it layer\_list\/}) & Clear content of
  current cell\\ \hline
\vr Commit() & Finalize changes in database\\ \hline
\vr Undo() & Undo last operation\\ \hline
\vr Redo() & Redo last undone operation\\ \hline
\vr SelectLast({\it types\/}) & Select most recent new object\\ \hline

% 030115
\multicolumn{2}{|c|}{\kb Current Transform}\\ \hline
\vr SetTransform({\it angle\_or\_string\/}, {\it reflection\/},
  {\it magnification\/}) & Set current transform\\ \hline
\vr StoreTransform({\it register\/}) & Save current transform parameters\\
  \hline
\vr RecallTransform({\it register\/}) & Recall current transform parameters\\
  \hline
\vr GetTransformString() & Return acode string for the current transform\\
  \hline
\vr GetCurAngle() & Return current transform angle\\ \hline
\vr GetCurMX() & Return current transform mirror-x\\ \hline
\vr GetCurMY() & Return current transform mirror-y\\ \hline
\vr GetCurMagn() & Return current transform magnification\\ \hline
\vr UseTransform({\it enable\/}, {\it x\/}, {\it y\/}) & Enable use of current
  transform\\ \hline

% 032217
\multicolumn{2}{|c|}{\kb Derived Layers}\\ \hline
\vr AddDerivedLayer({\it lname\/}, {\it index\/}, {\it lexpr\/}) & Add a
  derived layer definition\\ \hline
\vr RemDerivedLayer({\it lname\/}) & Remove a derived layer definition\\
  \hline
\vr IsDerivedLayer({\it lname\/}) & True if name matches a derived layer
  definition\\ \hline
\vr GetDerivedLayerIndex({\it lname\/}) & Return the index of the specified
  derived layer\\ \hline
\vr GetDerivedLayerExpString({\it lname\/}) & Return the layer expression
  string of the specified derived layer\\ \hline
\vr GetDerivedLayerLexpr({\it lname\/}, {\it noexp\/}) & Return a layer
  expression object for the specified derived layer\\ \hline
\vr EvalDerivedLayers({\it list\/}, {\it array\/}) & Evaluate the list of
  derived layers in an area\\ \hline
\vr ClearDerivedLayers({\it list\/}) & Clear geometry of derived layers
  in list\\ \hline

% 070516
\multicolumn{2}{|c|}{\kb Object Management by Handles}\\ \hline
\vr ListElecInstances() & List electrical cell instances from current cell\\
 \hline
\vr ListPhysInstances() & List physical cell instances from current cell\\
 \hline
\vr SelectHandle() & Return handle to a list of selected objects\\ \hline
\vr SelectHandleTypes({\it types\/}) & Return handle to a list of selected
  objects of given types\\ \hline
\vr AreaHandle({\it l\/}, {\it b\/}, {\it r\/}, {\it t\/}, {\it types\/}) &
  Return handle to a list of objects in area\\ \hline
\vr ObjectHandleDup({\it object\_handle\/}, {\it types\/}) & Duplicate handle
  with given object types\\ \hline
\vr ObjectHandlePurge({\it object\_handle\/}, {\it types\/}) & Remove from list
  objects with given types\\ \hline
\vr ObjectNext({\it object\_handle\/}) & Advance list to next object\\ \hline
\vr MakeObjectCopy({\it numpts\/}, {\it array\/}) & Create a phony object
  copy\\ \hline
\vr ObjectString({\it object\_handle\/}) & Return CIF-like string for
  object\\ \hline
\vr ObjectCopyFromString({\it object\_handle\/}, {\it layer}) & Return new
  object from CIF-like string\\ \hline
\vr FilterObjects({\it object\_list\/}, {\it template\_list\/}, {\it all\/},
  {\it touchok\/}, {\it remove\/}) & Select objects via template\\ \hline
\vr CheckObjectsConnected({\it object\_handle\/}) & Return 1 if objects in
  list form one group\\ \hline
\vr CheckForHoles({\it object\_handle\/}, {\it all\/}) & Return 1 if object(s)
  have ``holes''\\ \hline
\vr FilterObjectsA({\it object\_list\/}, {\it array\/}, {\it array\_size\/},
  {\it touchok\/}, {\it remove\/}) & Select objects via given polygon\\ \hline
\vr BloatObjects({\it object\_handle\/}, {\it all\/}, {\it dimen\/},
  {\it lname\/}, {\it mode\/}) & Create list of bloated objects\\ \hline
\vr EdgeObjects({\it object\_handle\/}, {\it all\/}, {\it dimen\/},
  {\it lname\/}, {\it mode\/}) & Create list of edge ``wire'' polygons\\ \hline
\vr ManhattanizeObjects({\it object\_handle\/}, {\it all\/}, {\it dimen\/},
  {\it lname\/}, {\it mode\/}) & Create list of Manhattanized objects\\ \hline
\vr GroupObjects({\it object\_handle\/}, {\it array\/}) & Create connected
 groups of objects\\ \hline
\vr JoinObjects({\it object\_handle\/}, {\it lname\/}) & Join touching objects
  in a list\\ \hline
\vr SplitObjects({\it object\_handle\/}, {\it all\/}, {\it lname\/},
  {\it vert\/}) & Split into trapezoids objects in a list\\ \hline
\vr DeleteObjects({\it object\_handle\/}, {\it all\/}) & Delete objects\\
  \hline
\vr SelectObjects({\it object\_handle\/}, {\it all\/}) & Select objects\\
  \hline
\vr DeselectObjects({\it object\_handle\/}, {\it all\/}) & Deselect objects\\
  \hline
\vr MoveObjects({\it object\_handle\/}, {\it all\/}, {\it refx\/},
  {\it refy\/}, {\it x\/}, {\it y\/}) & Move object(s)\\ \hline
\vr MoveObjectsToLayer({\it object\_handle\/}, {\it all\/}, {\it refx\/},
  {\it refy\/}, {\it x\/}, {\it y\/}, {\it oldlayer\/}, {\it newlayer\/}) &
  Move object(s) with layer change\\ \hline
\vr CopyObjects({\it object\_handle\/}, {\it all\/}, {\it refx\/},
  {\it refy\/}, {\it x\/}, {\it y\/}, {\it repcnt\/}) & Copy object(s)\\ \hline
\vr CopyObjectsToLayer({\it object\_handle\/}, {\it all\/}, {\it refx\/},
  {\it refy\/}, {\it x\/}, {\it y\/}, {\it oldlayer\/}, {\it newlayer\/},
  {\it repcnt\/}) & Copy object(s) with layer change\\ \hline
\vr CopyObjectsH({\it object\_handle\/}, {\it all\/}, {\it refx\/},
  {\it refy\/}, {\it x\/}, {\it y\/}, {\it oldlayer\/}, {\it newlayer\/},
  {\it todb\/}) & Copy object(s) to handle\\ \hline
\vr GetObjectType({\it object\_handle\/}) & Return the object's type code\\
  \hline
\vr GetObjectID({\it object\_handle\/}) & Return the object's id number\\
  \hline
\vr GetObjectArea({\it object\_handle\/}) & Return the object's area in 
  square microns\\ \hline
\vr GetObjectPerim({\it object\_handle\/}) & Return the object's perimeter
  in microns\\ \hline
\vr GetObjectCentroid({\it object\_handle\/}, {\it array\/}) & Compute the
  object's centroid point\\ \hline
\vr GetObjectBB({\it object\_handle\/}, {\it array\/}) & Return the object's
  bounding box\\ \hline
\vr SetObjectBB({\it object\_handle\/}, {\it array\/}) & Set the object's
  bounding box, scale object\\ \hline
\vr GetObjectListBB({\it object\_handle\/}, {\it array\/}) & Return the
  bounding box of all objects in list\\ \hline
\vr GetObjectXY({\it object\_handle\/}, {\it array\/}) & Return the object's
  reference point\\ \hline
\vr SetObjectXY({\it object\_handle\/}, {\it x\/}, {\it y\/}) & Set the
  object's reference point\\ \hline
\vr GetObjectLayer({\it object\_handle\/}) & Return the object's layer name\\
  \hline
\vr SetObjectLayer({\it object\_handle\/}, {\it layername\/}) & Set the object's
  layer\\ \hline
\vr GetObjectFlags({\it object\_handle\/}) & Return the object's flags\\ \hline
\vr SetObjectNoDrcFlag({\it object\_handle\/}, {\it value\/}) & Set or unset
 the {\vt NoDRC} object flag\\ \hline
\vr SetObjectMark1Flag({\it object\_handle\/}, {\it value\/}) & Set or unset
 the {\vt Mark1} object flag\\ \hline
\vr SetObjectMark2Flag({\it object\_handle\/}, {\it value\/}) & Set or unset
 the {\vt Mark2} object flag\\ \hline
\vr GetObjectState({\it object\_handle\/}) & Return the object's state\\ \hline
\vr GetObjectGroup({\it object\_handle\/}) & Return the object's conductor
  group number\\ \hline
\vr SetObjectGroup({\it object\_handle\/}, {\it group\_num\/}) & Set the
  object's conductor group number\\ \hline
\vr GetObjectCoords({\it object\_handle\/}, {\it array\/}) & Return the object's
  coordinates\\ \hline
\vr SetObjectCoords({\it object\_handle\/}, {\it array}, {\it size\/}) & Set the
  object's coordinates\\ \hline
\vr GetObjectMagn({\it object\_handle\/}) & Return the magnification of a
  subcell\\ \hline
\vr SetObjectMagn({\it object\_handle\/}, {\it magn\/}) & Set object's
  magnification, rescale object\\ \hline
\vr GetWireWidth({\it object\_handle\/}) & Return width of wire\\ \hline
\vr SetWireWidth({\it object\_handle\/}, {\it width\/}) & Set width of wire\\
  \hline
\vr GetWireStyle({\it object\_handle\/}) & Return wire end style\\ \hline
\vr SetWireStyle({\it object\_handle\/}, {\it code\/}) & Set wire end style\\
  \hline
\vr SetWireToPoly({\it object\_handle\/}) & Convert wire to polygon\\ \hline
\vr GetWirePoly({\it object\_handle\/}, {\it array\/}) & Return wire bounding
  polygon\\ \hline
\vr GetLabelText({\it object\_handle\/}) & Return text of label\\ \hline
\vr SetLabelText({\it object\_handle\/}, {\it text\/}) & Set text in label\\
  \hline
\vr GetLabelFlags({\it object\_handle\/}) & Return flags for label\\
  \hline
\vr SetLabelFlags({\it object\_handle\/}, {\it flags\/}) & Set flags for
  label\\ \hline
\vr GetInstanceArray({\it object\_handle\/}, {\it array\/}) & Return instance
  array parameters\\ \hline
\vr SetInstanceArray({\it object\_handle\/}, {\it array\/}) & Set instance array
  parameters, resize array\\ \hline
\vr GetInstanceXform({\it object\_handle\/}) & Return instance transformation
  string\\ \hline
\vr GetInstanceXformA({\it object\_handle\/}, {\it array\/}) & Return instance
  transformation in array\\ \hline
\vr SetInstanceXform({\it object\_handle\/}, {\it transform\/}) & Set instance
  transformation from string\\ \hline
\vr SetInstanceXformA({\it object\_handle\/}, {\it array\/}) & Set instance
  transformation from array\\ \hline
\vr GetInstanceMaster({\it object\_handle\/}) & Return name of instance
  master cell\\ \hline
\vr SetInstanceMaster({\it object\_handle\/}, {\it newname\/}) & Set instance
  master, replace instance\\ \hline
\vr GetInstanceName({\it object\_handle\/}) & Return name of instance\\ \hline
\vr SetInstanceName({\it object\_handle\/}, {\it newname\/}) & Set instance
  name property\\ \hline
\vr GetInstanceAltName({\it object\_handle\/}) & Return alternate name of
  instance\\ \hline
\vr GetInstanceType({\it object\_handle\/}) & Return instance type code\\
  \hline
\vr GetInstanceIdNum({\it object\_handle\/}) & Return instance id number\\
  \hline
\vr GetInstanceAltIdNum({\it object\_handle\/}) & Return instance alternate
  id number\\ \hline

\end{longtable}

Second group of functions for geometry editing

\begin{longtable}{|p{3.0in}|p{2.875in}|} \hline
\multicolumn{2}{|l|}{\kb Geometry Editing Functions 2}\\ \hline

% 032115
\multicolumn{2}{|c|}{\kb Cells, PCells, Vias, and Instance Placement}\\ \hline
\vr CheckPCellParam({\it library\/}, {\it cell\/}, {\it view\/},
  {\it pname\/}, {\it value\/}) & Validate a parameter value\\ \hline
\vr CheckPCellParams({\it library\/}, {\it cell\/}, {\it view\/},
  {\it params\/}) & Validate a parameter list\\ \hline
\vr CreateCell({\it cellname\/}, [{\it orig\_x\/}, {\it orig\_y\/}]) &
  Create new cell from selected objects\\ \hline
\vr CopyCell({\it name\/}, {\it newname\/}) & Copy a cell\\ \hline
\vr RenameCell({\it oldname\/}, {\it newname\/}) & Globally rename cell in
  memory, fix references\\ \hline
\vr DeleteEmpties({\it recurse\/}) & Delete empty cells\\ \hline
\vr Place({\it cellname\/}, {\it x\/}, {\it y\/} [, {\it refpt\/},
  {\it array\/}, {\it smash\/}, {\it usegui\/}, {\it tfstring\/}]) & Place
  an instance\\ \hline
\vr PlaceH({\it cellname\/}, {\it x\/}, {\it y\/} [, {\it refpt\/},
  {\it array\/}, {\it smash\/}, {\it usegui\/}, {\it tfstring\/}]) & Place
  an instance, return handle\\ \hline
\vr PlaceSetArrayParams({\it nx\/}, {\it ny\/}, {\it dx\/}, {\it dy\/}) &
  Set instrance arraying parameters\\ \hline
\vr PlaceSetPCellParams({\it library\/}, {\it cell\/}, {\it view\/},
  {\it params\/}) & Set pcell parameter string\\ \hline
\vr Replace({\it cellname\/}, {\it add\_xform\/}, {\it array\/}) & Replace
  an instance\\ \hline
\vr OpenViaSubMaster({\it vianame\/}, {\it defnstr\/}) & Define a standard
  via variant\\ \hline

% 100508
\multicolumn{2}{|c|}{\kb Clipping Functions}\\ \hline
\vr ClipAround({\it object\_handle1\/}, {\it all1\/}, {\it object\_handle2\/},
  {\it all2\/}) & Clip object around other objects\\ \hline
\vr ClipAroundCopy({\it object\_handle1\/}, {\it all1\/},
  {\it object\_handle2\/}, {\it all2\/}, {\it lname\/}) & Clip objects
  around other objects, return copies\\ \hline
\vr ClipTo({\it object\_handle1\/}, {\it all1\/}, {\it object\_handle2\/},
  {\it all2\/}) & Clip objects to other objects\\ \hline
\vr ClipToCopy({\it object\_handle1\/}, {\it all1\/}, {\it object\_handle2\/},
  {\it all2\/}, {\it lname\/}) & Clip objects to other objects, return
  copies\\ \hline
\vr ClipObjects({\it object\_handle\/}, {\it merge\/}) & Clip object list so
  no overlap\\ \hline
\vr ClipIntersectCopy({\it object\_handle1\/}, {\it all1\/},
  {\it object\_handle2\/}, {\it all2\/}, {\it lname\/}) &
  Exclusive-or objects or lists\\ \hline

% 012815
\multicolumn{2}{|c|}{\kb Other Object Management Functions}\\ \hline
\vr ChangeLayer() & Change layer of selected objects\\ \hline
\vr Bloat({\it dimen\/}, {\it mode\/}) & Bloat selected objects\\ \hline
\vr Manhattanize({\it dimen\/}, {\it mode\/}) & Manhattanize selected
  objects\\ \hline
\vr Join() & Join selected objects\\ \hline
\vr Decompose({\it vert\/}) & Convert selected objects to trapezoids\\ \hline
\vr Box({\it left\/}, {\it bottom\/}, {\it right\/}, {\it top\/}) & Create a
  box\\ \hline
\vr BoxH({\it left\/}, {\it bottom\/}, {\it right\/}, {\it top\/}) & Create a
  box, return handle\\ \hline
\vr Polygon({\it num\/}, {\it arraypts\/}) & Create a polygon\\ \hline
\vr PolygonH({\it num\/}, {\it arraypts\/}) & Create a polygon, return
  handle\\ \hline
\vr Arc({\it x\/}, {\it y\/}, {\it rad1X\/}, {\it rad1Y\/}, {\it rad2X\/},
  {\it rad2Y\/}, {\it ang\_start\/}, {\it ang\_end\/}) & Create an arc
  polygon\\ \hline
\vr ArcH({\it x\/}, {\it y\/}, {\it rad1X\/}, {\it rad1Y\/}, {\it rad2X\/},
  {\it rad2Y\/}, {\it ang\_start\/}, {\it ang\_end\/}) & Create an arc
  polygon, return handle\\ \hline
\vr Round({\it x\/}, {\it y\/}, {\it rad\/}) & Create a disk polygon\\ \hline
\vr RoundH({\it x\/}, {\it y\/}, {\it rad\/}) & Create a disk polygon,
  return handle\\ \hline
\vr HalfRound({\it x\/}, {\it y\/}, {\it rad\/}, {\it dir\/}) & Create a
  half-disk polygon\\ \hline
\vr HalfRoundH({\it x\/}, {\it y\/}, {\it rad\/}, {\it dir\/}) & Create a
  half-disk polygon, return handle\\ \hline
\vr Sides({\it numsides\/}) & Set the number of sides used for round objects\\
  \hline
\vr Wire({\it width\/}, {\it num\/}, {\it arraypts\/}, {\it end\_style\/}) &
  Create a wire\\ \hline
\vr WireH({\it width\/}, {\it num\/}, {\it arraypts\/}, {\it end\_style\/}) &
  Create a wire, return handle\\ \hline
\vr Label({\it text\/}, {\it x\/}, {\it y\/} [, {\it width\/}, {\it height\/},
  {\it flags\/}]) & Create a label\\ \hline
\vr LabelH({\it text\/}, {\it x\/}, {\it y\/} [, {\it width\/}, {\it height\/},
  {\it flags\/}]) & Create a label, return handle\\ \hline
\vr Logo({\it string\/}, {\it x\/}, {\it y\/} [, {\it width\/},
  {\it height\/}]) & Create physical text\\ \hline
\vr Justify({\it hj\/}, {\it vj\/}) & Set default text justification\\ \hline
\vr Delete() & Delete selected objects\\ \hline
\vr Erase({\it left\/}, {\it bottom\/}, {\it right\/}, {\it top\/}) & Erase
  objects in area\\ \hline
\vr EraseUnder() & Erase overlap with selected objects\\ \hline
\vr Yank({\it left\/}, {\it bottom\/}, {\it right\/}, {\it top\/}) & Grab
  geometry into buffer\\ \hline
\vr Put({\it x\/}, {\it y\/}, {\it bufnum\/}) & Place stored geometry\\ \hline
\vr Xor({\it left\/}, {\it bottom\/}, {\it right\/}, {\it top\/}) &
  Exclusive-or geometry in area\\ \hline
\vr Copy({\it fromx\/}, {\it fromy\/}, {\it tox\/}, {\it toy\/},
  {\it repcnt\/}) & Copy selected objects\\ \hline
\vr CopyToLayer({\it fromx\/}, {\it fromy\/}, {\it tox\/}, {\it toy\/},
  {\it oldlayer\/}, {\it newlayer\/}, {\it repcnt\/}) & Copy selected
  objects and change layer\\ \hline
\vr Move({\it fromx\/}, {\it fromy\/}, {\it tox\/}, {\it toy\/}) & Move selected
  objects\\ \hline
\vr MoveToLayer({\it fromx\/}, {\it fromy\/}, {\it tox\/}, {\it toy\/},
  {\it oldlayer\/}, {\it newlayer\/}) &
  Move selected objects and change layer\\ \hline
\vr Rotate({\it x\/}, {\it y\/}, {\it ang\/}, {\it remove\/}) & Rotate
  selected objects\\ \hline
\vr RotateToLayer({\it x\/}, {\it y\/}, {\it ang\/}, {\it oldlayer\/},
  {\it newlayer\/}, {\it remove\/}) & Rotate selected objects and change
  layer\\ \hline
\vr Split({\it x\/}, {\it y\/}, {\it flag\/}, {\it orient\/}) & Divide selected
  objects\\ \hline
\vr Flatten({\it depth\/}, {\it use\_merge\/}, {\it fast\_mode\/}) & Flatten
  hierarchy\\ \hline
\vr Layer({\it string\/}, {\it mode\/}, {\it depth\/}, {\it recurse\/},
  {\it noclear\/}, {\it use\_merge\/}, {\it fast\_mode\/}) &
  Apply geometric manipulations\\ \hline

% 030215
\multicolumn{2}{|c|}{\kb Property Management}\\ \hline
\vr PrpHandle({\it object\_handle\/}) & Return handle to a list of the
  object's properties\\ \hline
\vr GetPrpHandle({\it number\/}) & Return a handle to certain properties\\
  \hline
\vr CellPrpHandle() & Return handle to a list of all current cell properties\\
  \hline
\vr GetCellPrpHandle({\it number\/}) & Return handle to a list of specific
  current cell properties\\ \hline
\vr PrpNext({\it prpty\_handle\/}) & Advance to the next property\\ \hline
\vr PrpNumber({\it prpty\_handle\/}) & Return the property number\\ \hline
\vr PrpString({\it prpty\_handle\/}) & Return the property string\\ \hline
\vr PrptyString({\it obj\_or\_prp\_handle\/}, {\it number\/}) & Return the
  property string\\ \hline
\vr GetPropertyString({\it number\/}) & Return property string from
 selected object\\ \hline
\vr GetCellPropertyString({\it number\/}) & Return property string from
 current cell\\ \hline
\vr PrptyAdd({\it object\_handle\/}, {\it number\/}, {\it string\/}) & Add a
  property\\ \hline
\vr AddProperty({\it number\/}, {\it string\/}) & Add properties to selected
  objects\\ \hline
\vr AddCellProperty({\it number\/}, {\it string\/}) & Add property to current
  cell\\ \hline
\vr PrptyRemove({\it object\_handle\/}, {\it number\/}, {\it string\/}) &
  Remove a property\\ \hline
\vr RemoveProperty({\it number\/}, {\it string\/}) & Remove properties from
  selected objects\\ \hline
\vr RemoveCellProperty({\it number\/}, {\it string\/}) & Remove properties from
  current cell\\ \hline
\end{longtable}

These are the computational geometry functions:

\begin{longtable}{|p{3.0in}|p{2.875in}|} \hline
\multicolumn{2}{|l|}{\kb Computational Geometry and layer Expressions}\\ \hline
 
% 071415
\multicolumn{2}{|c|}{\kb Trapezoid lists and Layer Expressions}\\ \hline
\vr SetZref({\it arg\/}) & Set background clipping zoidlist\\ \hline
\vr GetZref() & Return background clipping zoidlist\\ \hline
\vr GetZrefBB({\it array\/}) & Return background clipping zoidlist
  bounding box\\ \hline
\vr AdvanceZref({\it clear\/}, {\it array\/}) & Establish or advance grid
  clipping area\\ \hline
\vr Zhead({\it zoidlist\/}) & Extract and return leading trapezoid\\ \hline
\vr Zvalues({\it zoidlist\/}, {\it array\/}) & Extract parameters of leading
  trapezoid\\ \hline
\vr Zlength({\it zoidlist\/}) & Return number of trapezoids in list\\ \hline
\vr Zarea({\it zoidlist\/}) & Return total area of trapezoids in list\\ \hline
\vr GetZlist({\it layersrc\/}, {\it depth\/}) & Create zoidlist from
  cell\\ \hline
\vr GetSqZlist({\it layername\/}) & Create zoidlist from
  selected objects\\ \hline
\vr TransformZ({\it zoidlist\/}, {\it refx\/}, {\it refy\/}, {\it newx\/},
  {\it newy\/}) & Apply a transformation to a zoidlist\\ \hline
\vr BloatZ({\it dimen\/}, {\it zoidlist\/}, {\it mode\/}) & Bloat a
  zoidlist\\ \hline
\vr ExtentZ({\it zoidlist\/}) & Find the bounding box of a zoidlist\\ \hline
\vr EdgesZ({\it dimen\/}, {\it zoidlist\/}, {\it mode\/}) & Create an edge
  zoidlist\\ \hline
\vr ManhattanizeZ({\it dimen\/}, {\it zoidlist\/}, {\it mode\/}) &
  Manhattanize a zoidlist\\ \hline
\vr RepartitionZ({\it zoidlist\/}) & Canonicalize for horizontal split\\
  \hline
\vr BoxZ({\it l\/}, {\it b\/}, {\it r\/}, {\it t\/}) & Create zoidlist from
  box\\ \hline
\vr ZoidZ({\it xll\/}, {\it xlr\/}, {\it yl\/}, {\it xul\/}, {\it xur\/},
  {\it yu\/}) & Create zoidlist from trapezoid\\ \hline
\vr ObjectZ({\it object\_handle\/}, {\it all\/}) & Create zoidlist from
  object(s)\\ \hline
\vr ParseLayerExpr({\it string\/}) & Create layer\_expr from string\\ \hline
\vr EvalLayerExpr({\it layer\_expr\/}, {\it zoidlist\/}, {\it depth\/},
  {\it isclear\/}) & Evaluate layer expression in zoidlist\\ \hline
\vr TestCoverageFull({\it layer\_expr\/}, {\it zoidlist\/}, {\it minsize\/}) &
  Test layer expression for full coverage of zoidlist\\ \hline
\vr TestCoveragePartial({\it layer\_expr\/}, {\it zoidlist\/},
  {\it minsize\/}) & Test layer expression for partial coverage of
  zoidlist\\ \hline
\vr TestCoverageNone({\it layer\_expr\/}, {\it zoidlist\/}, {\it minsize\/}) &
  Test layer expression for no coverage of zoidlist\\ \hline
\vr TestCoverage({\it layer\_expr\/}, {\it zoidlist\/}, {\it testfull\/}) &
  Test layer expression in zoidlist\\ \hline
\vr ZtoObjects({\it zoidlist\/}, {\it lname\/}, {\it join\/},
  {\it to\_dbase\/}) & Create objects from zoidlist\\ \hline
\vr ZtoTempLayer({\it longname\/}, {\it zoidlist\/}, {\it join\/}) &
  Put objects from zoidlist in layer\\ \hline
\vr ClearTempLayer({\it longname\/}) & Clear objects in layer\\ \hline
\vr ZtoFile({\it filename\/}, {\it zoidlist\/}, {\it ascii\/}) & Save
  trapezoid list in file\\ \hline
\vr ZfromFile({\it filename\/}) & Extract trapezoid list from file\\ \hline
\vr ReadZfile({\it filename\/}) & Read trapezoids from file into current
  cell\\ \hline
\vr ChdGetZlist({\it chd\_name\/}, {\it cellname\/}, {\it scale\/},
  {\it array\/}, {\it clip\/}, {\it all\/}) &
  Extract trapezoid list through CHD\\ \hline

% 110213
\multicolumn{2}{|c|}{\kb Operations}\\ \hline
\vr Filt({\it zoids\/}, {\it lexpr\/}) & Trapezoid filtering\\ \hline
\vr GeomAnd({\it zoids1\/} [, {\it zoids2\/}]) & Geometrical AND function\\ \hline
\vr GeomAndNot({\it zoids1\/}, {\it zoids2\/}) & Clip second list from
  first\\ \hline
\vr GeomCat({\it zoids1\/}, ...) & Concatenate zoidlists\\ \hline
\vr GeomNot({\it zoids1\/}) & Invert zoidlist\\ \hline
\vr GeomOr({\it zoids1\/}, ...) & Merge zoidlist\\ \hline
\vr GeomXor({\it zoids1\/} [, {\it zoids2\/}]) & Exclusive-Or zoidlists\\
  \hline

% 010509
\multicolumn{2}{|c|}{\kb Spatial Parameter Tables}\\ \hline
\vr ReadSPtable({\it filename\/}) & Create or replace a table\\ \hline
\vr NewSPtable({\it name\/}, {\it x0\/}, {\it dx\/}, {\it nx\/}, {\it y0\/},
  {\it dy\/}, {\it ny\/}) & Create a table\\ \hline
\vr WriteSPtable({\it name\/}, {\it filename\/}) & Write a table to a file\\
  \hline
\vr ClearSPtable({\it name\/}) & Destroy a table\\ \hline
\vr FindSPtable({\it name\/}, {\it array\/}) & Find a table\\ \hline
\vr GetSPdata({\it name\/}, {\it x\/}, {\it y\/}) & Obtain value from
  table\\ \hline
\vr SetSPdata({\it name\/}, {\it x\/}, {\it y\/}, {\it value\/}) & Set
  table value\\ \hline

% 010509
\multicolumn{2}{|c|}{\kb Polymorphic Flat Database}\\ \hline
\vr ChdOpenOdb({\it chd\_name\/}, {\it scale\/}, {\it cellname\/},
  {\it array\/}, {\it clip\/}, {\it dbname\/}) & Open a flat object
  database\\ \hline
\vr ChdOpenZdb({\it chd\_name\/}, {\it scale\/}, {\it cellname\/},
  {\it array\/}, {\it clip\/}, {\it dbname\/}) & Open a flat trapezoid
  database\\ \hline
\vr ChdOpenZbdb({\it chd\_name\/}, {\it scale\/}, {\it cellname\/},
  {\it array\/}, {\it dbname\/}, {\it dx\/}, {\it dy\/}, {\it bx\/},
  {\it by\/}) & Open a binned flat trapezoid database\\ \hline
\vr GetObjectsOdb({\it dbname\/}, {\it layer\_list\/}, {\it array\/}) &
  Read objects from database\\ \hline
\vr ListLayersDb({\it dbname\/}) & List the layers used in the database\\
  \hline
\vr GetZlistDb({\it dbname\/}, {\it layer\_name\/}, {\it zoidlist\/}) &
  Read trapezoids from database\\ \hline
\vr GetZlistZbdb({\it dbname\/}, {\it layer\_name\/}, {\it nx\/}, {\it ny\/}) &
  Read trapezoids from ZBDB database\\ \hline
\vr DestroyDb({\it dbname\/}) & Destroy a database\\ \hline
\vr ShowDb({\it dbname\/}, {\it array\/}) & Display database region\\ \hline

% 020109
\multicolumn{2}{|c|}{\kb Named String Tables}\\ \hline
\vr FindNameTable({\it tabname\/}, {\it create\/}) & Verify existence of or
  create named string table\\ \hline
\vr RemoveNameTable({\it tabname\/}) & Destroy named string table\\ \hline
\vr ListNameTables() & List existing named string tables\\ \hline
\vr ClearNameTables() & Destroy all named string tables\\ \hline
\vr AddNameToTable({\it tabname\/}, {\it name\/}, {\it value\/}) & Add
  name/value to named string table\\ \hline
\vr RemoveNameFromTable({\it tabname\/}, {\it name\/}) & Remove name from
  named string table\\ \hline
\vr FindNameInTable({\it tabname\/}, {\it name\/}) & Return value for name in
  named string table\\ \hline
\vr ListNamesInTable({\it tabname\/}) & Return list of names in
  named string table\\ \hline
\end{longtable}

These functions are specific to design rule checking:

\begin{longtable}{|p{3.0in}|p{2.875in}|} \hline
\multicolumn{2}{|l|}{\kb Design Rule Checking Functions}\\ \hline

% 010715
\multicolumn{2}{|c|}{\kb DRC}\\ \hline
\vr DRCstate({\it state\/}) & Set interactive DRC\\ \hline
\vr DRCsetLimits({\it batch\_cnt\/}, {\it intr\_cnt\/}, {\it intr\_time\/},
  {\it skip\_cells\/}) & Set DRC limit values\\ \hline
\vr DRCgetLimits({\it array\/}) & Return DRC limit values\\ \hline
\vr DRCsetMaxErrors({\it value\/}) & Set the batch mode error limit\\ \hline
\vr DRCgetMaxErrors() & Return the batch mode error limit\\ \hline
\vr DRCsetInterMaxObjs({\it value\/}) & Set the interactive mode object
  count limit\\ \hline
\vr DRCgetInterMaxObjs() & Return the interactive mode object count limit\\
  \hline
\vr DRCsetInterMaxTime({\it value\/}) & Set the interactive mode time limit\\
  \hline
\vr DRCgetInterMaxTime() & Return the interactive mode time limit\\ \hline
\vr DRCsetInterMaxErrors({\it value\/}) & Set the interactive mode error
  count limit\\ \hline
\vr DRCgetInterMaxErrors() & Return the interactive mode error count limit\\
  \hline
\vr DRCsetInterSkipInst({\it value\/}) & Set the interactive mode instance\
  skip flag\\ \hline
\vr DRCgetInterSkipInst() & Return the interactive mode instance skip flag\\
  \hline
\vr DRCsetLevel({\it level\/}) & Set DRC error reporting level\\ \hline
\vr DRCgetLevel() & Return DRC error reporting level\\ \hline
\vr DRCcheckArea({\it array\/}, {\it file\_handle\_or\_name\/}) &
  Perform DRC in area\\ \hline
\vr DRCchdCheckArea({\it chdname\/}, {\it cellname\/}, {\it gridsize\/},
  {\it array\/}, {\it file\_handle\_or\_name\/}, {\it flatten\/}) &
  Perform DRC in area using CHD\\ \hline
\vr DRCcheckObjects({\it file\_handle\/}) & Perform DRC for selected objects\\
  \hline
\vr DRCregisterExpr({\it expr\/}) & Register a layer expression\\ \hline
\vr DRCtestBox({\it left\/}, {\it bottom\/}, {\it right\/}, {\it top\/},
  {\it ld\/}) & Perform DRC for given box\\ \hline
\vr DRCtestPoly({\it num\/}, {\it points\/}, {\it ld\/}) & Perform DRC for
  given polygon\\ \hline
\vr DRCzList({\it layername\/}, {\it rulename\/}, {\it index\/},
  {\it source\/}) & Create test areas in returned trapezoid list\\ \hline
\vr DRCzListEx({\it source\/}, {\it target\/}, {\it inside\/},
  {\it outside\/}, {\it incode\/}, {\it outcode\/}, {\it dimen\/}) &
  Create test areas in returned trapezoid list\\ \hline
\end{longtable}

Functions specifically for the extraction system:

\begin{longtable}{|p{3.0in}|p{2.875in}|} \hline
\multicolumn{2}{|l|}{\kb Extraction Functions}\\ \hline

% 052409
\multicolumn{2}{|c|}{\kb Menu Commands}\\ \hline
\vr DumpPhysNetlist({\it filename\/}, {\it depth\/}, {\it modestring\/},
  {\it names\/}) & Dump physical netlist\\ \hline
\vr DumpElecNetlist({\it filename\/}, {\it depth\/}, {\it modestring\/},
  {\it names\/}) & Dump electrical netlist\\ \hline
\vr SourceSpice({\it filename\/}, {\it modestring\/}) & Update electrical
  from SPICE file\\ \hline
\vr ExtractAndSet({\it depth\/}, {\it modestring\/}) & Update electrical
  from physical\\ \hline
\vr FindPath({\it x\/}, {\it y\/}, {\it depth\/}, {\it use\_extract\/}) &
  Return objects in netlist\\ \hline
\vr FindPathOfGroup({\it groupnum\/}, {\it depth\/}) &
  Return objects in netlist\\ \hline

% 060716
\multicolumn{2}{|c|}{\kb Terminals}\\ \hline
\vr ListTerminals() & List cell contact terminals\\ \hline
\vr FindTerminal({\it name\/}, {\it index\/}, {\it use\_e\/},
  {\it xe\/}, {\it ye\/}, {\it use\_p\/}, {\it xp}, {\it yp\/}) &
  Find a cell connection terminal\\ \hline
\vr CreateTerminal({\it name\/}, {\it x\/}, {\it y\/}, {\it termtype\/}) &
  Create new contact terminal\\ \hline
\vr DestroyTerminal({\it thandle\/}) &
  Remove and destroy cell contact terminal\\ \hline
\vr GetTerminalName({\it thandle\/}) &
  Return terminal name\\ \hline
\vr SetTerminalName({\it thandle\/}, {\it name\/}) &
  Assign terminal name\\ \hline
\vr GetTerminalType({\it thandle}) &
  Return terminal type code\\ \hline
\vr SetTerminalType({\it thandle\/}, {\it termtype\/}) &
  Set terminal type\\ \hline
\vr GetTerminalFlags({\it thandle\/}) &
  Return terminal flags\\ \hline
\vr SetTerminalFlags({\it thandle\/}, {\it flags\/}) &
  Set terminal flags\\ \hline
\vr UnsetTerminalFlags({\it thandle\/}, {\it flags\/}) &
  Unset terminal flags\\ \hline
\vr GetElecTerminalLoc({\it thandle\/}, {\it index\/}, {\it array\/}) &
  Return electrical terminal location\\ \hline
\vr SetElecTerminalLoc({\it thandle\/}, {\it x\/}, {\it y\/}) &
  Assign electrical terminal location\\ \hline
\vr ClearElecTerminalLoc({\it thandle\/}, {\it x\/}, {\it y\/}) &
  Delete symbolic duplicate location\\ \hline

% 060716
\multicolumn{2}{|c|}{\kb Physical Terminals}\\ \hline
\vr ListPhysTerminals() & List physical cell contact terminals\\ \hline
\vr FindPhysTerminal({\it name\/}, {\it use\_p\/}, {\it xp}, {\it yp\/}) &
  Find a physical cell connection terminal\\ \hline
\vr CreatePhysTerminal({\it thandle\/}, {\it x\/}, {\it y\/}, {\it layer\/}) &
  Create new linkage to layout terminal\\ \hline
\vr HasPhysTerminal({\it thandle\/}) &
  Check if terminal has physical component\\ \hline
\vr DestroyPhysTerminal({\it thandle\/}) &
  Remove and destroy layout terminal linkage\\ \hline
\vr GetPhysTerminalLoc({\it thandle\/}, {\it array\/}) &
  Return layout terminal location\\ \hline
\vr SetPhysTerminalLoc({\it thandle\/}, {\it x\/}, {\it y\/}) &
  Assign layout terminal location\\ \hline
\vr GetPhysTerminalLayer({\it thandle\/}) &
  Return associated layer name\\ \hline
\vr SetPhysTerminalLayer({\it thandle\/}, {\it layer\/}) &
  Set layer name for hinting\\ \hline
\vr GetPhysTerminalGroup({\it thandle\/}) &
  Return associated physical group number\\ \hline
\vr GetPhysTerminalObject({\it thandle\/}) &
  Return handle to associated object\\ \hline

% 070516
\multicolumn{2}{|c|}{\kb Physical Conductor Groups}\\ \hline
\vr Group() & Run extraction\\ \hline
\vr GetNumberGroups() & Return number of groups\\ \hline
\vr GetGroupBB({\it group\/}, {\it array\/}) & Return bounding box of group\\
  \hline
\vr GetGroupNode({\it group\/}) & Return node of group\\ \hline
\vr GetGroupName({\it group\/}) & Return net or formal terminal name\\ \hline
\vr GetGroupNetName({\it group\/}) & Return net name\\ \hline
\vr GetGroupCapacitance({\it group\/}) & Return group capacitance\\ \hline
\vr CountGroupObjects({\it group\/}) & Count physical objects in group\\ \hline
\vr ListGroupObjects({\it group\/}) & Return list of objects in group\\ \hline
\vr CountGroupVias({\it group\/}) & Count standard vias or via cells used in
 the group\\ \hline
\vr ListGroupVias({\it group\/}) & Return list of standard via or via cell
 instances used in the group\\ \hline
\vr CountGroupDevContacts({\it group\/}) & Count device contacts in group\\
  \hline
\vr ListGroupDevContacts({\it group\/}) & Return list of device contacts in
  group\\ \hline
\vr CountGroupSubcContacts({\it group\/}) & Count subcircuit contacts in
  group\\ \hline
\vr ListGroupSubcContacts({\it group\/}) & Return list of subcircuit contacts
  in group\\ \hline
\vr CountGroupTerminals({\it group\/}) & Count cell connection terminals in
  group\\ \hline
\vr ListGroupTerminals({\it group\/}) & Return list of cell connection
  terminals in group\\ \hline
\vr ListGroupTerminalNames({\it group\/}) & Return list of cell contact
  terminal names in group\\ \hline
\vr CountGroupPhysTerminals({\it group\/}) & Count physical terminals in
  group\\ \hline
\vr ListGroupPhysTerminals({\it group\/}) & Return list of physical terminals
  in group\\ \hline

% 070809
\multicolumn{2}{|c|}{\kb Physical Devices}\\ \hline
\vr ListPhysDevs({\it name\/}, {\it pref\/}, {\it indices\/},
  {\it area\_array\/}) & Return list of physical devices\\ \hline
\vr GetPdevName({\it device\_handle\/}) & Return device name\\ \hline
\vr GetPdevIndex({\it device\_handle\/}) & Return device index\\ \hline
\vr GetPdevDual({\it device\_handle\/}) & Return corresponding electrical
  device\\ \hline
\vr GetPdevBB({\it device\_handle\/}, {\it array\/}) & Return device bounding
  box\\ \hline
\vr GetPdevMeasure({\it device\_handle\/}, {\it mname\/}) & Return device
  measurement\\ \hline
\vr ListPdevMeasures({\it device\_handle\/}) & Return list of measurement
  keywords\\ \hline
\vr ListPdevContacts({\it device\_handle\/}) & Return list of device contacts\\
  \hline
\vr GetPdevContactName({\it dev\_contact\_handle\/}) & Return device contact
  name\\ \hline
\vr GetPdevContactBB({\it dev\_contact\_handle\/}, {\it array\/}) & Return
  device contact bounding box\\ \hline
\vr GetPdevContactGroup({\it dev\_contact\_handle\/}) & Return device contact
  conductor group\\ \hline
\vr GetPdevContactLayer({\it dev\_contact\_handle\/}) & Return device contact
  layer\\ \hline
\vr GetPdevContactDev({\it dev\_contact\_handle\/}) & Return device containing
  contact\\ \hline
\vr GetPdevContactDevName({\it dev\_contact\_handle\/}) & Return name of
  device containing contact\\ \hline
\vr GetPdevContactDevIndex({\it dev\_contact\_handle\/}) & Return index of
  device containing contact\\ \hline

% 061116
\multicolumn{2}{|c|}{\kb Physical Subcircuits}\\ \hline
\vr ListPhysSubckts({\it name\/}, {\it index\/}, {\it l\/}, {\it b\/},
  {\it r\/}, {\it t\/}) & Return list of physical subcircuits\\ \hline
\vr GetPscName({\it subckt\_handle\/}) & Return master name of physical
  subcircuit\\ \hline
\vr GetPscIndex({\it subckt\_handle\/}) & Return index of physical subcircuit\\
  \hline
\vr GetPscIdNum({\it subckt\_handle\/}) & Return id number of physical
  subcircuit\\ \hline
\vr GetPscInstName({\it subckt\_handle\/}) & Return instance name of physical
  subcircuit\\ \hline
\vr GetPscDual({\it subckt\_handle\/}) & Return corresponding electrical
  subcircuit\\ \hline
\vr GetPscBB({\it subckt\_handle\/}, {\it array\/}) & Return physical
  subcircuit bounding box\\ \hline
\vr GetPscLoc({\it subckt\_handle\/}, {\it array\/}) & Return physical
  subcircuit placement location\\ \hline
\vr GetPscTransform({\it subckt\_handle\/}, {\it type\/}, {\it array\/}) &
  Return physical subcircuit orienttion string\\ \hline
\vr ListPscContacts({\it subckt\_handle\/}) & Return list of contacts\\ \hline
\vr IsPscContactIgnorable({\it subc\_contact\_handle\/}) & Return 1 if contact
  to ignored subcircuit\\ \hline
\vr GetPscContactName({\it subc\_contact\_handle\/}) & Return name of
  subcircuit\\ \hline
\vr GetPscContactGroup({\it subc\_contact\_handle\/}) & Return conductor group
  of contact\\ \hline
\vr GetPscContactSubcGroup({\it subc\_contact\_handle\/}) & Return group of
  contact in subcircuit\\ \hline
\vr GetPscContactSubc({\it subc\_contact\_handle\/}) & Return subcircuit
  containing contact\\ \hline
\vr GetPscContactSubcName({\it subc\_contact\_handle\/}) & Return name of
  subcircuit containing contact\\ \hline
\vr GetPscContactSubcIndex({\it subc\_contact\_handle\/}) & Return index of
  subcircuit containing contact\\ \hline
\vr GetPscContactSubcIdNum({\it subc\_contact\_handle\/}) & Return id number of
  subcircuit containing contact\\ \hline
\vr GetPscContactSubcInstName({\it subc\_contact\_handle\/}) & Return instnce
  name of subcircuit containing contact\\ \hline

% 100508
\multicolumn{2}{|c|}{\kb Electrical Devices}\\ \hline
\vr ListElecDevs({\it regex\/}) & Return list of electrical devices\\ \hline
\vr SetEdevProperty({\it devname\/}, {\it prpty\/}, {\it string\/}) & Set
  electrical device property\\ \hline
\vr GetEdevProperty({\it devname\/}, {\it prpty\/}) & Return electrical device
  property\\ \hline
\vr GetEdevObj({\it devname\/}) & Return electrical device subcell object\\
  \hline

% 052409
\multicolumn{2}{|c|}{\kb Resistance/Inductance Extraction}\\ \hline
\vr ExtractRL({\it conductor\_zoidlist\/}, {\it layername\/},
  {\it r\_or\_l\/}, {\it array\/}, {\it term\/}, ...) &
  Extract resistance or inductance from object\\ \hline
\vr ExtractNetResistance({\it net\_handle\/}, {\it spicefile\/},
  {\it array\/}, {\it term\/}, ...) &
  Extract resistance from wire net\\ \hline
\end{longtable}

Functions for electrical schematic editing:

\begin{longtable}{|p{3.0in}|p{2.875in}|} \hline
\multicolumn{2}{|l|}{\kb Schematic Editor Functions}\\ \hline

% 060616
\multicolumn{2}{|c|}{\kb Output Generation}\\ \hline
\vr Connect({\it for\_spice\/}) & Internally process the schematic\\ \hline
\vr ToSpice({\it spicefile\/}) & Write SPICE file\\ \hline

% 060616
\multicolumn{2}{|c|}{\kb Electrical Nodes}\\ \hline
\vr IncludeNoPhys({\it flag\/}) & Set {\et nophys} property usage\\ \hline
\vr GetNumberNodes() & Return number of nodes in circuit\\ \hline
\vr SetNodeName({\it node\/}, {\it name\/}) & Set text name for node\\ \hline
\vr GetNodeName({\it node\/}) & Return text name for node\\ \hline
\vr GetNodeNumber({\it name\/}) & Return node number for named node\\ \hline
\vr GetNodeGroup({\it node\/}) & Return corresponding group for node\\ \hline
\vr ListNodePins({\it node\/}) & Return list of connected cell contact
 terminals\\ \hline
\vr ListNodeContacts({\it node\/}) & Return list of connected instance
 terminals\\ \hline
\vr GetNodeContactInstance({\it terminal\_handle\/}) & Return handle to
 instance providing contact\\ \hline
\vr ListNodePinNames({\it node\/}) & Return list of connected cell contact
 terminal names\\ \hline
\vr ListNodeContactNames({\it node\/}) & Return list of connected instance
 terminal names\\ \hline

% 030115
\multicolumn{2}{|c|}{\kb Symbolic Mode}\\ \hline
\vr IsShowSymbolic() & True if current cell displayed symbolically in main
  window\\ \hline
\vr ShowSymbolic({\it show\/}) & Turn on/off symbolic display\\ \hline
\vr SetSymbolicFast({\it symb\/}) & Set symbolic mode of current cell,
  no display update\\ \hline
\vr MakeSymbolic() & Create simple symbolic representation\\ \hline

\end{longtable}


%------------------------------------------------------------------------------
\section{Main Functions 1}
\subsection{Current Cell}
\label{iffuncs}

\begin{description}
%------------------------------------
% 062109
\index{Edit function}
\item{(int) \vt Edit({\it name\/}, {\it symname\/})}\\
This function will read in the named file or cell and make it, or one
of the cells in the hierarchy, the current cell.  If the present cell
has been modified, in graphics mode the user is prompted for whether
to save the cell before reading the new one.  The {\it name} argument
can be null or empty, in which case the user will be prompted for a
file or cell to open for editing, if in graphics mode.  If not in
graphics mode, an empty cell is created in memory and made the current
cell.

The {\it name} provided can be an archive file, the name of an {\Xic}
cell, a library file, or the ``database name'' of a Cell Hierarchy
Digest (CHD).  If a CHD name or the name of an archive file is given,
the name of the cell to open can be provided as {\it symname\/}.  If
{\it symname} is null or empty, The CHD's default cell, or the top
level cell (the one not used as a subcell by any other cells in the
file) is the one opened for editing.  If there is more than one top
level cell, in graphics mode the user is presented with a pop-up
choice menu and asked to make a selection.  If the file is a library
file, the {\it symname} can be given, and it should be one of the
reference names from the library, or the name of a cell defined in the
library.  If {\it symname} is null or empty, in graphics mode a pop-up
listing the library contents will appear, allowing the user to select
a reference or cell.  If not in graphics mode, and the cell to edit
can not be determined, the current cell is unchanged, and nothing is
read.

See the table in \ref{features} for the features that apply during a
call to this function.  This function is consistent with the {\cb
Open} menu command in that cell name aliasing, layer filtering and
modification, and scaling are not available (unlike in the pre-3.0.0
version of this function).  If these features are needed, the {vt
OpenCell} function should be used instead.

The return value is one of the following integers, representing the
command status:

\begin{tabular}{|l|p{4.0in}|}\hline
-2 & The function call was reentered.  This is not likely to happen in
  scripts.\\ \hline
-1 & The user aborted the operation.\\ \hline
0 & The open failed: bad file name, parse error, etc.\\ \hline
1 & The operation succeeded.\\ \hline
2 & The read was successful on an archive with multiple top-level cells
  but the cells to edit can't be determined.  The current cell has not been
  set, but the cells are in memory.  The second argument could have been
  used to resolve the ambiguity.\\ \hline
3 & The cell name was the name of the device library or model library
  file, which has been opened for text editing (in graphic mode only).\\ \hline
\end{tabular}

%------------------------------------
% 100408
\index{OpenCell function}
\item{(int) \vt OpenCell({\it name\/}, {\it symname\/}, {\it curcell\/})}\\
This function will read a file into memory, similar to the {\vt Edit}
function.  The first two arguments are the same as would be passed to
{\vt Edit}.  The third argument is a boolean value.

See the table in \ref{features} for the features that apply during a
call to this function.

If {\it curcell} is nonzero, then this function will behave like the
{\vt Edit} function in switching the current cell to a newly-read
cell.  The only difference from {\vt Edit} is that scaling, layer
filtering and aliasing, and cell name modification are allowed, as in
the pre-3.0.0 versions of the {\vt Edit} function.  The return values
are those listed for the {\vt Edit} function.

If {\it curcell} is zero, the new cell will not be the current cell. 
Once in memory, the cell is available by its simple name, for use by
the {\vt Place} function for example.  If {\it name} is the name of an
archive or library file, {\it symname} is the cell or reference to
open, similar to the {\vt Edit} function.  In this mode, the return
value is 1 on success, 0 otherwise.

%------------------------------------
% 030115
\index{TouchCell function}
\item{(int) \vt TouchCell({\it cellname\/}, {\it curcell\/})}\\
If no cell exists in the current symbol table for the current mode
with the given name, create an empty cell for {\it cellname} and add
it to the symbol table.  If the boolean {\it curcell} is true, switch
the current cell to {\it cellname}.  This can be much faster than {\vt
Edit} or {\vt OpenCell} for cells already in memory.  The return value
is -1 on error, 0 if no new cell was created, or 1 if a new cell was
created.

%------------------------------------
% 012114
\index{Push function}
\item{(int) \vt Push({\it object\_handle\/})}\\
This function will push the editing context to the cell of the
instance referenced by the handle, that is, make it the currrent cell. 
The handle is the return value from the {\vt SelectHandle} or {\vt
AreaHandle} functions.  This is similar to the {\cb Push} command in
{\Xic}.  The editing context can be restored with the {\vt Pop}
function.  If the instance is an array, the 0,0 element will be pushed
(see {\vt PushElement}).

If successful, 1 is returned, otherwise 0 is returned.  This function
will fail if the handle passed is not a handle to an object list.

This function implicitly calls {\vt Commit} before the context change.

%------------------------------------
% 012114
\index{PushElement function}
\item{(int) \vt PushElement({\it object\_handle\/}, {\it xind\/},
  {\it yind\/})}\\
This is very similar to {\vt Push}, but allows passing indices which
select the instance element to push if the instance is arrayed.  The
indices are always effectively 0 in the {\vt Push} function.  An out
of range index value will cause the function to return 0 and not push
the context.  If both index values are zero, the function is identical
to {\vt Push}.  The selection of the array element only affects the
graphical display.

This function implicitly calls {\vt Commit} before the context change.

%------------------------------------
% 022206
\index{Pop function}
\item{(int) \vt Pop()}\\
This function will pop the editing context to the parent cell, to be
used after the {\vt Push} function or a {\cb Push} command in {\Xic}. 
The {\vt Pop} function always returns 1, and has no effect if there
was no corresponding push.

This function implicitly calls {\vt Commit} before the context change.

%------------------------------------
% 041704
\index{NewCellName function}
\item{(string) \vt NewCellName()}\\
This function returns a string which is a valid cell name that does
not conflict with any cell in the current symbol table.  The cell is
not actually created.  This can be used with the {\vt Edit} function
to open a new cell for editing, similar to the {\cb New} button in the
{\cb File Menu}.  This function never fails.

%------------------------------------
% 030104
\index{CurCellName function}
\item{(string) \vt CurCellName()}\\
The return value of this function is a string containing the name of
the current cell.

%------------------------------------
% 030104
\index{TopCellName function}
\item{(string) \vt TopCellName()}\\
The return value of this function is a string containing the name of
the top level cell in the hierarchy being edited.  This is different
from the current cell name while in a subedit (i.e., the {\cb Push}
command is active).

%------------------------------------
% 030104
\index{FileName function}
\item{(string) \vt FileName()}\\
This function returns the name of the file from which the current cell
was read.  If there is no such file, a null string is returned.

%------------------------------------
% 022012
\index{CurCellBB function}
\item{(int) \vt CurCellBB({\it array\/})}\\
This function will return the bounding box of the current cell, in
microns, in the {\it array}, as l, b, r, t.  The array must have size 4
or larger.  The function returns 1 on success, 0 if there is no
current cell.

In electrical mode, the bounding box returned will be for the
schematiic or symbolic representation, matching how the cell is
displayed in the main window.  See the {\vt CellBB} function for
an alternative.

%------------------------------------
% 101108
\index{SetCellFlag function}
\item{(int) \vt SetCellFlag({\it cellname\/}, {\it flagname\/}, {\it set\/})}\\
This will set a flag (see \ref{cellflags}) in the cell whose name is
passed as the first argument.  If this argument is 0, or a null or
empty string, the current cell is understood.  The second argument is
a string giving the flag name.  This must be the name of a
user-modifiable flag.  The third argument is a boolean indicating the
new flag state, a nonzero value will set the flag, zero will unset it. 
The return value is the previous flag status (0 or 1), or -1 on error. 
On error, a message can be obtained from {\vt GetError}.
 
{\bf Warning}:  This affects the user flags directly, and does {\bf
not} update the property used to hold flag status that is written to
disk when the cell is saved.  These flags should be set by setting the
{\bf Flags} property (property number 7105) with {\vt AddProperty} or
{\it AddCellProperty}, if the values need to persist when the cell is
written to disk and reread.

%------------------------------------
% 101108
\index{GetCellFlag function}
\item{(int) \vt GetCellFlag({\it cellname\/}, {\it flagname\/})}\\
This will query a flag (see \ref{cellflags}) in the cell whose name is
passed as the first argument.  If this argument is 0, or a null or
empty string, the current cell is understood.  The second argument is
a string giving the flag name, which can be any or the flag names. 
The return value is the flag status (0 or 1), or -1 on error.  On
error, a message can be obtained from {\vt GetError}.

%------------------------------------
% 033009
\index{Save function}
\item{(int) \vt Save({\it newname\/})}\\
This function will save to disk file the current cell, and its
descendents if the cell originated from an archive file.  If the
argument is null or the empty string, the current cell name is used,
suffixed with one of the following if saving as an archive:

\begin{tabular}{|l|l|}\hline
CGX &     {\vt .cgx}\\ \hline
CIF &     {\vt .cif}\\ \hline
GDSII &   {\vt .gds}\\ \hline
OASIS &   {\vt .oas}\\ \hline
\end{tabular}

The default format will be the format of the original input file,
though format conversion can be imposed by adding one of these
suffixes or ``{\vt .xic}'' to {\it newname\/}.  The cell is saved
unconditionally; there is no user prompt.

See the table in \ref{functions} for the features that apply during a
call to this function.

This function returns 1 on success, 0 otherwise.  On error, a message
is likely available from {\vt GetError}.

%------------------------------------
% 010406
\index{UpdateNative function}
\item{(int) \vt UpdateNative({\it dir\/})}\\
This will write to disk all of the modified cells in the current
hierarchy as native cell files in the directory given as the argument. 
If the argument is null or empty, cells will be written in the current
directory.  The return value is the number of cells written.

Note that only modified or internally created cells will be written. 
To write all cells as native cell files, use the {\vt ToXIC} function.

\end{description}

\subsection{Cell Info}

\begin{description}
%------------------------------------
% 022012
\index{CellBB function}
\item{(int) \vt CellBB({\it cellname\/}, {\it array\/} [, {\it symbolic\/}])}\\
This function will return the bounding box of the named cell in the
current mode, in microns, in the array, as l, b, r, t.  If {\it
cellname} is null or empty, the current cell is used.  The array must
have size 4 or larger.  The function returns 1 on success, 0 if the
cell is not found in memory.

The optional boolean third argument applies to electrical cells.  If
not given or set to false, the schematic bounding box is always
returned.  If this argument is true, and the cell has a symbolic
representation, the symbolic representation bounding box is returned,
or the function fails and returns 0 if the cell has no symbolic
representation.

%------------------------------------
% 020411
\index{ListSubcells function}
\item{(stringlist\_handle) \vt ListSubcells({\it cellname\/}, {\it depth\/},
 {\it array\/}, {\it incl\_top\/})}\\
This function returns a handle to a sorted list of subcell names found
under the named cell, to the given depth, and only if instantiated so
as to overlap a rectangular area (if given).  These apply to the
current mode, electrical or physical.  If {\it cellname} is null or
empty, the current cell is used.  The {\it depth} is the search depth,
which can be an integer which sets the maximum depth to search (0
means search {\it cellname} only and return its subcell names, 1 means
search {\it cellname} plus its subcells, etc., and a negative integer
sets the depth to search the entire hierarchy).  This argument can
also be a string starting with `{\vt a}' such as ``{\vt a}'' or ``{\vt
all}'' which indicates to search the entire hierarchy.

The cell will be read into memory if not already there.  The function
fails if the cell can not be found.

The {\it array} argument can be passed 0, which indicates no area
testing.  Otherwise, the array should be size four or larger, with the
values being the left ({\it array\/}[0]), bottom, right, and top
coordinates of a rectangular region of {\it cellname}.  Only cells
that are instantiated such that the instance bounding box, when
reflected to top-level coordinates, intersects the region will be
listed.

If the boolean {\it incl\_top} is nonzero, the top cell name ({\it
cellname\/}) will be included in the list, unless an array is given
and there is no overlap with the top cell.

The return is a handle to a list of cell names, and can be empty.  The
{\vt GenCells} or {\vt ListNext} functions can be used to iterate
through the list.

%------------------------------------
% 020411
\index{ListParents function}
\item{(stringlist\_handle) \vt ListParents({\it cellname\/})}\\
This function returns a list of cell names, each of which contain an
instance of the cell name passed as the argument.  These apply to the
current mode, electrical or physical.  If {\it cellname} is null or
empty, the current cell is used.

The function fails if the cell can not be found in memory.  
  
The return is a handle to a list of cell names, and can be empty.  The
{\vt GenCells} or {\vt ListNext} functions can be used to iterate
through the list.

%------------------------------------
% 020411
\index{InitGen function}
\item{(stringlist\_handle) \vt InitGen()}\\
This function returns a handle to a list of names of cells used in the
hierarchy of the current cell, either the physical or electrical part
according to the current mode.  Each cell is listed once only, and all
cells are listed, including the current cell which is returned last.

The return is a handle to a list of cell names, and can be empty.  The
{\vt GenCells} or {\vt ListNext} functions can be used to iterate
through the list.

%------------------------------------
% 020411
\index{CellsHandle function}
\item{(stringlist\_handle) \vt CellsHandle({\it cellname\/},
 {\it depth\/})}\\
This function returns a handle to a list of subcell names found in
{\it cellname}, to the given hierarchy depth.  If {\it cellname} is null
or empty, the current cell is used.  The {\it depth} is the search
depth, which can be an integer which sets the maximum depth to search
(0 means search {\it cellname} only and return its subcell names, 1
means search {\it cellname} plus its subcells, etc., and a negative
integer sets the depth to search the entire hierarchy).  This argument
can also be a string starting with `a' such as {\vt "a"} or {\vt
"all"} which indicates to search the entire hierarchy.  The listing
order is as a tree, with a subcell listed followed by the descent into
that subcell.

The cell will be read into memory if not already there.  The function
fails if the cell can not be found.

With ``{\vt all}'' passed, the output is similar to that of the {\vt
InitGen} function, except that the top-level cell name is not listed,
and duplicate entries are not removed ({\vt ListUnique} can be called
to remove duplicate names).

Be aware that the listing will generally contain lots of duplicate
names.  This function is not recommended for general hierarchy
traversal.

The return is a handle to a list of cell names, and can be empty.  The
{\vt GenCells} or {\vt ListNext} functions can be used to iterate
through the list.

%------------------------------------
% 020411
\index{GenCells function}
\item{(string) \vt GenCells({\it stringlist\_handle})}\\
This function returns a string containing the name of one of the
elements in the list whose handle is passed as the argument.  It
advances the handle to point to the next name.  The argument can be
the return value from one of the functions above, or any {\it
stringlist\_handle} variable.  A different name is returned for each
call.  The null string is returned after all names have been returned. 
This is identical to the {\vt ListNext} function.

Example:
\begin{quote}
This script will list all of the cells in the current hierarchy:
\begin{verbatim}
i = InitGen()
while ((name = GenCells(i)) != 0)
   Print(name)
end
\end{verbatim}
\end{quote}

\end{description}


\subsection{Database}

\begin{description}
%------------------------------------
% 020411
\index{Clear function}
\item{\vt Clear({\it cellname\/})}\\
If {\it cellname} is not empty, any matching cell and all its
descendents are cleared from the database, unless they are referenced
by another cell not being cleared.  If {\it cellname} is null or
empty, the entire database is cleared.  This function is obviously
very dangerous.

%------------------------------------
% 101412
\index{ClearAll function}
\item{\vt ClearAll({\it clear\_tech\/})}\\
This will clear all cells from the present symbol table, clear and
delete any other symbol tables that may be defined, and revert the
layer database.  If the boolean argument is nonzero, layers read from
the technology file will be cleared, otherwise the layer database is
reverted to the state just after the technology file was read.  This
function does {\bf not} automatically open a new cell.  This is for
server mode, to give the system a good scrubbing between jobs.

%------------------------------------
% 072904
\index{IsCellInMem function}
\item{(int) \vt IsCellInMem({\it cellname\/})}\\
This function returns 1 if the string {\it cellname} is the name of a
cell in the current symbol table, 0 otherwise.  If the string contains
a path prefix, it will be ignored, and the last (filename) component
used for the test.

%------------------------------------
% 051310
\index{IsFileInMem function}
\item{(int) \vt IsFileInMem({\it filename\/})}\\
This will compare the string {\it filename} to the source file names
saved with top-level cells in the current symbol table.  If {\it
filename} is a full path, the function returns 1 if an exact match is
found.  If {\it filename} is not rooted, the function returns 1 if the
last path component matches.  In either case, 0 is returned if no
match is seen.

%------------------------------------
% 030104
\index{NumCellsInMem function}
\item{(int) \vt NumCellsInMem()}\\
This function returns an integer giving the number of cells in the
current symbol table.

%------------------------------------
% 032215
\index{ListCellsInMem function}
\item{(stringlist\_handle) \vt ListCellsInMem({\it options\_str\/})}\\
This function returns a handle to a list of strings, sorted
alphabetically, giving the names of cells found in the current symbol
table.
  
A fairly extensive filtering capability is available, which is
configured through a string passed as the argument.  If 0 is passed,
or the options string is null or empty, all cells will be listed.

The string consists of a space-separated list of keywords, each of
which represents a condition for filtering.  The cells listed will be
the logical AND of all option clauses.  The keysords are described
with the {\cb Cell List Filter} panel in \ref{cellfilt}.

%------------------------------------
% 030104
\index{ListTopCellsInMem function}
\item{(stringlist\_handle) \vt ListTopCellsInMem()}\\
This function returns a handle to a list of strings, sorted
alphabetically, giving the names of top-level cells in the current
symbol table.  These are the cells that are not used as subcells, in
either physical or electrical mode.

%------------------------------------
% 030104
\index{ListModCellsInMem function}
\item{(stringlist\_handle) \vt ListModCellsInMem()}\\
This function returns a handle to a list of strings, sorted
alphabetically, giving the names of modified cells in the current
symbol table.  A cell is modified if the contents have changed since
the cell was read or last written to disk.

%------------------------------------
% 030104
\index{ListTopFilesInMem function}
\item{(stringlist\_handle) \vt ListTopFilesInMem()}\\
This function returns a handle to a list of strings, alphabetically
sorted, giving the source file names of the top-level cells in the
current symbol table.

\end{description}


\subsection{Symbol Tables}

\begin{description}
%------------------------------------
% 100408
\index{SetSymbolTable function}
\item{(string) \vt SetSymbolTable({\it tabname\/})}\\
This function will set the current symbol table to the table named in
the argument string.  If the {\it tabname} is null or empty, the
default ``{\vt main}'' table is understood.  If a table by the given
name does not exist, a new table will be created for that name.
   
The return value is a string giving the name of the active table
before the switch.

%------------------------------------
% 100408
\index{ClearSymbolTable function}
\item{(int) \vt ClearSymbolTable({\it destroy\/})}\\
This function will clear or destroy the current symbol table.  If the
boolean argument is nonzero, and the current table is not the ``{\vt
main}'' table, the current table and its contents will be destroyed. 
Otherwise, the current table will be cleared, i.e., all contained
cells will be destroyed.  If the current symbol table is destroyed, a
new current table will be installed from among the internal list of
existing tables.
   
This function always returns 1.

%------------------------------------
% 100408
\index{CurSymbolTable function}
\item{(string) \vt CurSymbolTable()}\\
This function returns a string giving the name of the current
symbol table.

\end{description}

\subsection{Display}

\begin{description}
%------------------------------------
% 030104
\index{Window function}
\item{(int) \vt Window({\it x\/}, {\it y\/}, {\it width\/}, {\it win\/})}\\
The window view is changed so that it is centered at {\it x\/}, {\it
y} and has width set by the third argument.  If the {\it width\/}
argument is less than or equal to zero, a centered, full view of the
current cell is obtained.  In this case, the {\it x\/}, {\it y\/}
arguments are ignored.  The {\it win} is an integer 0--4 which
specifies the window:

\begin{tabular}{ll}
0 & Main drawing window\\
1--4 & Sub-window (number as shown in title bar)
\end{tabular}

The function returns 1 on success, 0 if the indicated window does
not exist.

%------------------------------------
% 030104
\index{GetWindow function}
\item{(int) \vt GetWindow()}\\
This function returns the window number of the drawing window
that contains the pointer.  The window number is an integer 0--4:

\begin{tabular}{ll}
0 & Main drawing window\\
1--4 & Sub-window (number as shown in title bar)
\end{tabular}

If the pointer is not in a drawing window, 0 is returned.

%------------------------------------
% 030104
\index{GetWindowView function}
\item{(int) \vt GetWindowView({\it win}, {\it array\/})}\\
This function returns the view area (visible cell coordinates) of the
given window {\it win}, which is an integer 0--4 where 0 is the main
window and 1--4 represent sub-windows.  The view coordinates, in
microns, are returned in the {\it array}, in order L, B, R, T.  On
success, 1 is returned, otherwise 0 is returned and the {\it array} is
untouched.

%------------------------------------
% 030104
\index{GetWindowMode function}
\item{(int) \vt GetWindowMode({\it win})}\\
This function returns the display mode of the given window {\it win},
which is 0 for physical mode, 1 for electrical, or -1 if the window
does not exist.  The argument is an integer 0--4, where 0 represents
the main window and 1--4 indicate sub-windows.  This function is
identical to {\vt CurMode}.

%------------------------------------
% 030104
\index{Expand function}
\item{(int) \vt Expand({\it win\/}, {\it string\/})}\\
This sets the expansion mode for the display in the window specified
in {\it win\/}.  The {\it win} argument is an integer 0--4, where 0
refers to the main window, and 1--4 correspond to the sub-windows
brought up with the {\cb Viewport} command.  The {\it string} contains
characters which modify the display mode, as would be given to the
{\cb Expand} command in the {\cb View Menu}.

\begin{tabular}{ll}
integer  & set expand level\\
\vt n    & set level to 0\\
\vt a    & expand all\\
$+$      & increment expand level\\
$-$      & decrement expand level\\
\end{tabular}

%------------------------------------
% 030104
\index{Display function}
\item{(int) \vt Display({\it display\_string\/}, {\it win\_id\/},
{\it l\/}, {\it b\/}, {\it r\/}, {\it t\/})}\\
This function will render the current cell in a foreign X window.  The
X window id is passed as an integer in the second argument.  The first
argument is the X display string corresponding to the server in which
the window is cached.  The remaining arguments set the area to be
displayed, in microns.  The function returns 1 upon success, 0
otherwise.  This function is useful for rendering a layout if
interactive graphics is not enabled, such as in server mode.  This
function will not work under Microsoft Windows.

This is a primitive to allow {\Xic} to export graphics rendering
capability.  The intention is that this might be used in a {\et Tk}
script (for example) that is otherwise using {\Xic} in server mode as
a back-end.  The machine containing the window to be drawn into must
allow X access to the machine running the {\Xic} server (see the {\et
xhost} Unix command).

One can demonstrate the capability as follows.  The ``{\vt xwininfo
-children}'' Unix command can be used to find the window id of a
suitable {\it child} window in a running application.  The top-level
window given from {\et xwininfo} without the ``{\vt -children}''
argument is generally obscured by child windows, so this won't work. 
For example, an {\et xterm} window has a single child, which is the id
to use.  In server mode, a cell must be loaded for editing with the
{\vt Edit} function.  Then, a {\vt Display} command can be given,
something like
\begin{quote}\vt
    Display(":0", 0x1800015, -100, -100, 100, 100)
\end{quote}

The {\vt ":0"} is the display name for the local machine, assuming
that the {\Xic} server is also running on this machine.  In general,
this is the same as the {\et DISPLAY} environment variable, in the
form {\it hostname\/}{\vt :0}.  The second argument is the window id
returned from {\et xwininfo}.  The remaining arguments set the area to
display.  After giving the command, the window should be overwritten
with a display similar to a drawing window in {\Xic}.  However, if the
window is redrawn, it will revert to its previous contents.  The user
must set up expose event handling in a real application.  The
suggested way to do this is to pass the id of a pixmap to {\Xic}, and
then copy the pixmap to the destination window.  This is usually
faster than a direct write, and the pixmap can be used for backing
store for expose events.

%------------------------------------
% 030104
\index{FreezeDisplay function}
\item{(int) \vt FreezeDisplay({\it freeze\/})}\\
When this function is called with a nonzero argument, the graphical
display in the drawing windows will be frozen until a subsequent call
of this function with a zero argument, or the script terminates.  This
is useful for speeding execution, and eliminating distracting screen
drawing while a script is running.  When the function is called with a
zero argument, all drawing windows are refreshed.

%------------------------------------
% 030104
\index{Redraw function}
\item{(int) \vt Redraw({\it win\/})}\\
This function will redraw the window indicated by the argument, which
is 0 for the main window or 1--4 for the sub-windows.  The function
returns 0 if the argument does not correspond to an existing window, 1
otherwise.

\end{description}


\subsection{Exit}

\begin{description}
%------------------------------------
% 030104
\index{Exit function}
\item{\vt Exit()}\\
Calling this function terminates execution of the script.

%------------------------------------
% 030104
\index{Halt function}
\item{\vt Halt()}\\
Calling this function terminates execution of the script, equivalent
to {\vt Exit}.

\end{description}


\subsection{Annotation}

\begin{description}
%------------------------------------
% 120809
\index{AddMark function}
\item{(int) \vt AddMark({\it type\/}, {\it arguments\/} ...)}\\
This function will add a ``user mark'' to a display list, which is
rendered as highlighting in the current cell.  These can be used for
illustrative purposes.  The marks are not included in the design
database, but are persistent to the current cell and are remembered as
long as the current cell exists in memory.  Any call can have
associated marks, whether electrical or physical.  Marks are shown in
any window displaying the cell as the top level.  Marks are not shown
in expanded subcells.

The arguments that follow the type argument vary depending upon the
type.  The type argument can be an integer code, or a string whose
first character signifies the type.  The return value, if nonzero, is
a unique mark id, which can be passed to {\vt EraseMark} to erase the
mark.  A zero return indicates that an error occurred.

The table below describes the marks available.  All coordinates and
dimensions are in microns, in the coordinate system of the current
cell.  Each mark takes an optional attribute argument, which is an
integer whose set bits indicate a display property.  These bits are

\begin{description}
\item{\bf bit 0:}
  Draw with a textured (dashed) line if set, otherwise use a solid line.
\item{\bf bit 1:}
  Cause the mark to blink, using the selection colors.
\item{\bf bit 2:}
  Render the mark in an alternate color (bit 1 is ignored).
\end{description}

\begin{description}
\item{Type: 1 or {\vt "l"}\\
Arguments: {\it x1\/}, {\it y1\/}, {\it x2\/}, {\it y2\/}
[, {\it attribute\/}]}\\
  Draw a line segment from {\it x1\/},{\it y1\/} to {\it x2\/},{\it y2\/}.

\item{Type: 2 or {\vt "b"}\\
Arguments: {\it l\/}, {\it b\/}, {\it r\/}, {\it t\/} [, {\it attribute\/}]}\\
  Draw an open box, {\it l\/},{\it b\/} is lower-left corner and
 {\it r\/},{\it t\/} is upper-right corner.

\item{Type: 3 or {\vt "u"}\\
Arguments: {\it xl\/}, {\it xr\/}, {\it yb\/} [, {\it yt\/},
{\it attribute\/}]}\\
  Draw an open triangle.  The two base vertices are {\it xl\/},{\it
  yb\/} and {\it xr\/},{\it yb\/}.  The third vertex is ({\it
  xl\/}+{\it xr\/})/2,{\it yt\/}.  If {\it yt\/} is not given, it is
  set to make the triangle equilateral.

\item{Type: 4 or {\vt "t"}\\
Arguments: {\it yl\/}, {\it yu\/}, {\it xb\/} [, {\it xt\/},
{\it attribute\/}]}\\
  Draw an open triangle.  The two lower vertices are {\it xb\/},{\it
  yl\/} and {\it xb\/},{\it yu\/}.  The third vertex is {\it
  xt\/},({\it yl\/}+{\it yu\/})/2.  If {\it xt\/} is not given, it is
  set to make the triangle equilateral.

\item{Type: 5 or {\vt "c"}\\
Arguments: {\it xc\/}, {\it yc\/}, {\it rad\/} [, {\it attribute\/}]}\\
  Draw a circle of radius {\it rad\/} centered at {\it xc\/},{\it yc\/}.

\item{Type: 6 or {\vt "e"}\\
Arguments: {\it xc\/}, {\it yc\/}, {\it rx\/}, {\it ry\/}
[, {\it attribute\/}]}\\
  Draw an ellipse centered at xc,yc using radii rx and ry.

\item{Type: 7 or {\vt "p"}\\
Arguments: {\it numverts\/}, {\it xy\_array\/} [, {\it attribute\/}]}\\
  Draw an open polygon or path.  The number of vertices is given
  first, followed by an array of size {\vt 2*}{\it numverts} or larger
  that contains the vertex coordinates as x-y pairs.  For a polygon,
  The vertex list should be closed, i.e., the first and last vertices
  listed (and counted) should be the same.

\item{Type: 8 or {\vt "s"}\\
Arguments: {\it string\/}, {\it x\/}, {\it y\/} [, {\it width\/},
    {\it height\/}, {\it xform\/}, {\it attribute\/}]}\\
  Draw a text string.  The string is followed by the coordinates of
  the reference point, which for default justification is the
  lower-left corner of the bounding box.  The {\it width\/}, {\it
  height\/}, and {\it xform\/} arguments are analogous to those of the
  {\vt Label} script function, providing the rendering size and
  justification and transformation information.  Unlike the {\vt
  Label} function, the settings of the {\vt Justify} and {\vt
  UseTransform} functions are ignored, transformation and
  justification must be set through the {\it xform} argument.
\end{description}

%------------------------------------
% 041705
\index{EraseMark function}
\item{(int) \vt EraseMark({\it id\/})}\\
Remove a mark from the ``user marks'' display list.  The argument is
the id number returned from {\vt AddMark}.  If zero is passed instead,
all marks will be erased.  The return value is 1 if any marks were
erased.

%------------------------------------
% 120909
\index{DumpMarks function}
\item{(int) \vt DumpMarks({\it filename\/})}\\
This function will save the marks currently defined in the current
cell to a file.  If the argument is null or empty (or scalar 0), a
file name will be composed:  {\it cellname\/}.{\it mode\/}.{\vt
marks}, where {\it mode} is ``{\vt phys}'' or ``{\vt elec}''.  The
return is the number of marks written, or -1 if error.  On error, a
message may be available from {\vt GetError}.  If 0, no file was
produced, as no marks were found.

%------------------------------------
% 120909
\index{ReadMarks function}
\item{(int) \vt ReadMarks({\it filename\/})}\\
This function will read the marks found in a file into the current
cell.  The file must be in the format produced by {\vt DumpMarks}, and
apply to the same name and display mode as the current cell.  A null
or empty or 0 argument will imply a cell name composed as described
for {\vt DumpMarks} The return value is the number of marks read, or
-1 if error.  On error, a message may be available from {\vt
GetEreror}.

\end{description}


\subsection{Ghost Rendering}

% 030104
The {\vt PushGhost/PopGhost} functions are useful in scripts where an
object is created, and the user must click to place the object.  The
object's outline can be drawn and attached to the pointer,
facilitating placement.  Example:

\begin{quote}
\begin{verbatim}
array[2000]
# create some shape in array, nverts is actual size
...
ShowPrompt("Click to locate new object");
xy[2]
PushGhost(array, nverts)
ShowGhost(8)
if !Point(xy)
    Exit()
end
ShowGhost(0)
PopGhost()
# use xy to create object in database
\end{verbatim}
\end{quote}

\begin{description}
%------------------------------------
% 030104
\index{PushGhost function}
\item{(int) \vt PushGhost({\it array\/}, {\it numpts\/})}\\
This function allows a polygon to be added to the list of polygons
used for dynamic highlighting with the {\vt ShowGhost} function.  The
outline of the polygon will be ``attached'' to the mouse pointer.  The
return value is the number of polygons in the list, after the present
one is added.  The {\it array} is an array of x-y values forming the
polygon.  The {\it numpts} value is the number of x-y pairs that
constitute the polygon.  If this value is less than 2 or greater than
the real size of the array, the real size of the array will be
assumed.  The second argument is useful when the polygon data do not
entirely fill the array, and can be set to 0 otherwise.

%------------------------------------
% 030104
\index{PushGhostBox function}
\item{(int) \vt PushGhostBox({\it left\/}, {\it bottom\/}, {\it right\/},
{\it top\/})}\\
This function is similar to {\vt PushGhost}.  It allows a box outline
to be added to the list of polygons used for ghosting with the {\vt
ShowGhost} function.  The outline of the box will be ``attached'' to
the mouse pointer.  The return value is the number of polygons in the
list, after the present one is added.  The arguments are the
coordinates of the lower left and upper right corners of the box,
where ``0'' is the point attached to the mouse pointer.  The {\vt
PopGhost} function is used to remove the most recently added object
from the list.

%------------------------------------
% 092915
\index{PushGhostH function}
\item{(int) \vt PushGhostH({\it object\_handle\/}, {\it all\/})}\\
Push the outline of the figure referenced by the handle onto the ghost
list.  If boolean {\it all} is true, push all objects in the list
represented by the handle, otherwise push the single object at the
head of the list.  The return value is an integer count of the number
of outlines added to the ghost list.

%------------------------------------
% 030104
\index{PopGhost function}
\item{(int) \vt PopGhost()}\\
This function removes the last ghosting polygon passed to {\vt
PushGhost} or {\vt PushGhostBox} from the internal list, and returns
the number of polygons remaining in the list.

%------------------------------------
% 102913
\index{ShowGhost function}
\item{(int) \vt ShowGhost({\it type})}\\
Show dynamic highlighting.  This function turns on/off the ghosting,
i.e., the display of certain features which are ``attached'' to the
mouse pointer.  The argument is one of the numeric codes from the
table below.

\begin{tabular}{ll}
0 & Turn off ghosting\\
1 & full-screen horiz line, snapped to grid\\
2 & full-screen vert line, snapped to grid\\
3 & full-screen horiz line, not snapped\\
4 & full-screen vert line, not snapped\\
5 & vector from last point location to pointer\\
6 & box, snapped to grid\\
7 & box, not snapped\\
8 & display polygon list from {\vt PushGhost}\\
9 & vector from last point location to pointer\\
10 & vector from last point location to pointer\\
11 & vector from last point location to pointer\\
\end{tabular}

The modes 5, 9, 10, and 11 draw a vector from the last button 1 down
location to the pointer.  Mode 5 snaps to the grid, and snaps the
angle to multiples of 45 degrees when the angle is close.  If the {\et
Constrain45} variable is set, the angle is strictly constrained to
multiples of 45 degrees.  Mode 9 is similar, but does not snap to
grid.  Mode 10 is similar, but there are no angle constraints, except
that implicit in snapping to the grid.  Mode 11 is similar, but there
are no angle constraints and no grid snapping.

With the ghosting enabled, the {\vt Point} function returns
coordinates that are snapped to grid or not depending on the mode
passed to {\vt ShowGhost}.  Modes 1, 2, 5, 6, 8, and 10 are snapped to
grid.

If the {\vt UseTransform} function has been called to enable use of
the current transform, the current transform will be applied to the
displayed objects when using mode 8.  The translation supplied to {\vt
UseTransform} is ignored (the translation tracks the mouse pointer).

\end{description}


\subsection{Graphics}

The following functions represent an interface for exporting graphics
to a ``foreign'' X window.  In particular, the interface can be used
to draw into a window owned by a {\et Tk} script.  This interface is
not available on Microsoft Windows.

\begin{description}
%------------------------------------
% 030104
\index{GRopen function}
\item{(handle) \vt GRopen({\it display}, {\it window\/})}\\
This function returns a handle to a graphical interface that can be
used to export graphics to a foreign X window, possibly on another
machine.  The first argument is the X display string, corresponding to
the server which owns the target window.  The second argument is the X
window id of the target window to which graphics rendering is to be
exported.  If all goes well, and the user has permission to access the
window, a positive integer handle is returned.  If the open fails, 0
is returned.  The handle should be closed with the {\vt Close}
function when done.

%------------------------------------
% 030104
\index{GRcheckError function}
\item{(int) \vt GRcheckError()}\\
This function returns 1 if the previous operation by any of the GR
interface functions caused an X error, 0 otherwise.

%------------------------------------
% 030104
\index{GRcreatePixmap function}
\item{(drawable) \vt GRcreatePixmap({\it handle}, {\it width},
  {\it height\/})}\\
This function returns the X id of a new pixmap.  The first argument is
a handle returned from {\vt GRopen}.  The remaining arguments set the
size of the pixmap.  If the operation fails, 0 is returned.

%------------------------------------
% 030104
\index{GRdestroyPixmap function}
\item{(int) \vt GRdestroyPixmap({\it handle}, {\it pixmap\/})}\\
This function destroys a pixmap created with {\vt GRcreatePixmap}. 
The first argument is a handle returned from {\vt GRopen}.  The second
argument is the pixmap id returned from {\vt GRcreatePixmap}.  The
function returns 1 on success, 0 if there was an error.

%------------------------------------
% 030104
\index{GRcopyDrawable function}
\item{(int) \vt GRcopyDrawable({\it handle}, {\it dst}, {\it src},
  {\it xs}, {\it ys}, {\it ws}, {\it hs}, {\it x}, {\it y\/})}\\
This function is used to copy area between drawables, which can be
windows or pixmaps.  The first argument is a handle returned from {\vt
GRopen}.  The next two arguments are the ids of destination and source
drawables.  The area copied in the source drawable is given by the
next four arguments.  The coordinates are pixel values, with the
origin in the upper left corner.  If these four values are all zero,
the entire source drawable is understood.  The final two values give
the upper left corner of the copied-to area in the destination
drawable.

%------------------------------------
% 030104
\index{GRdraw function}
\item{(int) \vt GRdraw({\it handle}, {\it l}, {\it b}, {\it r},
  {\it t\/})}\\
This function renders an {\Xic} cell.  The first argument is a handle
returned from {\vt GRopen}.  The remaining arguments are the
coordinates of the cell to render, in microns.  The action is the same
as the {\vt Display} function.  The function returns 1 on success, 0
if there was an error.

%------------------------------------
% 030104
\index{GRgetDrawableSize function}
\item{(int) \vt GRgetDrawableSize({\it handle}, {\it drawable},
  {\it array\/})}\\
This function returns the size, in pixels, of a drawable.  The first
argument is a handle returned from {\vt GRopen}.  The second argument
is the id of a window or pixmap.  The third argument is an array of
size two or larger that will contain the pixel width and height of the
drawable.  Upon success, 1 is returned, and the array values are set,
otherwise 0 is returned.  The width is in the 0'th array element.

%------------------------------------
% 030104
\index{GRresetDrawable function}
\item{(drawable) \vt GRresetDrawable({\it handle}, {\it drawable\/})}\\
This function allows the target window of the graphical context to be
changed.  Then, the rendering functions will draw into the new window
or pixmap, rather than the one passed to {\vt GRopen}.  The return
value is the previous drawable id, or 0 if there is an error.

%------------------------------------
% 030104
\index{GRclear function}
\item{(int) \vt GRclear({\it handle\/})}\\
This function clears the window.  The argument is a handle returned
from {\vt GRopen}.  Upon success, 1 is returned, otherwise 0 is
returned.

%------------------------------------
% 030104
\index{GRpixel function}
\item{(int) \vt GRpixel({\it handle}, {\it x}, {\it y\/})}\\
This function draws a single pixel at the pixel coordinates given in
the second and third arguments, using the current color.  The first
argument is a handle returned from {\vt GRopen}.  Upon success, 1 is
returned, otherwise 0 is returned.

%------------------------------------
% 030104
\index{GRpixels function}
\item{(int) \vt GRpixels({\it handle}, {\it array}, {\it num\/})}\\
This function will draw multiple pixels using the current color.  The
first argument is a handle returned from {\vt GRopen}.  The second
argument is an array of pixel coordinates, taken as x-y pairs.  The
third argument is the number of pixels to draw (half the length of the
array).  Upon success, 1 is returned, otherwise 0 is returned.

%------------------------------------
% 030104
\index{GRline function}
\item{(int) \vt GRline({\it handle}, {\it x1}, {\it y1}, {\it x2},
  {\it y2\/})}\\
This function renders a line using the current color and line style. 
The first argument is a handle returned from {\vt GRopen}.  The next
four arguments are the endpoints of the line in pixel coordinates. 
Upon success, 1 is returned, otherwise 0 is returned.

%------------------------------------
% 030104
\index{GRpolyLine function}
\item{(int) \vt GRpolyLine({\it handle}, {\it array}, {\it num\/})}\\
This function renders a polyline in the current color and line style. 
The first argument is a handle returned from {\vt GRopen}.  The second
argument is an array containing vertex coordinates in pixels as x-y
pairs.  The line will be continued to each successive vertex.  The
third argument is the number of vertices (half the length of the
array).  Upon success, 1 is returned, otherwise 0 is returned.

%------------------------------------
% 030104
\index{GRlines function}
\item{(int) \vt GRlines({\it handle}, {\it array}, {\it num\/})}\\
This function renders multiple distinct lines, each using the current
color and line style.  The first argument is a handle returned by {\vt
GRopen}.  The second argument is an array of coordinates, in pixels,
which if taken four at a time give the x-y endpoints of each line. 
The third argument is the number of lines in the array (one fourth the
array length).  Upon success, 1 is returned, otherwise 0 is returned.

%------------------------------------
% 030104
\index{GRbox function}
\item{(int) \vt GRbox({\it handle}, {\it l}, {\it b}, {\it r},
  {\it t\/})}\\
This function renders a rectangular area in the current color with the
current fill pattern.  The first argument is a handle returned from
{\vt GRopen}.  The remaining arguments provide the diagonal vertices
of the rectangle, in pixels.  Upon success, 1 is returned, otherwise 0
is returned.

%------------------------------------
% 030104
\index{GRboxes function}
\item{(int) \vt GRboxes({\it handle}, {\it array}, {\it num\/})}\\
This function renders multiple rectangles, each using the current
color and fill pattern.  The first argument is a handle returned from
{\vt GRopen}.  the second argument is an array of pixel coordinates
which specify the boxes.  Taken four at a time, the values are the
upper-left corner (x-y), width, and height.  The third argument is the
number of boxes represented in the array (one fourth the array
length).  Upon success, 1 is returned, otherwise 0 is returned.

%------------------------------------
% 120809
\index{GRarc function}
\item{(int) \vt GRarc({\it handle}, {\it x0}, {\it y0}, {\it rx},
  {\it ry\/}, {\it theta1}, {\it theta2\/})}\\
This function renders an arc, using the current color and line style. 
The first argument is a handle returned from {\vt GRopen}.  The next
two arguments are the pixel coordinates of the center of the ellipse
containing the arc.  The remaining arguments are the x and y radii,
and the starting and ending angles.  The angles are in radians,
relative to the three-o'clock position, counter-clockwise.  Upon
success, 1 is returned, otherwise 0 is returned.

%------------------------------------
% 030104
\index{GRpolygon function}
\item{(int) \vt GRpolygon({\it handle}, {\it array}, {\it num\/})}\\
This function renders a polygon, using the current color and fill
pattern.  The first argument is a handle returned from {\vt GRopen}. 
The second argument is an array containing the vertices, as x-y pairs
of pixel coordinates.  The third argument is the number of vertices
(half the length of the array).  The polygon will be closed
automatically if the first and last vertices do not coincide.  Upon
success, 1 is returned, otherwise 0 is returned.

%------------------------------------
% 022713
\index{GRtext function}
\item{(int) \vt GRtext({\it handle}, {\it text}, {\it x}, {\it y},
  {\it flags\/})}\\
This function renders text in the current color.  The first argument
is a handle returned form {\vt GRopen}.  The second argument is the
text string to render.  The next two arguments give the anchor point
in pixel coordinates.  If there is no transformation, this will be the
lower-left of the bounding box of the rendered text.  The {\it flags}
argument specifies a label flags word as used in {\Xic} (see
\ref{labelflags}).  Only the bits of the least significant byte are
likely to be recognized.

Upon success, 1 is returned, otherwise 0 is returned.

%------------------------------------
% 030104
\index{GRtextExtent function}
\item{(int) \vt GRtextExtent({\it handle}, {\it text}, {\it array\/})}\\
This function returns the width and height in pixels needed to render
a text string.  The first argument is a handle returned from {\vt
GRopen}.  The second argument is the string to measure.  If the string
is null or empty, a ``typical'' single character width and height is
returned, which can be simply multiplied for the fixed-pitch font in
use.  The third argument is an array of size two or larger which will
receive the width (0'th index) and height.  The function returns 1 on
success, 0 otherwise.

%------------------------------------
% 030104
\index{GRdefineColor function}
\item{(int) \vt GRdefineColor({\it handle}, {\it red}, {\it green},
  {\it blue\/})}\\
This function will return a color code corresponding to the given
color.  The first argument is a handle returned from {\vt GRopen}. 
The next three arguments are color component values, each in a range
0--255, giving the red, green, and blue intensity.  The return value
is a color code representing the nearest displayable color to that
given.  If an error occurs, 0 (black) is returned.  The returned color
code can be passed to {\vt GRsetColor} to actually change the drawing
color.

%------------------------------------
% 030104
\index{GRsetBackground function}
\item{(int) \vt GRsetBackground({\it handle}, {\it pixel\/})}\\
This function sets the default background color assumed by the
graphics context.  The first argument is a handle returned from {\vt
GRopen}.  The second argument is a color code returned from {\vt
GRdefineColor}.  Upon success, 1 is returned, otherwise 0 is returned.

%------------------------------------
% 030104
\index{GRsetWindowBackground function}
\item{(int) \vt GRsetWindowBackground({\it handle}, {\it pixel\/})}\\
This function sets the color used to render the window background when
the window is cleared.  The first argument is a handle returned from
{\vt GRopen}.  The second argument is a color code returned from {\vt
GRdefineColor}.  The function returns 1 on success, 0 otherwise.

%------------------------------------
% 030104
\index{GRsetColor function}
\item{(int) \vt GRsetColor({\it handle}, {\it pixel\/})}\\
This function sets the current color, used for all rendering
functions.  The first argument is a handle returned from {\vt GRopen}. 
The second argument is a color code returned from {\vt GRdefineColor}. 
Upon success, 1 is returned, otherwise 0 is returned.

%------------------------------------
% 030104
\index{GRdefineLinestyle function}
\item{(int) \vt GRdefineLinestyle({\it handle}, {\it index},
  {\it mask\/})}\\
This function defines a line style.  The first argument is a handle
returned from {\vt GRopen}.  The second argument is an index value
1--15 which corresponds to an internal line style register.  The third
argument is an integer value whose bits set the line on/off pattern. 
the pattern starts with the most significant '1' bit in the {\it
mask}.  The '1' bits will be drawn.  The pattern continues to the
least significant bit, and is repeated as the line is rendered.  The
indices 1--10 contain pre-defined line styles, which can be
overwritten with this function.  The {\vt SetLinestyle} function is
used to set the pattern actually used for rendering.  Upon success, 1
is returned, otherwise 0 is returned.

%------------------------------------
% 030104
\index{GRsetLinestyle function}
\item{(int) \vt GRsetLinestyle({\it handle}, {\it index\/})}\\
This function sets the line style used to render lines.  The first
argument is a handle returned from {\vt GRopen}.  The second argument
is an integer 0--15 which corresponds to an internal style register. 
Index 0 is always solid, whereas the other values can be set with {\vt
GRdefineLinestyle}.  The function returns 1 on success, 0 otherwise.

%------------------------------------
% 030104
\index{GRdefineFillpattern function}
\item{(int) \vt GRdefineFillpattern({\it handle}, {\it index},
  {\it nx}, {\it ny} {\it array\_string\/})}\\
This function is used to define a fill pattern for rendering boxes and
polygons.  The first argument is a handle returned from {\vt GRopen}. 
The second argument is an integer 1--15 which corresponds to internal
fill pattern registers.  The next two arguments set the x and y size
of the pixel map used for the fill pattern.  These can take values of
8 or 16 only.  The final argument is a character string which contains
the pixel map.  The most significant bit of the first byte is the
upper left corner of the map.  The {\vt SetFillpattern} function is
used to set the fill pattern actually used for rendering.  The
function returns 1 on success, 0 otherwise.

%------------------------------------
% 030104
\index{GRsetFillpattern function}
\item{(int) \vt GRsetFillpattern({\it handle}, {\it index\/})}\\
This function sets the fill pattern used for rendering boxes and
polygons.  The first argument is a handle returned from {\vt GRopen}. 
The second argument is an integer index 0--15 which corresponds to
internal fill pattern registers.  The value 0 is always solid fill. 
The other values can be set with {\vt GRdefineFillpattern}.  Upon
success, 1 is returned, otherwise 0 is returned.

%------------------------------------
% 030104
\index{GRupdate function}
\item{(int) \vt GRupdate({\it handle\/})}\\
This function flushes the X queue and causes any pending operations to
be performed.  This should be called after completing a sequence of
drawing functions, to force a screen update.  Upon success, 1 is
returned, otherwise 0 is returned.

%------------------------------------
% 030104
\index{GRsetMode function}
\item{(int) \vt GRsetMode({\it handle}, {\it mode\/})}\\
This function sets the drawing mode used for rendering.  The first
argument is a handle returned from {\vt GRopen}.  The second argument
is one of the following:

\begin{tabular}{ll}
0 & normal drawing\\
1 & XOR\\
2 & OR\\
3 & AND-inverted\\
\end{tabular}

Modes 2,3 are probably not useful on other than 8-plane displays. 
The function returns 1 on success, 0 otherwise.

\end{description}


\subsection{Hard Copy}

The following functions provide an interface for plot and graphical
file output.  This is completely outside of the normal printing
interface.

\begin{description}
%------------------------------------
% 030104
\index{HClistDrivers function}
\item{(stringlist\_handle) \vt HClistDrivers()}\\
This function returns a handle to a list of available printer drivers. 
The returned handle can be processed by any of the functions that
operate on stringlist handles.

%------------------------------------
% 030104
\index{HCsetDriver function}
\item{(int) \vt HCsetDriver({\it driver\/})}\\
This function will set the current print driver to the name passed (as
a string).  The name must be one of the internal driver names as
returned from {\vt HClistDrivers}.  If the operation succeeds, the
function returns 1, otherwise 0 is returned.

%------------------------------------
% 030104
\index{HCgetDriver function}
\item{(string) \vt HCgetDriver()}\\
This function returns the internal name of the current driver.  If no
driver has been set, a null string is returned.

%------------------------------------
% 030104
\index{HCsetResol function}
\item{(int) \vt HCsetResol({\it resol\/})}\\
This function will set the resolution of the current driver to the
value passed.  The scalar argument should be one of the values
supported by the driver, as returned from {\vt HCgetResols}.  If the
resolution is set successfully, 1 is returned.  If no driver has been
set, or the driver does not support the given resolution, 0 is
returned.

%------------------------------------
% 030104
\index{HCgetResol function}
\item{(int) \vt HCgetResol()}\\
This function returns the resolution set for the current driver, or 0
if no driver has been set or the driver does not provide settable
resolutions.

%------------------------------------
% 030104
\index{HCgetResols function}
\item{(int) \vt HCgetResols({\it array\/})}\\
This function sets the array values to the resolutions supported by
the current driver.  The array must have size 8 or larger.  The return
value is the number of resolutions supported.  If no driver has been
set, or the driver has fixed resolution, 0 is returned.

%------------------------------------
% 030104
\index{HCsetBestFit function}
\item{(int) \vt HCsetBestFit({\it best\_fit\/})}\\
This function will set or reset the ``best fit'' flag for the current
driver.  In best fit mode, the image will be rotated 90 degrees if
this is a better match to the aspect ratio of the rendering area.  If
the operation succeeds, 1 is returned.  If there is no driver set or
the driver does not allow best fit mode, 0 is returned.  If the
argument is nonzero, best fit mode will be set if possible, otherwise
the mode is unset.

%------------------------------------
% 030104
\index{HCgetBestFit function}
\item{(int) \vt HCgetBestFit()}\\
This function returns 1 if the current driver is in ``best fit'' mode,
0 otherwise.

%------------------------------------
% 030104
\index{HCsetLegend function}
\item{(int) \vt HCsetLegend({\it legend\/})}\\
This function will set or reset the ``legend'' flag for the current
driver.  If set, a legend will be shown with the rendered image.  If
the operation succeeds, 1 is returned.  If there is no driver set or
the driver does not allow a legend, 0 is returned.  If the argument is
nonzero, the legend mode will be set if possible, otherwise the mode
is unset.

%------------------------------------
% 030104
\index{HCgetLegend function}
\item{(int) \vt HCgetLegend()}\\
This function returns 1 if the current driver has the ``legend'' mode
set, 0 otherwise.

%------------------------------------
% 030104
\index{HCsetLandscape function}
\item{(int) \vt HCsetLandscape({\it landscape\/})}\\
This function will set or reset the ``landscape'' flag for the current
driver.  If set, the image will be rotated 90 degrees.  If the
operation succeeds, 1 is returned.  If there is no driver set or the
driver does not allow landscape mode, 0 is returned.  If the argument
is nonzero, the landscape mode will be set if possible, otherwise the
mode is unset.

%------------------------------------
% 030104
\index{HCgetLandscape function}
\item{(int) \vt HCgetLandscape()}\\
This function returns 1 if the current driver has the ``landscape''
mode set, 0 otherwise.

%------------------------------------
% 030104
\index{HCsetMetric function}
\item{(int) \vt HCsetMetric({\it metric\/})}\\
This function sets a flag in the current driver which indicates that
the rendering area is given in millimeters.  If not set, the values
are taken in inches.  This pertains to the values passed to the {\vt
HCsetSize} function.  If the operation succeeds, 1 is returned.  If
there is no driver set, 0 is returned.  If the argument is nonzero,
the metric mode will be set if possible, otherwise the mode is unset.

%------------------------------------
% 030104
\index{HCgetMetric function}
\item{(int) \vt HCgetMetric()}\\
This function returns 1 if the current driver has the ``metric'' mode
set, 0 otherwise.

%------------------------------------
% 030104
\index{HCsetSize function}
\item{(int) \vt HCsetSize({\it x}, {\it y}, {\it w}, {\it h\/})}\\
This function sets the size and offset of the rendering area.  The
numbers correspond to the entries in the {\cb Print Control Panel}. 
The values are scalars, in inches unless metric mode is in effect
(with {\vt HCsetMetric}) in which case the values are in millimeters. 
The values are clipped to the limits provided in the technology file. 
Most drivers accept 0 for one of {\it w}, {\it h}, indicating auto
dimensioning mode.  The function returns 1 on success, 0 if no driver
has been set.  Not all drivers use all four parameters, unused
parameters are ignored.

%------------------------------------
% 030104
\index{HCgetSize function}
\item{(int) \vt HCgetSize({\it array\/})}\\
This function returns the rendering area parameters for the current
driver.  The array argument must have size 4 or larger.  The values
are returned in the order x, y, w, h.  If the function succeeds, the
values are set in the array and 1 is returned.  Otherwise, 0 is
returned.

%------------------------------------
% 030104
\index{HCshowAxes function}
\item{(int) \vt HCshowAxes({\it style\/})}\\
This function sets the style or visibility of axes shown in plots of
physical data (electrical plots never include axes).  The argument is
an integer 0--2, where 0 suppresses drawing of axes, 1 indicates plain
axes, and 2 (or anything else) indicates axes with a box at the
origin.  The return value is the previous setting.

%------------------------------------
% 030104
\index{HCshowGrid function}
\item{(int) \vt HCshowGrid({\it show\/}, {\it mode\/})}\\
This function determines whether or not the grid is shown in plots. 
If the first argument is nonzero, the grid will be shown, otherwise
the grid will not be shown.  The second argument indicates the type of
data affected:  zero for physical data, nonzero for electrical data. 
The return value is the previous setting.

%------------------------------------
% 030104
\index{HCsetGridInterval function}
\item{(int) \vt HCsetGridInterval({\it spacing\/}, {\it mode\/})}\\
This function sets the grid spacing used in plots.  The first argument
is the interval in microns.  The second argument indicates the type of
data affected:  zero for physical data, nonzero for electrical data. 
For electrical data, the spacing in microns is rather meaningless,
except as being relative to the default which is 1.0.  The return
value is the previous setting.

%------------------------------------
% 030104
\index{HCsetGridStyle function}
\item{(int) \vt HCsetGridStyle({\it linemod\/}, {\it mode\/})}\\
This function sets the line style used for the grid lines in plots. 
The first argument is an integer mask that defines the on-off pattern. 
The pattern starts at the most significant `1' bit and continues
through the least significant bit, and repeats.  Set bits are rendered
as the visible part of the pattern.  If the style is 0, a dot is shown
at each grid point.  Passing -1 will give continuous lines.  The
second argument indicates the type of data affected:  zero for
physical data, nonzero for electrical data.  The return value is the
previous setting.

%------------------------------------
% 071110
\index{HCsetGridCrossSize function}
\item{(int) \vt HCsetGridCrossSize({\it xsize\/}, {\it mode\/})}\\
This applies only to grids with style 0 (dot grid).  The {\it xsize}
is an integer 0--6 which indicates the number of pixels to draw in the
four compass directions around the central pixel.  Thus, for nonzero
values, the ``dot'' is rendered as a small cross.  The second argument
indicates the type of data affected:  zero for physical data, nonzero
for electrical data.  The return value is 1 if the cross size was set,
0 if the grid style was nonzero in which case the cross size was not
set.

%------------------------------------
% 030104
\index{HCsetGridOnTop function}
\item{(int) \vt HCsetGridOnTop({\it on\_top\/}, {\it mode\/})}\\
This function sets whether the grid lines are drawn after the geometry
(``on top'') or before the geometry.  If the first argument is
nonzero, the grid will be rendered on top.  The second argument
indicates the type of data affected:  zero for physical data, nonzero
for electrical data.  The return value is the previous setting.

%------------------------------------
% 030104
\index{HCdump function}
\item{(int) \vt HCdump({\it l}, {\it b}, {\it r}, {\it t}, {\it filename},
  {\it command\/})}\\
This is the function which actually generates a plot or graphics file. 
The first four arguments set the area in microns in current cell
coordinates to render.  If these values are all 0, a full view of the
current cell will be rendered.  The next argument is the name of the
file to use for the graphical output.  If this string is null or
empty, a temporary file will be used.  Under Windows, the final
argument is the name of a printer, as known to the operating system. 
These names can be obtained with {\vt HClistPrinters}.  Under
Unix/Linux, the last argument is a command string that will be
executed to generate a plot.  In any case if this argument is null or
empty, the plot file will be generated, but no further action will be
taken.  In the command string, the character sequence ``{\vt \%s}''
will be replaced by the file name.  If the sequence does not appear,
the file name will be appended.  If successful, 1 is returned,
otherwise 0 is returned, and an error message can be obtained with
{\vt HCerrorString}.

The {\it filename}, or the temporary file that is used if no {\it
filename} is given, is {\it not} removed.  The user must remove the
file explicitly.

The Windows Native driver (Windows only) has slightly different
behavior.  For this driver, the command string must specify a printer
name, and can not be null or empty.  If {\it filename} is not null or
empty, the output goes to that file and is {\it not} sent to the
printer.  Otherwise, the output goes to the printer.

%------------------------------------
% 030104
\index{HCerrorString function}
\item{(int) \vt HCerrorString()}\\
This function returns a string indicating the error generated by {\vt
HCdump}.  If there were no errors, a null string is returned.

%------------------------------------
% 030104
\index{HClistPrinters function}
\item{(stringlist\_handle) \vt HClistPrinters()}\\
Under Microsoft Windows, this function returns a handle to a list of
printer names available from the current host.  The first name is the
name of the default printer.  The remaining names, alphabetized,
follow.  If there are no printers available, or if not running under
Windows, the function returns 0.  The returned names can be supplied
to the {\vt HCdump} function to initiate a print job.

%------------------------------------
% 030104
\index{HCmedia function}
\item{(int) \vt HCmedia()}\\
This function sets the media index, which is used by the Windows
Native driver under Microsoft Windows only.  The media index sets the
assumed paper size.  The argument is one of the integers from the
table below.  The page dimensions are in points (1/72 inch).
    
\begin{tabular}{llll}
\kb Index & \kb Name & \kb Width & \kb Height\\
0 &  Letter     & 612  & 792\\
1 &  Legal      & 612  & 1008\\
2 &  Tabloid    & 792  & 1224\\
3 &  Ledger     & 1224 & 792\\
4 &  10x14      & 720  & 1008\\
5 &  11x17      & 792  & 1224\\
6 &  12x18      & 864  & 1296\\
7 &  17x22 ``C''  & 1224 & 1584\\
8 &  18x24      & 1296 & 1728\\
9 &  22x34 ``D''  & 1584 & 2448\\
10 & 24x36      & 1728 & 2592\\
11 & 30x42      & 2160 & 3024\\
12 & 34x44 ``E''  & 2448 & 3168\\
13 & 36x48      & 2592 & 3456\\
14 & Statement  & 396  & 612\\
15 & Executive  & 540  & 720\\
16 & Folio      & 612  & 936\\
17 & Quarto     & 610  & 780\\
18 & A0         & 2384 & 3370\\
19 & A1         & 1684 & 2384\\
20 & A2         & 1190 & 1684\\
21 & A3         & 842  & 1190\\
22 & A4         & 595  & 842\\
23 & A5         & 420  & 595\\
24 & A6         & 298  & 420\\
25 & B0         & 2835 & 4008\\
26 & B1         & 2004 & 2835\\
27 & B2         & 1417 & 2004\\
28 & B3         & 1001 & 1417\\
29 & B4         & 729  & 1032\\
30 & B5         & 516  & 729\\
\end{tabular}
  
The returned value is the previous setting of the media index.

\end{description}


\subsection{Keyboard}

\begin{description}
%------------------------------------
% 011114
\index{ReadMapfile function}
\item{(int) \vt ReadMapfile({\it mapfile\/})}\\
Read and assert a keyboard mapping file, as generated from within
{\Xic} with the {\cb Key Map} button in the {\cb Attributes Menu}.  If
the {\it mapfile} is not rooted, it is searched for in the current
directory, the user's home directory, and in the library search path,
in that order.  If success, 1 is returned, and the supplied mapping is
installed.  Otherwise, 0 is returned, and an error message is
available from {\vt GetError}.
\end{description}


\subsection{Libraries}

\begin{description}
%------------------------------------
% 030104
\index{OpenLibrary function}
\item{(int) \vt OpenLibrary({\it path\_name\/})}\\
This function will open the named library.  The name is either a full
path to the library file, or the name of a library file to find in the
search path.  Zero is returned on error, nonzero on success.

%------------------------------------
% 030104
\index{CloseLibrary function}
\item{(int) \vt CloseLibrary({\it path\_name\/})}\\
This function will close the named library, or all user libraries if
the argument is null.  The {\it path\_name} can be a full path to a
previously opened library file, or just the file name.  This function
always returns 1.

\end{description}


\ifoa
\subsection{OpenAccess}

These functions provide an interface to the OpenAccess database.  An
OpenAccess exception triggered by these functions will generate a
fatal error, terminating the script.  The functions that return an
integer that is not an explicit boolean result always return 1.

\begin{description}
%------------------------------------
% 030316
\index{OaVersion function}
\item{(string) \vt OaVersion()}\\
Return the version string of the connected OpenAccess database.
If none, a null string is returned.

%------------------------------------
% 030316
\index{OaIsLibrary function}
\item{(int) \vt OaIsLibrary({\it libname\/})}\\
Return 1 if the library named in the string argument is known to
OpenAccess, 0 if not.

%------------------------------------
% 030316
\index{OaListLibraries function}
\item{(stringlist\_handle) \vt OaListLibraries()}\\
Return a handle to a list of library names known to OpenAccess.

%------------------------------------
% 030316
\index{OaListLibCells function}
\item{(stringlist\_handle) \vt OaListLibCells({\it libname\/})}\\
Return a list of the names of cells contained in the OpenAccess
library named in the argument.

%------------------------------------
% 030316
\index{OaListCellViews function}
\item{(stringlist\_handle) \vt OaListCellViews({\it libname\/},
 {\it cellname\/})}\\
Return a handle to a list of view names found for the given cell in
the given OpenAccess library.

%------------------------------------
% 030316
\index{OaIsLibOpen function}
\item{(int) \vt OaIsLibOpen({\it libname\/})}\\
Return 1 if the OpenAccess library named in the argument is open,  
0 otherwise.

%------------------------------------
% 022316
\index{OaOpenLibrary function}
\item{(int) \vt OaOpenLibrary({\it libname\/})}\\
Open the OpenAccess library of the given name, where the name should
match a library defined in the {\vt lib.defs} or {\vt cds.lib} file. 
A library being open means that it is available for resolving
undefined references when reading cell data in {\Xic}.  The return is
1 on success, 0 if error.

%------------------------------------
% 022316
\index{OaCloseLibrary function}
\item{(int) \vt OaCloseLibrary({\it libname\/})}\\
Close the OpenAccess library of the given name, where the name should
match a library defined in the {\vt lib.defs} or {\vt cds.lib} file. 
A library being open means that it is available for resolving
undefined references when reading cell data in {\Xic}.  The return is
1 on success, 0 if error.

%------------------------------------
% 030316
\index{OaIsOaCell function}
\item{(int) \vt OaIsOaCell({\it libname\/}, {\it open\_only\/})}\\
Return 1 if a cell with the given name can be resolved in an
OpenAccess library, 0 otherwise.  If the boolean value {\it
open\_only} is true, only open libraries are considered, otherwise all
libraries are considered.

%------------------------------------
% 030316
\index{OaIsCellInLib function}
\item{(int) \vt OaIsCellInLib({\it libname\/}, {\it cellname\/})}\\
Return 1 if the given cell can be found in the OpenAccess library
given as the first argument, 0 otherwise.

%------------------------------------
% 030316
\index{OaIsCellView function}
\item{(int) \vt OaIsCellView({\it cellname\/}, {\it viewname\/},
  {\it open\_only\/})}\\
Return 1 if the cellname and viewname resolve as a cellview in an
OpenAccess library, 0 otherwise.  If the boolean {\it open\_only} is
true, only open libraries are considered, otherwise all libraries are
considered.

%------------------------------------
% 030316
\index{OaIsCellViewInLib function}
\item{(int) \vt OaIsCellViewInLib({\it libname\/}, {\it cellname\/},
  {\it viewname\/})}\\
Return 1 is the cellname and viewname resolve as a cellview in the
given OpenAccess library, 0 otherwise.

%------------------------------------
% 030316
\index{OaCreateLibrary function}
\item{(int) \vt OaCreateLibrary({\it libname\/}, {\it techlibname\/})}\\
This will create the library in the OpenAccess database if {\it
libname} currently does not exist.  This will also set up the
technology for the new library if {\it techlibname} is given (not null
or empty).  The new library will attach to the same library as {\it
techlibname}, or will attach to {\it techlibname} if it has a local
tech database.  If {\it techlibname} is given then it must exist. 

%------------------------------------
% 030316
\index{OaBrandLibrary function}
\item{(int) \vt OaBrandLibrary({\it libname\/})}\\
Set or remove the {\Xic} ``brand'' of the given library.  {\Xic} can
only write to a branded library.  If the boolean {\it branded} is
true, the library will have its flag set, otherwise the branded status
is unset.

%------------------------------------
% 030316
\index{OaIsLibBranded function}
\item{(int) \vt OaIsLibBranded({\it libname\/})}\\
Return 1 if the named library is ``branded'' (writable by {\Xic}), 0
otherwise.

%------------------------------------
% 030316
\index{OaDestroy function}
\item{(int) \vt OaDestroy({\it libname\/}, {\it cellname\/},
  {\it viewname\/})}\\
Destroy the named view from the given cell in the given OpenAccess
library.  If the {\it viewname} is null or empty, destroy all views
from the named cell, i.e., the cell itself.  If the {\it cellname} is
null or empty, undefine the library in the library definition ({\vt
lib.defs} or {\vt cds.lib}) file, and change the directory name to
have a ``{\vt .defunct}'' extension.  We don't blow away the data, the
user can revert by hand, or delete the directory.

%------------------------------------
% 030416
\index{OaLoad function}
\item{(int) \vt OaLoad({\it libname\/}, {\it cellname\/})}\\
If {\it cellname} is null or empty, load all cells in the OpenAccess
library named in {\it libname} into {\Xic}.  The current cell is not
changed.  Otherwise, load the cell and its hierarchy and make it the
current cell.  Whether the physical or electrical views are read, or
both, is determined by the value of the {\et OaUseOnly} variable.  If
the value is ``1'' or starts with `p' or `P', only the physical
(layout) views are read.  If the value is ``2'' or starts with `e' or
`E', only the electrical (schematic and symbol) views are read.  If
anything else or not set, both physical and electrical views are read.

%------------------------------------
% 030316
\index{OaReset function}
\item{(int) \vt OaReset()}\\
There is a table in {\Xic} that records the cells that have been
loaded from OpenAccess.  This avoids the ``merge control'' pop-up
which appears if a common subcell was previously read and is already
in memory, the in-memory cell will not be overwritten.  This function
clears the table, and should be called if this protection should be
ended, for example if the {\Xic} database has been cleared.

%------------------------------------
% 030416
\index{OaSave function}
\item{(int) \vt OaSave({\it libname\/}, {\it allhier\/})}\\
Write the current cell to the OpenAccess library whose name is given
in the first argument.  This must exist, and be writable from {\Xic}. 
Whether the physical or electrical views are written, or both, is
determined by the value of the {\et OaUseOnly} variable.  If the value
is ``1'' or starts with `p' or `P', only the physical (layout) views
are written.  If the value is ``2'' or starts with `e' or `E', only
the electrical (schematic and symbol) views are written.  If anything
else or not set, both physical and electrical views are written.  The
second argument is a boolean that if true (nonzero) indicates that the
entire cell hierarchy under the current cell should be saved. 
Otherwise, only the current cell is saved.
 
The actual view names used are given in the {\et OaDefLayoutView},
{\et OaDefSchematicView}, and {\et OaDefSymbolView} variables, or
default to ``layout'', ``schematic'', and ``symbol''.

%------------------------------------
% 030316
\index{OaAttachTech function}
\item{(int) \vt OaAttachTech({\it libname\/}, {\it techlibname\/})}\\
If {\it techlibname} has an attached tech library, then that library
will be attached to {\it libname}.  If {\it techlibname} has a local
tech database, then {\it techlibname} itself will be attached to {\it
libname}.  This will fail if {\it libname} has a local tech database. 
The local database should be destroyed first.

%------------------------------------
% 030316
\index{OaGetAttachedTech function}
\item{(string) \vt OaGetAttachedTech({\it libname\/})}\\
Return the name of the OpenAccess library providing the attached
technology, or a null string if no attachment.

%------------------------------------
% 030316
\index{OaHasLocalTech function}
\item{(int) \vt OaHasLocalTech({\it libname\/})}\\
Return 1 if the OpenAccess library has a local technology database, 0
if not.

%------------------------------------
% 030316
\index{OaCreateLocalTech function}
\item{(int) \vt OaHasLocalTech({\it libname\/})}\\
If the library does not have an attached or local technology database,
create a new local database.

%------------------------------------
% 030316
\index{OaDestroyTech function}
\item{(int) \vt OaDestroyTech({\it libname\/}, {\it unattach\_only\/})}\\
If {\it libname} has an attached technology library, unattach it.  If
the boolean second argument is false, and the library has a local
database, destroy the database.
\end{description}
\fi


\subsection{Mode}

\begin{description}
%------------------------------------
% 030315
\index{Mode function}
\item{(int) \vt Mode({\it window}, {\it mode\/})}\\
This function switches {\Xic} between physical and electrical modes,
or switches sub-windows between the two viewing modes.  The first
argument is an integer 0--4, where 0 represents the main window, in
which case the application mode is set, and 1--4 represent the
sub-windows, in which case the viewing mode of that sub-window is set. 
The sub-window number is the same number as shown in the window title
bar.

The second argument can be a number or a string.  If a number and the
nearest integer is not zero, the mode is electrical, otherwise
physical.  If a string that starts with `{\vt e}' or `{\vt E}', the
mode is electrical, otherwise physical.

The return value is the new mode setting (0 or 1) or
-1 if the indicated sub-window is not active.

%------------------------------------
% 030104
\index{CurMode function}
\item{(int) \vt CurMode({\it window\/})}\\
This function returns the current mode (physical or electrical) of the
main window or sub-windows.  The argument is an integer 0--4 where 0
represents the main window (and the application mode) and 1--4
represent sub-window viewing modes.  The return value is 0 for physical
mode, 1 for electrical mode, or -1 if the indicated sub-window does not
exist.  This function is identical to {\vt GetWindowMode}.

\end{description}


\subsection{Prompt Line}

\begin{description}
%------------------------------------
% 030104
\index{StuffText function}
\item{(int) \vt StuffText({\it string\/})}\\
The {\vt StuffText} function stores the {\it string} in a buffer,
which will be retrieved into the edit line on the next call to an
editing function.  The edit will terminate immediately, as if the user
has typed {\it string\/}.  Multiple lines can be stuffed, and will be
retrieved in order.  This function must be issued before the function
which invokes the editor.  Once a ``stuffed'' line is used, it is
discarded.

%------------------------------------
% 030104
\index{TextCmd function}
\item{(int) \vt TextCmd({\it string\/})}\\
This executes the command in string as if it were one of the keyboard ``!''
commands in {\Xic}.  The leading ``!'' is optional.  Examples:\\
\begin{quote}
\begin{tabular}{ll}
  \vt TextCmd("!")            & brings up an xterm\\
  \vt TextCmd("set ho deedo") & sets variable `ho'\\
  \vt TextCmd("!select c")    & selects all subcells\\
\end{tabular}
\end{quote}

%------------------------------------
% 030104
\index{GetLastPrompt function}
\item{(int) \vt GetLastPrompt()}\\
This function returns the most recent message that was shown on the
prompt line, or would normally have been shown if {\Xic} is not in
graphics mode.  Although the prompt line may have been erased, the
last message is available until the next message is sent to the prompt
line.  The text on the prompt line while in edit mode is not saved and
is not accessible with this function.  An empty string is returned if
there is no current message.  This function never fails.

\end{description}


\subsection{Scripts}

\begin{description}
%------------------------------------
% 030104
\index{ListFunctions function}
\item{(stringlist\_handle) \vt ListFunctions()}\\
This function will re-read all of the {\vt library} files in the
script search path, and return a handle to a string list of the
functions available from the libraries.

%------------------------------------
% 030104
\index{Exec function}
\item{(untyped) \vt Exec({\it script\/})}\\
This function will execute a script.  The argument is a string giving
the script name or path.  If the script is a file, it must have a
``{\vt .scr}'' extension.  The ``{\vt .scr}'' extension is optional in
the argument.  If no path is given, the script will be opened from the
search path or from the internal list of scripts read from the
technology file or added with the {\cb !script} command.  If a path is
given, that file will be executed, if found.  It is also possible to
reference a script which appears in a sub-menu of the {\cb User Menu}
by giving a modified path of the form ``{\vt @@/{\it
libname}/.../{\it scriptname}}''.  The {\it libname} is the name of
the script menu, the ...  indicates more script menus if the menu is
more than one deep, and the last component is the name of the script. 

The return value is the result of the expression following ``return''
if a {\vt return} statement caused termination of the script being
executed.  If the script did not terminate with a {\vt return}
statement with a following expression, the integer 1 is returned by
{\vt Exec}.  If the script indicated by the argument to {\vt Exec}
could not be found, integer 0 is returned.  If the {\vt return}
statement is used, the type of the return is determined by the type of
object being returned.

Example:  script1.scr
\begin{quote}
({\it executable lines\/})\\
{\vt return 3}
\end{quote}
in main script:
\begin{quote}
{\vt Print(Exec("script1"))    \# prints "3"}
\end{quote}

%------------------------------------
% 030104
\index{SetKey function}
\item{(int) \vt SetKey({\it password\/})}\\
This function sets the key used by {\Xic} to decrypt encrypted
scripts.  The password must be the same as that used to encrypt the
scripts.  This function returns 1 on success, i.e., the key has been
set, or 0 on failure, which shouldn't happen as even an empty string
is a valid password.

%------------------------------------
% 021913
\index{HasPython function}
\item{(int) \vt HasPython()}\\
This function returns 1 if the Python language support plug-in has
been successfully loaded, 0 otherwise.

%------------------------------------
% 022813
\index{RunPython function}
\item{(int) \vt RunPython({\it command\/} [, {\it arg\/}, ...])}\\
Pass a command string to the Python interpreter for evaluation.  The
first argument is a path to a Python script file.  Arguments that
follow are concatenated and passed to the script.  Presently, only
string and scalar type arguments are accepted.  The interpreter will
have available the entire {\Xic} scripting interface, though only the
basic data types are useful.  The Python interface description
provides information about the header lines needed to instantiate the
interface to {\Xic} from Python (see \ref{pyplugin}).

This function exists only if the Python language support plug-in has
been successfully loaded.  The function returns 1 on success, 0
otherwise with an error message available from {\vt GetError}.

%------------------------------------
% 022813
\index{RunPythonModuleFunc function}
\item{(int) \vt RunPythonModuleFunc({\it module\/}, {\it function\/}
 [, {\it arg} ...])}\\
This function will call the Python interpreter, to execute the module
function specified in the arguments.  The first argument is the name
of the module, which must be known to Python.  The second argument is
the name of the function within the module to evaluate.  Following are
zero or more function arguments, as required by the function.

This function exists only if the Python language support plug-in has
been successfully loaded.  The function returns 1 on success, 0
otherwise with an error message available from {\vt GetError}.

%------------------------------------
% 100814
\index{ResetPython function}
\item{(int) \vt ResetPython()}\\
Reset the Python interpreter.  It is not clear that a user would ever
need to call this.

This function exists only if the Python language support plug-in has
been successfully loaded.  The function always returns 1.

%------------------------------------
% 021913
\index{HasTcl function}
\item{(int) \vt HasTcl()}\\
This function returns 1 if the Tcl language support plug-in was
successfully loaded, 0 otherwise.

%------------------------------------
% 021913
\index{HasTk function}
\item{(int) \vt HasTk()}\\
This function returns 1 if the Tcl with Tk language support plug-in
was successfully loaded, 0 otherwise.

%------------------------------------
% 022813
\index{RunTcl function}
\item{(int) \vt RunTcl({\it command\/} [, {\it arg} ...])}\\
Pass a command string to the Tcl interpreter for evaluation.  The
first argument is a path to a Tck/Tk script.  If both Tcl and Tk are
available, the script file must have a {\vt .tcl} or {\vt .tk}
extension.  If only Tcl is available, there is no extension
requirement, but the file should contain only Tcl commands.  A Tcl
script ie executed linearly and returns.  A Tk script blocks, handling
events until the last window is destroyed, at which time it returns.

Arguments that follow are concatenated and passed to the script. 
Presently, only string and scalar type arguments are accepted.  The
interpreter will have available the entire {\Xic} scripting interface,
though only the basic data types are useful.  The Tcl/Tk interface
description provides more information.

This function exists only if the Tcl language support plug-in has been
successfully loaded.  The function returns 1 on success, 0 otherwise
with an error message available from {\vt GetError}.

%------------------------------------
% 100814
\index{ResetTcl function}
\item{(int) \vt ResetTcl()}\\
Reset the Tcl/Tk interpreter.  It is not clear that a user would ever
need to call this.

This function exists only if the Tcl language support plug-in has been
successfully loaded.  The function always returns 1.

%------------------------------------
% 022813
\index{HasGlobalVariable function}
\item{(int) \vt HasGlobalVariable({\it globvar\/})}\\
Return true if the passed string is the name of a global variable
currently in scope.  This is part of the exported global variable
interface to Python and Tcl.

%------------------------------------
% 022813
\index{GetGlobalVariable function}
\item{(int) \vt GetGlobalVariable({\it globvar\/})}\\
Return the value of the global variable whose name is passed.  The
function will generate a fatal error, halting the script, if the
variable is not found, so one may need to check existence with {\vt
HasGlobalVariable}.  The return type is the type of the variable,
which can be any known type.  This is for use in Python or Tcl
scripts, providing access to the global variables maintained in the
{\Xic} script interpreter.

%------------------------------------
% 022813
\index{SetGlobalVariable function}
\item{(int) \vt SetGlobalVariable({\it globvar\/}, {\it value\/})}\\
Set the value of the global variable named in the first argument.  The
function will generate a fatal error if the variable is not found, or
the assignment fails due to type mismatch.  This is for use in Python
or Tcl scripts, providing access to the global variables maintained in
the {\Xic} script interpreter.  Note that global variables can not be
created from Python or Tcl, but values can be set with this function. 
Global variables can be used to return data to a top-level native
script from a Tcl or Python sub-script.
\end{description}


\subsection{Technology File}

\begin{description}
%------------------------------------
% 020611
\index{GetTechName function}
\item{\vt GetTechName()}\\
This returns a string containing the current technology name, as set
in the technology file with the {\vt Technology} keyword.

%------------------------------------
% 030104
\index{GetTechExt function}
\item{(string) \vt GetTechExt()}\\
This returns a string containing the current technology file name
extension.

%------------------------------------
% 030104
\index{SetTechExt function}
\item{(int) \vt SetTechExt({\it extension})}\\
This sets the current technology file extension to the string
argument.  It alters the name of new technology files created with the
{\cb Save Tech} button in the {\cb Attributes Menu}.

%------------------------------------
% 032712
\index{TechParseLine function}
\item{(int) \vt TechParseLine({\it line})}\\
This function will parse and process a line of text is if read from a
technology file.  It can therefor modify parameters that are otherwise
set in the technology file, after a technology file has been read, or
if no technology file was read.
  
However, there are limitations.
\begin{enumerate}
\item{There is no macro processing done on the line, it is parsed
verbatim, and macro directives will not be understood.}

\item{There is no line continuation, all related text must appear in
the given string.}

\item{The print driver block keywords are not recognized, nor are any
other block forms, such as device blocks for extraction.}

\item{Layer block keywords are acceptable, however they must be given
in a special format, which is
\begin{quote}
  [{\vt elec}]{\vt layer} {\it layername} {\it layer\_block\_line...}
\end{quote}
i.e., the text must be prefaced by the {\vt layer}/{\vt eleclayer}
keyword followed by an existing layer name.  Note that new layers must
be created first, before calling this function.}
\end{enumerate}

If the line is recognized and successfully processed, the function
returns 1.  Otherwise, 0 is returned, and a message is available from
{\vt GetError}.

%------------------------------------
% 021913
\index{TechGetFkeyString function}
\item{(int) \vt TechGetFkeyString({\it fkeynum})}\\
This function returns the string which encodes the functional
assignment of a function key.  This is the same format as used in the
technology file for the F1Key -- F12Key keyword assignments.  The
argument is an integer with value 1--12 representing the function key
number.  The return value is a null string if the argument is out of
range, or if no assignment has been made.

%------------------------------------
% 021913
\index{TechSetFkeyString function}
\item{(int) \vt TechSetFkeyString({\it fkeynum}, {\it string\/})}\\
This function sets the string which encodes the functional assignment
of a function key.  This is the same format as used in the technology
file for the F1Key -- F12Key keyword assignments.  The first argument
is an integer with value 1--12 representing the function key number. 
The second argument is the string, or 0 to clear the assignment.  The
return value is 1 if an assignment was made, 0 if the first argument
is out of range.
\end{description}


\subsection{Variables}

\begin{description}
%------------------------------------
% 113009
\index{Set function}
\item{\vt Set({\it name\/}, {\it string\/})}\\
The {\vt Set} function allows variable {\it name} to be set to {\it
string} as with the {\cb !set} keyboard operation in {\Xic}.  Some
variables, such as the search paths, directly affect {\Xic} operation. 
The {\vt Set} function can also set arbitrary variables, which may be
useful to the script programmer.  To set a variable, both arguments
should be strings.  If the second argument is the constant zero ({\vt
0} or {\vt NULL}, not {\vt "0"}) or a null (not empty) string, the
variable will be unset if set.  As with {\cb !set}, forms like \$({\it
name}) are expanded.  If {\it name} matches the name of a previously
set variable, that variable's value string replaces the form. 
Otherwise, if {\it name} matches an environment variable, the
environment variable text replaces the form.

The {\vt Set} function will permanently change the variable value. 
See the {\vt PushSet} function for an alternative.

%------------------------------------
% 030104
\index{Unset function}
\item{\vt Unset({\it name\/})}\\
This function will unset the variable.  No action is taken if the
variable is not already set.  This is equivalent to {\vt Set}({\it
name\/}, 0).

%------------------------------------
% 113009
\index{PushSet function}
\item{\vt PushSet({\it name\/}, {\it string\/})}\\
This function is similar to {\vt Set}, however the previous value is
stored internally, and can be restored with {\vt PopSet}.  In
addition, all variables set (or unset) with {\vt PushSet} are reverted
to original values when the script exits, thus avoiding permanent
changes.  There can be arbitrarily many {\vt PushSet} and {\vt PopSet}
operations on a variable.

%------------------------------------
% 113009
\index{PopSet function}
\item{\vt PopSet({\it name\/})}\\
This reverts a variable set with {\vt PushSet} to its previous state. 
If the variable has not been set (or unset) with {\vt PushSet}, no
action is taken.

%------------------------------------
% 030304
\index{SetExpand function}
\item{(string) \vt SetExpand({\it string\/}, {\it use\_env\/})}\\
This function returns a copy of {\it string} which expands variable
references in the form \$({\it word}) in {\it string}.  The {\it word}
is expected to be a variable previously set with the {\vt Set}
function or {\cb !set} command.  The value of the variable replaces
the reference in the returned string.  If the integer {\it use\_env}
is nonzero, variables found in the environment will also be
substituted.  If {\it word} is not resolved, no change is made. 
Otherwise, in general, the token is replaced with the value of {\it
word}.

There is an exception to the direct-substitution rule.  If any
substitution string is of the form ``{\vt (...)}'', then the
parentheses and leading/trailing white space are stripped before
substitution, and the entire substituted string is enclosed in
parentheses if it is not already.  This is for convenience when adding
a directory to a search path (see \ref{searchpaths}) variable, and the
path is enclosed in parentheses.  See the {\cb !set} command
description in \ref{setcmd} for more information.

%------------------------------------
% 030104
\index{Get function}
\item{(string) \vt Get({\it name})}\\
The {\vt Get} function returns a string containing the value of {\it
name\/}, which has been previously set with the {\vt Set} function, or
otherwise from within {\Xic}.  A null string is returned if
the named variable has not been set.

%------------------------------------
% 072904
\index{JoinLimits function}
\item{\vt JoinLimits({\it flag})}\\
This is a convenience function to set/unset the variables which
control the polygon joining process, i.e., {\et JoinMaxPolyVerts},
{\et JoinMaxPolyQueue}, and {\et JoinMaxPolyGroup}.  If the argument
is zero, each of these variables is set to zero, removing all limits. 
If the argument is nonzero, the variables are unset, meaning that the
default limits will be applied.  The default limits generally speed
processing, but will often leave unjoined joinable pieces when complex
polygons are constructed.  The status of the variables will persist
after the script terminates.  This function has no return value.

\end{description}


\subsection{{\Xic} Version}

\begin{description}
%------------------------------------
% 030104
\index{VersionString function}
\item{(string) \vt VersionString()}\\
This function returns a string containing the current {\Xic}
version in a form like ``{\vt 2.5.40}''.

\end{description}


%------------------------------------------------------------------------------
\section{Main Functions 2}
\subsection{Arrays}

\begin{description}
%------------------------------------
% 030104
\index{ArrayDims function}
\item{(int) \vt ArrayDims({\it out\_array}, {\it array\/})}\\
This function returns the size (number of storage locations) of an
array, and possibly the size of each dimension.  Arrays can have from
one to three dimensions.  If the first argument is an array with size
three or larger, the size of each dimension of the array in the second
argument is stored in the first three locations of the first argument
array, with the 0'th index being the lowest order.  Unused dimensions
are saved as 0.  If the first argument is an integer 0, no dimension
size information is returned.  The size of the array (number of
storage locations, which should equal the product of the nonzero
dimensions) is returned by the function.

%------------------------------------
% 030104
\index{ArrayDimension function}
\item{(int) \vt ArrayDimension({\it out\_array}, {\it array\/})}\\
This function is very similar to {\vt ArrayDims}, and the arguments
have the same types and purpose as for that function.  The return
value is the number of dimensions used (1--3) if the second argument
is an array, 0 otherwise.  Unlike {\vt ArrayDims}, this function does
not fail if the second argument is not an array.

%------------------------------------
% 030104
\index{GetDims function}
\item{(int) \vt GetDims({\it array\/}, {\it out\_array\/})}\\
This is for backward compatibility.  This function is equivalent to
{\vt ArrayDimension}, but the two arguments are in reverse order. 
This function may disappear -- don't use.

%------------------------------------
% 030104
\index{DupArray function}
\item{(int) \vt DupArray({\it desc\_array}, {\it src\_array\/})}\\
This function duplicates the {\it src\_array} into the {\it
dest\_array}.  The {\it dest\_array} argument must be an unreferenced
array.  Upon successful return, the {\it dest\_array} will be a copy
of the {\it src\_array}, and the return value is 1.  If the {\it
dest\_array} can not be resized due to its being referenced by a
pointer, 0 is returned.  The function will fail if either argument is
not an array.

%------------------------------------
% 030104
\index{SortArray function}
\item{(int) \vt SortArray({\it array}, {\it size}, {\it descend},
  {\it indices\/})}\\
This function will sort the elements of the array passed as the first
argument.  The number of elements to sort is given in the second
argument.  The function will fail if {\it size} is negative, or will
return without action if {\it size} is 0.  The size is implicitly
limited to the size of the array.  The sorted values will be ascending
if the third argument is 0, descending otherwise.  The fourth
argument, if nonzero, is an array which will be filled in with the
index mapping applied to the array.  For example, if array[5] is moved
to array[0] during the sort, the value of indices[0] will be 5.  This
array will be resized if necessary, but the function will fail if
resizing fails.
    
If the array being sorted is multi-dimensional, the sorting will use
the internal linear order.  The return value is the actual number of
items sorted, which will be the value of size unless this was limited
by the actual array size.

\end{description}


\subsection{Bitwise Logic}

All numerical data are stored internally in double-precision floating
point representation.  These functions convert the internal values to
unsigned integer data, apply the operation, and return the
floating-point representation of the result.  This should be invisible
to the user, but assumes well-behaved numerics in the host computer.

\begin{description}
%------------------------------------
% 030304
\index{ShiftBits function}
\item{(unsigned int) \vt ShiftBits({\it bits\/}, {\it val\/})}\\
This function will shift the binary representation of the unsigned
integer {\it bits} by the integer {\it val\/}.  If {\it val} is
positive, the bits are shifted to the right, or if negative the bits
are shifted to the left.  The function returns the shifted value.

%------------------------------------
% 030304
\index{AndBits function}
\item{(unsigned int) \vt AndBits({\it bits1\/}, {\it bits2\/})}\\
This function returns the bitwise AND of the two arguments, which are
taken as unsigned integers.

%------------------------------------
% 030304
\index{OrBits function}
\item{(unsigned int) \vt OrBits({\it bits1\/}, {\it bits2\/})}\\
This function returns the bitwise OR of the two arguments, which are
taken as unsigned integers.

%------------------------------------
% 030304
\index{XorBits function}
\item{(unsigned int) \vt XorBits({\it bits1\/}, {\it bits2\/})}\\
This function returns the bitwise exclusive-OR of the two arguments,
which are taken as unsigned integers.

%------------------------------------
% 030304
\index{NotBits function}
\item{(unsigned int) \vt NotBits({\it bits\/})}\\
This function returns the bitwise NOT of the argument, which is taken
as an unsigned integer.

\end{description}


\subsection{Error Reporting}

The following functions provide an interface to the {\Xic} error
reporting and logging system.  The first two functions operate on the
``message'' which is a list of strings generated by errors encountered
in function calls.  Within {\Xic}, the message may or may not be added
to the error log, which is accessible via the functions below.  Logged
messages are included in the error log file, and will be displayed in
a pop-up on-screen.  If not added to the error log, the message may be
displayed in another type of pop-up window, or on the prompt line, or
may be placed in a conversion log file.

\begin{description}
%------------------------------------
% 030104
\index{GetError function}
\item{(string) \vt GetError()}\\
This returns the current error text.  Error messages generated by an
unsuccessful operation that opens, translates, or writes cells or
manipulates the database, can be retrieved with this function for
diagnostic purposes.  This function should be called immediately after
an error return is detected, since subsequent operations may clear or
change the error text.  If there are no recorded errors, a ``no
errors'' string is returned.  This function never fails and always
returns a message string.

%------------------------------------
% 030104
\index{AddError function}
\item{\vt AddError({\it string\/})}\\
This function will add a string to the current error message, which
can be retrieved with {\vt GetError}.  This is useful for error
reporting from user-defined functions.  Any number of calls can be
made, with the retrieved text consisting of a concatenation of the
strings, with line termination added if necessary, in reverse order of
the {\vt AddError} calls.  No other built-in function should be
executed between calls to {\vt AddError}, or between a call that
generated an error and a call to {\vt AddError}, as this will cause
the second string to overwrite the first.

%------------------------------------
% 030104
\index{GetLogNumber function}
\item{(int) \vt GetLogNumber()}\\
Return the integer index of the most recent error message dumped to
the errors log file.  The return value is 0 if there are no errors
recorded in the file.

%------------------------------------
% 030104
\index{GetLogMessage function}
\item{(string) \vt GetLogMessage({\it message\_num\/})}\\
Return the error message string corresponding to the integer argument,
as was appended to the errors log file.  The 10 most recent error
messages are available.  If the argument is out of range, a null
string is returned.  The range is the current index to (not including)
this index minus 10, or 0, whichever is larger.

%------------------------------------
% 101609
\index{AddLogMessage function}
\item{(int) \vt AddLogMessage({\it string}, {\it error\/})}\\
Apply a new message to the error/warning log file.  The second
argument is a boolean which if nonzero will add the string as an error
message, otherwise the message is added as a warning.  The return
value is the index assigned to the new message, or 0 if the string is
empty or null.
\end{description}


\subsection{Generic Handle Functions}

The following functions take as an argument any type of handle, though
some of these functions may do nothing if passed an inappropriate
handle type.  In particular, for functions that operate on lists, the
following handle types are meaningful:

\begin{tabular}{|l|l|}\hline
\bf Object & \bf Handle Type\\ \hline
string &  stringlist\_handle\\ \hline
object &  object\_handle\\ \hline
property &  prpty\_handle\\ \hline
device &  device\_handle\\ \hline
device contact &  dev\_contact\_handle\\ \hline
subcircuit &  subckt\_handle\\ \hline
subcircuit contact &  subc\_contact\_handle\\ \hline
terminal &  terminal\_handle\\ \hline
\end{tabular}

\begin{description}
%------------------------------------
% 030104
\index{NumHandles function}
\item{(int) \vt NumHandles()}\\
This returns the number of handles of all types currently in the hash
table.  It can be used as a check to make sure handles are being
properly closed (and thus removed from the table) in the user's
scripts.

%------------------------------------
% 030104
\index{HandleContent function}
\item{(int) \vt HandleContent({\it handle\/})}\\
This function returns the number of objects currently referenced by
the list-type handle passed as an argument.  The return value is 1 for
other types of handle.  The return value is 0 for an empty or closed
handle.

%------------------------------------
% 030104
\index{HandleTruncate function}
\item{(int) \vt HandleTruncate({\it handle\/}, {\it count\/})}\\
This function truncates the list referenced by the handle, leaving the
current item plus at most {\it count} additional items.  If {\it
count} is negative, it is taken as 0.  The function returns 1 on
success, or 0 if the handle does not reference a list or is not found.

%------------------------------------
% 030104
\index{HandleNext function}
\item{(int) \vt HandleNext({\it handle\/})}\\
This function will advance the handle to reference the next element in
its list, for handle types that reference a list.  It has no effect on
other handles.  If there were no objects left in the list, or the
handle was not found, 0 is returned, otherwise 1 is returned.

%------------------------------------
% 030104
\index{HandleDup function}
\item{(handle) \vt HandleDup({\it handle\/})}\\
This function will duplicate a handle and its underlying reference or
list of references.  The new handle is not associated with the old,
and should be iterated through or closed explicitly.  For file
descriptors, the return value is a duplicate descriptor to the
underlying file, with the same read/write mode and file position as
the original handle.  If the function succeeds, a handle value is
returned.  If the function fails, 0 is returned.

%------------------------------------
% 030104
\index{HandleDupNitems function}
\item{(handle) \vt HandleDupNitems({\it handle\/}, {\it count\/})}\\
This function acts similarly to {\vt HandleDup}, however for handles
that are references to lists, the new handle will reference the
current item plus at most {\it count} additional items.  For handles
that are not references to lists, the {\it count} argument is ignored. 
The new handle is returned on success, 0 is returned if there was an
error.

%------------------------------------
% 030104
\index{H function}
\item{(handle) \vt H({\it scalar\/})}\\
This function creates a handle from an integer variable.  This is
needed for using the handle values stored in the array created with
the {\vt HandleArray} function, or otherwise.  Array elements are
numeric variables, and can not be passed directly to functions
expecting handles.  This function performs the necessary data
conversion.

Example:
\begin{quote}
{\vt SomeFunction(H(handle\_array[3]))}.
\end{quote}

Array elements are always numeric variables, though it is possible to
assign a handle value to an array element.  In order to use as a
handle an array element so defined, the {\vt H} function must be
applied.  Since scalar variables become handles when assigned from a
handle, the {\vt H} function should never be needed for scalar
variables.

%------------------------------------
% 030104
\index{HandleArray function}
\item{(int) \vt HandleArray({\it handle\/}, {\it array\/})}\\
This function will create a new handle for every object in the list
referenced by the handle argument, and add that handle identifier to
the array.  Each new handle references a single object.  The array
argument is the name of a previously defined array variable.  The
array will be resized if necessary, if possible.  It is not possible
to resize an array referenced through a pointer, or an array with
pointer references.  The function returns 0 if the array cannot be
resized and resizing is needed.  The number of new handles is
returned, which will be 0 if the handle argument is empty or does not
reference a list.  The handles in the array of handle identifiers can
be closed conveniently with the {\vt CloseArray} function.  Since the
array elements are numeric quantities and not handles, they can not be
passed directly to functions expecting handles.  The {\vt H} function
should be used to create a temporary handle variable from the array
elements when a handle is needed:  for example, {\vt
HandleNext(H(array[2]))}.

%------------------------------------
% 110115
\index{HandleCat function}
\item{(int) \vt HandleCat({\it handle1\/}, {\it handle2\/})}\\
This function will add a copy of the list referenced by the second
handle to the end of the list referenced by the first handle.  Both
arguments must be handles referencing lists of the same kind.  The
return value is nonzero for success, 0 otherwise.

%------------------------------------
% 030104
\index{HandleReverse function}
\item{(int) \vt HandleReverse({\it handle\/})}\\
This function will reverse the order of the list referenced by the
handle.  Calling this function on other types of handles does nothing. 
The function returns 1 if the action was successful, 0 otherwise.

%------------------------------------
% 030104
\index{HandlePurgeList function}
\item{(int) \vt HandlePurgeList({\it handle1\/}, {\it handle2\/})}\\
This function removes from the list referenced by the second handle
any items that are also found in the list referenced by the first
handle.  Both handles must reference lists of the same type.  The
return value is 1 on success, 0 otherwise.

%------------------------------------
% 030104
\index{Close function}
\item{(int) \vt Close({\it handle\/})}\\
This function deletes and frees the handle.  It can be used to free up
resources when a handle is no longer in use.  In particular, for file
handles, the underlying file descriptor is closed by calling this
function.  The return value is 1 if the handle is closed successfully,
0 if the handle is not found in the internal hash table or some other
error occurs.

%------------------------------------
% 030104
\index{CloseArray function}
\item{(int) \vt CloseArray({\it array\/}, {\it size\/})}\\
This function will call {\vt Close} on the first {\it size} elements
of the array.  The array is assumed to be an array of handles as
returned from {\vt HandleArray}.  The function will fail if the {\it
array} is not an array variable.  The return value is always 1. 

\end{description}


\subsection{Memory Management}

\begin{description}
%------------------------------------
% 030104
\index{FreeArray function}
\item{(int) \vt FreeArray({\it array\/})}\\
This function will delete the memory used in the {\it array}, and
reallocate the size to 1.  This function may be useful when memory is
tight.  It is not possible to free an array it there are variables
that point to it.  This function returns 1 on success, 0 otherwise.

%------------------------------------
% 030104
\index{CoreSize function}
\item{(int) \vt CoreSize()}\\
This returns the total size of dynamically allocated memory used by
{\Xic}, in kilobytes.

\end{description}


\subsection{Script Variables}

\begin{description}
%------------------------------------
% 030104
\index{Defined function}
\item{(int) \vt Defined({\it variable\/})}\\
If a variable is referenced before it is assigned to, the variable has
no type, but behaves in all ways as a string set to the variable's
name.  This function returns 1 if the argument has a type assigned, or
0 if it has no type.

%------------------------------------
% 110115
\index{TypeOf function}
\item{(string) \vt TypeOf({\it variable\/})}\\
This function returns a string which indicates the type of variable
passed as an argument.  The possible returns are

\begin{tabular}{ll}
``{\vt none}''   & variable has no type\\
``{\vt scalar}'' & variable is a scalar number\\
``{\vt complex}'' & variable is a complex number\\
``{\vt string}'' & variable is a string\\
``{\vt array}''  & variable is an array\\
``{\vt zoidlist}''  & variable is a zoidlist\\
``{\vt layer\_expr}'' & variable is a layer\_expr\\
``{\vt handle}'' & variable is a handle to something\\
\end{tabular}

\end{description}


\subsection{Path Manipulation and Query}

\begin{description}
%------------------------------------
% 030104
\index{PathToEnd function}
\item{(int) \vt PathToEnd({\it path\_name}, {\it dir\/})}\\
This function manipulates path strings.  The string {\it path\_name}
can be anything, but it is usually one of ``{\vt Path}'', ``{\vt
LibPath}'', ``{\vt HlpPath}'', or ``{\vt ScrPath}'', i.e., the name of
a search path.  The string {\it dir\/} will be appended to the path if
it does not exist in the path, or is moved to the end if it does.  If
the {\it path\_name} is not a recognized path keyword, a variable of
that name will be created to hold the path.  This can be used to store
alternate paths.

%------------------------------------
% 030104
\index{PathToFront function}
\item{(int) \vt PathToFront({\it path\_name\/}, {\it dir\/})}\\
This is similar to the {\vt PathToEnd} function, but the {\it dir\/}
will be added or moved to the front of the path.

%------------------------------------
% 030104
\index{InPath function}
\item{(int) \vt InPath({\it path\_name\/}, {\it dir\/})}\\
This function returns 1 if {\it dir\/} is included in the path named in
{\it path\_name}, 0 otherwise.

%------------------------------------
% 030104
\index{RemovePath function}
\item{(int) \vt RemovePath({\it path\_name\/}, {\it dir\/})}\\
This function removes the directory {\it dir} from the search path, if
it is present.  The return value is 1 if the path was modified, 0
otherwise.  The {\it path\_name} argument has the same meaning as in
{\vt PathToEnd}.

\end{description}


\subsection{Regular Expressions}

\begin{description}
%------------------------------------
% 030104
\index{RegCompile function}
\item{(regex\_handle) \vt RegCompile({\it regex}, {\it case\_insens\/})}\\
This function returns a handle to a compiled regular expression, as
given in the first (string) argument.  The handle can be used for
string comparison in {\vt RegCompare}, and should be closed when no
longer needed.  The second argument is a flag; if nonzero the regular
expression is compiled such that comparisons will be case-insensitive. 
If zero, the test will be case-sensitive.  If the compilation fails,
this function returns 0, and an error message can be obtained from
{\vt RegError}.

%------------------------------------
% 030104
\index{RegCompare function}
\item{(int) \vt RegCompare({\it regex\_handle}, {\it string},
  {\it array\/})}\\
This function compares the regular expression represented by the
handle to the string given in the second argument.  If a match is
found, the function returns 1, and the match location is set in the
{\it array} argument, unless 0 is passed for this argument.  If an
array is passed, it must have size 2 or larger.  The 0'th array
element is set to the character index in the {\it string} where the
match starts, and the next array location is set to the character
index of the first character following the match.  This function
returns 0 if there is no match, and -1 if an error occurs.  If -1 is
returned, an error message can be obtained from {\vt RegError}.

%------------------------------------
% 030104
\index{RegError function}
\item{(string) \vt RegError({\it regex\_handle})}\\
This function returns an error message string produced by the failure
of {\vt RegCompile} or {\vt RegCompare}.  It can be called after one
of these functions returns an error value.  The argument is the handle
value returned from {\vt RegCompile}, which will be 0 if {\vt
RegCompile} fails.  A null string is returned if the handle is bogus.

\end{description}


\subsection{String List Handles}

The following group of functions relate to lists of strings accessed by
a handle.  Such lists are returned by functions that find, for
example, the list of layers in the current technology file, of a list
of subcells in the current cell.  Lists can also be defined by the
user and are quite convenient for some purposes.

\begin{description}
%------------------------------------
% 030104
\index{StringHandle function}
\item{(stringlist\_handle) \vt StringHandle({\it string\/},
 {\it sepchars\/})}\\
This function returns a handle to a list of strings which are derived
by splitting the {\it string} argument at characters found in the {\it
sepchars} string.  If {\it sepchars} is empty or null, the strings
will be separated by white space, so each string in the handle list
will be a word from the argument string.

%------------------------------------
% 010509
\index{ListHandle function}
\item{(stringlist\_handle) \vt ListHandle({\it arglist\/})}\\
This function creates a list of strings corresponding to the variable
number of arguments, and returns a handle to the list.  The arguments
are converted to strings in the manner of the {\vt Print} function,
however each argument corresponds to a unique string in the list.  The
strings are accessed in (left to right) order of the arguments.

If no arguments are given, a handle to an empty list is returned. 
Calls to {\vt ListAddFront} and/or {\vt ListAddBack} can be used to
add strings subsequently.

%------------------------------------
% 030104
\index{ListContent function}
\item{(string) \vt ListContent({\it stringlist\_handle\/})}\\
This function returns the string currently referenced by the handle,
and does {\it not} increment the handle to the next string in the
list.  If the handle is not found or contains no further list
elements, a null string is returned.  The function will fail if the
handle is not a reference to a list of strings.

%------------------------------------
% 060905
\index{ListReverse function}
\item{(int) \vt ListReverse({\it stringlist\_handle\/})}\\
This function reverses the order of strings in the stringlist handle
passed.  If the operation succeeds the return value is 1, or if the
list is empty or an error occurs the value is 0.


%------------------------------------
% 030104
\index{ListNext function}
\item{(string) \vt ListNext({\it stringlist\_handle\/})}\\
This function will return the string at the front of the list
referenced by the handle, and set the handle to reference the next
string in the list.  The function will fail if the handle is not a
reference to a list of strings.  A null string is returned if the
handle is not found, or after all strings in the list have been
returned.

%------------------------------------
% 030104
\index{ListAddFront function}
\item{(int) \vt ListAddFront({\it stringlist\_handle\/}, {\it string\/})}\\
This function adds {\it string} to the front of the list of strings
referenced by the handle, so that the handle immediately references
the new string.  The function will fail if the handle is not a
reference to a string list, or the given string is null.  The return
value is 1 unless the handle is not found, in which case 0 is
returned.

%------------------------------------
% 030104
\index{ListAddBack function}
\item{(int) \vt ListAddBack({\it stringlist\_handle\/}, {\it string\/})}\\
This function adds {\it string} to the back of the list of strings
referenced by the handle, so that the handle references the new string
after all existing strings have been cycled.  The function will fail
if the handle is not a reference to a string list, or the given string
is null.  The return value is 1 unless the handle is not found, in
which case 0 is returned.

%------------------------------------
% 030104
\index{ListAlphaSort function}
\item{(int) \vt ListAlphaSort({\it stringlist\_handle\/})}\\
This function will alphabetically sort the list of strings referenced
by the handle.  The function will fail if the handle is not a
reference to a list of strings.  The return value is 1 unless the
handle is not found, in which case 0 is returned.

%------------------------------------
% 030104
\index{ListUnique function}
\item{(int) \vt ListUnique({\it stringlist\_handle\/})}\\
This function deletes duplicate strings from the string list
referenced by the handle, so that strings remaining in the list are
unique.  The function will fail if the handle is not a reference to a
list of strings.  The return value is 1 unless the handle is not
found, in which case 0 is returned.

%------------------------------------
% 030104
\index{ListFormatCols function}
\item{(string) \vt ListFormatCols({\it stringlist\_handle\/},
 {\it columns\/})}\\
This function returns a string which contains the column formatted
list of strings referenced by the handle.  The {\it columns} argument
sets the page width in character columns.  This function is useful for
formatting lists of cell names, for example.  The return is a null
string if the handle is not found.  The function fails if the handle
does not reference a list of strings.

%------------------------------------
% 030104
\index{ListConcat function}
\item{(string) \vt ListConcat({\it stringlist\_handle\/},
 {\it sepchars\/})}\\
This function returns a string consisting of each string in the list
referenced by the handle separated by the {\it sepchars} string.  If
the {\it sepchars} string is empty or null, there is no separation
between the strings.  The function will fail if the handle does not
reference a list of strings.  A null string is returned if the handle
is not found.

%------------------------------------
% 030104
\index{ListIncluded function}
\item{(int) \vt ListIncluded({\it stringlist\_handle\/}, {\it string\/})}\\
This function compares {\it string} to each string in the list
referenced by the handle and returns 1 if a match is found (case
sensitive).  If no match, or the handle is not found, 0 is returned. 
The function will fail if the handle is not a reference to a list of
strings.

\end{description}


\subsection{String Manipulation and Conversion}

\begin{description}
%------------------------------------
% 030104
\index{Strcat function}
\item{(string) \vt Strcat({\it string1\/}, {\it string2\/})}\\
This function appends {\it string2\/} to {\it string1\/} and returns
the new string.  The `$+$' operator is overloaded to also perform this
function on string operands.

%------------------------------------
% 030104
\index{Strcmp function}
\item{(int) \vt Strcmp({\it string1\/}, {\it string2\/})}\\
This function returns an integer representing the lexical difference
between {\it string1} and {\it string2}.  This is the same as the
``{\vt strcmp}'' C library function, except that null strings are
accepted and have the minimum lexical value.  The comparison operators
are overloaded to also perform this function on string operands.

%------------------------------------
% 030104
\index{Strncmp function}
\item{(int) \vt Strncmp({\it string1\/}, {\it string2\/}, {\it n\/})}\\
This compares at most {\it n} characters in strings 1 and 2 and
returns the lexical difference.  This is equivalent to the C library
``{\vt strncmp}'' function, except that null strings are accepted and
have the minimum lexical value.

%------------------------------------
% 030104
\index{Strcasecmp function}
\item{(int) \vt Strcasecmp({\it string1\/}, {\it string2\/})}\\
This internally converts strings 1 and 2 to lower case, and returns
the lexical difference.  This is equivalent to the C library ``{\vt
strcasecmp}'' function, except that null strings are accepted and have
the minimum lexical value.

%------------------------------------
% 030104
\index{Strncasecmp function}
\item{(int) \vt Strncasecmp({\it string1\/}, {\it string2\/}, {\it n\/})}\\
This internally converts strings 1 and 2 to lower case, and compares
at most {\it n} characters, returning the lexical difference.  This is
equivalent to the C library ``{\vt strncasecmp}'' function.  except
that null strings are accepted and have the minimum lexical value.

%------------------------------------
% 030104
\index{Strdup function}
\item{(string) \vt Strdup({\it string\/})}\\
This function returns a new string variable containing a copy of the
argument's string.  An error occurs if the argument is not
string-type.  Note that this differs from assignment, which propagates
a pointer to the string data rather than copying.

%------------------------------------
% 030104
\index{Strtok function}
\item{(string) \vt Strtok({\it str\/}, {\it sep\/})}\\
The {\vt Strtok} function is used to isolate sequential tokens in a
string, {\it str\/}.  These tokens are separated in the string by at
least one of the characters in the string {\it sep\/}.  The first time
that {\vt Strtok} is called, {\it str} should be specified; subsequent
calls, wishing to obtain further tokens from the same string, should
pass 0 instead.  The separator string, {\it sep}, must be supplied
each time, and may change between calls.

The {\vt Strtok} function returns a reference to each subsequent token
in the string, after replacing the separator character with a NULL
character.  When no more tokens remain, a null string is returned. 
Note that this is destructive to {\it str}.

This function is similar to the C library ``{\vt strtok}'' function.

Example:  print the space-separated words
\begin{quote}\vt
  teststr = "here are$\backslash$tsome   words"\\
  word = Strtok(teststr, " $\backslash$t")\\
  Print("First word is", word);\\
  while (word = Strtok(0, " $\backslash$t"))\\
  \hspace*{2em} Print("Next word:", word)\\
  done\\
\end{quote}

%------------------------------------
% 072904
\index{Strchr function}
\item{(string) \vt Strchr({\it string\/}, {\it char\/})}\\
The second argument is an integer representing a character.  The
return value is a pointer into {\it string} offset to point to the
first instance of the character.  If the character is not in the
string, a null pointer is returned.  This is basically the same as the
C {\vt strchr} function.

%------------------------------------
% 072904
\index{Strrchr function}
\item{(string) \vt Strrchr({\it string\/}, {\it char\/})}\\
The second argument is an integer representing a character.  The
return value is a pointer into {\it string} offset to point to the
last instance of the character.  If the character is not in the
string, a null pointer is returned.  This is basically the same as the
C {\vt strrchr} function.

%------------------------------------
% 072904
\index{Strstr function}
\item{(string) \vt Strstr({\it string\/}, {\it char\/})}\\
The second argument is a string which is expected to be a substring of
the string.  The return value is a pointer into {\it string} to the
start of the first occurrence of the substring.  If there are no
occurrences, a null pointer is returned.  This is equivalent to the C
{\vt strstr} function.

%------------------------------------
% 072904
\index{Strpath function}
\item{(string) \vt Strpath({\it string\/})}\\
This returns a copy of the file name part of a full path given in
the string.

%------------------------------------
% 030104
\index{Strlen function}
\item{(int) \vt Strlen({\it string\/})}\\
This function returns the number of characters in {\it string\/}.

%------------------------------------
% 030104
\index{Sizeof function}
\item{(int) \vt Sizeof({\it arg\/})}\\
This function returns the allocated size of the argument, which is
mostly useful for determining the size of an array.  The return value
is

\begin{tabular}{ll}\\
string length & {\it arg} is a string\\
allocated array size & {\it arg} is an array\\
number of trapezoids & {\it arg} is a zoidlist\\
1 & {\it arg} is none of above\\
\end{tabular}

%------------------------------------
% 030104
\index{ToReal function}
\item{(scalar) \vt ToReal({\it string\/})}\\
The returned value is a variable of type scalar containing the
numeric value from the passed argument, which is a string.  The text
of the string should be interpretable as a numeric constant.  If the
argument is instead a scalar, the value is simply copied.

%------------------------------------
% 030104
\index{ToString function}
\item{(string) \vt ToString({\it real\/})}\\
The returned value is a variable of type string containing a text
representation of the passed variable, which is expected to be of type
scalar.  The format is the same as the C {\vt printf} function with
``{\vt \%g}'' as a format specifier.  If the argument is instead a
string, the returned value points to that string.

%------------------------------------
% 102114
\index{ToStringA function}
\item{(string) \vt ToStringA({\it real\/}, {\it digits\/})}\\
This will return a string containing the real number argument in SPICE
format, which is a form consisting of a fixed point number followed by
an alpha character or sequence which designates a scale factor.  These
are the same scale factors as used in the number parser.  though
``{\vt mils}'' is not used.  The second argument is an integer giving
the number of digits to print (in the range 2-15).  If out of this
range, a default of 6 is used.

If the first argument is a string, the string contents will be parsed
as a number, and the result output as described above.  If the parse
fails, the number is silently taken as zero.

%------------------------------------
% 101104
\index{ToFormat function}
\item{(string) \vt ToFormat({\it format\/}, {\it arg\_list})}\\
This function returns a string, formatted in the manner of the C {\vt
printf} function.  The first argument is a format string, as would be
given to {\vt printf}.  Additional arguments (there can be zero or
more) are the variables that correspond to the format specification. 
The type and position of the arguments must match the format
specification, which means that the variables passed must resolve to
strings or to numeric scalars.  All of the formatting options
described in the Unix manual page for {\vt printf} are available, with
the following exceptions:
\begin{enumerate}
\item{No random argument access.}
\item{At most one `{\vt *}' per substitution.}
\item{``{\vt \%p}'' will always print zero.}
\item{``{\vt \%n}'' is not supported.}
\end{enumerate}

The function fails if the first argument is not a string, is null, or
there is a syntax error or unsupported construct, or there is a type
or number mismatch between specification and arguments.

For example, the ``id'' returned from {\vt GetObjectID} prints as a
floating point value by default (since it is a large integer), which
is usually not useful.  One can print this as a hex value as follows:

\begin{quote}\vt
 id = GetObjectID(handle)\\
 Print("Id =", ToFormat("0x\%x", id))
\end{quote}

%------------------------------------
% 030104
\index{ToChar function}
\item{(string) \vt ToChar({\it integer\/})}\\
This function takes as its input an integer value for a character, and
returns a string containing a printable representation of the
character.  A null string is returned if the input is not a valid
character index.  This function can be used to preformat character
data for printing with the various print functions.

\end{description}


\subsection{Current Directory}

\begin{description}
%------------------------------------
% 030104
\index{Cwd function}
\item{\vt Cwd({\it path\/})}\\
This function changes the current working directory to that given by
the argument.  If {\it path\/} is null or empty, the change will be to
the user's home directory.  A tilde character (`\symbol{126}')
appearing in {\it path\/} is expanded to the user's home directory as
in a Unix shell.  The return value is 1 if the change succeeds, 0
otherwise.

%------------------------------------
% 030104
\index{Pwd function}
\item{(string) \vt Pwd()}\\
This function returns a string containing the absolute path to
the current directory.

\end{description}


\subsection{Date and Time}

\begin{description}
%------------------------------------
% 030104
\index{DateString function}
\item{(string) \vt DateString()}\\
This function returns a string containing the date and time in
the format
\begin{quote}\vt
    Tue Jun 12 23:42:38 PDT 2001
\end{quote}

%------------------------------------
% 020411
\index{Time function}
\item{(int) \vt Time()}\\
This returns a system time value, which can be converted to more
useful output by {\vt TimeToString} or {\vt TimeToVals}.  Actually,
the returned value is the number of seconds since the start of the
year 1970.

%------------------------------------
% 020411
\index{MakeTime function}
\item{(int) \vt MakeTime({\it array\/}, {\it gmt\/})}\\
This function takes the time fields specified in the array and returns
a time value is if returned from {\vt Time}.  If the boolean argument
{\it gmt} is nonzero, the interpretation is GMT, otherwise local time. 
The array must be size 9 or larger, with the values set as when
returned by the {\vt TimeToVals} function (below).

Under Windows, the {\it gmt} argument is ignored and local time is used.

%------------------------------------
% 020411
\index{TimeToString function}
\item{(string) \vt TimeToString({\it time\/}, {\it gmt\/})}\\
Given a time value as returned from {\vt Time}, this returns a 
string in the form
\begin{quote} \vt
Tue Jun 12 23:42:38 PDT 2001
\end{quote}

If the boolean argument {\it gmt} is nonzero, GMT will be used,
otherwise the local time is used.

%------------------------------------
% 020411
\index{TimeToVals function}
\item{(string) \vt TimeToVals({\it time\/}, {\it gmt\/}, {\it array\/})}\\
Given a time value as returned from {\vt Time}, this breaks out the
time/date into the array.  The array must have size 9 or larger.  If
the boolean argument {\it gmt} is nonzero, GMT is used, otherwise
local time is used.
  
The array values are set as follows.

\begin{tabular}{ll}
{\it array\/}[0] & seconds (0 - 59).\\
{\it array\/}[1] & minutes (0 - 59).\\
{\it array\/}[2] & hours (0 - 23).\\
{\it array\/}[3] & day of month (1 - 31).\\
{\it array\/}[4] & month of year (0 - 11).\\
{\it array\/}[5] & year - 1900.\\
{\it array\/}[6] & day of week (Sunday = 0).\\
{\it array\/}[7] & day of year (0 - 365).\\
{\it array\/}[8] & 1 if summer time is in effect, or 0.\\
\end{tabular}

The return value is a string containing an abbreviation of the local
timezone name, except under Windows where the return is an empty
string.

%------------------------------------
% 030104
\index{MilliSec function}
\item{(int) \vt MilliSec()}\\
This returns the elapsed time in milliseconds since midnight January
1, 1970 GMT.  This can be used to measure script execution time.

%------------------------------------
% 102504
\index{StartTiming function}
\item{(int) \vt StartTiming({\it array\/})}\\
This will initialize the values in the array, which must have size 3
or larger, for later use by the {\vt StopTiming} function.  The return
value is always 1.

%------------------------------------
% 102504
\index{StopTiming function}
\item{(int) \vt StopTiming({\it array\/})}\\
This will place time differences (in seconds) into the array, since
the last call to {\vt StartTiming} (with the same argument).  The
array must have size 3 or larger.  the components are:

\begin{tabular}{ll}\\
0 & Elapsed wall-clock time\\
1 & Elapsed user time\\
2 & Elapsed system time\\
\end{tabular}

The user time is the time the cpu spent executing in user mode.  The
system time is the time spent in the system executing on behalf of the
process.  This uses the UNIX {\vt getrusage} or {\vt times} system
calls, which may not be available on all systems.  If support is not
available, e.g., in Windows, the user and system entries will be zero,
but the wall-clock time is valid.  This function always returns 1.

\end{description}


\subsection{File System Interface}

\begin{description}
%------------------------------------
% 030104
\index{Glob function}
\item{(string) \vt Glob({\it pattern\/})}\\
This function returns a string which is a filename expansion of the
pattern string, in the manner of the C-shell.  The pattern can contain
the usual substitution characters {\vt *}, {\vt ?}, {\vt [ ]},
\{ \}.

Example:  Return a list of ``{\vt .gds}'' files in the current directory.
\begin{quote}\vt
list = Glob("*.gds")
\end{quote}

%------------------------------------
% 030104
\index{Open function}
\item{(file\_handle) \vt Open({\it file\/}, {\it mode\/})}\\
This function opens the file given as a string argument according to
the string {\it mode\/}, and returns a file descriptor.  The {\it
mode\/} string should consist of a single character:  `{\vt r}' for
reading, `{\vt w}' to write, or `{\vt a}' to append.  If the returned
value is negative, an error occurred.

%------------------------------------
% 030104
\index{Popen function}
\item{(file\_handle) \vt Popen({\it command\/}, {\it mode\/})}\\
This command opens a pipe to the shell command given as the first
argument, and returns a file handle that can be used to read and/or
write to the process.  The handle should be closed with the {\vt
Close} function.  This is a wrapper around the C library {\vt popen}
command so has the same limitations as the local version of that
command.  In particular, on some systems the mode may be reading or
writing, but not both.  The function will fail if either argument is
null or if the {\vt popen} call fails.

%------------------------------------
% 030104
\index{Sopen function}
\item{(file\_handle) \vt Sopen({\it host\/}, {\it port\/})}\\
This function opens a ``socket'' which is a communications channel to
the given {\it host} and {\it port}.  If the {\it host} string is null
or empty, the local host is assumed.  The {\it port} number must be
provided, there is no default.  If the open is successful, the return
value is an integer larger than zero and is a handle that can be used
in any of the read/write functions that accept a file handle.  The
{\vt Close} function should be called on the handle when the
interaction is complete.  If the connection fails, a negative number
is returned.  The function fails if there is a major error, such as no
BSD sockets support.

%------------------------------------
% 030104
\index{ReadLine function}
\item{(string) \vt ReadLine({\it maxlen\/}, {\it file\_handle\/})}\\
The {\vt ReadLine} function returns a string with length up to {\it
maxlen\/} filled with characters read from {\it file\_handle\/}.  The
{\it file\_handle\/} must have been successfully opened for reading
with a call to {\vt Open}, {\vt Popen}, or {\vt Sopen}.  The read is
terminated by end of file, a return character, or a null byte.  The
terminating character is not included in the string.  A null string is
returned when the end of file is reached, or if the handle is not
found.  The function will fail if the handle is not a file handle, or
{\it maxlen} is less than 1.

%------------------------------------
% 030104
\index{ReadChar function}
\item{(int) \vt ReadChar({\it file\_handle\/})}\\
The {\vt ReadChar} function returns a single character read from {\it
file\_handle\/}, which must have been successfully opened for reading
with an {\vt Open}, {\vt Popen}, or {\vt Sopen} call.  The function
returns EOF (-1) when the end of file is reached, or if the handle is
not found.  The function will fail if the handle is not a file handle.

%------------------------------------
% 041004
\index{WriteLine function}
\item{(int) \vt WriteLine({\it string\/}, {\it file\_handle\/})}\\
The {\vt WriteLine} function writes the content of {\it string\/} to
{\it file\_handle\/}, which must have been successfully opened for
writing or appending with an {\vt Open}, {\vt Popen}, or {\vt Sopen}
call.  The number of characters written is returned.  The function
will fail if the handle is not a file handle, or the {\it string} is
null.

This function has the unusual property that it will accept the
arguments in reverse order.

{\vt WriteLine} does not append a carriage return character to the
string.  See the {\vt PrintLog} function for a variable argument list
alternative that does append a return character.

%------------------------------------
% 041004
\index{WriteChar function}
\item{(int) \vt WriteChar({\it c, file\_handle\/})}\\
This function writes a single character {\it c\/} to {\it
file\_handle\/}, which must have been successfully opened for writing
or appending with a call to {\vt Open}, {\vt Popen}, or {\vt Sopen}. 
The function returns 1 on success.  The function will fail if the
handle is not a file handle, or the integer value of {\it c} is not in
the range 0--255.

This function has the unusual property that it will accept the
arguments in reverse order.

%------------------------------------
% 030104
\index{TempFile function}
\item{(string) \vt TempFile({\it prefix\/})}\\
This function creates a unique temporary file name using the prefix
string given, and arranges for the file of that name to be deleted
when the program terminates.  The file is not actually created.  The
return from this command is passed to the {\vt Open} command to
actually open the file for writing.

%------------------------------------
% 020411
\index{ListDirectory function}
\item{(stringlist\_handle) \vt ListDirectory({\it path},
  {\it filter\/})}\\
This function returns a handle to a list of names of files and/or
directories in the given directory.  If the {\it path} argument
is null or empty, the current directory is understood.  If the {\it
filter} string is null or empty, all files and subdirectories will be
listed.  Otherwise the {\it filter} string can be ``{\vt f}'' in which
case only regular files will be listed, or ``{\vt d}'' in which case
only directories will be listed.  If the directory does not exist or
can't be read, 0 is returned, otherwise the return value is a handle
to a list of strings.

%------------------------------------
% 020411
\index{MakeDir function}
\item{(int) \vt MakeDir({\it path})}\\
This function will create a directory, if it doesn't already exist. 
If the {\it path} specifies a multi-component path, all parent
directories needed will be created.  The function will fail if a null
or empty {\it path} is passed, otherwise the return value is 1 if no
errors, 0 otherwise, with a message available from {\vt GetError}. 
Passing the name of an existing directory is not an error.

%------------------------------------
% 020411
\index{FileStat function}
\item{(int) \vt FileStat({\it path}, {\it array\/})}\\
This function returns 1 if the file in {\it path} exists, and fills in
some data about the file (or directory).  If the file does not exist,
0 is returned, and the array is untouched.

The {\it array} must have size 7 or larger, or a value 0 can be passed
for this argument.  In this case, no statistics are returned, but the
function return still indicates file existence.

If an array is passed and the path points to an existing file or
directory, the array is filled in as follows:

\begin{description}
\item{\it array\/}[0]\\
Set to 0 if {\it path} is a regular file.  Set to 1 if {\it path} is a
directory.  Set to 2 if {\it path} is some other type of object.

\item{\it array\/}[1]\\
The size of the regular file in bytes, undefined if not a regular
file.

\item{\it array\/}[2]\\
Set to 1 if the present process has read access to the file, 0
otherwise.

\item{\it array\/}[3]\\
Set to 1 if the present process has write access to the file, 0
otherwise.

\item{\it array\/}[4]\\
Set to 1 if the present process has execute permission to the
file, 0 otherwise.

\item{\it array\/}[5]\\
Set to the user id of the file owner.

\item{\it array\/}[6]\\
Set to the last modification time.  This is in a system-encoded form,
use {\vt TimeToString} or {\vt TimeToVals} to convert.
\end{description}

%------------------------------------
% 020411
\index{DeleteFile function}
\item{(int) \vt DeleteFile({\it path})}\\
Delete the file or directory given in {\it path}.  If a directory, it
must be empty.  If the file or directory does not exist or was
successfully deleted, 1 is returned, otherwise 0 is returned with an
error message available from {\vt GetError}.

%------------------------------------
% 020411
\index{MoveFile function}
\item{(int) \vt MoveFile({\it from\_path}, {\it to\_path\/})}\\
Move (rename) the file {\it from\_path} to a new file {\it
to\_path\/}.  On success, 1 is returned, otherwise 0 is returned with
an error message available from {\vt GetError}. 

Except under Windows, directories can be moved as well, but only
within the same file system.

%------------------------------------
% 020411
\index{CopyFile function}
\item{(int) \vt CopyFile({\it from\_path}, {\it to\_path\/})}\\
Copy the file {\it from\_path} to a new file {\it to\_path\/}.  On
success, 1 is returned, otherwise 0 is returned with an error message
available from {\vt GetError}.

%------------------------------------
% 020411
\index{CreateBak function}
\item{(int) \vt CreateBak({\it path})}\\
If the path file exists, rename it, suffixing the name with a ``{\vt
.bak}'' extension.  If a file with this name already exists, it will
be overwritten.  The function returns 1 if the file was moved or
doesn't exist, 0 otherwise, with an error message available from {\vt
GetError}.

%------------------------------------
% 102214
\index{Md5Digest function}
\item{(string) \vt Md5Digest({\it path})}\\
Return a string containing an MD5 digest for the file whose path is
passed as the argument.  This is the same digest as returned from the
{\cb !md5} command, and from the command
\begin{quote}
{\vt openssl dgst -md5} {\it filepath}
\end{quote}
available on many Linux-like systems.

If the file can not be opened, an empty string is returned, and an
error message is available from {\vt GetError}.
\end{description}


\subsection{Socket and {\Xic} Client/Server Interface}
\begin{description}
%------------------------------------
% 100408
\index{ReadData function}
\item{(string) \vt ReadData({\it size\/}, {\it skt\_handle\/})}\\
This function will read exactly {\it size} bytes from a socket, and
return string-type data containing the bytes read.  The {\it
skt\_handle} must be a socket handle returned from {\vt Sopen}.  The
function will fail (halt the script) only if the {\it size} argument
is not an integer.  On error, a null string is returned, and a message
is available from {\vt GetError}.

Note that the string can contain binary data, and if reading an
ASCII string be sure to include the null termination byte.  With
binary data, the standard string manipulations may not work, and
in fact can easily cause a program crash.

%------------------------------------
% 100408
\index{ReadReply function}
\item{(string) \vt ReadReply({\it retcode\/}, {\it skt\_handle\/})}\\
This function will read a response message from the {\Xic} server.  It
expects the {\Xic} server protocol and can not be used for other
purposes.

The first argument is an array of size 3 or larger.  Upon return, {\it
retcode\/}[0] will contain the server return code, which is an integer
0--9, or possibly -1 on error.  The value in {\it retcode\/}[1] will be
the size of the message returned, which will be 0 or larger.  The
value in {\it retcode\/}[2] will be 0 on success, 1 on error.  If an
error occurred, an error message is available from {\vt GetError}.

The return code in {\it retcode\/}[0] can have the following response
types:

\begin{tabular}{ll}
0 & ok\\
1 & in block, waiting for ``end''\\
2 & error\\
3 & scalar data\\
4 & string data\\
5 & array data\\
6 & zlist data\\
7 & lexpr data\\
8 & handle data\\
9 & geometry data\\
-1 & error reading data from server\\
\end{tabular}

The return value is of string-type, and may be null or binary.  With
binary data, the standard string manipulations may not work, and in
fact can easily cause a program crash.  It is not likely that the
return will have any use other than as an argument to {\vt
ConvertReply}.

This function will fail (halt the script) only if the retcode argument
is bad.

%------------------------------------
% 100408
\index{ConvertReply function}
\item{(variable) \vt ConvertReply({\it message\/}, {\it retcode\/})}\\
This function will parse and analyze a return message from the {\Xic}
server, which has been received with {\vt ReadReply}.  The first
argument is the message returned from {\vt ReadReply}.  The second
argument is an array of size 3 or larger, and can be the same array
passed to {\vt ReadReply}.  The {\it retcode\/}[0] entry must be set
to the message return code, and {\it retcode\/}[1] must be set to the
size of the returned buffer.  These are the same values as set in {\vt
ReadReply}.

Upon return, {\it retcode\/}[2] will contain a ``data\_ok'' flag,
which will be nonzero if the message contained data and the data were
read properly.  The function will fail (by halting the script) if the
{\it retcode} argument is bad, i.e., not an array of size 3 or larger,
or the {\it message} argument is not string-type.

The response codes 0--2 contain no data and are status responses
from the server.  The data responses will set the type and data of
the function return, if successful.  The {\it retcode\/}[2] value
will be nonzero on success in these cases, and will always be
false if ``{\vt longmode}'' is not enabled.

Note that the type returned can be anything, and if assigned to a
variable that already has a different type, an error will occur.  The
{\vt delete} operator can be applied to the assigned-to variable to
clear its state, before the function call.

The response type 9 is returned from the {\vt geom} server function. 
This function will return a handle to a geometry stream, which can be
passed to {\vt GsReadObject}.

%------------------------------------
% 100408
\index{WriteMsg function}
\item{(int) \vt WriteMsg({\it string\/}, {\it skt\_handle\/})}\\
This function will write a message to a socket, adding the proper
network line termination.  The first argument is a string containing
the characters to write.  The second argument is a socket handle
obtained from {\vt Sopen}.  Any trailing line termination will be
stripped from the string, and the network termination ``{\vt
$\backslash$r$\backslash$n}'' will be added.

This function never fails (halts the script).  The return value is the
number of bytes written, or 0 on error.  On error, a message is
available from {\vt GetError}.

\end{description}


\subsection{System Command Interface}

\begin{description}
%------------------------------------
% 030104
\index{Shell function}
\item{(int) \vt Shell({\it command\/})}\\
The {\vt Shell} function will execute {\it command\/} under an
operating system shell.  The {\it command\/} string consists of an
executable name plus arguments, which should be meaningful to the
operating system.  The return value is the return code from the
command, as obtained by the shell.  The function will fail if the {\it
command} string is null or empty.

%------------------------------------
% 030104
\index{System function}
\item{(int) \vt System({\it command\/})}\\
This function sends the {\it command} string to the operating system
for execution.  This is an alias to the {\vt Shell} function.

%------------------------------------
% 100704
\index{getPID function}
\item{(int) \vt GetPID({\it parent\/})}\\
If the boolean argument is zero, this function returns the process ID
of the currently running {\Xic} process.  If the argument is nonzero,
the function returns the process ID of the parent process (typically a
shell).  The process ID is a unique integer assigned by the operating
system.

\end{description}


\subsection{Menu Buttons}

\begin{description}
%------------------------------------
% 030113
\index{SetButtonStatus function}
\item{(int) \vt SetButtonStatus({\it menu}, {\it button}, {\it set\/})}\\
This command sets the state of the specified button in the given menu
or button array, which must be a toggle button.  The button will be
``pressed'' if necessary to match the given state.

The first argument is a string giving the internal name of a menu.  If
the given name is null, empty, or ``{\vt main}'', all of the menus in
the main window will be searched.  The internal menu names are as
follows:

\begin{tabular}{ll}
\vt main       & Main window menus\\
\vt side       & {\cb Side Menu} buttons\\
\vt top        & {\cb Top Menu} buttons\\
\vt sub1       & Wiewport 1 menus\\
\vt sub2       & Wiewport 2 menus\\
\vt sub3       & Wiewport 3 menus\\
\vt sub4       & Wiewport 4 menus\\
\\
\vt file       & {\cb File Menu}\\
\vt cell       & {\cb Cell Menu}\\
\vt edit       & {\cb Edit Menu}\\
\vt mod        & {\cb Modify Menu}\\
\vt view       & {\cb View Menu}\\
\vt attr       & {\cb Attributes Menu}\\
\vt conv       & {\cb Convert Menu}\\
\vt drc        & {\cb DRC Menu}\\
\vt ext        & {\cb Extract Menu}\\
\vt user       & {\cb User Menu}\\
\vt help       & {\cb Help Menu}\\
\end{tabular}

The second argument is the button name, which is the code name given
in the tooltip window which pops up when the mouse pointer rests over
the button.  In the case of {\cb User Menu} command buttons, the name
is the text which appears on the button.  Only buttons and menus
visible in the current mode (electrical or physical) can be accessed.

It should be stressed that the string arguments refer to internal
names, and {\it not} (in general) the label printed on the button. 
For a button, this is the five character or fewer name that is shown
in the tooltip that pops up when the pointer is over the button.  The
same applies to the {\it menu} argument, however these names are not
available from running {\Xic}.  The internal menu names are provided
in the table above.

The identification of the menu is case insensitive.  In the lower
group of entries, only the first one or two characters have to match. 
Thus ``Convert'', ``c'', and ``crazy'' would all select the {\cb
Convert} menu, for example.  One character is sufficient, except for
`e' ({\cb Extract} and {\cb Edit}).  So, the menu argument can be the
menu label, or the internal name, or some simplification at the user's
discretion.  For the upper group, the entire menu name must be given.

If the third argument is nonzero, the button will be pressed if it is
not already engaged.  If the third argument is zero, the button will
be depressed if it is not already disengaged.  The return value is 1 if
the button state changed, 0 if the button state did not change, or -1
if the button was not found.

%------------------------------------
% 030104
\index{GetButtonStatus function}
\item{(int) \vt GetButtonStatus({\it menu}, {\it button\/})}\\
This command returns the status of the indicated menu button, which
should be a toggle button.  The two arguments are as described for
{\vt SetButtonStatus}.  The return value is 1 if the button is
engaged, 0 if the button is not engaged, or -1 if the button is not
found.

%------------------------------------
% 030104
\index{PressButton function}
\item{(int) \vt PressButton({\it menu}, {\it button\/})}\\
This command ``presses'' the indicated button.  This works with all
buttons, toggle or otherwise, and is equivalent to clicking on the
button with the mouse.  The two arguments, which identify the menu and
button, are described under {\vt SetButtonStatus}.  The return value
is 1 if the button was pressed, 0 if the button was not found.
\end{description}

The following four functions send raw events to the window system. 
They are used primarily for the run time logging in the {\vt
xic\_run.log} file.  The run log consists entirely of executable
statements, thus command scripts can be created by simply performing
operations in {\Xic}, and editing the {\vt xic\_run.log} file. 
Otherwise, these functions are not likely to be of much use to most
{\Xic} users.

\begin{description}
%------------------------------------
% 030104
\index{BtnDown function}
\item{\vt BtnDown({\it num}, {\it state}, {\it x}, {\it y}, {\it widget\/})}\\
This function generates a button press event dispatched to the widget
specified by the last argument.  The {\it num} is the button number: 
1 for left, 2 for middle, 3 for right.  The {\it state} is the
``modifier'' key state at the time of the event, and is the OR of 1 if
{\kb Shift} pressed, 4 if {\kb Control} pressed, 8 if {\kb Alt}
pressed, as in X windows.  Other flags may be given as per that spec,
but are not used by {\Xic}.  The coordinates are relative to the
window of the target, in pixels.  The {\it widget} argument is a
string containing a resource specifier for the widget relative to the
application, the syntax of which is dependent upon the specific user
interface.  A call to {\vt BtnDown} should be followed by a call to
{\vt BtnUp} on the same widget.  There is no return value.

%------------------------------------
% 030104
\index{BtnUp function}
\item{\vt BtnUp({\it num}, {\it state}, {\it x}, {\it y}, {\it widget\/})}\\
This function generates a button release event dispatched to the
widget specified by the last argument.  The {\it num} is the button
number:  1 for left, 2 for middle, 3 for right.  The {\it state} is
the ``modifier'' key state at the time of the event, and is the OR of
1 if {\kb Shift} pressed, 4 if {\kb Control} pressed, 8 if {\kb Alt}
pressed, as in X windows.  Other flags may be given as per that spec,
but are not used by {\Xic}.  The coordinates are relative to the
window of the target.  The {\it widget} argument is a string
containing a resource path for the widget relative to the application,
the syntax of which is dependent upon the specific user interface.  A
call to {\vt BtnUp} should only follow a call to {\vt BtnDown} on the
same widget.  There is no return value.

%------------------------------------
% 030104
\index{KeyDown function}
\item{\vt KeyDown({\it keysym}, {\it state}, {\it widget\/})}\\
This function generates a key press event dispatched to the widget
specified in the last argument.  The {\it keysym\/} is a code
representing the key te send.  The {\it state} and {\it widget}
arguments are as described for {\vt BtnDown}.  A call to {\vt KeyDown}
should followed by a call to {\vt KeyUp}, on the same widget.  There
is no return value.

%------------------------------------
% 030104
\index{KeyUp function}
\item{\vt KeyUp({\it keysym\/}, {\it state\/}, {\it widget\/})}\\
This function generates a key release event dispatched to the widget
specified in the last argument.  The {\it keysym\/} is a code
representing the key te send.  The {\it state} and {\it widget}
arguments are as described for {\vt BtnDown}.  A call to {\vt KeyUp}
should only follow a call to {\vt KeyDown}, on the same widget.  There
is no return value.

\end{description}


\subsection{Mouse Input}

\begin{description}
%------------------------------------
% 021310
\index{Point function}
\item{(int) \vt Point({\it array\/})}\\
This function blocks until mouse button 1 (left button) is pressed, or
the {\kb Esc} key is pressed, while the pointer is in a drawing
window.  The coordinates of the pointer at the time of the press are
returned in the array.  The return value is 0 if {\kb Esc} was pressed
or 1 for a button 1 press.  Buttons 2 and 3 have their normal effects
while this function is active, i.e., they are not handled in this
function.

\begin{quotation}
\noindent
Example:
\begin{verbatim}
a[2]
ShowPrompt("Click in a drawing window")
Point(a)
ShowPrompt("x=", a[0], "y=", a[1])
\end{verbatim}
\end{quotation}

When a ghost image is displayed with the {\vt ShowGhost} function, the
coordinates returned are either snapped to the grid or not, depending
on the mode number passed to {\vt ShowGhost}.  If no ghost image is
displayed, the nearest grid point is returned.

If the {\vt UseTransform} function has been called to enable use of
the current transform, the current transform will be applied to the
displayed objects when using mode 8.  The translation supplied to {\vt
UseTransform} is ignored (the translation tracks the mouse pointer).

%------------------------------------
% 110115
\index{Selection function}
\item{(int) \vt Selection()}\\
Block, but allow selections in drawing windows.  Return on any
keypress, or escape event.  Return the number of selected objects in
the selection list.

\end{description}


\subsection{Graphical Input}

\begin{description}
%------------------------------------
% 020109
\index{PopUpInput function}
\item{(string) \vt PopUpInput({\it message\/}, {\it default\/},
 {\it buttontext\/},, {\it multiline\/})}\\
This function will pop up a text-input widget, into which the user can
enter text.  The function blocks until the user presses the
affirmation button, at which time the text is returned, and the pop-up
disappears.  If the user instead presses the {\cb Dismiss} button or
otherwise destroys the pop-up, the script will halt.

The first argument is an explanatory string which is printed on the
pop-up.  If this argument is null or empty, a default message is used. 
Recall that passing 0 is equivalent to passing a null string.

The second argument is a string providing default text which appears
in the entry area when the pop-up appears.  If this argument is null
or empty there will be no default text.
 
The third argument is a string giving text that will appear on the
affirmation button.  If null or empty, the button will show a default
label.

The fourth argument is a boolean that when nonzero, a multi-line text
input widget will be used.  Otherwise, a single-line input widget will
be used.

%------------------------------------
% 020109
\index{PopUpAffirm function}
\item{(int) \vt PopUpAffirm({\it message\/})}\\
This button pops up a small window which allows the user to answer yes
or no to a question.  Deleting the window is equivalent to answering
no.  The argument is a string which should contain the text to which
the user responds.  When the user responds, the pop-up disappears, and
the return value is 1 if the user answered ``yes'', 0 otherwise.

%------------------------------------
% 020109
\index{PopUpNumeric function}
\item{(real) \vt PopUpNumeric({\it message\/}, {\it initval\/},
 {\it minval\/}, {\it maxval\/}, {\it delta\/}, {\it numdgt\/})}\\
This function pops up a small window which contains a ``spin button''
for numerical entry.  The user is able to enter a number directly, or
by clicking on the increment/decrement buttons.

The first argument is a string providing explanatory text.  The
second argument provides the initial numeric value.  The
{\it minval} and {\it maxval} arguments are the minimum and
maximum allowed values.  The {\it delta} argument is the delta to
increment or decrement when the user presses the up/down buttons.
These parameters are all real values.  The {\it numdgt} is an
integer value which sets how many places to the right of a decimal
point are shown.

If the user presses {\cb Apply}, the pop-up disappears, and this
function returns the current value.  If the user presses the {\cb
Dismiss} button or otherwise destroys the widget, the script will
halt.
\end{description}


\subsection{Text Input}

\begin{description}
%------------------------------------
% 030204
\index{AskReal function}
\item{(scalar) \vt AskReal({\it prompt}, {\it default\/})}\\
The two arguments are both strings, or 0 (equivalent to the predefined
constant {\vt NULL}).  The function will print the strings on the
prompt line, and the user will type a response.  The response is
converted to a real number which is returned by the function.  If
either argument is null, that part of the message is not printed.  The
{\it prompt} is immutable, but the {\it default} can be edited by the
user. 
\begin{quotation}
\noindent
Example:
\begin{verbatim}
a = AskReal("enter a value for a ", "2.5")
\end{verbatim}
\end{quotation}

%------------------------------------
% 030204
\index{AskString function}
\item{(string) \vt AskString({\it prompt\/}, {\it default\/})}\\
The two arguments and the return value are strings.  Similar to
the {\vt AskReal} function, however a string is returned.
\begin{quotation}
\noindent
Example:
\begin{verbatim}
title = AskString("Enter your title: ", "Senior Computer Geek")
\end{verbatim}
\end{quotation}

%------------------------------------
% 030204
\index{AskConsoleReal function}
\item{(scalar) \vt AskConsoleReal({\it prompt\/}, {\it default\/})}\\
This function prompts the user for a number, in the console window. 
It is otherwise similar to the {\vt AskReal} function. 

%------------------------------------
% 030204
\index{AskConsoleString function}
\item{(string) \vt AskConsoleString({\it prompt\/}, {\it default\/})}\\
This function prompts the user for a string, in the console window. 
It is otherwise similar to the {\vt AskString} function.

%------------------------------------
% 030204
\index{GetKey function}
\item{(int) \vt GetKey()}\\
This function blocks until any key is pressed.  The return value is a
key code, which is system dependent, but is generally the ``keysym''
of the key pressed.  If the value is less than 20, the value is an
internal code.

\end{description}


\subsection{Text Output}

\begin{description}
%------------------------------------
% 030104
\index{SepString function}
\item{(string) \vt SepString({\it string\/}, {\it repeat\/})}\\
This function returns a string that is created by repeating the {\it
string} argument {\it repeat} times.  The {\it repeat} value is an
integer in the range 1--132.  The function will fail if {\it string} is
null.

%------------------------------------
% 030204
\index{ShowPrompt function}
\item{(int) \vt ShowPrompt({\it arg\_list\/})}\\
Print the values of the arguments on the prompt line.  The
number of arguments is variable.
\begin{quotation}
\noindent
Example:
\begin{verbatim}
a = 2.5
b = "the value of a is "
ShowPrompt(b, a)
\end{verbatim}
\end{quotation}
This code fragment will print ``{\vt the value of a is 2.5}'' on the
prompt line.

If given without arguments, the prompt line will be erased, but
without disturbing the current message as returned with {\vt
GetLastPrompt}.  The function returns 1 if something is printed
(message updated), 0 otherwise.

%------------------------------------
% 030204
\index{SetIndent function}
\item{(int) \vt SetIndent({\it level\/})}\\
This function sets the indentation level used for printing with the
{\vt Print} and {\vt PrintLog} functions.  The argument is an integer
which specifies the column where printed output will start.  The
argument can also be a string in one of the following formats:

\begin{description}
\item{\vt "+{\it N\/}"}\\
{\it N} is an optional integer (default 1), increases indentation
by {\it N} columns.
\item{\vt "-{\it N\/}"}\\
{\it N} is an optional integer (default 1), decreases indentation
by {\it N} columns.
\item{\vt ""}\\
Empty string, does not change indentation.
\end{description}

The function returns the previous indentation level.

%------------------------------------
% 030204
\index{SetPrintLimits function}
\item{(int) \vt SetPrintLimits({\it num\_array\_elts}, {\it max\_zoids\/})}\\
While printing with the {\vt Print} family of functions, or when using
{\vt ListHandle}, the number of array points and trapezoids actually
printed is limited.  The default limits are 100 array points and 20
trapezoids.  This function allows these limits to be changed.  A value
for either argument of -1 will remove any limit, 0 will keep the
present limit, non-negative values will set the limit, and negative
values of -2 or less will revert to the default values.  This function
always returns 1 and never fails.

%------------------------------------
% 030204
\index{Print function}
\item{(int) \vt Print({\it arg\_list\/})}\\
This function will print the arguments on the console.  This is the
window from which {\Xic} was launched.  The number of arguments is
variable.  The printing is indented according to the level set with
the {\vt SetIndent} function.

Any type of variable can be printed.  Handles will be printed as a
string giving the handle type.  For a zoidlist variable, the
coordinates of the trapezoids are printed, one trapezoid per line, in
order x-lower-left, x-lower-right, y-lower, x-upper-left,
x-upper-right, y-upper.  Arrays are printed as a sequence of numbers. 
The number of array elements and trapezoids printed is limited to 100
and 20, respectively, but these limits can be changed or removed with
the {\vt SetPrintLimits} function.

%------------------------------------
% 030204
\index{PrintLog function}
\item{(int) \vt PrintLog({\it file\_handle\/}, {\it arg\_list\/})}\\
This works like the {\vt Print} function, however output goes to a
file previously opened for writing with the {\vt Open} function.  The
first argument is the file handle returned from {\vt Open}.  Following
arguments are printed to the file in order, using indentation set with
the {\vt SetIndent} function.  The function returns the number of
characters written.  The function will fail if the handle is not a
file handle.

%------------------------------------
% 030204
\index{PrintString function}
\item{(string) \vt PrintString({\it arg\_list\/})}\\
This works like the {\vt Print}, etc.  functions, however it returns a
string containing the text, and indentation as set with {\vt
SetIndent} is ignored.

%------------------------------------
% 030204
\index{PrintStringEsc function}
\item{(string) \vt PrintStringEsc({\it arg\_list\/})}\\
This works exactly like {\vt PrintString}, however, special characters
in any string supplied as an argument are shown in their
`$\backslash$' escape form.

%------------------------------------
% 030204
\index{Message function}
\item{(int) \vt Message({\it arg\_list\/})}\\
This function will print the arguments in a pop-up message window,
indentation is ignored.

%------------------------------------
% 030204
\index{ErrorMsg function}
\item{(int) \vt ErrorMsg({\it arg\_list\/})}\\
This function will print the arguments in a pop-up error window,
indentation is ignored.

%------------------------------------
% 030204
\index{TextWindow function}
\item{(int) \vt TextWindow({\it fname\/}, {\it readonly\/})}\\
This function brings up a text editor window loaded with the file
whose path is given in the {\it fname} string.  If the integer {\it
readonly} is 0, editing of the file is enabled, otherwise editing is
prevented.

\end{description}


%------------------------------------------------------------------------------
\section{Main Functions 3}

% 101812
Many of the layer-related functions take a ``standard layer
argument''.  This can be an integer index number into the layer table,
where the index is 1-based, and values less than 1 return the current
layer.  The argument can also be a string, giving a layer name in {\it
layer\/}[{\vt :}{\it purpose\/}] form, or an alias name.  If the
string is null or empty, the current layer is returned.

\subsection{Grid and Edge Snapping}

\begin{description}
%------------------------------------
% 101412
\index{SetMfgGrid function}
\item{(int) \vt SetMfgGrid({\it mfg\_grid\/})}\\
This will set the manufacturing grid to the value of the argument,
provided that the value is in the range 0.0 -- 100.0 microns.  When
the manufacturing grid is nonzero, the snap grid is constrained to
integer multiples of the manufacturing grid.  The function returns 1
if the argument is in range, in which case the value is accepted, 0
otherwise.

%------------------------------------
% 101412
\index{GetMfgGrid function}
\item{(real) \vt GetMfgGrid()}\\
This function returns the value of the manufacturing grid.  When
nonzero, the snap grid is constrained to integer multiples of the
manufacturing grid.

%------------------------------------
% 101412
\index{SetGrid function}
\item{(int) \vt SetGrid({\it interval\/}, {\it snap},  {\it win\/})}\\
This function sets the grid parameters for the window indicated by the
third argument, which is 0 for the main window or 1--4 for the
sub-windows.  The interval argument sets snap grid spacing, in
microns.  This value can be zero, in which case the present value is
retained.

The snap value is an integer in the range of -10 to 10.  If positive,
the number provides the number of snap grid intervals between fine
grid lines.  If negative, the absolute value is the number of fine
grid lines displayed per snap grid interval.  If zero, the present
setting is retained.
 
For electrical mode windows, the snap points must be on multiples of
one micron.  If not, this function returns 0 and the grid is
unchanged.  The function also returns 0 if the window argument does
not correspond to an existing window.  The return is 1 if the
operation succeeds.

The function does not redraw the window.  The {\vt Redraw()} function
can be called to redraw the window if necessary.

%------------------------------------
% 101412
\index{GetGridInterval function}
\item{(real) \vt GetGridInterval({\it win\/})}\\
This function returns the fine grid interval in microns for the grid
in the window indicated by the argument, which is 0 for the main
window or 1--4 for the sub-windows.  The function returns 0 if the
argument does not correspond to an existing window.

%------------------------------------
% 101412
\index{GetSnapInterval function}
\item{(real) \vt GetSnapInterval({\it win\/})}\\
This function returns the snap grid interval in microns for the grid
in the window indicated by the argument, which is 0 for the main
window or 1--4 for the sub-windows.  The function returns 0 if the
argument does not correspond to an existing window.

%------------------------------------
% 101412
\index{GetGridSnap function}
\item{(int) \vt GetGridSnap({\it win\/})}\\
This function returns the snap number for the grid in the window
specified by the argument, which is 0 for the main window or 1--4 for
the sub-windows.  The snap number determines the number of snap grid
intervals between fine grid lines if positive, or fine grid lines per
snap interval if negative.  The function returns 0 if the argument
does not correspond to an existing window.

%------------------------------------
% 121508
\index{ClipToGrid function}
\item{(int) \vt ClipToGrid({\it coord}, {\it win\/})}\\
The first argument to this function is a coordinate in microns.  The
return value is the coordinate, in microns, snapped to the nearest
snap point of the grid of the window given in the second argument. 
The second argument is 0 for the main window, or 1--4 for the
sub-windows.  The function fails if the window argument does not
correspond to an existing window.

Note that this function must be called twice for an x,y coordinate
pair.  This function ignores the edge-snapping modes, only taking into
account the grid resolution and snap values.

%------------------------------------
% 012815
\index{SetEdgeSnappingMode function}
\item{(int) \vt SetEdgeSnappingMode({\it win}, {\it mode\/})}\\
Change the edge snapping mode in a drawing window.  The first argument
is an integer representing the drawing window:  0 for the main window,
and 1--4 for subwindows.  The change will apply only to that window,
though changes in the main window will apply to new sub-windows.  The
second argument is an integer in the range 0--2.  The effects are

\begin{quote}
\begin{tabular}{ll}
0 & No edge snapping.\\
1 & Edge snapping is enabled in some commands.\\
2 & Edge snapping is always enabled.\\
\end{tabular}
\end{quote}
 
The return value is 1 if the window edge snapping was updated, 0
otherwise. 
 
%------------------------------------
% 012815
\index{SetEdgeOffGrid function}
\item{(int) \vt SetEdgeOffGrid({\it win}, {\it off\_grid\/})}\\
This will enable snapping to off-grid locations when edge snapping is
enabled, in the given window.  The first argument is an integer
representing the drawing window:  0 for the main window, and 1--4 for
subwindows.  The second argument is a boolean which will allow
off-grid snapping when true.  The return value is 1 if the window
parameter was updated, 0 otherwise.

%------------------------------------
% 012815
\index{SetEdgeNonManh function}
\item{(int) \vt SetEdgeNonManh({\it win}, {\it non\_manh\/})}\\
This will enable snapping to non-Manhattan edges when edge snapping is
enabled, in the given window.  The first argument is an integer
representing the drawing window:  0 for the main window, and 1--4 for
subwindows.  The second argument is a boolean which will allow
snapping to non-Manhattan edges when true.  The return value is 1 if
the window parameter was updated, 0 otherwise.

%------------------------------------
% 012815
\index{SetEdgeWireEdge function}
\item{(int) \vt SetEdgeWireEdge({\it win}, {\it wire\_edge\/})}\\
This will enable snapping to wire edges when edge snapping is enabled,
in the given window.  The first argument is an integer representing
the drawing window:  0 for the main window, and 1--4 for subwindows. 
The second argument is a boolean which will allow snapping to wire
edges when true.  The return value is 1 if the window parameter was
updated, 0 otherwise.

%------------------------------------
% 012815
\index{SetEdgeWirePath function}
\item{(int) \vt SetEdgeWirePath({\it win}, {\it wire\_path\/})}\\
This will enable snapping to the wire path when edge snapping is
enabled, in the given window.  The path is the set of line segments
that invisibly run along the center of the displayed wire, which,
along with the wire width and end style, actually defines the wire. 
The first argument is an integer representing the drawing window:  0
for the main window, and 1--4 for subwindows.  The second argument is
a boolean which will allow snapping to the wire path when true.  The
return value is 1 if the window parameter was updated, 0 otherwise.

%------------------------------------
% 012815
\index{GetEdgeSnappingMode function}
\item{(int) \vt GetEdgeSnappingMode({\it win})}\\
This function returns the edge snapping mode in effect for the given
window.  The argument is an integer representing the drawing window: 
0 for the main window, and 1--4 for subwindows.  The return value is
-1 if the window is not found, 0-2 otherwise.

\begin{quote}
\begin{tabular}{ll}
0 & No edge snapping.\\
1 & Edge snapping is enabled in some commands.\\
2 & Edge snapping is always enabled.\\
\end{tabular}
\end{quote}

%------------------------------------
% 012815
\index{GetEdgeOffGrid function}
\item{(int) \vt GetEdgeOffGrid({\it win})}\\
This returns the setting of the allow off-grid edge snapping flag for
the given window.  The argument is an integer representing the drawing
window:  0 for the main window, and 1-4 for subwindows.  The return
value is -1 if the window is not found, 0 or 1 otherwise tracking the
state of the flag.

%------------------------------------
% 012815
\index{GetEdgeNonManh function}
\item{(int) \vt GetEdgeNonManh({\it win})}\\
This returns the setting of the allow non-Manhattan edge snapping flag
for the given window.  The argument is an integer representing the
drawing window:  0 for the main window, and 1--4 for subwindows.  The
return value is -1 if the window is not found, 0 or 1 otherwise
tracking the state of the flag.

%------------------------------------
% 012815
\index{GetEdgeWireEdge function}
\item{(int) \vt GetEdgeWireEdge({\it win})}\\
This returns the setting of the allow wire-edge edge snapping flag for
the given window.  The argument is an integer representing the drawing
window:  0 for the main window, and 1--4 for subwindows.  The return
value is -1 if the window is not found, 0 or 1 otherwise tracking the
state of the flag.

%------------------------------------
% 012815
\index{GetEdgeWirePath function}
\item{(int) \vt GetEdgeWirePath({\it win})}\\
This returns the setting of the allow wire-path edge snapping flag for
the given window.  The argument is an integer representing the drawing
window:  0 for the main window, and 1--4 for subwindows.  The return
value is -1 if the window is not found, 0 or 1 otherwise tracking the
state of the flag.

%------------------------------------
% 012815
\index{SetRulerSnapToGrid function}
\item{(int) \vt SetRulerSnapToGrid({\it snap})}\\
This function sets the snap-to-grid behavior when creating rulers in
the {\cb Rulers} command.  When set, the mouse cursor will snap to
grid locations, otherwise not.  In either case the cursor may snap to
object edges if edge snapping is enabled.  If the {\cb Rulers} command
is active the mode will change immediately, otherwise the new mode
will apply when the command becomes active.  The return value is 0 or
1 representing the previous flag value.

%------------------------------------
% 012815
\index{SetRulerEdgeSnappingMode function}
\item{(int) \vt SetRulerEdgeSnappingMode({\it mode})}\\
This sets the edge snapping mode which is applied during the {\cb
Rulers} command.  This command has its own default edge snapping
state.  This function changes only the initial state when the command
starts, and will have no effect in a running command (use {\vt
SetEdgeSnappingMode} to alter the current setting).  The argument is
an integer 0--2.

\begin{quote}
\begin{tabular}{ll}
0 & No edge snapping.\\
1 & Edge snapping is enabled in some commands.\\
2 & Edge snapping is always enabled.\\
\end{tabular}
\end{quote}

The function returns -1 if the argument is out of range, or 0--2
representing the previous state otherwise.

%------------------------------------
% 012815
\index{SetRulerEdgeOffGrid function}
\item{(int) \vt SetRulerEdgeOffGrid({\it off\_grid})}\\
This sets the edge snapping allow off-grid flag which is applied
during the {\cb Rulers} command.  This command has its own default
edge snapping state.  This function changes only the initial state
when the command starts, and will have no effect in a running command
(use {\vt SetEdgeOffGrid} to alter the current setting).  The argument
is a boolean value which enables the flag when true.

The return value is 0 or 1 representing the previous flag state.

%------------------------------------
% 012815
\index{SetRulerEdgeNonManh function}
\item{(int) \vt SetRulerEdgeNonManh({\it non\_manh})}\\
This sets the edge snapping allow non-Manhattan flag which is applied
during the {\cb Rulers} command.  This command has its own default
edge snapping state.  This function changes only the initial state
when the command starts, and will have no effect in a running command
(use {\vt SetEdgeNonManh} to alter the current setting).  The argument
is a boolean value which enables the flag when true.

The return value is 0 or 1 representing the previous flag state.

%------------------------------------
% 012815
\index{SetRulerEdgeWireEdge function}
\item{(int) \vt SetRulerEdgeWireEdge({\it wire\_edge})}\\
This sets the edge snapping allow wire-edge flag which is applied
during the {\cb Rulers} command.  This command has its own default
edge snapping state.  This function changes only the initial state
when the command starts, and will have no effect in a running command
(use {\vt SetEdgeWireEdge} to alter the current setting).  The
argument is a boolean value which enables the flag when true.

The return value is 0 or 1 representing the previous flag state.

%------------------------------------
% 012815
\index{SetRulerEdgeWirePath function}
\item{(int) \vt SetRulerEdgeWirePath({\it wire\_path})}\\
This sets the edge snapping allow wire-path flag which is applied
during the {\cb Rulers} command.  This command has its own default
edge snapping state.  This function changes only the initial state
when the command starts, and will have no effect in a running command
(use {\vt SetEdgeWirePath} to alter the current setting).  The
argument is a boolean value which enables the flag when true.

The return value is 0 or 1 representing the previous flag state.

%------------------------------------
% 012815
\index{GetRulerSnapToGrid function}
\item{(int) \vt GetRulerSnapToGrid()}\\
This returns the present default snap-to-grid state used during the
{\cb Rulers} command.  The values are 0 or 1 depending on the state.

%------------------------------------
% 012815
\index{GetRulerEdgeSnappingMode function}
\item{(int) \vt GetRulerEdgeSnappingMode()}\\
The return value is an integer 0-2 representing the default edge
snapping mode to use during the {\cb Rulers} command.

\begin{quote}
\begin{tabular}{ll}
0 & No edge snapping.\\
1 & Edge snapping is enabled in some commands.\\
2 & Edge snapping is always enabled.\\
\end{tabular}
\end{quote}

%------------------------------------
% 012815
\index{GetRulerEdgeOffGrid function}
\item{(int) \vt GetRulerEdgeOffGrid()}\\
The return value is 0 or 1 depending on the setting of the edge
snapping allow off-grid flag which is the default in the {\cb Rulers}
command.

%------------------------------------
% 012815
\index{GetRulerEdgeNonManh function}
\item{(int) \vt GetRulerNonManh()}\\
The return value is 0 or 1 depending on the setting of the edge
snapping allow non-Manhattan flag which is the default in the {\cb
Rulers} command.

%------------------------------------
% 012815
\index{GetRulerEdgeWireEdge function}
\item{(int) \vt GetRulerEdgeWireEdge()}\\
The return value is 0 or 1 depending on the setting of the edge
snapping allow wire-edge flag which is the default in the {\cb Rulers}
command.

%------------------------------------
% 012815
\index{GetRulerEdgeWirePath function}
\item{(int) \vt GetRulerEdgeWirePath()}\\
The return value is 0 or 1 depending on the setting of the edge
snapping allow wire-path flag which is the default in the {\cb Rulers}
command.
\end{description}

\subsection{Grid Style}

\begin{description}
%------------------------------------
% 030204
\index{ShowGrid function}
\item{(int) \vt ShowGrid({\it on\/}, {\it win\/})}\\
This function sets whether or not the grid is shown in a window.  If
the first argument is nonzero, the grid will be shown, otherwise the
grid will not be shown.  The second argument is an integer
representing the drawing window:  0 for the main window, and 1--4 for
sub-windows.  The change will not be visible until the window is
redrawn (one can call {\vt Redraw}).  If success, 1 is returned, or 0
is returned if the window does not exist.

%------------------------------------
% 030204
\index{ShowAxes function}
\item{(int) \vt ShowAxes({\it style\/}, {\it win\/})}\\
This function sets the axes presentation style in physical mode
windows.  The first argument is an integer 0--2, where 0 suppresses
drawing of axes, 1 indicates plain axes, and 2 (or anything else)
indicates axes with a box at the origin.  The second argument is an
integer representing the drawing window:  0 for the main window, 1--4
for sub-windows.  Axes are never shown in electrical mode windows.  On
success, 1 is returned.  If the window does not exist or is not
showing a physical view, 0 is returned.  The change will not be
visible until the window is redrawn (one can call {\vt Redraw}).

%------------------------------------
% 030204
\index{SetGridStyle function}
\item{(int) \vt SetGridStyle({\it style}, {\it win\/})}\\
This function sets the line style used for grid rendering.  The first
argument is an integer mask that defines the on-off pattern.  The
pattern starts at the most significant `1' bit and continues through
the least significant bit, and repeats.  Set bits are rendered as the
visible part of the pattern.  If the style is 0, a dot is shown at
each grid point.  Passing -1 will give continuous lines.  The second
argument is an integer representing the drawing window:  0 for the
main window, 1--4 for sub-windows.  The function returns 1 on success, 0
if the window does not exist.  The change will not be visible until
the window is redrawn (one can call {\vt Redraw}).

%------------------------------------
% 030204
\index{GetGridStyle function}
\item{(int) \vt GetGridStyle({\it win\/})}\\
This function returns the line style mask used for rendering the grid
in the given window.  The mask has the interpretation described in the
description of {\vt SetGridStyle}.  The argument is an integer
representing the window:  0 for the main window, and 1--4 for
sub-windows.  If the window does not exist, 0 is returned.

%------------------------------------
% 071110
\index{SetGridCrossSize function}
\item{(int) \vt SetGridCrossSize({\it xsize}, {\it win\/})}\\
This applies only to grids with style 0 (dot grid).  The {\it xsize}
is an integer 0--6 which indicates the number of pixels to draw in the
four compass directions around the central pixel.  Thus, for nonzero
values, the ``dot'' is rendered as a small cross.  The second argument
is an integer representing the drawing window:  0 for the main window,
1--4 for subwindows.  The function returns 1 on success, 0 if the
window does not exist or the style is nonzero.  The change will not be
visible until the window is redrawn (one can call {\vt Redraw}).

%------------------------------------
% 071110
\index{GetGridCrossSize function}
\item{(int) \vt GetGridCrossSize({\it win\/})}\\
This returns an integer 0--6, which will be nonzero only for grid
style 0 (dot grid), and if the ``dots'' are being rendered as small
crosses via a call to {\vt SetGridCrossSize} or otherwise.  The
argument is an integer representing the window:  0 for the main
window, and 1--4 for subwindows.  If the window does not exist, 0 is
returned.

%------------------------------------
% 030204
\index{SetGridOnTop function}
\item{(int) \vt SetGridOnTop({\it ontop}, {\it win\/})}\\
This function sets whether the grid is shown above or below rendered
objects.  If the first argument is nonzero, the grid will be shown
above rendered objects.  The second argument is an integer
representing the drawing window:  0 for the main window and 1--4 for
sub-windows.  The function returns 1 on success, 0 if the window does
not exist.  The change will not be visible until the window is redrawn
(one can call {\vt Redraw}).

%------------------------------------
% 030204
\index{GetGridOnTop function}
\item{(int) \vt GetGridOnTop({\it win\/})}\\
This function returns 1 is the grid is shown on top of objects.  The
argument is an integer representing the drawing window:  0 for the
main window and 1--4 for sub-windows.  If the grid is shown below
rendered objects, 0 is returned.  If the window does not exist, -1 is
returned.

%------------------------------------
% 101412
\index{SetGridCoarseMult function}
\item{(int) \vt SetGridCoarseMult({\it mult\/}, {\it win\/})}\\
This sets the number of fine grid lines per coarse grid line.  The
first argument is an integer 1--50 that provides this multiple (it is
clipped to this range).  If 1, the coarse grid color is used for all
grid lines.  The second argument represents the drawing window whose
grid is being changed, 0 for the main drawing window, and 1--4 for
sub-windows.  The change will not be visible until the window is
redrawn (one can call {\vt Redraw()}).

The return value is 1 on success, 0 if the window does not exist.

%------------------------------------
% 101412
\index{GetGridCoarseMult function}
\item{(int) \vt GetGridCoarseMult({\it win\/})}\\
This returns the number of fine grid lines per coarse grid interval,
as being used in the drawing window indicated by the argument.  The
argument is 0 for the main drawing window, 1--4 for sub-windows.  If
the window does not exist, zero is returned.

%------------------------------------
% 101412
\index{SaveGrid function}
\item{(int) \vt SaveGrid({\it regnum\/}, {\it win\/})}\\
This will save a grid parameter set to a register.  The first argument
is a register index value 0--7.  Register 0 is used internally for the
``last'' value whenever grid parameters are changed, so is probably
not a good choice unless this behavior is expected.  These are the
same registers as used with the {\cb Grid Setup} panel, and are
associated with the {\vt PhysGridReg} and {\vt ElecGridReg} keyword
families in the technology file.

The second argument represents the drawing window whose grid
parameters are to be saved.  The value is 0 for the main drawing
window, and 1--4 for sub-windows.  Note that separate registers exist
for electrical and physical mode, so register numbers can be reused in
the two modes.

The return value is 1 on success, 0 if the indicated window does not
exist, or the register value is out of range.

%------------------------------------
% 101412
\index{RecallGrid function}
\item{(int) \vt RecallGrid({\it regnum\/}, {\it win\/})}\\
This will recall a grid parameter set from a register, and update the
grid of a drawing window.  The first argument is a register index
value 0--7.  Register 0 is used internally for the ``last'' value
whenever grid parameters are changed, so is probably not a good choice
unless this behavior is expected.  These are the same registers as
used with the {\cb Grid Setup} panel, and are associated with the {\vt
PhysGridReg} and {\vt ElecGridReg} keyword families in the technology
file.

The second argument represents the drawing window whose grid
parameters are to be saved.  The value is 0 for the main drawing
window, and 1--4 for sub-windows.  Note that separate registers exist
for electrical and physical mode, so register numbers can be reused in
the two modes.

The return value is 1 on success, 0 if the indicated window does not
exist.  The change will not be visible until the window is redrawn
(one can call {\vt Redraw()}).

\end{description}


\subsection{Current Layer}

\begin{description}
%------------------------------------
% 101312
\index{GetCurLayer function}
\item{(string) \vt GetCurLayer()}\\
This function returns a string containing the name of the current
layer.  If no current layer is defined, a null string is returned.

%------------------------------------
% 101312
\index{GetCurLayerIndex function}
\item{(int) \vt GetCurLayerIndex()}\\
This function returns the 1-based index of the current layer in the
layer table.  If no current layer is defined, 0 is returned.

%------------------------------------
% 030115
\index{SetCurLayer function}
\item{(int) \vt SetCurLayer({\it stdlyr\/})}\\
This function sets the current layer as indicated by the standard
layer argument.  The return value is the 1-based index of the previous
current layer in the layer table, or 0 if there was no current layer. 
This return can be passed as the argument to revert to the previous
current layer.

%------------------------------------
% 030115
\index{SetCurLayerFast function}
\item{(int) \vt SetCurLayerFast({\it stdlyr\/})}\\
This is like {\vt GetCurLayer}, but there is no visible update, i.e.,
the layer table indication, and the current layer shown in various
pop-ups, is unchanged.  This is for speed when drawing.  When drawing
is finished, this should be called with the original current layer, or
{\vt SetCurLayer} should be called with some layer.  The return value
is the 1-based index of the previous current layer in the layer table,
or 0 if there was no current layer.  This return can be passed as the
argument to revert to the previous current layer. 

%------------------------------------
% 030115
\index{NewCurLayer function}
\item{(int) \vt NewCurLayer({\it stdlyr\/})}\\
If the standard layer argument matches an existing layer, the current
layer is set to that layer.  Otherwise, a new layer is created, if
possible, and the current layer is set to the new layer.  The function
will fail if it is not possible to create a new layer, for example if
the name is not a valid layer name.

If the name is not in the {\it layer\/}{\vt :}{\it purpose} form, any
new layer created will use the default ``{\vt drawing}'' purpose.

The return value is the 1-based index of the previous current layer in
the layer table, or 0 if there was no current layer.  This return can
be passed as the argument to revert to the previous current layer.

%------------------------------------
% 101312
\index{GetCurLayerAlias function}
\item{(string) \vt GetCurLayerAlias()}\\
This function is deprecated, see {\vt GetLayerAlias}.  Return the
alias name of the current layer, or a null string if there is no
alias.

%------------------------------------
% 101312
\index{SetCurLayerAlias function}
\item{(int) \vt SetCurLayerAlias({\it alias\/})}\\
This function is deprecated, see {\vt SetLayerAlias}.  Set the alias
name of the current layer.  Returns 1 on success, 0 otherwise
(possibly indicating a name clash).

%------------------------------------
% 101312
\index{GetCurLayerDescr function}
\item{(string) \vt GetCurLayerDescr()}\\
This function is deprecated, see {\vt GetLayerDescr}.  Return the
description string of the current layer.  This will be null if no
description has been set.

%------------------------------------
% 101312
\index{SetCurLayerDescr function}
\item{(int) \vt SetCurLayerDescr({\it descr\/})}\\
This function is deprecated, see {\vt SetLayerDescr}.  Set the
description string of the current layer.  The return value is always
1.
\end{description}


\subsection{Layer Table}

\begin{description}
%------------------------------------
% 101312
\index{LayersUsed function}
\item{(int) \vt LayersUsed()}\\
This returns a count of the layers in the layer table for the current
display mode.

%------------------------------------
% 101312
\index{AddLayer function}
\item{(int) \vt AddLayer({\it name\/}, {\it index\/})}\\
This adds the named layer to the layer table, in the position
specified by the integer second argument.  If the second argument is
negative, the new layer will be added at the end, above all existing
layers.  If the index is 0, the new layer will be positioned at the
index of the current layer, and the current layer and those above
moved up.  Otherwise, the index is a 1-based index into the layer
table, where the new layer will be inserted.  The layer at that index
and those above will be moved up.

The name can match the name of an existing layer that has been
removed from the layer table.  It can also be a unique new name,
and a new layer will be created.  If the name matches an existing
layer in the table, a new layer will also be created, but with an
internally generated name.

The function will return 0 if it is not possible to create a new
layer, for example if the name is not a valid layer name.  On
success 1 is returned.

If the name is not in the {\it layer\/}{\vt :}{\it purpose} form, any
new layer created will use the default ``{\vt drawing}'' purpose.

%------------------------------------
% 101312
\index{RemoveLayer function}
\item{(int) \vt RemoveLayer({\it stdlyr]/})}\\
This removes the layer indicated by the standard layer argument from
the layer table if found.  This returns 1 if the layer is found and
removed, 0 otherwise.

%------------------------------------
% 101312
\index{RenameLayer function}
\item{(int) \vt RenameLayer({\it oldname\/}, {\it newname\/})}\\
The {\it oldname} is a standard layer argument.  The {\it newname} is
a string providing a new layer/purpose name in the {\it layer\/}[{\vt
:}{\it purpose\/}] form.  If no purpose field is given, the default
``{\vt drawing}'' purpose is assumed.  This renames the layer
specified in {\it oldname} to {\it newname\/}.  The renamed layer will
have any alias name removed.

This fails if {\it oldname} is unresolved or {\it newname} is null,
and returns 0 on error, with an error message available from {\vt
GetError}.

%------------------------------------
% 101312
\index{LayerHandle function}
\item{(stringlist\_handle) \vt LayerHandle({\it down\/})}\\
This function returns a handle to a list of the layer names from the
layer table.  If the argument is 0, the list is in ascending order. 
If the argument is nonzero, the list is in descending order.  The
layers used in the current display mode are listed.

%------------------------------------
% 101312
\index{GenLayers function}
\item{(string) \vt GenLayers({\it stringlist\_handle\/})}\\
This function returns a string containing a layer name from the layer
table.  The argument is the handle returned by {\vt LayerHandle}.  A
different layer is returned for each call.  The null string is
returned after all layers have been cycled through.  This is
equivalent to {\vt ListNext}.

%------------------------------------
% 101312
\index{GetLayerPalette function}
\item{(stringlist\_handle) \vt GetLayerPalette({\it regnum})}\\
The argument is an integer 0--7 corresponding to a layer palette
register, as used with the {\cb Layer Palette} panel, and associated
with the {\vt PhysLayerPalette} and {\vt ElecLayerPalette} technology
file keyword families.  The return value is a stringlist handle, where
the strings are the names of layers saved in the indexed palette
register corresponding to the display mode of the main drawing window.

If the palette register is empty, or the argument is out of range, a
scalar 0 is returned.

The register with index 0 is used internally to save the last {\cb
Layer Palette} user area before it pops down.  Thus, this index should
not be used unless this behavior is expected.

%------------------------------------
% 101312
\index{SetLayerPalette function}
\item{(int) \vt SetLayerPalette({\it list\/}, {\it regnum\/})}\\
The second argument is an integer 0--7 corresponding to a layer
palette register, as used with the {\cb Layer Palette} panel, and
associated with the {\vt PhysLayerPalette} and {\vt ElecLayerPalette}
technology file keyword families.

The first argument provides a list of layers, or null, to be saved
in the indexed palette register corresponding to the display mode
of the main drawing window.  If the argument is a scalar 0, or a
null string, the palette register will be cleared.  Otherwise this
argument can be a string consisting of space-separated layer
names, or a stringlist handle, where the strings are layer names.
The handle is unaffected by this function call.

The function returns 1 on success, 0 if the register index is out
of range.  The call will fail (halt the script) if a bad argument
is passed.

There is no checking of the validity of the string saved as palette
register data.
\end{description}


\subsection{Layer Database}

\begin{description}
%------------------------------------
% 032017
\index{GetLayerNum function}
\item{(int) \vt GetLayerNum({\it name\/})}\\
Return the component layer number given the component layer name. 
This is the {\it layer} part of the general {\it layer\/}[{\vt :}{\it
purpose\/}] layer name used in {\Xic}.  Each such name has a
corresponding number in the database.  If the name is not found, the
return value is -1, which is reserved and is not a valid component
layer number.

%------------------------------------
% 032017
\index{GetLayerName function}
\item{(string) \vt GetLayerName({\it num\/})}\\
Return the component layer name given the component layer number.  If
there is no name associated with the number, a null string is
returned.

%------------------------------------
% 101312
\index{IsPurposeDefined function}
\item{(int) \vt IsPurposeDefined({\it name\/})}\\
This returns 1 if the name matches a known purpose, 0 otherwise.

%------------------------------------
% 101312
\index{GetPurposeNum function}
\item{(int) \vt GetPurposeNum({\vt name\/})}\\
This will return a purpose number associated with the name.  If the
name is not recognized, is null or empty, or matches ``{\vt drawing}''
without case sensitivity, -1 is returned.  This is the {\vt drawing}
purpose number.

%------------------------------------
% 101312
\index{GetPurposeName function}
\item{(string) \vt GetPurposeName({\it num\/})}\\
Return a string giving the purpose name corresponding to the passed
purpose number.  If the purpose number is not recognized, or is the
{\vt drawing} purpose value of -1, a null string is returned.
\end{description}


\subsection{Layers}

\begin{description}
%------------------------------------
% 032017
\index{GetLayerLayerNum function}
\item{(int) \vt GetLayerLayerNum({\it stdlyr\/})}\\
Return the component layer number associated with the layer indicated
by the standard layer argument.

%------------------------------------
% 032017
\index{GetLayerPurposeNum function}
\item{(int) \vt GetLayerPurposeNum({\it stdlyr\/})}\\
Return the purpose number associated with the layer indicated by the
standard layer argument.

%------------------------------------
% 101312
\index{GetLayerAlias function}
\item{(string) \vt GetLayerAlias({\it stdlyr\/})}\\
This function returns a string containing the alias name of the layer
indicated by the standard layer argument.  The string will be null if
no alias is set.

%------------------------------------
% 101312
\index{SetLayerAlias function}
\item{(int) \vt SetLayerAlias({\it stdlyr\/}, {\it alias\/})}\\
This function sets the alias name of the layer indicated by the
standard layer first argument to the string given as the second
argument, as for the {\vt LppName} technology file keyword.  The alias
name is an optional secondary name for a layer/purpose pair.  Most if
not all functions that take a layer name argument will also accept an
alias name.

The alias name will hide other layers if there is a name clash.  This
can be used for layer remapping, but the user must be careful with
this.  Layer name comparisons are case-insensitive.

Unlike the normal layer names, the alias name can have arbitrary
punctuation, embedded white space, etc.  However, leading and trailing
white space is removed, and if the resulting string is empty or null,
the existing alias name (if any) will be removed.

The function returns 1 if the alias name is applied to the layer, 0 if
an error occurs.  It is not possible to set the same name on more than
one layer.

%------------------------------------
% 120114
\index{GetLayerDescr function}
\item{(string) \vt GetLayerDescr({\it stdlyr\/})}\\
This function returns a string containing the description of the layer
indicated by the argument, which is a standard layer argument or
derived layer name string.  If no description has been set, a null
string is returned.

%------------------------------------
% 120114
\index{SetLayerDescr function}
\item{(int) \vt SetLayerDescr({\it stdlyr\/}, {\it descr\/})}\\
This function sets the description of the layer indicated by the first
argument, which is a standard layer argument or a derived layer name
string, to the string given as the second argument.  The description
is an optional text string associated with the layer.  The function
always returns 1.

%------------------------------------
% 101312
\index{IsLayerDefined function}
\item{(int) \vt IsLayerDefined({\it name\/})}\\
The string argument contains a layer name.  This can be the standard
{\it layer\/}[{\vt :}{\it purpose\/}] form, or can be an alias name. 
This function returns 1 if the argument can be resolved as the name of
a layer in the layer table, in the current (electrical/physical) mode. 
If the layer can't be resolved, 0 is returned.  The function will fail
fatally if the argument is null or empty.

%------------------------------------
% 120114
\index{IsLayerVisible function}
\item{(int) \vt IsLayerVisible({\it stdlyr\/})}\\
The function returns 1 if the layer indicated by the argument, which
is a standard layer argument or a derived layer name string, is
currently visible (i.e., the visibility flag is set), 0 otherwise.  If
the layer is derived, the return is the flag status, derived layers
are never actually visible.

%------------------------------------
% 120114
\index{SetLayerVisible function}
\item{(int) \vt SetLayerVisible({\it stdlyr\/}, {\it visible\/})}\\
This will set the visibility of the layer indicated in the first
argument, which is a standard layer argument or a derived layer name
string.  The layer will be visible if the boolean second argument is
nonzero, invisible otherwise.  The previous visibility status is
returned.  If the layer is derived, the flag status is set, however
derived layers are never visible.

%------------------------------------
% 120114
\index{IsLayerSelectable function}
\item{(int) \vt IsLayerSelectable({\it stdlyr\/})}\\
The function returns 1 if the layer indicated by the argument, which
is a standard layer argument or a derived layer name string.  is
currently selectable (i.e., the selectability flag is set), 0
otherwise.

%------------------------------------
% 120114
\index{SetLayerSelectable function}
\item{(int) \vt SetLayerSelectable({\it stdlyr\/}, {\it selectable\/})}\\
This will set the selectability of the layer indicated in the first
argument, which is a standard layer argument or a derived layer name
string.  The layer will be selectable if the boolean second argument
is nonzero, not selectable otherwise.  The previous selectability
status is returned.

%------------------------------------
% 120114
\index{IsLayerSymbolic function}
\item{(int) \vt IsLayerSymbolic({\it stdlyr\/})}\\
The function returns 1 if the layer indicated by the argument, which
is a standard layer argument or a derived layer name string, is
currently symbolic (i.e., the {\vt Symbolic} attribute is set), 0
otherwise.

%------------------------------------
% 120114
\index{SetLayerSymbolic function}
\item{(int) \vt SetLayerSymbolic({\it stdlyr\/}, {\it symbolic\/})}\\
This will set the {\vt Symbolic} attribute of the layer indicated in
the first argument, which is a standard layer argument or a derived
layer name string.  The layer will be symbolic if the boolean second
argument is nonzero, not symbolic otherwise.  The previous symbolic
status is returned.

%------------------------------------
% 120114
\index{IsLayerNoMerge function}
\item{(int) \vt IsLayerNoMerge({\it stdlyr\/})}\\
The function returns 1 if the {\vt NoMerge} attribute is set in the
layer indicated by the argument, which is a standard layer argument or
a derived layer name string, 0 otherwise.

%------------------------------------
% 120114
\index{SetLayerNoMerge function}
\item{(int) \vt SetLayerNoMerge({\it stdlyr\/}, {\it nomerge\/})}\\
This will set the {\vt NoMerge} attribute of the layer indicated in
the first argument, which is a standard layer argument or a derived
layer name string.  The layer will be given the {\vt NoMerge}
attribute if the boolean second argument is nonzero, or the attribute
will be removed if present otherwise.  The previous {\vt NoMerge}
status is returned.

%------------------------------------
% 120114
\index{GetLayerMinDimension function}
\item{(real) \vt GetLayerMinDimension({\it stdlyr\/})}\\
The return value is the {\vt MinWidth} design rule value in microns
for the layer indicated by the argument, which is a standard layer
argument or a derived layer name string.  If there is no {\vt
MinWidth} rule, or the DRC package is not available, 0 is returned.

%------------------------------------
% 120114
\index{GetLayerWireWidth function}
\item{(real) \vt GetLayerWireWidth({\it stdlyr\/})}\\
The function returns the default wire width for the layer indicated by
the argument, which is a standard layer argument or a derived layer
name string.

%------------------------------------
% 120114
\index{AddLayerGdsOutMap function}
\item{(int) \vt AddLayerGdsOutMap({\it stdlyr\/}, {\it layer\_num\/},
  {\it datatype\/})}\\
This function will add a mapping from the layer in the first argument
(a standard layer argument or a derived layer name string) to the given
GDSII layer number and data type.  The layer number and data type are
integers which define the layer in the GDSII world.  When a GDSII file
is written, the present layer will appear on the given layer number
and data type in the GDSII file.  It is possible to have multiple
mappings of the layer, in which case the geometry from the named layer
will appear on each layer number/data type given.

The function returns 1 on success, or 0 if the layer number or
data type number is out of range.  The acceptable range for the
layer number and data type is [0 -- 65535].

%------------------------------------
% 120114
\index{RemoveLayerGdsOutMap function}
\item{(int) \vt RemoveLayerGdsOutMap({\it stdlyr\/}, {\it layer\_num\/},
  {\it datatype\/})}\\
This function will remove a GDSII output layer mapping for the layer
indicated in the first argument (a standard layer argument or a
derived layer name string).  The mapping may have been applied in the
technology file, with the {\cb Tech Parameter Editor} panel from the
{\cb Attributes Menu}, or by calling the {\vt AddLayerGdsOutMap}
function.  The mappings removed match the given layer number and data
type integers provided.  These are in the range [-1 -- 65535], where
the value '-1' indicates a wild-card which will match all layer
numbers or data types.

The return value is -1 if the layer number or data type is out of
range.  Otherwise, the return value is the number of mappings
removed.

%------------------------------------
% 120114
\index{AddLayerGdsInMap function}
\item{(int) \vt AddLayerGdsInMap({\it stdlyr\/}, {\it string\/})}\\
This function adds a GDSII input mapping record to the layer whose
name is indicated in the first argument (a standard layer argument or
a derived layer name string).  The second argument is a string listing
the layer numbers and data types which will map to the named layer, in
the same syntax as used in the technology file.  This is ``{\it l1
l2-l3} ..., {\it d1 d2-d3} ...", where there are two comma separated
fields.  The left field consists of individual layer numbers and/or
ranges of layer numbers, similarly the right field consists of
individual data types and/or ranges of data types.  Each field can
have an arbitrary number of space-separated terms.  For each layer
listed or in a range, all of the data types listed or in a range will
map to the named layer.  There can be multiple input mappings applied
to the named layer.

The function returns 0 if there was a syntax error.  The function
returns 1 if the mapping is successfully added.

%------------------------------------
% 120114
\index{ClearLayerGdsInMap function}
\item{(int) \vt ClearLayerGdsInMap({\it stdlyr\/})}\\
This function deletes all of the GDSII input mappings applied to the
layer indicated in the argument, which is a standard layer argument or
a derived layer name string.  These mappings may have been applied
through the technology file, added with the {\cb Tech Parameter
Editor} from the {\cb Attributes Menu}, or added with the {\vt
AddLayerGdsInMap} function.  This function returns 0 if the layer name
does not exist in the symbol table for the current display mode
(physical or electrical).  Otherwise, the return value is the number
of mapping records deleted.

%------------------------------------
% 120114
\index{SetLayerNoDRCdatatype function}
\item{(int) \vt SetLayerNoDRCdatatype({\it stdlyr\/}, {\it datatype\/})}\\
This function assigns a data type to be used for objects with the DRC
skip flag set.  The first argument is a standard layer argument
indicating a physical layer or a derived layer name string.  The
second argument is the data type in the range [0 -- 65535], or -1.  If
-1 is given, any previously defined data type is cleared.  The
function returns 0 if the layer name can't be resolved, or the data
type is out of range.  The value 1 is returned on success.

\end{description}

\subsection{Layers -- Extraction Support}
% 120114

These functions mainly support the extraction system, but are
maintained in the main program and are therefor accepted in feature
sets where the extraction system is disabled.

Many of the layer-related functions take a ``standard layer
argument''.  This can be an integer index number into the layer table,
where the index is 1-based, and values less than 1 return the current
layer.  The argument can also be a string, giving a layer name in {\it
layer\/}[{\vt :}{\it purpose\/}] form, or an alias name.  If the
string is null or empty, the current layer is returned.

\begin{description}
%------------------------------------
% 120114
\index{SetLayerExKeyword function}
\item{(string) \vt SetLayerExKeyword({\it stdlyr\/}, {\it string\/})}\\
The first argument is a standard layer argument indicating a physical
layer, or a derived layer name string.  The {\it string} argument is
an extraction keyword and associated text, as would appear in a layer
block in the technology file.  The specification will be applied to
the layer, overriding existing settings and possibly causing
incompatible or redundant existing keywords to be deleted.  This is
similar to the editing functions of the {\cb Tech Parameter Editor}
from the {\cb Attributes Menu}, when using the {\cb Extract} or {\cb
Physical} pages.

The return is a status or error string, which may be null.
  
The following keywords can be specified:
\begin{quote}\vt
    Conductor\\
    Routing\\
    GroundPlane\\
    GroundPlaneDark\\
    GroundPlaneClear\\
    TermDefault\\
    Contact\\
    Via\\
    Dielectric\\
    DarkField\\
    Thickness\\
    Rho\\
    Sigma\\
    Rsh\\
    EpsRel\\
    Capacitance\\
    Lambda\\
    Tline\\
    Antenna
\end{quote}

%------------------------------------
% 120114
\index{SetCurLayerExKeyword function}
\item{(string) \vt SetCurLayerExKeyword({\it string\/})}\\
This is similar to {\vt SetLayerExKeyword}, but applies to the current
layer.  This function is deprecated and not recommended for use in new
scripts.

%------------------------------------
% 120114
\index{RemoveLayerExKeyword function}
\item{(int) \vt RemoveLayerExKeyword({\it stdlyr\/}, {\it keyword\/})}\\
The first argument is a standard layer argument indicating a physical
layer, or a derived layer name string.  This will remove the
specification for the extract keyword given in the argument from the
layer.  The argument must be one of the extraction keywords, i.e.,
those listed for {\vt SetCurLayerExKeyword}.  The return value is 1 if
a specification was removed, 0 otherwise.

%------------------------------------
% 120114
\index{RemoveCurLayerExKeyword function}
\item{(int) \vt RemoveCurLayerExKeyword({\it keyword\/})}\\
This is similar to {\vt RemoveLayerExKeyword} but applies to the
current layer.  This function is deprecated and not recommended for
use in new scripts.

%------------------------------------
% 120114
\index{IsLayerConductor function}
\item{(int) \vt IsLayerConductor({\it stdlyr\/})}\\
The function returns 1 if the {\et Conductor} keyword is given or
implied for the layer indicated by the argument, which is a standard
layer argument or a derived layer name string, 0 otherwise.

%------------------------------------
% 120114
\index{IsLayerRouting function}
\item{(int) \vt IsLayerRouting({\it stdlyr\/})}\\
The function returns 1 if the {\et Routing} keyword is given for the
layer indicated by the argument, which is astandard layer argument or
a derived layer name string, 0 otherwise.

%------------------------------------
% 120114
\index{IsLayerGround function}
\item{(int) \vt IsLayerGround({\it stdlyr\/})}\\
The function returns 1 if one of the {\et GroundPlane} keywords was
given for the layer indicated by the argument, which is a standard
layer argument or a derived layer name string, 0 otherwise.

%------------------------------------
% 120114
\index{IsLayerContact function}
\item{(int) \vt IsLayerContact({\it stdlyr\/})}\\
The function returns 1 if the {\et Contact} keyword is given for the
layer indicated by the argument, which is a standard layer argument or
a derived layer name string, 0 otherwise.

%------------------------------------
% 120114
\index{IsLayerVia function}
\item{(int) \vt IsLayerVia({\it stdlyr\/})}\\
The function returns 1 if the {\et Via} keyword is given for the layer
indicated by the argument, which is a standard layer argument or a
derived layer name string, 0 otherwise.

%------------------------------------
% 120114
\index{IsLayerDielectric function}
\item{(int) \vt IsLayerDielectric({\it stdlyr\/})}\\
The function returns 1 if the {\et Dielectric} keyword is given for
the layer indicated by the argument, which is a standard layer
argument or a derived layer name string, 0 otherwise.

%------------------------------------
% 120114
\index{IsLayerDarkField function}
\item{(int) \vt IsLayerDarkField({\it stdlyr\/})}\\
The function returns 1 if the {\et DarkField} keyword is given or
implied for the layer indicated by the argument, which is a standard
layer argument or a derived layer name string, 0 otherwise.

%------------------------------------
% 120114
\index{GetLayerThickness function}
\item{(real) \vt GetLayerThickness({\it stdlyr\/})}\\
The function returns the value of the {\et Thickness} parameter given
for the layer indicated by the argument, which is a standard layer
argument or a derived layer name string.

%------------------------------------
% 120114
\index{GetLayerRho function}
\item{(real) \vt GetLayerRho({\it stdlyr\/})}\\
The function returns the resistivity in ohm-meters of the layer
indicated by the argument, which is a standard layer argument or a
derived layer name string, as given by the {\vt Rho} or {\vt Sigma}
parameters, if given.  If neither of these is given, and {\vt Rsh} and
{\vt Thickness} are given, the return value will be {\vt Rsh}*{\vt
Thickness}.

%------------------------------------
% 120114
\index{GetLayerResis function}
\item{(real) \vt GetLayerResis({\it stdlyr\/})}\\
The function returns the sheet resistance for the layer indicated
by the argument, which is a standard layer argument or a derived layer
name string.  This will be the value of the {\vt Rsh} parameter, if
given, or the values of {\vt Rho}/{\vt Thickness}, if {\vt Rho} or
{\vt Sigma} and {\vt Thickness} are given, or 0 if no value is
available.

%------------------------------------
% 120114
\index{GetLayerEps function}
\item{(real) \vt GetLayerEps({\it stdlyr\/})}\\
The function returns the relative dielectric constant for the layer
indicated by the argument, which is a standard layer argument or
derived layer name string, as given by the {\et EpsRel} parameter if
applied.

%------------------------------------
% 120114
\index{GetLayerCap function}
\item{(real) \vt GetLayerCap({\it stdlyr\/})}\\
The function returns the per-area capacitance for the layer indicated
by the argument, which is a standard layer argument or a derived layer
name string.

%------------------------------------
% 120114
\index{GetLayerCapPerim function}
\item{(real) \vt GetLayerCapPerim({\it stdlyr\/})}\\
The function returns the per-perimeter capacitance for the layer
indicated by the argument, which is a standard layer argument or a
derived layer name string.

%------------------------------------
% 120114
\index{GetLayerLambda function}
\item{(real) \vt GetLayerLambda({\it stdlyr\/})}\\
The function returns the value of the {\et Lambda} parameter for the
layer indicated by the argument, which is a standard layer argument or
a derived layer name string.

\end{description}


\subsection{Selections}

\begin{description}
%------------------------------------
% 100412
\index{SetLayerSpecific function}
\item{(int) \vt SetLayerSpecific({\it state\/})}\\
If the boolean state value is nonzero, all layers except for the
current layer will become unselectable.  Otherwise, all layers will be
set to their default selectability state.  The return value is always
1.

%------------------------------------
% 052409
\index{SetLayerSearchUp function}
\item{(int) \vt SetLayerSearchUp({\it state\/})}\\
This function will set layer-search-up selection mode (see
\ref{selcontrol}) if the argument is nonzero, or normal mode
otherwise.  The return value is 1 or 0 representing the previous
layer-search-up mode status.

%------------------------------------
% 030204
\index{SetSelectMode function}
\item{(string) \vt SetSelectMode({\it ptr\_mode\/}, {\it area\_mode\/},
  {\it sel\_mode\/})}\\
This function allows the various selection modes to be set.  These are
the same modes that can be set with the {\cb Selection Control Panel}
provided by the {\cb layer} button.  If an input value is given as -1,
that particular parameter will be unchanged.  Otherwise, the possible
values are

\begin{tabular}{|l|l|l|} \hline
\it ptr\_mode & \it area\_mode & \it sel\_mode\\ \hline
0 Normal & 0 Normal   & 0 Normal\\ \hline
1 Select & 1 Enclosed & 1 Toggle\\ \hline
2 Modify & 2 All      & 2 Add\\ \hline
         &            & 3 Remove\\ \hline
\end{tabular}

The return value is a string, where the first three characters are the
previous values of {\it ptr\_mode}, {\it area\_mode}, and {\it
sel\_mode} as {\it integers}, not ASCII characters.

%------------------------------------
% 052409
\index{SetSelectTypes function}
\item{(int) \vt SetSelectTypes({\it string\/})}\\
This function allows setting of the object types that can be selected. 
This provides the default selection types, but does not apply to
functions that provide an explicit argument for selection types.
  
The string argument consists of a sequence of characters whose
presence indicates that the corresponding object type is selectable. 
These are:

\begin{quote}
\begin{tabular}{ll}
\vt c & cell instances\\
\vt b & boxes\\
\vt p & polygons\\
\vt w & wires\\
\vt l & labels\\
\end{tabular}
\end{quote}

Other characters are ignored.  If the string is null, empty, or
contains none of the listed characters, all objects are enabled, as if
the string ``{\vt cbpwl}'' was entered.
  
This function always returns 1.

%------------------------------------
% 030204
\index{Select function}
\item{(int) \vt Select({\it left\/}, {\it bottom\/}, {\it right\/},
 {\it top\/}, {\it types\/})}\\
This function performs a selection operation in the rectangle defined
by the first four arguments (given in microns).  The fifth argument is
a string whose characters serve to enable selection of a given type of
object:  `{\vt b}' for boxes, `{\vt p}' for polygons, `{\vt w}' for
wires, `{\vt l}' for labels, and `{\vt c}' for instances.  If this
string is empty or null, then all objects will be selected.  Any
matching object that touches or overlaps the selection box will have
its selection status toggled.  For example,
\begin{quote}
  {\vt Select(-INFINITY, -INFINITY, INFINITY, INFINITY, "c")}
\end{quote}
will select all subcells.

For more complex selections based on object types, etc., the {\vt
TextCmd} function can be used to call the {\cb !select} command.

% 030204
\index{Deselect function}
\item{(int) \vt Deselect()}\\
This function deselects all selected objects.

\end{description}


\subsection{Pseudo-Flat Generator}

\begin{description}
%------------------------------------
% 030204
\index{FlatObjList function}
\item{(object\_handle) \vt FlatObjList({\it l\/}, {\it b\/}, {\it r\/},
   {\it t\/}, {\it depth\/})}\\
This function provides access to the ``pseudo-flat'' object access
functions that are part of internal DRC routines in {\Xic}.  This
enables cycling through objects in the database without regard to the
cell hierarchy.  The first four arguments are the coordinates in
microns of the bounding box to search in.  The {\it depth} is the
search depth, which can be an integer 0 or larger which sets the
maximum depth to search (0 means search the current cell only, 1 means
search the current cell plus the subcells, etc., and a negative
integer sets the depth to search the entire hierarchy).  This argument
can also be a string starting with `{\vt a}' such as ``{\vt a}'' or
``{\vt all}'' which indicates to search the entire hierarchy.

The return value is a list of box, polygon, and wire objects found in
the given region on the current layer.  Label and subcell objects are
never returned.  If {\it depth} is 0, the actual object pointers are
returned in the list, and all of the object manipulation functions are
available.  Otherwise, the list references copies of the actual
objects, transformed to the coordinate space of the current cell.

The copies of the objects can use substantial memory if the list is
very long.  The {\vt FlatObjGen} function provides another access
interface that can use less memory.

%------------------------------------
% 030204
\index{FlatObjGen function}
\item{(handle) \vt FlatObjGen({\it l\/}, {\it b\/}, {\it r\/},
   {\it t\/}, {\it depth\/})}\\
This function provides access to the ``pseudo-flat'' object access
functions that are part of internal DRC routines in {\Xic}.  This
enables cycling through objects in the database without regard to the
cell hierarchy.  The first four arguments are the coordinates in
microns of the bounding box to search in.  The {\it depth} is the
search depth, which can be an integer 0 or larger which sets the
maximum depth to search (0 means search the current cell only, 1 means
search the current cell plus the subcells, etc., and a negative
integer sets the depth to search the entire hierarchy).  This argument
can also be a string starting with `{\vt a}' such as ``{\vt a}'' or
``{\vt all}'' which indicates to search the entire hierarchy.

Similar to {\vt FlatObjList}, objects on the current layer are
returned, but through an intermediate handle rather than through a
list, which can require significant memory.  This function returns a
special handle which is passed to the {\vt FlatGenNext} function to
actually retrieve the objects.  Although this handle can be passed to
the generic handle functions, most of these functions will have no
effect.  {\vt HandleContent} will return 1, or 0 if the handle is
exhausted.  {\vt HandleNext} will advance to the next object without
saving the object.  The other functions will return 0 and do nothing. 
The {\vt Close} function should be called to delete the handle unless
the handle is iterated to completion with {\vt FlatGenNext} or {\vt
HandleNext}.

If {\it depth} is 0, the object pointers returned from {\vt
FlatGenNext} represent the actual object, and all object manipulation
functions are available.  Otherwise, transformed copies of the actual
objects are returned, and there are restrictions on the operations
that can be performed (see \ref{objmanh}).

%------------------------------------
% 030204
\index{FlatObjGenLayers function}
\item{(handle) \vt FlatObjGenLayers({\it l\/}, {\it b\/}, {\it r\/},
   {\it t\/}, {\it depth}, {\it layers\/})}\\
This function is very similar to {\vt FlatObjGen}, however it returns
objects from layers named in the {\it layers} string.  If the string
is null or empty, objects on all layers will be returned.  Otherwise,
the string is a space separated list of layer names.  The names are
expected to match layers in the current display mode.  Names that do
not match any layer are silently ignored, though the function fails if
no layer can be recognized.

%------------------------------------
% 030204
\index{FlatGenNext function}
\item{(object\_handle) \vt FlatGenNext({\it handle\/})}\\
This takes as an argument the handle returned from {\vt FlatObjGen} or
{\vt FlatObjGenLayers}, and returns an object handle which contains a
single object returned from the generator.  If the {\it depth}
argument passed to these functions was nonzero, the objects are
transformed copies.  The returned handles should be closed after use
by calling {\vt Close}, or by calling an iterating function such as
{\vt HandleNext} or {\vt ObjectNext}.

A new handle is returned for each call of this function, until no
further objects are available in which case this function returns 0,
and the handle passed as the argument will be closed.

%------------------------------------
% 030204
\index{FlatGenCount function}
\item{(int) \vt FlatGenCount({\it handle\/})}\\
This function returns the number of objects that can be generated with
the generator handle passed, which must be returned from {\vt
FlatObjGen} or {\vt FlatObjGenLayers}.  Generator handles do not cache
an internal list of objects, so that the number of objects is unknown,
which is why {\vt HandleContent} returns 1 for generator handles. 
This function duplicates the generator context and iterates through
the loop, counting returned objects.  This can be an expensive
operation.

%------------------------------------
% 030204
\index{FlatOverlapList function}
\item{(object\_handle) \vt FlatOverlapList({\it object\_handle},
  {\it touch\_ok}, {\it depth}, {\it layers\/})}\\
This function returns a handle to a list of objects that touch or
overlap the object referenced by the {\it object\_handle} argument. 
If {\it touch\_ok} is nonzero, objects that touch but have zero
overlap area will be included; if {\it touch\_ok} is zero these
objects will be skipped.  The {\it depth} is the search depth, which
can be an integer which sets the maximum depth to search (0 means
search the current cell only, 1 means search the current cell plus the
subcells, etc., and a negative integer sets the depth to search the
entire hierarchy).  This argument can also be a string starting with
`{\vt a}' such as ``{\vt a}'' or ``{\vt all}'' which indicates to
search the entire hierarchy.  If {\it depth} is not 0, the objects
returned are transformed copies, otherwise the actual objects are
returned.  The {\it layer} argument is a string containing
space-separated layer names of the layers to search for objects.  If
this is empty or null, all layers will be searched.  The function
fails if the handle argument is not a handle to an object list.  The
return value is a handle to a list of objects, or 0 if no overlapping
or touching objects are found.

Only boxes, polygons, and wires are returned.  The reference object
can be any object.  If the reference object is a subcell, objects from
within the cell will be returned if {\it depth} is nonzero.

\end{description}


\subsection{Geometry Measurement}

\begin{description}
%------------------------------------
% 030204
\index{Distance function}
\item{(real) \vt Distance({\it x\/}, {\it y\/}, {\it x1\/}, {\it y1\/})}\\
This function computes the distance between two points, given in
microns, returning the distance between the points in microns.

%------------------------------------
% 030204
\index{MinDistPointToSeg function}
\item{(real) \vt MinDistPointToSeg({\it x\/}, {\it y\/}, {\it x1\/},
 {\it y1\/}, {\it x2\/}, {\it y2\/}, {\it aret\/})}\\
This function computes the shortest distance from {\it x,y} to the
line segment defined by the next four arguments.  The {\it aret} is an
array of size at least 4, used for returned coordinates.  If no return
is needed, this argument can be set to 0.  Upon return of a value
greater than 0, the first two values in {\it aret} are {\it x} and
{\it y}, the next two values are the point on the segment closest to
{\it x,y}.  All values are in microns.

%------------------------------------
% 030204
\index{MinDistPointToObj function}
\item{(real) \vt MinDistPointToObj({\it x\/}, {\it y\/},
 {\it object\_handle\/}, {\it aret\/})}\\
This function computes the minimum distance from the point {\it x,y}
to the boundary of the object given by the handle.  The {\it aret} is
an array of size at least 4 for return coordinates.  If the return is
not needed, this argument can be given as 0.  Upon return of a value
greater than 0, the first two values of aret will be {\it x} and {\it
y}, the next two values will be the point on the boundary of the
object closest to {\it x,y}.  The function returns 0 if {\it x,y}
touch or are enclosed in the object.  The function will fail if the
handle is not a reference to an object list.  If there is an internal
error, -1 is returned.  All coordinates are in microns.

%------------------------------------
% 030204
\index{MinDistSegToObj function}
\item{(real) \vt MinDistSegToObj({\it x1\/}, {\it y1\/}, {\it x2\/},
 {\it y2\/}, {\it object\_handle\/}, {\it aret\/})}\\
This function computes the minimum distance from the line segment
defined by the first four arguments to the boundary of the object
given by the handle.  The {\it aret} is an array of size at least 4
for return coordinates.  If the return is not needed, this argument
can be given as 0.  Upon return of a value greater than 0, the first
two values of {\it aret} will be the point on the line segment nearest
the object, the next two values will be the point on the boundary of
the object nearest to the line segment.  The function returns 0 if the
line segment touches or overlaps the object.  The function will fail
if the handle is not a reference to an object list.  If there is an
internal error, -1 is returned.  All coordinates are in microns.

%------------------------------------
% 030204
\index{MinDistObjToObj function}
\item{(real) \vt MinDistObjToObj({\it object\_handle1\/},
 {\it object\_handle2\/}, {\it aret\/})}\\
This function computes the minimum distance between the two objects
referenced by the handles.  The {\it aret} is an array of size at
least 4 for return coordinates.  If the return is not needed, this
argument can be given as 0.  Upon return of a value greater than 0,
the first two values of {\it aret} will be the point on the boundary
of the first object nearest the second object, the next two values
will be the point on the boundary of the second object nearest to the
first object.  The function returns 0 if the objects touch or overlap. 
The function will fail if either handle is not a reference to an
object list.  If there is an internal error, -1 is returned.  All
coordinates are in microns.

%------------------------------------
% 030204
\index{MaxDistPointToObj function}
\item{(real) \vt MaxDistPointToObj({\it x\/}, {\it y\/},
 {\it object\_handle\/}, {\it aret\/})}\\
This function finds the vertex of the object referenced by the handle
farthest from the point {\it x,y} and returns this distance.  The {\it
aret} is an array of size at least 4 for return coordinates.  If the
return is not needed, this argument can be given as 0.  Upon return of
a value greater than 0, the first two values of aret will be {\it x}
and {\it y}, the next two values will be the vertex of the object
farthest from {\it x,y\/}.  The function will fail if the handle is
not a reference to an object list.  If there is an internal error, -1
is returned.  All coordinates are in microns.

%------------------------------------
% 030204
\index{MaxDistObjToObj function}
\item{(real) \vt MaxDistObjToObj({\it object\_handle1\/},
 {\it object\_handle2\/}, {\it aret\/})}\\
This function finds the pair of vertices, one from each object, that
are farthest apart.  Both handles can be the same.  The {\it aret} is
an array of size at least 4 for return coordinates.  If the return is
not needed, this argument can be given as 0.  Upon return of a value
greater than 0, the first two values of {\it aret} will be the vertex
from the first object, the next two values will be the vertex from the
second object.  The function will fail if either handle is not a
reference to an object list.  If there is an internal error, -1 is
returned.  All coordinates are in microns.

%------------------------------------
% 030204
\index{Intersect function}
\item{(int) \vt Intersect({\it object\_handle1\/},
 {\it object\_handle2\/}, {\it touchok\/})}\\
This function determines whether the two objects referenced by the
handles touch or overlap.  The return value is 1 if the objects touch
or overlap, 0 if the objects do not touch or overlap, or -1 if either
handle points to an empty list or some other error occurred.  The
function fails if either handle is not a reference to an object list. 
If the {\it touchok} argument is nonzero, 1 will be returned if the
objects touch but do not overlap.  If touchok is 0, objects must
overlap (have nonzero intersection area) for 1 to be returned.

\end{description}


%------------------------------------------------------------------------------
\section{Layout File Input/Output Functions}
\subsection{Layer Conversion Aliasing}

% 101412
There is provision for a layer aliasing mechanism which is applied
when a data file is read.  This capability is exported through an
interface consisting of the {\et UseLayerAlias} and {\et LayerAlias}
variables, and the script functions described below.

This is different from the {\vt LppName} aliasing which applies to
{\Xic} layers, and is built into the layer database.  The conversion
aliases apply only while a layout file is being read.

\begin{description}
%------------------------------------
% 101412
\index{ReadLayerCvAliases function}
\item{(int) \vt ReadLayerCvAliases({\it handle\_or\_filename\/})}\\
The argument can be either a string giving a file name, or a file
handle as returned from the {\vt Open} function or equivalent (opened
for reading).  This function will read layer aliases, adding the
definitions to the layer alias table.  The format consists of lines of
the form
\begin{quote}
      {\it name\/}$=${\it newname}
\end{quote}
where both {\it name} and {\it newname} are four-character CIF-type
layer names, and there is one definition per line.  Lines with a
syntax error or bad layer name are silently ignored.  When the layer
alias table is active, layers read from an input file will be
substituted, i.e., if a layer named {\it name} is read, it will be
replaced with {\it newname}.  For data formats that use layer number
and datatype numbers, such as GDSII, the layer names should be in the
form of a four or eight-byte hex number, using upper case, where the
left bytes represent the hex value of the layer number, zero padded,
and the right bytes represent the zero padded datatype number.  The
eight-byte form should be used if the layer or datatype is larger than
255.  Alternatively, the decimal form L,D is accepted for layer
tokens, where the decimal layer and datatype numbers are separated by
a comma with no space.

The function returns 1 on success, 0 otherwise.

%------------------------------------
% 101412
\index{DumpLayerCvAliases function}
\item{(int) \vt DumpLayerCvAliases({\it handle\_or\_filename\/})}\\
The argument can be either a string giving a file name, or a file
handle as returned from the {\vt Open} function or equivalent (opened
for writing).  This function will dump the layer alias table.  The
format consists of lines of the form
\begin{quote}
      {\it name\/}$=${\it newname}
\end{quote}
with one definition per line, where {\it name} and {\it newname} are
CIF-type four character layer names, with {\it newname} being the
replacement.  The function returns 1 on success, 0 otherwise.

%------------------------------------
% 101412
\index{ClearLayerCvAliases function}
\item{(int) \vt ClearLayerCvAliases()}\\
This function will remove all entries in the layer alias table.
The function always returns 1.

%------------------------------------
% 101412
\index{AddLayerCvAlias function}
\item{(int) \vt AddLayerCvAlias({\it lname\/}, {\it new\_lname\/})}\\
This function will add the layer name string {\it new\_lname} as an
alias for the layer name string {\it lname} to the layer alias table. 
If an error occurs, or an alias for {\it lname} already exists in the
table (it will not be replaced) the function returns 0.  The function
otherwise returns 1.

%------------------------------------
% 101412
\index{RemoveLayerCvAlias function}
\item{(int) \vt RemoveLayerCvAlias({\it lname\/})}\\
This function removes any alias for {\vt lname} from the layer alias
table.  The function always returns 1.

%------------------------------------
% 101412
\index{GetLayerCvAlias function}
\item{(string) \vt GetLayerCvAlias({\it lname\/})}\\
This function returns a string containing the alias for the passed
layer name string, obtained from the layer alias table.  If no alias
exists for {\it lname}, a null string is returned.

\end{description}


\subsection{Cell Name Mapping}

\begin{description}
%------------------------------------
% 100108
\index{SetMapToLower function}
\item{(int) \vt SetMapToLower({\it state}, {\it rw})}\\
This function sets a flag which causes upper case cell names to be
mapped to lower case when reading, writing, or format converting
archive files.  The first argument is a boolean value which if nonzero
indicates case conversion will be applied, and if zero case conversion
will be disabled.

The second argument is a boolean value that if zero indicates that
case conversion will be applied when reading or format converting
archive files, and nonzero will apply case conversion when writing an
archive file from memory.
 
Within {\Xic}, this flag can also be set from the panels available
from the {\cb Convert Menu}.  The internal effect is to set or clear
the {\et InToLower} or {\et OutToLower} variables.  The return value
is the previous setting of the variable.

%------------------------------------
% 100108
\index{SetMapToUpper function}
\item{(int) \vt SetMapToUpper({\it state}, {\it rw})}\\
This function sets a flag which causes lower case cell names to be
mapped to upper case when reading, writing, or format converting
archive files.  The first argument is a boolean value which if nonzero
indicates case conversion will be applied, and if zero case conversion
will be disabled.

The second argument is a boolean value that if zero indicates that
case conversion will be applied when reading or format converting
archive files, and nonzero will apply case conversion when writing an
archive file from memory.
 
Within {\Xic}, this flag can also be set from the panels available
from the {\cb Convert Menu}.  The internal effect is to set or clear
the {\et InToUpper} or {\et OutToUpper} variables.  The return value
is the previous setting of the variable.

\end{description}


\subsection{Cell Table}
\label{celltab}

\begin{description}
%------------------------------------
% 100108
\index{CellTabAdd function}
\item{(int) \vt CellTabAdd({\it cellname\/}, {\it expand\/})}\\
This function is used to add cell names to the cell table for the
current symbol table.  The {\it cellname} must match a name in the
global string table, which includes all cells read into memory or
referenced by a CHD in memory.

If the boolean argument {\it expand} is nonzero, and the name matches
a cell in the main database, the cell and all of the cells in its
hierarchy will be added to the table, otherwise only the named cell
will be added.  It is not an error to add the same cell more than
once, duplicates will be ignored.

\index{UseCellTab variable}
If the {\it UseCellTab} variable is set, when a Cell Hierarchy Digest
(CHD) is used to process a cell hierarchy for anything other than
reading cells into the main database, cells listed in the cell table
will override cells of the same name in the CHD.  Thus, for example,
one can substitute modified versions of cells as a layout file is
being written.

The return value is 1 if all goes well, 0 if the table is not
initialized or the cell is not found.

%------------------------------------
% 100108
\index{CellTabCheck function}
\item{(int) \vt CellTabCheck({\it cellname\/})}\\
This function returns 1 if {\it cellname} is in the current cell
table.  If the {\it cellname} is valid but {\it cellname} is not in
the table, 0 is returned.  If the cellname is invalid (not a known
cell name) or the cell table is uninitialized, the return value is -1.

%------------------------------------
% 100108
\index{CellTabRemove function}
\item{(int) \vt CellTabRemove({\it cellname\/})}\\
If {\it cellname} is found in the current cell table, it will be
removed.  If the name was found in the table and removed, the return
value is 1, otherwise the function returns 0.

%------------------------------------
% 100108
\index{CellTabList function}
\item{(stringlist\_handle) \vt CellTabList()}\\
This function returns a handle to a list of cell name strings obtained
from the current cell table.  If the table is empty, a scalar 0 is
returned.

%------------------------------------
% 100108
\index{CellTabClear function}
\item{(int) \vt CellTabClear()}\\
This function will clear the current cell table.  The function always
returns 1.

\end{description}


\subsection{Windowing and Flattening}
\begin{description}
%------------------------------------
% 022916
\index{SetConvertFlags function}
\item{(int) \vt SetConvertFlags({\it use\_window}, {\it clip},
     {\it flatten\/}, {\it ecf\_level\/}, {\it rw\/})}\\
This function sets the status of flags used in format conversions and
when writing output.  The first three arguments correspond to the {\cb
Use Window}, {\cb Clip to Window}, and {\cb Flatten Hierarchy} buttons
in the {\cb Format Conversion} panel and similar.  A nonzero integer
value will set the flag, 0 will reset the flag.

The {\it ecf\_level} is an integer 0--3 which sets the empty cell
filtering level, as described for the {\cb Format Conversion} panel in
\ref{ecfilt}.  The values are

\begin{tabular}{ll}
0 & No empty cell filtering.\\
1 & Apply pre- and post-filtering.\\
2 & Apply pre-filtering only.\\
3 & Apply post-filtering only.\\
\end{tabular}

The {\it rw} argument is a boolean value that if zero indicates that
the flags will be applied when converting archive files, as if set
from the {\cb Format Conversion} panel, and also apply to the {\vt
FromArchive} script function.  With {\it rw} nonzero, the flags apply
when writing output with the {\cb Export Control} panel, or when using
the {\vr Export} and {\vt To}{\it XXX} script functions.  In this
case, the {\it no\_empties} flag is ignored, and the windowing is
ignored except when flattening.

The data window can be set with the {\vt SetConvertArea} script
function.  To apply clipping, both the {\it use\_window} and {\it
clip} flags must be set.

This function returns the previous value of the internal variable that
contains the flags.  The two ecf filter bits encode the filtering
level as above.  The bits are:

\begin{tabular}{ll}
\it flatten     & \vt 0x1\\
\it use\_window & \vt 0x2\\
\it clip        & \vt 0x4\\
\it ecf level0  & \vt 0x8\\
\it ecf level1  & \vt 0x10\\
\end{tabular}

%------------------------------------
% 022916
\index{SetConvertArea function}
\item{(int) \vt SetConvertArea({\it l}, {\it b}, {\it r}, {\it t\/},
  {\it rw\/})}\\
This function sets the rectangular area used to filter or clip objects
during format conversion or file writing.  The first four arguments
are the window coordinates in microns, in the coordinate system of the
top level cell, after scaling (if any).

The {\it rw} argument is a boolean value that if zero indicates that
the values will be applied when converting archive files, as if set
from the {\cb Format Conversion} panel, and also apply when using the
{\vt FromArchive} script function.  With {\it rw} nonzero, the values
apply when writing output with the {\cb Export Control} panel, or when
using the {\vt Export} and {\vt To}{\it XXX} script functions.  In
this case, windowing is ignored except when flattening.

Use of the window can be enabled with the
{\vt SetConvertFlags} script function.

The function always returns 1.

\end{description}


\subsection{Scale Factor}
\begin{description}
%------------------------------------
% 022916
\index{SetConvertScale function}
\item{(real) \vt SetConvertScale({\it scale\/}, {\it which\/})}\\
This sets the scale used for conversions.  There are three such
scales, and the one to set is specified by the second argument, which
is an integer 0--2.
\begin{description}
\item{{\it which} = 0}\\
Set the scale used when converting an archive file directly to another
format with the\newline
{\vt FromArchive} script function or similar, or with the {\cb Format
Conversion} panel.
\item{{\it which} = 1}\\
Set the scale used when writing a file with the {\vt Export} and {\vt
To{\it XXX\/}} script functions or similar, or the {\cb Export
Control} panel.
\item{{\it which} = 2}\\
Set the scale used when reading a file into {\Xic} with the {\vt Edit}
or {\vt OpenCell} functions or similar, or from the {\cb Import
Control} panel in {\Xic}.
\end{description}
Script functions that read, write, or convert archive file data will
in general make use of one of these scale factors, however if the
function takes a scale value as an argument, that value will be used
rather than the values set with this function.

The scale argument is a real value in the inclusive range 0.001 --
1000.0.  The return value is the previous scale value.

\end{description}


\subsection{Export Flags}

\begin{description}
%------------------------------------
% 022816
\index{SetStripForExport function}
\item{(int) \vt SetStripForExport({\it state\/})}\\
This function sets the state of the {\cb Strip For Export} flag.  When
set, output from the conversion functions will contain physical
information only.  This should be applied when generating output for
mask fabrication.  See the {\cb Export Control} panel
description for more information.  If the integer argument is nonzero,
the state will be set active.  The return value is the previous state
of the flag.

%------------------------------------
% 022816
\index{SetSkipInvisLayers function}
\item{(int) \vt SetSkipInvisLayers({\it code\/})}\\
This function sets the variable which controls how invisible layers
are treated by the output conversion functions.  Layer visibility is
set by clicking in the layer table with mouse button 2, or through the
{\vt SetLayerVisible} script function.  If {\it code} is 0 or
negative, invisible layers will be converted.  If {\it code} is 1,
invisible physical layers will not be converted.  If {\it code} is 2,
invisible electrical layers will not be converted.  if {\it code} is 3
or larger, both electrical and physical invisible layers will not be
converted.  The return value is the previous code, which represents
the state of the {\et SkipInvisible} variable, and the check boxes in
the {\cb Export Control} panel.

\end{description}


\subsection{Import Flags}

\begin{description}
%------------------------------------
% 022816
\index{SetMergeInRead function}
\item{(int) \vt SetMergeInRead({\it state\/})}\\
This function controls the setting of an internal flag which enables
merging of boxes and coincident objects while a file is being read. 
This flag is set from within {\Xic} in the {\cb Import Control}
panel.  If the integer argument is nonzero, the flag will be set.  The
return value is the previous state of the flag.

\end{description}


\subsection{layout File Format Conversion}

\begin{description}
%------------------------------------
% 100308
\index{FromArchive function}
\item{(int) \vt FromArchive({\it file\_or\_chd}, {\it destination})}\\
This function will read an archive (GDSII, CIF, CGX, or OASIS) file
and translate the contents to another format.  The {\it file\_or\_chd}
argument is a string giving a path to the source archive file, or the
name of a Cell Hierarchy Digest (CHD) in memory.

The type of file written is implied by the {\it destination}.  If the
{\it destination} is null or empty, native cell files will be created
in the current directory.  If the {\it destination} is the name of an
existing directory, native cell files will be created in that
directory.  Otherwise, the extension of the {\it destination}
determines the file type:

\begin{tabular}{ll}
CGX   & \vt .cgx\\
CIF   & \vt .cif\\
GDSII & \vt .gds, .str, .strm, .stream\\
OASIS & \vt .oas\\
\end{tabular}

Only these extensions are recognized, however CGX and GDSII allow
an additional {\vt .gz} which will imply compression.

See the table in \ref{functions} for the features that apply during a
call to this function.

The value 1 is returned on success, 0 otherwise, with possibly an
error message available from {\vt GetError}.

%------------------------------------
% 100108
\index{FromTxt function}
\item{(int) \vt FromTxt({\it text\_file\/}, {\it gds\_file\/})}\\
This function will translate a text file in the format produced by the
{\vt ToTxt} function into a GDSII format file.  This is useful after
text mode editing has been performed on the file, to repair corruption
or incompatibilities.  If {\it gds\_file} is null or empty, the name
is generated from the {\it text\_file} and given a ``{\vt .gds}''
suffix.

%------------------------------------
% 072710
\index{FromNative function}
\item{(int) \vt FromNative({\it dir\_path\/}, {\it archive\_file\/})}\\
This function will translate native cell files found in the directory
given in {\it dir\_path} into an archive file given in the second
argument.  The format of the archive file produced is determined by
the file extension provided, as for the {\vt FromArchive} function. 
All native cell files found in the directory, except those with a
``{\vt .bak}'' extension or whose name is the same as a device library
symbol, are translated and concatenated, independently of any
hierarchical relationship between the cells.

See the table in \ref{functions} for the features that apply during a
call to this function.  The supported manipulations are cell name
aliasing, layer filtering, and scaling.  Windowing manipulations and
flattening are not supported.  If a file named ``{\vt aliases.alias}''
exists in the {\it dir\_path}, it will be used as an input alias list
for conversion.  Each line consists of a native cell name followed by
an alias to be used in the archive file, separated by white space.

The value 1 is returned on success, 0 otherwise, with possibly an
error message available from {\vt GetError}.

\end{description}


\subsection{Export Layout File}

\begin{description}
%------------------------------------
% 100108
\index{SaveCellAsNative function}
\item{(int) \vt SaveCellAsNative({\it cellname\/}, {\it directory\/})}\\
Save the cell named in the first (string) argument, which must exist
in the current symbol table, to a native format file in the {\it
directory\/}.  If the directory string is null or empty (or 0 is
passed for this argument), the cell is saved in the current directory. 

See the table in \ref{features} for the features that apply during a
call to this function.

This functions returns 1 on success, 0 otherwise, with an error
message likely available from {\vt GetError}.

%------------------------------------
% 022816
\index{Export function}
\item{(int) \vt Export({\it filepath\/}, {\it allcells\/})}\\
This function exports design data to a disk file (or files).  It can
perform the same operations as the {\vt To}{\it XXX} functions also
described in this section.  The type of file produced is set by the
extension found on the {\it filepath} string.  Recognized extensions
are

\begin{quote}
\begin{tabular}{ll}
native  & \vt .xic\\
CGX     & \vt .cgx\\
CIF     & \vt .cif\\
GDSII   & \vt .gds, .str, .strm, .stream\\
OASIS   & \vt .oas\\
\end{tabular}
\end{quote}

Only these extensions are recognized, however CGX and GDSII allow an
additional ``{\vt .gz}'' which will imply compression.  For native
cell file output, the {\it filepath} must provide a path to an
existing directory.  If none of the other formats is matched, and the
{\it filepath} exists as a directory, then native cell files will be
written to that directory.  Alternatively, if the {\it filepath} has a
``{\vt .xic}'' extension, and the {\it filepath} with the {\vt .xic}
stripped is an existing directory, or the {\it filepath} including the
{\vt .xic} is an existing directory (checked in this order), again
native cell files will be written to that directory.

The second argument is a boolean.  If false, then the current cell
hierarchy is written to output.  If true, all cells found in the
current symbol table will be written to output.  In either case, by
default cells that are sub-masters or library cells are not written
unless the controlling variables are set, as from the {\cb Export
Control} panel.  The other controls for windowing, flattening,
scaling, and cell name mapping found in this panel apply as well, as
do their underlying variables.  These flags and values can also be set
with the {\vt SetConvertFlags}, {\vt SetConvertArea}, and {\vt
SetConvertScale} functions, and others that apply to output
generation.  When writing all files, any windowing or flattening in
force is ignored.

See the table in \ref{features} for the features that apply during a
call to this function.

The function return 1 on success, 0 otherwise with an error message
available from {\vt GetError}.

%------------------------------------
% 100108
\index{ToXIC function}
\item{(int) \vt ToXIC({\it destination\_dir\/})}\\
The {\vt ToXIC} function will write the current cell hierarchy to disk
files in native format, no questions asked.  The argument is the
directory where the {\Xic} files will be created.  If this argument is
a null or empty string or zero, the {\Xic} files will be created in
the current directory.

See the table in \ref{features} for the features that apply during a
call to this function.

This functions returns 1 on success, 0 otherwise, with an error
message likely available from {\vt GetError}.

%------------------------------------
% 062109
\index{ToCGX function}
\item{(int) \vt ToCGX({\it cgx\_name\/})}\\
This function will write the current cell hierarchy to a CGX format
file on disk.  The argument is the name of the CGX file to create.  If
the {\it cgx\_name} is null or an empty string, the name used will be
the top level cell name suffixed with ``{\vt .cgx}''.

See the table in \ref{features} for the features that apply during a
call to this function.

This functions returns 1 on success, 0 otherwise, with an error
message likely available from {\vt GetError}.

%------------------------------------
% 062109
\index{ToCIF function}
\item{(int) \vt ToCIF({\it cif\_name\/})}\\
This function will write the current cell hierarchy to a CIF format
file on disk.  The argument is the name of the CIF file to create.  If
the {\it cif\_name\/} is null or an empty string, the name used will
be the top level cell name suffixed with ``{\vt .cif}''.

See the table in \ref{features} for the features that apply during a
call to this function.

This functions returns 1 on success, 0 otherwise, with an error
message likely available from {\vt GetError}.

%------------------------------------
% 062109
\index{ToGDS function}
\item{(int) \vt ToGDS({\it gds\_name\/})}\\
This function will write the current cell hierarchy to a GDSII format
file on disk.  The argument is the name of the GDSII file to create. 
If the {\it gds\_name\/} is null or an empty string, the name used
will be the top level cell name suffixed with ``{\vt .gds}''.

See the table in \ref{features} for the features that apply during a
call to this function.

This functions returns 1 on success, 0 otherwise, with an error
message likely available from {\vt GetError}.

%------------------------------------
% 062109
\index{ToGdsLibrary function}
\item{(int) \vt ToGdsLibrary({\it gds\_name}, {\it cellname\_list\/})}\\
This function will create a GDSII file from a list of cells in memory. 
The first argument is the name of the GDSII file to create.  The
second argument is a string consisting of space-separated cell names. 
The cells must be in memory, in the current symbol table.  Both
arguments must provide values as there are no defaults.  The GDSII
file will contain the hierarchy under each cell given, but any cell is
added once only.  The resulting file will in general contain multiple
top-level cells.

See the table in \ref{features} for the features that apply during a
call to this function.

This functions returns 1 on success, 0 otherwise, with an error
message likely available from {\vt GetError}.

%------------------------------------
% 062109
\index{ToOASIS function}
\item{(int) \vt ToOASIS({\it oas\_name\/})}\\
This function will write the current cell hierarchy to an OASIS format
file on disk.  The argument is the name of the OASIS file to create. 
If the {\it oas\_name} is null or an empty string, the name used will
be the top level cell name suffixed with ``{\vt .oas}''.

See the table in \ref{features} for the features that apply during a
call to this function.

This functions returns 1 on success, 0 otherwise, with an error
message likely available from {\vt GetError}.

%------------------------------------
% 100108
\index{ToTxt function}
\item{(int) \vt ToTxt({\it archive\_file\/}, {\it text\_file\/})}\\
This function will create an ASCII text file {\it text\_file\/} from
the contents of the archive file.  The human-readable text file is
useful for diagnostics.  If {\it text\_file} is null or empty, the
name is derived from the {\it archive\_file} and given a ``{\vt
.txt}'' extension.  No output is produced for CIF, since these are
already in readable format.

The third argument is a string, which can be passed to specify the
range of the conversion.  If this argument is passed 0, or the string
is null or empty, the entire archive file will be converted.  The
string is in the form

\begin{quote}
    [{\it start\_offs\/}[{\vt -}{\it end\_offs\/}]] [{\vt -r}
      {\it rec\_count\/}] [{\vt -c} {\it cell\_count\/}]
\end{quote}

The square brackets indicate optional terms.  The meanings are

\begin{description}
\item{\it start\_offs}\\
An integer, in decimal or ``{\vt 0x}'' hex format (a hex integer
preceded by ``{\vt 0x}'').  The printing will begin at the first
record with offset greater than or equal to this value.

\item{\it end\_offs}\\
An integer in decimal or ``{\vt 0x}'' hex format.  If this value is
greater than {\it start\_offs}, the last record printed is at most the
one containing this offset.  If given, this should appear after a
`{\vt -}' character following the {\it start\_offs}, with no space.

\item{\it rec\_count}\\
A positive integer, at most this many records will be printed.

\item{\it cell\_count}\\
A non-negative integer, at most the records for this many cell
definitions will be printed.  If given as 0, the records from the {\it
start\_offs} to the next cell definition will be printed.
\end{description}

See the table in \ref{features} for the features that apply during a
call to this function.

The function returns 1 on success, 0 otherwise with an error message
possibly available from {\vt GetError}.

\end{description}


\subsection{Cell Hierarchy Digest}

% 100108
The Cell Hierarchy Digest (CHD) is a data structure for saving a
description of a cell hierarchy in compact form.  The CHD can be used
to access data in the original file, without having to load the file,
in an efficient manner.  This capability is accessible from a set of
script functions described below.  This capability applies to physical
data only.

\begin{description}
%------------------------------------
% 120110
\index{FileInfo function}
\item{(string) \vt FileInfo({\it filename\/}, {\it handle\_or\_filename\/},
  {\it flags\/})}\\
This function provides information about the archive file given by the
first argument.  If the second argument is a string giving the name of
a file, output will go to that file.  If the second argument is a
handle returned from the {\vt Open} function or similar (opened for
writing), output goes to the handle stream.  In either case, the
return value is a null string.  If the second argument is a scalar 0,
the output will be in the form of a string which is returned.

The third argument is an integer or string which determines the type
of information to return.  If an integer, the bits are flags that
control the possible data fields and printing modes.  The string form
is a space or comma-separated list of text tokens or hex integers. 
The hex numbers or equivalent values for the text tokens are or'ed
together to form the flags integer.
  
This is really just a convenience wrapper around the {\vt ChdInfo}
function.  See the description of that function for a description of
the flags.  In this function, the following keyword flags will show as
follows:

\begin{description}
\item{\vt alias}\\
No aliasing is applied.
\item{\vt flags}\\
The flags will always be 0.
\end{description}

On error, a null string is returned, with an error message likely
available from {\vt GetError}.

%------------------------------------
% 100108
\index{OpenCellHierDigest function}
\item{(chd\_name) \vt OpenCellHierDigest({\it filename\/},
 {\it info\_saved\/})}\\
This function returns an access name to a new Cell Hierarchy Digest
(CHD), obtained from the archive file given as the argument.  The new
CHD will be listed in the {\cb Cell Hierarchy Digests} panel, and the
access name is used by other functions to access the CHD.

See the table in \ref{features} for the features that apply during a
call to this function.  In particular, the names of cells saved in the
CHD reflect any aliasing that was in force at the time the CHD was
created.

The file is opened from the library search path, if a full path is not
provided.  The CHD is a data structure that provides information about
the hierarchy in compact form, and does not use that main database. 
The second argument is an integer that determines the level of
statistical information about the hierarchy saved.  This info is
available from the {\vt ChdInfo} function and by other means.  The
values can be:

\begin{quote}
\begin{tabular}{|l|l|} \hline
0 & No information is saved.\\ \hline
1 & Only total object counts are saved (default).\\ \hline
2 & Object totals are saved per layer.\\ \hline
3 & Object totals are saved per cell.\\ \hline
4 & Objects counts are saved per cell and per layer.\\ \hline
\end{tabular}
\end{quote}

The larger the value, the more memory is required, so it is best to
only save information that will be used.

If the {\vt ChdEstFlatMemoryUse} function will be called from the new
CHD, the per-cell totals {\it must} be specified (value 3 or 4) or the
estimate will be wildly inaccurate.

The CHD refers to physical information only.  On error, a null string
is returned, and an error message may be available with the {\vt
GetError} function.

%------------------------------------
% 012111
\index{WriteCellHierDigest function}
\item{(int) \vt WriteCellHierDigest({\it chd\_name\/}, {\it filename\/},
 {\it incl\_geom\/}, {\it no\_compr\/})}\\
This function will write a disk file representation of the Cell
Hierarchy Digest (CHD) associated with the access name given as the
first argument, into the file whose name is given as the second
argument.  Subsequently, the file can be read with {\vt
ReadCellHierDigest} to recreate the CHD.  The file has no other
use and the format is not documented.

The CHD (and thus the file) contains offsets onto the target archive,
as well as the archive location.  There is no checksum or other
protection currently, so it is up to the user to make sure that the
target archive is not moved or modified while the CHD is
potentially or actually in use.

If the boolean argument {\it incl\_geom} is true, and the CHD has a
linked CGD (as from {\vt ChdLinkCgd}), then geometry records will be
written to the file as well.  When the file is read, a new CGD will be
created and linked to the new CHD.  Presently, the linked CGD must
have memory or file type, as described for {\vt OpenCellGeomDigest}.

The boolean argument {\it no\_compr}, if true, will skip use of
compression of the CHD records.  This is unnecessary and not
recommended, unless compatibility with {\Xic} releases earlier than
3.2.17, which did not support compression, is needed.

The function returns 1 if the file was written successfully, 0
otherwise, with an error message likely available from {\vt GetError}.

%------------------------------------
% 012211
\index{ReadCellHierDigest function}
\item{(string) \vt ReadCellHierDigest({\it filename\/}, {\it cgd\_type\/})}\\
This function returns an access name to a new cell Hierarchy Digest
(CHD) created from the file whose name is passed as an argument.  The
file must have been created with {\vt WriteCellHierDigest}, or with
the {\cb Save} button in the {\cb Cell Hierarchy Digests} panel.

If the file was written with geometry records included, a new Cell
Geometry Digest (CGD) may also be created (with an internally
generated access name), and linked to the new CHD.  If the integer
argument {\it cgd\_type} is 0, a ``memory'' CGD will be created, which
has the compressed geometry data stored in memory.  If {\it cgd\_type}
is 1, a ``file'' CGD will be created, which will use offsets to obtain
geometry from the CHD file when needed.  If {\it cgd\_type} is any
other value, or the file does not contain geometry records, no CGD
will be produced.

On error, a null string is returned, with an error message probably
available from {\vt GetError}.

%------------------------------------
% 100108
\index{ChdList function}
\item{(stringlist\_handle) \vt ChdList()}\\
This function returns a handle to a list of access strings to Cell
Hierarchy Digests that are currently in memory.  The function never
fails, though the handle may reference an empty list.

%------------------------------------
% 100108
\index{ChdChangeName function}
\item{(int) \vt ChdChangeName({\it old\_chd\_name\/},
  {\it new\_chd\_name\/})}\\
This function allows the user to change the access name of an existing
Cell Hierarchy Digest (CHD) to a user-supplied name.  The new name
must not already be in use by another CHD.

The first argument is the access name of an existing CHD, the second
argument is the new access name, with which the CHD will subsequently
be accessed.  This name can be any text string, but can not be null.

The function returns 1 on success, 0 otherwise, with an error message
likely available from {\vt GetError}.

%------------------------------------
% 100108
\index{ChdIsValid function}
\item{(int) \vt ChdIdValid({\it chd\_name\/})}\\
This function returns one if the string argument is an access name of
a Cell Hierarchy Digest currently in memory, zero otherwise.

%------------------------------------
% 100108
\index{ChdDestroy function}
\item{(int) \vt ChdDestroy({\it chd\_name\/})}\\
If the string argument is an access name of a Cell Hierarchy Digest
(CHD) currently in memory, the CHD will be destroyed and its memory
freed.  One is returned on success, zero otherwise, with an error
message likely available with {\vt GetError}.

%------------------------------------
% 021912
\label{ChdInfo}
\index{ChdInfo function}
\item{(string) \vt ChdInfo({\it chd\_name\/}, {\vt handle\_or\_filename\/},
 {\it flags\/})}\\
This function provides information about the archive file represented
by the Cell Hierarchy Digest (CHD) whose access name is given as the
first argument.  If the second argument is a string giving the name of
a file, output will go to that file.  If the second argument is a
handle returned from the {\vt Open} function or similar (opened for
writing), output goes to the handle stream.  In either case, the
return value is a null string.  If the second argument is a scalar 0,
the output will be in the form of a string which is returned.

The third argument is an integer or string which determines the type
of information to return.  If an integer, the bits are flags that
control the possible data fields and printing modes.  The string form
is a space or comma-separated list of text tokens (from the list
below, case insensitive) or hex integers.  The hex numbers or
equivalent values for the text tokens are or'ed together to form the
flags integer.
  
If this argument is 0, all flags except for {\vt allcells}, {\vt
instances}, {\vt flags}, {\vt instcnts}, and {\vt instcntsp} are
implied.  Thus, the sometimes very lengthly cells/instances listing
is skipped by default.  To obtain all available information, pass
{\vt -1} or {\vt all} as the flags value.

\begin{tabular}{|l|l|l|} \hline
\bf Keyword & \bf Value & \bf Description\\ \hline
\vt filename & \vt 0x1 & File name.\\ \hline
\vt filetype & \vt 0x2 & File type (``{\vt CIF}'', ``{\vt CGX}'',
  ``{\vt GDSII}'', or ``{\vt OASIS}'').\\ \hline
\vt unit & \vt 0x4 & File unit in meters (e.g., GDSII M-UNIT).\\ \hline
\vt alias & \vt 0x8 & Applied cell name aliasing modes.\\ \hline

\vt reccounts & \vt 0x10 & Table of record type counts (file format
  dependent).\\ \hline
\vt objcounts & \vt 0x20 & Table of object counts.\\ \hline
\vt depthcnts & \vt 0x40 & Tabulate the number of cell instances at each
  hierarchy level.\\ \hline
\vt estsize & \vt 0x80 & Print estimated memory needed to read file
  into {\Xic}.\\ \hline

\vt estchdsize & \vt 0x100 & Print size of data structure used to
  provide info.\\ \hline
\vt layers & \vt 0x200 & List of layer names found, as for {\vt ChdLayers}
  function.\\ \hline
\vt unresolved & \vt 0x400 & List any cells that are referenced
  but not defined in the file.\\ \hline
\vt topcells & \vt 0x800 & Top-level cells.\\ \hline

\vt allcells & \vt 0x1000 & All cells.\\ \hline
\vt offsort & \vt 0x2000 & Sort cells by offset in archive file.\\ \hline
\vt offset & \vt 0x4000 & Print offsets of cell definitions in archive
  file.\\ \hline
\vt instances & \vt 0x8000 & List instances with cells.\\ \hline

\vt bbs & \vt 0x10000 & List bounding boxes with cells, and attributes
  with instances.\\ \hline
\vt flags & \vt 0x20000 & Unused.\\ \hline
\vt instcnts & \vt 0x40000 & Count cell instances and report totals.\\ \hline
\vt instcntsp & \vt 0x80000 & Count cell instances and report totals per
  master.\\ \hline

\vt all & \vt -1 & Set all flags.\\ \hline
\end{tabular}

The information provided by these flags is more fully described below.

\begin{description}
\item{\vt filename}\\
Print the name of the archive file for which the information
applies.

\item{\vt filetype}\\
Print a string giving the format of the archive file:  one of ``{\vt
CIF}'', ``{\vt CGX}'', ``{\vt GDSII}'', or ``{\vt OASIS}''.

\item{\vt unit}\\
This is a file parameter giving the value of one unit in meters.  In
GDSII files, this is obtained from the M-UNIT record.  The value is
typically 1e-9, which means that a coordinate value of 1000
corresponds to one micron.

\item{\vt alias}\\
Print a string giving the cell name aliasing modes that were in
effect when the CHD was created.

\item{\vt reccounts}\\
Print a table of the counts for record types found in the archive.
This is format-dependent.

\item{\vt objcounts}\\
Print a table of object counts found in the archive file.  The table
contains the following keywords, each followed by a number.

\begin{tabular}{|l|l|} \hline
\bf Keyword & \bf Description\\ \hline
\vt Records & Total record count\\ \hline
\vt Cells & Number of cell definitions\\ \hline
\vt Boxes & Number of rectangles\\ \hline
\vt Polygons & Number of polygons\\ \hline
\vt Wires & Number of wire paths\\ \hline
\vt Avg Verts & Average vertex count per poly or wire\\ \hline
\vt Labels & Number of (non-physical) labels\\ \hline
\vt Srefs & Number of non-arrayed instances\\ \hline
\vt Arefs & Number of arrayed instances\\ \hline
\end{tabular}

If the per-layer counts option was set when the CHD was created,
additional lines will display the object counts as above, broken out
per-layer.

\item{\vt depthcnts}\\
A table of the number of cell instantiations at each hierarchy level
is printed, for each top-level cell found in the file.  The count for
depth 0 is 1 (the top-level cell), the count at depth 1 is the number
of subcells of the top-level cell, depth 2 is the number of subcells
of these subcells, etc.  Arrays are expanded, with each element
counting as an instance placement.  A total is printed, the same
value that would be obtained from the {\vt instcnts} flag.

\item{\vt estsize}\\
This flag will enable printing of the estimated memory required to
read the entire file into {\Xic}.  The system must be able to provide
at least this much memory for a read to succeed.

\item{\vt estchdsize}\\
Print an estimate of the memory required by the present CHD.

By default, a compression mechanism is used to reduce the data storage
needed for instance lists.  The {\et NoCompressContext} variable, if
set, will turn off use of compression.  If compression is used, the
{\vt extcxsize} field will include compression statistics.  The
``ratio'' is the space actually used to the space used if not
compressed.

\item{\vt layers}\\
Print a list of the layer names encountered in the archive, as for the
{\vt ChdLayers} function.

\item{\vt unresolved}\\
This will list cells that are referenced but not defined in the file. 
These will also be listed if {\vt allcells} is given.  A valid archive
file will not contain unresolved references.

\item{\vt topcells}\\
List the top-level cells, i.e., the cells in the file that are not
used as a subcell by another cell in the file.  If {\vt allcells} is
also given, only the names are listed, otherwise the cells are listed
including the {\vt offset}, {\vt instances}, {\vt bbs}, and {\vt
flags} fields if these flags are set.  The list will be sorted as per
{\vt offsort}.

\item{\vt allcells}\\
All cells found in the file are listed by name, including the {\vt
offset}, {\vt instances}, {\vt bbs}, and {\vt flags} fields if these
flags are also given.  The list will be sorted as per {\vt offsort}.
\end{description}

The following flags apply only if at least one of {\vt topcells} or
{\vt allcells} is given.

\begin{description}
\item{\vt offsort}\\
If this flag is set, the cells will be listed in ascending order of
the file offset, i.e., in the order in which the cell definitions
appear in the archive file.  If not set, cells are listed
alphabetically.

\item{\vt offset}\\
When set, the cell name is followed by the offset of the cell
definition record in the archive file.  This is given as a decimal
number enclosed in square brackets.

\item{\vt instances}\\
For each cell, the subcells used in the cell are listed.  The subcell
names are indented and listed below the cell name.

\item{\vt bbs}\\
For each cell the bounding box is shown, in L,B R,T form.  For
subcells, the position, transformation, and array parameters are
shown.  Coordinates are given in microns.  The subcell
transformation and array parameters are represented by a
concatenation of the following tokens, which follow the subcell
reference position.  These are similar to the transformation
tokens found in CIF, and have the same meanings.

\begin{tabular}{ll}
\vt MY & Mirror about the x-axis.\\
{\vt R}{\it i\/},{\it j\/} & Rotate by an angle given by the vector
  {\it i\/},{\it j\/}.\\
{\vt M}{\it mag} & Magnify by {\it mag\/}.\\
{\vt A}{\it nx\/},{\it ny\/},{\it dx\/},{\it dy\/} & Specifies
  an array, {\it nx} x {\it ny} with spacings {\it dx},
  {\it dy\/}.\\
\end{tabular}

Note:  for technical reasons, the cell bounding boxes in CHDs do
{\it not} include empty cells, unlike the bounding boxes computed
in the main database, which will include the placement location
points.

\item{\vt flags}\\
This is currently unused and ignored.

\item{\vt instcnts}\\
Print the total number of cell instantiations found in the hierarchy. 
Arrays are expanded, i.e., each element of an array counts as an
instance placement.

\item{\vt instcntsp}\\
Similar to {\vt instcnts}, but print the total instantiations for
each master cell.

\item{\vt all}\\
This enables all flags.
\end{description}

On error, a null string is returned, with an error message likely
available from {\vt GetError}.

This function is similar to the {\cb !fileinfo} command and to the
{\vt FileInfo} script function.

%------------------------------------
% 100108
\index{ChdFileName function}
\item{(string) \vt ChdFileName({\it chd\_name\/})}\\
This function returns a string containing the full pathname of the
file associated with the Cell Hierarchy Digest (CHD) whose access name
was given in the argument.  A null string is returned on error, with
an error message likely available from {\vt GetError}.

%------------------------------------
% 100108
\index{ChdFileType function}
\item{(string) \vt ChdFileType({\it chd\_name\/})}\\
This function returns a string containing the file format of the
archive file associated with the Cell Hierarchy Digest (CHD) whose
access name was given in the argument.  A null string is returned on
error, with an error message likely available from {\vt GetError}. 
Other possible returns are ``{\vt CIF}'', ``{\vt GDSII}'', ``{\vt
CGX}'', and ``{\vt OASIS}''.

%------------------------------------
% 100108
\index{ChdTopCells function}
\item{(stringlist\_handle) \vt ChdTopCells({\it chd\_name\/})}\\
This function returns a handle to a list of strings that contain the
top-level cell names in the Cell Hierarchy Digest (CHD) whose access
name was given in the argument (physical cells only).  The top-level
cells are those not used as a subcell by another cell in the CHD.  A
scalar zero is returned on error, with an error message likely
available from {\vt GetError}.

%------------------------------------
% 022209
\index{ChdListCells function}
\item{(stringlist\_handle) \vt ChdListCells({\it chd\_name\/},
 {\it cellname\/}, {\it mode\/}, {\it all\/})}\\
This function returns a handle to a list of cellnames from among those
found in the CHD, whose access name is given as the first argument. 
There are two basic modes, depending on whether the boolean argument
{\it all} is true or not.

If {\it all} is true, the {\it cellname} argument is ignored, and the
list will consist of all cells found in the CHD.  If the integer {\it
mode} argument is 0, all physical cell names are listed.  If {\it
mode} is 1, all electrical cell names will be returned.  If any other
value, the listing will contain all physical and electrical cell
names, with no duplicates.

If {\it all} is false, the listing will contain the names of all cells
under the hierarchy of the cell named in the {\it cellname} argument
(including {\it cellname}).  If {\it cellname} is 0, empty, or null,
the default cell for the CHD is assumed, i.e., the cell which has been
configured, or the first top-level cell found.  The {\it mode}
argument is 0 for physical cells, nonzero for electrical cells (there
is no merging of lists in this case).

On error, a scalar 0 is returned, and a message may be available
from {\vt GetError}.

%------------------------------------
% 100108
\index{ChdLayers function}
\item{(stringlist\_handle) \vt ChdLayers({\it chd\_name\/})}\\
This function returns a handle to a list of strings that contain the
names of layers used in the file represented by the Cell Hierarchy
Digest whose access name is passed as the argument (physical cells
only).  For file formats that use a layer/datatype, the names are
four-byte hex integers, where the left two bytes are the zero-padded
hex value of the layer number, and the right two bytes are the
zero-padded value of the datatype number.  This applies for
GDSII/OASIS files that follow the standard convention that layer and
datatype numbers are 0--255.  If either number is larger than 255, the
layer ``name'' will consist of eight hex bytes, the left four for
layer number, the right four for datatype.

The layers listing is available only if the CHD was created with info
available, i.e., {\vt OpenCellHierDigest} was called with the {\it
info\_saved} argument set to a value other than 0.

Each unique combination or layer name is listed.  A scalar zero is
returned on error, in which case an error message may be available
from {\vt GetError}.

%------------------------------------
% 041010
\index{ChdInfoMode function}
\item{(int) \vt ChdInfoMode({\it chd\_name\/})}\\
This function returns the saved info mode of the Cell Hierarchy Digest
whose access name is passed as the argument.  This is the {\it
info\_saved} value passed to {\vt OpenCellHierDigest}.  The values
are:

\begin{tabular}{ll}
0 & no information is saved.\\
1 & only total object counts are saved.\\
2 & object totals are saved per layer.\\
3 & object totals are saved per cell.\\
4 & objects counts are saved per cell and per layer.\\
\end{tabular}

If the CHD name is not resolved, the return value is -1, with an error
message available from {\vt GetError}.

%------------------------------------
% 041010
\index{ChdInfoLayers function}
\item{(stringlist\_handle) \vt ChdInfoLayers({\it chd\_name\/},
 {\it cellname\/})}\\
This is identical to the {\vt ChdLayers} function when the {\it
cellname} is 0, null, or empty.  If the CHD was created with {\vt
OpenCellHierDigest} with the {\it info\_saved} argument set to 4
(per-cell and per-layer info saved), then a {\it cellname} string can
be passed.  In this case, the return is a handle to a list of layers
used in the named cell.  A scalar 0 is returned on error, with an
error message probably available from {\vt GetError}.

%------------------------------------
% 041010
\index{ChdInfoCells function}
\item{(stringlist\_handle) \vt ChdInfoCells({\it chd\_name\/})}\\
If the CHD whose access name is given as the argument was created with
{\vt OpenCellHierDigest} with the {\it info\_saved} argument set to 3
(per-cell data saved) or 4 (per-cell and per-layer data saved), then
this function will return a handle to a list of cell names from the
source file.  On error, a scalar 0 is returned, with an error message
probably available from {\vt GetError}.

%------------------------------------
% 041010
\index{ChdInfoCounts function}
\item{(int) \vt ChdInfoCounts({\it chd\_name\/}, {\it cellname\/},
  {\it layername\/}, {\it array\/})}\\
This function will return object count statistics in the {\it
array\/}, which must have size 4 or larger.  The counts are obtained
when the CHD, whose access name is given as the first argument, was
created.  The types of counts available depend on the {\it
info\_saved} value passed to {\vt OpenCellHierDigest} when the CHD was
created.

The array is filled in as follows:

\begin{tabular}{ll}
{\it array\/}[0] & Box count.\\
{\it array\/}[1] & Polygon count.\\
{\it array\/}[2] & Wire count.\\
{\it array\/}[3] & Vertex count (polygons plus wires).\\
\end{tabular}

The following counts are available for the various {\it info\_saved}
modes.

\begin{description}
\item{\it info\_saved} = 0\\
No information is available.

\item{\it info\_saved} = 1\\
Both {\it cellname} and {\it layername} arguments are ignored, the
return provides file totals.

\item{\it info\_saved} = 2\\
The {\it cellname} argument is ignored.  If {\it layername} is 0,
null, or empty, the return provides file totals.  Otherwise, the
return provides totals for {\it layername}, if found.

\item{\it info\_saved} = 3\\
The {\it layername} argument is ignored.  If {\it cellname} is 0,
null, or empty, the return represents file totals.  Otherwise, the
return provides totals for {\it cellname}, if found.

\item{\it info\_saved} = 4\\
If both arguments are 0, null, or empty, the return represents file
totals.  If {\it cellname} is 0, null, or empty, the return represents
totals for the layer given.  If {\it layername} is 0, null, or empty,
the return provides totals for the cell name given.  If both names are
given, the return provides totals for the given layer in the given
cell.
\end{description}

If a cell or layer is not found, or data are not available for some
reason, or an error occurs, the return value is 0, and an error
message may be available from {\vt GetError}.  Otherwise, the return
value is 1, and the array is filled in.

%------------------------------------
% 100108
\index{ChdCellBB function}
\index{NoReadLabels variable}
\item{(int) \vt ChdCellBB({\it chd\_name\/}, {\it cellname\/}, {\it array\/})}\\
This returns the bounding box of the named cell.  The {\it cellname}
is a string giving the name of a physical cell found in the Cell
Hierarchy Digest (CHD) whose access name is given in the first
argument.

The {\it cellname}, if nonzero, must be the cell name after any
aliasing that was in force when the CHD was created.  If {\it
cellname} is passed 0, the default cell for the CHD is understood. 
This is a cell name configured into the CHD, or the first top-level
cell found in the archive file.

The values are returned in the {\it array}, which must have size 4 or
larger.  the order is l,b,r,t.  One is returned on success, zero
otherwise, with an error message likely available from {\vt GetError}.

The cell bounding boxes for geometry are computed as the file is read,
so that if the {\et NoReadLabels} variable is set during the read,
i.e., when {\vt OpenCellHierDigest} is called, text labels will not
contribute to the bounding box computation.

%------------------------------------
% 062209
\index{ChdSetDefCellName function}
\item{(int) \vt ChdSetDefCellName({\it chd\_name\/}, {\it cellname\/})}\\
This will set or unset the configuration of a default cell namein the
Cell Hierarchy Digest whose access name is given in the first
argument.

If the {\it cellname} argument in not 0 or null, it must be a cell
name after any aliasing that was in force when the CHD was created,
that exists in the CHD.  This will set the default cell name for the
CHD which will be used subsequently by the CHD whenever a cell name is
not otherwise specified.  The current default cell name is returned
from the {\vt ChdDefCellName} function.  If {\it cellname} is 0 or
null, the default cell name is unconfigured.  In this case, the CHD
will use the first top-level cell found (lowest offset on the archive
file).  A top-level cell is one that is not used as a subcell by any
other cell in the CHD.

One is returned on success, zero otherwise, with an error message
likely available with {\vt GetError}.

%------------------------------------
% 062209
\index{ChdDefCellName function}
\item{(string) \vt ChdDefCellName({\it chd\_name\/})}\\
This will return the default cell name of the Cell Hierarchy Digest
whose access name is given in the argument.  This will be the cell
name configured (with {\vt ChdSetDefCellName}), or if no cell name is
configured the return will be the name of the first top-level cell
found (lowest offset on the archive file).  A top-level cell is one
that is not used as a subcell by any other cell in the CHD.

On error, a null string is returned, with an error message likely
available from {\vt GetError}.

%------------------------------------
% 012111
\index{ChdLoadGeometry function}
\item{(int) \vt ChdLoadGeometry({\it chd\_name\/})}\\
This function will read the geometry from the original layout file
from the Cell Hierarchy Digest (CHD) whose access name is given in the
argument into a new Cell Geometry Digest (CGD) in memory, and
configures the CHD to link to the new CGD for use when reading.  The
new CGD is given an internally-generated access name, and will store
all geometry data in memory.  The new CGD will be destroyed when
unlinked. 

This is a convenience function, one can explicitly create a CGD   
(with {\vt OpenCellGeomDigest}) and link it to the CHD (with   
{\vt ChdLinkCgd}) if extended features are needed.   

See the table in \ref{features} for the features that apply during a
call to this function.

The return value is 1 on success, 0 otherwise, with an error message
likely available from {\vt GetError}.

%------------------------------------
% 012111
\index{ChdLinkCgd function}
\item{(int) \vt ChdLinkCgd({\it chd\_name\/}, {\it cgd\_name\/})}\\
This function links or unlinks a Cell Geometry Digest (CGD) whose
access name is given as the second argument, to the Cell Hierarchy
Digest (CHD) whose access name is given as the first argument.  With a
CGD linked, when the CHD is used to access geometry data, the data
will be obtained from the CGD, if it exists in the CGD, and from the
original layout file if not provided by the CGD.  The CGD is a
``geometry cache'' which resides in memory.

If the {\it cgd\_name} is null or empty (0 can be passed for this
argument) any CGD linked to the CHD will be unlinked.  If the CGD was
created specifically to link with the CHD, such as with {\vt
ChdLoadGeometry}, it will be freed from memory, otherwise it will be
retained.

This function returns 1 on success, 0 otherwise with an error message
likely available from {\vt GetError}.

%------------------------------------
% 100108
\index{ChdGetGeomName function}
\item{(string) \vt ChdGetGeomName({\it chd\_name\/})}\\
The string argument is an access name for a Cell Hierarchy Digest
(CHD) in memory.  If the CHD exists and has an associated Cell
Geometry Digest (CGD) linked (e.g., {\vt ChdLoadGeometry} was called),
this function returns the access name of the CGD.  If the CHD is not
found or not configured with a CGD, a null string is returned.

%------------------------------------
% 012111
\index{ChdClearGeometry function}
\item{(int) \vt ChdClearGeometry({\it chd\_name\/})}\\
This function will clear the link to the Cell Geometry Digest within
the Cell Hierarchy Digest.  If a CGD was linked, and it was created
explicitly for linking into the CHD as in {\vt ChdLoadGeometry}, the CGD
will be freed, otherwise it will be retained.  The return value is 1
if the CHD was found, 0 otherwise, with a message available from {\vt
GetError}.

This function is identical to {\vt ChdLinkCgd} with a null second
argument.

%------------------------------------
% 040409
\index{ChdSetSkipFlag function}
\item{(int) \vt ChdSetSkipFlag({\it chd\_name\/}, {\it cellname\/},
   {\it skip\/})}\\
This will set/unset the skip flag in the Cell Hierarchy Digest (CHD)
whose access name is given in the first argument for the cell named in
{\it cellname} (physical only).

The {\it cellname}, if nonzero, must be the cell name after any
aliasing that was in force when the CHD was created.  If {\it
cellname} is passed 0, the default cell for the CHD is understood. 
This is a cell name configured into the CHD, or the first top-level
cell found in the archive file.

With the skip flag set, the cell is ignored in the CHD, i.e., the cell
and its instances will not be included in output or when reading into
memory when the CHD is used to access layout data.  The last argument
is a boolean value:  0 to unset the skip flag, nonzero to set it.  The
return value is 1 if a flag was altered, 0 otherwise, with an error
message likely available from {\vt GetError}.

%------------------------------------
% 040409
\index{ChdClearSkipFlags function}
\item{(int) \vt ChdClearSkipFlags({\it chd\_name\/})}\\
This will clear the skip flags for all cells in the Cell Hierarchy
Digest whose access name is given in the argument.  The skip flags are
set with {\vt SetSkipFlag}.  The return value is 1 on success, 0
otherwise, with an error message likely available with {\vt GetError}.

%------------------------------------
% 022209
\index{ChdCompare function}
\item{(int) \vt ChdCompare({\it chd\_name1\/}, {\it cname1\/},
 {\vt chd\_name2\/}, {\it cname2\/}, {\it layer\_list\/}, {\it skip\_layers\/},
 {\it maxdiffs\/},\newline
 {\it obj\_types\/}, {\it geometric\/}, {\it array\/})}\\
This will compare the contents of two cells, somewhat similar to the
{\cb !compare} command and the {\cb Compare Layouts} operation in the
{\cb Convert Menu}.  However, only one cell pair is compared, taking
account only of features within the cells.  The {\vt ChdCompareFlat}
function is similar, but flattens geometry before comparison.

When comparing subcells, arrays will be expanded into individual
instances before comparison, avoiding false differences between
arrayed and unarrayed equivalents.  The returned handles (if any)
contain differences, as lists of object copies.  Properties are
ignored.

The arguments are:
\begin{description}
\item{\it chd\_name1}\\
Access name of a Cell Hierarchy Digest (CHD) in memory.

\item{\it cname1}\\
Name of cell in {\it chd\_name1} to compare, if null (0 passed) the
default cell in {\it chd\_name1} is used.

\item{\it chd\_name2}\\
If not null or empty (one can pass 0 for this argument), the name of
another CHD.

\item{\it cname2}\\
Name of cell in the second CHD, or in memory, to compare.  If null, or
0 is passed, and a second CHD was specified, the second CHD's default
cell is understood.  Otherwise, the name will be assumed the same as
{\it cname1\/}.

\item{\it layer\_list}\\
String of space-separated layer names, or zero which implies all
layers.

\item{\it skip\_layers}\\
If this boolean value is nonzero and a {\it layer\_list} was given,
the layers in the list will be skipped.  Otherwise, only the layers in
the list will be compared (all layers if {\it layer\_list} is passed
zero).

\item{\it maxdiffs}\\
The function will return after recording this many differences.  If 0
or negative, there is no limit.

\item{\it obj\_types}\\
String consisting of the layers {\vt c,b,p,w,l}, which determines
objects to consider (subcells, boxes, polygons, wires, and labels), or
zero.  If zero, ``{\vt cbpw}'' is the default, i.e., labels are
ignored.  If the geometric argument is nonzero, all but '{\vt c}' will
be ignored, and boxes, polygons, and wires will be compared.

\item{\it geometric}\\
If this boolean value is nonzero, a geometric comparison will be
performed, otherwise objects are compared directly.

\item{\it array}\\
This is a two-element or larger array, or zero.  If an array is
passed, upon return the elements are handles to lists of box, polygon,
and wire object copies (labels and subcells are not returned):  {\it
array\/}[0] contains a list of objects in handle1 and not in handle2,
and {\it array\/}[1] contains objects in handle2 and not in handle1. 
The {\vt H} function must be used on the array elements to access the
handles.  If the argument is passed zero, no object lists are
returned.
\end{description}

The cells for the current mode (electrical or physical) are compared. 
The scalar return can take the following values:

\begin{tabular}{ll}
-1 & An error occurred, with a message possibly available
      from the {\vt GetError} function.\\
0 & Successful comparison, no differences found.\\
1 & Successful comparison, differences found.\\
2 & The cell was not found in {\it chd\_name1\/}.\\
3 & The cell was not found in {\it chd\_name2\/}.\\
4 & The cell was not found in either source.\\
\end{tabular}

%------------------------------------
% 022209
\index{ChdCompareFlat function}
\item{(int) \vt ChdCompareFlat({\it chd\_name1\/}, {\it cname1\/},
 {\vt chd\_name2\/}, {\it cname2\/}, {\it layer\_list\/}, {\it skip\_layers\/},
 \newline
 {\it maxdiffs\/}, {\it area\/}, {\it coarse\_mult\/}, {\it find\_grid\/},
 {\it array\/})}\\
This will compare the contents of two hierarchies, using a flat
geometry model similar to the flat options of the {\cb !compare}
command and the {\cb Compare Layouts} operation in the {\cb Convert
Menu}.  The {\vt ChdCompare} function is similar, but does not
flatten.

The returned handles (if any) contain the differences, as lists of
objects.  Properties are ignored.

The arguments are:
\begin{description}
\item{\it chd\_name1}\\
Access name of a Cell Hierarchy Digest (CHD) in memory.

\item{\it cname1}\\
Name of cell in {\it chd\_name1} to compare, if null (0 passed) the
default cell in {\it chd\_name1} is used.

\item{\it chd\_name2}\\
Access name of another CHD in memory.  This argument can not be null
as in {\vt ChdCompare}, flat comparison to memory cells is
unavailable.

\item{\it cname2}\\
Name of cell in the second CHD to compare.  If null, or 0 is passed,
the second CHD's default cell is understood.

\item{\it layer\_list}\\
String of space-separated layer names, or zero which implies all
layers.

\item{\it skip\_layers}\\
If this boolean value is nonzero and a {\it layer\_list} was given,
the layers in the list will be skipped.  Otherwise, only the layers in
the list will be compared (all layers if {\it layer\_list} is passed
zero).

\item{\it maxdiffs}\\
The function will return after recording this many differences.  If 0
or negative, there is no limit.

\item{\it area}\\
This argument can be an array of size 4 or larger, or 0.  If an array,
it contains a rectangle description in order L,B,R,T in microns, which
specifies the area to compare.  If 0 is passed, the area compared will
contain the two hierarchies entirely.

\item{\it coarse\_mult}\\
The comparison is performed in the manner described for the {\vt
ChdIterateOverRegion} function, using a fine grid and a coarse grid. 
This argument specifies the size of the coarse grid in multiples of
the fine grid size.  All of the geometry needed for a coarse grid cell
is brought into memory at once, so this size should be consistent with
memory availability and layout feature density.  Values of 1--100 are
accepted for this argument, with 20 a reasonable initial choice.

\item{\it fine\_grid}\\
Comparison is made within a fine grid cell.  The optimum fine grid
size depends on factors including layout feature density and memory
availability.  Larger sizes usually run faster, but may require
excessive memory.  The value is given in microns, with the acceptable
range being 1.0 -- 100.0 microns.  A reasonable initial choice is
20.0, but experimentation can often yield better performance.

\item{\it array}\\
This is a two-element or larger array, or zero.  If an array is
passed, upon return the elements are handles to lists of box, polygon,
and wire object copies (labels and subcells are not returned):  {\it
array\/}[0] contains a list of objects in handle1 and not in handle2,
and {\it array\/}[1] contains objects in handle2 and not in handle1. 
The {\vt H} function must be used on the array elements to access the
handles.  If the argument is passed zero, no object lists are
returned.
\end{description}

The cells for the physical mode are compared, it is not possible to
compare electrical cells in flat mode.  The return value is an
integer, -1 on error (with a message likely available from {\vt
GetError}), 0 if no differences were seen, or positive giving the
number of differences seen.

%------------------------------------
% 062209
\index{ChdEdit function}
\item{(int) \vt ChdEdit({\it chd\_name\/}, {\it scale\/}, {\it cellname\/})}\\
This will read the given cell and its descendents into memory and open
the cell for editing, similar to the {\vt Edit} function, however the
layout data will be accessed through the Cell Hierarchy Digest whose
access name is given in the first argument.  The return value takes
the same values as the {\vt Edit} function return.

See the table in \ref{features} for the features that apply during a
call to this function.

The {\it scale} will multiply all coordinates in cells opened, and
can be in the range 0.001 -- 1000.0.

The {\it cellname}, if nonzero, must be the cell name after any
aliasing that was in force when the CHD was created.  If {\it
cellname} is passed 0, the default cell for the CHD is understood. 
This is a cell name configured into the CHD, or the first top-level
cell found in the archive file.

%------------------------------------
% 062209
\index{ChdOpenFlat function}
\item{(int) \vt ChdOpenFlat({\it chd\_name\/}, {\it scale\/}, {\it cellname\/},
 {\it array\/}, {\it clip\/})}\\
This will read the cell named in the {\it cellname} string and its
subcells into memory, creating a flat cell with the same name.  The
Cell Hierarchy Digest (CHD) whose access name is given in the first
argument is used to obtain the layout data.

See the table in \ref{features} for the features that apply during a
call to this function.  Text labels are ignored.

The {\it cellname}, if nonzero, must be the cell name after any
aliasing that was in force when the CHD was created.  If {\it
cellname} is passed 0, the default cell for the CHD is understood. 
This is a cell name configured into the CHD, or the first top-level
cell found in the archive file.

If the cell already exists in memory, it will be overwritten.

The {\it scale} will multiply all coordinates read, and can be in the
range 0.001 -- 1000.0.

If the {\it array} argument is passed 0, no windowing will be used. 
Otherwise the array should have four components which specify a
rectangle, in microns, in the coordinates of {\it cellname}.  The
values are

\begin{tabular}{ll}
{\it array\/}{\vt [0]} & X left\\
{\it array\/}{\vt [1]} & Y bottom\\
{\it array\/}{\vt [2]} & X right\\
{\it array\/}{\vt [3]} & Y top\\
\end{tabular}

If an array is given, only the objects and subcells needed to render
the window will be read.

If the boolean value {\it clip} is nonzero and an array is given,
objects will be clipped to the window.  Otherwise no clipping is done.

Before calling {\vt ChdOpenFlat}, the memory use can be estimated by
calling the {\vt ChdEstFlatMemoryUse} function.  An overall
transformation can be set with {\vt ChdSetFlatReadTransform}, in which
case the area given applies in the ``root'' coordinates.

The return value is 1 on success, 0 on error, or -1 if an interrupt
was received.  In the case of an error return, an error message may be
available through {\vt GetError}.

%------------------------------------
% 100108
\index{ChdSetFlatReadTransform function}
\item{(real) \vt ChdSetFlatReadTransform({\it tfstring\/}, {\it x\/},
 {\it y\/})}\\
This rather arcane function will set up a transformation which will be
used during calls to the following functions:
\begin{quote} \vt
ChdOpenFlat\\
ChdWriteSplit\\
ChdGetZlist\\
ChdOpenOdb\\
ChdOpenZdb\\
ChdOpenZbdb
\end{quote}

The transform will be applied to all of the objects read through the
CHD with these functions.  Why might this function be used?  Consider
the following:  suppose we have a CHD describing a cell hierarchy, the
top-level cell of which is to be instantiated under another cell we'll
call ``root'', with a given transformation.  We would like to consider
the objects from the CHD from the perspective of the ``root'' cell. 
This function would be called to set the transformation, then one of
the flat read functions would be called and the returned objects
accumulated.  The returned objects will have coordinates relative to
the ``root'' cell, rather than relative to the top-level cell of the
CHD.

The {\it tfstring} describes the rotation and mirroring part of the
transformation.  It is either one of the special tokens to be
described, or a sequence of the following tokens:

\begin{description}
\item{\vt MX}\\
 Flip the X axis.
\item{\vt MY}\\
 Flip the Y axis.
\item{\vt R}{\it nnn}\\
 Rotate by {\it nnn} degrees.  The {\it nnn} must be one of 0, 45, 90,
 135, 180, 225, 270, 315.
\end{description}

White space can appear between tokens.  The operations are performed
in order.  Note that, e.g., ``{\vt MXR90}'' is very different from
``{\vt R90MX}''.

Alternatively, the {\it tfstring} can contain a single ``Lef/Def''
token as listed below.  The second column is the equivalent string
using the syntax previously described.

\begin{quote}
\begin{tabular}{ll}
\vt N & null or empty or {\vt R0}\\
\vt S & \vt R180\\
\vt W & \vt R90\\
\vt E & \vt R270\\
\vt FN & \vt MX\\
\vt FS & \vt MY\\
\vt FW & \vt MYR90\\
\vt FE & \vt MXR90\\
\end{tabular}
\end{quote}

The {\it x} and {\it y} are the translation part of the
transformation.  These are coordinates, given in microns.

If {\it tfstring} is null or empty, no rotations or mirroring will be
used.

The function returns 1 on success, 0 if the {\it tfstring} contains an
error.

%------------------------------------
% 100108
\index{ChdEstFlatMemoryUse function}
\item{(real) \vt ChdEstFlatMemoryUse({\it chd\_name\/}, {\vt cellname\/},
 {\it array\/}, {\it counts\_array\/})}\\
This function will return an estimate of the memory required to
perform a {\vt ChdOpenFlat} call.  The first argument is the access
name of an existing Cell Hierarchy Digest that was created with
per-cell object counts saved (e.g., a call to {\vt OpenCellHierDigest}
with the {\it info\_saved} argument set to 3 or 4).

The {\it cellname}, if nonzero, must be the cell name after any
aliasing that was in force when the CHD was created.  If {\it
cellname} is passed 0, the default cell for the CHD is understood. 
This is a cell name configured into the CHD, or the first top-level
cell found in the archive file.

The third argument is an array of size four or larger that contains
the rectangular area as passed to the {\vt ChdOpenFlat} call.  The
components are
\begin{quote}
\begin{tabular}{ll}
{\it array\/}[0] & X left\\
{\it array\/}[1] & Y bottom\\
{\it array\/}[2] & X right\\
{\it array\/}[3] & Y top\\
\end{tabular}
\end{quote}
This argument can also be zero to indicate that the full area of the
top level cell is to be considered.
 
The final argument is also an array of size four or larger, or zero. 
If an array is passed, and the function succeeds, the components are
filled with the following values:
\begin{quote}
\begin{tabular}{ll}
{\it counts\_array\/}[0] & estimated total box count\\
{\it counts\_array\/}[1] & estimated total polygon count\\
{\it counts\_array\/}[2] & estimated total wire count\\
{\it counts\_array\/}[3] & estimated total vertex count\\
\end{tabular}
\end{quote}
These are counts of objects that would be saved in the top-level cell
during the {\vt ChdOpenFlat} call.  These are estimates, based on area
normalization, and do not include any clipping or merging.  The vertex
count is an estimate of the total number of polygon and wire vertices.

The return value is an estimate, in megabytes, of the incremental
memory required to perform the {\vt ChdOpenFlat} call.  This does not
include normal overhead.

%------------------------------------
% 022916
\index{ChdWrite function}
\item{(int) \vt ChdWrite({\it chd\_name\/}, {\it scale\/}, {\it cellname\/},
 {\it array\/}, {\it clip\/}, {\it all\/}, {\it flatten\/},
 {\it ecf\_level\/}, {\it outfile\/})}\\
This will write the cell named in the {\it cellname} string to the
output file given in {\it outfile}, using the Cell Hierarchy Digest
whose access name is given in the first argument to obtain layout
data.

If the {\it outfile} is null or empty, the geometry will be
``written'' as cells in the main database, hierarchically if {\it all}
is true.  This allows windowing to be applied when converting a
hierarchy, which will attempt to convert only objects and cells needed
to render the window area.  This has the potential to hopelessly
scramble your in-memory design data so be careful.

See the table in \ref{features} for the features that apply during a
call to this function.

The {\it cellname}, if nonzero, must be the cell name after any
aliasing that was in force when the CHD was created.  If {\it
cellname} is passed 0, the default cell for the CHD is understood. 
This is a cell name configured into the CHD, or the first top-level
cell found in the archive file.

If the boolean argument {\it all} is nonzero, the hierarchy under the
cell is written, otherwise only the named cell is written.  If the
{\it outfile} is null or empty, native cell files will be created in
the current directory.  If the {\it outfile} is the name of an
existing directory, native cell files will be created in that
directory.  Otherwise, the extension of the {\it outfile} determines
the file type:

\begin{tabular}{ll}
CGX   & \vt .cgx\\
CIF   & \vt .cif\\
GDSII & \vt .gds, .str, .strm, .stream\\
OASIS & \vt .oas\\
\end{tabular}

Only these extensions are recognized, however CGX and GDSII allow
an additional {\vt .gz} which will imply compression.

The {\it scale} will multiply all coordinates read, and can be in
the range 0.001 -- 1000.0.

If the {\it array} argument is passed 0, no windowing will be used. 
Otherwise the array should have four components which specify a
rectangle, in microns, in the coordinates of {\it cellname}.  The
values are

\begin{tabular}{ll}
{\it array\/}{\vt [0]} & X left\\
{\it array\/}{\vt [1]} & Y bottom\\
{\it array\/}{\vt [2]} & X right\\
{\it array\/}{\vt [3]} & Y top\\
\end{tabular}

If an array is given, only the objects and subcells needed to render
the window will be written.

If the boolean value {\it clip} is nonzero and an array is given,
objects will be clipped to the window.  Otherwise no clipping is done.

If the boolean value {\it all} is nonzero, the hierarchy under {\it
cellname} is written, otherwise not.  If windowing is applied, this
applies only to {\it cellname}, and not subcells.

If the boolean variable {\it flatten} is nonzero, the objects in the
hierarchy under {\it cellname} will be written into {\it cellname},
i.e., flattened.  The {\it all} argument is ignored in this case. 
Otherwise, no flattening is done.

The {\it ecf\_level} is an integer 0--3 which sets the empty cell
filtering level, as described for the {\cb Format Conversion} panel in
\ref{ecfilt}.  The values are

\begin{tabular}{ll}
0 & No empty cell filtering.\\
1 & Apply pre- and post-filtering.\\
2 & Apply pre-filtering only.\\
3 & Apply post-filtering only.\\
\end{tabular}

The return value is 1 on success, 0 on error, or -1 if an interrupt
was received.  In the case of an error return, an error message may be
available through {\vt GetError}.

%------------------------------------
% 022916
\index{ChdWriteSplit function}
\item{(int) \vt ChdWriteSplit({\it chd\_name\/}, {\it cellname\/},
 {\it basename\/}, {\it array\/}, {\it regions\_or\_gridsize\/},\\
 {\it numregions\_or\_bloatval\/}, {\it maxdepth\/}, {\it scale\/},
 {\it flags\/})}\\
This function will read the geometry data through the a Cell Hierarchy
Digest (CHD) whose name is given as the first argument, into a
collection of files representing rectangular regions of the top-level
cell.  Each output file contains only the cells and geometry necessary
to represent the region.  The regions can be specified as a list of
rectangles, or as a grid.

See the table in \ref{features} for the features that apply during a
call to this function.

\begin{description}
\item{\it cellname}\\
The {\it cellname\/}, if nonzero, must be the cell name after any
aliasing that was in force when the CHD was created.  If {\it
cellname} is passed 0, the default cell for the CHD is understood. 
This is a cell name configured into the CHD, or the first top-level
cell found in the archive file.

\item{\it basename}\\
The {\it basename} is a cell path name in the form
\begin{quote}
    [/{\it path\/}/{\it to\/}/]{\it basename\/}.{\it ext\/},
\end{quote}
where the extension {\it ext} gives the type of file to create.
One of the following extensions must be provided:
\begin{quote}
\begin{tabular}{ll}
CGX output & \vt .cgx\\
CIF output & \vt .cif\\
GDSII output & \vt .gds, .str, .strm, .stream\\
OASIS output & \vt .oas, .oasis\\
\end{tabular}
\end{quote}

A ``{\vt .gz}'' second extension is allowed following CGX and GDSII
extensions in which case the files will be compressed using the {\vt
gzip} format.

When writing a list of regions, the output files will be named in the
form {\it basename\_N\/}.{\it ext}, where the .{\it ext} is the
extension supplied, and {\it N} is a 0--based index of the region,
ordered as given.  When writing a grid, the output files will be named
in the form {\it basename\_X\_Y\/}.{\it ext\/}, where the .{\it ext}
is the extension supplied, and {\it X\/},{\it Y} are integer indices
representing the grid cell (origin is the lower-left corner).  If a
directory path is prepended to the {\it basename\/}, the files will be
found in that directory (which must exist, it will not be created).

\item{\it array}\\
The {\it array} argument can be 0, or the name of an array of size
four or larger that contains a rectangle specification, in microns, in
order L,B,R,T.  If given, the rectangle should intersect the bounding
box of the top-level cell ({\it cellname\/}).  Only cells and geometry
within this area will be written to output.  If 0 is passed, the
entire bounding box of the top cell is understood.

When writing grid files, the origin of the grid, before bloating, is
at the lower-left corner of the area to be output.

\item{\it regions\_or\_gridsize}\\
This argument can be an array, or a scalar value.  If an array, the
array consists of one or more rectangular area specifications, in
order L,B,R,T in microns.  These are the regions that will be written
to output files.

If this argument is a number, it represents the size of a square grid
cell, in microns.

\item{\it bloatval}\\
If an array was passed as the previous argument, then this argument is
an integer giving the number of regions in the array to be written. 
The size of the array is at least four times the number of regions.

If instead a grid value was given in the previous argument, then
this argument provides a bloating value.  The grid cells will be
bloated by this value (in microns) if the value is nonzero.  A
positive value pushes out the grid cell edges by the value given, a
negative value does the reverse.

\item{\it maxdepth}\\
This integer value applies only when flattening, and sets the maximum
hierarchy depth for include in output.  If 0, only objects in the
top-level cell will be included,

\item{\it scale}\\
This is a scale factor which will be applied to all output.  The {\it
gridsize\/}, {\it bloatval\/}, and {\it array} coordinates are the
sizes found in output, and are independent of the scale factor.  The
valid range is 0.001 -- 1000.0.

\item{\it flags}\\
This argument is a {\bf string} consisting of specific letters,
the presence of which sets one of several available modes.  These are
\begin{quote}
\begin{tabular}{ll}
\vt p & parallel\\
\vt f & flatten\\
\vt c & clip\\
{\vt n}[{\it N\/}] & empty cell filtering\\
\vt m & map names\\
\end{tabular}
\end{quote}

The character recognition is case-insensitive.  A null or empty string
indicates no flags set.

\begin{description}
\item{\vt p}\\
If {\vt p} is given, a parallel writing algorithm is used.  Otherwise,
the output files are generated in sequence.  The files should be
identical from either writing mode.  The parallel mode may be a little
faster, but requires more internal memory.  When writing in parallel,
the user may encounter system limitations on the number of file
descriptors open simultaneously.

\item{\vt f}\\
If {\vt f} is given, the output will be flattened.  When flattening,
an overall transformation can be set with {\vt
ChdSetFlatReadTransform}, in which case the given area description
would apply in the ``root'' coordinates.

If not given, the output files will be hierarchical, but only the
subcells needed to render the grid cell area, each containing only
the geometry needed, will be written.

\item{\vt c}\\
If {\vt c} is given, objects will be clipped at the grid cell
boundaries.  This also applies to objects in subcells, when not
flattening.

\item{\vt n}[{\it N\/}]\\
The `{\vt n}' can optionally be followed by an integer 0--3.  If no
integer follows, `3' is understood.  This sets the empty cell
filtering level as described for the {\cb Format Conversion} panel in
\ref{ecfilt}.  The values are

\begin{tabular}{ll}
0 & No empty cell filtering (no operation).\\
1 & Apply pre- and post-filtering.\\
2 & Apply pre-filtering only.\\
3 & Apply post-filtering only.\\
\end{tabular}

\item{\vt m}\\
If {\vt m} is given, and {\vt f} is also given (flattening), the
top-level cell names in the output files will be modified so as to be
unique in the collection.  A suffix ``\_{\it N\/}'' is added to the
cell name, where {\it N} is a grid cell or region index.  The index is
0 for the lower-left grid cell, and is incremented in the sweep order
left to right, bottom to top.  If writing regions, the index is
0--based, in the order of the regions given.  Furthermore, a native
cell file is written, named ``{\it basename\/}\_root'', which calls
each of the output files.  Loading this file will load the entire
output collection, memory limits permitting.
\end{description}
\end{description}

The function returns 1 on success, 0 otherwise, with an error message
likely available from {\vt GetError}.


%------------------------------------
% 100408
\index{ChdCreateReferenceCell function}
\item{(int) \vt ChdCreateReferenceCell({\it chd\_name\/}, {\it cellname\/})}\\
This function will create a reference cell (see \ref{refcell}) in
memory.  A reference cell is a special cell that references a cell
hierarchy in an archive file, but does not have its own content. 
Reference cells can be instantiated during editing like any other
cell, but their content is not visible.  When a reference cell is
written to disk as part of a cell hierarchy, the hierarchy of the
reference cell is extracted from its source and streamed into the
output.

The first argument is a string giving the name of a Cell Hierarchy
Digest (CHD) already in memory.  The second argument is the name of a
cell in the CHD, which must include aliasing if aliasing was applied
when the CHD was created.  This will also be the name of the reference
cell.  A cell with this name should not already exist in current
symbol table.

Although the CHD is required for reference cell creation, it is not
required when the reference cell is written, but will be used if
present.  The archive file associated with the CHD should not be moved
or altered before the reference cell is written to disk.

A value 0 is returned on error, with a message probably available from
{\vt GetError}.  The value 1 is returned on success.

%------------------------------------
% 100408
\index{ChdLoadCell function}
\item{(int) \vt ChdLoadCell({\it chd\_name\/}, {\it cellname\/})}\\
This function will load a cell into the main editing database, and
subcells of the cell will be loaded as reference cells (see
\ref{refcell}).  This allows the cell to be edited, without loading
the hierarchy into memory.  When written to disk as part of a
hierarchy, the cell hierarchies of the reference cells will be
extracted from the input source and streamed to output.

The first argument is a string giving the name of a Cell Hierarchy
Digest (CHD) already in memory.  The second argument is the name of a
cell in the CHD, which must include aliasing if aliasing was applied
when the CHD was created.  This cell will be read into memory.  Any
subcells used by the cell will be created in memory as reference
cells, which a special cells which have no content but point to a
source for their content.

Although the CHD is required for reference cell creation, it is not
required when the reference cell is written, but will be used if
present.  The archive file associated with the CHD should not be
moved or altered before the reference cell is written to disk.

A value 0 is returned on error, with a message probably available from
{\vt GetError}.  The value 1 is returned on success.

%------------------------------------
% 030113
\index{ChdIterateOverRegion function}
\item{\parbox[t]{6in}{(int) \vt ChdIterateOverRegion({\it chd\_name\/},
 {\it cellname\/}, {\it funcname\/}, {\it array\/}, {\it coarse\_mult\/},
 {\it fine\_grid\/},\\
 \hspace*{1.5cm}{\it bloat\_val})}}\\
This function is an interface to a system which creates a logical
rectangular grid over a cell hierarchy, then iterates over the
partitions in the grid, performing some action on the logically flattened
geometry.

A Cell Hierarchy Digest (CHD) is used to obtain the flattened
geometry, with or without the assistance of a Cell Geometry Digest
(CGD).  There are actually two levels of gridding:  the coarse grid,
and the fine grid.  The area of interest is first logically
partitioned into the coarse grid.  For each cell of the coarse grid, a
``ZBDB'' special database (see \ref{specdb}) is created, using the
fine grid.  For example, one might choose 400x400 microns for the
coarse grid, and 20x20 microns for the fine grid.  Thus, geometry
access is in 400x400 ``chunks''.  The geometry is extracted,
flattened, and split into separate trapezoid lists for each fine grid
area, for each layer.

As each fine grid cell is visited, a user-supplied script function is
called.  The operations performed are completely up to the user, and
the framework is intended to be as flexible as possible.  As an
example, one might extract geometric parameters such as density,
minimum line width and spacing, for use by a process analysis tool. 
Scalar parameters can be conveniently saved in spatial parameter
tables (SPTs, see \ref{spt}).

The first argument is the access name of a CHD in memory.  The second
argument is the top-level cell from the CHD, or if passed 0, the CHD's
default cell will be used.

The third argument is the name of a user-supplied script function
which will implement the user's calculations.  The function should
already be in memory before {\vt ChdIterateOverRegion} is called. 
This function is described in more detail below.

The {\it array} argument can be 0, in which case the area of interest
is the entire top-level cell.  Otherwise, the argument should be an
array of size four or larger containing the rectangular area of
interest, in order L,B,R,T in microns.  The coarse and find grid
origin is at the lower left corner of the area of interest.

The {\it fine\_grid} argument is the size of the fine grid (which is
square) in microns.  The {\it coarse\_mult} is an integer representing
the size of the coarse grid, in {\it fine\_grid} quanta.

The {\it bloat\_val} argument specifies an amount, in microns, that
the grid cells (both coarse and fine) should be expanded when
returning geometry.  Geometry is clipped to the bloated grid.  Thus,
it is possible to have some overlap in the geometry returned from
adjacent grid cells.  This value can be negative, in which case grid
cells will effectively shrink.

The callback function has the following prototype.
\begin{quote}
(int) {\vt callback}({\it db\_name\/}, {\it j\/}, {\it i\/},
{\it spt\_x\/}, {\it spt\_y\/}, {\it data}, {\it cell\_name}, {\it chd\_name})
\end{quote}
The function definition must start with the {\it db\_name} and include
the arguments in the order shown, but unused arguments to the right of
the last needed argument can be omitted.

\begin{description}
\item{\it db\_name} (string)\\
The access name of the ZBDB database containing geometry.

\item{\it j} (integer)\\
The X index of the current fine grid cell.

\item{\it i} (integer)\\
The Y index of the current fine grid cell.

\item{\it spt\_x} (real)\\
The X coordinate value in microns of the current grid cell in a spatial
parameter table:\\
{\it coarse\_grid\_cell\_left} + {\it j\/}*{\it fine\_grid\_size} +
 {\it fine\_grid\_size\/}{\vt /2}

\item{\it spt\_y} (real)\\
The Y coordinate value in microns of the current grid cell in a spatial
parameter table:\\
{\it coarse\_grid\_cell\_bottom} + {\it i\/}*{\it fine\_grid\_size} +
 {\it fine\_grid\_size}{\vt /2}

\item{\it data} (real array)\\
An array containing miscellaneous parameters, described below).

\item{\it cell\_name} (string)\\
The name of the top-level cell.

\item{\it chd\_name} (string)\\
The access name of the CHD.
\end{description}

The {\it data} argument is an array that contains the following
parameters.

\begin{tabular}{ll}
\bf index & \bf description\\
0 & The spatial parameter table column size.\\
1 & The spatial parameter table row size.\\
2 & The fine grid period in microns.\\
3 & The coarse grid period in microns.\\
4 & The amount of grid cell expansion in microns.\\
5 & Area of interest left in microns.\\
6 & Area of interest bottom in microns.\\
7 & Area of interest right in microns.\\
8 & Area of interest top in microns.\\
9 & Coarse grid cell left in microns.\\
10 & Coarse grid cell bottom in microns.\\
11 & Coarse grid cell right in microns.\\
12 & Coarse grid cell top in microns.\\
13 & Fine grid cell left in microns.\\
14 & Fine grid cell bottom in microns.\\
15 & Fine grid cell right in microns.\\
16 & Fine grid cell top in microns.\\
\end{tabular}

The trapezoid data for the grid cells can be accessed, from within the
callback function, with the {\vt GetZlistZbdb} function.
\begin{quote}
{\vt GetZlistZbdb}({\it db\_name\/}, {\it layer\_lname\/}, {\it j\/},
{\it i\/})
\end{quote}

Example:\\
Here is a function that simply prints out the fine grid indices, and
the number of trapezoids in the grid location on a layer named ``{\vt
M1}''.
\begin{quote}\vt
function myfunc(dbname, j, i, x, y, prms)\\
\hspace*{2em}zlist = GetZlistZbdb(dbname, "M1", j, i)\\
\hspace*{2em}Print("Location", j, i, "contains", Zlength(zlist),
 "zoids on M1")\\
endfunc
\end{quote}

If the function returns a nonzero value, the operation will abort.
If there is no explicit return statement, the return value is 0.
\begin{quote}
{\vt if (}{\it some error\/}{\vt )}\\
\hspace*{2em}{\vt return 1}\\
{\vt end}
\end{quote}

If all goes well, {\vt ChdIterateOverRegion} returns 1, otherwise 0 is
returned, with an error message possibly available from {\vt
GetError}.

This function is intended for OEM users, customization is possible. 
Contact Whiteley Research for more information.

%------------------------------------
% 030113
\index{ChdWriteDensityMaps function}
\item{(int) \vt ChdWriteDensityMaps({\it chd\_name\/}, {\it cellname\/},
 {\it array\/}, {\it coarse\_mult\/}, {\it fine\_grid\/}, {\it bloat\/},
 {\it save})}\\
This function uses the same framework as {\vt ChdIterateOverRegion},
but is hard-coded to extract density values only.  The {\it
chd\_name\/}, {\it cellname\/}, {\it array\/}, {\it coarse\_mult\/},
and {\it fine\_grid} arguments are as described for that function.

When called, the function will iterate over the given area, and
compute the fraction of dark area for each layer in a fine grid cell,
saving the values in a spatial parameter table (SPT, see \ref{spt}). 
The access names of these SPTs are in the form {\it cellname\/}.{\it
layername\/}, where {\it cellname} is the name of the top-level cell
being processed.  The {\it layername} is the name of the layer,
possibly in hex format as used elsewhere.

If the boolean {\it save} argument is nonzero, the SPTs will be
retained in memory after the function returns.  Otherwise, the SPTs
will be dumped to files in the current directory, and destroyed.  The
file names are the same as the SPT names, with a ``{\vt .spt}''
extension added.  These files can be read with {\vt ReadSPtable}, and
are in the format described for that function, with the ``reference
coordinates'' the central points of the fine grid cells.

If all goes well, {\vt ChdWriteDensityMaps} returns 1, otherwise 0 is
returned, with an error message possibly available from {\vt
GetError}.
\end{description}


\subsection{Cell Geometry Digest}

\begin{description}
%------------------------------------
% 012111
\index{OpenCellGeomDigest function}
\item{(string) \vt OpenCellGeomDigest({\it idname\/}, {\it string\/},
  {\it type\/})}\\
This function returns an access name to a new Cell Geometry Digest
(CGD) which is created in memory.  A CGD is a data structure that
provides access to cell geometry saved in compact form, and does not
use the main cell database.  The CGD refers to physical data only. 
The new CGD will be listed in the {\cb Cell Geometry Digests} panel,
and the access name is used by other functions to access the CGD.

See the table in \ref{features} for the features that apply during a
call to this function.  In particular, the names of cells saved in the
CGD reflect any aliasing that was in force at the time the CGD was
created.

The first argument is a specified access name (which will be returned
on success).  This name can not be in use, meaning that the name can
not access an existing CGD which is currently linked to a CHD.  If
there is a name match to an unlinked CGD, the new CGD will replace the
old (which is destroyed).  This argument can be passed 0 or an empty
string.  If a null or empty string is passed, a new access name will
be generated and assigned.

The third argument is an integer 0--2 which specifies the type of CGD
to create.  The second ({\it string\/}) argument depends on what type
of CGD is being created.

\begin{description}
\item{Type 0 (actually, {\it type} not 1 or 2)}\\
This will create a ``memory'' CGD, where all geometry data will be
stored in memory, in highly-compressed form.  This provides the most
efficient access, but very large databases may exceed memory
limitations.

In this mode, the {\it string} argument can be one of the following:
\begin{enumerate}
\item{A layout (archive) file.  The file will be read and the
geometry extracted.}
\item{The access name of a Cell Hierarchy Digest (CHD) in memory.
The CHD will be used to read the geometry from the file it
references.}
\item{A saved CHD file.  The file will be read, and a new CHD will
be created in memory.  This CHD will be used to read the geometry
from the file referenced.}
\item{A saved CGD file name.  The file will be read into an
in-memory CGD.}
\end{enumerate}

Files are opened from the library search path, if a full path is
not provided.

\item{Type 1}\\
This will create a ``file'' CGD, where geometry data are stored in a
CGD file on disk, and geometry is retrieved when needed via saved
file offsets.  This uses less memory, but is not quite as fast as
saving geometry data in memory.  It is generally much faster than
reading geometry from the original layout file since 1) the data
are highly compressed, and 2) the objects are pre-sorted by layer.

In this mode, the {\it string} is a path to a saved CGD file, or to a
saved CHD file containing geometry records.  The in-memory CGD will
access this file.  The file is opened from the library search path, if
a full path is not provided.

\item{Type 2}\\
This will create a stub CGD which obtains geometry information from a
remote host which is running {\Xic} in server mode.  The server must
have a CGD in memory, from which data are obtained.

In this mode, the {\it string} must be in the format
\begin{quote}
{\it hostname\/}[{\vt :}{\it port\/}]{\vt /}{\it idname}
\end{quote}

The [...] indicates ``optional'' and is not literal.  The
{\it hostname} is the network name of the machine running the server. 
If the server is using a non-default port number, the same port number
should be provided after the host name, separated by a colon. 
Following the hostname or port is the access name on the server of the
CGD to access, separated by a forward slash.  The entire string should
contain no white space.
\end{description}

On error, a null string is returned, and an error message may be
available with the {\vt GetError} function.

%------------------------------------
% 030209
\index{NewCellGeomDigest function}
\item{(string) \vt NewCellGeomDigest()}\\
This function creates a new, empty Cell Geometry Digest, and returns
the access name.  The {\vt CgdAddCells} function can be used to add
cell geometry.

%------------------------------------
% 012111
\index{WriteCellGeomDigest function}
\item{(int) \vt WriteCellGeomDigest({\it cgd\_name\/}, {\it filename\/})}\\
This function will write a disk file representation of the Cell
Geometry Digest (CGD) associated with the access name given as the
first argument, into the file whose name is given as the second
argument.  Subsequently, the file can be read with {\vt
OpenCellGeomDigest} to recreate the CGD.  The file has no other
use and the format is not documented.

The function returns 1 if the file was written successfully, 0
otherwise, with an error message likely available from {\vt GetError}.

%------------------------------------
% 100108
\index{CgdList function}
\item{(stringlist\_handle) \vt CgdList()}\\
This function returns a handle to a list of access strings to Cell
Geometry Digests that are currently in memory.  The function never
fails, though the handle may reference an empty list.

%------------------------------------
% 100108
\index{CgdChangeName function}
\item{(int) \vt CgdChangeName({\it old\_cgd\_name\/},
 {\it new\_cgd\_name\/})}\\
This function allows the user to change the access name of an existing
Cell Geometry Digest (CGD) to a user-supplied name.  The new name must
not already be in use by another CGD.

The first argument is the access name of an existing CGD, the second
argument is the new access name, with which the CGD will subsequently
be accessed.  This name can be any text string, but can not be null.

The function returns 1 on success, 0 otherwise, with an error message
likely available from {\vt GetError}.

%------------------------------------
% 100108
\index{CgdIsValid function}
\item{(int) \vt CgdIsValid({\it cgd\_name\/})}\\
This function returns one if the string argument is an access name of
a Cell Geometry Digest currently in memory, zero otherwise.

%------------------------------------
% 012111
\index{CgdDestroy function}
\item{(int) \vt CgdDestroy({\it cgd\_name\/})}\\
The string argument is the access name of a Cell Geometry Digest (CGD)
currently in memory.  If the CGD is not currently linked to a Cell
Hierarchy Digest (CHD), then the CGD will be destroyed and its memory
freed.  One is returned on success, zero otherwise, with an error
message likely available with {\vt GetError}.

%------------------------------------
% 030209
\index{CgdIsValidCell function}
\item{(int) \vt CgdIsValidCell({\it cgd\_name\/}, {\it cellname\/})}\\
This function will return 1 if a Cell Geometry Digest (CGD) with an
access name given as the first argument exists and contains data for
the cell whose name is given as the second argument.  Otherwise, 0 is
returned.

%------------------------------------
% 030209
\index{CgdIsValidLayer function}
\item{(int) \vt CgdIsValidLayer({\it cgd\_name\/}, {\it cellname\/},
 {\it layername\/})}\\
This function returns 1 if the {\it cgd\_name} is an access name of a
Cell Geometry Digest (CGD) in memory, which contains a cell {\it
cellname} that has data for layer {\it layername}.  Otherwise, 0 is
returned.

%------------------------------------
% 012111
\index{CgdRemoveCell function}
\item{(int) \vt CgdRemoveCell({\it cgd\_name\/}, {\it cellname\/})}\\
This function will remove and destroy the data for the cell {\it
cellname} from the Cell Geometry Digest (CGD) with access name {\it
cgd\_name\/}.  This applies to all CGD types, as described for {\vt
OpenCellGeomDigest}.  If the CGD is accessing geometry from a remote
server, the cell data are removed from the server.

The names of cells that have been removed are retained, and can be
checked with {\vt CgdIsCellRemoved}.

If the CGD is found and it contains {\it cellname\/}, the cell data
are destroyed and the function returns 1.  Otherwise, 0 is returned,
with an error message available from {\vt GetError}.

%------------------------------------
% 012111
\index{CgdIsCellRemoved function}
\item{(int) \vt CgdIsCellRemoved({\it cgd\_name\/}, {\it cellname\/})}\\
This function returns 1 if a CGD is found with access name as given in
{\it cgd\_name}, and the {\it cellname} is the name of a cell that has
been removed from the CGD, for example with {\vt CgdRemoveCell}. 
Otherwise, the return value is 0.

%------------------------------------
% 012111
\index{CgdRemoveLayer function}
\item{(int) \vt CgdRemoveLayer({\it cgd\_name\/}, {\it cellname\/},
 {\it layername\/})}\\
If the Cell Geometry Digest (CGD) exists, and contains data for a cell
{\it cellname} that contains data for {\it layername\/}, the {\it
layername} data will be deleted from the {\it cellname} record, and
the function returns 1.  Otherwise, 0 is returned, with an error
message likely available from {\vt GetError}.

This applies to memory and file type CGDs, as described for {\vt
OpenCellGeomDigest}.  The data, if found, are freed, and (unlike {\vt
CgdRemoveCell}) no record of removed layers is retained.  This
actually reduces memory use only for memory type CGDs.

%------------------------------------
% 012111
\index{CgdAddCells function}
\item{(int) \vt CgdAddCells({\it cgd\_name\/}, {\it chd\_name\/},
 {\it cells\_list\/})}\\
This function will add a list of cells to the Cell Geometry Digest
(CGD) whose access name is given as the first argument.  The cells
will be read using the Cell Hierarchy Digest (CHD) whose access name
is given as the second argument.

This, and the {\vt CgdRemoveCell} function can be used to implement a
cache for cell data.  When a CHD is used for access, and a CGD has
been linked to the CHD, the CHD will read geometry information for
cells in the CGD from the CGD, and cells not found in the CGD will be
read from the layout file.  Thus, if memory is tight, one can put only
the heavily-used cells into the CGD, instead of all cells.

If the CGD already contains data for a cell to add, the data will be
overwritten with the new cell data.
    
For the {\it cells\_list} argument, one can pass either a handle to a
list of strings that contain cell names, or a string containing
space-separated cell names.  If a cell named in the list is not found
in the CHD, it will be silently ignored.

This applies to memory and file type CGDs, as described for {\vt
OpenCellGeomDigest}.  The geometry records are saved in memory,
whether or not the CGD is file type.  Individual records set the
access method, so it is possible to have mixed file access and memory
access records in the same CGD.

On success, 1 is returned.  If an error occurs, 0 is returned, and a
message may be available from {\vt GetError}.

%------------------------------------
% 100108
\index{CgdContents function}
\item{(stringlist\_handle) \vt CgdContents({\it cgd\_name\/},
 {\it cellname\/}, {\it layername\/})}\\
This function returns content listings from the Cell Geometry Digest
(CGD) whose access name is given in the first argument.  The remaining
string arguments give the cell name and layer name to query.  Either
or both of these arguments can be null (passed 0).

If the {\it cellname} is null, a handle to a list if strings giving
the cell names in the CGD is returned.  otherwise, the {\it cellname}
must be a cell name from the CGD.

If {\it layername} is null, the return value is handle to a list of
layer name strings for layers used in {\it cellname}.  If {\it
layername} is not null, it should be one of the layer names contained
in the {\it cellname}.

The return value when both {\it cellname} and {\it layername} are
non-null is a handle to a list of two strings.  The first string gives
the integer number of bytes of compressed geometry for the cell/layer. 
The second string gives the size of the geometry string after
decompression.  The compressed size can be 0, in which case
compression was not used as the block is too small for compression to
be effective.

If the arguments are unresolved, the return value is a scalar 0.

%------------------------------------
% 100108
\index{CgdOpenGeomStream function}
\item{(gs\_handle) \vt CgdOpenGeomStream({\it cgd\_name\/},
 {\it cellname\/}, {\it layername\/})}\\
This function creates a handle to an iterator for decompressing the
geometry in a Cell Geometry Digest (CGD).  The first argument is the
access name of the CGD.  The second argument is the name of one of the
cells contained in the CGD.  The third argument is the name of a layer
used by the cell.  The cells and layers in the CGD can be listed with
{\vt CgdContents}.

The return value is a handle to an incremental reader, loaded with the
compressed geometry for the cell and layer.  This can be passed to
{\vt GsReadObject} to obtain the geometrical objects.

The {\vt Close} function can be used to destroy the reader.  It will
be closed automatically if\newline
{\vt GsReadObject} iterates through all objects contained in the
stream.

A scalar 0 is returned if the arguments are not resolved.

%------------------------------------
% 100108
\index{GsReadObject function}
\item{(object\_handle) \vt GsReadObject({\it gs\_handle\/})}\\
This function takes the handle created with {\vt CgdOpenGeomStream}
and returns an object handle which points to a single object.  A
different object will be returned with each call until all objects
have been returned, at which time the geometry stream handle is
closed.  Further calls will return a scalar 0.

The {\vt ConvertReply} function can also return a handle for use by
this function.

%------------------------------------
% 100108
\index{GsDumpOasisText function}
\item{(int) \vt GsDumpOasisText({\it gs\_handle\/})}\\
This function will dump the geometry stream in OASIS ASCII text
representation to the console window (standard output).  The handle is
freed.  This may be useful for debugging.

\end{description}


\subsection{Assembly Stream}

% 033009
These functions implement a functionality similar to the {\cb
!assemble} command.

\begin{description}
%------------------------------------
% 033009
\index{StreamOpen function}
\item{(stream\_handle) \vt StreamOpen({\it outfile})}\\
Open an assembly stream to the file {\it outfile}.  The file format
that will be used is obtained from the extension of the name given,
which must be one of

\begin{tabular}{ll}
CGX   & \vt .cgx\\
CIF   & \vt .cif\\
GDSII & \vt .gds, .str, .strm, .stream\\
OASIS & \vt .oas\\
\end{tabular}

If successful, a handle to the stream control structure is returned,
which can be passed to other functions which require this data type. 
A scalar zero is returned on error.  The returned handle is used to
implement processing of archive data similar to the {\cb !assemble}
command.

%------------------------------------
% 100108
\index{StreamTopCell function}
\item{(int) \vt StreamTopCell({\it stream\_handle}, {\it cellname\/})}\\
Define the name of a top-level cell that will be created in the output
stream.  At most one definition is possible in a stream.  If
successful 1 is returned, otherwise 0 is returned.
 
%------------------------------------
% 033009
\index{StreamSource function}
\item{(int) \vt StreamSource({\it stream\_handle}, {\it file\_or\_chd\/},
  {\it scale\/}, {\it layer\_filter\/}, {\it name\_change\/})}\\
This function will add a source specification to a stream.  The
specification can refer to either an archive file, or to a Cell
Hierarchy Digest (CHD).  Upon successful return, the source will be
queued for writing to the stream (initiated with {\vt StreamRun}). 
Arguments set various modes and conditions that will apply during the
write.

This function specifies the equivalent of a Source Block as described
for the {\cb !assemble} command.  The {\vt StreamInstance} function is
used to add ``Placement Blocks''.

\begin{description}
\item{\it stream\_handle}\\
Handle to the stream object.

\item{\it file\_or\_chd}\\
This argument can be either a string giving a path to an archive
file, or the access name of a Cell Hierarchy Digest in memory.

\item{\it scale}\\
This is a scaling factor which applies only when streaming the entire
file, which will occur if no instances are specified for the source
with the {\vt StreamInstance} function.  It is ignored if an
instance is specified.  When used, all coordinates read from the
source file will be multiplied by the factor, which can be in the
range 0.001 -- 1000.0.

\item{\it layer\_filter}\\
This is a switch integer that enables or disables use of the layer
filtering and aliasing capability.  If 0, no layer filtering or
aliasing will be done.  If nonzero, layer filtering and aliasing will
be be performed when reading from the source, according to the present
values of the variables listed below.  These values are saved, so that
the variables can subsequently change.
\begin{quote} \vt
LayerList\\
UseLayerList\\
LayerAlias\\
UseLayerAlias
\end{quote}

If needed, these variables should be set to the desired values before
calling this function, then reset to the previous values after the
call.  This can be done with the {\vt Get} and {\vt Set} functions.

\item{name\_change}\\
This is a switch integer that enables or disables use of the Cell Name
Mapping capability.  If 0, no cell name changes are done, except that
if a name clash is detected, a new name will be supplied, similar to
the auto-aliasing feature.  If nonzero, cell name mapping will be
performed when the source is read according to the present values of
the variables listed below.  These values are saved, so that the
variables can subsequently change.
\begin{quote} \vt
InCellNamePrefix\\
InCellNameSuffix\\
InToLower\\
InToUpper
\end{quote}

If needed, these variables should be set to the desired values before
calling this function, then reset to the previous values after the
call.  This can be done with the {\vt Get} and {\vt Set} functions.
\end{description}

The function returns one on success, zero otherwise with an error
message probably available through {\vt GetError}.

%------------------------------------
% 022916
\index{StreamInstance function}
\item{(int) \vt StreamInstance({\it stream\_handle}, {\it cellname\/},
  {\it x\/}, {\it y\/}, {\it my\/}, {\it rot\/}, {\it magn\/},
  {\it scale\/}, {\it no\_hier\/},\newline
  {\it ecf\_level\/}, {\it flatten\/}, {\it array\/}, {\it clip\/})}\\
This function will add a placement name to the most recently added
source file (using {\vt StreamSource}).  A source must have been
specified before this function can be called successfully.  This
function specifies the equivalent of a Placement Block as described
for the {\cb !assemble} command.

The {\it cellname} must match the name of a cell found in the source,
including any aliasing in effect.  There are two consequences of
calling this function:  the named cell and possibly its subcell
hierarchy will be written to output, and if a top cell was specified
(with {\vt StreamTopCell}), an instance of the named cell will be
placed in the top cell.  The placement is governed by the {\it x\/},
{\it y\/}, {\it my\/}, {\it ang\/}, and {\it magn\/} arguments, which
are ignored if there is no top cell.

The {\it x\/},{\it y} are the translation coordinates of the cell
origin.  The {\it my} is a flag indicating Y-reflection before
rotation.  The {\it ang} is the rotation angle, in degrees, and must
be a multiple of 45 degrees.  The {\it magn} is the magnification
factor for the placement.  These apply to the instantiation only, and
have no effect on the cell definitions.

The remaining arguments affect the cell definitions that are created
in the output file.

\begin{description}
\item{\it scale}\\
This is a scale factor by which all coordinates are scaled in cell
definition output, and is a real number in the range 0.001 -- 1000.0. 
This is different from the {\it magn} factor, which applies only to
the instance placement.

\item{\it no\_hier}\\
This is a boolean value that when nonzero indicates that only the
named cell, and not its hierarchy, is written to output.  This can
cause the output file to have unresolved references.

\item{\it ecf\_level}\\
This is an integer 0--3 which specifies the empty cell filtering level
as described for the {\cb Format Conversion} panel in \ref{ecfilt}. 
The values are

\begin{tabular}{ll}
0 & No empty cell filtering.\\
1 & Apply pre- and post-filtering.\\
2 & Apply pre-filtering only.\\
3 & Apply post-filtering only.\\
\end{tabular}

\item{\it flatten}\\
If the boolean variable {\it flatten} is nonzero, the objects in the
hierarchy under {\it cellname} will be created in {\it cellname\/},
thus only one cell, containing all geometry, will be written.

\item{\it array}\\
If the {\it array} argument is passed 0, no windowing will be used. 
Otherwise the {\it array} should have four components which specify a
rectangle, in microns, in the coordinates of {\it cellname\/}.  The
values are

\begin{tabular}{ll}
{\it array\/}{\vt [0]} & X left\\
{\it array\/}{\vt [1]} & Y bottom\\
{\it array\/}{\vt [2]} & X right\\
{\it array\/}{\vt [3]} & Y top\\
\end{tabular}

If an {\it array} is given, only the objects and subcells needed to
render the window will be written.

\item{\it clip}\\
If the boolean value {\it clip} is nonzero and an {\it array} is
given, objects will be clipped to the window.  Otherwise no clipping
is done.
\end{description}

The function returns one on success, zero otherwise with an error
message probably available through {\vt GetError}.

%------------------------------------
% 100108
\index{StreamRun function}
\item{(int) \vt StreamRun({\it stream\_handle})}\\
This function will initiate the writing from the sources previously
specified with {\vt SteamSource} into the output file.  The real work
is done here.  The function returns one on success, zero otherwise
with an error message probably available through {\vt GetError}.

\end{description}


%------------------------------------------------------------------------------
\section{Geometry Editing Functions 1}
\subsection{General Editing}

\begin{description}
%------------------------------------
% 041704
\index{ClearCell function}
\item{(int) \vt ClearCell({\it undoable\/}, {\it layer\_list\/})}\\
This function will clear the content of the present mode (electrical
or physical) part of the current cell.  If the first argument is
nonzero, the deletions will be added to the internal undo list,
otherwise not.  The latter is more efficient, though this makes the
deletions irreversible.  The second argument, if null or empty,
indicates that all objects on all layers will be deleted, including
subcells.  Otherwise this can be set to a string containing a
space-separated list of layer names, following an optional special
character `{\vt !}' or `{\vt \symbol{94}}' which must be the first
character in the string if used.  If the special character does not
appear, the deletions apply only to the layers listed.  If the special
character appears, the deletions apply only to the layers {\it not}
listed.  Recall that the internal name for the layer that contains
subcells ls ``{\vt \$\$}'', thus for example using ``{\vt !  \$\$}''
would delete all geometry but retain the subcells.

The return value is the number of objects deleted.

%------------------------------------
% 030204
\index{Commit function}
\item{\vt Commit()}\\
The Commit functions terminates the present operation, adding it to
the undo list.  It will also redisplay any changes.  This function
should be called after each change or after a group of related
changes.  It is implicitly called when a script exits.

%------------------------------------
% 030204
\index{undo}
\index{Undo function}
\item{\vt Undo()}\\
This function will undo the most recent operation.

%------------------------------------
% 030204
\index{redo}
\index{Redo function}
\item{\vt Redo()}\\
This function will redo the last undone operation.

%------------------------------------
% 030204
\index{SelectLast function}
\item{(int) \vt SelectLast({\it types\/})}\\
This function selects objects that have been created by the script
functions since the last call to {\vt Commit} or {\vt SelectLast}
(which calls {\vt Commit}), according to {\it type\/}.  The {\it
type\/} argument is a string whose characters serve to enable
selection of a given type of object:  `{\vt b}' for boxes, `{\vt p}'
for polygons, `{\vt w}' for wires, `{\vt l}' for labels, and `{\vt c}'
for instances.  If this string is empty or null, then all objects will
be selected.  Objects that are created using {\vt PressButton} or
otherwise using {\Xic} input implicitly call {\vt Commit}, so can't be
selected in this manner.
\end{description}


\subsection{Current Transform}

\begin{description}
%------------------------------------
% 030115
\index{SetTransform function}
\item{(int) \vt SetTransform({\it angle\_or\_string\/}, {\it reflection\/},
 {\it magnification\/})}\\
This function sets the ``current transform'' to the values provided. 
It is similar in action to the controls in the {\cb Current Transform}
panel.  The first argument can be a floating point angle that will be
snapped to the nearest multiple of 45 degrees in physical mode, 90
degrees in electrical mode.  If bit 1 of {\it reflection\/} is set, a
reflection of the x-axis is specified.  If bit 2 of {\it reflection\/}
is set, a reflection of the y-axis is specified.  The {\it
magnification\/} sets the scaling applied to transformed objects, and
is accepted only while in physical mode.  It is ignored if less than
or equal to zero.

The first argument can alternatively be a string, in the format as
returned from {\vt GetTransformString}.  The string will be parsed,
and if no error the transform will be set.  The two remaining
arguments are ignored, but must be given (0 can be passed for both).

The return value is 1 on success, 0 otherwise.

Examples:
\begin{quote}\rr
Set rotation 180, mirror the X axis:\\
\hspace*{1em}{\vt SetTransform(180, 1, 1)} or
  {\vt SetTransform("R180MX", 0, 0)}\\
Set rotation 180, mirror the Y axis:\\
\hspace*{1em}{\vt SetTransform(180, 2, 1)} or
  {\vt SetTransform("R180MY", 0, 0)}\\
Set rotation 180, mirror both X,Y axes:\\
\hspace*{1em}{\vt SetTransform(180, 3, 1)} or
  {\vt SetTransform("R180MYMX", 0, 0)}\\
\end{quote}

%------------------------------------
% 030204
\index{StoreTransform function}
\item{(int) \vt StoreTransform({\it register\/})}\\
This function will save the current transform settings into a
register, which can be recalled with {\vt RecallTransform}.  The
argument is a register number 0--5.  These correspond to the ``last''
and registers 1--5 in the {\cb Current Transform} pop-up.  This
function returns 1 on success, 0 if the argument is out of range.

%------------------------------------
% 030204
\index{RecallTransform function}
\item{(int) \vt RecallTransform({\it register\/})}\\
This function will restore the transform settings previously saved
with {\vt StoreTransform}.  The argument is a register number 0--5. 
These correspond to the ``last'' and registers 1--5 in the {\cb
Current Transform} pop-up.  This function returns 1 on success, 0 if
the argument is out of range.

%------------------------------------
% 030215
\index{GetTransformString function}
\item{(string) \vt GetTransformString()}\\
Return a string describing the current transform, an empty string will
indicate the identity transform.  The string is a sequence of tokens
and contains no white space.  It is the same format used to indicate
the current transform in the {\Xic} status line.  The tokens are:

\begin{quote}
[{\vt R}{\it ang\/}][{\vt MY}][{\vt MX}][{\vt M}{\it magn\/}]
\end{quote}

The square brackets indicate that each token is optional and do not
appear in the string.  If the rotation angle is nonzero, the first
token will appear, where {\it ang} is the angle in degrees.  This is
an integer multiple of 45 degrees in physical mode, 90 degrees in
electrical mode, larger than zero and smaller than 360.

If reflection of Y or X is in force, one or both of the mext two
tokens will appear.  These are literal.  If the magnification is not
unity, the final token will appear, with {\it magn} being a real
number in the range 0.001 through 1000.0.
  
The order of the tokens must be as shown.

The returned string, or one in the same format, can be passed to
the first argument of {\vt SetTransform}.

%------------------------------------
% 030204
\index{GetCurAngle function}
\item{(int) \vt GetCurAngle()}\\
This returns the rotation angle of the current transform, in degrees. 
This will be 0, 45, 90, 135, 180, 225, 270, 315 in physical mode, or
0, 90, 180, 270 in electrical mode.  The {\vt SetTransform} function
can be used to set the rotation angle.

%------------------------------------
% 030204
\index{GetCurMX function}
\item{(int) \vt GetCurMX()}\\
This returns 1 if the current transform mirrors the x-axis, 0
otherwise.  The {\vt SetTransform} function can be used to set the
mirror transformations.

%------------------------------------
% 030204
\index{GetCurMY function}
\item{(int) \vt GetCurMY()}\\
This returns 1 if the current transform mirrors the y-axis, 0
otherwise.  The {\vt SetTransform} function can be used to set the
mirror transformations.

%------------------------------------
% 030204
\index{GetCurMagn function}
\item{(real) \vt GetCurMagn()}\\
This returns the magnification component of the current transform. 
The {\vt SetTransform} function can be used to set the magnification.

%------------------------------------
% 041705
\index{UseTransform function}
\item{(int) \vt UseTransform({\it enable\/}, {\it x\/}, {\it y\/})}\\
This command enables and disables use of the current transform in the
{\vt ShowGhost} function, as well as the functions that create
objects:  {\vt Box}, {\vt Polygon}, {\vt Arc}, {\vt Wire}, and {\vt
Label}.  The functions {\vt Move}, {\vt Copy}, {\vt Logo}, and {\vt
Place} naturally use the current transform and are unaffected by this
function.

All arguments are numeric.  If the first argument is nonzero, the
current transformation will be used in subsequent calls to the
functions listed above.  If the first argument is zero, the current
transform is ignored by these functions.  The remaining arguments
provide the translation applied to the object being created, before
the current transform is applied.

If {\vt UseTransform(1, ...)} has been given, {\vt ShowGhost} will
apply the current transform to the list of objects to display, using
the pointer location as the translation rather than the {\it x}, {\it
y} supplied to {\vt UseTransform}, which are ignored.  The other
functions listed above will create the object after applying the
current transform, using {\it x}, {\it y}.

In some scripts, it will be necessary to call {\vt UseTransform(1,
...)} twice, once to enable {\vt ShowGhost}, and again after the
location for the new object is obtained.  In particular, if {\vt
Point} is used to obtain the coordinate, {\vt UseTransform} should be
called before {\vt Point} (so the ghost drawing will be accurate) and
again with the coordinates returned from {\vt Point} before the new
object is created.

The {\vt Box} function will actually create a polygon if the current
transform is being used and the rotation angle is 45 degrees or one of
the other non-Manhattan angles.  The {\vt Polygon} function will
actually create a box if the rotated figure can be so represented. 
The {\vt Polygon} function will never create boxes unless use of the
current transform is enabled.

Below is an example script that will place boxes on the current layer
where the user clicks.  Note that the size and rotation angle of the
box can be changed while in the script through the {\cb Transform
Menu}.

\begin{verbatim}
ShowPrompt("Click to place boxes")
PushGhostBox(0, 0, 1, 1)
UseTransform(1, 0, 0)
while (1)
    ShowGhost(8)
    a[2]
    if !Point(a)
        ShowPrompt("")
        Exit()
    end
    ShowGhost(0)
    UseTransform(1, a[0], a[1])
    Box(0, 0, 1, 1)
    Commit()
end
\end{verbatim}

\end{description}


% funcs_geom1:drvlyr 032217
\subsection{Derived Layers}
\label{drvlyrfunc}
%------------------------------------

These functions provide an interface to the derived layer capability
(see \ref{drvlyr}).  Derived layers are invisible internal layers that
imply geometry resulting from evaluation of a layer expression, which
may involve normal layers and other derived layers.  Derived layers
are recognized by name in layer expressions.

There are actually two implementations of derived layer functionality. 
The interface functions allow explicit choice of which evaluation
method to use.  Within {\Xic}, this detail is generally invisible to
the user.

In the original implementation, developed for the DRC system, the
geometry of derived layers must be created or updated before the
derived layer is referenced.  In use, reference to a derived layer in
a layer expression retrieves this geometry, very similar to what
happens when a normal layer is referenced.  Ordinarily, the derived
layer geometry will be cleared after final use.  This method may be
fast when the same layer expressions must be evaluated many times, so
it seems a good match for DRC, where it is used.

To use this method, the interface function {\vt EvalDerivedLayers} is
called to create the geometry for each derived layer that will be
evaluated.  Then, {\vt GetDerivedLayerLexpr} is called with a boolean
true second argument to get the evaluation objects as needed, which
are evaluated to create new geometry.  When done, {\vt
ClearDerivedLayers} is called to destroy the precomputed geometry.

In the second mode of operation, when the parse tree for the derived
layer is created, references to derived layers will be recursively
parsed and stitched into the tree.  The final parse tree will contain
normal layers only, and can therefor be evaluated in any context,
without the need for precomputed geometry caches.

With this method, there is no need to call {\vt EvalDerivedLayers} and
{\vt ClearDerivedLayers}, as there is no use of cached geometry.  The
evaluation object is returned from {\vt GetDerivedLayerLexpr} with a
boolean false second argument.

\begin{description}
% 010815
\index{AddDerivedLayer function}
\item{(int) \vt AddDerivedLayer({\it lname\/}, {\it index\/},
 {\it lexpr\/})}\\
This will add a derived layer to the database, under the name given in
the first argument.  The second argument is an integer layer number
for the layer, which is used for ordering when the derived layers are
printed, for example to an updated technology file.  If not positive,
{\Xic} will generate a number to be used for a new layer.  Numbers
need not be unique, sorting is alphabetic among derived layer names
with the same index number.  If a derived layer of the same name
already exists, it will be silently overwritten.

The third argument is a string starting with an optional keyword
followed by a layer expression, separated by space.  The keyword is
one of {\vt join}, {\vt split}, or {\vt splitv}.  These are the same
keywords, and have the same effects, as is explained for the {\et
DerivedLayer} keyword in the technology file.  The expression can
reference by name ordinary layers and derived layers.  The expression
is not parsed until evaluation time.

The function fails if either the {\it lname} or {\it lexpr} are null
or empty strings.

% 120114
\index{RemDerivedLayer function}
\item{(int) \vt RemDerivedLayer({\it lname\/})}\\
If a derived layer exists with the given name, remove the definition
from the internal registry, so that the derived layer definition and
any existing geometry becomes inaccessible.  The derived layer
definition can be restored with {\vt AddDerivedLayer}.  If the derived
layer is found and removed, this function will return 1, otherwise 0
is returned.

% 120114
\index{IsDerivedLayer function}
\item{(int) \vt IsDerivedLayer({\it lname\/})}\\
This function will return 1 if the string argument matches a derived
layer name in the database, 0 otherwise.  Matching is
case-insensitive.

The name can be in the form ``{\it layer\/}{\vt :}{\it purpose\/}'' as
for normal {\Xic} layers, however the entire token is taken verbatim. 
This is a subtle difference from normal layers, where for example
``{\vt m1:drawing}'' and ``{\vt m1}'' are equivalent (the {\vt
drawing} purpose being the default).  As derived layer names, the two
would differ, and the notion of a purpose does not apply to derived
layers.

% 120114
\index{GetDerivedLayerIndex function}
\item{(int) \vt GetDerivedLayerIndex({\it lname\/})}\\
This returns a positive integer which is the layer index number of the
derived layer whose name was given, or 0 if no derived layer can be
found with that name (case insensitive).

% 120114
\index{GetDerivedLayerExpString function}
\item{(string) \vt GetDerivedLayerExpString({\it lname\/})}\\
This returns the layer expression string for the derived layer whose
name is passed.  If the derived layer is not found, a null string is
returned.

% 032217
\index{GetDerivedLayerLexpr function}
\item{(layer\_expr) \vt GetDerivedLayerLexpr({\it lname\/}, {\it noexp\/})}\\
This returns a parsed layer expression object created from the layer
expression of the derived layer whose name is passed.  This can be
passed to other functions which can use this data type.  If there is a
parse error, the function fails fatally.  Otherwise the return is a
valid parse tree object.

The boolean second argument will suppress derived layer expansion if
set.

There are two ways to handle derived layers.  Generally, layer
expression parse trees are expanded (second argument is false),
meaning that when a derived layer is encountered, the parser
recursively descends into the layer's expression.  The resulting tree
references only normal layers, and evaluation is straightforward.

A second approach might be faster.  The parse trees are not expanded
(second argument is true), and a parse node to a derived layer
contains a layer descriptor, just as for normal layers.  Before any
computation, {\vt EvalDerivedLayers} must be called, which actually
creates database objects in a database for the derived layer. 
Evaluation involves only finding the geometry in the search area, as
for a normal layer.

% 032217
\index{EvalDerivedLayers function}
\item{(string) \vt EvalDerivedLayers({\it list\/}, {\it array\/})}\\
Derived layer evaluation objects (such as the return from {\vt
GetDerivedLayerLexpr}) that are not recursively expanded must have
derived layer geometry precomputed before use.  This function creates
derived layer geometry for this purpose.

Evaluation creates the geometry described by the layer expression. 
Derived layers are never visible, so this geometry is internal, but
can be accessed, e.g., by design rule evaluation functions, or used to
create normal layers with the {\cb !layer} command or the {\cb
Evaluate Layer Expression} panel from the {\cb Edit Menu}.

The first argument is a string containing a list of derived layer
names, separated by commas or white space.  The function will evaluate
these derived layers, and any derived layers referenced in their layer
expressions, in an order such that the derived layers will be
evaluated before being referenced during another evaluation.

All geometry created will exist in the current cell, and the layer
expressions will source all levels of the hierarchy.  Any geometry
left in the current cell from a previous evaluation will be cleared
first.  Derived layer geometry in subcells is ignored.

The second argument can set the area where the layers will be
evaluated, which can be any rectangular region of the current cell. 
This can be an array of size four or larger, specifying left, bottom,
right, and top coordinates in microns in the 0, 1, 2, 3 indices.  The
argument can also be a scalar 0 which indicates to use the entire
current cell.

The return is a string listing all of the derived layers evaluated,
which will include derived layers referenced by the original list but
not included in the list.  This should be passed to {\vt
ClearDerivedLayers} when finished using the layers.

% 120114
\index{ClearDerivedLayers function}
\item{(int) \vt ClearDerivedLayers({\it list\/})}\\
The argument is a string containing a list of derived layer names,
separated by commas or white space.  This may be the return from {\vt
EvalDerivedLayers}.  All of the layers listed will be cleared in the
current cell.  If a layer name is not resolved as a derived layer, it
is silently ignored.  Clearing already clear layers is not an error. 
Derived layers should be cleared after their work is done, to recycle
memory.  The return value is an integer count of the number of derived
layers that were cleared.

\end{description}

% funcs_geom1:objbh 022712
\subsection{Object Management by Handles}
\label{objmanh}

The following functions provide a fairly complete interface to
database objects.

Internally, most of the ``{\vt Set}..." functions in this group modify
objects via application of the pseudo-properties (see
\ref{pseudoprops}).  This allows modification of most objects and
types, with the restrictions listed in the table below.  Without
restrictions, the functions can act on database objects or the
``object copies'' which are memory objects not part of any cell.  The
objects can be from electrical or physical cells, and the containing
cell (if any) need not be the current cell.  However, a restriction
when working with copies is that the object type can not be changed.

\begin{quote}
\begin{tabular}{lp{4in}}
boxes & no restrictions\\
polys & no restrictions\\
wires & can't accept electrical wires on the active (SCED) layer\\
labels & no restrictions\\
instances & can't accept electrical instances\\
\end{tabular}
\end{quote}

As mentioned, some of the functions generate or accept lists of
``object copies''.  These are objects that are not included in the
object database for any cell.  A list of copies behaves in most
respects like an ordinary object list.  The The {\vt CopyObjects}
function can be used to create a new database object from a copy.  The
handle manipulation functions such as {\vt HandleCat} work, but lists
of copies can {\it not} be mixed with lists of database objects, {\vt
HandleCat} will fail quietly if this is attempted.  Copies can not be
selected.

\begin{description}
%------------------------------------
% 062116
\index{ListElecInstances function}
\item{(object\_handle) \vt ListElecInstances()}\\
This function returns a handle to a complete list of cell instances
found in the electrical part of the current cell.  Operation is
identical in electrical and physical modes.  In the schematic, cell
instances represent subcircuits, devices, and pins.  The ``{\vt
GetInstance}'' functions described below can be used to obtain
information about the instances.

%------------------------------------
% 062116
\index{ListPhysInstances function}
\item{(object\_handle) \vt ListPhysInstances()}\\
This function returns a handle to a complete list of cell instances
found in the physical layout of the current cell.  Operation is
identical in electrical and physical modes.  The ``{\vt GetInstance}''
functions described below can be used to obtain information about the
instances.

%------------------------------------
% 030204
\index{SelectHandle function}
\item{(object\_handle) \vt SelectHandle()}\\
This function returns a handle to the list of objects currently
selected.  The list is copied internally, and so is unchanged if the
objects are subsequently deselected.

A handle to the object list is returned.  The {\vt ObjectNext}
function is used to advance the handle to point to the next object in
the list.  The {\vt HandleContent} function returns the number of
objects remaining in the list.

%------------------------------------
% 030204
\index{SelectHandleTypes function}
\item{(object\_handle) \vt SelectHandleTypes({\it types\/})}\\
This function returns a handle to a list of objects that are currently
selected, but only the types of objects specified in the argument are
included.  The argument is a string which specifies the types of
objects to include.  If zero or an empty string is passed, all types
are included, and the function is equivalent to {\vt SelectHandle}. 
Otherwise the characters in the string signify which objects to
include:

\begin{tabular}{ll}\\
`{\vt b}' & boxes\\
`{\vt p}' & polygons\\
`{\vt w}' & wires\\
`{\vt l}' & labels\\
`{\vt c}' & subcells
\end{tabular}

For example, passing ``{\vt pwb}'' would include polygons, wires, and
boxes only.  The order of the characters is unimportant.

%------------------------------------
% 030204
\index{AreaHandle function}
\item{(object\_handle) \vt AreaHandle({\it l\/}, {\it b\/}, {\it r\/},
 {\it t\/}, {\it types\/})}\\
This function creates a list of objects that touch the rectangular
area specified by the first four coordinates (which are the left,
bottom, right, and top values of the rectangle).  The fifth argument
is a string which specifies the types of objects to include.  If zero
or an empty string is passed, all types are included, otherwise the
characters in the string signify which objects to include:

\begin{tabular}{ll}\\
`{\vt b}' & boxes\\
`{\vt p}' & polygons\\
`{\vt w}' & wires\\
`{\vt l}' & labels\\
`{\vt c}' & subcells
\end{tabular}

For example, passing ``{\vt pwb}'' would list polygons, wires, and
boxes only.  The order of the characters is unimportant.

A handle to the object list is returned.  The {\vt ObjectNext}
function is used to advance the handle to point to the next object in
the list.  The {\vt HandleContent} function returns the number of
objects remaining in the list.

%------------------------------------
% 030204
\index{ObjectHandleDup function}
\item{(object\_handle) \vt ObjectHandleDup({\it object\_handle\/},
 {\it types\/})}\\
This function creates a new handle and list of objects.  The new
object list consists of those objects in the list referenced by the
argument whose types are given in the string {\it types} argument.  If
zero or an empty string is passed, all types are included, otherwise
the characters in the string signify which objects to include:

\begin{tabular}{ll}\\
`{\vt b}' & boxes\\
`{\vt p}' & polygons\\
`{\vt w}' & wires\\
`{\vt l}' & labels\\
`{\vt c}' & subcells
\end{tabular}

The return value is a handle, or 0 if an error occurred.  Note that
the new handle may be empty if there were no matching objects.  The
function will fail if the handle argument is not a pointer to an
object list.

%------------------------------------
% 030204
\index{ObjectHandlePurge function}
\item{(int) \vt ObjectHandlePurge({\it object\_handle\/}, {\it types\/})}\\
This function will purge from the list of objects referenced by the
handle argument objects with types listed in the {\it types} string. 
If zero or an empty string is passed, all types are deleted, otherwise
the characters in the string signify which objects to delete:

\begin{tabular}{ll}\\
`{\vt b}' & boxes\\
`{\vt p}' & polygons\\
`{\vt w}' & wires\\
`{\vt l}' & labels\\
`{\vt c}' & subcells
\end{tabular}

The return value is the number of objects remaining in the list.  The
function will fail if the handle argument does not reference a list of
objects.

%------------------------------------
% 030204
\index{ObjectNext function}
\item{(int) \vt ObjectNext({\it object\_handle\/})}\\
This function is called with a handle to a list of objects, and causes
the handle to reference the next object in the list.  If there are no
more objects, the handle is closed, and this function returns zero. 
Otherwise, 1 is returned.  This function will fail if the handle
passed is not a handle to an object list.

%------------------------------------
% 030204
\index{MakeObjectCopy function}
\item{(object\_handle) \vt MakeObjectCopy({\it numpts}, {\it array\/})}\\
This function creates an object copy from the {\it numpts} coordinate
pairs in the {\it array}.  The function returns an object list handle
referencing the ``copy'', which can be used in the same manner as
copies of ``real'' objects.  The coordinate list must be closed, i.e.,
the last coordinate pair must be the same as the first.  If the
coordinates represent a rectangle, a box object is created, otherwise
the object is a polygon.  Coordinates are in microns, relative to the
origin of the current cell.  The object is associated with the current
layer (but of course it really does not exist on that layer).

%------------------------------------
% 080705
\index{ObjectString function}
\item{(string) \vt ObjectString({\it object\_handle\/})}\\
This function returns a CIF-like string describing the object pointed
to by the given object handle.  This provides all of the geometric
information for the object.  Strings of this format can be reconverted
to object copies with the {\vt ObjectCopyFromString} function.

On error or for an empty handle, a null string is returned.  The
function will fail if the argument is not a handle to an object list.

%------------------------------------
% 080705
\index{ObjectCopyFromString function}
\item{(object\_handle) \vt ObjectCopyFromString({\it string\/}, {\vt layer})}\\
This function will create an object copy from the CIF-like string, as
generated by the {\vt ObjectString} function.  Boxes, polygons, and
wires are supported, labels and subcells will not return a handle. 
The object will be associated with the layer named in the second
argument.  The layer will be created if it does not exist.  Only
physical layers are accepted.

On success, a handle to an object list containing the new copy is
returned.  On error, a scalar zero is returned.  The function will
fail if the string is null or a new layer cannot be created.

%------------------------------------
% 100508
\index{FilterObjects function}
\item{(object\_handle) \vt FilterObjects({\it object\_list},
{\it template\_list}, {\it all\/}, {\it touchok}, {\it remove\/})}\\
This function creates a handle to a list of objects that is a subset
of the objects contained in the {\it object\_list}.  The objects in
the new list are those that touch or overlap objects in the {\it
template\_list}, which is also a handle to a list of objects.

If {\it all} is nonzero, all of the objects in the {\it
template\_list} will be used for comparison, otherwise only the head
object in the template list will be used.

If {\it touchok} is nonzero, objects in the object list that touch but
do not overlap the template object(s) will be added to the new list,
otherwise not.

If {\it remove} is nonzero, objects that are added to the new list are
removed from the {\it object\_list}, otherwise the {\it object\_list}
is not touched.  The function will fail if the handle arguments are of
the wrong type.  The return value is a new handle to a list of
objects.

%------------------------------------
% 030204
\index{FilterObjectsA function}
\item{(object\_handle) \vt FilterObjectsA({\it object\_list},
{\it array}, {\it array\_size}, {\it touchok}, {\it remove\/})}\\
This function creates a handle to a list of objects, which consist of
the objects in the {\it object\_list} that touch or overlap the
polygon defined in the {\it array}.  The {\it array\_size} is the
number of x-y coordinates represented in the array.  In the array, the
values are x-y coordinate pairs representing the polygon vertices, and
the first pair must match the last pair (i.e., the figure must be
closed).  The values are specified in microns.  If {\it touchok} is
nonzero, objects that touch but do not overlap the polygon will be
added to the list, otherwise not.  If {\it remove} is nonzero, objects
that are added to the new list are removed from the {\it
object\_list}, otherwise the {\it object\_list} is not touched.

The function will fail if {\it array\_size} is less than 4, or the
size of the array is less than twice {\it array\_size}, or if the
handle argument is not a handle to a list of objects.  The return
value is a new handle to a list of objects.

%------------------------------------
% 100408
\index{CheckObjectsConnected function}
\item{(int) \vt CheckObjectsConnected({\it object\_handle\/})}\\
This function returns 1 unless the list contains objects on the layer
of the first object in the list that are mutually disjoint, meaning
that there exist two objects and one can not draw a curve from the
interior of one to the other without crossing empty area.  If disjoint
objects are found, 0 is returned.

%------------------------------------
% 030204
\index{CheckForHoles function}
\item{(int) \vt CheckForHoles({\it object\_handle\/}, {\it all\/})}\\
This function returns 1 if the object, or collection of objects, has
``holes'', i.e., uncovered areas completely surrounded by geometry. 
The first argument is a handle to a list of objects.  If the second
argument is nonzero, the geometry represented by all objects in the
list is checked.  If zero, only the first object (which might be a
complex polygon containing holes) is checked.  If no holes are found,
0 is returned.

When {\it all} is true, only objects on the same layer as the first
object in the list are considered.

%------------------------------------
% 100508
\index{BloatObjects function}
\item{(object\_handle) \vt BloatObjects({\it object\_handle}, {\it all\/},
 {\it dimen\/}, {\it lname\/}, {\it mode\/})}\\
This function returns a handle to a list of object copies which are
bloated versions of the objects referenced by the handle argument,
similar to the {\cb !bloat} command.  The passed handle and objects
are not affected.  Edges will be pushed outward or pulled inward by
{\it dimen} (positive values push outward).  The {\it dimen} is given
in microns.

The {\it all} argument is a boolean that if nonzero indicates that all
objects in the list referenced by the handle may be processed.  If
zero, only the first object in the list will be processed.

The {\it lname} argument is a layer name.  If this argument is zero,
or a null or empty string, all objects on the returned list are
associated with the layer of the first object in the passed list, and
only objects on this layer in the passed list are processed. 
Otherwise, the layer will be created if it does not exist, and all new
objects will be associated with this layer, and all objects in the
passed list will be processed.

The {\it mode} argument is an integer that specifies the algorithm to
use for bloating.  Giving zero specifies the default algorithm.  See
the description of the {\cb !bloat} command (\ref{bloatcmd}) for
documentation of the algorithms available. 

The {\vt DeleteObjects} function can be called to delete the old
objects.  The {\vt CopyObjects} function can be called on the returned
objects to add them to the database.  This function returns a handle
to the new list upon success, or 0 if there are no objects.  The
function will fail if the first argument is not a handle to a list of
objects or copies, or the {\it lname} argument is non-null and not a
valid layer name.

This function uses the {\et JoinMax}{\it XXX} variables in processing. 
There is no effect on objects in the list whose handle is passed as
the first argument, or on the handle.

%------------------------------------
% 100508
\index{EdgeObjects function}
\item{(object\_handle) \vt EdgeObjects({\it object\_handle}, {\it all\/},
 {\it dimen\/}, {\it lname\/}, {\it mode\/})}\\
This function creates new polygon copies that cover the edges of the
figures in the passed handle.  The {\it dimen} is half the effective
path width of the generated wire-like shapes that cover the edges.

If the boolean argument {\it all} is nonzero, all of the objects in
the passed list may be processed, otherwise only the object at the
head of the list will be processed.

The {\it lname} argument is a layer name.  If this argument is zero,
or a null or empty string, all objects on the returned list are
associated with the layer of the first object in the passed list, and
only objects on this layer in the passed list are processed. 
Otherwise, the layer will be created if it does not exist, and all new
objects will be associated with this layer, and all objects in the
passed list will be processed.

The {\it mode} is an integer which specifies the algorithm to use.  The
algorithms are described with the {\vt EdgesZ} function.

The {\vt DeleteObjects} function can be called to delete the old
objects.  The {\vt CopyObjects} function can be called on the returned
objects to add them to the database.  This function returns a handle
to the new list upon success, or 0 if there are no objects.  The
function will fail if the first argument is not a handle to a list of
objects or copies, or the {\it lname} argument is non-null and not a
valid layer name.

%------------------------------------
% 100508
\index{ManhattanizeObjects function}
\item{(object\_handle) \vt ManhattanizeObjects({\it object\_handle},
 {\it all\/}, {\it dimen\/}, {\it lname\/}, {\it mode\/})}\\
This function will convert the objects pointed to by the handle
argument into a list of copies, which is referenced by the returned
handle.  The supplied objects and handle are not affected.  Each new
object is a Manhattan approximation of the original object.  The {\it
dimen} argument is the minimum height or width in microns of
rectangles created to approximate the non-Manhattan parts.

The {\it all} argument is a boolean that if nonzero indicates that all
objects in the list referenced by the handle may be processed.  If
zero, only the first object in the list will be processed.

The {\it lname} argument is a layer name, or zero.  If a layer name is
given, the new objects will be associated with that layer, which will
be created if it does not exist.  If 0 or an empty string is passed,
the new objects will be associated with the layer of the original
object.

The {\it mode} argument is a boolean value which selects one of two
Manhattanizing algorithms to employ.  These algorithms are described
with the {\cb !manh} command.

The function will fail if the first argument is not a handle to a list
of objects or copies, or the {\it lname} argument is non-null and not
a valid layer name, or the {\it dimen} argument is smaller than 0.01. 
On success, a handle to the list of copies is returned.  Each object
in the returned list is a box or Manhattan polygon which approximates
one of the original objects.  Of course, if the original objects were
all Manhattan, the shapes will be unchanged, though the coordinates
will be moved to a {\it dimen} grid if the gridding mode ({\it mode}
nonzero) is given.

The {\vt DeleteObjects} function can be called to delete the old
objects.  The {\vt CopyObjects} function can be called on the returned
objects to add them to the database.

This function uses the {\et JoinMax}{\it XXX} variables in processing. 
There is no effect on objects in the list whose handle is passed as
the first argument, or on the handle.

%------------------------------------
% 030204
\index{GroupObjects function}
\item{(int) \vt GroupObjects({\it object\_handle\/}, {\it array\/})}\\
This function acts on the first object in the list and all other
objects on the same layer found in the list.  The objects are copied,
then sorted into groups, so that each group forms a single figure,
i.e., no two members of the same group are disjoint.  The groups are
then joined into polygons, and a handle to each group is returned in
the array.  The array will be resized if necessary.  The returned
value is the number of groups, corresponding to the used entries in
the array.  The {\vt H} function should be used on the array elements
to convert the values to an object handle data type, similar to the
treatment of the array returned from the {\vt HandleArray} function. 
The {\vt CloseArray} function can be used to close the handles.  The
created objects are copies, so are not added to the database.

This function uses the {\et JoinMax}{\it XXX} variables in processing. 
There is no effect on objects in the list whose handle is passed as
the first argument, or on the handle.  The value 0 is returned on
error or if the list is empty.

%------------------------------------
% 040904
\index{JoinObjects function}
\item{(object\_handle) \vt JoinObjects({\it object\_handle}, {\it lname\/})}\\
This function will combine the objects in the list passed as the first
argument, if possible, into a new list of object copies, which is
returned.  The passed handle and objects are not affected.  All
objects in the returned list will be associated with the layer named
in the second argument.  This layer will be created if it does not
exist, and the output will consist of the joined outlines of all of
the objects in the passed list, from any layer.  If 0, or a null or
empty string is passed, the new objects will be associated with the
layer of the first object in the passed list, and only the outlines of
objects on this layer found in the passed list will contribute to the
result.

The {\vt DeleteObjects} function can be called to delete the old
objects.  The {\vt CopyObjects} function can be called on the returned
objects to add them to the database.  This function returns a handle
to the new list upon success, or 0 if there are no objects.  The
function will fail if the first argument is not a handle to a list of
objects or copies, or the {\it lname} argument is non-null and not a
vail layer name.

This function uses the {\et JoinMax}{\it XXX} variables in processing. 
There is no effect on objects in the list whose handle is passed as
the first argument, or on the handle.

%------------------------------------
% 100508
\index{SplitObjects function}
\item{(object\_handle) \vt SplitObjects({\it object\_handle}, {\it all\/},
 {\it lname}, {\it vert\/})}\\
This function will split the objects in the list passed as the first
argument into horizontal or vertical trapezoids (polygons or boxes)
and return a list of the new objects.  The new objects are ``object
copies'' and are not added to the database.

If the boolean argument {\it all} is nonzero, all of the objects in
the list referenced by the handle will be processed.  Otherwise, only
the first object will be processed.

The new objects are placed on the layer with the name given in {\it
lname}, which is created if it does not exist, independent of the
originating layer of the objects.  If a null string or 0 is passed for
{\it lname}, the target layer will be the layer of the first object
found in the object list.

The {\it vert} argument is an integer which if nonzero indicates a
vertical decomposition, otherwise a horizontal decomposition is
produced.

The handle and objects passed are untouched.  The {\vt DeleteObjects}
function can be called to delete the old objects.  The {\vt
CopyObjects} function can be called on the returned objects to add them
to the database.  This function returns a handle to the new list upon
success, or 0 if there are no objects.  The function will fail if the
first argument is not a handle to a list of objects or copies, or the
{\it lname} argument is non-null and not a valid layer name.

%------------------------------------
% 100508
\index{DeleteObjects function}
\item{(int) \vt DeleteObjects({\it object\_handle\/}, {\it all\/})}\\
Calling this function will delete referenced objects from the current
cell.  If the boolean argument {\it all} is nonzero, all objects in
the list will be deleted.  Otherwise, only the first object in the
list will be deleted.  Once deleted, the objects are no longer
referenced by the handle, which may become empty as a result.

This function will fail if the handle passed is not a handle to an
object list.  The number of objects deleted is returned.

%------------------------------------
% 100508
\index{SelectObjects function}
\item{(int) \vt SelectObjects({\it object\_handle\/}, {\it all\/})}\\
This function will select objects referenced by the handle.  If the
boolean argument {\it all} is nonzero, all objects in the list will be
selected.  Otherwise, only the first object in the list will be
selected.

It is not possible to select object copies, 0 is returned if the
passed handle represents copies.  Otherwise the return value is the
number of newly selected objects.

This function will fail if the handle passed is not a handle to an
object list.

%------------------------------------
% 100508
\index{DeselectObjects function}
\item{(int) \vt DeselectObjects({\it object\_handle\/}, {\it all\/})}\\
This function will deselect objects referenced by the handle.  If the
boolean argument {\it all} is nonzero, all objects in the list will be
deselected.  Otherwise, only the first object in the list will be
deselected.

It is not possible to select object copies, 0 is returned if the
passed handle represents copies.  Otherwise the return value is the
number of newly deselected objects.

This function will fail if the handle passed is not a handle to an
object list.

%------------------------------------
% 100508
\index{MoveObjects function}
\item{(int) \vt MoveObjects({\it object\_handle}, {\it all\/}, {\it refx},
 {\it refy}, {\it x}, {\it y})}\\
This function is similar to the {\vt Move} function, however it
operates on the object(s) referenced by the handle.  An object is
moved such that the coordinate {\it refx}, {\it refy} is translated to
{\it x}, {\it y}.  The current transform will be applied to the move. 
If {\it all} is nonzero, all objects in the list are moved, otherwise
only the object currently referenced is moved.  The function returns
the number of objects moved.  This function will fail if the handle
passed is not a handle to an object list.

If the handle references object copies, each copy is translated and
possibly transformed as described above.  The handle will subsequently
reference the modified object.

%------------------------------------
% 100508
\index{MoveObjectsToLayer function}
\item{(int) \vt MoveObjectsToLayer({\it object\_handle}, {\it all\/},
 {\it refx\/}, {\it refy\/}, {\it x\/}, {\it y\/}, {\it oldlayer\/},
 {\it newlayer\/})}\\
This is similar to the {\vt MoveObjects} function, but allows layer
change.  If {\it newlayer} is 0, null, or empty, {\it oldlayer} is
ignored and the function behaves identically to {\vt MoveObjects}. 
Otherwise the {\it newlayer} string must be a layer name.  If {\it
oldlayer} is 0, null, or empty, all moved objects are placed on {\it
newlayer\/}.  Otherwise, {\it oldlayer} must be a layer name, in which
case only objects on {\it oldlayer} will be placed on {\it
newlayer\/}, other objects will remain on the same layer.  Subcell
objects are moved as in {\vt MoveObjects}, i.e., the layer arguments
are ignored.

%------------------------------------
% 101415
\index{CopyObjects function}
\item{(int) \vt CopyObjects({\it object\_handle}, {\it all\/}, {\it refx},
 {\it refy}, {\it x}, {\it y}, {\it repcnt\/})}\\
This function is similar to the {\vt Copy} function, however it
operates on the object(s) referenced by the handle.  An object is
copied such that the coordinate {\it refx}, {\it refy} is translated
to {\it x}, {\it y}.

The {\it repcnt} is an integer replication count in the range
1--100000, which will be silently taken as one if out of range.  If
not one, multiple copies are made, at multiples of the translation
factors given.

The current transform will be applied to the copy.  If {\it all} is
nonzero, all of the objects in the list are copied, otherwise only the
object currently being referenced is copied.  The function returns the
number of objects copied.  This function will fail if the handle
passed is not a handle to an object list.

If the handle references object copies, the object copies that are
referenced remains untouched, however the new objects, translated and
possibly transformed as described above, are added to the database.
The {\it repcnt} argument is ignored in this case.

%------------------------------------
% 082009
\index{CopyObjectsToLayer function}
\item{(int) \vt CopyObjectsToLayer({\it object\_handle\/}, {\it all\/},
 {\it refx\/}, {\it refy\/}, {\it x\/}, {\it y\/}, {\it oldlayer\/},
 {\it newlayer\/}, {\it repcnt\/})}\\
This is similar to the {\vt CopyObjects} function, but allows layer
change.  If {\it newlayer} is 0, null, or empty, {\it oldlayer} is
ignored and the function behaves identically to {\vt CopyObjects}. 
Otherwise the {\it newlayer} string must be a layer name.  If {\it
oldlayer} is 0, null, or empty, all copied objects are placed on {\it
newlayer\/}.  Otherwise, {\it oldlayer} must be a layer name, in which
case only objects on {\it oldlayer} will be placed on {\it
newlayer\/}, other objects will remain on the same layer.  Subcell
objects are copied as in {\vt CopyObjects}, i.e., the layer arguments
are ignored.

%------------------------------------
% 092615
\index{CopyObjectsH function}
\item{(object\_handle) \vt CopyObjectsH({\it object\_handle\/}, {\it all\/},
 {\it refx\/}, {\it refy\/}, {\it x\/}, {\it y\/}, {\it oldlayer\/},
 {\it newlayer\/}, {\it todb\/})}\\
This function returns an object handle, containing copies of the
objects in the handle passed as the first argument.  If boolean {\it
all} is set, all passed objects will be copied, otherwise only the
first object in the list will be copied.  The next four arguments set
the copy translation, with {\it refx} and {\it refy} in the passed
object translated to {\it x}, {\it y} in the copy.  The current
transform is also applied to the copy.

The two layer name arguments behave as in {\vt CopyObjectToLayer}.  If
{\it newlayer} is 0, null, or empty, {\it oldlayer} is ignored and no
object layers will change.  Otherwise the {\it newlayer} string must
be a layer name.  If {\it oldlayer} is 0, null, or empty, all copied
objects are placed on {\it newlayer}.  Otherwise, {\it oldlayer} must
be a layer name, in which case only objects on {\it oldlayer} will be
placed on {\it newlayer}, other objects will remain on the same layer. 
Subcell objects are copied as in {\vt CopyObjects}, i.e., the layer
arguments are ignored.

The final argument is a boolean that when true, the copies are added
to the database, and the returned handle points to the database
objects.  If false, the returned handle contains ``object copies''
which do not appear in the database.  Note that when copies are added
to the database, unlike other copy functions merging is disabled, and
the replication feature is not available.

%------------------------------------
% 030204
\index{GetObjectType function}
\item{(string) \vt GetObjectType({\it object\_handle\/})}\\
This function returns a one-character string representing the type of
object referenced by the handle argument.  If the handle is invalid, a
null string is returned.  The types are:

\begin{tabular}{ll}\\
`{\vt b}' & boxes\\
`{\vt p}' & polygons\\
`{\vt w}' & wires\\
`{\vt l}' & labels\\
`{\vt c}' & subcells
\end{tabular}

This function will fail if the handle passed is not a handle to an
object list.

%------------------------------------
% 030204
\index{GetObjectID function}
\item{(int) \vt GetObjectID({\it object\_handle\/})}\\
This function returns a unique id number for the object.  The id is
actually the address of the object in the process memory, so it is
valid only for the current {\Xic} process.  If the referenced object
is a copy, the id returned is the address of the real object, not the
copy.  If no object is referenced by the handle, 0 is returned.  The
function fails if the handle is not an object list type.

%------------------------------------
% 072904
\index{GetObjectArea function}
\item{(int) \vt GetObjectArea({\it object\_handle\/})}\\
Return the area in square microns of the object pointed to by the
handle.  Zero is returned for a defunct handle or upon error.

%------------------------------------
% 072904
\index{GetObjectPerim function}
\item{(int) \vt GetObjectPerim({\it object\_handle\/})}\\
Return the perimeter in microns of the object pointed to by the
handle.  Zero is returned for a defunct handle or upon error.

%------------------------------------
% 062715
\index{GetObjectCentroid function}
\item{(int) \vt GetObjectCentroid({\it object\_handle\/}, {\it array\/})}\\
Return the centroid coordinates in microns of the object pointed to by
the handle.  The second argument is an array of size two or larger
that will contain the centroid coordinates upon successful return. 
The return value is zero for a defunct handle or upon error, one if
success.

%------------------------------------
% 030204
\index{GetObjectBB function}
\item{(int) \vt GetObjectBB({\it object\_handle\/}, {\it array\/})}\\
This function loads the left, bottom, right, and top coordinates of
the object's bounding box (in microns) into the {\it array} passed. 
This function will fail if the handle passed is not a handle to an
object list, or if the size of the array is less than 4.  The return
value is 1 if successful, 0 otherwise.

%------------------------------------
% 021912
\index{SetObjectBB function}
\item{(int) \vt SetObjectBB({\it object\_handle\/}, {\it array\/})}\\
This function will alter the shape of the object pointed to by the
handle such that it has the bounding box passed.  The {\it array}
contains the left, bottom, right, and top coordinates, in microns. 
This function will fail if the handle passed is not a handle to an
object list, or if the size of the array is less than 4.  The return
value is 1 if successful, 0 otherwise.  This function has no effect on
subcells, but other types of object will be rescaled to the new
bounding box.

%------------------------------------
% 120615
\index{GetObjectListBB function}
\item{(int) \vt GetObjectListBB({\it object\_handle\/}, {\it array\/})}\\
This is similar to {\vt GetObjectBB}, but computes the bounding box of
all objects in the list of objects referenced by the handle.  not just
the list head.  The function loads the left, bottom, right, and top
coordinates of the aggregate bounding box (in microns) into the array
passed.  This function will fail if the handle passed is not a handle
to an object list, or if the size of the array is less than 4.  The
return value is a count of the objects in the list.

%------------------------------------
% 011110
\index{GetObjectXY function}
\item{(int) \vt GetObjectXY({\it object\_handle\/}, {\it array\/})}\\
This function will retrieve the ``XY'' position from the object
pointed to by the handle into the array, which must have size 2 or
larger.  This is a coordinate, in microns, the interpretation of which
depends on the object type.  For boxes, that value is the lower-left
corner of the box.  For wires and polygons, the value is the first
vertex in the coordinate list.  For labels, the value is the text
anchor position.  For subcells, the value is the instantiation point,
the same as the translation in the instantiation transform.

On success, the return value is 1, with the array values set. 
Otherwise, 0 is returned.

%------------------------------------
% 011110
\index{SetObjectXY function}
\item{(int) \vt SetObjectXY({\it object\_handle\/}, {\it x\/}, {\it y\/})}\\
This function will set the ``XY'' coordinate of the object pointed to
by the handle, as if setting the {\et XprpXY} pseudo-property number
7215 on the object.  This has the effect of moving the object to a new
location.  The interpretation of the coordinate, which is supplied in
microns, depends on the type of object.  For boxes, the lower-left
corner will assume the new value.  For polygons and wires, the object
will be moved so that the first vertex in the coordinate list will
assume the new value.  For labels, the text will be anchored at the
new value, and for subcells, the new value will set the translation
part of the instantiation transform.

A value of 1 is returned if the operation succeeds, and the object
will be moved.  On failure, 0 is returned.

%------------------------------------
% 030204
\index{GetObjectLayer function}
\item{(string) \vt GetObjectLayer({\it object\_handle\/})}\\
This function returns the name of the layer on which the object
referenced by the handle is defined.  For subcells, this layer is
named ``{\vt \$\$}'', but objects will return a layer from the layer
table.  This function will fail if the handle passed is not a handle
to an object list.  A stale handle will return a null string.

%------------------------------------
% 021912
\index{SetObjectLayer function}
\item{(int) \vt SetObjectLayer({\it object\_handle\/}, {\it layername\/})}\\
This function will move the object to the layer named in the string
{\it layername}.  This will have no effect on subcells.  A value 1 is
returned if successful, 0 otherwise.  This function will fail if the
handle passed is not a handle to an object list.

%------------------------------------
% 070516
\index{GetObjectFlags function}
\item{(int) \vt GetObjectFlags({\it object\_handle\/})}\\
This function returns internal flag data from the object referenced by
the handle.  This function will fail if the handle passed is not a
handle to an object list.  A stale handle will return 0.

The following flags are defined:\\
\begin{tabular}{|l|l|p{4in}|} \hline
\bf Name         &   \bf Bit  & \bf Description\\ \hline\hline
\vt MergeDeleted & \vt 0x1    & Object has been deleted due to merge.\\ \hline
\vt MergeCreated & \vt 0x2    & Object has been created due to merge.\\ \hline
\vt NoDRC        & \vt 0x4    & Skip DRC tests on this object.\\ \hline
\vt Expand       & \vt 0x8    & Five flags are used to keep track of cell
 expansion in main plus four sub-windows, in cell instances only.\\ \hline
\vt Mark1        & \vt 0x100  & General purpose application flag.\\ \hline
\vt Mark2        & \vt 0x200  & General purpose application flag.\\ \hline
\vt MarkExtG     & \vt 0x400  & Extraction system, in grouping phonycell.\\
 \hline
\vt MarkExtE     & \vt 0x800  & Extraction system, in extraction phonycell.\\
 \hline
\vt InQueue      & \vt 0x1000 & Object is in selection queue.\\ \hline
\vt NoMerge      & \vt 0x4000 & Object will not be merged.\\ \hline
\vt IsCopy       & \vt 0x8000 & Object is a copy, not in database.\\ \hline
\end{tabular}

The bitwise logic functions such as {\vt AndBits} can be used to check
the state of the flags.  Of these, only {\vt NoDRC}, {\vt Mark1}, and
{\vt Mark2} can be arbitrarily set by the user, using functions
described below.

%------------------------------------
% 070516
\index{SetObjectNoDrcFlag function}
\item{(int) \vt SetObjectNoDrcFlag({\it object\_handle\/}, {\it value\/})}\\
This will set the state of the {\vt NoDRC} flag of the object
referenced by the handle.  The second argument is a boolean
representing the flag state.  This can be called on any object, but is
only significant for boxes, polygons, and wires in the database. 
Objects with this flag set are ignored during design rule checking.

The return value is 0 or 1 representing the previous state of the
flag, or -1 on error.

%------------------------------------
% 070516
\index{SetObjectMark1Flag function}
\item{(int) \vt SetObjectMark1Flag({\it object\_handle\/}, {\it value\/})}\\
This will set the state of the {\vt Mark1} flag of the object
referenced by the handle.  The second argument is a boolean
representing the flag state.  This can be called on any object.  The
flag is unused by {\Xic}, but can be set and tested by the user for
any purpose.  The flag persists as long as the object is in memory.

The return value is 0 or 1 representing the previous state of the
flag, or -1 on error.

%------------------------------------
% 070516
\index{SetObjectMark2Flag function}
\item{(int) \vt SetObjectMark2Flag({\it object\_handle\/}, {\it value\/})}\\

This will set the state of the {\vt Mark2} flag of the object
referenced by the handle.  The second argument is a boolean
representing the flag state.  This can be called on any object.  The
flag is unused by {\Xic}, but can be set and tested by the user for
any purpose.  The flag persists as long as the object is in memory.

The return value is 0 or 1 representing the previous state of the
flag, or -1 on error.

%------------------------------------
% 030204
\index{GetObjectState function}
\item{(int) \vt GetObjectState({\it object\_handle\/})}\\
This function returns a status value for the object referenced by the
handle.  The status values are:

\begin{tabular}{ll}\\
0 & normal state\\
1 & object is selected\\
2 & object is deleted\\
3 & object is incomplete\\
4 & object is internal only\\
\end{tabular}

Only values 0 and 1 are likely to be seen.  This function will fail if
the handle passed is not a handle to an object list.  A stale handle
will return 0.

%------------------------------------
% 033009
\index{GetObjectGroup function}
\item{(int) \vt GetObjectGroup({\it object\_handle\/})}\\
This function returns the conductor group number of the object, which
is a non-negative integer or possibly -1 in certain cases, and is
assigned internally by the extraction system.  This is used by the
extraction system to establish connectivity nets of boxes, polygons,
and wires, and for subcell indexing.  If extraction is unavailable or
not being used, then an arbitrary integer can be applied for other
uses with the {\vt SetObjectGroup} function.

This function will fail if the handle passed is not a handle to an
object list.  If no group has been assigned, or the handle is stale,
or the object is part of the ``ground'' group, 0 is returned. 
Otherwise, any assigned number will be returned. 

%------------------------------------
% 033009
\index{SetObjectGroup function}
\item{(int) \vt SetObjectGroup({\it object\_handle\/}, {\it group\_num\/})}\\
This function will assign the group number to the object.  All objects
and instances may receive a group number, which is an arbitrary
integer.  The group number is usually assigned and used by the
extraction system, and should {\bf not} be assigned with this function
if extraction is being used.  However, if extraction is unavailable or
not being used, then this function allows an arbitrary integer to be
associated with an object, which might be useful.  Beware that this
number is zeroed if the object is modified, or in copies.

The {\vt GetObjectGroup} function can be used to obtain the group
number of an object or cell instance.

This function will fail if the handle passed is not a handle to an
object list.  If the group number is successfully assigned, 1 is
returned, 0 is returned otherwise.

%------------------------------------
% 030204
\index{GetObjectCoords function}
\item{(int) \vt GetObjectCoords({\it object\_handle\/}, {\it array\/})}\\
This function will obtain the vertex list for polygons and wires, or
the bounding box vertices of other objects, starting from the lower
left corner and working clockwise.  If an array is passed, the vertex
coordinates are copied into the array, and the vertex count is
returned.  The array will contain the x, y values of the vertices, in
microns, if successful.  The coordinates are copied only if the array
is large enough, or can be resized.  If the array is a pointer to a
too small array, or the array is too small but has other variables
pointing to it, resizing is impossible and the copying is skipped.  In
this case, the returned value is the negative vertex count.  If 0 is
passed instead of the array, the (positive) vertex count is returned. 
Zero is returned if there is an error.  This function will fail if the
handle passed is not a handle to an object list.

%------------------------------------
% 030204
\index{SetObjectCoords function}
\item{(int) \vt SetObjectCoords({\it object\_handle\/}, {\it array\/},
 {\it size\/})}\\
This function will modify a physical object to have the vertex list
passed in the array.  The size is the number of vertices (one half the
size of the array used).  For all but wires, the first and last
vertices must coincide, thus the minimum number of vertices is four. 
The array consists of x, y coordinates of the vertices.  If the
operation is successful, 1 is returned, otherwise 0 is returned.  The
coordinates in the array are in microns.  If the coordinates represent
a rectangle, the new object will be a box, if it was previously a
polygon or box.  A box may be converted to a polygon if the
coordinates are not those of a rectangle.  For labels, the coordinates
must represent a rectangle, and the label will be stretched to the new
box.  The function has no effect on instances.  This function will
fail if the handle passed is not a handle to an object list.

%------------------------------------
% 030204
\index{GetObjectMagn function}
\item{(real) \vt GetObjectMagn({\it object\_handle\/})}\\
This function returns the magnification part of the transform if the
object referenced by the handle is a subcell, or 1.0 for other
objects.  Only physical subcells can have non-unit magnification. 
This function will fail if the handle passed is not a handle to an
object list.  A stale handle returns 0.

%------------------------------------
% 021912
\index{SetObjectMagn function}
\item{(int) \vt SetObjectMagn({\it object\_handle\/}, {\it magn\/})}\\
This will set the magnification of the subcell referenced by the
handle, or scale other physical objects.  The real number {\it magn}
must be between .001 and 1000 inclusive.  If the operation is
successful, 1 is returned, otherwise 0 is returned.  This function
will fail if the handle passed is not a handle to an object list.

%------------------------------------
% 030204
\index{GetWireWidth function}
\item{(real) \vt GetWireWidth({\it object\_handle\/})}\\
This function will return the wire width if the object referenced by
the handle is a wire, otherwise 0 is returned.  This function will
fail if the handle passed is not a handle to an object list.

%------------------------------------
% 030204
\index{SetWireWidth function}
\item{(int) \vt SetWireWidth({\it object\_handle\/}, {\it width\/})}\\
This function will set the width of the wire referenced by the handle
to the given {\it width} (in microns).  If the operation is
successful, 1 is returned, otherwise 0 is returned.  This function
will fail if the handle passed is not a handle to an object list.

%------------------------------------
% 030204
\index{GetWireStyle function}
\item{(int) \vt GetWireStyle({\it object\_handle\/})}\\
This function returns the end style code of the wire pointed to by the
handle, or -1 if the object is not a wire.  The codes are

\begin{tabular}{ll}\\
0 & flush ends\\
1 & projecting rounded ends\\
2 & projecting square ends
\end{tabular}

This function will fail if the handle passed is not a handle to an
object list.

%------------------------------------
% 030204
\index{SetWireStyle function}
\item{(int) \vt SetWireStyle({\it object\_handle\/}, {\it code\/})}\\
This function will change the end style of the wire referenced by the
handle to the given {\it code}.  The code is an integer which can take
the following values

\begin{tabular}{ll}\\
0 & flush ends\\
1 & projecting rounded ends\\
2 & projecting square ends
\end{tabular}

If the operation succeeds, 1 is returned, otherwise 0.  This can apply
to physical wires only.  This function will fail if the handle passed
is not a handle to an object list.

%------------------------------------
% 030204
\index{SetWireToPoly function}
\item{(int) \vt SetWireToPoly({\it object\_handle\/})}\\
This function converts the wire object referenced by the handle to a
polygon object.  If the conversion is done, the handle will reference
the new polygon object.  The conversion will be done only if the wire
has nonzero width.  If the wire is not a copy, the wire object in the
database will be converted to a polygon.  Otherwise, only the copy
will be changed.  Upon success, the function returns 1, otherwise 0 is
returned.  The function fails if the argument is not a handle to an
object list.

%------------------------------------
% 030204
\index{GetWirePoly function}
\item{(int) \vt GetWirePoly({\it object\_handle}, {\it array})}\\
This function returns the polygon used for rendering a wire.  This
will be different from the wire vertices, if the wire has nonzero
width.  The first argument is a handle to an object list which
references a wire object.  The second argument is an array which will
hold the polygon coordinates.  This argument can be 0, if the polygon
points are not needed.  The array will be resized if necessary (and
possible).  The return value is the number of vertices required or
used in the polygon.  If an error occurs, the return value is 0.  If
an array is passed which can't be resized because it is referenced by
a pointer, the return value is a negative value, the negative vertex
count required.  The function will fail if the first argument is not a
handle to an object list, or the second argument is not an array or
zero.  The coordinates returned in the array are in microns, relative
to the origin of the current cell.

%------------------------------------
% 030204
\index{GetLabelText function}
\item{(string) \vt GetLabelText({\it object\_handle\/})}\\
This function returns the label text if the object referenced by the
handle is a label.  Otherwise, a null string is returned.  The actual
text is always returned, and not the symbolic text that is shown
on-screen for script and long text labels.  This function will fail if
the handle passed is not a handle to an object list.

%------------------------------------
% 021912
\index{SetLabelText function}
\item{(int) \vt SetLabelText({\it object\_handle\/}, {\it text\/})}\\
This function will set the label text of a label referenced by the
handle.  Setting the text in this manner will cause a long-text label
to revert to a normal label.  If the operation succeeds, the return
value is 1, otherwise 0 is returned.  This function will fail if the
handle passed is not a handle to an object list.

%------------------------------------
% 022713
\index{GetLabelFlags function}
\item{(int) \vt GetLabelFlags({\it object\_handle\/})}\\
This function returns the flags word used to specify a number of label
presentation attributes, as described in \ref{labelflags}.

This function will fail if the handle passed is not a handle to an
object list.

The function was named {\vt GetLabelXform} in releases prior to 3.3.1,
and is still recognized by that name, though this is deprecated and
undocumented.

%------------------------------------
% 022713
\index{SetLabelFlags function}
\item{(int) \vt SetLabelFlags({\it object\_handle\/}, {\it flags\/})}\\
This function will apply the given flags to the label referenced by
the handle.  The flags are the label flags used by {\Xic} and
described in \ref{labelflags}.  If the operation is successful, 1 is
returned, otherwise 0 is returned.  This function will fail if the
handle passed is not a handle to an object list.

The function was named {\vt SetLabelXform} in releases prior to 3.3.1,
and is still recognized by that name, though this is deprecated and
undocumented.

%------------------------------------
% 030204
\index{GetInstanceArray function}
\item{(int) \vt GetInstanceArray({\it object\_handle\/}, {\it array\/})}\\
This function fills in the {\it array}, which must have size of four
or larger, with the array parameters for the instance referenced by
the handle.  If the operation succeeds, 1 is returned, and the array
components have the following values, relative to the untransformed
coordinates:

\begin{tabular}{ll}
array[0] & number of cells along x\\
array[1] & number of cells along y\\
array[2] & center to center x spacing (in microns)\\
array[3] & center to center y spacing (in microns)\\
\end{tabular}

If the operation fails, 0 is returned.  This function will fail if the
handle passed is not a handle to an object list.

%------------------------------------
% 030204
\index{SetInstanceArray function}
\item{(int) \vt SetInstanceArray({\it object\_handle\/}, {\it array\/})}\\
This function will change the array parameters of the instance
referenced by the handle to the indicated values.  The {\it array}
values are in the format as returned from {\vt GetInstanceArray}. 
Only physical mode subcells can be changed by this function, arrays
are not supported in electrical mode.  If the operation succeeds, 1 is
returned, otherwise 0 is returned.  This function will fail if the
handle passed is not a handle to an object list.

%------------------------------------
% 030204
\index{GetInstanceXform function}
\item{(string) \vt GetInstanceXform({\it object\_handle\/})}\\
This function returns a string giving the CIF transformation code for
the instance referenced by the handle.  If the object is not an
instance, a null string is returned.  This function will fail if the
handle passed is not a handle to an object list.

%------------------------------------
% 030204
\index{GetInstanceXformA function}
\item{(string) \vt GetInstanceXformA({\it object\_handle\/}, {\it array\/})}\\
This function fills in the {\it array}, which must have size 4 or
larger, with the components of the transformation of the instance
referenced by the handle.  The values are:

\begin{tabular}{ll}
{\it array\/}{\vt [0]} & 1 if mirror-y, 0 if no mirror-y\\
{\it array\/}{\vt [1]} & angle in degrees\\
{\it array\/}{\vt [2]} & translation x\\
{\it array\/}{\vt [3]} & translation y\\
\end{tabular}

This is the same data as provided by the {\vt GetInstanceXform}
function, but in numerical rather than string form.  The transform
components are applied in the order as found in the array, i.e.,
mirror first, then rotate, then translate.  The function returns 1 if
successful, 0 otherwise.  It will fail if the handle passed is not a
handle to an object list.

%------------------------------------
% 030204
\index{SetInstanceXform function}
\item{(int) \vt SetInstanceXform({\it object\_handle\/}, {\it transform\/})}\\
This function applies the given {\it transform} to the instance
referenced by the handle.  The {\it transform} is in the form of a CIF
transformation string, as returned by {\vt GetInstanceXform}.  Note
that coordinates in the transform string are in internal units (1 unit
= .001 micron).  Only physical-mode subcells can be modified by this
function.  If the operation succeeds, 1 is returned, otherwise 0 is
returned.  This function will fail if the handle passed is not a
handle to an object list.

%------------------------------------
% 030204
\index{SetInstanceXformA function}
\item{(int) \vt SetInstanceXformA({\it object\_handle\/}, {\it array\/})}\\
This function applies the given transform parameters in the {\it
array} to the instance referenced by the handle.  The parameters are:

\begin{tabular}{ll}
{\it array\/}{\vt [0]} & 1 if mirror-y, 0 if no mirror-y\\
{\it array\/}{\vt [1]} & angle in degrees\\
{\it array\/}{\vt [2]} & translation x\\
{\it array\/}{\vt [3]} & translation y\\
\end{tabular}

Only physical-mode subcells can be modified by this function.  If the
operation succeeds, 1 is returned, otherwise 0 is returned.  The
transform components are applied in the order as found in the array,
i.e., mirror first, then rotate, then translate.  The function returns
1 if successful, 0 otherwise.  It will fail if the handle passed is
not a handle to an object list.

%------------------------------------
% 062116
\index{GetInstanceMaster function}
\item{(string) \vt GetInstanceMaster({\it object\_handle\/})}\\
Note: prior to 4.2.12, this function was called {\vt GetInstanceName}.

This function returns the master cell name of the instance referenced
by the handle.  If the object is not an instance, a null string is
returned.  This function will fail if the handle passed is not a
handle to an object list.  The cell instance can be electrical or
physical, and operation is identical in electrical and physical mode.

%------------------------------------
% 062116
\index{SetInstanceMaster function}
\item{(int) \vt SetInstanceMaster({\it object\_handle\/}, {\it newname\/})}\\
Note: prior to 4.2.12, this function was called {\vt SetInstanceName}.

This currently works with physical cell data only.

This function will replace the instance referenced by the handle with
an instance of the cell given as {\it newname}, in the parent cell of
the referenced instance.  The current transform is added to the
transform of the new instance.  This function will fail if the handle
passed is not a handle to an object list.  If successful, 1 is
returned, otherwise 0 is returned.

%------------------------------------
% 060616
\index{GetInstanceName function}
\item{(string) \vt GetInstanceName({\it object\_handle\/})}\\
Note:  prior to 4.2.12, this function returned the name of the
instance master cell.  The \newline{\vt GetInstanceMaster} function now
performs that operation.

This function returns a name for the electrical cell instance
referenced by the handle.  This is the name of the object, as would
appear in a generated SPICE file.

Currently, for physical instances, and for unnamed electrical
instances, a null string is returned.

Internally, names are generated in the following way.  Each device has
a prefix, as specified in the technology file.  The prefix for
subcircuits is ``X'', which is defined internally.  The prefixes
follow (or should follow) SPICE conventions.  The database of instance
placements is scanned in order of the placement location (upper-left
corner of the instance bounding box) top to bottom, then left to
right.  Each instance encountered is given an index number as a count
of the same prefix previously encountered in the scan.  The prefix
followed by the index forms the instance name.  This will identify
each instance uniquely, and the sequencing is predictable from spatial
location in the schematic.  For example.  {\vt X1} will be above or
to the left of {\vt X2}.

Rather than the internal name.  this function will return an assigned
name, if one has been given using {\vt SetInstanceName} or by setting
the name property,

The index number can be obtained as an integer with {\vt
GetInstanceIdNum}.  See also \newline{\vt GetInstanceAltName} for a
different subcircuit name style.

%------------------------------------
% 060616
\index{SetInstanceName function}
\item{(int) \vt SetInstanceName({\it object\_handle\/}, {\it newname\/})}\\
Note:  prior to 4.2.12, this function would re-master the instance,
the same as the present \newline{\vt SetInstanceMaster} function.

This will set a name for the electrical instance referenced by the
handle, which is in effect applying a name property to the instance. 
this makes sense for devices, subcircuits, and terminal devices.  The
new name will be used when generating netlist output, so should
conform to any requirements, for example SPICE conventions, being in
force.

If the string is null or 0, any applied name will be deleted,
equivalent to ``removing'' a name property.

The return value is 1 on success, 0 otherwise.

%------------------------------------
% 060616
\index{GetInstanceAltName function}
\item{(string) \vt GetInstanceAltName({\it object\_handle\/})}\\
This returns an alternative instance name for the electrical
subcircuit cell instance referenced by the handle.  The format is the
master cell name, followed by an underscore, followed by an integer. 
The integer is zero-based and sequential among instances of a given
master.  For example, instances of master ``{\vt foo}'' would have
names {\vt foo\_0}, {\vt foo\_1}, etc.  This is more useful an some
cases than the SPICE-style names {\vt X1}, {\vt X2}, ...  as returned
by {\vt GetInstanceName}.

For electrical device instances, this function returns the same
name As the {\vt GetInstanceName} function.

The {\vt GetInstanceAltIdNum} function returns the index number used,
as an integer.  This is different from the regular index, where every
instance, of whatever type, has a unique index.  Here, instances of
each master each have an index count starting from zero.  The order
that instances appear, however, is the same in both lists. 

Presently, this function returns a null string for physical instances.

%------------------------------------
% 060616
\index{GetInstanceType function}
\item{(string) \vt GetInstanceType({\it object\_handle\/})}\\
This function will return a string consisting of a single letter that
indicates the type of cell instance referenced by the handle.  The
function will fail if the handle is of the wrong type.  A null string
is returned it the object referenced is not a cell instance. 
Otherwise, the following strings may be returned.

These apply to electrical cell instances.
\begin{description}
\item{``b''}\\
The instance is ``bad''.  There has been an error.

\item{``n''}\\
The instance type is ``null'' meaning that it has no electrical
significance in a schematic.
  
\item{``g''}\\
The instance is a ground pin.  It has a ``hot spot'' that when placed
forces a ground contact at that location.

\item{``t''}\\
This is a terminal device, which has a name label and hot spot.  When
placed, it forces a contact to a net named in the label at the hot
spot location.

\item{``d''}\\
The instance represents a device, such as a resistor, capacitor, or
transistor.

\item{``m''}\\
This is a macro, which implements a subcircuit that is placed in the
schematic, as a ``black box''.  Unlike a subcircuit, a macro has no
sub-structure.

\item{``s''}\\
This is an instance of a circuit cell, i.e., a subcircuit.  Its master
contains instances of devices and other objects representing a
circuit.
\end{description}

For physical instances, at present there is only one return.
\begin{description}
\item{``p''}\\
   This is a physical instance.
\end{description}

%------------------------------------
% 060616
\index{GetInstanceIdNum function}
\item{(int) \vt GetInstanceIdNum({\it object\_handle\/})}\\
This function returns the integer index number used in electrical
device and subcircuit instance names.  See the {\vt GetInstanceName}
description for information about how the numbers are computed.  Each
subcircuit will have a unique number.  Devices are numbered according
to their prefix strings, each unique prefix has its own number
sequence.  These values are always non-negative.

For physical instances, an internal indexing number used by the
extraction system is returned.

This function will return -1 on error.

%------------------------------------
% 060616
\index{GetInstanceAltIdNum function}
\item{(int) \vt GetInstanceAltIdNum({\it object\_handle\/})}\\
This returns an alternative index for electrical subcircuits, as used
in the {\vt GetInstanceAltName} function.  Every subcircuit master
will have its instances numbered sequentially starting with 0.  The
ordering is set by the instance placement location in the schematic,
top to bottom then left to right, with the upper-left corner of the
bounding box being the reference location.

For other instances, the return value is the same as {\vt
GetInstanceIdNum}.
\end{description}

%------------------------------------------------------------------------------
\section{Geometry Editing Functions 2}
\subsection{Cells, PCells, Vias, and Instance Placement}

\begin{description}
%------------------------------------
% 102414
\index{CheckPCellParam function}
\item{(int) \vt CheckPCellParam({\it library\/}, {\it cell\/},
{\it view\/}, {\it pname\/}, {\it value\/})}\\
The first three arguments specify a parameterized cell.
\ifoa
If {\it library} is not given as a scalar 0, it is the name of the
OpenAccess library containing the pcell super-master, whose name is
given in the {\it cell} argument.  The {\it view} argument can be
passed a scalar 0 to indicate that the OpenAccess view name is ``{\vt
layout}'', or the actual view name can be passed if different.  For
{\Xic} native pcells not stored in OpenAccess, the {\it library} and
{\it view} should both be 0 (zero).
\else
The {\it library} and {\it view} arguments passed to this function
should always be 0 (zero).
\fi

The {\it pname} is a string containing a parameter name for a
parameter of the specified pcell, and the {\it value} argument is
either a scalar or string value.  The function returns 1 if the value
is not forbidden by a constraint, 0 otherwise.

%------------------------------------
% 102414
\index{CheckPCellParams function}
\item{(int) \vt CheckPCellParams({\it library\/}, {\it cell\/},
{\it view\/}, {\it params\/})}\\
The first three arguments specify a parameterized cell.
\ifoa
If {\it library} is not given as a scalar 0, it is the name of the
OpenAccess library containing the pcell super-master, whose name is
given in the {\it cell} argument.  The {\it view} argument can be
passed a scalar 0 to indicate that the OpenAccess view name is ``{\vt
layout}'', or the actual view name can be passed if different.  For
{\Xic} native pcells not stored in OpenAccess, the {\it library} and
{\it view} should both be 0 (zero).
\else
The {\it library} and {\it view} arguments passed to this function
should always be 0 (zero).
\fi

The {\it params} argument is a string providing the parameter values
in the format of the {\et pc\_params} property as applied to
sub-masters and instances.  i.e., values are constants and constraints
are not included.  The function returns 1 if no parameter has a value
forbidden by a constraint, 0 otherwise.

%------------------------------------
% 062109
\index{CreateCell function}
\item{(int) \vt CreateCell({\it cellname\/}, [{\it orig\_x\/},
  {\it orig\_y\/}])}\\
This will create a new cell from the contents of the selection
queue, with the given name, which can not already be in use.  The new
cell is created in memory only, with the modified flag set so as to
generate a reminder to the user to save the cell to disk when exiting
{\Xic}.  This provides functionality similar to the {\cb Create Cell}
button in the {\cb Edit Menu}.

If the optional coordinate pair {\it orig\_x} and {\it orig\_y} are
given (in microns), then this point will be the new cell origin in
physical mode only.  Otherwise, the lower-left corner of the bounding
box of the objects will be the new cell origin.  In electrical mode,
the cell origin is selected to keep contacts on-grid, and the origin
arguments are ignored.

By default, this function will fail if a cell of the same name already
exists in the current symbol table.  However, if the {\et
CrCellOverwrite} variable is set, existing cells will be overwritten
with the new data, and the function will succeed.

%------------------------------------
% 100408
\index{CopyCell function}
\item{(int) \vt CopyCell({\it name\/}, {\it newname\/})}\\
This function will copy the cell in memory named {\it name} to {\it
newname}.  The function returns 1 if the operation was successful, 0
otherwise.  The {\it name} cell must exist in memory, and the {\it
newname} can not clash with an existing cell or library device.

%------------------------------------
% 030104
\index{RenameCell function}
\item{(int) \vt RenameCell({\it oldname\/}, {\it newname\/})}\\
This function will rename the cell in memory named {\it oldname} to
{\it newname\/}, and update all references.  The function returns 1 if
the operation was successful, 0 otherwise.  The {\it oldname} cell
must exist in memory, and the {\it newname} can not clash with an
existing cell or library device.

%------------------------------------
% 062109
\index{Delete Empties function}
\item{(int) \vt DeleteEmpties({\it recurse\/})}\\
This function will delete empty cells found in the hierarchy under the
current cell.  This operation can not be undone.  The argument is an
integer flag; if zero, one pass is done, and all empty cells are
deleted.  If the argument is nonzero, additional passes are done to
delete cells that are newly empty due to their subcells being deleted
on the previous pass.  The top-level cells is never deleted.  The
return value is the number of cells deleted.

%------------------------------------
% 030215
\index{Place function}
\item{(int) \vt Place({\it cellname\/}, {\it x\/}, {\it y\/} [, {\it refpt\/},
  {\it array\/}, {\it smash\/}, {\it usegui\/}, {\it tfstring\/}])}\\
This function places an instance of the named cell at {\it x\/}, {\it
y\/}.  The first argument is of string type and contains the name of
the cell to place.  The string can consist of two space-separated
words.  If so, the first word may be a CHD name, an archive file name,
or a library
\ifoa
name (including OpenAccess when available).
\else
name.
\fi

The interpretation is similar to the {\cb new} selection in the {\cb
Open} command in the {\cb File Menu}.  In the case of two words, the
second word is the name of the cell to extract from the source
specified as the first word.  If only one word is given, it can be an
archive file name in which case the top-level cell is understood, or a
CHD name in which case the default cell is understood, or it can be
the name of a cell available as a native cell from a library or the
search path, or already exist in memory.

The second two arguments define the placement location, in microns.

The remaining arguments are optional, meaning that they need not
be given, but all arguments to the left must be given.
       
The {\it refpt} argument is an integer code that specifies the
reference point which will correspond to {\it x\/}, {\it y} after
placement.  The values can be
\begin{quote}
\begin{tabular}{ll}
0 & the cell origin (the default)\\
1 & the lower left corner\\
2 & the upper left corner\\
3 & the upper right corner\\
4 & the lower right corner\\
\end{tabular}
\end{quote}
The corners are those of the untransformed array or cell.
       
In electrical mode, if the cell has terminals, this code is ignored,
and the location of the first terminal is the reference point.  If the
cell has no terminals, the corner reference points are snapped to the
nearest grid location.  This is to avoid producing off-grid terminal
locations.

The {\it array} argument, if given, can be a scalar, or the name of an
array containing four numbers.  This argument specifies the arraying
parameters for the instance placement, which apply in physical mode
only.  If a scalar 0 is passed, the placement will not be arrayed,
which is also the case if this argument does not appear and is always
true in electrical mode.  If the scalar is nonzero, then the placement
will use the current array parameters, as displayed in the {\cb Cell
Placement Control} pop-up, or set with the {\vt PlaceSetArrayParams}
function.  If the argument is the name of an array, the array contains
the arraying parameters.  These parameters are:

\begin{quote}
\begin{tabular}{ll}
{\it array\/}[0] & NX, integer number in the X direction.\\
{\it array\/}[1] & NY, integer number in the Y direction.\\
{\it array\/}[2] & DX, the real value spacing between cells in
  the X direction, in microns.\\
{\it array\/}[3] & DY, the real value spacing between cells in
  the Y direction, in microns.\\
\end{tabular}
\end{quote}

The NX and NY values will be clipped to the range of 1 through 32767. 
The DX and DY are edge to adjacent edge spacing, i.e., when zero the
elements will abut.  If DX or DY is given the negative cell width or
height, so that all elements appear at the same location, the
corresponding NX or NY is taken as 1.  Otherwise, there is no
restriction on DX or DY.

If the boolean value {\it smash} is given and nonzero (TRUE), the cell
will be flattened into the parent, rather than placed as an instance. 
The flatten-level is 1, so subcells of the cell (if any) become
subcells of the parent.  This argument is ignored if the cell being
placed is a parameterized cell (pcell).

The {\it usegui} argument applies only when placing a pcell.  If
nonzero (TRUE), the {\cb Parameters} panel will appear, and the
function will block until the user dismisses the panel.  The panel can
be used to set cell parameters before instantiation.  Initially, the
parameters will be shown with default values, or values that were last
given to {\vt PlaceSetPCellParams}.  If the {\it usegui} argument is
not given or zero (FALSE), the default parameter set as updated with
parameters given to {\vt PlaceSetPCellParams} will be used to
instantiate the cell immediately.

The final argument can be a null string or scalar 0 which is
equivalent, an empty string, or a transform description in the format
returned by {\vt GetTransformString}.  If null or not given, the
arguemnt is ignored.  In this case, the cell will be transformed
before placement according to the current transform.  Otherwise, the
given transformation will be used when placing the instance.  An empty
string is taken as the identity transform.  If the {\vt UseTransform}
mode is in effect, the current transform will be added to the string
transform, giving an overall transfromation that will match geometry
placement in this mode.

On success, the function returns 1, 0 otherwise.

%------------------------------------
% 030215
\index{PlaceH function}
\item{(object\_handle) \vt PlaceH({\it cellname\/}, {\it x\/}, {\it y\/}
  [, {\it refpt\/}, {\it array\/}, {\it smash\/}, {\it usegui\/},
  {\it tfstring\/}])}\\
This is similar to the {\vt Place} function, however it returns a
handle to the newly created instance.  However, if the {\it smash}
boolean is true or on error, a scalar 0 is returned.

%------------------------------------
% 030215
\index{PlaceSetArrayParams function}
\item{(int) \vt PlaceSetArrayParams({\it nx\/}, {\it ny\/}, {\it dx\/},
{\it dy\/})}\\
This function provides array parameters which may be used when
instantiating physical cells.  These parameters will appear in the
{\cb Cell Placement Control} panel.  The arguments are:
  
\begin{quote}
\begin{tabular}{ll}
{\it nx\/} & Integer number in the X direction.\\
{\it ny\/} & Integer number in the Y direction.\\
{\it dx\/} & The real value spacing between cells in
  the X direction, in microns.\\
{\it dy\/} & The real value spacing between cells in
  the Y direction, in microns.\\
\end{tabular}
\end{quote}
  
The {\it nx} and {\it ny} values will be clipped to the range of 1
through 32767.  The {\it dx} and {\it dy} are edge to adjacent edge
spacing, i.e., when zero the elements will abut.  If {\it dx} or {\it
dy} is given the negative cell width or height, so that all elements
appear at the same location, the corresponding {\it nx} or {\it ny} is
taken as 1.  Otherwise, there is no restriction on {\it dx} or {\it
dy}.

The function returns 1 and sets the array parameters in physical mode. 
In electrical mode, the function returns 0 and does nothing.

%------------------------------------
% 102414
\index{PlaceSetPCellParams function}
\item{(int) \vt PlaceSetPCellParams({\it library\/}, {\it cell\/},
{\it view\/}, {\it params\/})}\\
This sets the default parameterized cell (pcell) parameters used when
instantiating the pcell indicated by the {\it libname\/}/{\it
cell\/}/{\it view\/}.
\ifoa
If {\it library} is not given as a scalar 0, it is the name of the
OpenAccess library containing the pcell super-master, whose name is
given in the {\it cell} argument.  The {\it view} argument can be
passed a scalar 0 to indicate that the OpenAccess view name is ``{\vt
layout}'', or the actual view name can be passed if different.  For
{\Xic} native pcells not stored in OpenAccess, the library and view
should both be 0 (zero).
\else
The {\it library} and {\it view} arguments passed to this function
should always be 0 (zero).
\fi

The {\it params} argument is a string providing the parameter values
in the format of the {\et pc\_params} property as applied to
sub-masters and instances, i.e., values are constants and constraints
are {\bf not} included.  Not all parameters need be given, only those
with non-default values.

Be aware that there is no immediate constraint testing of the
parameter values given to this function, though bad values will cause
subsequent instantiation of the named pcell to fail.  The {\vt
CheckPCellParams} fuction can be used to validate the params list
before calling this function.  When giving parameters for non-native
pcells, it is recommended that the type specification prefixes be
used, though an attempt is made internally to recognize and adapt to
differing types.

The saved parameter set will be used for all instantiations of the
pcell, until changed with another call to {\vt PlaceSetPCellParams}. 
The placement is done with the {\vt Place} script function, as for
normal cells.

In graphical mode, the given parameter set will initialize the {\cb
Parameters} pop-up.

This function manages an internal table of cellname/parameter list
associations.  If 0 is given for all arguments, the table will be
cleared.  If the {\it params} argument is 0, the specified entry will
be removed from the table.  When the script terminates, parameter
lists set with this function will revert to the pre-script values. 
Entries that were cleared by passing null arguments are {\bf not}
reverted, and remain cleared.

The function returns 1 on success, 0 if an error occurred, with an
error message available from {\vt GetError}.

%------------------------------------
% 030215
\index{Replace function}
\item{(int) \vt Replace({\it cellname\/}, {\it add\_xform\/}, {\it array\/})}\\
This will replace all selected subcells with {\it cellname\/}.  The same
transformation applied to the previous instance is applied to the
replacing instance.  In addition, if {\it add\_xform\/} is nonzero, the
current transform will be added.  The function returns 1 if
successful, 0 if the new cell could not be opened.

The {\it array} argument can be a scalar, or the name of an array
containing four numbers.  This argument specifies the arraying
parameters for the instance placement, which apply in physical mode
only.  If a scalar 0 is passed, the placement will retain the same
arraying parameters as the previous instance.  If the scalar is
nonzero, then the placement will use the current array parameters, as
displayed in the {\cb Cell Placement Control} pop-up, or set with the
{\vt PlaceSetArrayParams} function.  If the argument is the name of an
array, the array contains the arraying parameters.  These parameters
are:
\begin{quote}
\begin{tabular}{ll}
{\it array\/}[0] & NX, integer number in the X direction.\\
{\it array\/}[1] & NY, integer number in the Y direction.\\
{\it array\/}[2] & DX, the real value spacing between cells in
  the X direction, in microns.\\
{\it array\/}[3] & DY, the real value spacing between cells in
  the Y direction, in microns.\\
\end{tabular}
\end{quote}

The NX and NY values will be clipped to the range of 1 through 32767. 
The DX and DY are edge to adjacent edge spacing, i.e., when zero the
elements will abut.  If DX or DY is given the negative cell width or
height, so that all elements appear at the same location, the
corresponding NX or NY is taken as 1.  Otherwise, there is no
restriction on DX or DY.

%------------------------------------
% 032115
\index{OpenViaSubMaster function}
\item{(int) \vt OpenViaSubMaster({\it vianame\/}, {\it defnstr\/})}\\
This function will create if necessary and return the name of a
standard via sub-master cell in memory.  The first argument is the
name of a standard via, as defined in the technology file or imported
from Virtuoso.  The second argument contains a string that specifies
the parameters that differ from the default values.  This can be null
or empty if no non-default values are used.  The format is the same as
described for the {\et stdvia} property, with the standard via name
token stripped (see \ref{stdviaprp}).

On success, a name is returned.  One can use this name with the {\vt
Place} function to instantiate the via.  Otherwise, a fatal error is
triggered.
\end{description}


\subsection{Clipping Functions}

\begin{description}
%------------------------------------
% 100508
\index{ClipAround function}
\item{(int) \vt ClipAround({\it object\_handle1}, {\it all1\/},
 {\it object\_handle2}, {\it all2\/})}\\
This function will clip out the pieces of objects in the second handle
list that intersect with objects in the first handle list.

If the boolean value {\it all1} is nonzero, all objects in the first
handle are used for clipping, otherwise only the first object is used. 
If the boolean value {\it all2} is nonzero, all objects in the second
handle list may be clipped, otherwise only the first object in the
list is a candidate for clipping.  Only boxes, polygons, and wires
that appear in the second handle list will be clipped.  The objects in
the first handle list can be of any type, and labels and subcells will
use the bounding box.  The objects in the second list must be database
objects, if they are are copies, no clipping is performed.  The
objects in the first list can be copies.

The newly created objects are added to the front of the second handle
list, and the original object is removed from the list.  The return
value is the number of objects created, or -1 if either handle is
empty or some other error occurred.  The function fails if either
handle does not reference an object list.

%------------------------------------
% 100508
\index{ClipAroundCopy function}
\item{(object\_handle) \vt ClipAroundCopy({\it object\_handle1},
 {\it all1\/}, {\it object\_handle2}, {\it all2\/}, {\it lname})}\\
This function is similar to {\vt ClipAround}, however no new objects
are created in the database, and neither of the lists passed as
arguments is altered.  Instead, a new object list handle is returned,
which references a list of ``copies'' of objects that are created by
the clipping.  The new objects are the pieces of the object or objects
referenced by the second handle that do not intersect the object or
objects referenced by the first handle.

If the boolean value {\it all1} is nonzero, all objects in the first
handle are used for clipping, otherwise only the first object is used. 
If the boolean value {\it all2} is nonzero, all objects in the second
handle list may be clipped, otherwise only the first object in the
list is a candidate for clipping.  Only boxes, polygons, and wires
that appear in the second handle list will be clipped.  The objects in
the first handle list can be of any type, and labels and subcells will
use the bounding box.  The objects in the second list can be database
objects or copies.

If {\it lname} is a non-empty string, it is taken as the name for a
layer on which all of the returned objects will be placed.  The layer
will be created if it does not exist.  If zero or an empty or null
string is passed, the object copies will retain the layer of the
original object from the second handle list.

The returned list can be used by most functions that expect a list of
objects, however they are not copies of ``real'' objects.  If no new
object copy would be created by clipping, the function returns 0.  The
function will fail if either handle is not an object-list handle.

%------------------------------------
% 100508
\index{ClipTo function}
\item{(int) \vt ClipTo({\it object\_handle1}, {\it all1\/},
  {\it object\_handle2}, {\it all2\/})}\\
This function will clip objects referenced by the second handle to the
boundaries of objects referenced by the first handle.

If the boolean value {\it all1} is nonzero, all objects in the first
handle are used for clipping, otherwise only the first object is used. 
If the boolean value {\it all2} is nonzero, all objects in the second
handle list may be clipped, otherwise only the first object in the
list is a candidate for clipping.  Only boxes, polygons, and wires
that appear in the second handle list will be clipped.  The objects in
the first handle list can be of any type, and labels and subcells will
use the bounding box.  The objects in the second list must be database
objects, if they are are copies, no clipping is performed.  The
objects in the first list can be copies.

The newly created objects are added to the front of the second handle
list, and the original object is removed from the list.  The return
value is the number of objects created, or -1 if either handle is
empty or some other error occurred.  The function fails if either
handle does not reference an object list.

%------------------------------------
% 100508
\index{ClipToCopy function}
\item{(object\_handle) \vt ClipToCopy({\it object\_handle1}, {\it all1\/},
  {\it object\_handle2}, {\it all2\/}, {\it lname})}\\
This function is similar to {\vt ClipTo}, however no new objects are
created in the database, and neither of the lists passed as arguments
is altered.  Instead, a new object list handle is returned, which
references a list of ``copies'' of objects that are created by the
clipping.  The new objects are the pieces of the object or objects
referenced by the second handle that intersect the object or objects
referenced by the first handle.

If the boolean value {\it all1} is nonzero, all objects in the first
handle are used for clipping, otherwise only the first object is used. 
If the boolean value {\it all2} is nonzero, all objects in the second
handle list may be clipped, otherwise only the first object in the
list is a candidate for clipping.  Only boxes, polygons, and wires
that appear in the second handle list will be clipped.  The objects in
the first handle list can be of any type, and labels and subcells will
use the bounding box.  The objects in the second list can be database
objects or copies.

If {\it lname} is a non-empty string, it is taken as the name for a
layer on which all of the returned objects will be placed.  The layer
will be created if it does not exist.  If zero or an empty or null
string is passed, the object copies will retain the layer of the
original object from the second handle list.

The returned list can be used by most functions that expect a list of
objects, however they are not copies of ``real'' objects.  If no new
object copy would be created by clipping, the function returns 0.  The
function will fail if either handle is not an object-list handle.

%------------------------------------
% 030204
\index{ClipObjects function}
\item{(int) \vt ClipObjects({\it object\_handle\/}, {\it merge\/})}\\
This function will clip boxes, polygons, and wires in the list on the
same layer as the first such object in the list so that none of these
objects overlap.  Newly created objects are added to the front of the
handle list, and deleted objects are removed from the list.  Objects
in the list that are not on the same layer as the first box, polygon,
or wire or are not boxes, polygons or wires are ignored.  If the merge
argument is nonzero, adjacent new objects will be merged, otherwise
the pieces will remain separate objects.  If successful, the number of
newly created objects is returned, otherwise -1 is returned.  The
function will fail if the handle does not reference an object list.

%------------------------------------
% 100508
\index{ClipIntersectCopy function}
\item{(object\_handle) \vt ClipIntersectCopy({\it object\_handle1},
 {\it all1\/}, {\it object\_handle2}, {\it all2\/}, {\it lname})}\\
This function returns a list of object copies which represent the
exclusive-or of box, polygon, and wire objects in the two object lists
passed.  The lists are not altered in any way, and the new objects,
being ``copies'', are not added to the database.  Objects found in the
lists that are not boxes, polygons, or wires are ignored.  The new
objects are placed on the layer with the name given in {\it lname},
which is created if it does not exist, independent of the originating
layer of the objects.  If a null string or 0 is passed for {\it
lname}, the target layer is the first layer found in {\it
object\_handle1}, or {\it object\_handle2} if {\it object\_handle1} is
empty.  The {\it all1} and {\it all2} are integer arguments indicating
whether to use only the first object in the list, or all objects in
the list.  If nonzero, then all boxes, polygons, and wires in the
corresponding list will be used, otherwise only the first box,
polygon, or wire will be processed.  On success, a handle to a list of
object copies is returned, zero is returned otherwise.  A fatal error
is triggered if either argument is not a handle to a list of objects.

\end{description}


\subsection{Other Object Management Functions}

\begin{description}
%------------------------------------
% 100408
\index{ChangeLayer function}
\item{(int) \vt ChangeLayer()}\\
This function will change the layer of all selected geometry to the
current layer.  This is similar to the functionality of the {\cb
Chg Layer} button in the {\cb Modify Menu}.

%------------------------------------
% 100412
\index{Bloat function}
\item{(int) \vt Bloat({\it dimen\/}, {\it mode\/})}\\
Each selected object is bloated by the given dimension, similar to the
{\cb !bloat} command.  The returned value is 0 on success, or 1 if
there was a runtime error.  This function will return 1 if not called
in physical mode.

The second argument is an integer that specifies the algorithm to use
for bloating.  Giving zero specifies the default algorithm.  See the
description of the {\cb !bloat} command (\ref{bloatcmd}) for
documentation of the algorithms available. 

%------------------------------------
% 100412
\index{Manhattanize function}
\item{(int) \vt Manhattanize({\it dimen\/}, {\it mode\/})}\\
Each selected non-Manhattan polygon or wire is converted to a
Manhattan polygon or box approximation, similar to the {\cb !manh}
command.  The first argument is a size in microns representing the
smallest dimension of the boxes created to approximate the
non-Manhattan parts.  The second argument is a boolean value that
specifies which of two algorithms to use.  These algorithms are
described with the {\cb !manh} command.

The returned value is 0 on success, or 1 if there was a runtime error. 
This function will return 1 if not called in physical mode.  The
function will fail if the {\it dimen} argument is smaller than 0.01.

%------------------------------------
% 030204
\index{Join function}
\item{(int) \vt Join()}\\
The selected objects that touch or overlap are merged together into
polygons, similar to the {\cb !join} command.  The returned value is 0
on success, 1 if there is a runtime error.  This function will return
1 if not called in physical mode.

%------------------------------------
% 030204
\index{Decompose function}
\item{(int) \vt Decompose({\it vert\/})}\\
The selected polygons and wires are decomposed into elemental
non-overlapping trapezoids (polygons) similar to the {\cb !split}
command.  If the integer argument is nonzero, the decomposition favors
a vertical orientation, otherwise the splitting favors horizontal. 
The returned value is 0 if called in physical mode, 1 if not called in
physical mode (an error).

%------------------------------------
% 030204
\index{object creation!boxes}
\index{Box function}
\item{(int) \vt Box({\it left\/}, {\it bottom\/}, {\it right\/},
 {\it top\/})}\\
The four arguments are real values specifying the coordinates of a
rectangle in microns.  Calling this function will generate a box on
the current layer with the given coordinates.  This provides
functionality similar to the {\cb box} menu button.

If the {\vt UseTransform} function has been called to enable use of
the current transform, the current transform will be applied to given
coordinates before the box is created.  The translation supplied to
{\vt UseTransform} is added to the coordinates before the current
transform is applied.

The {\vt Box} function will actually create a polygon if the current
transform is being used and the rotation angle is 45 degrees or one of
the other non-Manhattan angles.

%------------------------------------
% 021913
\index{object creation!boxes}
\index{BoxH function}
\item{(object\_handle) \vt BoxH({\it left\/}, {\it bottom\/}, {\it right\/},
 {\it top\/})}\\
This is similar to the {\vt Box} function, but will return a handle to
the new object.  On error, a scalar 0 is returned.

%------------------------------------
% 030204
\index{Polygon function}
\index{object creation!polygons}
\item{(int) \vt Polygon({\it num\/}, {\it arraypts\/})}\\
This function creates a polygon on the current layer.  The second
argument is an array of values, taken as x-y pairs.  The first pair of
values must be the same as the last, i.e., the path must be closed. 
The first argument is the number of pairs of coordinates in the array. 
This provides functionality similar to the {\cb polyg} menu button.

If the {\vt UseTransform} function has been called to enable use of
the current transform, the current transform will be applied to the
given coordinates before the polygon is created.  The translation
supplied to {\vt UseTransform} is added to the coordinates before the
current transform is applied.

The {\vt Polygon} function will actually create a box if the rotated
figure can be so represented.  The {\vt Polygon} function will never
create boxes unless use of the current transform is enabled.

%------------------------------------
% 021913
\index{PolygonH function}
\index{object creation!polygons}
\item{(object\_handle) \vt PolygonH({\it num\/}, {\it arraypts\/})}\\
This is similar to the {\vt Polygon} function, but will return a
handle to the new object.  On error, a scalar 0 is returned.

%------------------------------------
% 012815
\index{Arc function}
\item{(int) \vt Arc({\it x\/}, {\it y\/}, {\it rad1X\/}, {\it rad1Y\/},
 {\it rad2X\/}, {\it rad2Y\/}, {\it ang\_start\/}, {\it ang\_end\/})}\\
This produces a circular or elliptical solid or ring-like figure,
providing functionality similar to the {\cb round}, {\cb donut}, and
{\cb arc} buttons in the physical side menu.

\begin{tabular}{ll}
\it x, y & center coordinates\\
\it rad1X, rad1Y & x and y inner radii\\
\it rad2X, rad2Y & x and y outer radii\\
\it ang\_start & starting angle in degrees\\
\it ang\_end & ending angle in degrees\\
\end{tabular}

All dimensions are given in microns.  The first two arguments provide
the center coordinates.  The second two arguments are the inner radius
in the X and Y directions.  If these differ, the inner radus will be
elliptical, otherwise it will be circular.  If both are zero, the
figure will not have an inner surface.

Similarly, the next two arguments specify the outer radius, X and Y
directions separately.  Both are required to be larger than the inner
radius counterpart.

The final two arguments are the start and end angle, given in degrees. 
If {\it ang\_start\/} and {\it ang\_end\/} are equal, a donut (ring
figure) is produced.  If the outer and inner radii are equal, a solid
figure is produced.  Angles are defined from the positive x-axis, in a
counter-clockwise sense.  The arc is generated in a clockwise
direction.

If the {\vt UseTransform} function has been called to enable use of
the current transform, the current transform will be applied to the
arc coordinates before the arc is created.  The translation supplied
to {\vt UseTransform} is added to the coordinates before the current
transform is applied.

The function returns 1 on success, 0 otherwise.

%------------------------------------
% 021913
\index{ArcH function}
\item{(object\_handle) \vt ArcH({\it x\/}, {\it y\/}, {\it rad1X\/},
 {\it rad1Y\/}, {\it rad2X\/}, {\it rad2Y\/}, {\it ang\_start\/},
 {\it ang\_end\/})}\\
This is similar to the {\vt Arc} function, but will return a handle to
the new object.  On error, a scalar 0 is returned.


%------------------------------------
% 012815
\index{Round function}
\item{(int) \vt Round({\it x\/}, {\it y\/}, {\it rad\/})}\\
This a simplification of the {\vt Arc} function which simply creates a
circular disk object at the location specified in the first two
arguments.  All dimensions are in microns.  The third argument
specifies the radius.
  
The function returns 1 on success, 0 otherwise.

%------------------------------------
% 012815
\index{RoundH function}
\item{(object\_handle) \vt RoundH({\it x\/}, {\it y\/}, {\it rad\/})}\\
This is similar to the {\vt Round} function, but will return a handle
to the new object.  On error, a scalar 0 is returned.

%------------------------------------
% 012815
\index{HalfRound function}
\item{(int) \vt HalfRound({\it x\/}, {\it y\/}, {\it rad\/},
 {\it dir\/})}\\
This is a simplification of the {\vt Arc} function which creates a
half-circular figure.  The first two arguments indicate the center of
an equivalent full circle, i.e., it is the midpoint of the flat edge. 
The {\it dir} argument is an integer 0--7 which specifies the
orientation, in increments of 45 degrees.  With 0, the flat section is
horizontal with the curved surface on top.  The dir rotates clockwise,
so that a value of 2 would produce a figure that looks like the letter
D.
  
The function returns 1 on success, 0 otherwise.

%------------------------------------
% 012815
\index{HalfRoundH function}
\item{(object\_handle) \vt HalfRoundH({\it x\/}, {\it y\/}, {\it rad\/},
 {\it dir\/})}\\
This is similar to the {\vt HalfRound} function, but will return a
handle to the new object.  On error, a scalar 0 is returned.

%------------------------------------
% 021515
\index{Sides function}
\item{(int) \vt Sides({\it numsides\/})}\\
This sets the number of segments to use in generating round objects,
for the current display mode (electrical or physical).  The function
returns the present value for this parameter.  This is similar to the
{\cb sides} side menu button in physical mode.  It simply sets the
{\et RoundFlashSides} variable, or clears the variable if the number
of sides given is the default.  Similarly, in electrical mode it is
similar to the {\cb sides} entry in the menu from the {\cb shape}
button in the side menu, and sets or clears the {\et
ElecRoundFlashSides} variable.

%------------------------------------
% 030204
\index{Wire function}
\index{object creation!wires}
\item{(int) \vt Wire({\it width\/}, {\it num\/}, {\it arraypts\/},
 {\it end\_style\/})}\\
This function creates a wire on the current layer.  The first argument
is the width of the wire in microns.  The third argument is the name
of an array of coordinates, taken as x-y pairs.  The second argument
is the number of coordinate pairs in the array.  The fourth argument
is 0, 1, or 2 to set the end style to flush, rounded, or extended,
respectively.  This provides the functionality of the {\cb wire} menu
button.

If the {\vt UseTransform} function has been called to enable use of
the current transform, the current transform will be applied to the
given coordinates before the wire is created.  The translation
supplied to {\vt UseTransform} is added to the coordinates before the
current transform is applied.  The variable {\et NoWireWidthMag} will
suppress changes to the wire width due to the magnification component
of the current transform when set.

%------------------------------------
% 021913
\index{WireH function}
\index{object creation!wires}
\item{(object\_handle) \vt WireH({\it width\/}, {\it num\/}, {\it arraypts\/},
 {\it end\_style\/})}\\
This is similar to the {\vt Wire} function, but will return a handle
to the new object.  On error, a scalar 0 is returned.

%------------------------------------
% 022713
\index{Label function}
\index{object creation!labels}
\item{(int) \vt Label({\it text}, {\it x}, {\it y} [, {\it width\/},
{\it height\/}, {\it flags\/})]}\\
This function creates a label on the current layer.  The function
takes a variable number of arguments, but the first three must be
present.  The first argument is of string type and contains the label
text.  The next two arguments specify the x and y coordinates of the
label reference point.

The remaining arguments are optional.  The {\it width} and {\it
height} specify the size of the bounding box into which the text will
be rendered, in microns.  if both are zero or negative or not given, a
default size will be used.  If only one is given a value greater than
zero, the other will be computed using a default aspect ratio.  If
both are greater than zero, the text will be squeezed or stretched to
conform.

The {\it flags} argument is a label flags word used in {\Xic} to set
various label attributes, as described in \ref{labelflags}.  If given,
the {\vt Justify} function and {\vt UseTransform} function settings
will be ignored, and these attributes will be set from the {\it
flags}.  If {\it flags} is not given, the functions will set the
justification and transformation.

This function always returns 1.

%------------------------------------
% 022713
\index{LabelH function}
\index{object creation!labels}
\item{(object\_handle) \vt LabelH({\it text}, {\it x}, {\it y}
 [, {\it width\/}, {\it height\/}, {\it xform\/})]}\\
This is similar to the {\vt Label} function, but will return a handle
to the new object.  On error, a scalar 0 is returned.

%------------------------------------
% 100408
\index{Logo function}
\item{(int) \vt Logo({\it string\/}, {\it x\/}, {\it y\/} [, {\it width\/},
{\it height\/}])}\\
This creates and places physical text, i.e., text that is constructed
with database polygons that will appear in the mask layout.  The
function takes a variable number of arguments, but the first three
must be present.  The first argument is of string type and contains
the label text.  The next two arguments specify the x and y
coordinates of the reference point, which is dependent on the current
justification, as set with the {\vt Justify} function.  The default is
the lower-left corner of the bounding box.  The text will be
transformed according to the current transform.

The remaining arguments are optional.  The {\it width} and {\it
height} specify the approximate size of the rendered text.  Unlike the
{\vt Label} function, the text aspect ratio is fixed.  The first of
{\it height} or {\it width} which is positive will be used to set the
``pixel'' size used to render the text, by dividing this value by the
character cell height or width of the default font.  Thus, the
rendered text size will only be accurate for this font, and will scale
with the number of pixels used in the ``pretty'' fonts.  One must
experiment with a chosen font to obtain accurate sizing.  If neither
parameter is given and positive, a default size will be used.

This provides the functionality of the {\cb logo} menu button, and is
sensitive to the following variables.

\begin{quote}\et
 LogoEndStyle\\
 LogoPathWidth\\
 LogoAltFont\\
 LogoPrettyFont\\
 LogoToFile
\end{quote}

This function always returns 1.

%------------------------------------
% 030204
\index{Justify function}
\item{(int) \vt Justify({\it hj\/}, {\it vj\/})}\\
This sets the justification for text created with the {\cb logo} and
{\cb label} commands and corresponding script functions.  The
arguments can have the following values:

\begin{tabular}{lll}
\it hj/vj & horizontal & vertical\\
0 & left & bottom\\
1 & center & center\\
2 & right & top\\
\end{tabular}

Values out of range will preserve the present justification setting.
The function always returns 1.

%------------------------------------
% 030204
\index{Delete function}
\index{object deletion}
\item{(int) \vt Delete()}\\
This function deletes all selected objects from the database.

%------------------------------------
% 030204
\index{Erase function}
\index{object erasing}
\item{(int) \vt Erase({\it left\/}, {\it bottom\/}, {\it right\/},
 {\it top\/})}\\
This function erases the rectangular area defined by the arguments. 
Polygons, wires, and boxes are appropriately clipped.  The erase
function has no effect on subcells or labels.  This provides an erase
capability similar to the {\cb erase} menu button.

%------------------------------------
% 100412
\index{EraseUnder function}
\item{(int) \vt EraseUnder()}\\
This function will erase geometry from unselected objects that
intersect with objects that are selected.  This is equivalent to the
{\cb Erase Under} command in {\Xic}.  This function always returns 1.

%------------------------------------
% 030204
\index{Yank function}
\item{(int) \vt Yank({\it left\/}, {\it bottom\/}, {\it right\/},
 {\it top\/})}\\
This function puts the geometry in the specified rectangle in yank
buffer 0.  It can be placed with the {\vt Put} function, or the {\cb
put} command.  This provides a yank capability similar to the {\cb
erase} button in the side menu.

%------------------------------------
% 030204
\index{Put function}
\item{(int) \vt Put({\it x\/}, {\it y\/}, {\it bufnum\/})}\\
This puts the contents of the indicated yank buffer in the current
layout, with the lower left at {\it x\/}, {\it y\/}.  The {\it bufnum\/} is
the yank buffer index, which can be 0--4.  Buffer 0 is the most recent
yank or erase, buffer 1 is the next most recent, etc.  This provides
functionality similar to the {\cb put} button in the side menu.

%------------------------------------
% 030204
\index{Xor function}
\index{object invert}
\item{(int) \vt Xor({\it left\/}, {\it bottom\/}, {\it right\/},
 {\it top\/})}\\
This function exclusive-or's the rectangular area defined by the
arguments with boxes, polygons, and wires on the current layer. 
Existing objects become clear areas.  This provides functionality
similar to the {\cb xor} button in the side menu.

%------------------------------------
% 082009
\index{Copy function}
\index{object copy}
\item{(int) \vt Copy({\it fromx\/}, {\it fromy\/}, {\it tox\/},
 {\it toy\/}, {\it repcnt\/})}\\
Copies of selected objects are created and placed such that the point
specified by the first two arguments is moved to the location
specified by the second two arguments.

The {\it repcnt} is an integer replication count in the range
1--100000, which will be silently taken as one if out of range.  If
not one, multiple copies are made, at multiples of the translation
factors given.

This provides functionality similar to the {\cb Copy} button in the
{\cb Modify Menu}.  The return value is 1 if there were no errors and
something was copied, 0 otherwise.

%------------------------------------
% 082009
\index{CopyToLayer function}
\item{(int) \vt CopyToLayer({\it fromx\/}, {\it fromy\/}, {\it tox\/},
 {\it toy\/}, {\it oldlayer\/}, {\it newlayer\/}, {\it repcnt\/})}\\
This is similar to the {\vt Copy} function, but allows layer change. 
If {\it newlayer} is 0, null, or empty, {\it oldlayer} is ignored and
the function behaves identically to {\vt Copy}.  Otherwise the {\it
newlayer} string must be a layer name.  If {\it oldlayer} is 0, null,
or empty, all copied objects are placed on {\it newlayer\/}. 
Otherwise, {\it oldlayer} must be a layer name, in which case only
objects on {\it oldlayer} will be placed on {\it newlayer\/}, other
objects will remain on the same layer.  Subcell objects are copied as
in {\vt Copy}, i.e., the layer arguments are ignored.

%------------------------------------
% 100408
\index{Move function}
\index{object move}
\item{(int) \vt Move({\it fromx\/}, {\it fromy\/}, {\it tox\/},
 {\it toy\/})}\\
This function moves the selected objects such that the reference point
specified in the first two arguments is moved to the point specified
by the second two arguments.  This provides functionality similar to
the {\cb Move} button in the {\cb Modify Menu}.  The return value is 1
if there were no errors and something was moved, 0 otherwise.

%------------------------------------
% 030204
\index{MoveToLayer function}
\item{(int) \vt MoveToLayer({\it fromx\/}, {\it fromy\/}, {\it tox\/},
 {\it toy\/}, {\it oldlayer\/}, {\it newlayer\/})}\\
This is similar to the {\vt Move} function, but allows layer change. 
If {\it newlayer} is 0, null, or empty, {\it oldlayer} is ignored and
the function behaves identically to {\vt Move}.  Otherwise the {\it
newlayer} string must be a layer name.  If {\it oldlayer} is 0, null,
or empty, all moved objects are placed on {\it newlayer\/}. 
Otherwise, {\it oldlayer} must be a layer name, in which case only
objects on {\it oldlayer} will be placed on {\it newlayer\/}, other
objects will remain on the same layer.  Subcell objects are moved as
in {\vt Move}, i.e., the layer arguments are ignored.

%------------------------------------
% 081908
\index{Rotate function}
\index{object rotation}
\item{(int) \vt Rotate({\it x\/}, {\it y\/}, {\it ang\/}, {\it remove\/})}\\
The selected objects are rotated counter-clockwise by {\it ang\/} (in
degrees) about he point specified in the first two arguments.  This
provides functionality similar to the {\cb spin} button in the side
menu.

If the boolean argument {\it remove} is true (nonzero), the original
objects will be deleted.  Otherwise, the original objects are
retained, and will become deselected.

The return value is 1 if there were no errors and something was
rotated, 0 otherwise.

Note:  in releases prior to 3.0.5, the {\it remove} argument was 
absent and effectively 0 in the current function implementation.

%------------------------------------
% 081908
\index{RotateToLayer function}
\item{(int) \vt RotateToLayer({\it x\/}, {\it y\/}, {\it ang\/},
 {\it oldlayer\/}, {\it newlayer\/}, {\it remove\/})}\\
This is similar to the {\vt Rotate} function, but allows layer change. 
If {\it newlayer} is 0, null, or empty, {\it oldlayer} is ignored and
the function behaves identically to {\vt Rotate}.  Otherwise the {\it
newlayer} string must be a layer name.  If {\it oldlayer} is 0, null,
or empty, all rotated objects are placed on {\it newlayer\/}. 
Otherwise, {\it oldlayer} must be a layer name, in which case only
objects on {\it oldlayer} will be placed on {\it newlayer\/}, other
objects will remain on the same layer.  Subcell objects are rotated as
in {\vt Rotate}, i.e., the layer arguments are ignored.

If the boolean argument {\it remove} is true (nonzero), the original
objects will be deleted.  Otherwise, the original objects are
retained, and will become deselected.

The return value is 1 if there were no errors and something was
rotated, 0 otherwise.

Note:  in releases prior to 3.0.5, the {\it remove} argument was 
absent and effectively 0 in the current function implementation.

%------------------------------------
% 030204
\index{Split function}
\item{(int) \vt Split({\it x\/}, {\it y\/}, {\it flag\/}, {\it orient\/})}\\
This will sever selected objects along a vertical or horizontal line
through {\it x\/}, {\it y\/} if {\it flag\/} is nonzero.  If {\it orient\/}
is 0, the break line is vertical, otherwise it is horizontal.  If {\it
flag\/} is zero, the function will return 1 if an object would be split,
0 otherwise, though no objects are actually split.  This provides
functionality similar to the {\cb break\/} button in the side menu.

%------------------------------------
% 102913
\index{Flatten function}
\index{hierarchy of cells}
\index{flatten hierarchy}
\item{(int) \vt Flatten({\it depth\/}, {\it use\_merge\/},
  {\it fast\_mode\/})}\\
The selected subcells are flattened into the current cell, recursively
to the given depth, similar to the effect of the {\cb Flatten} button
in the {\cb Edit Menu}.

The {\it depth} argument may be an integer representing the depth into
the hierarchy to flatten:  0 for top-level subcells only, 1 to include
second-level subcells, etc.  This argument can also be a string
starting with `{\vt a}' to signify flattening all levels.  A negative
depth also signifies flattening all levels.

The {\it use\_merge} argument is a boolean which if nonzero indicates
that new objects will be merged with existing objects when added to
the current cell.  This is the same merging as specified in the {\cb
Editing Setup} panel from the {\cb Edit Menu}, or corresponding
variables.

If the boolean argument {\it fast\_mode} is nonzero, ``fast'' mode is used,
meaning that there will be no undo list generation and no object
merging.  This is not undoable so should be used with care.

The function returns 1 on success, 0 otherwise, with an error message
probably available from {\vt GetError}.

%------------------------------------
% 082809
\index{Layer function}
\item{\vt Layer({\it string}, {\it mode\/}, {\it depth}, {\it recurse\/},
 {\it noclear\/}, {\it use\_merge\/}, {\it fast\_mode\/})}\\
This is very similar to the {\cb !layer} command, and operations from
the {\cb Evaluate Layer Expression} panel brought up with the {\cb
Layer Expression} button in the {\cb Edit Menu}.  The {\it string} is
of the form
\begin{quote}
  ``{\it new\_layer\_name} [=] {\it layer\_expression}''.
\end{quote}

The {\it mode} argument is an integer which sets the split/join mode,
similar to the keywords in the {\cb !layer} command, and the buttons
in the {\cb Evaluate Layer Expression} panel.  Only the two
least-significant bits of the integer value are used.

\begin{quote}
\begin{tabular}{ll}
0 & default\\
1 & horizontal split\\
2 & vertical split\\
3 & join\\
\end{tabular}
\end{quote}

The {\it depth} is the search depth, which can be an integer which
sets the maximum depth to search (0 means search the current cell
only, 1 means search the current cell plus the subcells, etc., and a
negative integer sets the depth to search the entire hierarchy).  This
argument can also be a string starting with `{\vt a}' such as ``{\vt
a}'' or ``{\vt all}'' which specifies to search the entire hierarchy.

The {\it recurse} argument is a boolean value which corresponds to the
``{\vt -r}'' option of the {\cb !layer} command, or the {\cb
Recursively create in subcells} check box in the {\cb Evaluate Layer
Expression} panel.  If nonzero, evaluation will be performed in
subcells to depth, using only that cell's geometry.  When zero,
geometry is created in the current cell only, using geometry found in
subcells to depth.

If the boolean argument {\it noclear} is true, the target layer will
not be cleared before expression evaluation.  This corresponds to the
``{\vt -c}'' option of the {\cb !layer} command, and the {\cb Don't
clear layer before evaluation} button in the {\cb Evaluate Layer
Expression} panel.

The boolean argument {\it use\_merge} corresponds to the ``{\vt -m}''
option in the {\cb !layer} command, and the {\cb Use object merging
while processing} check box in the {\cb Evaluate Layer Expression}
panel.  When nonzero, new objects will be merged with existing objects
when added to a cell.

The {\it fast\_mode} argument is a boolean value that corresponds to
the ``{\vt -f}'' option in the {\cb !layer} command, and the {\cb Fast
mode} check box in the {\cb Evaluate Layer Expression} panel.  When
nonzero, undo list processing and merging are skipped for speed and to
reduce memory use.  However, the result is not undoable so this flag
should be used with care.

There is no return value; the function either succeeds or will
terminate the script on error.
\end{description}


\subsection{Property Management}

% 030515
The functions described in this section provide an interface for
working with properties.

When specifying the property ``number'' for electrical mode
properties, either a number or string equivalent can be used.  The
string equivalent is a prefix of one of the supported property names. 
In addition, some of the properties have a letter that any word that
starts with the letter will indicate that property.  The idea was that
each property could be keyed by a single letter, and this is almost
still true ({\et node} is the exception).

The following table identifies the recognized strings.  Not all of
these properties apply in all functions.  The listed order is the
order of testing, the first match yields the equivalence.


\begin{tabular}{|l|l|l|} \hline
\bf Number & \bf Name & \bf String\\ \hline\hline
1  & \et model  & prefix\\ \hline
2  & \et value  & prefix\\ \hline
3  & \et param  & prefix\\ \hline
3  & \et initc  & prefix\\ \hline
4  & \et other  & prefix\\ \hline
11 & \et name   & prefix\\ \hline
5  & \et nophys & prefix or starts with `y' or `Y'\\ \hline
6  & \et virtual & prefix or starts with `t' ot `T'\\ \hline
7  & \et flatten & prefix\\ \hline
8  & \et range  & prefix\\ \hline
10 & \et node   & prefix\\ \hline
18 & \et nosymb & prefix or starts with `s' or `S'\\ \hline
20 & \et macro  & prefix or starts with `c' or `C'\\ \hline
21 & \et devref & prefix\\ \hline
\end{tabular}

The {\et initc} is an archaic alias for the {\et param} property that
is still recognized.  In some functions, an additonal keyword ``{\vt
all}'' is recognized in a way that has significance to the function. 
If the string does not match, an error is indicated.

\begin{description}
%------------------------------------
% 030215
\index{PrpHandle function}
\item{(prpty\_handle) \vt PrpHandle({\it object\_handle})}\\
This function returns a handle to the list of properties of the object
referenced by the passed object handle.  The function fails if the
argument is not a valid object handle, use {\vt CellPrpHandle} to list
cell properties.

%------------------------------------
% 040106
\index{GetPrpHandle function}
\item{(prpty\_handle) \vt GetPrpHandle({\it number\/})}\\
Since there can be arbitrarily many properties defined with the same
number, a generator function is used to read properties one at a time. 
This function returns a handle to a list of the properties that match
the {\it number} passed.  This applies to the first object in the
selection queue (the most recent object selected).  The returned value
is used by other functions to actually retrieve the property text.

If the {\it number} argument is a prefix of ``{\vt all}'', then any
property string will be returned.  In physical mode, the {\it number}
argument should otherwise be an integer.  In electrical mode, the {\it
number} argument can have string form as described in the introduction
to this section.

%------------------------------------
% 030215
\index{CellPrpHandle function}
\item{(prpty\_handle) \vt CellPrpHandle()}\\
This function returns a handle to the list of properties of the
current cell, applicable to the current display mode in the main
window.

%------------------------------------
% 040106
\index{GetCellPrpHandle function}
\item{(prpty\_handle) \vt GetCellPrpHandle({\it number\/})}\\
Since there can be arbitrarily many properties defined with the same
number, a generator function is used to read properties one at a time. 
This function returns a handle to a list of the properties that match
the {\it number} passed, from the current cell.  The returned value is
used by other functions to actually retrieve the property text.

A prefix of the string ``{\vt all}'' can be passed for the {\it
number} argument, in which case the handle will reference all
properties of the cell.  In physical mode, the {\it number} argument
should otherwise be an integer.  In electrical mode, the {\it number}
argument can have string form as described in the introduction to this
section.

%------------------------------------
% 030215
\index{PrpNext function}
\item{(int) \vt PrpNext({\it prpty\_handle})}\\
This function causes the referenced property of the passed handle to
be advanced to the next in the list.  If there are no other properties
in the list, the handle is closed, and 0 is returned.  Otherwise, the
handle (same as the argument) is returned.  The number of remaining
reference objects can be obtained with the {\vt HandleContent} function.

%------------------------------------
% 030215
\index{PrpNumber function}
\item{(int) \vt PrpNumber({\it prpty\_handle})}\\
This function returns the number of the property referenced by the
handle.

%------------------------------------
% 030215
\index{PrpString function}
\item{(string) \vt PrpString({\it prpty\_handle})}\\
This function returns the string of the property referenced by the
handle.  The ``raw'' string is returned, meaning that if the property
comes from an electrical object, all of the detail from the internal
property string is returned.

%------------------------------------
% 030215
\index{PrptyString function}
\item{(string) \vt PrptyString({\it obj\_or\_prp\_handle\/}, {\it number\/})}\\
The first argument can be a property handle, or an object handle.  If
a property handle is given, the function returns the string of the
first property referenced by the handle that matches the {\it number}. 
If the {\it number} argument is a prefix of ``{\vt all}'', then any
property string will be returned.  In physical mode, the {\it number}
argument should otherwise be an integer.  In electrical mode, the {\it
number} argument can be a string, as described in the introduction to
this section.  The handle is set to reference the next property in the
reference list, following the one returned.  When there are no more
properties, this function returns a null string.

If the first argument is an object handle, the function returns the
strings from properties or pseudo-properties for the object referenced
by the handle.

In physical mode, the function will locate a property with the given
number, and return its string.  If no property is found with that
number, and a pseudo-property for the object matches the number, then
the pseudo-property string is returned.  If no matching
pseudo-property is found, a null string is returned.  Note:  objects
can be modified through setting pseudo-properties using the {\vt
PrptyAdd} function.

In electrical mode, the number argument can be a string, as described
in the introduction to this section.  In the case of an object handle,
the ``{\vt all}'' keyword is not supported.

The function will fail if the argument is not a valid object or
property handle.  Use\\ {\vt GetCellPropertyString} to obtain strings
from cell properties.

If the requested property is a {\et name} property of an electrical
device or subcircuit, only the name is returned (the internal property
string is more complex).  Otherwise the ``raw'' string is returned.

%------------------------------------
% 030215
\index{GetPropertyString function}
\item{(string) \vt GetPropertyString({\it number\/})}\\
This function searches the selection queue for an object with a
property matching {\it number\/}.  The string for the first such
property found is returned.  A null string is returned if no matching
property was found.

%------------------------------------
% 030215
\index{GetCellPropertyString function}
\item{(string) \vt GetCellPropertyString({\it number\/})}\\
This function searches the properties of the current cell, and returns
the string for the first property found that matches {\it number}.  If
no match, a null string is returned.

%------------------------------------
% 030515
\index{PrptyAdd function}
\item{(int) \vt PrptyAdd({\it object\_handle}, {\it number\/},
 {\it string\/})}\\
This function will create a new property using the {\it number} and
{\it string} provided, on the object referenced by the handle.  The
object must be defined in the current cell.  The function will fail if
the handle is invalid.  Use {\vt CellPropertyAdd} to add properties to
the current cell.

In physical mode, the property number can take any non-negative value. 
This includes property numbers that are used by {\Xic} for various
purposes in the range 7000--7199.  Unless the user is expecting the
{\Xic} interpretation of the property number, these numbers should be
avoided.  It is the caller's responsibility to ensure that the
properties in this range are applied to the appropriate objects, in
the correct context and with correct syntax, as there is little or no
checking.  Adding some properties in this range such as {\et flags},
{\et flatten}, or a pcell property will automatically remove an
existing property with the same number, if any.

The pseudo-properties in the range 7200--7299 will have their
documented effect when applied, and no property is added (see
\ref{pseudoprops}),

In electrical mode, it is possible to set these properties of
device instances:
\begin{quote}
{\et name}, {\et model}, {\et value}, {\et param}, {\et devref},
{\et other}, {\et range} {\et nophys}, {\et symblc}
\end{quote}
and the following properties of subcircuit instances:
\begin{quote}
{\et name}, {\et param}, {\et other}, {\et flatten},
{\et range} {\et nophys}, {\et symblc}.
\end{quote}
Attempts to set properties not listed here will silently fail.  The
object must be defined in the current cell, thus the mode must be
electrical.

If the function succeeds, 1 is returned.  otherwise 0 is returned.

%------------------------------------
% 030515
\index{AddProperty function}
\item{(int) \vt AddProperty({\it number}, {\it string})}\\
This function adds a property with the given number and string to all
selected objects.

In physical mode, the property number can take any non-negative value. 
This includes property numbers that are used by {\Xic} for various
purposes in the range 7000--7199.  Unless the user is expecting the
{\Xic} interpretation of the property number, these numbers should be
avoided.  It is the caller's responsibility to ensure that the
properties in this range are applied to the appropriate objects, in
the correct context and with correct syntax, as there is little or no
checking.

The pseudo-properties in the range 7200--7299 will have their
documented effect when applied, and no property is added,

In electrical mode, it is possible to set these properties of
device instances:
\begin{quote}
{\et name}, {\et model}, {\et value}, {\et param}, {\et devref},
{\et other}, {\et range} {\et nophys}, {\et symblc}
\end{quote}
and the following properties of subcircuit instances:
\begin{quote}
{\et name}, {\et param}, {\et other}, {\et flatten},
{\et range} {\et nophys}, {\et symblc}.
\end{quote}
Attempts to set properties not listed here will silently fail.  The
object must be defined in the current cell, thus the mode must be
electrical.

The number of properties added plus the number of pseudo-properties
applied is returned.

%------------------------------------
% 030515
\index{AddCellProperty function}
\item{(int) \vt AddCellProperty({\it number}, {\it string})}\\
This function adds a property to the current cell.

In physical mode, the property number can take any non-negative value. 
This includes property numbers that are used by {\Xic} for various
purposes in the range 7000--7199.  Unless the user is expecting the
{\Xic} interpretation of the property number, these numbers should be
avoided.  It is the caller's responsibility to ensure that the
properties in this range are applied to the appropriate objects, in
the correct context and with correct syntax, as there is little or no
checking.  Adding some properties in this range such as {\et flags},
{\et flatten}, or a pcell property will automatically remove an
existing property with the same number, if any.

Numbers in the pseudo-property range 7200--7299 will do nothing.

In electrical mode, it is possible to set the {\et param}, {\et
other}, {\et virtual}, {\et flatten}, {\et macro}, {\et node}, {\et
name}, and {\et symbolic} properties of the current cell.  The last
three are not ``user settable'' but are needed when building up a new
circuit cell in memory, as in the scripts produced by the {\cb
!mkscript} command.  The string should have the format as read from a
native cell file.

The function returns 1 if the operation was successful, 0 otherwise.

%------------------------------------
% 030515
\index{PrptyRemove function}
\item{(int) \vt PrptyRemove({\it object\_handle}, {\it number\/},
 {\it string\/})}\\
This function will remove properties matching the given {\it number}
and {\it string} from the object referenced by the handle.

In physical mode, the property number can take any non-negative value. 
This includes property numbers that are used by {\Xic} for various
purposes in the range 7000--7199.  It is the caller's responsibility
to make sure that removal of properties in this range is appropriate. 
Giving numbers in the pseudo-property range 7200--7299 will do
nothing.

If the {\it string} is null or empty, only the {\it number} is used
for comparison, and all properties with that number will be removed. 
Otherwise, if the {\it string} is a prefix of the property string and
the numbers match, the property will be removed.

In electrical mode, it is possible to remove these properties of
device instances:
\begin{quote}
{\et name}, {\et model}, {\et value}, {\et param}, {\et devref},
{\et other}, {\et range} {\et nophys}, {\et symblc}
\end{quote}
and the following properties of subcircuit instances:
\begin{quote}
{\et name}, {\et param}, {\et other}, {\et flatten},
{\et range} {\et nophys}, {\et symblc}.
\end{quote}
Attempts to remove properties not listed here will silently fail. 
Except for {\et other}, the string argument is ignored.  For {\et
other} properties, the string is used as above to identify the
property to delete.

Objects must be defined in the current cell.  The function returns
the number of properties removed.

%------------------------------------
% 030515
\index{RemoveProperty function}
\item{(int) \vt RemoveProperty({\it number}, {\it string})}\\
This function will remove properties from selected objects.

In physical mode, the property number can take any non-negative value. 
This includes property numbers that are used by {\Xic} for various
purposes in the range 7000--7199.  It is the caller's responsibility
to make sure that removal of properties in this range is appropriate. 
Giving numbers in the pseudo-property range 7200--7299 will do
nothing.

If the {\it string} is null or empty, only the {\it number} is used
for comparison, and all properties with that number will be removed. 
Otherwise, if the {\it string} is a prefix of the property string and
the numbers match, the property will be removed.

In electrical mode, it is possible to remove these properties of
device instances:
\begin{quote}
{\et name}, {\et model}, {\et value}, {\et param}, {\et devref},
{\et other}, {\et range} {\et nophys}, {\et symblc}
\end{quote}
and the following properties of subcircuit instances:
\begin{quote}
{\et name}, {\et param}, {\et other}, {\et flatten},
{\et range} {\et nophys}, {\et symblc}.
\end{quote}
Attempts to remove properties not listed here will silently fail. 
Except for {\et other}, the string argument is ignored.  For {\et
other} properties, the string is used as above to identify the
property to delete.
 
The number of properties removed is returned.

%------------------------------------
% 030515
\index{RemoveCellProperty function}
\item{(int) \vt RemoveCellProperty({\it number}, {\it string})}\\
This function will remove properties from the current cell.

In physical mode, the property number can take any non-negative value. 
This includes property numbers that are used by {\Xic} for various
purposes in the range 7000--7199.  It is the caller's responsibility
to make sure that removal of properties in this range is appropriate. 
Giving numbers in the pseudo-property range 7200--7299 will do
nothing.

If the {\it string} is null or empty, only the {\it number} is used
for comparison, and all properties with that number will be removed. 
Otherwise, if the {\it string} is a prefix of the property string and
the numbers match, the property will be removed.

In electrical mode, it is possible to remove the {\et param}, {\et
other}, {\et virtual}, {\et flatten}, and {\et macro} properties of
the current cell.  Except for {\et other}, the string argument is
ignored.  For {\et other} properties, the string is used as above to
identify the property to delete.

The function returns the number of properties removed.
\end{description}


%------------------------------------------------------------------------------
\section{Computational Geometry and Layer Expressions}

\subsection{Trapezoid Lists and Layer Expressions}
\label{zoidlistarg}

For the functions described below, a ``zoidlist'' argument can
actually have the following data types:

\begin{tabular}{|l|p{4in}|} \hline
zoidlist & Obviously\\ \hline
integer zero & Implies an empty zoidlist\\ \hline
integer nonzero & Implies the reference zoidlist\\ \hline
string & The string is parsed as a layer expression,
  which is evaluated, and the result used\\ \hline
layer\_expr & evaluate layer expression, use result\\ \hline
\end{tabular}


\begin{description}
%------------------------------------
% 010509
\index{SetZref function}
\item{(int) \vt SetZref({\it arg\/})}\\
This function sets the reference zoidlist.  The reference zoidlist
represents the current ``background'' needed by some functions and
operators which manipulate zoidlists.  For example, when a zoidlist is
polarity inverted, the reference zoidlist specifies the boundary of
the inversion, i.e., the inverse of an empty zoidlist would be the
reference zoidlist.

The reference zoidlist can be set from various types of object passed
as the {\it arg}.  This can be a zoidlist, or an object handle, or an
array of size 4 or larger, which contains rectangle coordinates in
microns in order left, bottom, right, top.  The argument can also be
the constant 0, in which case the reference zoid list will be the
boundary of the physical current cell, or a large ``infinity'' box if
there is no current cell.  This is the default if no reference zoid
list is given.

This function will return 1 and fails only if the argument is not
an appropriate type.

%------------------------------------
% 103104
\index{GetZref function}
\item{(zoidlist) \vt GetZref()}\\
This function returns the current reference zoidlist, which will be
empty if no reference area has been set with {\vt SetZref} or
otherwise.

%------------------------------------
% 010509
\index{GetZrefBB function}
\item{(int) \vt GetZrefBB({\it array\/})}\\
This will return the bounding box of the reference zoidlist, as
returned from {\vt GetZref}.  If the reference zoidlist is empty, the
bounding box of the current cell is returned.  The coordinates are in
microns, in order left, bottom, right, top.  On success, the function
returns 1.  If there is no reference zoidlist or current cell, 0 is
returned.

%------------------------------------
% 010509
\index{AdvanceZref function}
\item{(int) \vt AdvanceZref({\vt clear\/}, {\it array\/})}\\
This function allows iteration over a given area by establishing a
grid over the area and incrementally setting the reference area (see
{\vt SetZref}) to elements of the grid.  The grid is aligned from the
lower-left corner of the given area and iteration advances right and
up.  The reference area is set to the intersection of the grid element
area and the given area.  The size of the square grid elements is
given by the {\et PartitionSize} variable, or defaults to 100 microns
if this variable is not set.

The second argument is an array of size 4 or larger, or 0.  If 0, the
given area is taken to be the bounding box of the current cell. 
Otherwise, the array elements define the given rectangular area, in
microns, in order left, bottom, right, top.

With the boolean first argument set to zero, the function will set the
reference area to the first (lower left) or next grid element
intersection area and return 1.  The function will return zero when it
advances past the last grid element that overlaps the given area, at
which time the reference area is returned to the default value.  Thus,
this function can be used in a loop to limit the computation area for
each iteration, for large cells that would be inefficient to process
in one step.
 
If the first argument is nonzero, the internal state is cleared.  This
should be called if the iteration is not complete and one wishes to
start a new loop.

%------------------------------------
% 103104
\index{Zhead function}
\item{(zoidlist) \vt Zhead({\it zoidlist\/})}\\
This function will remove the first trapezoid from the passed
trapezoid list, and return it as a new list.  If the passed list is
empty, the returned list will be empty.  If the passed list contains a
single trapezoid, it will become empty.

%------------------------------------
% 103104
\index{Zvalues function}
\item{(int) \vt Zvalues({\it zoidlist\/}, {\it array\/})}\\
This function will return the coordinates of the first trapezoid in
the list in the array, which must have size 6 or larger.  The order of
the values is

\begin{tabular}{ll}
0 & x lower-left\\
1 & x lower-right\\
2 & y lower\\
3 & x upper-left\\
4 & x upper-right\\
5 & y upper\\
\end{tabular}

On success, 1 is returned.  If the passed trapezoid list is empty,
the return value is 0 and the array is untouched.

%------------------------------------
% 103104
\index{Zlength function}
\item{(int) \vt Zlength({\it zoidlist\/})}\\
This function returns the number of trapezoids contained in the list
passed as an argument.

%------------------------------------
% 032405
\index{Zarea function}
\item{(int) \vt Zarea({\it zoidlist\/})}\\
This function returns the total area of the trapezoids contained in
the list passed as an argument, in square microns.  This does not
account for overlapping trapezoids, call {\vt GeomOr} first if
overlapping trapezoids are present (lists returned from the script
functions have already been clipped/merged unless otherwise noted).

%------------------------------------
% 071915
\index{GetZlist function}
\item{(zoidlist) \vt GetZlist({\it layersrc}, {\it depth\/})}\\
This function returns a zoidlist from the layer source given in
the first argument, which is a string in the form
\begin{quote}
{\it lname}[{\vt .}{\it stname\/}][{\vt .}{\it cellname\/}]
\end{quote}
Any of {\it lname}, {\it stname}, {\it cname} can be 
double-quoted, which must be true if the token contains the 
separation char `{\vt .}'.  The {\it stname} is the name of a
symbol table, the {\it cname} is tha name of a cell found in the   
symbol table.  If there are only two fields, the second field is
{\it cname}, and the current symbol table is understood.  If no     
{\it cname} is given, the current cell is understood.

The returned list is clipped to the current reference area (see {\vt
SetZref}).  The second argument is the hierarchy depth to search,
which can be a non-negative integer or a string starting with `{\vt
a}' to indicate ``{\vt all}''.  If not called in physical mode, an
empty list is returned.

The layer specification can also be given in the form
\begin{quote}
{\it lname\/}{\vt .@}{\it dbname}
\end{quote}
where {\it dbname} is the name of a saved database.  Operation will be
similar to the {\vt GetZlistDb} script function.

%------------------------------------
% 111709
\index{GetSqZlist function}
\item{(zoidlist) \vt GetSqZlist({\it layername})}\\
This function returns a trapezoid list derived from objects in the
selection queue on the layer whose name is passed as the argument. 
Labels are ignored, as are subcells unless the layer name is the
special name ``\$\$'', in which case the subcell bounding boxes are
returned.

This function can be called successfully only in physical mode.

%------------------------------------
% 072606
\index{TransformZ function}
\item{(zoidlist) \vt TransformZ({\it zoidlist\/}, {\it refx\/},
  {\it refy\/}, {\it newx\/}, {\it newy\/})}\\
Return a transformed copy of the passed trapezoid list.  The transform
should have been set previously with {\vt SetTransform} or equivalent. 
The original list is not touched and can be closed if no longer
needed.  The function internally converts each input trapezoid to a
polygon, applies the transformation to the polygon coordinates, then
decomposes the polygons into a new trapezoid list, which is returned.
 
The remaining arguments are ``reference'' and ``new'' coordinates,
which provide for translations.  The reference point is the point
about which rotations and mirroring are performed, and is translated
to the new location, if different.

%------------------------------------
% 100408
\index{BloatZ function}
\item{(zoidlist) \vt BloatZ({\it dimen}, {\it zoidlist\/}, {\it mode\/})}\\
This function returns a new zoidlist which is a bloated version of the
zoidlist passed as an argument (similar to the {\cb !bloat} command). 
Edges will be pushed outward or pulled inward by {\it dimen} (positive
values push outward).  The {\it dimen} is given in microns.

The third argument is an integer that specifies the algorithm to use
for bloating.  Giving zero specifies the default algorithm.  See the
description of the {\cb !bloat} command (\ref{bloatcmd}) for
documentation of the algorithms available. 

%------------------------------------
% 010715
\index{ExtentZ function}
\item{(zoidlist) \vt ExtentZ({\it zoidlist\/})}\\
This will return a zoidlist with at most one component:  a rectangle
giving the bounding box of the list given as an argument.  If the
passed list is null, the return is a null list.

%------------------------------------
% 100408
\index{EdgesZ function}
\item{(zoidlist) \vt EdgesZ({\it dimen}, {\it zoidlist\/}, {\it mode\/})}\\
This returns a list of zoids that in some way describe edges in the
zoid list passed.  The {\it dimen} is given in microns.

The {\it mode} is an integer which specifies the algorithm to use to
define the edges.  The values 0--3 are equivalent to the {\vt BloatZ}
function returning edges only, with the four corner fill-in modes.

\begin{description}
\item{\bf mode 0}\\
Provides an edge template as from the {\vt BloatZ} function with
corner fill-in mode 0 (rounded corners).

\item{\bf mode 1}\\
Provides an edge template as from the {\vt BloatZ} function with
corner fill-in mode 1 (flat corners).

\item{\bf mode 2}\\
Provides an edge template as from the {\vt BloatZ} function with
corner fill-in mode 2 (projected corners).

\item{\bf mode 3}\\
Provides an edge template as from the {\vt BloatZ} function with
corner fill-in mode 3 (no corner fill).

\item{\bf mode 4}\\
The zoid list is logically merged into distinct polygons, and a
``halo'' extending outside of the polygon by width {\it dimen}
(positive value taken) is constructed.  The trapezoids describing the
halo are returned.

\item{\bf mode 5}\\
The zoid list is logically merged into distinct polygons, and a wire
object is constructed using each polygon vertex list.  The wire width
is twice the {\it dimen} value passed.  The trapezoid list
representing the wire area is returned.  This may fail and give
strange shapes if the dimensions of a polygon are smaller than half
the wire width.

\item{\bf mode 6}\\
For each zoid in the {\it zoidlist} argument, a new zoid is
constructed from each edge that covers the area within +/- {\it dimen}
normal to the edge.  The list of new zoids is returned.
\end{description}

%------------------------------------
% 100408
\index{ManhattanizeZ function}
\item{(zoidlist) \vt ManhattanizeZ({\it dimen}, {\it zoidlist\/},
  {\it mode\/})}\\
This function returns a new zoidlist which is a Manhattan
approximation of the zoidlist passed as an argument (similar to the
{\cb !manh} command).  The first argument is the minimum rectangle
width or height in microns used to approximate non-Manhattan pieces. 
The third argument is a boolean which specifies which of the two
algorithms to employ.  These algorithms are described with the {\cb
!manh} command, though in this function there is no reassembly into
polygons.

All of the returned trapezoids are rectangles.  The function will fail
if the argument is smaller than 0.01.

%------------------------------------
% 091306
\index{RepartitionZ function}
\item{(zoidlist) \vt RepartitionZ({\it zoidlist\/})}\\
This is a rather obscure function that conditions a list of trapezoids
so that the area covered will be constructed with trapezoids that are
as long (horizontally) as possible.  Logically, this is what would
happen if the initial trapezoid list was converted to distinct
polygons, then split back into trapezoids.

%------------------------------------
% 030204
\index{BoxZ function}
\item{(zoidlist) \vt BoxZ({\it l}, {\it b}, {\it r}, {\it t\/})}\\
This function returns a zoidlist containing a single trapezoid which
represents the box given in the arguments.  The given coordinates are
in microns.  This function never fails.

%------------------------------------
% 030204
\index{ZoidZ function}
\item{(zoidlist) \vt ZoidZ({\it xll}, {\it xlr}, {\it yl},
 {\it xul}, {\it xur}, {\it yu\/})}\\
This function returns a zoidlist containing a single horizontal
trapezoid which represents the horizontal trapezoid given in the
arguments.  The six numbers must represent a non-degenerate figure or
the function will fail.  The given coordinates are in microns.

%------------------------------------
% 030204
\index{ObjectZ function}
\item{(zoidlist) \vt ObjectZ({\it object\_handle} {\it all\/})}\\
This function returns a zoidlist which is generated by fracturing the
outlines of the objects in the {\it object\_handle}.  If {\it all} is
0, only the first object in the list is used.  If {\it all} is
nonzero, all objects in the list are used.  This function will fail if
the first argument is not a handle to an object list.

%------------------------------------
% 030204
\index{ParseLayerExpr function}
\item{(layer\_expr) \vt ParseLayerExpr({\it string\/})}\\
This function returns a variable which contains a parse tree for a
layer expression contained in the string passed as an argument.  The
resulting variable is used to rapidly evaluate the layer expression. 
The return value can not be assigned or otherwise manipulated, and can
only be passed to functions that expect this variable type.  The
function will fail on a parse error in the layer expression.

%------------------------------------
% 030204
\index{EvalLayerExpr function}
\item{(zoidlist) \vt EvalLayerExpr({\it layer\_expr}, {\it zoidlist},
  {\it depth}, {\it isclear\/})}\\
This function evaluates the layer expression passed as the first
argument.  The first argument can be a string containing the layer
expression, or a return from {\vt ParseLayerExpr}.  If the
second argument is nonzero, it is taken as a reference zoidlist.  If
0, the current reference zoidlist (as set with {\vt SetZref}) will be
used.  The third argument is the depth into the cell hierarchy to
process.  This can be an integer, with 0 representing the current cell
only, or a string starting with `{\vt a}' to indicate use of all
levels of the hierarchy.  If {\it isclear} is 0, the returned zoidlist
will represent all areas within the reference where the layer
expression is ``true''.  if {\it isclear} is nonzero, the complement
regions will be returned.  The function will fail on a parse or
evaluation error.

%------------------------------------
% 071415
\index{TestCoverageFull function}
\item{(int) \vt TestCoverageFull({\it layer\_expr}, {\it zoidlist},
  {\it minsize\/})}\\
This function will return an integer value indicating the coverage of
the layer expression given in the first argument over the regions
described in the second argument.  The first argument can be a string
containing a layer expression, or a return from {\vt
ParseLayerExpression}.  If the second argument is 0, the current
reference zoidlist as set with {\vt SetZref} is assumed.  This
defaults to tha area of the current cell.

The third argument is an integer which gives the minimum dimension in
internal units of trapezoids which will be considered in the result. 
Sub-dimensional trapezoids are ignored.  This minimizes false-positive
tests due to ``slivers'' caused by clipping errors in non-Manhattan
geometry.  If the geomentry is known to be Manhattan, 0 can be used. 
If 45's only, 2 is recommended, otherwise 4.  Negative values are
taken as zero.

The function tests each dark-area trapezoid from the layer expression
against the reference zoid list.  It will return immediately on the
first such zoid that is not fully covered by the reference zoid list.

The return value is 0 if there was only one trapezoid from the layer
expression, and it did not overlap the reference zoid list. 
Otherwise, if all layer expression trapezoids were covered by the
reference zoid list, 2 is returned, or 1 if not.  Note that 1 will be
returned if there is no intersection and more than one layer
expression trapezoid.  Use {\vt TestCoveragePartial} to fully
distinguish the not-full case.  The present function is most efficient
for determining when the layer expression dark area is or is not fully
covered.

%------------------------------------
% 071415
\index{TestCoveragePartial function}
\item{(int) \vt TestCoveragePartial({\it layer\_expr}, {\it zoidlist},
  {\it minsize\/})}\\
This function will return an integer value indicating the coverage of
the layer expression given in the first argument over the regions
described in the second argument.  The first argument can be a string
containing a layer expression, or a return from {\vt
ParseLayerExpression}.  If the second argument is 0, the current
reference zoidlist as set with {\vt SetZref} is assumed.  This
defaults to tha area of the current cell.

The third argument is an integer which gives the minimum dimension in
internal units of trapezoids which will be considered in the result. 
Sub-dimensional trapezoids are ignored.  This minimizes false-positive
tests due to ``slivers'' caused by clipping errors in non-Manhattan
geometry.  If the geomentry is known to be Manhattan, 0 can be used. 
If 45's only, 2 is recommended, otherwise 4.  Negative values are
taken as zero.

The function tests each dark-area trapezoid from the layer expression
against the reference zoid list.  It will return immediately on the
first such zoid that is partially covered by the reference zoid list,
of after finding both a fully covered zoid and a fully uncovered zoid.

The return value is 0 if there is no dark area from the layer
expression that intersects the reference zoid list, 2 if the layer
expression dark area falls entirely in the reference zoid list, and 1
if coverage is partial.  This test is a bit expensive but provides
definitive results,

%------------------------------------
% 071415
\index{TestCoverageNone function}
\item{(int) \vt TestCoverageNone({\it layer\_expr}, {\it zoidlist},
  {\it minsize\/})}\\
This function will return an integer value indicating the coverage of
the layer expression given in the first argument over the regions
described in the second argument.  The first argument can be a string
containing a layer expression, or a return from {\vt
ParseLayerExpression}.  If the second argument is 0, the current
reference zoidlist as set with {\vt SetZref} is assumed.  This
defaults to tha area of the current cell.

The third argument is an integer which gives the minimum dimension in
internal units of trapezoids which will be considered in the result. 
Sub-dimensional trapezoids are ignored.  This minimizes false-positive
tests due to ``slivers'' caused by clipping errors in non-Manhattan
geometry.  If the geomentry is known to be Manhattan, 0 can be used. 
If 45's only, 2 is recommended, otherwise 4.  Negative values are
taken as zero.

The function tests each dark-area trapezoid from the layer expression
against the reference zoid list.  It will return immediately on the
first such zoid that is not completely uncovered by the reference zoid
list.

The return value is 0 if there is no dark area from the layer
expression that intersects the reference zoid list, 1 otherwise. 
This test is most efficient when determining whether or not the layer
expression dark area intersects the reference list.

%------------------------------------
% 071415
\index{TestCoverage function}
\item{(int) \vt TestCoverage({\it layer\_expr}, {\it zoidlist},
  {\it testfull\/})}\\
This function is deprecated and should not be used in new scripts. 
The {\vt TestCoverageFull}, {\vt TestCoveragePartial}, and {\vt
TestCoverageNone} functions are replacements.

When the boolean {\it testfull} is true, this function is identical to
{\vt TestCoveragePartial} with a {\it minsize} value of 4.  When {\it
testfull} is false, this function is equivalent to {\vt
TestCoverageNone} again with a {\it minsize} of 4.

%------------------------------------
% 030204
\index{ZtoObjects function}
\item{(object\_handle) \vt ZtoObjects({\it zoidlist},
  {\it lname}, {\it join}, {\it to\_dbase\/})}\\
This function will create a list of objects from a zoidlist.  The
objects will be created on the layer whose name is given in the second
argument, which will be created if it does not already exist.  If this
argument is 0, the current layer will be used.  If the {\it join}
argument is nonzero, the objects created will comprise a minimal set
of polygons that enclose all of the trapezoids.  If the {\it join}
argument is 0, the objects will be have the same geometry as the
individual trapezoids.  If the {\it to\_dbase} argument is nonzero, the
new objects will be added to the database.  Otherwise, the new objects
will be ``copies'' that can be manipulated with other functions that
accept object copies, but they will not appear in the database.  The
function will fail if not called in physical mode, or the layer could
not be created.

%------------------------------------
% 030204
\index{ZtoTempLayer function}
\item{(int) \vt ZtoTempLayer({\it longname}, {\it zoidlist},
  {\it join\/})}\\
This function creates a temporary layer using {\it longname}, and adds
the content of the {\it zoidlist} to the new layer, in the current
cell.  If the temporary layer for {\it longname} exists, it will be
used, with existing geometry untouched.  If {\it join} is nonzero, the
zoidlist will be added as a minimal set of polygons, otherwise each
zoid will be added as a box or polygon.  The function returns 1 on
success, 0 otherwise.  This works in physical mode only.

%------------------------------------
% 030204
\index{ClearTempLayer function}
\item{(int) \vt ClearTempLayer({\it longname\/})}\\
This function will clear all of the objects in the current cell from
the given layer, without saving them in the undo list.  If successful,
1 is returned, otherwise 0 is returned.  This works in physical mode
only.

%------------------------------------
% 121708
\index{ZtoFile function}
\item{(int) \vt ZtoFile({\it filename\/}, {\it zoidlist\/}, {\it ascii\/})}\\
Save the zoidlist in a file, whose name is given in the first
argument.  The zoidlist can be recovered with {\vt ZfromFile}.

There are two file formats available.  If the boolean argument {\it
ascii} is nonzero, a human-readable ASCII text file is produced.  Each
line contains the six numbers that describe a trapezoid, using the
following C-style format string:
\begin{quote} \vt
"yl=\%d yu=\%d ll=\%d ul=\%d lr=\%d ur=\%d"
\end{quote}
The numbers are integer values in internal units (usually 1000 units
per micron).

If the {\it ascii} argument is zero, the file is in OASIS format,
using a single dummy cell (named ``zoidlist'') and layer (``0100''),
and uses only TRAPEZOID and CTRAPEZOID geometry records.  The OASIS
representation is more compact and is the appropriate choice for very
large trapezoid collections.

The function returns 1 if successful, 0 otherwise.

%------------------------------------
% 103104
\index{ZfromFile function}
\item{(zoidlist) \vt ZfromFile({\it filename\/})}\\
Read the file, which was produced by {\vt ZtoFile}, and return the
list of trapezoids it contains.  If an error occurs in reading or an
interrupt is received, this function will fail (halting the script). 
Otherwise a zoidlist will always be returned, but the list may be
empty.

%------------------------------------
% 012711
\index{ReadZfile function}
\item{(int) \vt ReadZfile({\it filename\/})}\\
This will read a trapezoid list file whose name is specified as the
required string argument.  This is an ASCII file consisting of two
types of lines:

\begin{enumerate}
\item{Trapezoid lines, in the ASCII format used by {\vt ZfromFile} and
produced by {\vt ZtoFile}, i.e., in the format:
\begin{quote}
\vt yl=\%d yu=\%d ll=\%d ul=\%d lr=\%d ur=\%d
\end{quote}
}

\item{Layer designation lines in the form:
\begin{quote}
{\vt L} {\it layer\_name}
\end{quote}
The {\it layer\_name} should be an {\Xic}-style name for a 
layer, the layer will be created if it does not exist.
}
\end{enumerate}

When a layer designation line is encountered, the trapezoids that have
been read since the file start or last layer designator are written
into the current cell on the specified layer.  Thus, each block of
trapezoid lines must be followed by a layer designation line for the
trapezoids to be recognized. 

However, if the file contains no layer designation lines, all   
trapezoids will be added to the current cell on the current layer.  

Lines that are not recognized as one of these two forms are ignored.

This function always returns 1.  The function will fail if the file
can not be opened.

%------------------------------------
% 082809
\index{ChdGetZlist function}
\item{(zoidlist) \vt ChdGetZlist({\it chd\_name\/}, {\it cellname\/},
{\it scale\/}, {\it array\/}, {\it clip\/}, {\it all\/})}\\
This function will create and return a trapezoid list created from
objects read through the Cell Hierarchy Digest (CHD) whose access name
is given in the first argument.

See the table in \ref{features} for the features that apply during a
call to this function.  An overall transformation can be set with {\vt
ChdSetFlatReadTransform}, in which case the area given applies in the
``root'' coordinates.

The {\it cellname}, if nonzero, must be the cell name after any
aliasing that was in force when the CHD was created.  If {\it
cellname} is passed 0, the default cell for the CHD is understood. 
This is a cell name configured into the CHD, or the first top-level
cell found in the archive file.

The {\it scale} factor will be applied to all coordinates.  The
accepted range is 0.001 -- 1000.0.

If the {\it array} argument is passed 0, no windowing will be used. 
Otherwise the array should have four components which specify a
rectangle, in microns, in the coordinates of {\it cellname}.  The
values are

\begin{tabular}{ll}
{\it array\/}{\vt [0]} & X left\\
{\it array\/}{\vt [1]} & Y bottom\\
{\it array\/}{\vt [2]} & X right\\
{\it array\/}{\vt [3]} & Y top\\
\end{tabular}

If an array is given, only the objects and subcells needed to render
the window will be processed.

If the boolean value {\it clip} is nonzero and an array is given,
trapezoids will be clipped to the window.  Otherwise no clipping is
done.

If the boolean variable {\it all} is nonzero, the objects in the
hierarchy under {\it cellname} will be transformed and added to the
trapezoid list, i.e., the list will be a flat representation of the
entire hierarchy.  Otherwise, only objects in {\it cellname} are
processed.

\end{description}


\subsection{Operations}

\begin{description}
%------------------------------------
% 110213
\index{Filt function}
\item{(zoidlist) \vt Filt({\it zoids}, {\it lexpr\/})}\\
This function is rather specialized.  First, the trapezoids passed by
the handle in the first argument are separated into groups of
mutually-connected trapezoids.  Each group is like a wire net.  We
throw out the groups that do not intersect with nonzero area the dark
area implied by the layer expression second argument.  The return
value is a handle to a list of the trapezoids that remain.

%------------------------------------
% 030204
\index{GeomAnd function}
\item{(zoidlist) \vt GeomAnd({\it zoids1} [, {\it zoids2\/}])}\\
This function takes either one or two arguments, each of which is
taken as a zoidlist after possible conversion as described in the text
for this section.  If one argument is given, the return is a zoidlist
consisting of the intersection regions between zoids in the argument
list.  If two arguments are given, the return is a list of
intersecting regions between the two argument lists.

%------------------------------------
% 030204
\index{GeomAndNot function}
\item{(zoidlist) \vt GeomAndNot({\it zoids1}, {\it zoids2\/})}\\
This function takes two arguments, each of which is taken as a
zoidlist after possible conversion as described in the text for this
section.  The return is a list of regions covered by the first list
that are not covered by the second.

%------------------------------------
% 030204
\index{GeomCat function}
\item{(zoidlist) \vt GeomCat({\it zoids1} [, ...])}\\
This function takes one or more arguments, each of which is taken as a
zoidlist after possible conversion as described in the text for this
section.  The return is a list of all regions from each of the
arguments.  There is no attempt to clip or merge the returned list.

%------------------------------------
% 030204
\index{GeomNot function}
\item{(zoidlist) \vt GeomNot({\it zoids\/})}\\
This function takes one argument, which is taken as a zoidlist after
possible conversion as described in the text for this section.  The
return is a list of zoids representing the areas of the reference area
not covered by the argument list.

%------------------------------------
% 030204
\index{GeomOr function}
\item{(zoidlist) \vt GeomOr({\it zoids1}, ...)}\\
This function takes one or more arguments, each of which is taken as a
zoidlist after possible conversion as described in the text for this
section.  The return is a list of all regions from each of the
arguments, merged and clipped so that no elements overlap.

%------------------------------------
% 030204
\index{GeomXor}
\item{(zoidlist) \vt GeomXor({\it zoids1} [, {\it zoids2\/}])}\\
This function takes one or two arguments, each of which is taken as a
zoidlist after possible conversion as described in the text for this
section.  If one argument is given, the return is a list of areas
where one and only one zoid from the argument has coverage (note that
this is not exclusive-or, in spite of the function name).  If two
arguments are given, the return is the exclusive-or of the two lists,
i.e., the areas covered by either list but not both.

\end{description}


\subsection{Spatial Parameter Tables}
\label{spt}

\begin{description}
%------------------------------------
% 100508
\index{ReadSPtable function}
\item{(int) \vt ReadSPtable({\it filename\/})}\\
This function reads a specification file for a spatial parameter
table.  A spatial parameter table is a two dimensional array of
floating point values, which can be accessed via x-y coordinate pairs. 
The user can define any number of such tables, each of which is given
a unique identifying keyword.  Tables remain defined until explicitly
destroyed, or until {\vt ClearAll} is called.

The tables are input through a file, which uses the following format:

\begin{quote}
{\it keyword X DX NX Y DY NY}\\
{\it X Y value}\\
{\vt ...}
\end{quote}

Blank lines and lines that begin with punctuation are ignored.
There is one ``header'' line with the following entries:

\begin{description}
\item{\it keyword}\\
Arbitrary word for identification.  An existing database with the same
identifier will be replaced.
\item{\it X}\\
Reference coordinate in microns.
\item{\it DX}\\
Grid spacing in X direction, in microns, must be $>$ 0.
\item{\it NX}\\
Number of grid cells in X direction, must be $>$ 0.
\item{\it Y}\\
Reference coordinate in microns.
\item{\it DY}\\
Grid spacing in Y direction, in microns, must be $>$ 0.
\item{\it NY}\\
Number of grid cells in Y direction, must be $>$ 0.
\end{description}

The header line is followed by data lines that supply a value to the
cells.  The {\it X\/},{\it Y} given in microns specifies the cell.  A
second access to a cell will simply overwrite the data value for that
cell.  Unwritten cells will have a zero value.

The function returns 1 on success, 0 otherwise with an error message
available from the {\vt GetError} function.

%------------------------------------
% 100508
\index{NewSPtable function}
\item{(int) \vt NewSPtable({\it name\/}, {\it x0\/}, {\it dx\/}, {\it nx\/},
  {\it y0\/}, {\it dy\/}, {\it ny\/})}\\
This will create a new, empty spatial parameter table in memory,
replacing any existing table with the same name.  The first argument
is a string giving a short name for the table.  The table origin is at
{\it x0\/}, {\it y0} (in microns).  The unit cell size is given by
{\it dx\/}, {\it dy\/} in microns, and the number of cells along x and
y is {\it nx\/}, {\it ny\/}.

The function returns 1 on success, 0 otherwise, with a message
available from {\vt GetError}.

%------------------------------------
% 100508
\index{WriteSPtable function}
\item{(int) \vt WriteSPtable({\it name\/})}\\
This will write the named spatial parameter table to a file.  The
return value is 1 on success, 0 otherwise, with an error message
available from {\vt GetError}.

%------------------------------------
% 060905
\index{ClearSPtable function}
\item{(int) \vt ClearSPtable({\it name\/})}\\
This will destroy the spatial parameter table whose keyword matches
the string given.  If a numeric 0 ({\vt NULL}) or a null string is
passed, all spatial parameter tables will be destroyed.  The return
value is the number of tables destroyed.

%------------------------------------
% 010509
\index{FindSPtable function}
\item{(int) \vt FindSPtable({\it name\/}, {\it array\/})}\\
This function returns 1 if a spatial parameter table with the given
name exists in memory, 0 otherwise.  The {\it array} is an array of
size 6 or larger, or the constant 0.  If an array name is passed, and
the named table exists, the array is filled in with the following
table parameters:

\begin{tabular}{ll}
{\it array\/}{\vt [0]} & origin x in microns\\
{\it array\/}{\vt [1]} & x spacing in microns\\
{\it array\/}{\vt [2]} & row size\\
{\it array\/}{\vt [3]} & origin y in microns\\
{\it array\/}{\vt [4]} & y spacing in microns\\
{\it array\/}{\vt [5]} & column size\\
\end{tabular}

%------------------------------------
% 100508
\index{GetSPdata function}
\item{(real) \vt GetSPdata({\it name\/}, {\it x\/}, {\it y\/})}\\
This function returns the value from the spatial parameter table keyed
by {\it name\/}, at coordinate {\it x\/},{\it y} given in microns.  If
{\it x\/},{\it y} is out of range, 0 is returned.  The function fails
(halts execution) if the table can't be found.

%------------------------------------
% 100508
\index{SetSPdata function}
\item{(int) \vt SetSPdata({\it name\/}, {\it x\/}, {\it y\/}, {\it value\/})}\\
This function will set the data cell corresponding to {\it x\/},{\it
y} (in microns) of the named spatial parameter table to the {\it
value\/}.  The return value is 1 if successful, 0 if {\it x\/},{\it y}
is out of range, or some other error occurs.  The function fails
(halts execution) if the table can't be found.

\end{description}


\subsection{Polymorphic Flat Database}
\label{specdb}

% 100508
There functions are related to creating and using ``special''
databases.  A special database is a spatially sorted container for
objects or trapezoids (not cell instances or cells), with varying
internal formats.  The following script functions expose this
functionality.

\begin{description}
%------------------------------------
% 082809
\index{CxOpenOdb function}
\item{(int) \vt ChdOpenOdb({\it chd\_name\/}, {\it scale\/}, {\it cellname\/},
  {\it array\/}, {\it clip\/}, {\it dbname})}\\
This function will create a ``special database'' of the objects read
through the Cell Hierarchy Digest (CHD) whose access name is passed as
the first argument.

See the table in \ref{features} for the features that apply during a
call to this function.  An overall transformation can be set with {\vt
ChdSetFlatReadTransform}, in which case the area given applies in the
``root'' coordinates.

The {\it scale} factor will be applied to all coordinates.  The
accepted range is 0.001 -- 1000.0.

The {\it cellname}, if nonzero, must be the cell name after any
aliasing that was in force when the CHD was created.  If {\it
cellname} is passed 0, the default cell for the CHD is understood. 
This is a cell name configured into the CHD, or the first top-level
cell found in the archive file.

The {\it array}, if not 0, is an array of
four values or larger giving a rectangular area of {\it cellname} to
read.  The values are in microns, in order L,B,R,T.  If zero, the
entire cell bounding box is understood.  If the boolean value {\it
clip} is nonzero, objects will be clipped to the array, if given.  The
{\it dbname} is a string which names the database.  This can be any
short name string.  The database can be retrieved or cleared using
this name.

The return value is 1 on success, 0 otherwise, with an error message
likely available from {\vt GetError}.

%------------------------------------
% 082809
\index{CxOpenZdb function}
\index{ZDB database}
\item{(int) \vt ChdOpenZdb({\it chd\_name\/}, {\it scale\/}, {\it cellname\/},
  {\it array\/}, {\it clip\/}, {\it dbname})}\\
This function will create a ``special database'' of the trapezoid
representations of objects read through the Cell Hierarchy Digest
(CHD) whose access name is passed as the first argument.

See the table in \ref{features} for the features that apply during a
call to this function.  An overall transformation can be set with {\vt
ChdSetFlatReadTransform}, in which case the area given applies in the
``root'' coordinates.

The {\it scale} factor will be applied to all coordinates.  The
accepted range is 0.001 -- 1000.0.

The {\it cellname}, if nonzero, must be the cell name after any
aliasing that was in force when the CHD was created.  If {\it
cellname} is passed 0, the default cell for the CHD is understood. 
This is a cell name configured into the CHD, or the first top-level
cell found in the archive file.

The {\it array}, if not 0, is an array of four values or larger giving
a rectangular area of {\it cellname} to read.  The values are in
microns, in order L,B,R,T.  If zero, the entire cell bounding box is
understood.  If the boolean value {\it clip} is nonzero, trapezoids
will be clipped to the array, if given.  The {\it dbname} is a string
which names the database.  This can be any short name string.  The
database can be retrieved or cleared using this name.

The return value is 1 on success, 0 otherwise, with an error message
likely available from {\vt GetError}.

%------------------------------------
% 082809
\index{CxOpenZbdb function}
\index{ZBDB database}
\item{(int) \vt ChdOpenZbdb({\it chd\_name\/}, {\it scale\/}, {\it cellname\/},
  {\it array\/}, {\it dbname}, {\it dx\/}, {\it dy\/},
  {\it bx\/}, {\it by\/})}\\
This function will create a ``special database'' of the trapezoid
representations of objects read through the Cell Hierarchy Digest
(CHD) whose access name is passed as the first argument.  This will
open a database similar to {\vt ChdOpenZdb}, however the trapezoids
will be saved in binned lists. 

See the table in \ref{features} for the features that apply during a
call to this function.  An overall transformation can be set with {\vt
ChdSetFlatReadTransform}, in which case the area given applies in the
``root'' coordinates.

The {\it scale} factor will be applied to all coordinates.  The
accepted range is 0.001 -- 1000.0.

The {\it cellname}, if nonzero, must be the cell name after any
aliasing that was in force when the CHD was created.  If {\it
cellname} is passed 0, the default cell for the CHD is understood. 
This is a cell name configured into the CHD, or the first top-level
cell found in the archive file.

The {\it array}, if not 0, is an array of four values or larger giving
a rectangular area of {\it cellname} to read.  The values are in
microns, in order L,B,R,T.  If zero, the entire cell bounding box is
understood.  The {\it dbname} is a string which names the database. 
This can be any short name string.  The database can be retrieved or
cleared using this name.

The {\it dx\/}, {\it dy} are the grid spacing values for the bins, in
microns.  These values must be positive.  The {\it bx\/}, {\it by} are
non-negative overlap bloat values for the bins.  The actual bins are
bloated by these values in the x and y directions.  The trapezoids
will be clipped to the bins.

The return value is 1 on success, 0 otherwise, with an error message
likely available from {\vt GetError}.

%------------------------------------
% 100508
\index{GetObjectsOdb function}
\item{(object\_handle) \vt GetObjectsOdb({\it dbname\/}, {\it layer\_list\/},
  {\it array\/})}\\
This returns a handle to a list of objects, extracted from a named
database created with {\vt ChdOpenOdb}.  The first argument is a
database name string as given to {\vt ChdOpenOdb}.  This function will
work only with databases produced by that function.

The second argument is a string containing a space-separated list of
layer names, or 0.  Objects for each of the given layers will be
obtained.  Objects on the same layer will be grouped together, with
groups ordered as in the {\it layer\_list}.  If this argument is 0,
all layers will be used, ordered bottom-up as in the layer table.

The third argument is an array, as passed to {\vt ChdOpenOdb}, or 0. 
If 0, all objects for the specified layers in the database will be
retrieved.  Otherwise, only those objects with bounding boxes that
overlap the array rectangle with nonzero area will be retrieved.  The
objects retrieved are copies of the database objects, which are not
affected.

%------------------------------------
% 010509
\index{ListLayersDb function}
\item{(stringlist\_handle) \vt ListLayersDb({\it dbname\/})}\\
This function returns a handle to a list of layer name strings, naming
the layers used in the database.  It applies to all of the database
types.  On error, a scalar 0 is returned.

%------------------------------------
% 071915
\index{GetZlistDb function}
\item{(zoidlist) \vt GetZlistDb({\it dbname\/}, {\it layer\_name\/},
  {\it zoidlist\/})}\\
This returns a zoidlist associated with a layer, extracted from a
named database created with {\vt ChdOpenOdb}, {\vt ChdOpenZdb}, or {\vt
ChdOpenZbdb}.  The first argument is a database name string as given to
{\vt ChdOpenOdb} or equivalent.  The second argument is the associated
layer name.

The third argument is the reference trapezoid list.  If the database
was opened with {\vt ChdOpenOdb} or {\vt ChdOpenZdb}, the returned
zoidlist will be clipped to the reference list.  If the database was
opened with {\vt ChdOpenZbdb}, the trapezoids for the bin containing the
center of the first trapezoid in the reference list will be returned. 
In all cases, the returned trapezoids are copies, the database is not
affected.

See also the {\vt GetZlist} function, which can work similarly.

%------------------------------------
% 010509
\index{GetZlistZbdb function}
\item{(zoidlist) \vt GetZlistZbdb({\it dbname\/}, {\it layer\_name\/},
  {\it nx\/}, {\it ny\/})}\\
Return the zoidlist for the given bin and layer.  This applies only to
databases opened with {\vt ChdOpenZbdb}.  The 0,0 bin is in the lower left
corner.

%------------------------------------
% 100508
\index{DestroyDb function}
\item{(int) \vt DestroyDb({\it dbname\/})}\\
This function will free and clear the special database named in the
argument.  This is the database name as given to {\vt ChdOpenOdb} or
equivalent.  If the argument is 0, then all special databases will be
freed and cleared.  This function always returns 1.

%------------------------------------
% 100408
\index{ShowDb function}
\item{(int) \vt ShowDb({\it dbname\/}, {\it array\/})}\\
This function will pop up a window displaying the area given in the
array of the special database named in {\it dbname}.  The array
argument is in the same format as passed to {\vt ChdOpenOdb} or
equivalent.  If passed 0, the bounding box containing all objects in
the database is understood.  The return value is the window number of
the new window (1--4) or -1 if an error occurred.

\end{description}


\subsection{Named String Tables}
\index{hash tables}
\index{named string tables}

% 020109
This interface provides general purpose string hash tables.  The hash
tables are useful for saving and retrieving a string-keyed integer
value, and for detecting or preventing the occurrence of duplicate
strings in a list.  The hash tables are persistent until explicitly
freed, i.e., they remain in memory after a script completes (if not
destroyed), and can be invoked by subsequent scripts.  Each hash table
is accessed by an arbitrary user-supplied name, and there is no limit
on the number of tables that can be created.

\begin{description}
%------------------------------------
% 020109
\index{FindNameTable function}
\item{(int) \vt FindNameTable({\it tabname\/}, {\it create\/})}\\
This function will create or verify the existence of a named string
hash table.  The named tables are available for use in scripts, for
associating a string with an integer and for efficiently ensuring
uniqueness in a collection of strings.  The named tables persist until
explicitly destroyed.

The {\it tabname} is an arbitrary name token used to access a named
hash table.  This function returns 1 if the named hash table exists, 0
otherwise.  If the boolean argument {\it create} is nonzero, if the
named table does not exist, it will be created, and 1 returned.

%------------------------------------
% 020109
\index{RemoveNameTable function}
\item{(int) \vt RemoveNameTable({\it tabname\/})}\\
This function will destroy a named hash table, as created with {\vt
FindNameTable} in create mode.  It the table exists, it will be
destroyed, and 1 is returned.  If the given name does not match an
existing table, 0 is returned.

%------------------------------------
% 020109
\index{ListNameTables function}
\item{(stringlist\_handle) \vt ListNameTables()}\\
This function returns a handle to a list of names of named hash tables
currently in memory.

%------------------------------------
% 020109
\index{ClearNameTables function}
\item{(int) \vt ClearNameTables()}\\
This functions destroys all named hash tables in memory.

%------------------------------------
% 020109
\index{AddNameToTable function}
\item{(int) \vt AddNameToTable({\it tabname\/}, {\it name\/}, {\it value\/})}\\
This will add a string and associated integer to a named hash table. 
The hash table whose name is given as the first argument must exist in
memory, as created with {\vt FindNameTable} in create mode.  The {\it
name} can be any non-null and non-empty string.  The {\it value} can
be any integer, however, the value -1 is reserved for internal use as
a ``not in table'' indication.

If {\it name} is inserted into the table, 1 is returned.  If {\it
name} already exists in the table, or the table does not exist, 0 is
returned.  The {\it value} is ignored if the {\it name} already exists
in the table, the existing value is not updated.

%------------------------------------
% 020109
\index{RemoveNameFromTable function}
\item{(int) \vt RemoveNameFromTable({\it tabname\/}, {\it name\/})}\\
This will remove the {\it name} string from the named hash table whose
name is given as the first argument.  If the {\it name} string is
found and removed, 1 is returned.  Otherwise, 0 is returned.

%------------------------------------
% 020109
\index{FindNameInTable function}
\item{(int) \vt FindNameInTable({\it tabname\/}, {\it name\/})}\\
This function will return the data value saved with the {\it name}
string in the table whose name is given as the first argument.  If the
table is not found, or the {\it name} string is not found, -1 is
returned.  Otherwise the returned value is that supplied to {\vt
AddNameToTable} for the {\it name} string.  Note that it is a bad
idea to use -1 as a data value.

%------------------------------------
% 020109
\index{ListNamesInTable function}
\item{(stringlist\_handle) \vt ListNamesInTable({\it tabname\/})}\\
This function returns a handle to a list of the strings saved in the
hash table whose name is supplied as the first argument.
\end{description}


%------------------------------------------------------------------------------
\section{Design Rule Checking Functions}
\subsection{DRC}

The following functions relate to the design rule checking subsystem.

\begin{description}
%------------------------------------
% 030204
\index{DRCstate function}
\index{design rules!state}
\item{(int) \vt DRCstate({\it state\/})}\\
This function sets the interactive DRC state, and returns the existing
state.  If the argument is 0, interactive DRC is turned off.  If
nonzero, interactive DRC is turned on.  If greater than 1, error
messages will not pop up.  The return value is the present state,
which is a value of 0--2, similarly interpreted.

%------------------------------------
% 010715
\index{DRCsetLimits function}
\item{(int) \vt DRCsetLimits({\it batch\_cnt}, {\it intr\_cnt},
  {\it intr\_time}, {\it skip\_cells\/})}\\
{\bf Deprecated in favor of DRCsetMaxErrors and similar.}

This function sets the limits used in design rule checking.  Each
argument, if negative, will cause the related value to be unchanged by
the function call.  For the first three arguments, the value ``0'' is
interpreted as ``no limit''.

\begin{description}
\item{\it batch\_cnt}\\
This sets the maximum number of errors to record in batch-mode error
checking.  When this number is reached, the checking is aborted. 
Values 0 -- 100000 are accepted.
\item{\it intr\_cnt}\\
This sets the maximum number of objects tested in interactive DRC. 
The testing aborts when this count is reached.  Values of 0 -- 100000
are accepted.
\item{\it intr\_time}\\
This sets the maximum time allowed for interactive DRC testing.  The
value given is in milliseconds, and values of 0 -- 30000 are accepted.
\item{\it skip\_cells}\\
If nonzero, testing of newly placed, moved, or copied subcells is
skipped in interactive DRC.  If zero, subcells will be tested.  This
can be a lengthly operation.
\end{description}

This function always returns 1.  Out-of-range arguments are set to the
maximum permissible values.

%------------------------------------
% 010715
\index{DRCgetLimits function}
\item{(int) \vt DRCgetLimits({\it array\/})}\\
{\bf Deprecated in favor of DRCgetMaxErrors and similar.}

This function fills the {\it array}, which must have size 4 or larger,
with the current DRC limit values.  These are, in order,

\begin{tabular}{ll}
$[0]$ & The batch error count limit.\\
$[1]$ & The interactive object count limit.\\
$[2]$ & The interactive time limit in milliseconds.\\
$[3]$ & A flag which indicates interactive DRC is skipped for subcells.\\
\end{tabular}

The return value is always 1.  The function fails if the array
argument is bad.

%------------------------------------
% 010715
\index{DRCsetMaxErrors function}
\item{(int) \vt DRCsetMaxErrors({\it value\/})}\\
Set the maximum violation count allowed before a batch DRC run is
terminated.  If set to 0, no limit is imposed.  The value is clipped
to the acceptable range 0 -- 100,000.  If not set, a value 0 (no
limit) is assumed.  The function returns the previous value.

%------------------------------------
% 010715
\index{DRCgetMaxErrors function}
\item{(int) \vt DRCgetMaxErrors()}\\
Returns the maximum violation count before a batch DRC run is
terminated.  If set to 0, no limit is imposed.

%------------------------------------
% 010715
\index{DRCsetInterMaxObjs function}
\item{(int) \vt DRCsetInterMaxObjs({\it value\/})}\\
Set the maximum number of objects tested in interctive DRC.  Further
testing is skipped when this value is reached.  A value of 0 imposes
no limit.  The passed value is clipped to the acceptable range 0 --
100,000, the value used if not set is 1000.  The function returns the
previous setting.

%------------------------------------
% 010715
\index{DRCgetInterMaxObjs function}
\item{(int) \vt DRCgetInterMaxObjs()}\\
Return the maximum number of objects tested in interctive DRC. 
Further testing is skipped when this value is reached.  A value of 0
imposes no limit.

%------------------------------------
% 010715
\index{DRCsetInterMaxTime function}
\item{(int) \vt DRCsetInterMaxTime({\it value\/})}\\
Set the maximum time in milliseconds allowed for interactive DRC
testing after an operation.  The testing will abort after this limit,
returning program control to the user.  If set to 0, no time limit is
imposed.  the passed value is clipped to the acceptable range 0 -
30,000.  If not set, a value of 5000 (5 seconds) is used.  The
function returns the previous value.

%------------------------------------
% 010715
\index{DRCgetInterMaxTime function}
\item{(int) \vt DRCgetInterMaxTime()}\\
Return the maximum time in milliseconds allowed for interactive DRC
testing after an operation.  The testing will abort after this limit,
returning program control to the user.  If set to 0, no time limit is
imposed.

%------------------------------------
% 010715
\index{DRCsetInterMaxErrors function}
\item{(int) \vt DRCsetInterMaxErrors({\it value\/})}\\
Set the maximum number of errors allowed in interactive DRC testing
after an operation.  Further testing is skipped after this count is
reached.  A value of 0 imposes no limit.  The value will be clipped to
the acceptable rnge 0 -- 1000.  If not set, a value of 100 is used. 
The function returns the previous value.

%------------------------------------
% 010715
\index{DRCgetInterMaxErrors function}
\item{(int) \vt DRCgetInterMaxErrors()}\\
Return the maximum number of errors allowed in interactive DRC testing
after an operation.  Further testing is skipped after this count is
reached.  A value of 0 imposes no limit.

%------------------------------------
% 010715
\index{DRCsetInterSkipInst function}
\item{(int) \vt DRCsetInterSkipInst({\it value\/})}\\
If the boolean argument is nonzero, cell instances will not be checked
for violations in interactive DRC.  The test can be lengthly and the
user may want to defer such testing.  The return value is 0 or 1
representing the previous setting.

%------------------------------------
% 010715
\index{DRCgetInterSkipInst function}
\item{(int) \vt DRCgetInterSkipInst()}\\
The return value of this function is 0 or 1 representing whether cell
instances are skipped (if 1) in interactive DRC testing.

%------------------------------------
% 030204
\index{DRCsetLevel function}
\index{design rules!level}
\item{(int) \vt DRCsetLevel({\it level\/})}\\
This function sets the DRC error recording level to the argument.  The
argument is interpreted as follows:

\begin{tabular}{ll}
0 or negative & One error is reported per object.\\
1             & One error of each type is reported per object.\\
2 or larger   & All errors are reported.\\
\end{tabular}

This function always succeeds, and the previous level (0, 1, 2) is
returned.

%------------------------------------
% 030204
\index{DRCgetLevel function}
\item{(int) \vt DRCgetLevel()}\\
This function returns the current error reporting level for design
rule checking.  Possible values are

\begin{tabular}{ll}
0 & One error is reported per object.\\
1 & One error of each type is reported per object.\\
2 & All errors are reported.\\
\end{tabular}

This function always succeeds.

%------------------------------------
% 022309
\index{DRCcheckArea function}
\item{(int) \vt DRCcheckArea({\it array\/}, {\it file\_handle\_or\_name\/})}\\
This function performs batch-mode design rule checking in the current
cell.

The {\it array} argument is an array of size 4 or larger, or 0 can be
passed for this argument.  If an array is passed, it represents a
rectangular area where checking is performed, and the values are in
microns in order L,B,R,T.  If 0 is passed, the entire area of the
current cell is checked.

The second argument can be a file handle opened with the {\vt Open}
function for writing, or the name of a file to open, or an empty
string, or a null string or (equivalently) the scalar 0.  This sets
the destination for error recording.  If the argument is null or 0, a
file will be created in the current directory using the name template
``{\vt drcerror.log.}{\it cellname\/}'', where {\it cellname} is the
current cell.  If an empty string is passed (give {\vt ""} as the
argument), output will go to the error log, and appear in the pop-up
which appears on-screen.  If a string is given, it is taken as a file
name to open.

The function returns an integer, either the number of errors found or
-1 on error.  If -1 is returned, an error message is probably
available from the {\vt GetError} function.

%------------------------------------
% 010815
\index{DRCchdCheckArea function}
\item{(int) \vt DRCchdCheckArea({\it chdname\/}, {\it cellname\/},
  {\it gridsize\/}, {\it array\/}, {\it file\_handle\_or\_name\/},
  {\it flatten})}\\
This function performs a batch-mode DRC of the given top-level cell,
from the Cell Hierarchy Digest (CHD) whose access name is given as the
first argument.  Unlike other DRC commands, this function does not
require that the entire layout be in memory, thus it is theoretically
possible to perform DRC on designs that are too large for available
memory. 

If the given {\it cellname} is null or 0 is passed, the default cell
for the named CHD is assumed.

The checking is performed on the areas of a grid, and only the cells
needed to render the grid area are read into memory temporarily.  The
gridsize argument gives the size of this grid, in microns.  If 0 is
passed, no grid is used, and the entire layout will be read into
memory, as in the normal case.  If a negative value is passed, the
value associated with the {\et DrcPartitionSize} variable is used. 
The chosen grid size should be small enough to avoid page swapping,
but too-small of a grid will lengthen checking time (larger is better
in this regard).  The user can experiment to find a reasonable value
for their designs.  A good starting value might be 400.0 microns.

The {\it array} argument is an array of size 4 or larger, or 0 can be
passed for this argument.  If an array is passed, it represents a
rectangular area where checking is performed, and the values are in
microns in order L,B,R,T.  If 0 is passed, the entire area of the {\it
cellname} is checked.

The {\it file\_handle\_or\_name} argument can be a file handle opened
with the {\vt Open} function for writing, or the name of a file to
open, or an empty or null string or the scalar 0.  This sets the
destination for error recording.  If the argument is null, empty or 0,
a file will be created in the current directory using the name
template ``{\vt drcerror.log.}{\it cellname}'', where {\it cellname}
is the top-level cell being checked.  If a string is given, it is
taken as a file name to open.  There is no provision for sending
output to the on-screen error logger, unlike in the {\vt DRCcheckArea}
function.

If the boolean argument {\it flatten} is true, the geometry will be
flattened as it is read into memory.  This will make life simpler and
faster for the DRC evaluation functions, at the expense of (probably)
much larger memory use.  The user can experiment to find if this
option provides any speed benefit.

The function returns an integer, either the number of errors found or
-1 on error.  If -1 is returned, an error message is probably
available from the {\vt GetError} function.

%------------------------------------
% 030204
\index{DRCcheckObjects function}
\item{(int) \vt DRCcheckObjects({\it file\_handle\/})}\\
This function checks each selected object for design rule violations. 
The {\it file\_handle\/} argument is a file descriptor returned from
the {\vt Open} function, or 0.  If a file descriptor is passed, output
goes to that file, otherwise output goes to the on-screen error
logger.  This function returns the number of errors found.

%------------------------------------
% 030204
\index{DRCregisterExpr function}
\item{(expr\_handle) \vt DRCregisterExpr({\it expr\/})}\\
This function creates and tags a parse tree from the string argument,
which is a layer expression, for later use, and returns a handle to
the expression.  This avoids the overhead of parsing the expression on
each function call.  The returned value is used by other functions
(currently just the two below).

%------------------------------------
% 030204
\index{DRCtestBox function}
\item{(int) \vt DRCtestBox({\it left\/}, {\it bottom\/}, {\it right\/},
 {\it top\/}, {\it expr\_handle\/})}\\
This function tests a rectangular area specified by the first four
arguments for regions where a layer expression is true.  The {\it
expr\_handle\/} argument is the handle of a layer expression returned
by {\vt DRCregisterExpr}.  The returned value is 0 if the expression
is nowhere true, 1 if the expression is true somewhere but not
everywhere, and 2 if the expression is true everywhere in the test
region.

%------------------------------------
% 030204
\index{DRCtestPoly function}
\item{(int) \vt DRCtestPoly({\it num\/}, {\it points\/},
 {\it expr\_handle\/})}\\
This function tests a polygon area for regions where a layer
expression is true.  The first argument is the number of points in the
polygon.  The second argument is the name of an array variable
containing the polygon data.  The polygon data are stored sequentially
as x,y pairs, and the last point must be the same coordinate as the
first.  The length of the vector must be at least two times the value
passed for the first argument.  The {\it expr\_handle\/} argument is
the handle of a layer expression returned by {\vt DRCregisterExpr}. 
The returned value is 0 if the expression is nowhere true, 1 if the
expression is true somewhere but not everywhere, and 2 if the
expression is true everywhere in the test region.

%------------------------------------
% 010715
\index{DRCzList function}
\item{(zoidlist) \vt DRCzList({\it layername\/}, {\it rulename\/},
 {\it index\/}, {\it source\/})}\\
This function will access existing design rule definitions, and use
the associated test region generator to create a new trapezoid list,
which is returned.  For example, in a {\vt MinSpaceTo} rule test, we
construct a ``halo'' around source polygons.  If this halo intersects
any target polygons, a violation would be flagged.  The list of
trapezoids that constitute the halos around the source polygons is the
return of this function.

The first three arguments specify an existing design rule.  The rule
is defined on the layer named in the first argument (a string).  The
type of rule is given as a string in the second argument.  This is the
name of an ``edge'' rule, which uses test regions constructed along
edges to evaluate the rule.  Valid names are the user-defined rules
and

\begin{quote} \vt
MinEdgeLength\\
MaxWidth\\
MinWidth\\
MinSpace\\
MinSpaceTo\\
MinSpaceFrom\\
MinOverlap\\
MinNoOverlap
\end{quote}

The third argument is an integer index which specifies the rule to
choose if there is more than one of the named type assigned to the
layer.  The index is zero based, and indicates the position of the
rule when listed in the window of the {\cb Design Rule Editor} panel
from the {\cb Edit Rules} button in the {\cb DRC Menu}, relative to
and counting only rules of the same type.  The is also the order as
first seen by {\Xic}, as read from the technology file or created
interactively.

The fourth argument is a ``zoidlist'' as is taken by many of the
functions that deal with layer expressions and trapezoid lists, as
explained for those functions (see \ref{zoidlistarg}).  If the value
passed is a scalar 0, then geometry is obtained from the full
hierarchy of the current cell.  In this case, the created test areas
will be identical to those created during a DRC run.  It may be
instructive to create a visible layer from this result, to see where
testing is being performed.

If the argument instead passes trapezoids, the result will be creation
of the test regions as if the passed trapezoids were features on the
layer or {\vt Region} associated with the rule.  The actual features
on the layer are ignored.

The function will fail and halt execution if the first three
arguments do not indicate an existing design rule definition.

%------------------------------------
% 010715
\index{DRCzListEx function}
\item{(zoidlist) \vt DRCzListEx({\it source\/}, {\it target\/},
 {\it inside\/}, {\it outside\/}, {\it incode\/}, {\it outcode\/},
 {\it dimen\/})}\\
This is similar to {\vt DRCzList}, however it does not reference an
existing rule.  Instead, it accesses the test area generator directly,
effectively creating an internal, temporary rule.

The first argument is a ``zoidlist'' as expected by other functions
that accept this argument type (see \ref{zoidlistarg}).  Unlike for
{\vt DRCzList}, this argument can not be zero or null.

The second argument is a string providing a target layer expression. 
This may be scalar 0 or null.  The {\it inside} and {\it outside}
arguments are strings providing layer expressions that will select
which parts of an edge will be used for test area generation.  The
{\it inside} is the area inside the figure at the edge, and {\it
outside} is just outside of the figure along the edge.  Either can be
null or scalar 0.

The {\it incode} and {\it outcode} are integer values 0--2 which
indicate how the inside and outside expressions are to be interpreted
with regard to defining the ``active'' part of the edge.  The values
have the following interpretations:

\begin{quote}
\begin{tabular}{ll}
0 & Don't care, the value expression is ignored.\\
1 & The active parts of the edge are where the expression is clear.\\
2 & The active parts of the edge are where the expression is dark.\\
\end{tabular}
\end{quote}

The {\it dimen} is the width of the test area, in microns.  It must be
a positive real number.

If all goes well, a trapezoid list reprseenting the effective test
areas is returned.

\end{description}


%------------------------------------------------------------------------------
\section{Extraction Functions}
\subsection{Menu Commands}

The functions in this section provide an interface to the extraction
system.  This interface is by no means complete, but it allows many
common operations to be performed and allows traversal and information
retrieval.

\begin{description}
%------------------------------------
% 061516
\index{DumpPhysNetlist function}
\item{(int) \vt DumpPhysNetlist({\it filename}, {\it depth},
 {\it modestring\/}, {\it names\/})}\\
This function dumps a netlist file extracted from the physical part of
the database, much like the {\cb Dump Phys Netlist} command in the
{\cb Extract Menu}.  The {\it filename} argument is a file name which
will receive the output.  If null or empty, the file will be the base
name of the current cell with ``{\vt .physnet}'' appended.  The {\it
depth} argument specifies the depth of the hierarchy to process.  If
an integer, 0 represents the current cell only, 1 includes the first
level subcells, etc.  A negative integer specifies to process the
entire hierarchy.  This argument can also be a string beginning with
the letter `{\vt a}', which will process all levels of the hierarchy.

The third argument is a string, consisting of characters from the
table below, which set the mode of the command.  These are analogous
to the check boxes that appear with the {\cb Dump Phys Netlist}
command.  If a character does not appear in the string, that option is
turned off.  If it appears in lower case, the option is turned on, and
if it appears in upper case, the option will be set by the present
value of the corresponding {\cb !set} variable.  The characters can
appear in any order.

\begin{tabular}{|l|l|l|} \hline
{\kb character} & {\kb option} & {\kb corresponding variable}\\ \hline
{\vt n} & {\cb net} & {\et PnetNet}\\ \hline
{\vt d} & {\cb devs} & {\et PnetDevs}\\ \hline
{\vt s} & {\cb spice} & {\et PnetSpice}\\ \hline
{\vt b} & {\cb list bottom-up} & {\et PnetBottomUp}\\ \hline
{\vt g} & {\cb show geometry} & {\et PnetShowGeometry}\\ \hline
{\vt c} & {\cb include wire cap} & {\et PnetIncludeWireCap}\\ \hline
{\vt a} & {\cb list all cells} & {\et PnetListAll}\\ \hline
{\vt l} & {\cb ignore labels} & {\et PnetNoLabels}\\ \hline
\end{tabular}

The final argument, if not null or empty, contains a space-separated
list of physical format names, each of which must match a {\vt
PnetFormat} name in the format library file, or option names from the
table above.  The names that contain white space should be
double-quoted.

For each cell, a field in the output is generated for each format
choice implicit in the {\it modestring} or given in the {\it names\/}. 
In most cases, only one format is probably wanted.  The option text in
the table above can also be included in the {\it names\/}, which is
equivalent to giving the corresponding lower-case letter in the {\it
modestring\/}.  The {\it modestring} setting will have precedence if
there is a conflict.  If both the {\it modestring} and the {\it names}
string are empty or null, an effective mode string consisting of all
of the upper-case option letters is used.

Example:  print a SPICE file
\begin{quote}
{\vt DumpPhysNetlist("myfile.cir", "a", "s", 0)}\\
or\\
{\vt DumpPhysNetlist("myfile.cir", "a", 0, "spice")}
\end{quote}

If the function succeeds, 1 is returned, otherwise 0 is returned.

%------------------------------------
% 061516
\index{DumpElecNetlist function}
\item{(int) \vt DumpElecNetlist({\it filename}, {\it depth},
 {\it modestring\/}, {\it names\/})}\\
This function dumps a netlist file extracted from the electrical part
of the database, much like the {\cb Dump Elec Netlist} command in the
{\cb Extract Menu}.  The {\it filename} argument is a file name which
will receive the output.  If null or empty, the file will be the base
name of the current cell with ``{\vt .elecnet}'' appended.  The {\it
depth} argument specifies the depth of the hierarchy to process.  If
an integer, 0 represents the current cell only, 1 includes the first
level subcells, etc.  A negative integer specifies to process the
entire hierarchy.  This argument can also be a string beginning with
the letter `{\vt a}', which will process all levels of the hierarchy.

The third argument is a string, consisting of characters from the
table below, which set the mode of the command.  These are analogous
to the check boxes that appear with the {\cb Dump Elec Netlist}
command.  If a character does not appear in the string, that option is
turned off.  If it appears in lower case, the option is turned on, and
if it appears in upper case, the option will be set by the present
value of the corresponding {\cb !set} variable.  The characters can
appear in any order.

\begin{tabular}{|l|l|l|} \hline
{\kb character} & {\kb option} & {\kb corresponding variable}\\ \hline
{\vt n} & {\cb net} & {\et EnetNet}\\ \hline
{\vt s} & {\cb spice} & {\et EnetSpice}\\ \hline
{\vt b} & {\cb list bottom-up} & {\et EnetBottomUp}\\ \hline
\end{tabular}

The final argument, if not null or empty, contains a space-separated
list of electrical format names, each of which must match an {\vt
EnetFormat} name in the format library file, or option names from the
table above.  The names that contain white space should be double
quoted.

For each cell, a field in the output is generated for each format
choice implicit in the {\it modestring} or given in the {\it names}. 
In most cases, only one format is probably wanted.  The option text in
the table above can also be included in the {\it names}, which is
equivalent to giving the corresponding lower-case letter in the {\it
modestring\/}.  The {\it modestring} setting will have precedence if
there is a conflict.  If both the {\it modestring} and the {\it names}
string are empty or null, an effective mode string consisting of all
of the upper-case option letters is used.

If the function succeeds, 1 is returned, otherwise 0 is returned.

%------------------------------------
% 030204
\index{SourceSpice function}
\item{(int) \vt SourceSpice({\it filename}, {\it modestring\/})}\\
This function will parse a SPICE file, adding to or updating the
electrical part of the database with the devices and subcircuits
found.  This is equivalent to the {\cb Source SPICE} command in the
{\cb Extract Menu}.  The first argument is a path to the SPICE file to
process.

The final argument is a string, consisting of characters from the
table below, which set the mode of the command.  These are analogous
to the check boxes that appear with the {\cb Source SPICE} command. 
If a character does not appear in the string, that option is turned
off.  If it appears in lower case, the option is turned on, and if it
appears in upper case, the option will be set by the present value of
the corresponding {\cb !set} variable.  The characters can appear in
any order.  If the string is empty or null, all options will be set by
the corresponding variables.

\begin{tabular}{|l|l|l|} \hline
{\kb character} & {\kb option} & {\kb corresponding variable}\\ \hline
{\vt a} & {\cb all devs} & {\et SourceAllDevs}\\ \hline
{\vt r} & {\cb create} & {\et SourceCreate}\\ \hline
{\vt l} & {\cb clear} & {\et SourceClear}\\ \hline
\end{tabular}

If the operation succeeds, 1 is returned, otherwise 0 is returned.

%------------------------------------
% 113009
\index{ExtractAndSet function}
\item{(int) \vt ExtractAndSet({\it depth}, {\it modestring\/})}\\
This function performs extraction on the physical part of the
database, updating the electrical part.  This is equivalent to the
{\cb Source Physical} command in the {\cb Extract Menu}.  The first
argument indicates the depth of the hierarchy to process.  This can be
an integer:  0 means process the current cell only, 1 means process
the current cell plus the subcells, etc., and a negative integer sets
the depth to process the entire hierarchy.  This argument can also be
a string starting with `{\vt a}' such as ``{\vt a}'' or ``{\vt all}''
which indicates to process the entire hierarchy.

The final argument is a string, consisting of characters from the
table below, which set the mode of the command.  These are analogous
to the check boxes that appear with the {\cb Source Physical} command. 
If a character does not appear in the string, that option is turned
off.  If it appears in lower case, the option is turned on, and if it
appears in upper case, the option will be set by the present value of
the corresponding {\cb !set} variable.  The characters can appear in
any order.  If the string is empty or null, all options will be set by
the corresponding variables.

\begin{tabular}{|l|l|l|} \hline
{\kb character} & {\kb option} & {\kb corresponding variable}\\ \hline
{\vt a} & {\cb all devs} & {\et NoExsetAllDevs}\\ \hline
{\vt r} & {\cb create} & {\et NoExsetCreate}\\ \hline
{\vt l} & {\cb clear} & {\et ExsetClear}\\ \hline
{\vt c} & {\cb include wire cap} & {\et ExsetIncludeWireCap}\\ \hline
{\vt n} & {\cb ignore labels} & {\et ExsetNoLabels}\\ \hline
\end{tabular}

If the operation succeeds, 1 is returned, otherwise 0 is returned. 
This function does not redraw the windows.

%------------------------------------
% 051809
\index{FindPath function}
\item{(object\_handle) \vt FindPath({\it x\/}, {\it y\/}, {\it depth},
 {\it use\_extract\/})}\\
This function returns a handle to a list of copies of physical
conducting objects in a wire net.  The {\it x},{\it y} point (microns,
in the physical part of the current cell) should intersect a
conducting object, and the list will consist of this object plus
connected objects.  The {\it depth} argument is an integer or a string
beginning with ``{\vt a}'' (for "all") which gives the hierarchy
search depth.  Only objects in cells to this depth will be considered
for addition to the list (0 means objects in the current cell only). 
If the boolean value {\it use\_extract} is nonzero, the main
extraction functions will be used to determine the connectivity.  If
the value is zero, the connectivity is established through geometry. 
This is similar to the {\cb Select Path} and {\cb "Quick" Path} modes
available in the {\cb Path Selection Control} panel.

The return value is a handle to a list of object copies, or 0 if
no objects are found.

%------------------------------------
% 052409
\index{FindPathOfGroup function}
\item{(object\_handle) \vt FindPathOfGroup({\it groupnum\/}, {\it depth})}\\
This function returns a handle to a list of copies of physical
conducting objects in the group number from the current cell given, to
the given depth.  The depth argument is an integer or a string
beginning with ``{\vt a}'' (for ``{\vt all}'') which gives the
hierarchy search depth.  Only objects in cells to this depth will be
considered for addition to the list (0 means objects in the current
cell only).

The function will fail (halt the script) on a major error.  If the
group number is out of range, or a ``minor'' error occurs, the
function will return a scalar 0, and an error message should be
available from {\vt GetError}.

Otherwise, the return value is a handle to a list of object copies, or
the list may be empty if the group has no physical objects.

\end{description}


\subsection{Terminals}

Here, a ``terminal'' refers to a {\et node} property of an electrical
device or circuit.  Both masters and instances have such properties,
though their internal structure differs a bit.  A ``terminal\_handle''
is a handle to a list of terminals, that can be passed to functions
that provide information about or operate on node properties.

In the next section, we introduce ``physical terminals'', which are
different objects.  A physical terminal is a data structure that
stores information about the physical aspects of a terminal, including
its location in the layout, an object that it may be bound to, and the
associated layer.  If a schematic has a layout and has been
associated, then each terminal (node property) has a pointer to the
corresponding physical terminal, and vice-versa.  Thus, in general
either object can be used to reference data.  In fact, in most of the
functions in this section and the next, the "handle" argument can be a
handle to either a node property or physical terminal.

However, cells that are electrical-only will not have physical
terminals, and similarly, a layout without a corresponding schematic
will lack node properties.  In these cases, only the existing object
type can be used.

\begin{description}
%------------------------------------
% 041113
\index{ListTerminals function}
\item{(terminal\_handle) \vt ListTerminals()}\\
Return a handle containing a list of the connection terminals of the
current cell.  These correspond to the normal contact terminals as
would be defined with the {\cb subct} command, as represented by {\et
node} properties of the electrical cell view.  On success, a handle is
returned containing the terminal list.  If there are no terminals
defined or some other error occurs, a scalar 0 is returned.

%------------------------------------
% 041113
\index{FindTerminal function}
\item{(terminal\_handle) \vt FindTerminal({\it name\/}, {\it index\/},
 {\it use\_e\/}, {\it xe\/}, {\it ye\/}, {\it use\_p\/}, {\it xp\/},
 {\it yp\/})}\\
This function will return a handle referencing a single terminal, if
one can be found among the current cell contact terminals that matches
the arguments.  The arguments specify parameters, any of which can be
ignored.  The non-ignored parameters must all match.

The {\it name} can be a string that will match an applied terminal
name (not a default name generated by {\Xic}).  The argument will be
ignored if a scalar 0 or null or empty string is passed.

The {\it index} is the terminal order index, or -1 if the parameter is
to be ignored.  This is the number that is shown within the terminal
box in the {\cb subct} command.

If {\it use\_e} is a nonzero value, the next two arguments are taken
as a location in the electrical drawing.  These are specified in
fictitious ``microns'' which represent 1000 internal units.  These are
the numbers displayed in the coordinate readout area while a schematic
is being edited.  A location match will depend of whether the
electrical cell is symbolic or not.  If symbolic, a location match to
any of the placement locations will count as a match (terminals can
have more than one ``hot spot'' in the symbolic display).  If {\it
use\_e} is 0, the two arguments that follow are ignored and can be any
numeric values.

Similarly, if {\it use\_p} is nonzero, the next two arguments
represent a coordinate in the layout, given in (real) microns.  If a
physical terminal is placed at the given location, a match will be
indicated.  If {\it use\_p} is zero, the two arguments that follow are
ignored, and can be set to any numeric values.

The arguments should provide at least one matchable parameter. 
Internally, the list of terminals is scanned, and the first matching
terminal found is returned, referenced by a handle.  If no terminals
match, a scalar zero is returned.

%------------------------------------
% 041113
\index{CreateTerminal function}
\item{(terminal\_handle) \vt CreateTerminal({\it name\/}, {\it x\/}, {\it y\/},
 {\it termtype\/})}\\
This function will create a new terminal in the schematic of the
current cell.  If a {\it name} string is passed, the terminal will be
given that name.  If this argument is a scalar 0 or a null or empty
string, the terminal will not have an assigned name but will use an
internally generated name.  The terminal will be placed at the
location indicated by the {\it x} and {\it y} arguments, which are in
fictitious ``microns'' representing 1000 database units.  These are
the same coordinates as displayed in the coordinate readout while a
schematic is being edited.

The {\it termtype} argument can be a scalar integer or a keyword, from
the list below.  This will assign a type to the terminal.  The type is
not used by {\Xic}, but this facility may be useful to the user.

\begin{quote}
\begin{tabular}{ll}
0 & {\vt input}\\
1 & {\vt output}\\
2 & {\vt inout}\\
3 & {\vt tristate}\\
4 & {\vt clock}\\
5 & {\vt outclock}\\
6 & {\vt supply}\\
7 & {\vt outsupply}\\
8 & {\vt ground}\\
\end{tabular}
\end{quote}

Keyword matching is case-insensitive.  If the argument is not
recognized, and the default ``{\vt input}'' will be used.

The function returns a handle that references the new terminal on
success, or a scalar zero otherwise.

%------------------------------------
% 041213
\index{DestroyTerminal function}
\item{(int) \vt DestroyTerminal({\it thandle\/})}\\
This function will destroy the terminal referenced by the passed
handle, and will close the handle.  This destroys the terminal, which
is actually a {\et node} property of the electrical current cell, and
the linkage into the physical layout, if any.  If a terminal was
destroyed, value one is returned, or zero on error.

%------------------------------------
% 060716
\index{GetTerminalName function}
\item{(string) \vt GetTerminalName({\it thandle\/})}\\
Return a string containing the name of the terminal or physical
terminal referenced by the handle passed as an argument.  Both objects
have name fields that track.  However, if no name was assigned, for a
terminal a default name generated by {\Xic} is returned, whereas the
return from a physical terminal will be null.

%------------------------------------
% 060716
\index{SetTerminalName function}
\item{(int) \vt SetTerminalName({\it thandle\/}, {\it name\/})}\\
The first argument is a handle that references a terminal or physical
terminal.  The second argument is a string which gives a name to
apply.  It can also be a scalar 0, or if null or empty any existing
assigned name will be removed.  Both terminals and physical terminals
have names that track, this will change both, when both objects exist. 
The return value is one on success, zero if error.

%------------------------------------
% 060716
\index{GetTerminalType function}
\item{(int) \vt GetTerminalType({\it thandle\/})}\\
Return a type code for the terminal referenced by the handle passed as
an argument, which can also be a handle to the corresponding physical
terminal.  A non-negative return represents success.  The code
represents the terminal type set by the user.  The terminal type is
not used by {\Xic}, but is available for user applications.  The
defined types are listed below.  The default is type 0.

\begin{quote}
\begin{tabular}{ll}
0 & {\vt input}\\
1 & {\vt output}\\
2 & {\vt inout}\\
3 & {\vt tristate}\\
4 & {\vt clock}\\
5 & {\vt outclock}\\
6 & {\vt supply}\\
7 & {\vt outsupply}\\
8 & {\vt ground}\\
\end{tabular}
\end{quote}

%------------------------------------
% 060716
\index{SetTerminalType function}
\item{(int) \vt SetTerminalType({\it thandle\/}, {\it termtype\/})}\\
This function will apply a terminal type to the terminal referenced by
the handle passed as the first argument, which can also be a handle to
the corresponding physical terminal.  The second argument is either an
integer, or a string keyword, from the list below.

\begin{quote}
\begin{tabular}{ll}
0 & {\vt input}\\
1 & {\vt output}\\
2 & {\vt inout}\\
3 & {\vt tristate}\\
4 & {\vt clock}\\
5 & {\vt outclock}\\
6 & {\vt supply}\\
7 & {\vt outsupply}\\
8 & {\vt ground}\\
\end{tabular}
\end{quote}

The function returns one if the type is set successfully, zero
otherwise.

%------------------------------------
% 060716
\index{GetTerminalFlags function}
\item{(int) \vt GetTerminalFlags({\it thandle\/})}\\
Return the flags for the terminal referenced by the handle passed as
an argument, which can also be a handle to the corresponding physical
terminal.  The return value is an integer with bits representing flags
as listed in the table below.  On error, the return value is -1.

\begin{description}
\item{{\vt -x1} ({\vt BYNAME})}\\
The terminal makes connections in the schematic by name rather   
than by location.   
\item{{\vt 0x2} ({\vt VIRTUAL})}\\
No longer used, reserved.
\item{{\vt 0x4} ({\vt FIXED})}\\
The physical terminal has been placed by the user, and {\Xic}
should never move it.   
\item{{\vt 0x8} ({\vt SCINVIS})}\\
The electrical terminal will not be shown in schematics.   
\item{{\vt 0x10} ({\vt SYINVIS})}\\
The electrical terminal will not be shown in the symbol.   
\item{{\vt 0x100} ({\vt UNINIT})}\\
The terminal is not initialized (internal).
\item{{\vt 0x200} ({\vt LOCSET})}\\
The physical terminal location has not been set (internal).   
\item{{\vt 0x400} ({\vt POINTS})}\\
Set when the terminal has multiple hot-spots.
\item{{\vt 0x800} ({\vt NOPHYS})}\\
Set if the terminal has no physical implementation, such as a
temperature node.  Such terminals have no physical terminals.
\end{description}

%------------------------------------
% 060716
\index{SetTerminalFlags function}
\item{(int) \vt SetTerminalFlags({\it thandle\/}, {\it flags\/})}\\
This will set the first five flags listed for {\vt GetTerminalFlags}
in the terminal referenced by the first argument, which can also be a
handle to the corresponding physical terminal.  All but the five least
significant bits in the {\it flags} integer are ignored.  The bits
that are set will set the corresponding flag in the terminal, unset
bits are ignored.  The value one is returned on success, zero
otherwise.

%------------------------------------
% 060716
\index{UnsetTerminalFlags function}
\item{(int) \vt UnsetTerminalFlags({\it thandle\/}, {\it flags\/})}\\
This will unset the first five flags listed for {\vt GetTerminalFlags}
in the terminal referenced by the first argument, which can also be a
handle to the corresponding physical terminal.  All but the five least
significant bits in the {\it flags} integer are ignored.  The bits
that are set will unset the corresponding flag in the terminal, unset
bits are ignored.  The value one is returned on success, zero
otherwise.

%------------------------------------
% 060716
\index{GetElecTerminalLoc function}
\item{(int) \vt GetElecTerminalLoc({\it thandle\/}, {\it index\/},
 {\it array\/})}\\
This will return terminal locations in the electrical schematic, of
the terminal referenced by the first argument.  This argument can also
be a handle to the corresponding physical terminal.  The return is
dependent on whether the electrical cell is symbolic or not.  Values
for {\it x} and {\it y} are returned in the {\it array}, which must
have size two or larger.  The returned values are in fictitions
``microns'' that correspond to 1000 database units.  This is the same
coordinate system indicated by the coordinate readout when editing a
schematic.

If the electrical cell is not symbolic, the integer {\it index}
argument must be zero, and the terminal location in the schematic is
returned.

If the electrical cell is symbolic, there can be arbitrarily many
``copies'' of the terminal, representing multiple ``hot spots'' where
the terminal can make connections.  The {\it index} argument is a
0-based index for these locations.  To get all of the locations, one
should call this function repeatedly while incrementing the index from
zero.  A return value of zero indicates that the index is out of range
(or some error occurred).  A return value of one indicates success,
with the array containing the location.

%------------------------------------
% 060716
\index{SetElecTerminalLoc function}
\item{(int) \vt SetElecTerminalLoc({\it thandle\/}, {\it x\/}, {\it y\/})}\\
This function specifies a location for the terminal referenced by the
first argument, for use in electrical mode.  The {\it x} and {\it y}
are coordinates in fictitions ``microns'' which are 1000 database
units.  This is the same coordinate system used in the coordinate
readout when editing a schematic.  As for most of these functions, the
first argument can also be a handle to the corresponding physical
terminal.
  
The function behaves differently depending on whether the electrical
current cell is symbolic or not.  If the electrical current cell not
symbolic, the passed coordinates set the terminal location within the
schematic.  Otherwise, in symbolic mode, there can be arbitrarily many
locations set.  The function will add the passed location to the list
of locations for the terminal, if it is not already using the
location.

The function returns one on success, zero otherwise.

%------------------------------------
% 060716
\index{ClearElecTerminalLoc function}
\item{(int) \vt ClearElecTerminalLoc({\it thandle\/}, {\it x\/}, {\it y\/})}\\
This function applies only when the electric current cell is in
symbolic mode.  When true, a terminal may be displayed in arbitrarily
many locations, representing different possible connection points. 
The {\it x} and {\it y} are coordinates in fictitions ``microns''
which are 1000 database units.  This is the same coordinate system
used in the coordinate readout when editing a schematic.  If the
coordinates match a hot spot of the terminal, that location is
deleted.

It is not possible to delete the last location, there is always at
least one active location.  Calling this function when the electrical
current cell is not symbolic has no effect.  The function returns one
on success, zero if error.

As for most of these functions, the first argument can also be a
handle to the corresponding physical terminal.
\end{description}


\subsection{Physical Terminals}

As noted in the description in the previous section, physical
terminals are a separate data structure that save layout information
about the terminal, such as effective location, the layer attached to,
or an object attached to.  When a schematic exists and has been
associated, the physical terminal and the electrical node property are
linked, so access to one automatically provides access to the other. 
Thus, most of the the functions in this section that access physical
terminals will also take a handle to a regular terminal equivalently,
as did the functions in the previous section.  However, if one data
type does not exist, for the function to succeed, the exissting data
type must be passed, and it must contain the data to be accessed.

Physical terminals that correspond to cell connection points are
stored with the physical data, and are therefor potentially available
when there is no schematic.  Most commonly, however, they are created
upon reading the electrical data for a cell.

\begin{description}
% 060716
\index{ListPhysTerminals function}
\item{(physterm\_handle) \vt ListPhysTerminals()}\\
This returns a handle to a list of physical terminal structures that
correspond to the cell connection points, as obtained from the
physical part of the current cell.

%------------------------------------
% 061016
\index{FindPhysTerminal function}
\item{(physterm\_handle) \vt FindPhysTerminal({\it name\/},
 {\it use\_p\/}, {\it xp\/}, {\it yp\/})}\\
This attempts to find a physical terminal structure by name or
location.  If a name is given, i.e., the argument is not null or 0,
then it will match the name of the terminal returned.  If the boolean
{\it use\_p} is nonzero (true), then the coordinates {\it xp} and {\it
yp\/}, given in microns, will match the placement location of the
returned terminal.  If both name and coordinates are given, both must
match.

An empty handle (scalar 0) is returned if there is no matching
physical terminal found.

%------------------------------------
% 041113
\index{CreatePhysTerminal function}
\item{(int) \vt CreatePhysTerminal({\it thandle\/}, {\it x\/}, {\it y\/},
 {\it layer\/})}\\
As created, (electrical) terminals do not contain the data structures
necessary for a corresponding terminal in the physical layout.  This
is fine as-is, if the user is intending to only work with a schematic,
or if the terminal does not have an actual physical counterpart. 
However, in general one must create the physical terminal.

This function will create a new physical terminal, if one of the same
name does not currently exist.  The first argument can be a handle to
a terminal (electrical node) or a string giving a name.  In the first
case, the new physical terminal is created, given the name of the
electrical terminal, and the linkage established.  In the second case,
which does not require the existance of the electrical schematic, the
physical terminal is created under the given name, and saved in the
physical data.  It will be linked to corresponding electrical data
during association, when possible.

The {\it x} and {\it y} give the initial terminal location in the
layout in microns.  The layer argument can be scalar 0, which is
ignored, or the name of a layer.  The layer must have the {\et
Routing} keyword applied.  If given, this will set the layer hint for
the new terminal.

The return value is 1 on success, 0 otherwise.  It is not an error if
the physical terminal already exists, the function will return 1 and
perform no other operation in that case.

%------------------------------------
% 061016
\index{HasPhysTerminal function}
\item{(int) \vt HasPhysTerminal({\it thandle\/})}\\
This function returns 1 if the (electrical) terminal referenced by the
handle argument has a physical terminal link, 0 if no link has been
assigned.  On error, a value -1 is returned.

%------------------------------------
% 061016
\index{DestroyPhysTerminal function}
\item{(int) \vt DestroyPhysTerminal({\it thandle\/})}\\
This will unlink and destroy the physical terminal data structure that
maintains the terminal linkage into the physical layout, if any.  The
argument can be a handle to the corresponding electrical terminal, or
to the physical terminal itself.  In the latter case, the passed
handle will be closed.  The electrical terminal (if any) will still be
valid, as will its handle if that was passed.  The function returns
one on success, zero if an error occurs.

%------------------------------------
% 061016
\index{GetPhysTerminalLoc function}
\item{(int) \vt GetPhysTerminalLoc({\it thandle\/}, {\it array\/})}\\
Return the layout location for the physical terminal referenced by the
handle passed as an argument.  The first argument can alternatively be
a handle to the corresponding electrical terminal.  The second
argument is an array of size two or larger which will receive the x-y
coordinate, in microns.  The function returns one on success, zero
otherwise.

%------------------------------------
% 061016
\index{SetPhysTerminalLoc function}
\item{(int) \vt SetPhysTerminalLoc({\it thandle\/}, {\it x\/}, {\it y\/})}\\
Set the location of the physical terminal referenced by the first
argument to the layout coordinate given, in microns.  The first
argument can also be a handle to the corresponding electrical
terminal.  Generally, physical terminal locations are set by {\Xic},
using extraction results.  However, this may fail, requiring that the
user provide a location for one or more terminals.  Terminals that
have been placed by the user (using this function) will by default
remain fixed in the location.  The function returns one on success,
zero if an error occurs.

%------------------------------------
% 061016
\index{GetPhysTerminalLayer function}
\item{(string) \vt GetPhysTerminalLayer({\it thandle\/})}\\
Return a string containing the layer name for the physical terminal
referenced by the handle passed as an argument.  A handle to the
corresponding electrical terminal is also accepted.  Non-virtual
physical terminals are associated with an object on a {\et Routing}
layer.  A null string is returned if there is no associated layer.

%------------------------------------
% 060716
\index{SetPhysTerminalLayer function}
\item{(int) \vt SetPhysTerminalLayer({\it thandle\/}, {\it layer\/})}\\
This function will set the associated layer hint on the physical
terminal referenced by the handle passed as the first argument.  A
handle to the corresponding electrical terminal is also accepted.  If
the second argument is the name of a physical layer which has the {\et
Routing} keyword set, the terminal hint layer will be set to that
layer.  If the second argument is a scalar 0, or a null or empty
string, any existing hint layer will be removed.  The function returns
one on success, zero otherwise.

%------------------------------------
% 061016
\index{GetPhysTerminalGroup function}
\item{(int) \vt GetPhysTerminalGroup({\it thandle\/})}\\
This function will return the conductor group number to which the
physical terminal referenced by the argument is assigned.  A handle to
the corresponding electrical terminal is also accepted.  The group
assignment is made during extraction and association.  The return
value is a non-negative integer on success, or -1 if
extraction/association has not been run (or been reverted), or -2 if
some error occurred.

%------------------------------------
% 061016
\index{GetPhysTerminalObject function}
\item{(object\_handle) \vt GetPhysTerminalObject({\it thandle\/})}\\
Return a handle to a physical object that is associated with the
physical terminal referenced by the handle passed as an argument.  A
handle to the coresponding electrical terminal is also accepted. 
Physical terminals are associated with underlying conducting objects
as part of the connectivity algorithm.  Not all terminals have an
associated object, in which case they are ``virtual".  An empty handle
(scalar 0) is returned in this case.
\end{description}


\subsection{Physical Conductor Groups}

\begin{description}
%------------------------------------
% 030204
\index{Group function}
\item{(int) \vt Group()}\\
This function will run the grouping and device extraction algorithm on
the current physical cell.  The grouping algorithm identifies the wire
nets.  The returned value is the number of groups used, or 0 if an
error occurs.  The group index extends from 0 through the number
returned minus one.  Group 0 is the ground group, if a ground plane
layer has been defined.

%------------------------------------
% 030204
\index{GetNumberGroups function}
\item{(int) \vt GetNumberGroups()}\\
This returns the number of conductor groups allocated by the
extraction process in the physical part of the current cell.  The
group index passed to other functions should be less than this value.

%------------------------------------
% 030204
\index{GetGroupBB function}
\item{(int) \vt GetGroupBB({\it group}, {\it array\/})}\\
This function returns the bounding box of the conductor group whose
index is passed as the first argument.  The coordinates, in microns
relative to the current physical cell origin, are returned in the
{\it array}, which must have size 4 or larger.  If the function
succeeds, 1 is returned, otherwise 0 is returned.  The saved order is
L, B, R, T.

%------------------------------------
% 030204
\index{GetGroupNode function}
\item{(int) \vt GetGroupNode({\it group\/})}\\
This function returns the node number from the electrical database
which corresponds to the physical group index passed as the argument. 
If the association failed, -1 is returned.

%------------------------------------
% 030204
\index{GetGroupName function}
\item{(string) \vt GetGroupName({\it group\/})}\\
This will return a string containing a name for the group whose number
is passed as the argument.  The name is the name of a formal terminal
attached to the group, or the net name if no formal terminal.  If the
group has no name, a null string is returned.

%------------------------------------
% 030204
\index{GetGroupNetName function}
\item{(string) \vt GetGroupNetName({\it group\/})}\\
This will return a string containing the net name for the group whose
number is passed as the argument.  If the group has no net name, a
null string is returned.

%------------------------------------
% 030204
\index{GetGroupCapacitance}
\item{(real) \vt GetGroupCapacitance({\it group\/})}\\
This will return the capacitance assigned to the group whose index is
passed as the argument.  If no capacitance has been assigned.  0 is
returned.

%------------------------------------
% 061016
\index{CountGroupObjects function}
\item{(int) \vt CountGroupObjects({\it group\/})}\\
Return the number of physical objects that implement the group.  If
there is an error, such as the argument being out of range, -1 is
returned.

%------------------------------------
% 070516
\index{ListGroupObjects function}
\item{(object\_handle) \vt ListGroupObjects({\it group\/})}\\
This function returns a handle to the list of objects associated with
the current physical cell which constitute the group, as found by the
extraction system.  These may or may not correspond to actual objects
in the cell.  For example, the objects returned have been processed by
the {\vt Conductor Exclude} directive, so would possibly be clipped
versions of the original objects.  Additionally, objects from
wire-only subcells and vias that have been logically flattened during
extraction will be included.  Objects from flattened via instances
will mave the {\vt MergeCreated} ({\vt 0x1}) flag set, which can be
tested with {\vt GetObjectFlags}.  This allows the caller to filter
out redundant metal if standard vias are used, in addition to the
objects, to represent the net.

The argument is the group number.  The returned objects are copies, so
can not be modified or selected.  If an error occurs, 0 is returned.

%------------------------------------
% 070516
\index{CountGroupVias function}
\item{(int) \vt CountGroupVias({\it group\/})}\\
Return the number of via instances used to implement the group, from
the extraction system.  This is the number of vias that would be
returned by {\vt ListGroupVias} (below).  If there is an error, such
as the group number argument being out of range, -1 is returned.

%------------------------------------
% 070516
\index{ListGroupVias function}
\item{(object\_handle) \vt ListGroupVias({\it group\/})}\\
This function returns a handle to the list of via instances associated
with the current physical cell which are used in the group, as
obtained from the extraction system.  This may include vias that were
``promoted'' due to the logical flattening of wire-only subcells
during extraction.  Vias in such cells are treated as if they reside
in their parent cells, recursively.

The argument is the group number.  The via instances are copies, so
can not be modified or selected.  If an error occurs, 0 is returned.

%------------------------------------
% 061016
\index{CountGroupDevContacts function}
\item{(int) \vt CountGroupDevContacts({\it group\/})}\\
This function returns a count of the number of device contacts which
are assigned to the conductor group whose index is passed as the
argument.  If an error occurs, -1 is returned.

%------------------------------------
% 030204
\index{ListGroupDevContacts function}
\item{(dev\_contact\_handle) \vt ListGroupDevContacts({\it group\/})}\\
This function returns a handle to the list of device contacts which
are assigned to the conductor group whose index is passed as the
argument.  If an error occurs, 0 is returned.

%------------------------------------
% 061016
\index{CountGroupSubcContacts function}
\item{(int) \vt CountGroupSubcContacts({\it group\/})}\\
This function returns a count of subcircuit contacts associated with
the group index passed as the argument.  If an error occurs, -1 is
returned.

%------------------------------------
% 030204
\index{ListGroupSubcContacts function}
\item{(subc\_contact\_handle) \vt ListGroupSubcContacts({\it group\/})}\\
This function returns a handle to a list of subcircuit contacts
associated with the group index passed as the argument.  If an error
occurs, 0 is returned.

%------------------------------------
% 061016
\index{CountGroupTerminals function}
\item{(int) \vt CountGroupTerminals({\it group\/})}\\
Return a count of cell connection terminals associated with the group
number passed as an argument.  If an error occurs, -1 is returned.

%------------------------------------
% 061016
\index{ListGroupTerminals function}
\item{(terminal\_handle) \vt ListGroupTerminals({\it group\/})}\\
This will return a handle to a list of formal terminals associated
with the group number passed as an argument.  If an error occurs, 0 is
returned.  If the group contains no formal terminals, the list will be
empty.

%------------------------------------
% 061016
\index{ListGroupTerminalNames function}
\item{(stringlist\_handle) \vt ListGroupTerminalNames({\it group\/})}\\
This function returns a list of names of the cell connection terminals
assigned to the conductor group whose index is passed as the argument. 
If an error occurs, 0 is returned.  If the group contains no cell
connection terminals, the list will be empty.

%------------------------------------
% 061016
\index{CountGroupPhysTerminals function}
\item{(int) \vt CountGroupPhysTerminals({\it group\/})}\\
Return a count of the physical terminal descriptors from the
physical cell that are associated with the group number given.

%------------------------------------
% 061016
\index{ListGroupPhysTerminals function}
\item{(physterm\_handle) \vt ListGroupPhysTerminals({\it group\/})}\\
Return a handle to a list of the physical terminal descriptors from
the physical cell that are associated with the group number given.

\end{description}


\subsection{Physical Devices}

\begin{description}
%------------------------------------
% 070809
\index{ListPhysDevs function}
\item{(device\_handle) \vt ListPhysDevs({\it name}, {\it pref}, {\it indices},
  {\it area\_array\/})}\\
This function returns a handle to a list of devices extracted from the
physical part of the current cell.  The first two arguments are
strings which match the {\vt Name} and {\vt Prefix} fields from the
technology file Device block of the device to list.  Either or both of
these arguments can be null or empty, in which case no devices are
excluded by the comparison, i.e., such values act as wildcards.

The third argument is a string providing a list of device indices, or
ranges of indices, to allow.  These are integers that are unique to
each instance of a device type in a cell.  If this argument is null or
empty, all indices will be returned.  Each token in the string is an
integer (e.g., ``2''), or range of integers (e.g., ``1-4''), using the
hyphen (minus sign) to separate the minimum and maximum index to
include.  The tokens are separated by white space and/or commas.  For
example, ``1,3-5,7,9-12''.

The final argument, if not 0, is an array of size four or larger
containing rectangle coordinates, in microns, in order L,B,R,T.  If 0
is passed for this argument, the entire cell is searched for devices. 
Otherwise, only the area provided will be searched.

On success, a handle is returned, otherwise 0 is returned.  The handle
can be used in the functions that take a device handle as an argument. 
This is {\it not} an object handle.  The returned device handle can be
manipulated with the generic handle functions, and like other handles
should be iterated through or explicitly closed when no longer needed.

%------------------------------------
% 030204
\index{GetPdevName function}
\item{(string) \vt GetPdevName({\it device\_handle\/})}\\
This function returns a string containing the name of the device
referenced by the handle.  The name string is composed of the {\vt
Name} field for the device (from the {\et Device Block}), followed by
an underscore, followed by the device index number.  If the handle is
defunct or some other error occurs, a null string is returned.

%------------------------------------
% 030204
\index{GetPdevIndex function}
\item{(int) \vt GetPdevIndex({\it device\_handle\/})}\\
This function returns the index of the device referenced by the handle
passed as an argument.  The index is an integer which is unique among
the devices of a given type.  If the handle is defunct or an error
occurs, -1 is returned.

%------------------------------------
% 030204
\index{GetPdevDual function}
\item{(object\_handle) \vt GetPdevDual({\it device\_handle\/})}\\
This function returns an object handle which references the dual
device in the electrical database to the physical device referenced by
the argument.  If association failed for the device, 0 is returned. 
The dual device is a subcell obtained from the device library.

%------------------------------------
% 030204
\index{GetPdevBB function}
\item{(int) \vt GetPdevBB({\it device\_handle}, {\it array\/})}\\
This function obtains the bounding box of the device referenced by the
first argument.  The coordinates, in microns using the origin of the
current physical cell, are returned in the {\it array}, which must
have size 4 or larger.  If the function succeeds, 1 is returned,
otherwise the returned value is 0.  The saved order is L, B, R, T.

%------------------------------------
% 030204
\index{GetPdevMeasure function}
\item{(real) \vt GetPdevMeasure({\it device\_handle}, {\it mname\/})}\\
This function returns a device parameter corresponding to a {\vt
Measure} line given in the Device block for the device referenced by
the first argument.  The second argument is a string giving the name
from a {\vt Measure} line.  The returned value is the measured
parameter, or 0 if there was an error.

%------------------------------------
% 030204
\index{ListPdevMeasures function}
\item{(stringlist\_handle) \vt ListPdevMeasures({\it device\_handle\/})}\\
This function returns a string list handle corresponding to a list of
the names associated with {\vt Measure} lines in the Device block for
the device referenced by the handle.  These are the names that can be
passed to {\vt GetPdevMeasure} to perform the measurement.  If an error
occurs, 0 is returned.

%------------------------------------
% 030204
\index{ListPdevContacts function}
\item{(dev\_contact\_handle) \vt ListPdevContacts({\it device\_handle\/})}\\
This function returns a handle to a list of contact descriptors for
the device referenced by the argument.  The returned handle can be
passed to the functions below to obtain information about the device
contacts.  If there is an error, 0 is returned.  The returned handle
can be manipulated with the generic handle functions, and like other
handles should be iterated through or closed explicitly when no longer
needed.

%------------------------------------
% 030204
\index{GetPdevContactName function}
\item{(string) \vt GetPdevContactName({\it dev\_contact\_handle\/})}\\
This function returns the name string of the contact referenced by the
argument.  Contact names are assigned in the Device block for the
device containing the contact.  If an error occurs, a null string is
returned.

%------------------------------------
% 030204
\index{GetPdevContactBB function}
\item{(int) \vt GetPdevContactBB({\it dev\_contact\_handle}, {\it array\/})}\\
This function returns the bounding box of the contact referenced by
the first argument.  The coordinates, in microns relative to the
origin of the physical current cell, are returned in the {\it array},
which must have size 4 or larger.  If the operation is successful, 1
is returned, otherwise 0 is returned.

%------------------------------------
% 030204
\index{GetPdevContactGroup function}
\item{(int) \vt GetPdevContactGroup({\it dev\_contact\_handle\/})}\\
This function returns the conductor group index to which the contact
referenced by the argument is assigned.  If there is an error, -1 is
returned.

%------------------------------------
% 030204
\index{GetPdevContactLayer function}
\item{(string) \vt GetPdevContactLayer({\it dev\_contact\_handle\/})}\\
This function returns the name string of the layer to which the
contact referenced by the argument is assigned.  All contacts are
assigned to layers which have the {\et Conductor} attribute.  If there
is an error, a null string is returned.

%------------------------------------
% 030204
\index{GetPdevContactDev function}
\item{(device\_handle) \vt GetPdevContactDev({\it dev\_contact\_handle\/})}\\
This function returns a handle to the device containing the contact
referenced by the argument.  If an error occurs, 0 is returned.  The
returned handle should be closed (for example, with the {\vt Close}
function) when no longer needed.

%------------------------------------
% 030204
\index{GetPdevContactDevName function}
\item{(string) \vt GetPdevContactDevName({\it dev\_contact\_handle\/})}\\
This function returns the name of the device containing the contact
referenced by the argument.  A null string is returned on error.

%------------------------------------
% 030204
\index{GetPdevContactDevIndex function}
\item{(int) \vt GetPdevContactDevIndex({\it dev\_contact\_handle\/})}\\
This returns the index number of the device to which the contact,
referenced by the passed handle, is associated.  Each device of a
given type has an index number assigned, which is unique in the
containing cell.  On error, -1 is returned.  A valid index is 0 or
larger.

\end{description}


\subsection{Physical Subcircuits}

\begin{description}
%------------------------------------
% 030204
\index{ListPhysSubckts function}
\item{(subckt\_handle) \vt ListPhysSubckts({\it name}, {\it index},
  {\it l}, {\it b}, {\it r}, {\it t\/})}\\
This function returns a handle to a list of subcircuits from the
physical part of the current cell.  Subcircuits are subcells which
contain devices or sub-subcells that contain devices.  Subcells that
contain only wire are typically not saved internally as subcircuits. 
The first argument is a string name which will match the returned
subcircuits.  If this argument is null or empty, then this test will
not exclude any subcircuits to be returned.  The second argument is
the index number of the subcircuit to be returned.  If the value is
-1, subcells with any index will be returned.  The remaining four
values define a rectangular area, given in microns relative to the
current physical cell origin, where subcircuits will be searched for. 
If all four values are 0, the entire cell will be searched.  The
returned handle references subcircuits, and is distinct from device
handles and object handles.  The handle can be passed to the generic
handle functions, and like other handles should be iterated through or
closed when no longer needed.  The function returns 0 if an error
occurs.

%------------------------------------
% 030204
\index{GetPscName function}
\item{(string) \vt GetPscName({\it subckt\_handle\/})}\\
This function returns the cell name corresponding to the subcircuit
referenced by the handle.  if an error occurs, a null string is
returned.

%------------------------------------
% 061116
\index{GetPscIndex function}
\item{(int) \vt GetPscIndex({\it subckt\_handle\/})}\\
This function returns the index of the subcircuit referenced by the
argument.  The index is a zero-based sequence for each subcircuit
master.  If an error occurs, -1 is returned.

%------------------------------------
% 061116
\index{GetPscIdNum function}
\item{(int) \vt GetPscIdNum({\it subckt\_handle\/})}\\
This function returns the ID number of the subcircuit referenced by
the argument.  The ID number is unique among all instances in the
parent cell.  If an error occurs, -1 is returned.

%------------------------------------
% 061116
\index{GetPscInstName function}
\item{(string) \vt GetPscInstName({\it subckt\_handle\/})}\\
This function returns an instance name corresponding to the subcircuit
instance referenced by the handle.  This is the cell name, followed by
an underscore, followed by the index number.  if an error occurs, a
null string is returned.

%------------------------------------
% 030204
\index{GetPscDual function}
\item{(object\_handle) \vt GetPscDual({\it subckt\_handle\/})}\\
This function returns an object handle which references the subcell in
the electrical database which is the dual of the physical subcircuit
referenced by the argument.  If the association fails, 0 is returned.

%------------------------------------
% 030204
\index{GetPscBB function}
\item{(int) \vt GetPscBB({\it subckt\_handle}, {\it array})}\\
This function returns the bounding box of the subcircuit referenced by
the first argument.  The coordinates, in microns relative to the
origin of the current physical cell, are returned in the array, which
must have size 4 or larger.  If the operation succeeds, 1 is returned,
otherwise 0 is returned.

%------------------------------------
% 061116
\index{GetPscLoc function}
\item{(int) \vt GetPscLoc({\it subckt\_handle}, {\it array})}\\
This returns the instance placement location, in microns, in the array
passed as a second argument.  The array must have size two or larger. 
On success, the function returns 1, and the array location 0 will
contain the X value, and the 1 location will contain the Y value. 
Zero is returned on error, with the array values undefined.

%------------------------------------
% 061116
\index{GetPscTransform function}
\item{(int) \vt GetPscTransform({\it subckt\_handle}, {\it type\/},
  {\it array})}\\
This function returns a string describing the instance orientation. 
There are presently three format types, specified by the second
argument.  If this argument is zero, then the {\Xic} transformation
string is returned.  This is the same CIF-like encoding as used for
the current transformation in the status line of {\Xic}.  In this case
the third argument is ignored and can be zero.

If the second argument is one, the return will be a Cadence DEF
orientation code.  In addition, if an array of size two or larger is
passed as a third argument, the values will be filled in with the X
and Y origin correction values implied by the transformation.  In a
DEF transformation, the lower left corner position of the bounding box
is invariant, implying that there is an additional translation after
rotation/mirroring to enforce this.  Pass 0 for this argument if these
values aren't needed.

In DEF, there is no support for 45, 135, 225, and 315 rotations, a
null string is returned in these cases.  Magnification is ignored.

If the second argument is any other value, the OpenAccess strings are
returned, otherwise all is as for DEF.

The following table lists equivalent orientation codes for DEF,
OpenAccess, and {\Xic}.  The {\bf Origin} column indicates the
position of the original lower-left corner after the operation.

\begin{quote}
\begin{tabular}{llll}
LEF/DEF & OpenAccess & {\Xic} & Origin\\
N  & R0   & R0     & LL\\
W  & R90  & R90    & LR\\
S  & R180 & R180   & UR\\
E  & R270 & R270   & UL\\
FN & MY   & MX     & LR\\
FW & MX90 & R270MY & LL\\
FS & MX   & MY     & UL\\
FE & MY90 & R90MX  & UR\\
\end{tabular}
\end{quote}

%------------------------------------
% 030204
\index{ListPscContacts function}
\item{(subc\_contact\_handle) \vt ListPscContacts({\it subckt\_handle})}\\
This function returns a handle to a list of subcircuit contacts
associated with the subcircuit referenced by the handle.  The returned
handle is a distinct type, in particular subcircuit contacts are
different from device contacts.  The return handle can be used with
the functions which query information about subcircuit contacts, or
with the generic handle functions.  If an error occurs, this function
returns 0.

%------------------------------------
% 030204
\index{IsPscContactIgnorable function}
\item{(int) \vt IsPscContactIgnorable({\it subc\_contact\_handle\/})}\\
If the subcircuit associated with the contact referenced from the
argument is flattened or ignored, return 1.  Otherwise 0 is returned. 
When 1 is returned, the contact can usually be skipped in listings.

%------------------------------------
% 030204
\index{GetPscContactName function}
\item{(string) \vt GetPscContactName({\it subc\_contact\_handle\/})}\\
This function returns a name string, if available, from the subcircuit
contact referenced by the argument.  If the subcircuit does not
provide a name, the returned string will be a number giving the
subcircuit group contacted.  A null string is returned on error.

%------------------------------------
% 030204
\index{GetPscContactGroup function}
\item{(int) \vt GetPscContactGroup({\it subckt\_contact\_handle})}\\
This function returns the group index in the current cell
corresponding to the subcircuit contact referenced by the argument. 
If an error occurs, this function returns -1.

%------------------------------------
% 030204
\index{GetPscContactSubcGroup function}
\item{(int) \vt GetPscContactSubcGroup({\it subckt\_contact\_handle})}\\
This function returns the group index in the subcircuit associated
with the subcircuit contact referenced by the argument.  On error, the
function returns -1.

%------------------------------------
% 030204
\index{GetPscContactSubc function}
\item{(subckt\_handle) \vt GetPscContactSubc({\it subckt\_contact\_handle})}\\
This function returns a handle to the subcircuit which is associated
with the subcircuit contact referenced by the argument.  On error, the
function return 0.

%------------------------------------
% 030204
\index{GetPscContactSubcName function}
\item{(string) \vt GetPscContactSubcName({\it subc\_contact\_handle\/})}\\
This function returns a string containing the name of the subcircuit
associated with the contact referenced by the argument.  A null string
is returned on error.

%------------------------------------
% 030204
\index{GetPscContactSubcIndex function}
\item{(int) \vt GetPscContactSubcIndex({\it subc\_contact\_handle\/})}\\
This function returns the index of the subcircuit associated with the
contact referenced by the argument.  Each subcircuit of a given kind
has an index number that is unique in the containing cell.  On error,
-1 is returned.  Valid index values are 0 and larger.

%------------------------------------
% 061116
\index{GetPscContactSubcIdNum function}
\item{(int) \vt GetPscContactSubcIdNum({\it subc\_contact\_handle\/})}\\
This function returns the ID number of the subcircuit associated with
the contact referenced by the argument.  Each subcircuit has an ID
number that is unique in the containing cell.  On error, -1 is
returned.  Valid index values are 0 and larger.

%------------------------------------
% 061116
\index{GetPscContactSubcInstName function}
\item{(string) \vt GetPscContactSubcInstName({\it subc\_contact\_handle\/})}\\
This function returns a string containing an instance name of the
subcircuit associated with the contact referenced by the argument. 
The instance name consists of the cell name followed by an underscore,
which is followed by the index.  A null string is returned on error.

\end{description}


\subsection{Electrical Devices}

\begin{description}
%------------------------------------
% 030204
\index{ListElecDevs function}
\item{(stringlist\_handle) \vt ListElecDevs({\it regex\/})}\\
This function returns a handle to a list of strings containing device
names from the electrical database.  The names correspond to devices
used in the current circuit.  The argument is a regular expression
used to filter the device names.  If the argument is null or empty,
all devices are listed.  This function returns 0 on error.

%------------------------------------
% 030204
\index{SetEdevProperty function}
\item{(int) \vt SetEdevProperty({\it devname}, {\it prpty}, {\it string\/})}\\
This function is used to set property values of electrical devices and
mutual inductors.  It is equivalent to the {\vt Set} command, or the
keyboard {\cb !set} command, with the {\vt \@}{\it devname}.{\it
prpty} syntax.  The first argument is the name of a device in the
current circuit.  This is the value of a {\et name} property for some
device.  The second argument is a string giving the property type to
set or modify.  The possible strings are prefixes of ``{\vt name}'',
``{\vt model}'', ``{\vt value}'', ``{\vt param}'', ``{\vt other}'',
and ``{\vt nophys}''.  The single character string "{\vt n}" implies
{\et name}, and (additionally) "{\vt y}" implies {\et nophys}.  If the
string is unrecognized, the property type defaults to {\et other}.  If
the device is a mutual inductor, only the {\et name} and {\et value}
properties can be applied.  The final argument is a string containing
the body of the property.  If the string is null or empty, the
property is removed (or reset to the default in the case of the {\et
name} property).  The function returns 1 on success, 0 otherwise.

%------------------------------------
% 030204
\index{GetEdevProperty function}
\item{(string) \vt GetEdevProperty({\it devname}, {\it prpty\/})}\\
This function returns a string containing the text of the specified
property for the given device.  The two arguments have the same format
and interpretation as the first two arguments of {\vt
SetEdevProperty}, i.e., the device name and property name.  The return
value is a string containing the text for that property.  If the
device or property does not exist or some other error occurs, a null
string is returned.

%------------------------------------
% 030204
\index{GetEdevObj function}
\item{(object\_handle) \vt GetEdevObj({\it devname\/})}\\
This function returns a handle to the electrical subcell from the
device library corresponding to the given device name.  If an error
occurs, 0 is returned.

\end{description}


\subsection{Resistance/Inductance Extraction}

\begin{description}
%------------------------------------
% 052609
\index{extractRL function}
\item{(int) \vt ExtractRL({\it conductor\_zoidlist\/}, {\it layername\/},
 {\it r\_or\_l\/}, {\it array\/}, {\it term\/}, ...)}\\
This will use the square-counting system to estimate the resistance or
inductance of a conducting object with respect to two or more
terminals.  The first argument is a trapezoid list representing a
single conducting area, on the layer given in the second argument. 
The layer keywords set electrical parameters used in the estimation.

\begin{description}
\item{For Resistance:}\\
  The {\et Rsh} layer keyword gives the ohms-per-square of the
  material.  If not set, the value is computed from {\et Rho} or {\et
  Sigma} and {\et Thickness} if these are set.  If these keywords are
  also not given, a value of 1.0 is assumed.

\item{For Inductance:}\\
  The {\vt Tline} keyword supplies the appropriate parameters.  In
  this case, the material is assumed to be over a ground plane covered
  by dielectric.
\end{description}

The third argument is a boolean which if nonzero indicates inductance
estimation, and zero indicates resistance estimation.

The fourth argument is an array which will hold the return values,
which will be resized if necessary.  The zeroth component of the array
gives the number of returned values, which are returned in the rest of
the array.  If there are two terminals, the number of returned values
is 1.  For more than two terminals, the number of returned values is
{\it n\/}{\vt *}({\it n\/}--1)/2, where {\it n} is the number of
terminals.  The values are the effective two-terminal decomposition
for terminals {\it i\/},{\it j} ({\it i} {\vt !=} {\it j\/}) in the
order, e.g., for {\it n} = 4; 01, 02, 03, 12, 13, 23.

The following arguments are trapezoid lists representing the
terminals.  Arguments that are not trapezoid lists will be ignored. 
There must be at least two terminals passed.  Terminal areas should be
spatially disjoint, and in the computation, the terminal areas are
clipped by the conductor area.  Terminals are assigned numbers in
left-to-right order.

The algorithm is most efficient if all coordinates are on some grid. 
This provide for efficient tiling of the structure.

Structures that require a very large number of tiles may require
excessive time and memory to compute, and/or suffer from a loss of
accuracy.  The approximate threshold is $10^5$ tiling squares. 
Non-Manhattan shapes have strict internal limiting of tile count. 
Manhattan structures can require an arbitrarily large number of tiles,
thus the potential for resource overuse.

The return value is always 1.  The function will fail (terminating the
script) if an error is encountered.

%------------------------------------
% 052409
\index{extractNetResistance function}
\item{(int) \vt ExtractNetResistance({\it net\_handle\/},
 {\it spicefile\/}, {\it array\/}, {\it term\/}, ...)}\\
This function will extract resistance of a conductor net, taking into
account multiple conducting layers connected by vias.  The resistance
decomposition of each conducting object and its vias and/or terminals
is computed using the algorithm used by the {\vt ExtractRL} function. 
The resistance of the connected network is then computed, with respect
to the terminals specified.

The first argument is a handle to a list of objects as returned from
{\vt FindPath} or {\vt FindPathOfGroup}.

The second argument is a string giving a file name, which will contain
a generated SPICE listing representing the extracted resistor network. 
In the SPICE file, each terminal and each via are assigned node
numbers.  A comment indicates the range of numbers used for terminals. 
If this argument is 0 (NULL) or an empty string, no SPICE file is
written.

The third argument is an array which will hold the return values,
which will be resized if necessary.  The zeroth component of the array
gives the number of returned values, which are returned in the rest of
the array.  If there are two terminals, the number of returned values
is 1.  For more than two terminals, the number of returned values is
{\it n\/}{\vt *}({\it n\/}--1)/2, where {\it n} is the number of
terminals.  The values are the effective two-terminal decomposition
for terminals {\it i\/},{\it j} ({\it i} {\vt !=} {\it j\/}) in the
order, e.g., for {\it n} = 4; 01, 02, 03, 12, 13, 23.

The following arguments are trapezoid lists representing the
terminals.  Arguments that are not trapezoid lists will be ignored. 
There must be at least two terminals passed.  Terminal areas should be
spatially disjoint, and in the computation, the terminal areas are
clipped by the conductor area.  Terminals are assigned numbers in
left-to-right order.

The return value is always 1.  The function will fail (terminating the
script) if an error is encountered.

\end{description}


%------------------------------------------------------------------------------
\section{Schematic Editor Functions}
\subsection{Symbolic Mode}

\begin{description}
%------------------------------------
% 060616
\index{Connect function}
\item{(int) \vt Connect({\it for\_spice\/})}\\
This function establishes the circuit connectivity for the current
hierarchy.  If the boolean {\it for\_spice} is false, then devices
with the {\et nophys} property set are ignored, and the netlist will
have the ``shorted'' {\et nophys} devices shorted out.  This is
appropriate for LVS and other extraction system operations.

If {\it for\_spice} is true, the {\et nophys} devices are included,
and not shorted.  This applies when generating output for SPICE
simulation.

The function returns 1 on success, 0 otherwise.  If the schematic is
already processed and current, the function will return immediately. 
The schematic is implicitly processed before most internal operations
that make use of the schematic, so it is unlikely that the user will
need to call this function.

%------------------------------------
% 091306
\index{ToSpice function}
\item{(int) \vt ToSpice({\it spicefile\/})}\\
This function will dump a SPICE file from the current cell to a file
of the given name.  If the argument is null or an empty string, the
name will be that of the current cell with a ``{\vt .cir}'' suffix. 
Any existing file of the same name will be moved, and given a ``{\vt
.bak}'' extension.  The return value is 1 on success, 0 otherwise.

\end{description}

\subsection{Electrical Nodes}

\begin{description}
%------------------------------------
% 011210
\index{IncludeNoPhys function}
\item{(int) \vt IncludeNoPhys({\it flag\/})}\\
This sets an internal mode which applies to the other functions in
this group.  If the boolean {\it flag} argument is nonzero, devices
with the {\et nophys} property set will be considered when generating
the connectivity and node mapping structures.  This has relevance when
a device has the shorted option to {\et nophys} set, as such devices
will be considered as normal devices with the flag set.  If the flag
is unset, these devices will be taken as short circuits, which of
course alters the node assignments.

Internally, the extraction functions always take these devices as
shorted, and they are otherwise ignored.  When generating a SPICE file
during simulation or with other commands in the side menu, these
devices are included as normal devices.  The present state of the
netlist data structures will reflect the state of the last operation.

Setting this flag will cause rebuilding of the data structures to the
requested state if necessary when one of the functions in this section
is called.  This persists until some other function, such as an
extraction or SPICE listing function is called, at which time the
internal state of the flag may change.  Thus, this function may need
to be called repeatedly ahead of the functions in this section.

The return value is the previous value of the internal flag.

%------------------------------------
% 030204
\index{GetNumberNodes function}
\item{(int) \vt GetNumberNodes()}\\
Return the size of the internal node map.  The internal node numbers
range from 0 up to but not including this value.  The return value is
0 on error or if the cell is empty.

%------------------------------------
% 050809
\index{SetNodeName function}
\item{(int) \vt SetNodeName({\it node}, {\it name\/})}\\
This function associates the string {\it name} with the node number
given in the first argument.  This affects the electrical database,
and is equivalent to setting a node name with the node mapping
facility available in the side menu in electrical mode.  Netlist
output will use the given string name rather than a default name,
however if the existing default name matches a global node name, the
user-supplied name will be ignored.  If the name given is null or
empty, any existing given name is deleted, and netlist output will use
the node number.  The function returns 1 on success, 0 otherwise.

%------------------------------------
% 050809
\index{GetNodeName function}
\item{(string) \vt GetNodeName({\it node\/})}\\
This function returns a string name for the given node number.  If a
name has been given for that node, the name is returned, otherwise an
internally generated default name is returned.  If the operation
fails, a null string is returned.

%------------------------------------
% 030204
\index{GetNodeNumber function}
\item{(int) \vt GetNodeNumber({\it name\/})}\\
This function returns the node number corresponding to the name string
passed as an argument.  If no mapping to the string is found, -1 is
returned.

%------------------------------------
% 030204
\index{GetNodeGroup function}
\item{(int) \vt GetNodeGroup({\it node\/})}\\
This function returns the group index in the physical cell that
corresponds to the given node number.  On error, -1 is returned.

%------------------------------------
% 060616
\index{ListNodePins function}
\index{ListNodeTerminals function}
\item{(terminal\_handle) \vt ListNodePins({\it node\/})}\\
Note:  This and {\vt ListNodeContacts} replace {\vt
ListNodeTerminals}, which was removed in 4.2.12.

Return a handle to the list of cell connection terminals bound to the
internal node number supplied as the argument.  There probably will be
at most one such connection.

%------------------------------------
% 060616
\index{ListNodeContacts function}
\item{(terminal\_handle) \vt ListNodeContacts({\it node\/})}\\
Note:  This and {\vt ListNodePins} replace {\vt ListNodeTerminals},
which was removed in 4.2.12.

Return a handle to a list of device and subcircuit connection
terminals bound to the specified node.

%------------------------------------
% 060616
\index{GetNodeContactInstance function}
\item{(object\_handle) \vt GetNodeContactInstance({\it terminal\_handle\/})}\\
For a handle to an instance contact, such as returned from {\vt
ListNodeContacts}, this function will return a handle to the device or
subcircuit instance that provides the contact.

%------------------------------------
% 060616
\index{ListNodePinNames function}
\index{ListNodeTerminalNames function}
\item{(stringlist\_handle) \vt ListNodePinNames({\it node\/})}\\
Note:  This and {\vt ListNodeContactNames} replace {\vt
ListNodeTerminalNames}, which was removed in 4.2.12.

Return a list of cell connection terminal names that connect to the
given node.  There is likely at most one cell connection per node.

%------------------------------------
% 060616
\index{ListNodeContactNames function}
\item{(stringlist\_handle) \vt ListNodeContactNames({\it node\/})}\\
Note:  This and {\vt ListNodePinNames} replace {\vt
ListNodeTerminalNames}, which was removed in 4.2.12.

Return a list of device and subcircuit contact names that connect to
the given node.
\end{description}

\subsection{Symbolic Mode}

\begin{description}
%------------------------------------
% 030115
\index{IsShowSymbolic function}
\item{(int) \vt IsShowSymbolic()}\\
This function will return 1 if the current cell is being displayed in
symbolic form in the main window, 0 otherwise.  The return is always 0
in physical mode.

%------------------------------------
% 030115
\index{ShowSymbolic function}
\item{(int) \vt ShowSymbolic({\it show\/})}\\
This will set symbolic mode of the current cell, and display the
symbolic representation, if possible, in the main window.  The effect
is similar to the effect of pressing or un-pressing the {\cb symbl}
button in the electrical side menu.  The function call must be made in
electrical display mode.  When symbolic mode is asserted, by passing a
boolean true argument, the current cell will be displayed in symbolic
mode, unless the {\cb No Top Symbolic} button in the {\cb Main Window}
sub-menu of the {\cb Attributes Menu} is pressed.  The return value is
1 on success, 0 if some error occurred, with an error message likely
available from {\vt GetError}.

%------------------------------------
% 030115
\index{SetSymbolicFast function}
\item{(int) \vt SetSymbolicFast({\it symb\/})}\\
This will enable or disable symbolic mode of the current cell.  It
differs from {\vt ShowSymbolic} in two ways.  First, it applies only
to cells with a symbolic representation, meaning that it has a
symbolic form which may or may not be visible.  Second, it will change
the status of a flag in the cell, but there will be no updating of the
screen or other internal things (such as undo logging).  The caller
must reset to the original state before a screen redisplay or any
major operation.  This is much faster than calling {\vt ShowSymbolic},
and can be used when making quick changes to a cell.

The return value is 1 if the current cell was previously actively
symbolic, 0 otherwise.  In physical mode the return value is always 0
and the function has no effect.

%------------------------------------
% 113009
\index{MakeSymbolic function}
\item{(int) \vt MakeSymbolic()}\\
This will create a very simple symbolic representation of the
electrical view of the current cell, consisting of a box with a name
label, and wire stubs containing the terminals.  Any existing symbolic
representation will be overwritten (but the operation can be undone). 
In electrical mode, symbolic mode will be asserted.

On success, 1 is returned, 0 otherwise.

\end{description}

