% -----------------------------------------------------------------------------
% Xic Manual
% (C) Copyright 2013, Whiteley Research Inc., Sunnyvale CA
% $Id: format.tex,v 1.82 2017/03/22 22:27:01 stevew Exp $
% -----------------------------------------------------------------------------


% -----------------------------------------------------------------------------
% symfilefmt 022816
\chapter{Design Data File Formats}

\index{strip button}
This section describes the extensions to the CIF and GDSII formats
used by {\Xic}.  The CGX file format, designed to be a more efficient
replacement for GDSII and released into the public domain by Whiteley
Research Inc., is described below as well.  The extensions to CIF and
GDSII are designed to accommodate the electrical information and
certain properties.  When strict conformance to the standard format is
required, such as when exporting physical layouts to a mask vendor,
the {\cb Strip For Export} button in the {\cb Export Control} panel
should be used to strip out all extensions, leaving only the physical
layout.

The GDSII (Stream) format is owned by Cadence, Inc., and is described
in documentation available from Cadence (specifically the
``\underline{Design Data Translators Reference},'' which is updated
periodically), which may be available on the internet.

\index{cell names}
In {\Xic}, cell names are not limited in length.  The cell names can
contain any characters valid in a file name with the exception of the
semicolon (`;'), which is reserved in the CIF-like syntax used for
native cell files.  For portability, it is recommended that cell
names use only the GDSII allowed characters, which are the
alpha-numerics plus `\_', `\$', and `?'.  In older GDSII
specifications (release 3), cell names were limited to 32 characters,
so it may be wise to observe this limit in {\Xic}.

Archive files created by {\Xic} generally consist of two records, the
first containing the physical information, and the second containing
electrical information.  If there is no electrical information, the
second block is not written.  Each block is an individual
representation of the archive file type, i.e., they each parse as a
complete ``file''.  GDSII files written in this manner are generally
portable to other CAD systems (format extensions appear in the
electrical block only), as reading will terminate at the end of the
physical block, and the following electrical block will be ignored. 
However, in specific instances where this proves not to be true, the
{\cb Strip For Export} button, which eliminates the electrical block,
should be used. 

The same comments apply to OASIS and CGX formats, however CGX is not
known to be supported by other CAD vendors at this time.

CIF files use extensions unique to {\Xic}, so will likely not be
portable to other CAD systems unless {\cb Strip For Export} is used. 
These extensions are described below.

% 022916
\section{GDSII Format and Extensions}
\index{GDSII file format}

The GDSII format provides a compact, binary representation of a design
hierarchy.  Though the standard format is intended for physical data,
minor extensions are used by {\Xic} to allow storage of the electrical
information as well.

A GDSII file consists of a sequence of data blocks.  The first four
bytes of the block provide the block type and size.  Integers and
floating point numbers have defined representations in GDSII, so
conversion is necessary from most machine representations.  This
section will describe the extensions only.

A GDSII file can be decomposed into an ASCII representation with the
{\cb Format Conversion} panel found in the {\cb Convert Menu}.  The
resulting file prints out the characteristics of each block, plus the
block offset within the file and messages indicating extensions and
errors.  A file in this format can be reconverted to GDSII. 
Generally, there is a one to one correspondence between items in the
text representation and blocks in the GDSII file, thus one can learn
much about the structure of the GDSII file by examining the text
representation.

\index{Strip For Export button}
When the {\cb Strip For Export} button in the {\cb Export Control}
panel is active, GDSII files produced from this panel adhere to the
strict GDSII standard and contain only physical data.  Otherwise the
file produced will contain extensions.  Such files are not guaranteed
to be readable by other software (but they generally are).

\index{GDSII extensions}
The file with extensions contains two concatenated GDSII record sets. 
The first contains the physical data.  The physical records are zero
padded to the next 2Kb block boundary, at which point the electrical
records begin.  Beyond this arrangement the extensions are as follows:

\begin{enumerate}
\item The data size limitation on attribute strings is increased and the
total size of attribute lists is unconstrained.
Attribute strings can be up to 16Kb in length.

\item Attribute records can appear ahead of cell definitions, thus
giving properties to cell definitions.  This violates the standard
GDSII record sequencing.
\end{enumerate}

Both of these extensions are necessary to accommodate to properties found
in the electrical design data, and are used in the electrical part of
the file only.

% 022916
\subsection{Physical Mode Cell Properties}

Certain features, such as parameterized cells, require cell properties
in physical mode.  Cell properties are also used to save the grid/snap
values in the top-level cell, and can be added by the user to support
other applications.

The GDSII format has no provision for storing properties of cell
definitions.  In electrical mode, {\Xic} uses the format extension
mentioned above.  We can't use extensions in physical mode, since that
would make the files non-portable, so we have to fake it.

\index{CELL PROPERTIES label}
In releases prior to 2.5.66, the cell properties were saved in a dummy
label.  This label was written on layer/datatype 0/0 at the origin,
and was given the text string ``{\vt CELL PROPERTIES}''.  Physical
cells with properties would have this label added in GDSII output. 
When reading in the GDSII file, the label would be stripped, and the
properties from the label object would be applied to the containing
cell.

However, when using direct conversions from the {\cb Format
Conversion} panel from the {\cb Convert Menu}, the file would be
converted as-is, so that if converting to {\Xic} native cell files
(for example), the converted cells would contain the ``{\vt CELL
PROPERTIES}'' labels and would {\bf not} have the properties set.

\index{SNAPNODE record}
\index{PLEX record}
Release 2.5.66 and later no longer create a ``{\vt CELL
PROPERTIES}'' label.  Instead, a ``SNAPNODE'' record with a PLEX
number {\vt 0xffffff} is created, at the origin on layer 0 with
nodetype 0.  This is an obscure data type that is more likely to be
invisible in the GDSII database.  Unlike a label, it should not be
visible in most other readers.

The current release reader will still process the ``{\vt CELL
PROPERTIES}'' label if found, for backwards compatibility.  In
addition, it will also process these labels, and the SNAPNODE records,
when doing direct conversions, so that the properties are assigned
correctly in this case.

Neither construct is/was added to the output file if the {\et
StripForExport} button or variable is active.

{\bf COMPATIBILITY WARNING}\\
{\Xic} releases prior to 2.5.66 will not process cell properties in
GDSII files created with this release and subsequent.  Physical mode
cell properties are used by {\Xic} to implement parameterized cells,
to save the grid parameters used in the top-level cell, and can be
added by the user for third-party purposes.  Loss of cell properties
will cause parameterized cells to lose the parameterization feature,
but still behave as normal cells.  Loss of the grid parameters may
require the user to reset these manually.  Files read with older
versions will generate ``{\vt unsupported record type PLEX}'' warnings
in the log file if any of the new-style records are encountered.

% 101212
\section{The CIF File Format}
\index{CIF format}

The Cal (Tech) Intermediate Format (CIF) was developed at Cal Tech in
the earliest days of design automation.  The format, such as it is
(there are many dialects), is public domain.  Though possibly still
used in educational and research environments, it is unusual in
current commercial IC engineering.

The format used in native cell files and the device library file is
an extension of the CIF file format.  Through extension, this format is
robust enough to meet the needs of {\Xic} while retaining the syntactic
simplicity of the original format.  This section outlines the basic
syntax of CIF, while the next section will provide details about the
extensions used by {\Xic}.

In CIF, ``lines'' are terminated with semicolons.  The line feed and
carriage return characters are taken as white space and ignored, and
may not even be present, so the ``lines'' are actually logical only.

\index{CIF format!comment}
Comments in CIF are enclosed in parentheses.  Comments are ignored in
CIF, however {\Xic} uses special comment lines for various purposes,
as will be seen in the next section.
\begin{quote}\vt
(This is an example comment);
\end{quote}
Note that this (and all) CIF lines must be terminated with `;'.

The first one or two non-whitespace characters of a line (i.e.,
following `;') are used as a command key.  In strict CIF, this key is
a letter, though numbers have been adopted as widely-used extensions.

Historically, in CIF the word ``symbol'' has been used to refer to
what in current terminology is referred to as a cell.  When describing
CIF, the terms ``symbol'' and ``cell'' are used as synonyms.

\index{CIF format!symbol definition}
The {\vt DS} (define symbol) directive begins a symbol (cell) definition.
\begin{quote}\vt
DS {\it symnum A B\/};
\end{quote}
In strict CIF, symbols do not have names, but are referenced by symbol
number.  The assigned symbol number (an integer) follows {\vt DS}. 
The remaining two parameters are for scaling.  Each coordinate in the
symbol is scaled by {\it A\/}/{\it B\/}.  The use of two integers
rather than a single floating point value was once considered a speed
optimization.  {\Xic} never uses the {\it symnum} or scaling factors in
native files:
\begin{quote}\vt
DS 0 1 1;
\end{quote}

\index{CIF format!symbol termination}
The definition of a symbol is terminated with a {\vt DF} line,
\begin{quote}\vt
DF;
\end{quote}

\index{CIF format!layer}
Within the symbol definition, there are layer directives, followed
by geometry specifications, and subcell calls.  A layer directive
consists of a line with the form
\begin{quote}\vt
L {\it layername\/};
\end{quote}
where {\it layername} is a name for a layer.  In traditional CIF, the
{\it layername} is an alphanumeric text token of four characters or
fewer.  All geometry which follows a layer declaration will be
assigned to that layer, until the next layer declaration.

\index{CIF format!box}
There are three types of CIF geometric objects used by {\Xic}:  boxes,
polygons, and wires.  Boxes have the form
\begin{quote}\vt
B {\it width height x y} [{\it rx ry\/}];
\end{quote}
where the first two parameters are the box width and height, and the
second two parameters are the coordinates of the midpoint of the box. 
The last two parameters are optional, and indicate a rotation.  The
two numbers define a vector with respect to the origin, and the angle
represents the angle by which the box should be rotated.  {\Xic} never
uses the rotation parameters for boxes.  Non-Manhattan rotated boxes
are converted to polygons.

Ordinarily, boxes are rendered according to the attributes of the
layer on which the box is defined.  In {\Xic} electrical mode, boxes
on the SCED layer use that attribute, however boxes on other layers
are rendered as a dotted outline with no fill.  The SCED layer
defaults to solid fill, other layers default to empty fill.  All
physical layers default to empty fill.

\index{CIF format!polygon}
Polygons are specified with {\vt P} followed by x-y coordinate pairs. 
The first and last coordinates must be the same, indicating closure of
the polygon.
\begin{quote}\vt
P {\it x0 y0 x1 y1 ... xN yN x0 y0\/};
\end{quote}
The polygon is rendered using the fill attributes of the layer on which
the polygon is defined.  There should be at least four pairs of
coordinates defined for a polygon.

\index{CIF format!wire}
Wires are fixed-width paths.  A wire is specified with {\vt W}
followed by the width, which is followed by x-y coordinate pairs
representing the path.
\begin{quote}
W {\it width x0 y0 x1 y1 ... xN yN\/};
\end{quote}
In electrical mode, the basic line primitive is a zero width wire.  In
physical mode, wires are defined with finite width as a physical
necessity.  The coordinates will form the vertices of the path.  A
wire can technically consist of a single vertex, which will be
rendered as a box with the width of the wire.  This construct is
disallowed in {\Xic}, and should be avoided.  Wires are rendered with the
fill attribute of the layer on which the wire is defined.

\index{CIF format!symbol call}
\index{CIF format!transformation}
A symbol call (subcell) is indicated with a {\vt C}, followed by the symbol
number, followed by a transformation specification.  The transformation
if made up of components representing translation, rotation, and
reflection.  Translation is indicated by {\vt T} followed by the translation:
\begin{quote}
T {\it x y}
\end{quote}
Rotation is specified by {\vt R}, followed by two numbers which represent
a vector with respect to the origin, the angle of which is the angle of
rotation.  Many parsers recognize only orthogonal rotations.  {\Xic}
recognizes angles that are multiples of 45 degrees.
\begin{quote}
R {\it rx ry}
\end{quote}
Mirroring about the y-axis is specified with
\begin{quote}
MX
\end{quote}
and about the x-axis with
\begin{quote}
MY
\end{quote}
The transformation specification is a concatenation of these directives,
which are evaluated in sequence to obtain the coordinate mapping from the
cell coordinates in the symbol being instantiated to the cell coordinates
of the parent of the instance.  The overall syntax is
\begin{quote}\vt
C {\it symnum transform\/};
\end{quote}
where an example would be
\begin{quote}\vt
C 0 R 1 0 T -1000 0;
\end{quote}

The parsing is terminated with an end directive:
\begin{quote}\vt
END
\end{quote}
This line need not be terminated with a semicolon.

The base coordinate system specified for CIF uses 100 units per micron.

% 032017
\section{CIF Format Extensions}
\label{cifext}
\index{CIF extensions}
\index{native file format!CIF extensions}

There have been numerous extensions to the CIF syntax used to enhance
the capabilities of the original format.  Some of these extensions
have been accepted widely and have become essentially part of the
standard.  {\Xic} uses these extensions, plus some further extensions,
in native format files and in files converted to CIF without the {\cb
Strip For Export} button active.  These extensions to the basic CIF
syntax are enumerated below.  Unless stated otherwise, the extension
is applied identically in native cell files and CIF output.

When writing a cell hierarchy in CIF format, when the top-level cell
is known, the writer will add a transform-less symbol call of the
top-level symbol just before the final {\vt End} line.  Thus, the two
final lines look like
\begin{quote}
{\vt C} {\it top\_cell\_number\/}{\vt ;}\\
{\vt End}
\end{quote}
MOSIS specifically requires this.  If the top level cell is unknown,
which is true when translating directly from another format, this is
skipped.  {\Xic} does not require or use this line.

\begin{enumerate}
\index{CIF extensions!layer name}
\item{Layer names can be arbitrarily long in CIF files generated or
read by {\Xic}, there is no four-character limit as in traditional
CIF.  In order to produce traditional CIF output, layer names should
follow the traditional CIF limitation.  Unlike some extensions, there
is no provision for enforcing traditional CIF output, when layer
names are arbitrary.  Prior to release 3.3.0, {\Xic} used CIF-style
layer names.}

\index{CIF extensions!semicolon hiding}
\item{If a semicolon is preceded by a backslash character, the reader
will strip the backslash and propagate the semicolon as an ordinary
character, and {\bf not} as a record terminator.  Thus, label and
property strings may contain semicolons if they are ``hidden'' with a
backslash.}

\item{Comment, label, and property strings can be arbitrarily long. 
Other interpreters, and older releases of {\Xic}, may limit these
lengths.  Beware that the GDSII and CGX file formats have a 64KB
record size limit and cannot accommodate strings that would overrun
this limit (see the {\et GdsTruncateLongStrings} variable).}

\index{CIF extensions!cell name}
\item The {\vt DS} (define symbol) line is always followed by a cell name
extension line of the form\\
{\vt 9 {\it symbol\_name\/};}\\
This extension is widely used, and is a standard means for including
the symbol names within the CIF framework.

In native cell files, however, the {\vt DS} line is {\it preceded} by
the symbol name line.

\item In {\Xic} releases prior to 3.0.0, the symbol number part of
an instance call was set to 0, i.e., the call sequence was always
\begin{quote}
{\vt C 0} ...{\vt ;}
\end{quote}
when cell name extensions were used.  Since cell names were provided
through the extensions, the cell numbering is unneeded.  In current
releases, the cell numbering is retained and will appear in the
instance calls, in all CIF output.

\item In CIF, the name of the cell being instantiated may precede the ``{\vt
C ...}'' (symbol call) line, using the same format as associated with
the {\vt DS} line, i.e.
\begin{quote}
{\vt 9} {\it master\_name\/}{\vt ;}\\
{\vt C} {\it N} {\vt ...;}
\end{quote}
This is redundant in CIF, since the master name can be obtained from
the symbol number.  It is required in native cell files, as there is
no symbol numbering.

\index{native file format!archive reference}
In native cell files only, instead of a cell name, the string can
contain two words separated by white space.  The first word is a path
to an archive
\ifoa
file (CIF, GDSII, etc.), library, or the name of an
OpenAccess library.
\else
(CIF, GDSII, etc.) or library file.
\fi
The second token would be the cell or library reference name.  If the
first word is a path to an archive file, the cell name is optional if
the file contains only one top-level cell, this cell would be
understood.  When the cell is read into {\Xic}, the master and its
hierarchy will be read from the specified source.

For example:
\begin{quote}
{\vt 9 /path/to/directory/containing/myfile.gds mycell;}\\
{\vt C 0 ...;}\\
\end{quote}

If a native cell with instance calls in this form is saved as a native
cell, the instance calls will retain this special form.

Cell files that contain instance calls of this form can be produced in
a couple of ways.  One would first read the target archive cell into
{\Xic}, then create a new cell, and place one or more instances of the
target cell.  One can then write the new cell to disk as a native
cell, and modify the instance calls with a text editor. 
Alternatively, one can use the rename function of the {\cb Cells
Listing} panel to change the name of the target cell to the two words
as would appear in the instance call, i.e., for example the full path
to the source file name followed by the cell name, separated by space. 
The rename will accept this form.  Then write the current cell to a
native cell file.  There will be no need for text editing in this
case.  After doing this, however, you have a cell in memory with a
bizarre name, best to clear the database or restart {\Xic}.

\index{CIF extensions!labels}
\item Labels are specified with a unique syntax:\\
{\vt 94 <<{\it label string\/}>> {\it x y flags width height\/};}\\
This is a further extension of a widely-used extension for labels,
which does not have the {\it flags}, {\it width}, or {\it height}
fields and the delimiters around the label.  The original extension
also required that the string contain no white space.

The {\it width\/} and {\it height\/} are the dimensions of the
untransformed bounding box of the label.  The label will be stretched
to fill this area.  The label is surrounded by {\vt << >>}.  The {\it
x\/} and {\it y\/} are the reference coordinates, which by default is
the lower left corner of the bounding box.  The {\it flags\/} entry
specifies transformations applied to the label at the reference point,
and other rendering information.  See \ref{labelflags} for more
information.

\item Cells and instances can be preceded by properties of the general
form\\
\index{native file format!Property extension}
{\vt 5 {\it prop\_num prop\_string\/};}\\
The property number {\it prop\_num} is an arbitrary integer.  The
property string begins with the first non-space character following
the integer, and ends with the semicolon (the semicolon is not
included).  The string can contain any alphanumeric, punctuation or
white space but not `;' for obvious reasons.  There are a number of
properties used by {\Xic}, particularly in electrical mode.  This
extension is widely used.
\end{enumerate}

{\Xic} writes the electrical information in a second symbol definition
which immediately follows the physical cell definition in native
files, but after the terminating token of the physical cell.
Similarly, when {\Xic} writes a CIF file without the {\cb Strip For
Export} function active, the electrical CIF representation immediately
follows the physical CIF data, after the termination token. 

In {\Xic} releases prior to 3.0.0, the cell terminator was the single
character {\vt E}.  This was used in both native cell files and
unstripped CIF.  In the present release, the cell terminator is always
``{\vt End}'' in CIF, ``{\vt E}'' in native cell files..

Whether or not these extensions are used when writing CIF output is
controlled by a set of flags, which can be individually set from the
{\cb CIF} page of the {\cb Export Control} panel.  Actually, there are
two banks of these flags, one bank is used when {\cb Strip For Export}
is set, the other bank is used when {\cb Strip for Export} is unset. 
In the case of {\cb Strip For Export} set, the flags all default to 0,
so no extensions are used.  In the case of {\cb Strip For Export}
unset, the flags all default to 1, so all extensions are used.

The user can set these flags individually through the {\cb Extension
Flags} menu in the {\cb CIF} page of the {\cb Export Control} panel. 
The bank of flags being set is determined by the state of the {\cb
Strip For Export} button and variable.

The flags in the menu have the following effects.

\begin{description}
\item{\cb scale extension}\\
Traditional CIF has a fixed resolution of 100 units per micron.
This extension will add a comment of the form
\begin{quote}
{\vt (RESOLUTION} {\it nnn\/}{\vt );}
\end{quote}
near the top of the file, and use {\it nnn} as the file resolution. 
The CIF reader must check for this comment and scale numerical values
accordingly.

{\Xic} normally uses internal units in unstripped CIF and native
files, signaled with the addition of a comment line ahead of the first
symbol definition something like:
\begin{quote}
{\vt (RESOLUTION 1000);}
\end{quote}
{\Xic} will look for this comment, and interpret the coordinates
accordingly.  If no comment is found, the CIF default of 100 units per
micron is assumed.  {\Xic} will always use internal units when writing a
CIF file when this extension is enabled, and 100 units otherwise.

\item{\cb cell properties}\\
Properties may be applied to cell definitions, ahead of the {\vt DS}.

\item{\cb inst name comment}\\
Comments of form
\begin{quote}
{\vt (SymbolCall} {\it cellname\/}{\vt );}
\end{quote}
are added ahead of instance `{\vt C}' calls.

\item{\cb inst name extension}\\
Text in the form
\begin{quote}
{\vt 9} {\it cellname}{\vt ;}
\end{quote}
is added ahead of instance `{\vt C}' calls.

\item{\cb inst magn extension}\\
Cell instance `{\vt C}' calls can be preceded by a
magnification extension of the form
\begin{quote}
{\vt 1 Magnify} {\it magn}{\vt ;}
\end{quote}
where {\it magn} is a magnification factor.  All internal structure of
the cell will be scaled by the given factor, which is a floating point
number greater than zero.  This extension will appear in physical cell
descriptions only.  It is unique to {\Xic}.

\item{\cb inst array extension}\\
Cell instance `{\vt C}' calls can be preceded by an array
extension of the form
\begin{quote}
{\vt 1 Array} {\it x dx ny dy}{\vt ;}
\end{quote}
where {\it nx} and {\it ny} are the number of cells to array in the x
and y directions, and {\it dx} and {\it dy} are the spacing between
cells.  This extension was used in earlier CAD programs.

\item{\cb inst bound extension}\\
Cell instance `{\vt C}' calls can be preceded by a bounding box
extension of the form
\begin{quote}
{\vt 1 Bound} {\it left bottom right top}{\vt ;}
\end{quote}
The {\it left}, {\it bottom}, {\it right}, {\it top} are the
coordinates of the parent cell defining the bounding box of the
subcell.  This extension is not currently used, though it is written
into the files.  It is unique to {\Xic}.

\item{\cb inst properties}\\
Properties may be added ahead ahead of instance `{\vt C}' calls.

\item{\cb obj properties}\\
Properties may be added ahead of {\vt B} (boxes), {\vt P} (polygons),
and {\vt W} (wires).

\item{\cb wire extension}\\
The end style of wires is not part of traditional CIF.  In this
extension, text of the form
\begin{quote}
{\vt 1 7033 PATHTYPE} {\it n\/}{\vt ;}
\end{quote}
may be added ahead of wires to specify an end style.  The values of
{\it n} are 0 (flush ends), 1 (rounded ends), or 2 (extended ends, the
default).

This extension was used in {\Xic} prior to 2.5.23.  It has been
superseded by {\cb wire extension new}, which will have precedence if
both extensions are enabled.

\item{\cb wire extension new}\\
This overrides {\cb wire extension}, wires include an end-style
designation:
\begin{quote}
{\vt W0 | W1 | W2}  {\it width} {\it x-y data}{\vt ;}
\end{quote}
The end style of wires is not part of traditional CIF.  In this
extension, the wire end style 0-2 immediately follows the `{\vt W}',
with the rest of the line as in traditional CIF.  The end style is the
same as the GDSII path type:  0 for flush ends, 1 for rounded ends,
and 2 for extended square ends.

This extension was introduced in release 2.5.23.  Older releases
of {\Xic} are not compatible with this extension.

\item{\cb text extension}\\
Label string text is enclosed in {\vt << >>}, and may include white
space.  Without this extension, white space characters in the label
text will be replaced with underscores.  In both cases, semicolons are
replaced with underscores.  This extension applies with any of the
label format choices.
\end{description}

% 072510
\section{Native Cell File Format}
\index{native file format}

Native cell files use a modified CIF format.  Each file can contain
two cell definitions:  one for physical geometry, and the second for
electrical (schematic) data.  The physical information is found first. 
The electrical cell definition is optional, it may be absent or empty. 
An empty physical cell definition is created if there is no physical
information, thus all native cell files produced by {\Xic} will
contain a physical cell definition.  The parser will also recognize
schematic files from the {\et Jspice3} program, which have a similar
format, but no physical cell definition. 

The basic file layout is shown below.

\begin{quote}\vt
(Symbol {\it symbol\_name\/});\\
({\it RCS\_ID\/});\\
({\it program\_version date\/});\\
(PHYSICAL);\\
(RESOLUTION 1000);\\
9 {\it symbol\_name\/};\\
DS 0 1 1;\\
{\it physical data\/} ...\\
DF;\\
E\\
(ELECTRICAL);\\
(RESOLUTION 1000);\\
9 {\it symbol\_name\/};\\
DS 0 1 1;\\
{\it electrical data\/} ...\\
DF;\\
E\\
{\it SPICE listing}
\end{quote}

\index{PHYSICAL keyword}
The first line is a mandatory CIF comment giving the symbol (cell)
name.  The second line is an optional comment providing an ID field to
be used with the RCS/CVS code control programs.  These programs are
used to manage large projects with multiple designers.  Other comment
lines may follow, in particular a comment line containing the creating
program version and creation date is added by {\Xic}.  The next line
is a CIF comment containing the word ``PHYSICAL''.  This indicates
that the following cell definition contains physical data.  If this
line is not found, some time consuming tests are performed to figure
out what exactly is in the file.

\index{RESOLUTION keyword}
An optional ``{\vt (RESOLUTION 1000);}'' comment line follows.  This
indicates that coordinates in the physical part of the file use 1000
units per micron.  If the line is not present, 100 units per micron is
assumed.  This was the default for early versions of {\Xic}, and
follows from the implicit CIF assumption.  The use of resolutions
other than 100 represents an extension of the CIF syntax.

The integer following ``{\vt RESOLUTION}'' can be 100 or any of the
values supported for the {\et DatabaseResolution} variable, for the
physical cell.  For the electrical cell, only the values 100 and 1000
are allowed.

\index{ELECTRICAL keyword}
The electrical part of the file is optional, and starts with a CIF
comment containing the word ``ELECTRICAL'', followed by the resolution
comment and the electrical cell description.  Either cell description
can be empty, i.e., a {\vt DS 0 1 1;} line followed by {\vt DF;} and
{\vt E}.  Finally, if the cell was written in schematic mode and is a
top-level cell (containing no terminal nodes) a SPICE listing of the
circuit is added to the bottom of the cell file.  Such files can be
read directly into the {\WRspice} program for simulation.  The SPICE
listing has no relevance to {\Xic}.

In release 3.1.5 and later, the terminating line of a native cell file
can have ``{\vt n}'' or ``{\vt nd}'' (case insensitive) following the
``{\vt E}'', as in normal CIF.  In earlier releases, anything after
``{\vt E}'' would cause a syntax error.

The format of the Physical cell data adheres to the extended CIF
described in the preceding sections.  Electrical cell descriptions use
the same extensions, however the array extension never appears, as
arrays are not available in electrical mode.  The major difference in
the files is the large number of properties assigned in electrical
mode.


% -----------------------------------------------------------------------------
% cgxformat 061713
\section{Computer Graphics Exchange (CGX) Format}
\label{cgxformat}
\index{CGX format}
The Computer Graphics eXchange (CGX) format is a simple binary data
format somewhat similar to GDSII, but designed to be more compact. 
Like GDSII, files consist of a sequential list of variable-length
records.  It has simplified record structure, but extensions in data
flexibility.  If is more compact than GDSII and is more efficient to
read and write.

The advantages of CGX are smaller files and faster read/write than
GDSII.  This format was developed by Whiteley Research Inc., but is
hereby placed in the public domain without restriction.

The file extension is ``{\vt .cgx}''.  Gzipped files (``{\vt
cgx.gz}'') are supported.  {\Xic} will automatically identify this
file type, and can read, write, and convert to {\Xic} files just as
GDSII.

\subsection{CGX Format Identifier}

The first three bytes of a CGX file are `c', `g', and `x'. The fourth
byte is an integer format level.  A parser designed to handle a
certain level will accept that level and any value lower.  Presently,
the only existing level is 0, thus this byte should be set to 0.

\subsection{CGX Data Types}

CGX uses the same long (4--byte) and short (2--byte) integer formats
as GDSII, and the same 8--byte floating point format.  These are the
only numerical data types defined.

A date is stored as 8 bytes, as shown in the following table.  These
are the same numerical fields as used in GDSII, though the format is
different (bytes are used where possible, rather than shorts).  The
third column gives the value in terms of the members of the {\vt tm}
structure from the C library.

\begin{tabular}{lll}
  short & year     & \vt tm\_year + 1900\\
  byte  & month    & \vt tm\_mon + 1\\
  byte  & day      & \vt tm\_mday\\
  byte  & hour     & \vt tm\_hour\\
  byte  & minute   & \vt tm\_min\\
  byte  & second   & \vt tm\_sec\\
  byte  & 0 &\\
\end{tabular}

Strings are stored in the same manner as in GDSII.  The null
terminator in not written, however a null byte will be added to
strings of odd length, so that record sizes are always even.

\subsection{CGX Data Records}

The four--byte file header is followed by any number of data records,
the last of which signals the end of data.  There are 11 defined
record types.  Each record begins with a 4-byte header:

\begin{tabular}{lll}
short & \vt recsize\\
byte  & \vt rectype\\
byte  & \vt flags\\
\end{tabular}

The {\vt recsize} field is a short unsigned integer giving the total
record size, including the header.  Thus, as in GDSII, records are
limited to 64K bytes in length.  The record size will always be an
even number.  The {\vt rectype} byte is set to a small integer to
define the type of record.  The {\vt flags} byte is used in some of
the record types, otherwise it is ignored.

The defined record types are given in the table below.

\begin{tabular}{ll}
\bf\it rectype & \bf name\\
  0 & \vt LIBRARY\\
  1 & \vt STRUCT\\
  2 & \vt CPRPTY\\
  3 & \vt PROPERTY\\
  4 & \vt LAYER\\
  5 & \vt BOX\\
  6 & \vt POLY\\
  7 & \vt WIRE\\
  8 & \vt TEXT\\
  9 & \vt SREF\\
  10 & \vt ENDLIB\\
\end{tabular}

It is allowable to define additional record types for local or
proprietary purposes.  If a parser encounters an unknown record type,
it may skip over the record, ignoring it.

\paragraph{{\vt LIBRARY} record}

The {\vt LIBRARY} record should be the first data record in the file,
and can appear once only.

The {\vt flags} byte of the record header can be used for a version
number, which identifies in some way the remaining data in the file.

The {\vt LIBRARY} record contains the following fields:

\begin{tabular}{lll}
\kb bytes & \kb field name & \kb purpose\\
8 & \vt  munit    &     machine units\\
8 & \vt  uunit    &     user units\\
8 & \vt  cdate    &     library creation date\\
8 & \vt  mdate    &     library modification date\\
? & \vt  libname  &     library name string\\
\end{tabular}

The first two fields are double-precision numbers that define the
scale factors for the data in the file.  These are interpreted in the
same way as the similar fields in the header of a GDSII file.

The second two fields represent creation and modification dates for
the file content.

A name string for the library follows.  Strings are null-byte
terminated, and an additional null byte is added if necessary so that
the total length is even.

\paragraph{{\vt STRUCT} record}

The {\vt STRUCT} record opens a cell structure.  Records that follow
will be assigned to that cell, until another {\vt STRUCT} record is
seen.

The header {\vt flags} byte is not used.

The {\vt STRUCT} record contains the following fields:

\begin{tabular}{lll}
\kb bytes & \kb field name & \kb purpose\\
8 & \vt cdate    &    creation date\\
8 & \vt mdate    &    modification date\\
? & \vt strname  &    structure name string\\
\end{tabular}

The first two fields provide creation and modification dates for the
structure.  These are followed by a string giving a name for the
structure.  This name should be unique in the file.

\paragraph{{\vt CPRPTY} record}

Zero or more {\vt CPRPTY} records can appear following a {\vt STRUCT}
record.  These are properties that are applied to the cell.

The header {\vt flags} byte is not used.

The {\vt CPRPTY} record contains the following fields:

\begin{tabular}{lll}
\kb bytes & \kb field name & \kb purpose\\
4 & \vt number    &   property number\\
? & \vt string    &   property string\\
\end{tabular}

Any number or string is allowed.

\paragraph{{\vt PROPERTY} record}

Zero or more {\vt PROPERTY} records can appear ahead of {\vt BOX},
{\vt POLY}, {\vt WIRE}, {\vt TEXT}, and {\vt SREF} records.  It
assigns a property to the object that follows.

The header {\vt flags} byte is not used.

The {\vt PROPERTY} record contains the following fields:

\begin{tabular}{lll}
\kb bytes & \kb field name & \kb purpose\\
4 & \vt number   &    property number\\
? & \vt string   &    property string\\
\end{tabular}

Any number or string is allowed.

\paragraph{{\vt LAYER} record}

A {\vt LAYER} record can appear after a {\vt STRUCT}, and must appear
before any of {\vt BOX}, {\vt POLY}, {\vt WIRE}, {\vt TEXT} in the
{\vt STRUCT}.  The layer context will persist until the next {\vt
LAYER} or {\vt STRUCT} record.

The header {\vt flags} byte is not used.

The {\vt LAYER} record contains the following fields:

\begin{tabular}{lll}
\kb bytes & \kb field name & \kb purpose\\
2 & \vt number     &   layer number\\
2 & \vt datatype   &   data type\\
? & [{\vt lname}]  &   optional layer name\\
\end{tabular}

The layer number and data type are sufficient, and have the same
interpretation as in GDSII.  Alternatively or in addition, a
string giving a layer name can be supplied.

\paragraph{{\vt BOX} record}

A {\vt BOX} record can appear after a {\vt LAYER} record has been
issued.  The {\vt BOX} record defines one or more rectangular data
objects.

The header {\vt flags} byte is not used.

The {\vt BOX} record contains the following fields:

\begin{tabular}{lll}
\kb bytes & \kb field name & \kb purpose\\
4 & \vt left        &  left value\\
4 & \vt bottom      &  bottom value\\
4 & \vt right       &  right value\\
4 & \vt top         &  top value\\
? & [{\vt repeat\/}] & repeat for multiple boxes\\
\end{tabular}

The first four integers define a box, and a record can contain
multiple box definitions (four integers per box).  Each box is given
the properties currently in effect, and is assigned to the layer
currently in effect.

\paragraph{{\vt POLY} record}

A {\vt POLY} record can appear after a {\vt LAYER} record has been
issued.  The {\vt POLY} record defines a polygon object.

The header {\vt flags} byte is not used.

The {\vt POLY} record contains the following fields:

\begin{tabular}{lll}
\kb bytes & \kb field name & \kb purpose\\
? & \vt xy   &    coordinate pairs, path must be closed\\
\end{tabular}

Coordinates use four-byte integers.  The first and last coordinate
pair (x--y values) must be the same.  There must be at least four
coordinate pairs.

\paragraph{{\vt WIRE} record}

A {\vt WIRE} record can appear after a {\vt LAYER} record has been
issued.  A {\vt WIRE} record specifies a single wire (path) data
object.

The header {\vt flags} field contains a value in the range 0--2 which
sets the end style of the wire:

\begin{tabular}{ll}
   0 & flush ends\\
   1 & rounded ends\\
   2 & extended square ends\\
\end{tabular}

This is the same as the pathtype in GDSII.

The {\vt WIRE} record contains the following fields:

\begin{tabular}{lll}
\kb bytes & \kb field name & \kb purpose\\
4 & \vt  width  &  path width\\
? & \vt  xy  &  coordinate pairs (1 pair or more)\\
\end{tabular}

\paragraph{{\vt TEXT} record}

A {\vt TEXT} record can appear after a {\vt LAYER} record has been
issued.  A {\vt TEXT} record specifies a non-physical text object.

The header {\vt flags} byte is an orientation code:

\begin{tabular}{ll}
bits 0-1  & rotate the text about the anchor\\
          & 00 no rotation\\
          & 01 90 degrees\\
          & 10 180 degrees\\
          & 11 270 degrees\\
bit 2     & mirror y after rotation\\
bit 3     & shift rotations to 45, 135, 225, 315 degrees\\
bits 4-5  & horizontal justification, 00 left, 01 center, 10,11 right\\
bits 6-7  & vertical justification, 00 bottom, 01 center, 10,11 top\\
\end{tabular}

Note that this is identical to the lowest byte of the {\Xic}
label flags (see \ref{labelflags}).

The {\vt TEXT} record contains the following fields:

\begin{tabular}{lll}
\kb bytes & \kb field name & \kb purpose\\
4 & \it x       &  x position\\
4 & \it y       &  y position\\
4 & \it width   &  field width\\
? & \it label   &  label text\\
1 & \it flags   &  additional flags\\
\end{tabular}

The {\it width} gives the physical equivalent width of the text.  The
height is determined by the font used for rendering.

In order to accommodate additional flags, a flags byte is ``hidden''
behind the label text.  The flags byte follows the null byte that
terminates the text string.  A null byte may be added following the
flags byte to make the total byte count even.  The flag bits are:

\begin{tabular}{ll}
\vt SHOW & 0x1\\
\vt HIDE & 0x2\\
\vt TLEV & 0x4\\
\vt LIML & 0x8\\
\end{tabular}
 
These flags are described with the {\et XprpXform} pseudo-property in
\ref{pseudoprops}.

\paragraph{{\vt SREF} record}

The {\vt SREF} record describes an instance, or an array of instances. 

The header {\vt flags} byte can have any of the following bits set.

\begin{tabular}{ll}
\vt ANGLE   & 0x1\\
\vt MAGN    & 0x2\\
\vt REFLECT & 0x4\\
\vt ARRAY   & 0x8\\
\end{tabular}

The {\vt SREF} record contains the following fields:

\begin{tabular}{lll}
\kb bytes & \kb field name & \kb purpose\\
4 & \vt  x      & x coordinate\\
4 & \vt  y      & y coordinate\\
8 & \vt  angle  & rotation angle, if {\vt ANGLE} flag only\\
8 & \vt  magnif & magnification, if {\vt MAGN} flag only\\
4 & \vt  cols   & array columns, if {\vt ARRAY} flag only\\
4 & \vt  rows   & array rows, if {\vt ARRAY} flag only\\
16 & \vt xy[4]  & aref points (like GDSII), if {\vt ARRAY} flag only\\
? & \vt  sname  & referenced structure name\\
\end{tabular}

If the {\vt ANGLE} flag is set, the cell is to be rotated by an
angle, in degrees, found in the record.  If the {\vt MAGN} bit is
set, the cell is scaled by a value found in the record.  If the {\vt
REFLECT} bit is set, the instance will be reflected about the
x--axis, as in GDSII.  If the {\vt ARRAY} bit is set, the instance
is arrayed, as in GDSII, where x, y, and xy give the three orientation
points, as in a GDSII AREF record.  Unless the corresponding bit is
set, the corresponding data are not in the record.
  
\paragraph{{\vt ENDLIB} record}

The {\vt ENDLIB} record must be the last record of the file.  It
contains no data.


% -----------------------------------------------------------------------------
% oasis 022916
\section{OASIS Format}

As integrated circuit mask layouts inexorably increase in complexity,
the fundamental limitations of the industry standard GDSII file format
have become a bottleneck.  A major weakness of the GDSII format is
inefficient data representation, which leads to very large files. 
File sizes of tens of gigabytes are not uncommon, leading to
difficulties in transmission, data integrity, and consumption of
hardware resources.

The Open Artwork System Interchange Standard (OASIS) was designed by
the SEMI consortium ({\vt http://www.semi.org}) as a modern
alternative to the GDSII standard.  A draft specification (SEMI
Document 3626 2003/04/23) of the OASIS format standard was circulated,
and subsequently adopted with very minor changes (SEMI P39-1105).  The
final standard document is available from the SEMI organization.

The main objective of the OASIS standard is efficient representation
of mask layout geometry, both in hierarchical and flat
representations.  The format makes use of a number of techniques to
this end.

\begin{itemize}
\item{A compact variable-size integer representation is used.  Along
with heavy use of offsets, one and two byte integers can be used
extensively in place of the larger fixed-size integers used in other
formats.}

\item{Extensive use of modal variables greatly reduces repeated
information.}

\item{String and name referencing by number eliminates repetition of
these data.}

\item{A flags byte indicates the presence or absence of certain data
fields in most records, so that unused or unset values do not need
to be included in the stream.}

\item{Special compact representations for trapezoids and other
common features save space.}

\item{An encoding mechanism for repetitions can be used to
consolidate arrays of objects.}

\item{A data compression mechanism is supported.}
\end{itemize}

As a ``typical'' example, the sizes in the table below illustrate the
space-saving capability of the OASIS format.  This lists the size of a
GDSII file, and the size of the resulting OASIS file as converted with
{\Xic} with the main available options.

\begin{tabular}{ll}
\bf File & \bf Size (bytes)\\
GDSII file &       7669760\\
OASIS, basic &     1643804\\
plus repetitions & 1153071\\
plus name tables & 1067157\\
plus compression & 816225\\
\end{tabular}

\subsection{OASIS Support in {\Xic}}

{\Xic} was the first (to our knowledge) commercial implementation of
the OASIS format.  Some limited tools have been made available from
Mentor Graphics (GDSII/OASIS translator), and SoftJin (GDSII/OASIS
translator and text mode converters).  We recommend {\vt anuvad} from
SoftJin \newline({\vt http://www.softjin.com/html/anuvad.htm}), which
has been our ``reference'' in establishing portability.

This capability was designed from the draft SEMI-3626 document, but
has incorporated changes from the final specification.

This section describes the OASIS capabilities in {\Xic}.  The present
status of OASIS support in {\Xic} is complete, the bottom line being

\begin{enumerate}
\item{{\Xic} can read any spec-conforming OASIS file.}
\item{OASIS output from {\Xic} is readable by any other
    spec-conforming tool.}
\item{Exceptions to the above are {\bf bugs}, please report!}
\end{enumerate}

OASIS is one of the supported archive formats, along with GDSII, CIF,
and CGX.  CGX (Computer Graphics eXchange) format is another
``improved'' GDSII developed and placed in the public domain by
Whiteley Research.  The archive formats have the following
capabilities in {\Xic}:

\begin{itemize}
\item{Files can be read directly into {\Xic}, either using the {\cb
Open} command, or with similar buttons and functions in {\Xic}.}

\item{Files can be converted directly to another (or the same) archive
format, or to {\Xic} native cell files, from the {\cb Format Conversion}
function in the {\cb Convert Menu}.  While converting, scaling,
windowing (clipped or not) and flattening can be employed.  There is
also provision for selecting the layers to convert.}

\item{{\Xic} can output a hierarchy in memory to any of the
archive formats.  The default format is the format of origin, if
any.}

\item{The random access of cells from the file, such as with the {\cb
Contents} function of the {\cb Files Listing} or the library access
mechanism applies to all archive formats.}

\item{The Cell Hierarchy Digest (CHD) and Cell Geometry Digest (CGD)
features, which facilitate working with very large files (too large to
fit into the main memory database) apply to all archive formats.}

\item{The script function that splits a file spatially into pieces,
{\vt ChdWriteSplit} applies to all archive formats for both input and
output.}
\end{itemize}

\subsection{Characteristics of OASIS Output From {\Xic}}

The basic OASIS file generated by {\Xic} has the characteristics
listed below.

\begin{itemize}
\item{By default, all strings are saved locally as strings, i.e., no
indirection is used, so there are no {\vt <name>} records.  This can
be changed with the {\et OasWriteNameTab} variable which is connected
to {\cb Use string tables} check boxes in the {\cb Format Conversion}
and {\cb Export Control} panels.}

\item{By default, no REPETITION records are generated for {\vt
<geometry>} records.  If the {\et OasWriteRep} variable or the
corresponding check box is set, REPETITION record types may be
generated.  This option attempts to recognize arrays of identical
objects when writing OASIS files.}

\item{By default, appropriate three and four-sided polygons will be
written as TRAPEZOID or CTRAPEZOID records, however this can be
disabled with the {\et OasWriteNoTrapezoids} variable.}

\item{By default, wires (paths) will retain that data type.  However,
rectangular two-vertex paths will be converted to a more compact
rectangle representation if {\et OasWriteWireToBox} is set.}

\item{The following record types are not generated by {\Xic}: 
CIRCLE, XNAME, XELEMENT, XGEOMETRY.}

\item{When writing OASIS files with {\et StripForExport} set, i.e.,
writing physical data only, and when using string tables, the offset
table is placed in the END record.  With {\et StripForExport} not set,
in general we write two sequential OASIS databases into the file, the
first for physical data, the second for electrical.  This is a
{\Xic}-specific extension.  In this case, string tables are used in
the physical part only, and the offset table is placed in the START
record.  PAD records are added to avoid overwriting data since this is
a non-sequential operation.  In all cases, strict-mode tables are
used.

Note:  If a design contains physical data only, the electrical records
are absent, so that the file becomes conventional.  Even if electrical
records are present, the reader will probably ignore them (as does
{\vt anuvad-0.2}).  However, when exporting physical data, for
portability {\et StripForExport} should always be set.

The string tables themselves are written just ahead of the END
record in all cases (when tables are used).}

\item{OASIS files generated by {\Xic} release 3.2.2 and later have a
file property named ``XIC\_SOURCE'', with no content.  This identifies
the file as originating from {\Xic} or a derivative.}

\item{All integer values are 32-bit limited, except for values that
represent offsets into the file, which may be 64-bit.}

\item{The OASIS format does not provide a native code to indicate a
rounded-end wire.  For wires that have rounded ends, i.e., that
originated as GDSII PATHTYPE=1, the half-width extension is specified,
and the PATH record is given an empty (info byte = 0x4) property named
``XIC\_ROUNDED\_END''.}

\item{The OASIS format does not provide codes for TEXT element
presentation.  In {\Xic}, these are used for on-screen labels, and are
treated by {\Xic} as any other database object, but they will not
appear on the mask layout.  Thus, at least for {\Xic} internal use,
TEXT presentation attributes are important.  They are stored in a
property applied to TEXT records named ``XIC\_LABEL''.  The XIC\_LABEL
property contains two unsigned integers.  The first is the width of
the label, in database units.  The second is the label flags word used
by {\Xic} which specifies many presentation attributes.  See
\ref{labelflags} for more information.}

\item{OASIS text labels can contain only printable ASCII characters
and the space character, thus some trickery is used to support
multi-line labels.  In OASIS files generated from {\Xic}, the
following non-printing characters are replaced with the indicated
character sequence when encountered in label text:
\begin{quote}
\begin{tabular}{ll}
{\vt 0xa} (line feed) & ``{\vt $\backslash$n}''\\
{\vt 0xd} (carriage return) & ``{\vt $\backslash$r}''\\
{\vt 0x9} (tab) & ``{\vt $\backslash$t}''\\
\end{tabular}
\end{quote}

The OASIS reader will perform the reverse conversion, if the
XIC\_SOURCE property is found in the file, meaning that it was written
by {\Xic}.}

\item{All other properties, which might be given to CELL or {\vt
<geometry>} records, are named \newline``XIC\_PROPERTY'' and consist of
concatenated number/string pairs.  {\Xic} uses properties indexed by a
number, with string-type data, so that the XIC\_PROPERTY consists of
the list of properties as known to {\Xic} for that cell or object.}
\end{itemize}

There are several options in {\Xic} that modify OASIS input/output. 
Many of these can be controlled by check boxes in the {\cb OASIS} page
in the {\cb Export Control} and {\cb Format Conversion} panels from
the {\cb Convert Menu}, which reflect the status of the variables
(which can also be set with the {\cb !set} command or equivalent). 

\begin{tabular}{ll}
\multicolumn{2}{l}{\bf Convert Menu - Input and ASCII Output}\\
\et OasReadNoChecksum & Ignore checksum in OASIS input file\\
\et OasPrintNoWrap & Use one line per record in OASIS ASCII output\\
\et OasPrintOffset & Add file offsets to OASIS ASCII output\\ \\

\multicolumn{2}{l}{\bf Convert Menu - Output}\\
\et OasWriteCompressed & Compress records in OASIS output\\
\et OasWriteNameTab & Use string table referencing in OASIS output\\
\et OasWriteRep & Try to combine similar objects in OASIS output\\
\et OasWriteChecksum & Compute and add checksum to OASIS output\\
\et OasWriteNoTrapezoids & Don't convert polys to trapezoids\\
\et OasWriteWireToBox & Convert wires to boxes when possible\\
\et OasWriteNoGCDcheck & Don't look for common divisors in repetitions\\
\et OasWriteUseFastSort & Use faster but less effective sorting\\
\et OasWriteNoXicTextPrps & Don't write certain text properties\\
\end{tabular}

\subsection{Requirements And Limitations for Reading OASIS}

{\Xic} can very likely read any OASIS file that meets the published
specification.  Exceptions should be reported as bugs!

\begin{itemize}
\item{Properties are ignored, unless the name matches one of those
understood by {\Xic} (see above).  The file properties set an internal
variable but otherwise do nothing.}

\item{The XNAME, XELEMENT, and XGEOMETRY records are ignored.}

\item{The CIRCLE record will create a polygon object approximating a
circle, with the number of sides using the internal variable in
{\Xic}.}

\item{The TRAPEZOID and CTRAPEZOID records will create a polygon object.}

\item{The REPETITION records found in PLACEMENT records will define a
cell array if possible (i.e., represents a periodic Manhattan
configuration), otherwise individual cell instances will be created
and replicated.  In {\vt <geometry>} records, any REPETITION is
accepted, but the repetition is decomposed and separate objects are
created in memory.}
\end{itemize}


% -----------------------------------------------------------------------------
% libraries 041215
\section{Library Files}
\label{libfile}
\index{library files}
Library files are {\Xic} input files which contain references to
cells, other libraries, or cell definitions.  The format of a library
file is as follows:

\begin{quote}\vt
(Library {\it libname\/});\\
{\vt \#} {\it any comments}\\
\\
{\vt \#} {\it optional keywords to implement conditional flow}\\
{\vt Define} [{\vt eval}] {\it name} [{\it value\/}]\\
{\vt If} {\it expression}\\
{\vt IfDef} {\it name}\\
{\vt IfnDef} {\it name}\\
{\vt Else}\\
{\vt Endif}\\
\\
Property {\it number string}\\
...\\
Alias {\it alias refname}\\
...\\
Reference {\it name path} [{\it cellname\/}]\\
...\\
Directory {\it path}\\
...\\
(Symbol {\it symname\/});\\
{\it symbol definition}\\
E\\
...\\
\end{quote}

The first line must begin with ``{\vt (Library }'', which designates a
library file to {\Xic}.  The {\it libname} on this line following {\vt
Library} is ignored, but by convention is the library file name. 
Within the file are three kinds of data fields:  properties,
references, and cells.  Any line starting with a pound sign (`\#') is
taken as a comment and ignored.  Blank lines are ignored.

It is recommended that library files be given a ``{\vt .lib}''
extension.  This is not a strict requirement, except that the listing
of libraries from the search path provided in the {\cb Libraries List}
button in the {\cb File Menu} will contain only files with this
extension.

All library files (including the device library) support a limited
macro capability.  The macro capability makes use of the generic macro
preprocessor provided in {\Xic}, which is described in \ref{macro}. 
The reader should refer to this section for a full description of the
preprocessor capabilities.  The preprocessor provides a few predefined
macros used for testing (and customizing for) release number,
operating system, etc.  The keyword names, which correspond to the
generic names as described for the macro preprocessor, are
case-insensitive and listed in the following table.

\begin{tabular}{ll}
\bf Keyword & \bf Function\\
\vt Define & Define a macro.\\
\vt If & Conditional evaluated test.\\
\vt IfDef & Conditional definition test.\\
\vt IfnDef & Conditional non-definition test.\\
\vt Else & Conditional else clause.\\
\vt Endif & Conditional end clause.\\
\end{tabular}

These can be used to conditionally determine which parts of the file
are actually loaded when the library is read.  Presently, there is no
macro expansion or text substitution in lines of text in the library,
the macros simply implement flow control.  Otherwise, they work the
same as similar keywords in the technology file (see \ref{techmacros})
and in scripts (see \ref{preproc}), and are reminiscent of the
preprocessor directives in the C/C++ programming language.

Properties are used in the device library file (which is a special
library file which must exist in order to use electrical mode), and
are described in the description of the device library file format.

Aliases provide alternative names by which data records can be
obtained from the library.  In particular, for the device library,
this facilitates accessing library devices under alternative names. 
For example, in older device libraries, the terminal device was named
``{\vt vcc}'', while the present name of a similar terminal is ``{\vt
tbar}''.  The addition of

\begin{quote}
{\vt Alias vcc tbar}
\end{quote}

will satisfy references to the {\vt vcc} terminal device in older
designs.

References associate a name with a cell, or another library.  For a
cell, {\it name} (above) is the name by which the cell will be added
to the database when opened, and the name that will appear in
selection listings.  The {\it path} is a path to the file containing
the cell, which can be native ({\Xic}), or a path to an archive file
containing the cell.  If the path contains white space characters, it
should be single or double quoted.

Aliases may be used to provide alternative names.

If the path points to an archive file, the {\it cellname} argument can
be set to the name of the cell in the file.  Note that this does not
have to be the same as {\it name}.  Opening {\it name} will open the
cell referenced and add it to the database as {\it name}.  Any
subcells that have references in the same library file will be opened
under the library reference name.  All other cell name aliasing is
suppressed, except for {\et AutoRename}.

If {\it cellname} is not given, opening a reference to an archive file
with multiple cells will cause a pop-up to appear, allowing the user
to choose which cell to open.  In this case, the cell will be opened
under its own name.

If {\it path} points to another library file, then {\it cellname}, if
given, indicates which reference in the library to open, i.e., it
should be one of the {\it name\/}s in the referenced library.  In this
case, the cell will be opened as {\it name} in the original library. 
If {\it cellname} is not given, a pop-up will appear allowing the user
to choose which library element in the referenced library to open.  A
cell selected in this way will be opened as {\it name} in the
referenced library.  Thus the {\vt Reference} keyword provides a means
for multiple-level indirection through the library files.

The {\vt Directory} keyword is followed by a full path to a directory. 
Every layout or library file found in the directory is logically added
as a {\vt Reference}, but with no {\it cellname} given.  This keyword
provides an easy way to reference a collection of cell files, for
example.

Cells can be defined within libraries by including the native-format
body in the library file.  The first line of the cell must start with
``{\vt (Symbol }''.  The symbol text should contain both the
electrical and physical blocks.  The cells in the device library
file are special in that they contain only an electrical block, so are
not representative.  Cells can be added to a library with a text
editor, by copying from native cell files.  The name of the cell is
actually given by the lines with format like ``{\vt 9} {\it
symbolname\/};'' and the {\it symbolname} in the ``{\vt (Symbol} {\it
symbolname\/}{\vt );}'' is actually ignored.  The user need not
concern themselves with details of the format, it is sufficient to
simply copy the entire {\Xic} cell file into the library, however any
trailing SPICE listing should be excluded, including the ``{\vt
*Generated by Xic}...'' line.

\index{ example library file}
\subsection{Example Library File}
The example below uses the {\vt Reference} directive only, which is
common.  It illustrates some of the types of references that are
possible.

\begin{verbatim}
(Library demo.lib);

# simple reference to native cell
Reference acell            /usr/.../xic_cells/acell

# simple reference to native cell, with name change
Reference buffer           /usr/.../xic_cells/cell32

# browsable reference to a GDSII file
Reference gdsfile.gds      /usr/.../gdsfile.gds

# reference to cell in GDSII file, with name change
Reference mux8_1           /usr/.../gdsfile.gds      MUX

# reference to cell in CIF file
Reference and5             /usr/.../ciffile.cif      and5

# browsable reference to another library
Reference stdcells_25.lib  /usr/.../stdcells_25.lib

# indirect references to cells in another library
Reference orgate_25        /usr/.../stdcells_25.lib  orgate
Reference andgate_25       /usr/.../stdcells_25.lib  andgate

# Reference to all layout files and libraries found in a directory
Directory                  /home/joe/devices
\end{verbatim}


% -----------------------------------------------------------------------------
% devlib 041715
\section{Device Library File}
\label{devdotlib}
\index{device.lib file}
\index{device library file}
The device library file is a special library file which contains all
of the information required to render and otherwise support the
devices available in the electrical mode of {\Xic}.  It is expected to
be found along the library search path.  The search is always
performed in the current directory first, whether or not this is
indicated by the search path.  The default name for this file is
``{\vt device.lib}'', however this name can be changed with the {\vt
DeviceLibrary} keyword in the technology file.  Only one device
library is used, and the first file found in the search path with a
matching name is read.

In present {\Xic} releases, devices are not required to be supplied in
the device library, however only devices referenced in the library
will appear in the device selection menus.  When loading a design
produced in Virtuoso, for example, the schematic symbols are imported
as cells which function as devices, but are not automatically included
in any device library.  As long as {\Xic} can find these device cells
when the layout file is loaded, all is well.  This is automatic if the
device cells are kept in the layout files, which is true by default. 
However, the device cells can be added to the device library in
various ways, as described below.  This has the following advantages:

\begin{enumerate}
\item{The devices are available in the device menus available from the
side menu in electrical mode.  This facilitates use of the device in
other designs.}

\item{The devices are no longer included in the layout files, reducing
their size.  However, the library must be provided along with the
files when exporting the design.}
\end{enumerate}

Devices can be either primitive devices as used by SPICE, or
subcircuit macros.  If the device represents a subcircuit macro, the
name of the subcircuit is given as a {\et model} property, and that
subcircuit must exist in one of the model library files.  For example,
a device named ``opamp'' could be added to the device library file. 
Then the user would set the {\et model} property to something like
``ua741'' which would have a subcircuit definition somewhere in the
model library files (perhaps in a directory containing SPICE models
obtained from a semiconductor manufacturer).

There are four classes of device that may appear in the device library
file.  The first class consists of basic elements such as resistors,
capacitors, and semiconductor devices which have physical
implementations in a layout and are known elements in SPICE.  The
second class consists of voltage and current sources, which are known
elements in SPICE but do not have physical equivalents in a layout. 
The third class applies to macros, which expand to a subcircuit in
SPICE.  These may or may not have an actual physical embodiment.  The
fourth class are terminals, which are used in the electrical schematic
to provide connections.  These are not used in SPICE, but are used to
establish connectivity when producing SPICE input.  They have no
direct physical implementation, but imply physical connections.

The first line of the file must be in the form
\begin{quote}\vt
(Library {\it filename});
\end{quote}
This is the signature used in all library files.

\index{device library file!comments in}
Comment lines, which are ignored when the file is parsed, begin with
the `{\vt \#}' character, and can appear anywhere outside of the
device definitions except on the first line.  Lines containing only
white space are ignored.

\subsection{Device Library Global Properties}
\label{devglobprop}
\index{global properties}
\index{device library file!properties}

The device library file handles ``global properties''.  These
properties appear at the beginning of the file, after the initial line
but before the definitions.  The syntax is
\begin{quote}\vt
Property {\it identifier string}
\end{quote}
where {\vt Property} appears literally, {\it identifier} is a keyword
or equivalent integer as described below, and the rest of the line
constitutes the {\it string}.  There can be any number of these lines.

The following properties are recognized:

\begin{description}
\index{SpiceDotSave property}
\item{\vt SpiceDotSave}
This property is identified by the keyword ``{\vt SpiceDotSave}" or by
the integer 20.

The {\it string} consists of a SPICE key letter for a device (such as
`{\vt R}' for a resistor), followed by the name of a parameter known
to SPICE for that device.  While a SPICE deck is being created, and if
this property was given, each device in the circuit that is keyed by
that letter will trigger the addition of a line in the SPICE file in
the form
\begin{quote}
{\vt .save @}{\it name\/}[{\it param\/}]
\end{quote}
The {\it name} is the name of the device, and the {\it param} is the
parameter name given in the property.  This construct forms a vector
name which the directive ensures will be saved during simulation, and
thus be available for output.  This is the means by which device
parameter data are made available {\it by default} in SPICE runs
initiated from SPICE output generated by {\Xic}.  {\WRspice} and other
SPICE3-derivative simulators will recognize this form, however only
{\WRspice} will actually save the vector in interactive mode.  SPICE3
ignores {\vt .save} lines, except in batch mode.

This property is used in the supplied {\vt device.lib} file, for
current sources and the ``{\vt c}'' (current) parameter.  The {\cb
branch} property for current sources references ``{\vt @}{\it
name\/}{\vt [c]}'', so that it is important to ensure that this vector
is saved.  Thus, the appropriate global property is
\begin{quote}
{\vt Property SpiceDotSave I c}\\
or equivalently\\
{\vt Property 20 I c}
\end{quote}
This will produce lines in the SPICE output like
\begin{quote}
{\vt .save @isource[c]}
\end{quote}
for a current source named {\vt isource}.

\index{DefaultNode property}
\item{\vt DefaultNode}\\
This property is identified by the keyword ``{\vt DefaultNode}''
or by the integer 21.

This property is used for providing a default node name for the last
node listed in a SPICE output device line.  This allows the use of a
three-node MOS device, with the substrate node connected
automatically.  The feature is enabled by adding the following
property line at the top of the device library file:

\begin{quote}
{\vt Property DefaultNode} {\it device\_name num\_nodes node\_name}\\
or equivalently\\
{\vt Property 21} {\it device\_name num\_nodes node\_name}
\end{quote}

The parameters are:
\begin{quote}
\begin{tabular}{ll}
\it device\_name: & name of device (e.g., {\vt nmos})\\
\it num\_nodes:   & number of nodes expected by SPICE\\
\it node\_name:   & name of node to be added\\
\end{tabular}
\end{quote}

For example,
\begin{quote}
{\vt Property 21 nmos 4 NSUB}
\end{quote}

A {\vt Property} line should be added for each device which has a default
node.  The respective device descriptions in the device library file
should also be modified to remove the substrate mode.  The supplied {\vt
device.lib} file contains MOS models with this feature included,
and also standard models.

Using the example above, a SPICE output deck will contain lines like
\begin{quote}
    {\vt M1 1 2 3 NSUB ...}
\end{quote}
Also, there will be a line added at the top of the deck:
\begin{quote}
    {\vt .global NSUB}
\end{quote}
This line tells {\WRspice} to not modify this node name during
subcircuit expansion.  The user must explicitly add a connection to
the global node, usually to a voltage source.  This can be
accomplished in {\Xic} by placing a terminal device, and modifying the
terminal name to the node name (NSUB).

\index{DeviceKey property}
\item{\vt DeviceKey}\\
This property is identified by the keyword ``{\vt DeviceKey}'' or by
the integer 22.

{\bf Although still recognized, this property is obsolete and should
not be used.} The {\vt DeviceKeyV2} property syould be used instead.

There is an internal table of mappings from letters to devices, in
accordance with the definitions and traditions of SPICE.  For example,
`{\vt r}' (case insensitive) maps to a resistor device.  It is
possible to define new device keys, overriding the defaults.  It is
also possible to define multi-letter keys.

These keys apply when {\Xic} reads a SPICE file and maps devices to
those found in the {\vt device.lib} file.

The format for the property specification is
\begin{quote}
{\vt Property DeviceKey} {\it prefix opt val nnodes nname pname}\\
or equivalently\\
{\vt Property 22} {\it prefix opt val nnodes nname pname}
\end{quote}

\begin{description}
\item{\it prefix}\\
This is a short (usually single-character) device identification prefix,
the first character if which must be a letter.
\item{\it opt}\\
This is a binary value, the token can be {\vt 0}, {\vt no}, or {\vt
off} if unset, or {\vt 1}, {\vt yes}, or {\vt on} if set.  If set,
then the presence of the last connection node of the device is
optional (such as for a BJT, which has an optional substrate node).
\item{\it val}\\
This is a binary value as above.  If set, text following the nodes is
saved in a {\cb value} property and the {\cb model} property is unset,
as for voltage/current source devices.
\item{\it nnodes}\\
An integer giving the number of device nodes, including the optional
node if any.
\item{\it nname}\\
The device name, or the n-type device, in the library.
\item{\it pname}\\
If this is not 0 or missing, it is the name of the p-type library
device.
\end{description}

\index{DeviceKeyV2 property}
\item{\vt DeviceKeyV2}\\
This property is identified by the keyword ``{\vt DeviceKeyV2}'' or by
the integer 23.

This is an extended version of the now-obsolete {\vt DeviceKey}
property that supports current-controlled sources and switch, and will
allow any number of optional nodes.  The older format is still
recognized, but can not be used to create standard keys for these
devices.

The format for the property specification is
\begin{quote}
{\vt Property DeviceKeyV2} {\it prefix min max devs val nname pname}\\
or equivalently\\
{\vt Property 23} {\it prefix min max devs val nname pname}
\end{quote}

\begin{description}
\item{\it prefix}\\
This is a short (usually single-character) device identification prefix,
the first character if which must be a letter.
\item{\it min}\\
This is the minimum number of nodes used by the device.
\item{\it max}\\
This is the maximum number of nodes used by the device.  The
difference from the {\it min} is the number of optional nodes.
\item{\it devs}\\
This is the number of device reference names, which is one for
current-controlled sources and switch.
\item{\it val}\\
This is a binary value, the token can be {\vt 0}, {\vt no}, or {\vt
off} if unset, or {\vt 1}, {\vt yes}, or {\vt on} if set.  If set,
text following the nodes is saved in a {\cb value} property and the
{\cb model} property is unset, as for voltage/current source devices.
\item{\it nname}\\
The device name, or the n-type device, in the library.
\item{\it pname}\\
If this is not 0 or missing, it is the name of the p-type library
device.
\end{description}

The internal table provides the following defaults.

\begin{tabular}{|l|l|l|l|l|l|l|}\hline
\it prefix & \it min & ]it max & \it devs & \it val & \it nname & \it pname\\
  \hline
\vt a & 2 & 2 & 0 & true &  \vt vsrc &     0\\ \hline
\vt b & 2 & 3 & 0 & false & \vt jj   &     0\\ \hline
\vt c & 2 & 2 & 0 & false & \vt cap  &     0\\ \hline
\vt d & 2 & 2 & 0 & false & \vt dio  &     0\\ \hline
\vt e & 2 & 4 & 4 & true  & \vt vcvs &     0\\ \hline
\vt f & 2 & 2 & 1 & true  & \vt cccs &     0\\ \hline
\vt g & 4 & 4 & 0 & true  & \vt vccs &     0\\ \hline
\vt h & 2 & 2 & 1 & true  & \vt ccvs &     0\\ \hline
\vt i & 2 & 2 & 0 & true  & \vt isrc &     0\\ \hline
\vt j & 3 & 3 & 0 & false & \vt njf  & \vt pjf\\ \hline
\vt k & 0 & 0 & 2 & false &     0    &     0\\ \hline
\vt l & 2 & 2 & 0 & false & \vt ind  &     0\\ \hline
\vt m & 4 & 4 & 0 & false & \vt nmos & \vt pmos\\ \hline
\vt n & 0 & 0 & 0 & false &     0    &     0\\ \hline
\vt o & 4 & 4 & 0 & false & \vt ltra &     0\\ \hline
\vt p & 0 & 0 & 0 & false &     0    &     0\\ \hline
\vt q & 3 & 4 & 0 & false & \vt npn  & \vt pnp\\ \hline
\vt r & 2 & 2 & 0 & false & \vt res  &     0\\ \hline
\vt s & 4 & 4 & 0 & false & \vt sw   &     0\\ \hline
\vt t & 4 & 4 & 0 & false & \vt tra  &     0\\ \hline
\vt u & 4 & 4 & 0 & false & \vt urc  &     0\\ \hline
\vt v & 2 & 2 & 0 & true  & \vt vsrc &     0\\ \hline
\vt w & 2 & 2 & 1 & false & \vt csw  &     0\\ \hline
\vt x & 0 & 0 & 0 & false &     0    &     0\\ \hline
\vt y & 0 & 0 & 0 & false &     0    &     0\\ \hline
\vt z & 3 & 3 & 0 & false & \vt nmes & \vt pmes\\ \hline
\end{tabular}

The user working with MOS technology may need to understand and set
this property for ``{\vt m}'' (MOS) devices.  For LVS, is is required
that the electrical and physical MOS devices assume the same number of
nodes.  The device library provides a choice of three-terminal ({\cb
nmos}, {\cb pmos}) and four-terminal ({\cb nmos1}, {\cb pmos1})
devices.  Although either type of device can be placed in a schematic
that is used for simulation, for comparison to the physical layout
consistency is required with the MOS device extraction templates
defined in the technology file {\vt Device} blocks (see
\ref{devblock}).

For consistency, there are two choices:
\begin{enumerate}
\item{The technology defines a three-terminal ``nmos'' device, and the
schematics exclusively use the {\cb nmos} schematic symbol (similar
for pmos).  In this case, substrate/well connectivity is simply
ignored in comparisons.}

\item{The technology file defines a four-terminal ``nmos1'' device,
and the schematics use the {\cb nmos1} schematic symbol (similar for
pmos).  In this case, the substrate/well connection at each transistor
is included in the connectivity comparison.}
\end{enumerate}

Three and four terminal devices of the same sex can not be mixed in
physical extraction, however they can be different for p and n
devices.  For example, in a process where only the pmos devices reside
in a defined ``tub'', it might be convenient to use three-terminal
nmos devices, and four terminal pmos devices.  In this case, the
technology file should define extraction devices for a three-terminal
``nmos'', and four terminal ``pmos1''.  The standard {\vt device.lib}
file should include the line

\begin{quote}
\vt Property DeviceKeyV2 m 4 4 0 false nmos pmos1
\end{quote}

and the user should remember to use the three-terminal {\cb nmos} and
four-terminal {\cb pmos1} library devices exclusively in schematics
that will be used with physical data.

The {\cb !devkeys} command dumps the current keys to the console
window, which can be useful for debugging this capability.
\end{description}

\subsection{Device Library Aliases}

The device library may use the {\vt Alias} keyword
\begin{quote}
{\vt Alias} {\it alias} {\it libcellname}
\end{quote}
to define alternate names for devices contained in the library.  The
alternate names can be used equivalently when referencing devices from
the library.  Aliases, however, will {\bf not} appear in the device
menu displayed from the electrical side menu in {\Xic}

The {\vt device.lib} file distributed with {\Xic} provides aliases to
terminal devices whose names have been changed from those used in
earlier {\Xic} releases, thus providing backward compatibility.  The
device names were changed in release 3.2.22.

\begin{tabular}{ll}
\bf old name & \bf current name\\
\vt vcc  & \vt tbar\\
\vt vbus & \vt tbus\\
\end{tabular}

\subsection{Device Library Devices}
\index{device library file!syntax}

A device is simply an electrical cell definition.  It is distinguished
as a device by the presence and values of certain properties.  Devices
have no subcells or sub-devices, they must contain only geometry. 
They can not contain physical data.

The supplied {\vt device.lib} file contains a collection or rather
plain looking generic device models that correspond to the devices
supported by SPICE.  Additional devices are often created when a
layout is imported from Cadence Virtuoso.  These correspond to the
schematic symbols of devices used in Virtuoso schematics.  Devices can
be created by the user through use of the {\cb Save As Device} button
in the {\cb File Menu}.  These can be used to supplement or replace
devices provided in the default library.

Device definitions as native cell definitions, whether inline in the
{\vt device.lib} file, or in separate files, may or may not have an
empty physical part.  This is in contrast with normal cells, where a
physical part is required.  In releases prior to 4.1.12, inline cell
definitions could not have an empty physical block, but in a cell file
the empty physical block was required. 

Devices can be referenced in three ways:
\begin{enumerate}
\item{By using the {\vt Reference} keyword to reference a cell
containing the device definition.}

\item{By using the {\vt Directory} keyword to reference a directory
that contains device definitions.  This is particularly useful as a
way to include a user's special devices.  One can ``install'' the
device by simply copying it into a directory.}

\item{By including the device definition in the library file as an
inline cell definition.  This is the format used in the supplied {\vt
device.lib} file.  In early {\Xic} releases, this was the only way to
define devices.}
\end{enumerate}

The native syntax for device definitions, as used in the supplied {\vt
device.lib} file, is described below.  This is the same CIF-like file
format as used in native cell files.  The syntax as described applies
to native-format device cell files as well as devices inlined into the
{\vt device.lib} file.  However, for stand-alone cell descriptions,
other file formats can be used.

In these cells, there is no physical representation, however an empty
physical representation can appear.  The default resolution is 100
units per micron (as in CIF and native cell files), however the {\vt
(RESOLUTION 1000);} comment can appear, which indicates 1000 units per
micron, as in ordinary cells.  Each device entry has the following
format:

\begin{quote}{\vt}
(Symbol {\it symname\/});\\
5 {\it property\/};\\
5 ... ;\\
9 {\it symname\/};\\
DS 0 1 1;\\
L SCED;\\
{\it geometry\/} ...\\
{\it more layers/geometry\/} ...\\
DF;\\
E
\end{quote}

The first line is a CIF comment stating the device name, e.g., for a
capacitor one might have
\begin{quote}
{\vt (Symbol cap);}
\end{quote}
This line signals the beginning of a device definition to the function
that automatically updates the device library file after a device is
edited (see \ref{devedit}), so must appear as shown for that feature
to work correctly.

This is followed by property specification lines, which begin with the
number `{\vt 5}', and a cell name definition, which begins with the
number `{\vt 9}'.  The property lines can occur in any order. 
Technically, the property lines are optional, however the name line is
mandatory.  All lines in the symbol specification parts of the file
must end with a semicolon (;), except for the symbol termination line
``{\vt E}''.  While the device is being parsed, the `;' is actually
taken to be the line terminator, so that logical lines can span
several printed lines.

The name line begins with `{\vt 9}' in the first column, followed by
the symbol name (space separated), and ending with a semicolon
(without space).  This line actually defines the name of the device,
as known to {\Xic}.  The property lines define the device terminals
and other parameters through the property mechanism.  Each line begins
with `{\vt 5}' in the first column, followed by the property number,
followed by other data, and finally terminated with a semicolon. 
Refer to properties description (Appendix \ref{prptyspec}) for
information about properties and their syntax.  If the device
represents a subcircuit macro, the name property must be keyed with
the character `{\vt x}' or `{\vt X}'.

After the property lines comes a CIF define symbol directive:
\begin{quote}
{\vt DS 0 1 1;}
\end{quote}
The next line is a directive to use the SCED layer, which is the
active layer in the drawing:
\begin{quote}
{\vt L SCED;}
\end{quote}
The drawing in the cell should be on this layer to visually match the
other elements, however there is no real requirement for this.  There
are additional layers in the default technology which can be used,
typically for highlighting.  The geometry used in a device has no
electrical significance, i.e., no connectivity, and exists for visual
purposes only.

The devices in the supplied {\vt device.lib} file use 100 units per
internal ``micron'' for historical reasons.  Be advised that if a {\vt
(RESOLUTION 1000);} line appears at the top of the device definition,
1000 units will be assumed for the device.  Devices that are edited by
{\Xic} or added through {\Xic} editing will use 1000 units.

After the geometry comes the CIF directive to end symbol definition:
\begin{quote}
{\vt DF;}
\end{quote}
The last line of the device definition contains the single character
\begin{quote}
{\vt E}
\end{quote}
which indicates the end of the device symbol definition.  Note that
in this case there is no terminating semicolon.

As an example, here is a sample library entry for a resistor:
\index{device library file!example entry}

\begin{quote}
\begin{verbatim}
# resistor
(Symbol res);
5 10 -1 0 0 0 + 0 0 0;
5 10 -1 1 0 -1000 - 0 0 0;
5 11 R 0;
5 15 -100 -100 0 -1 "<v>/<value>";
9 res;
DS 0 1 1;
L SCED;
W 0 0 -1000 0 -750 -100 -700 100 -600 -100 -500 100 -400 -100 -300 0 -250 0 0;
L ETC1;
W 0 -100 -75 -100 -125;
W 0 -125 -100 -75 -100;
DF;
E
\end{verbatim}
\end{quote}

The property lines (lines beginning with `5') represent two node
definitions, a name, and a branch, in that order.
The `W' line (wire) following the SCED layer declaration represents
the path used to render the resistor schematic symbol.  The other two
`W' lines, following the ETC1 layer declaration, represents a `+' mark
used to distinguish the positive end of the resistor, and the target
upon which the user clicks to obtain the resistor current, in conjunction
with the {\et branch} property.

\index{device library file!editing}
The device library file can be viewed or edited from within {\Xic}
through the {\cb Open} command.  If ``device.lib'' (or the actual file
name) is given in response to the cell-to-edit prompt, a text editing
window displaying the file appears.  Actually, the current device
library file is first copied to the current directory (if it is not
already there), and the copy is opened for editing.  After saving
changes and quitting the text editor, the internal device database is
rebuilt from the device library file in the current directory.

Devices from the library can also be edited graphically, and devices
added, from within {\Xic}.  This will be described in the following
section.

The terminal device is a special non-physical object used for tying
different parts of the circuit together without a wire, and for
assigning node names.  The library can contain multiple, functionally
equivalent terminal devices under various names, each possibly with a
different visual style.  The name label of a terminal device defaults
to the device name, but can be changed by editing the label text once
placed.  It is important that the {\et name} property of the device
begin with the character `{\vt @}'.

In the library, any device that has no {\et name} property and exactly
one {\et node} property will be taken as a ground terminal device.  A
terminal device will also have exactly one {\et node} property, but
must have a {\et name} property with a name string starting with the
`{\vt @}' character.

% not in help
See section \ref{devmenu} and the subsections that follow for more
information about the device menu and the various devices provided
in the distributed {\vt device.lib} file.
 

% -----------------------------------------------------------------------------
% modlib 102208
\section{Model Library Files}

\index{model.lib file}
\index{model library file}
Devices such as transistors require model specification for generation
of SPICE simulation input.  {\Xic} has a mechanism for handling large
numbers of device models, while at the same time providing interactive
editing capability.  Models for devices ({\vt .model} lines) and
subcircuits ({\vt .subckt} lines) are read from model library file
found along the library search, and from files found in particular
subdirectories of the directories in the path.

The first model library file found in the search path is searched for
models and subcircuits, as are any files found in subdirectories with
a specified name.  The default name for model library files is ``{\vt
model.lib}'', however another name can be specified in the technology
file with the {\vt ModelLibrary} keyword.  The default name for
subdirectories to search for device models is ``{\vt models}'', and
this name can be changed in the technology file with the {\vt
ModelSubdirs} keyword.

In the search, the current directory is always searched first, whether
or not this is actually specified in the search path.  Only the first
model library file is read, which allows the user to override a system
model library file with a custom version.  All files found in {\vt
models} subdirectories will be searched.  The names of the files found
in {\vt models} subdirectories are unimportant, but files existing in
these directories should contain SPICE models, though it is not an
error if no models or subcircuits are found in a file.

As with the device library file, the model library file can be edited
using the {\cb Open} command in {\Xic}.  One simply enters the name
(e.g., ``{\vt model.lib}'') when prompted for the cell name to edit. 
A text editing window appears.  If the file was not found in the
current directory, it is copied to the current directory.  Otherwise,
the file in the current directory is copied to a file with the same
name but with a ``{\vt .bak}'' extension.  When editing is complete
and changes saved, the model database is rebuilt, using the model
library file in the current directory.

\index{models subdirectories}
\index{.model lines}
The ``{\vt models}'' subdirectories might be used with large
collections of files provided by semiconductor manufacturers. 
Typically, the package supplied from the manufacturer contains a
number of files, each describing a device sold by the manufacturer. 
In most cases, all that is required to make these models available to
{\Xic} is to move the files into a {\vt models} subdirectory of a
directory in the library path.  All of these files found will be added
to the database.

The format is that of SPICE, where the first line of each model starts
with {\vt .model} (case insensitive), and the text for that model is
assumed to extend to the next {\vt .model} or {\vt .subckt} line or
end of file.

Subcircuits as well as models are are tabulated.  A subcircuit begins
with {\vt .subckt}, and ends on a line starting with {\vt .ends}. 
Models and subcircuits defined within a subcircuit are not accessible
as separate library references.

Any line which begins with `{\vt \#}' or `{\vt *}' is treated as a
comment and ignored.

The text of any of the files in the {\vt models} subdirectories must
not change while {\Xic} is active.  If the text changes after the time
that {\Xic} caches the file offsets to the models, the model text that
{\Xic} will extract from the file will very likely be bogus.  If the
model library file is edited with the {\cb Open} command and saved,
all offset tables are updated.

\subsection{MOS Model Spatial Binning}
\index{MOS model binning}
 
When {\Xic} generates a SPICE netlist, it automatically includes the
text of the required models.  For MOS devices, i.e., devices keyed by
the letter `{\vt m}', a spatial binning model selection scheme is
available.  This same binning mode is available for MOS devices in
{\WRspice}.  When running {\WRspice} from {\Xic}, the SPICE text is
composed by {\Xic}, so the it is usually necessary to resolve the
spatial binning within {\Xic}.
 
Complete information is available in the description of the
{\WRspice} MOS model.
 
The {\vt L} and {\vt W} parameters values found in the MOS device {\et
param} property are used to key different models.  The {\et model}
property specifies the basename of the model.  The variations are
found in the model library under the basename suffixed with ``{\vt
.1}'', ``{\vt .2}'', etc.  Each of these models may contain parameters
{\vt LMIN}, {\vt LMAX}, {\vt WMIN}, {\vt WMAX} which specify the
parameter window for the model.  The model with the window containing
the device {\vt L}/{\vt W} is the one chosen.  If a model is missing
one of the min/max parameter sets, it will match any value of the
parameter.
 
