% -----------------------------------------------------------------------------
% Xic Manual
% (C) Copyright 2009, Whiteley Research Inc., Sunnyvale CA
% $Id: cellmenu.tex,v 1.12 2017/03/22 07:29:56 stevew Exp $
% -----------------------------------------------------------------------------

% -----------------------------------------------------------------------------
% xic:cellmenu 102308
\chapter{The Cell Menu:  {\Xic} Cell Navigation and Information}
\index{Cell menu}

The {\cb Cell Menu} contains the {\cb Push}/{\cb Pop} commands that
enable pushing the viewing/editing context into the hierarchy, and
returning.  Other commands provide information about cells and allow
other manipulations.

In {\Xic}, there is a notion of the ``current cell''.  This is the
cell hierarchy shown in the main window.  The current cell is acted on
by many of the commands in {\Xic}, and in particular only the current
cell can be modified.  The current cell can be set in many ways,
including using the {\cb Open} command in the {\cb File Menu}, or the
{\cb Cells Listing} panel from the present menu.  One can set the
current cell to a subcell with the {\cb Push} command.  This can be
used in conjunction with the {\cb Info} command in the {\cb View Menu}
to push to the cell containing a selected object, to any depth in the
hierarchy.  The {\cb Pop} command can be used to climb back up the
hierarchy to the original current cell. 

\begin{tabular}{|l|l|l|l|} \hline
\multicolumn{4}{|c|}{\kb Cell Menu}\\ \hline
\kb Label & \kb Name & \kb Pop-up & \kb Function\\ \hline\hline
\et Push & \vt push & none & Make subcell the current cell\\ \hline
\et Pop & \vt pop & none & Make parent cell the current cell\\ \hline
\et Symbol Tables & \vt stabs & \cb Symbol Tables & List of cell symbol
  tables\\ \hline
\et Cells List & \vt cells & \cb Cells Listing & List cells in memory\\ \hline
\et Show Tree & \vt tree & \cb Cell Hierarchy Tree & Display cell hierarchy\\
  \hline
\end{tabular}


% -----------------------------------------------------------------------------
% xic:push 042909
\section{The {\cb Push} Button: Push Editing Context}
\label{pushcmd}
\index{Push button}
\index{editing subcells}
Pressing the {\cb Push} button in the {\cb Cell Menu} will push the
editing context to a subcell.  This means that the subcell becomes the
``current cell'', and editing operations can be performed in this
cell.  The {\cb Pop} command in the {\cb Cell Menu} can be used to
return to the original current cell.

If, when the {\cb Push} button is pressed, the {\cb Info} command is
active and an object is selected that is not in the current cell, The
editing context will be pushed to the cell containing that object,
which may be arbitrarily deep in the hierarchy.

Otherwise, if any subcells are selected, the editing context will be
pushed to the most recently selected subcell.  If no subcell has been
selected, the user is asked to select one.

The pushed-to cell is displayed in true orientation, with or without
the surrounding context shown as set with the {\cb Show Context in
Push} button in the {\cb Main Window} sub-menu in the {\cb Attributes
Menu} or in the sub-window {\cb Attributes} menu.  The surrounding
context is generally shown with reduced illumination to visually
differentiate the current cell from the context.  The illumination
percentage can be set in the {\cb Window Attributes} panel (from the
{\cb Attributes Menu}), or equivalently by setting the {\et
ContextDarkPcnt} variable to a value 1-100 (100 indicates no
darkening).

The history of which cells have been pushed to and popped from is
saved.  Assume that previously one has pushed into the hierarchy and
popped back.  When the {\cb Push} button is active, pressing the {\kb
Enter} key will push down one level and deactivate the button. 
Holding the {\kb Ctrl} key while pressing {\kb Enter} will suppress
the button deactivation, so that one can press {\kb Enter} repeatedly
to push deeper into the hierarchy, following the last push sequence. 
Pressing {\kb Shift-Enter} will cycle backwards, i.e., pop, with
button deactivation controlled by the {\kb Ctrl} key as above.  Unless
the {\kb Ctrl} key was up during the last context change, the
{\cb Push} command is still active and one must press {\kb Esc}
before the cell can be edited.

If instead of pressing {\kb Enter} a subcell is clicked on, the
subcell is pushed to in the usual way, and all past history below
the present level is removed.


% -----------------------------------------------------------------------------
% xic:pop 062109
\section{The {\cb Pop} Button: Pop Context}
\index{pop button}
\index{editing subcells}
\index{editing context}
Pressing the {\cb Pop} button in the {\cb Cell Menu} will pop the
editing context back to the parent cell, if the {\cb Push} command has
been employed.

If the user switches between physical and electrical mode while a push
is active, the symbol currently being edited remains the target, but
the cell becomes top-level (not in a push) in the new mode.  If the
original mode is returned to without editing a different cell, the
push stack is retained.  If a new cell is edited in the new mode,
through a push or otherwise, the original push context is lost.  This
context is also lost if the {\cb Clear} function in the {\cb Cells
Listing} from the {\cb Cell Menu} is invoked.


% -----------------------------------------------------------------------------
% xic:stabs 062109
\index{Symbol Tables button}
\index{symbol tables}
\section{The {\cb Symbol Tables} Button:  Switch Symbol Table}
\label{symtab}

The {\cb Symbol Tables} panel is brought up with the {\cb Symbol
Tables} button in the {\cb Cell Menu}.  A ``symbol'' is a cell name,
which applies to corresponding physical and electrical cells.  A
symbol table is a container (a hash table) which holds cell
definitions in memory for rapid access by name.  Within a symbol
table, all cells have unique names, and an attempt to add a cell with
an existing name will simply overwrite the existing cell in the table. 
On program startup, a default symbol table is provided, which will
contain all cells unless the user intervenes.

It is possible to have multiple symbol tables available.  This allows
different versions of a cell with the same name to exist in memory
concurrently, though in different symbol tables.  It also provides a
means for the user to "start fresh" without actually destroying cells
in memory.

This pop-up manages the symbol tables that are currently allocated. 
It is possible to add or delete symbol tables, and to switch between
the tables.  The table in use contains the cell ``memory'' that is
currently available.

The option menu to the left provides the means for switching between
existing tables.  Each table has a name, which is listed in the menu. 
Initially, only one table, named ``{\vt main} is available.

The {\cb Add} button allows a new symbol table to be created and added
to the list.  The user is asked to provide a name for the table.  This
name can be just about any text string, however if the name already
exists in the table list, a new table is not created.  The table
corresponding to the name becomes the current table.  Although
non-alphanumeric characters can be included in the name, this will
require that the name be double-quoted if used in the extended layer
name syntax of layer expressions or the {\cb !layer} command.

The {\cb Clear} button will clear and destroy the contents of the
current table.  After confirmation, if there are modified cells, the
user will be given a chance to save them to disk.  If the user does
not abort, all cells in the current table will be destroyed, and the
table will be empty except for the default ``{\vt noname}'' cell which
will be read from disk if it exists, and this will become the current
cell.

The {\cb Destroy} button will destroy the current table, and its
contents.  It is not possible to destroy the ``{\vt main}'' table,
the button is disabled when that table is current.  After
confirmation, if there are modified cells, the user will be given a
chance to save them to disk.  If the user does not abort, all cells in
the table, and the table itself, will be destroyed.  After the table
is destroyed, one of the remaining tables will become the new current
table.

Note that when switching between tables, the current cell in use at
the time of the switch is saved, and recalled when the user returns to
that table.


% -----------------------------------------------------------------------------
% xic:cells 030316
\index{Cells List button}
\index{cells panel}
\index{listing cells}
\section{The {\cb Cells List} Button: Cell Listing Panel}

The {\cb Cells List} button in the {\cb Cell Menu} is used to bring up
the {\cb Cells Listing} panel, providing a listing of cell names.  The
cells listed are dependent upon the context, as will be described, and
can be filtered for various criteria.  The panel can be used to select
cells for editing or placement, among other useful features.

If the {\cb Display} button in the {\cb Cell Hierarchy Digests} panel
is active, i.e., the program is in hierarchy display mode, the cells
listing is obtained from the CHD currently being displayed.  In this
case, filtering (to be described) does not apply.  Otherwise, the
listing is obtained from the cells presently in memory, in the current
symbol table.

To the right of the {\cb Dismiss} button is a drop-down menu which
provides a choice or electrical or physical display mode for the cells
list.  The initial selection will be the same as the current display
mode.  The cells listed will have been created in the selected mode. 

The display of the cell names is paged.  The number of entries
displayed per page can be set with the {\et ListPageEntries} variable,
or defaults to 5000 if this variable is unset (variables can be set
with the {\cb !set} command).  If the listing requires multiple pages,
a page selection menu will appear to the left of the {\cb Dismiss}
button.

\index{cells panel!listing}
Cell names are listed in columns.  The top level cells (those that are
not used as subcells of another cell) are shown with an asterisk `*',
and a plus sign `+' appears for modified cells.

The listing is a drag source, cell names can be dragged and dropped
into drawing windows, to display or edit that cell.

\subsection{Cells Listing Command Buttons}

A cell name can be selected by clicking on the name.  Only one name
can be selected at once, and it will be highlighted.

A number of buttons appear along the left edge of the panel.  Without
a selection, these buttons are grayed.  Selected names are acted on by
buttons of the panel, which become active when a selection is made. 
The buttons enable functionality described below.

\index{cells panel!Clear button}
\index{delete cells}
\begin{description}
\item{\cb Clear}\\
The {\cb Clear} button is available when listing cells from memory,
but not in CHD display mode.

This button will clear top-level cells (those not used as a subcell by
any other cell in memory, and marked with an asterisk in the list) or
all cells from memory.  If a top-level cell is selected in the text
area, that cell and its descendents which are not referenced outside
of the hierarchy are removed from memory, after confirmation.  There
is no ``undo'' of this operation.  If the cell is not top-level in
both electrical and physical modes, the command exits with a warning
message.  If no cell is selected, the entire symbol table will be
cleared (after confirmation).  The user is first given a chance to
save any unsaved work.  The current editing cell becomes the next cell
given on the command line, or the default ``{\vt noname}'' cell if no
other cell was specified.  This command can not be undone, and
anything cleared is very definitely gone.

\index{cells panel!Tree}
\index{tree diagram}
\item{\cb Tree}\\
The {\cb Tree} button is available in normal and CGD display modes,
and is active when a cell name is selected.

The {\cb Tree} button is used to bring up the {\cb Cell Hierarchy
Tree} pop-up, which can also be initiated with the {\cb Show Tree}
button in the {\cb Cell Menu} (for the current cell).  From the {\cb
Tree} button in the {\cb Cells Listing} panel, the {\cb Cell Hierarchy
Tree} pop-up will display the hierarchy of the selected cell.

\index{cells panel!edit}
\index{editing cells}
\item{\cb Open}\\
The {\cb Open} button is available when listing cells from memory, but
not in CHD display mode.  The button is active when a cell name is
selected.

Pressing the {\cb Open} button will load the selected cell into the
main window, for display or editing.  Cells can also be dragged from
the listing and dropped into drawing windows, with a similar effect.

\index{cells panel!Place}
\index{master cells}
\item{\cb Place}\\
The {\cb Place} button is available when listing cells from memory,
but not in CHD display mode, and is not available in the {\Xiv}
feature set.  When available, it is active when a cell name is
selected.

Pressing the {\cb Place} button will cause the selected cell to
become the current master cell, and the {\cb Cell Placement Control}
panel will appear.  Instances of the master can be created by pressing
the {\cb Place} button in the {\cb Cell Placement Control} panel, then
clicking on locations in a drawing window.

\index{cells panel!Copy button}
\index{copy cells}
\item{\cb Copy}\\
The {\cb Copy} button is available when listing cells from memory, but
not in CHD display mode, and is not available in the {\Xiv} feature
set.  When available it is active when a cell name is selected.

The {\cb Copy} button allows an existing cell to be duplicated under a
new name.  The user must explicitly save the copied cell to disk if
the new cell is not placed in a hierarchy saved as an archive file,
otherwise the copied cell will be lost when the program is exited,
though the new cell is marked as ``modified'' so the user will be
prompted to save it when exiting.  Pressing {\cb Copy} will cause a
dialog box to appear asking for a new name for the cell.  A copy will
be made if the user enters a valid new name, which must not already be
in use.  The new name will become highlighted in the cell listing. 

Any cell can be copied.  Copies will always be created with the
IMMUTABLE and LIBRARY flags (see below) unset.

\index{cells panel!Replace button}
\index{replace cells}
\index{cells panel!cell replacement}
\item{\cb Replace}\\
The {\cb Replace} button is available when listing cells from memory,
but not in CHD display mode, and is not available in the {\Xiv}
feature set.  When available, it is active when a cell name is
selected, and at least on cell instance is selected in a drawing
window.

The button allows replacement is cell instances selected in a drawing
window to be replaced with instances of the selected cell.  Pressing
the button brings up a confirmation pop-up.  A `yes' response will
initiate the replacement.  The current transform is ignored when
replacing cells from this panel, which is different from the {\cb
Replace} function in the {\cb Cell Placement Control} panel from the
side menu.

When a cell is replaced, the placement of the new cell is determined
in physical mode by the setting of the {\cb Origin}/{\cb Lower Left}
buttons in the {\cb Cell Placement Control} panel (though it may not
be visible).  When {\cb Lower Left} is active, the lower left corner
of the replacing cell corresponds to the lower left corner of the
replaced cell, otherwise the cell's origins are used.  In electrical
mode, the reference terminal (the first connection point) is always
placed at the same location as the reference terminal of the replaced
cell.

\index{cells panel!Rename}
\index{rename cells}
\item{\cb Rename}\\
The {\cb Rename} button is available when listing cells from memory,
but not in CHD display mode, and is not available in the {\Xiv}
feature set.  When available, it is active when a cell name is
selected.

The {\cb Rename} button allows a cell in memory to be given a new
name.  All references to the cell throughout the symbol table will be
changed to call the new name.  This is useful to avoid name clashes in
designs intended to be merged with other designs.  Note that the newly
named cell should be explicitly saved as a file if in native format,
or it may be lost when the user exits.  The cell will be saved in the
hierarchy if an ancestor cell is written to an archive file.  The user
must remember to save any cells which call the renamed cell (the
MODIFIED flag is set for these cells, so that the user is warned at
program exit).

Pressing the {\cb Rename} button brings up a dialog box asking for the
new name.  The renaming is effective if a valid new name, which must
not already be in use, is given.

Leading and trailing white space is stripped from the name, and any
non-empty name is accepted, though a warning is issued if the name
contains a character that may cause trouble.  The GDSII specification
allows alpha-numerics plus `{\vt \$}' (dollar sign), `{\vt \_}'
(underscore), and `{\vt ?}' (question mark).  A character not in this
list will trigger the warning.  The user should stick to valid cell
names when possible.

Cells with the IMMUTABLE flag (see below) set can not be renamed. 
Cells with the LIBRARY flag set can be renamed, which will unset the
LIBRARY flag.

\index{cells panel!Search}
\item{\cb Search}\\
The {\cb Search} button is available in normal and CGD display modes.

In normal display mode, when the {\cb Search} button is pressed, the
listing will initially contain only cells in the hierarchy of the
current cell, selections in the listing are ignored.  If the user
clicks in a drawing window displaying the current cell, the listing
will then contain only cells with instances that appear under the
click location.  If the user drags button 1 to define a rectangle in a
drawing window displaying the current cell, only cells that have
instances that appear in the drag rectangle will be listed.  These
operations can be repeated, the listing will be updated after each
operation.  Pressing the {\cb Search} button again to deactivate it
will revert to listing all cells in the current symbol table.

In CHD display mode, when the {\cb Search} button is pressed, the
listing will contain cells found in the CHD, including and under the
cell currently being displayed in the main window.  Clicking or
dragging in the window will restrict the cell listing as in the normal
display mode.

\index{InfoInternal variable}
The label at the top of the {\cb Cells Listing} will show the search
area coordinates in microns, unless the {\et InfoInternal} variable is
set, in which case internal units are given.

\index{cells panel!flags}
\item{\cb Flags}\\
The {\cb Flags} button is available in normal mode only.

Cells in the main database have two flags which can be modified by the
user.  The IMMUTABLE flag indicates that the cell is read-only, and
can not be modified or renamed.  The LIBRARY flag indicates that the
cell was read through the library mechanism.  Cells with the LIBRARY
flag set are not included when writing output, unless the {\cb Include
Library Cells} check box in the {\cb Export Control} panel is
active, or equivalently the {\et KeepLibMasters} variable is set.

Cells read into the database through the library mechanism will have
both the IMMUTABLE and LIBRARY flags set.  The panel that appears when
the {\cb Flags} button is pressed allows the user to change the flag
states, and corresponding cell behavior.

If no cell name is selected, all of the cells listed in the {\cb Cells
Listing} will be displayed in the {\cb Set Cell Flags} panel, along
with colored indicators of the status of the two flags.  If a cell
name is selected, only the selected cell will be listed in the {\cb
Set Cell Flags} panel upon pressing {\cb Flags}.  Clicking on the
indicators will toggle the indicators.  The indicators can also be set
globally with the buttons above the listing.  The {\cb Apply} button
must be pressed to actually change the flags in the cells.

Cell flags can also be listed and set/unset with the {\cb !setflag}
command.

If the IMMUTABLE flag of the current cell is set, user interface
editing features are disabled.  The {\cb Enable Editing} button in the
{\cb Edit Menu} can also be used to set the state of the IMMUTABLE
flag of the current cell.

Setting the LIBRARY flag is a means to prevent cell definitions from
appearing in the output file when the hierarchy is written.  It is
occasionally necessary to use this feature to enforce resolution of
cells from another source in a subsequent read, perhaps from a
different library or another layout.

It is also useful on occasion to create a customized library cell,
which will become part of the user's cell collection.  In this case,
the LIBRARY and IMMUTABLE flags for the library cell would be unset,
and the cell modified to the user's needs, and the user's cell
hierarchy written to disk.  On subsequent reads, the user's version of
the cell, which will exist in the file, will satisfy the references,
rather than the version from the library.

Another way to accomplish this, perhaps somewhat safer, would be to
copy the library cell to a new name (using {\cb Copy}), and reference
instances of the copy instead of the library cell.  Copies do not have
the flags set (unless reset by the user).

\index{cells panel!info}
\index{cell info}
\item{\cb Info}\\
The {\cb Info} button is available in normal and CGD display modes.

In normal display mode, the {\cb Info} button produces a pop-up that
provides information about subcells and other objects, as from the
{\cb Info} button in the {\cb View Menu}.  If a cell name has been
selected in the listing, the {\cb Cell Hierarchy Tree} pop-up, or in a
drawing window, pressing the {\cb Info} button will display a window
containing information about the cell.  This information includes the
size, number of objects and subcells, and cells for which the selected
cell is a subcell.  If this button is pressed when there is no
selected cell name, the info window will also appear, but contain no
data.  In any case, when the info window is visible, clicking on
objects in drawing windows will reload the window with information
about the object.

In CHD display mode, information contained in the CHD is shown, for a
selected cell or the displayed top-level cell if there are no
selections.  The information in the CGD is dependent upon the
parameters used when the CHD was created.

\index{cells panel!Show}
\index{find cells}
\item{\cb Show}\\
The {\cb Show} button is available in normal and CGD display modes.

The {\cb Show} button enables a mode where cell instances are
highlighted in the main drawing window.  If a cell name has been
selected in the listing, all instances of the cell will be outlined in
the highlighting color.  The outlines apply to all instances of the
cell, regardless of the level in the hierarchy or expansion status. 
This facilitates finding instances of a cell in a complex hierarchy. 
The display will track the currently selected cell name in the
listing.  If no selection, no highlighting is shown, until a selection
is made.  Only one cell can be highlighted at once.  The number of
instances found of the selected cell will be printed in the prompt
area.

\index{cells panel!Filter}
\item{\cb Filter}\\
This button brings up the {\cb Cell List Filter} panel, with which one
can limit the cell list to those with specific attributes.  After
specifying the filtering criteria, pressing the {\cb Apply} button in
the panel will update the listing.  The next section describes this
panel.
\end{description}

%-------------
% cellfilt 031915
\subsection{Cell Filtering}
\label{cellfilt}
\index{cell name filtering}

The {\cb Cell List Filter} panel appears when the {\cb Filter} button
in the {\cb Cells Listing} panel, which is obtained from the {\cb Cell
Menu}.  This provides criteria that enables a cell to be listed in the
{\cb Cells Listing}.

Each entry contains two check boxes, with logic such that at most one
can be set at a time.  Each is assiciated with some assertion about a
cell.  If the left box is checked, the cell will be listed if the
assertion is {\bf not} true.  If the right box is checked, the cell
will be listed if the assertion is true.  If neither is checked, the
assertion is not tested.

A cell will be listed if all tests indicate that the cell should be
listed.  If no tests are done, the cell will be listed by default. 
The available tests are described below.

\begin{description}
\item{\cb Immutable}\\
List cells with the IMMUTABLE flag set.
\item{\cb not Immutable}\\
List cells with the IMMUTABLE flag {\bf not} set.

\item{\cb Via sub-master}\\
List cells that are standard via sub-masters (physical only).
\item{\cb not Via sub-master}\\
List cells that are {\bf not} standard via sub-masters (physical only).

\item{\cb Library}\\
List cells with the LIBRARY flag set.
\item{\cb not Library}\\
List cells with the LIBRARY flag {\bf not} set.

\item{\cb PCell sub-master}\\
List cells that are parameterized cell sub-masters (physical only).
\item{\cb not PCell sub-master}\\
List cells that are {\bf not} parameterized cell sub-masters (physical
only).

\item{\cb Device}\\
List cells that are devices (electrical only).
\item{\cb not Device}\\
List cells that are {\cb not} devices (electrical only).

\item{\cb Top level}\\
List cells that are top level (not used as a subcell).
\item{\cb not Top level}\\
List cells that are {\bf not} top level.

\item{\cb Modified}\\
List cells that are modified, i.e., have been changed in some way.
\item{\cb not Modified}\\
List cells that are {\bf not} modified.

\item{\cb With alt}\\
List cells that have an alternate-mode cell defined, i.e., in the
physical listing, list cells if an electrical mode cell of the
same name exists.
\item{\cb not With alt}\\
List cells that have no alternate-mode cell defined.

\item{\cb Reference}\\
List reference cells.  These are special
cells that reference another cell hierarchy.
\item{\cb not Reference}\\
List cells that are {\bf not} reference cells.

\item{\cb Parent cells}\\
This makes use of the text entry area on the same line which can
contain a list of cell names.  List cells that use at least one of the
listed cells as subcells.  If the text entry is empty, list cells that
contain subcells.
\item{\cb not Parent cells}
List cells that do not contain any of the cells listed in the text
area as subcells, or list cells that contain no subcells if the text
area is empty.

\item{\cb Subcell}\\
This makes use of the text entry area on the same line which can
contain a list of cell names.  List cells that are subcells of any of
the listed cells.  If the text area is empty, list cells that are used
as a subcell of another cell in memory.
\item{\cb not Subcell}\\
List cells that are not a subcell of any of the cells listed in the
text area.  If the text area is empty, list cells that are not used as
a subcell.

\item{\cb With layers}\\
This makes use of the text entry area on the same line which can
contain a list of layer names.  List cells that contain objects on any
of the layers listed.  If the text area is empty, list cells that
contain any geometry.
\item{\cb not With layers}\\
List cells that do not have any geometry on the listed layers.  If the
text area is empty, list cells that have no geometry.

\item{\cb With flags}\\
This makes use of the text entry area on the same line which can
contain a list of flag names (see \ref{cellflags}).  At least one flag
must be given or the test is ignored.  List cells that have one or
more of the listed flags set.
\item{\cb not With flags}\\
The text area must have at least one entry or the test is ignored. 
List cells that do not have any of the listed flags set.

\item{\cb From filetypes}\\
This makes use of the text entry area on the same line which can
contain a list of file type names from among ``{\vt none}'', ``{\vt
native}'', ``{\vt cgx}'', ``{\vt cif}'', ``{\vt gds}'', ``{\vt
oasis}'', and ``{\vt openaccess}''.  Only the first two letters are
needed.  List cells that were read from one of the listed file types. 
Internally generated cells will have type ``{\vt none}''.  If the list
of types is empty, the test is ignored.
\item{\cb not From filetypes}\\
List cells that were not read from the listed file types.  The test is
ignored if the type list is empty.
\end{description}

When the {\cb Apply} button is pressed, the cell listing in the {\cb
Cells Listing} panel will be updated to reflect the given filtering
criteria.

The filtering state can be saved to and recalled from five registers,
through the {\cb Store} and {\cb Recall} menus.  There are separate
register sets for electrical and physical display modes.

The filter state can also be expressed as a string, using keywords. 
Presently, this is used only by the {\vt ListCellsInMem} script
function.  Each keyword or keyword/value pair represents a clause, and
the displayed cells are the logical AND of the clauses given.  The
available clauses are described below.

\begin{description}
\item{\vt immutable}\\
List cells with the IMMUTABLE flag set.
\item{\vt notimmutable}\\
List cells the IMMUTABLE flag {\bf not} set.

\item{\vt viasubm}\\
List cells that are standard via sub-masters (physical only).
\item{\vt notviasubm}\\
List cells that are {\bf not} standard via sub-masters (physical only).

\item{\vt library}\\
List cells with the LIBRARY flag set.
\item{\vt notlibrary}\\
List cells with the LIBRARY flag {\bf not} set.

\item{\vt pcellsubm}\\
List cells that are parameterized cell sub-masters (physical only).
\item{\vt notpcellsubm}\\
List cells that are {\bf not} parameterized cell sub-masters (physical
only).

\item{\vt device}\\
List device cells.
\item{\vt notdevice}\\
List cells that are {\bf not} device cells.

\item{\vt toplev}\\
List cells that are not used as a subcell, i.e., top-level cells.
\item{\vt nottoplev}\\
List cells that are used as a subcell, i.e., {\bf not} top-level.

\item{\vt modified}\\
List cells with the MODIFIED flag set.
\item{\vt notmodified}\\
List cells with the MODIFIED flag {\bf not} set.

\item{\vt withalt}\\
List cells that have an alternate-mode cell defined, i.e., in the
physical listing, keep cells if an electrical mode cell of the
same name exists.
\item{\vt notwithalt}\\
List cells without an alternate-mode cell defined.

\item{\vt reference}\\
List reference cells.
\item{\vt notreference}\\
List cells that are {\bf not} reference cells.

\item{{\vt parent "}{\it cellname1 cellname2} {\vt ..."}}\\
This keyword requires a following quoted list of cell names.  List
cells that use at least one of the cells in the list as subcells.  If
the cell list is empty, specified by two quote marks {\vt ""}, list
cells that have subcells.
\item{{\vt notparent "}{\it cellname1 cellname2} {\vt ..."}}\\
This keyword requires a following quoted list of cell names.  List
cells that do not have any of the listed cells as subcells.  If the
cell list is empty, specified by two quote marks {\vt ""}, list cells
that have no subcells.

\item{{\vt subcell "}{\it cellname1 cellname2} {\vt ..."}}\\
This keyword requires a following quoted list of cell names.  List
cells that are used as a subcell in one or more of the listed cells. 
If the cell list is empty, specified by two quote marks {\vt ""}, list
cells used as a subcell (same as {\vt nottoplev})
\item{{\vt nosubcell "}{\it cellname1 cellname2} {\vt ..."}}\\
This keyword requires a following quoted list of cell names.  List
cells that are not used as a subcell in any of the listed cells.  If
the cell list is empty, specified by two quote marks {\vt ""}, list
cells that are not used as a subcell (same as {\vt toplev}).

\item{{\vt layer "}{\it layername1 layername2} {\vt ..."}}\\
This keyword requires a following quoted list of layer names.  List
cells that have objects on one or more of the listed layers.  If the
layer list is empty, specified by two quote marks {\vt ""}, list cells
that have some geometry on any layer.
\item{{\vt notlayer "}{\it layername1 layername2} {\vt ..."}}\\
This keyword requires a following quoted list of layer names.  List
cells that do not have geometry on any of the listed layers.  If the
layer list is empty, specified by two quote marks {\vt ""}, list cells
that have no geometry.

\item{{\vt flag "}{\it flagname1 flagname2} {\vt ..."}}\\
This keyword requires a following quoted list of flag names (see
\ref{cellflags}).  List cells that have at least one of the listed
flags set.  If the list is empty, the clause is ignored.
\item{{\vt notflag "}{\it flagname1 flagname2} {\vt ..."}}\\
This keyword requires a following quoted list of flag names.  List
cells that have none one of the listed flags set.  If the list is
empty, the clause is ignored.

\item{{\vt ftype "}{\it filetype1 filetype2} {\vt ..."}}\\
This keyword requires a following quoted list of file types, from
``{\vt none}'', ``{\vt native}'', ``{\vt gds}'', ``{\vt cgx}'', ``{\vt
oasis}'', ``{\vt cif}'', and ``{\vt openaccess}''.  Only the first two
letters of the type names are necessary.  List cells that were read
from one of the listed file types.  Internally generated cells will
have type ``{\vt none}''.  If the list is empty, the clause is
ignored.
\item{{\vt notftype "}{\it filetype1 filetype2} {\vt ..."}}\\
This keyword requires a following quoted list of file types, as above. 
List cells that were read from a file type that is not in the list. 
If the list is empty, the clause is ignored.
\end{description}

Examples:
\begin{quote} \vt
notlibrary layer "M1 M2" parent cell1 notparent cell2
\end{quote}
List cells that are not library cells and that contain objects on {\vt
M1} or {\vt M2}, and contain {\vt cell1} but don't contain {\vt
cell2}.

\begin{quote} \vt
subcell maincell layer BASE notlayer VIA notparent ""
\end{quote}
List subcells of {\vt maincell} that have objects on layer {\vt BASE}
but have no objects on layer {\vt VIA} and that have no subcells.

%-----------------
% cellflags 032015
\subsection{Cell Flags}
\label{cellflags}
\index{cell flags}

Cells in memory contain a number of flags.  Most of these are used
internally and can not be set by the user.  All set flags can be seen
in the {\cb Info} windows when cell data are shown.

The table below lists all flags, with a brief description.

\begin{tabular}{|l|l|l|} \hline
\bf Flag Name & User Set & Set When, or Description\\ \hline \hline

\vt BBVALID     & N & Cell bounding box is valid\\ \hline
\vt BBSUBNG     & N & A subcell has unknown bounding box\\ \hline
\vt ELECTR      & N & Cell contains electrical data\\ \hline
\vt SYMBOLIC    & N & Cell has active symbolic representation\\ \hline
\vt CONNECT     & N & Connectivity info is current\\ \hline
\vt GPINV       & N & Inverted ground plane current\\ \hline
\vt DSEXT       & N & Devices and subcircuits extracted\\ \hline
\vt DUALS       & N & Physical/electrical duality established\\ \hline
\vt UNREAD      & N & Created to satisfy unsatisfied reference\\ \hline
\vt COMPRESSED  & N & Save hierarchy in compressed form\\ \hline
\vt SAVNTV      & N & Save in native format before exit\\ \hline
\vt ALTERED     & N & Cell data were altered when read\\ \hline
\vt CHDREF      & N & Cell is a reference\\ \hline
\vt DEVICE      & N & Cell represents a device symbol\\ \hline
\vt LIBRARY     & Y & Cell is from a user library\\ \hline
\vt IMMUTABLE   & Y & Cell is read-only\\ \hline
\vt OPAQUE      & Y & Cell content is ignored in extraction\\ \hline
\vt CONNECTOR   & Y & Cell is a connector\\ \hline
\vt SPCONNECT   & Y & SPICE connectivity info is current\\ \hline
\vt USER0       & Y & User flag 0\\ \hline
\vt USER1       & Y & User flag 1\\ \hline
\vt PCELL       & N & Cell is a PCell sub- or super-master\\ \hline
\vt PCSUPR      & N & Cell is a PCell super-master\\ \hline
%\ifoa
\vt PCOA        & N & Cell is a PCell sub-master from OpenAccess\\ \hline
%\fi
\vt PCKEEP      & N & PCell sub-master read from file\\ \hline
\vt STDVIA      & N & Cell is a standard via sub-master\\ \hline
\end{tabular}

The flags with a Y in the second column can be set by the user, with
the {\vt SetCellFlag} script function and in other places, depending
on the flag.

The first two user-modifiable flags are normally controlled by {\Xic},
however it is possible for the user to change their state through the
{\cb Flags} button in the {\cb Cells Listing} panel, and through the
{\vt SetCellFlag} script function.

\begin{description}
\item{\vt LIBRARY}\\
This flag is set for cells that were read into memory through the
library (see \ref{libraries}) mechanism.  By default, these cells are
not included when a hierarchy is written to disk.

\item{\vt IMMUTABLE}\\
This indicates that the cell is read-only and can't be edited.  This
will be set for cells read into memory through the library mechanism.
\end{description}

The remaining flags are completely under control of the user, they are
not set by {\Xic}.  These are set via the properties mechanism, from
the {\cb Cell Property Editor} ({\cb Flags} property) or with the {\vt
SetCellFlag} script function.  Using a property to control these flags
provides persistence when saved to disk.

\begin{description}
\item{\vt OPAQUE}\\
The physical contents of the cell should be ignored in extraction.

\item{\vt CONNECTOR}\\
The cell is a via or other connector that contains no devices.

\item{\vt
USER0}, {\vt USER1}\\
Convenience flags for the user.  {\Xic} does not use these, but they
may be useful in some application.
\end{description}

% -----------------------------------------------------------------------------
% xic:tree 022916
\section{The {\cb Show Tree} Button: Show Cell Hierarchy}
\index{tree diagram}
\index{Show Tree button}

The {\cb Show Tree} button in the {\cb Cell Menu} brings up the {\cb
Cell Hierarchy Tree} window, which presents a tree diagram
representing cell hierarchy.  Each subcell is initially shown
unexpanded, but these can be expanded by clicking on the expander
symbol.  Subcells can be unexpanded by clicking again in the same
location.  The glyph used to represent the expander is dependent on
the GTK theme in use, and may take different forms.  Clicking
elsewhere in the line will select the subcell name, for use by the
{\cb Info}, {\cb Open}, and {\cb Place} buttons.

When the main drawing window is in CHD display mode, meaning that the
{\cb Display} button in the {\cb Cell Hierarchy Digests} panel is
engaged, the {\cb Cell Hierarchy Tree} will display cells from the
displayed CHD, rooted at the default cell of the CHD.  Otherwise, the
listing represents cells in memory, rooted at the current cell.  The
{\cb Tree} button in the {\cb Cells Listing} panel can also be used to
display the {\cb Cell Hierarchy Tree}, rooted at other cells in memory
or in the displayed CHD.

Pressing the {\cb Info} button will display information about the
selected cell.  In CHD display mode, this is information stored in the
CHD when the CHD was created.  In normal mode, this is the same {\cb
Info} window available in the {\cb View Menu}.  Initially, this window
will contain information about the selected cell, though subsequent
clicks in a drawing window will generate info about other objects.

The {\cb Open} button is only available in normal display mode. 
Pressing {\cb Open} will open the selected cell in the main drawing
window, and make it the current cell for editing and selections.

The {\cb Place} button, also available in normal display mode only,
will pop up the {\cb Cell Placement Control} panel, loaded with the
selected cell.  This enables instantiation of the cell.  The {\cb
Place} button is not available in the {\Xiv} feature set.

Pressing the {\cb Update} button will rebuild the tree internally and
redisplay.  The tree does not automatically track changes in the cell
hierarchy due to editing, the {\cb Update} button can be used to
update the tree manually if needed.

The label at the bottom of the panel provides an indication of the
complexity of the tree.  The total ``nodes'' would be the number of
lines in the display if all items were expanded.  The depth is the
maximum hierarchy depth found.

The listing is a drag source.  Cell names can be dragged and dropped
into drawing windows, to display or edit that cell.

