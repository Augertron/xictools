% -----------------------------------------------------------------------------
% Xic Manual
% (C) Copyright 2009, Whiteley Research Inc., Sunnyvale CA
% $Id: language.tex,v 1.40 2016/06/22 04:29:02 stevew Exp $
% -----------------------------------------------------------------------------

\chapter{The {\Xic} Scripting Language}

% -----------------------------------------------------------------------------
% macro 062116
\section{The Macro Preprocessor}
\label{macro}
\index{macros}
As part of the scripting language support, a macro preprocessor
package is provided, which is used by {\Xic} when reading various
types of input.  This input includes scripts, library and menu (``{\vt
.scm}'') files, and the technology file.  This section describes the
common features of this macro processing system.

A macro is a text token that usually references another piece of text. 
When lines of text are ``macro expanded'', the tokens that are
recognized as macro names are removed, and replaced by the text
associated with the macro name.  This is done recursively, as the
replacement text may itself contain macro names.

In other cases, macros can be used to identify blocks of text to be
discarded when a file is being read.  The macro system applies
conditional testing based on the existence of a defined macro name, or
whether a macro name is set to a certain value, and marks blocks of
text for inclusion or exclusion accordingly.

This section will describe the common functionality of the macro
preprocessor, and will be referred to in the sections describing the
format of the various types of input.  Not all features are used in
all cases, and the exact keyword names (but not the functionality)
will vary for different input types.  For example, the keyword which
defines a macro is ``{\vt \#define}'' in scripts, but ``{\vt Define}''
may be used in other types of file.

\subsection{Predefined Macros}
\index{predefined macros}
\index{macros!predefined}

The macro preprocessor defines several macro names that are common to
all instances of the preprocessor and apply in all cases where the
preprocessor is in use.  These names are the same in all cases, they
do not differ with different file types.  The predefined macro names
can not be undefined or set to a different value, attempts to do so
will trigger an error.  These are the following:

\begin{description}
\item{\vt RELEASE}\\
First implemented: release 3.0.5\\
The macro name {\vt RELEASE} is predefined to the release number code. 
The release number code is a five digit integer {\it xyzz\/}0,
corresponding to release {\it x.y.z}.  The {\it x} (always 3) and {\it
y} are one digit fields, {\it zz} is a two-digit field, 0 padded.  The
trailing 0 is a historical anachronism.  For example, for release
3.2.5, the macro is predefined to ``{\vt 32050}''.

\item{\vt GENERATION}\\
First implemented: release 4.1.10\\
This is set to the generation part of the release triplet, which is
``4'' for the current generation 4.

\item{\vt MAJOR}\\
First implemented: release 4.1.10\\
This is set to the middle number of the release triplet, for 
example for release 4.1.10, {\vt MAJOR} is set to ``1''.

\item{\vt MINOR}\\
First implemented: release 4.1.10\\
This is set to the rightmost number or the release triplet, for
example for release 4.1.10, {\vt MINOR} is set to ``10''.

\item{\vt OSNAME}\\
First implemented: release 4.2.12\\
This is set to the distribution name of the program, for example
``{\vt LinuxRHEL7\_64}''.

\item{\vt OSTYPE}\\
First implemented: release 3.2.19\\
This macro name is set to one of the following words, depending on the
operating system target of the running program.  Note that this is
determined at compile time, so is static in the program binary, and
may not be the ``real'' operating system if running under an emulator. 
For example, a Linux binary running under FreeBSD would still indicate
``{\vt Linux}''.

\begin{tabular}{ll}
\bf Distribution Target & \bf Keyword\\
Any Linux & ``{\vt Linux}''\\
Windows & ``{\vt Windows}''\\
FreeBSD & ``{\vt UNIX}''\\
Any Apple & ``{\vt OSX}''\\
\end{tabular}

\item{\vt OSBITS}\\
First implemented: release 3.2.19\\
This macro is set to either ``{\vt 32}'' or ``{\vt 64}'', depending on
whether the program was compiled for 32- or 64-bit memory addresses. 
This is determined at compile time, so that a 32-bit binary running on
a 64-bit operating system would indicate ``{\vt 32}''.

\item{\vt XTROOT}\\
First implemented: release 3.2.19\\
This macro is defined to be the system xictools installation location
path as assumed by the running program.  It reflects the status of
environment variables or other means of defining this path, and will
revert to a default.  This directory is typically ``{\vt
/usr/local/xictools}'' in non-Windows programs.  The Windows
path is similar but may include a drive specifier and use back instead
of forward slash separators.

\item{\vt PROGROOT}\\
First implemented: release 3.3.1\\
This macro is defined to be the system installation location path for
the running program as assumed by the running program.  It reflects
the status of environment variables or other means of defining this
path, and will revert to a default.  For example, this directory is
typically\linebreak[4] ``{\vt /usr/local/xictools/xic}'' for the
{\Xic} program, in non-Windows programs.  The Windows path is similar
but may include a drive specifier and use back instead of forward
slash separators.

\item{product name}\\
First implemented: release 3.0.5\\
In releases prior to 4.0.9, exactly one of the macro names ``{\vt
Xic}'', ``{\vt XicII}'', or ``{\vt Xiv}'' would be defined, depending
upon which of the programs was being run.  The name is not defined to
any text, but one can test whether or not a given name is defined.  In
release 4.0.9 and later, the separate {\XicII} and {\Xiv} programs
were discontinued, but the functionality lives on as feature sets of
{\Xic}.  The {\vt Xic} symbol is always defined when running {\Xic}
for any feature set, and is therefor rather useless but provides some
backward compatibility.

This macro has the property that instances of the macro are not
replaced (with an empty string) when macro-expanding, i.e., macro
substitution is inhibited (4.2.12 and later).
 
\item{feature set name}\\
First implemented:  release 4.0.9\\
The macro ``{\vt FEATURESET}'' will be defined to one of three
strings, depending upon the feature set running.  If all features are
enabled, the string is {\vt "FULL"}.  If the {\XicII} (EDITOR)
permission set is running, the string is {\vt "EDITOR"}.  If the
{\Xiv} (VIEWER) feature set is running, the string is {\vt "VIEWER"}. 
The macro can be tested with forms similar to
\begin{quote}\vt
If FEATURESET == "FULL"\\
...\\
Endif
\end{quote}

\item{technology name}\\
First implemented: release 3.2.18\\
If the technology file uses the {\vt Technology} keyword to define a
name for the technology, that name will be predefined as a macro
name.  The name is not defined to any text, but one may test whether
or not a given name is defined.

This macro has the property that instances of the macro are not
replaced (with an empty string) when macro-expanding, i.e., macro
substitution is inhibited (4.2.12 and later).
\end{description}

These macros are always available, and additional predefined macros
may be available in the various contexts, which are documented
elsewhere.

\subsection{Generic Macro Keywords}
\index{macros!generic keywords}

The following keywords may vary between different contexts where the
macro processor is used.  The actual keywords are programmable within
the macro preprocessor system, so as to better match the syntax of the
file format to which the preprocessor is being applied.  Here, we will
use italicized generic names for these keywords, but the
correspondence to actual keyword names (given in the documentation for
the specific file formats) should be obvious.  The square brackets
indicate ``optional''.

\begin{description}
\item\parbox[b]{4in}{\rr
{\it DEFINE} [{\vt eval}] {\it token}\\
{\it DEFINE} [{\vt eval}] {\it token}({\it arg\/},
  {\it arg\/}1, ..., {\it arg\/}n) [{\it text\_containing\_args}]\\
}\\
The macro name {\it token} may use alphanumeric characters and
underscores, and must start with an alpha or underscore character. 
The name is optionally immediately followed by an argument list in
parentheses.  The arguments are arbitrary alphanumeric plus underscore
tokens that start with an alpha or underscore and are separated by
commas.  This is the same syntax used in the C language preprocessor
for {\vt \#define} lines.  The remainder of the line is the
substitution string.

If the optional ``{\vt eval}'' keyword is not included, the
replacement text, if any, will replace the macro in lines of text
being macro expanded.

If ``{\vt eval}'' is included (this is verbatim but case-insensitive),
the replacement text is assumed to be executable as a single line
script.  The script will be executed, and the result (or return value)
will be converted to a text string (if necessary) and taken as the
replacement text.

\item{\it IF expression}\\
The {\it expression} is a constant expression which can contain macros
previously defined with {\it DEFINE}, predefines, and functions from
the script library files or otherwise available in memory.  The {\it
expression} is evaluated numerically, and if the result is nonzero (as
an integer), the block that follows until the corresponding {\it ELSE}
or {\it ENDIF} is read.  If the result is 0 (as an integer), the block
of lines that follow is skipped.

\item{\it IFDEF token}\\
If {\it token} has been defined, either with {\it DEFINE} or as a
predefined macro, reading resumes at the following line.  Otherwise,
reading resumes at the line following the next {\it ELSE} or {\it
ENDIF\/}.

\item{\it IFNDEF token}\\
If {\it token} has not been defined, reading resumes at the
following line.  Otherwise, reading resumes at the line following
the next {\it ELSE} or {\it ENDIF\/}.

\item{\it ELSE}
Used in conjunction with {\it IF}, {\it IFDEF} and {\it IFNDEF\/}.

\item{\it ENDIF}\\
Used to terminate an {\it IF}, {\it IFDEF}, {\it IFNDEF\/}, or
{\it ELSE} block.
\end{description}

In various contexts, other special keywords may be recognized.
These are described elsewhere.

{\bf Examples:}

The examples below illustrate some simple constructs that improve
portability of input files, using the predefined macros and generic
keywords.  In real input, the actual keywords appropriate for the type
of file should be used.

The {\it IF} keyword, and product name and {\vt RELEASE} predefines,
were implemented in release 3.0.5, so use is not compatible with older
releases.  Nevertheless, files can be made portably version dependent
through use of {\it IFDEF} and/or {\it IFNDEF\/}.

\begin{quote}
{\it IFNDEF} {\vt RELEASE}\\
\# old release\\
{\it text\/}...\\
{\it ELSE}\\
{\it IF} {\vt RELEASE == 30050}\\
\# release xic-3.0.5\\
{\it text\/}...\\
{\it ELSE}\\
\# a later release\\
{\it text\/}...\\
{\it ENDIF}\\
{\it ENDIF}\\
\end{quote}

Often, it is necessary to know what operating system is being used. 
Usually, there are really only two categories:  Windows, and
everything else.

\begin{quote}
{\it IF} {\vt OSTYPE == "Windows"}\\
\# running Windows\\
{\it text\/}...\\
{\it ELSE}\\
\# not running Windows\\
{\it text\/}...\\
{\it ENDIF}\\
\end{quote}

It may be necessary to disable certain setup if not running the full
{\Xic} feature set, for example, if the same file is used for
different {\Xic} feature sets.

\begin{quote}
{\it IF} {\vt FEATURESET == "FULL"}\\
\# running Xic\\
{\it text\/}...\\
{\it ELSE}\\
{\it IF} {\vt FEATURESET == "EDITOR"}\\
\# running XicII\\
{\it text\/}...\\
{\it ELSE}\\
{\it IF} {\vt FEATURESET == "VIEWER"}\\
\# running Xiv\\
{\it text\/}...\\
{\it ELSE}\\
\# impossible!\\
{\it ENDIF}\\
{\it ENDIF}\\
{\it ENDIF}\\
\end{quote}


% -----------------------------------------------------------------------------
% xicscript 110815
\section{Introduction to {\Xic} Scripts}
\label{scriptintro}
\index{user-definable commands}
\index{scripts}

{\Xic} supports a scripting language and user-definable commands
(scripts).  These commands can be associated with buttons in the {\cb
User Menu}.  Scripts may also be used in ``script labels'', which are
labels placed in a drawing which execute the script when clicked on. 
Scripts are also used in user-defined design rules, and are the basis
for the protocol used in the {\Xic} server mode.  Scripts are also
integral to the native parameterized cell capability.  A library of
several hundred built-in functions callable from scripts provide
control over virtually all of the program capabilities.

In addition to the native scripting capability, {\Xic} provides a
plug-in interface to the popular open-source Python and Tcl/Tk
scripting languages. 

The scripting capability can be used to provide commands that quickly
generate complex geometry for microwave integrated circuits, for
example.  Another application is to produce simple, often-needed
geometry such as vias or device structures.  This powerful capability
provides the user with the tools to automate many tasks.

Script files are created using a text editor, perhaps most
conveniently from within the debugger built into {\Xic}, which is
accessible from the {\cb Debugger} button in the {\cb User Menu}. 
Scripts can be executed within the debugging environment, which offers
single stepping, breakpoints, and other features.  The language is
rather generic and somewhat reminiscent of the C programming language.

\section{The Scripting Language}

A script consists of command lines, each containing one or more
syntactically complete statements.  Lines may be continued by adding a
backslash character at the end of the line, which ``hides'' the return
character.  Parentheses are used as delimiters to enforce execution
order, and to enclose arguments to functions.  Arrays of up to three
dimensions are supported, with the array indices separated by commas
and enclosed in square brackets.  Array names are taken as addresses,
and may be passed to functions, and used in arithmetic expressions. 
There are no address or pointer operators, however a pointer mechanism
does exist.

If a line begins with the pound sign `{\vt \#}' the line will be
ignored by the parser, unless the line contains a ``preprocessor''
directive, described in \ref{preproc}.  Preprocessor directives can be
used to comment out blocks of lines.  The character sequence `{\vt
//}' at the start of a line also indicates a comment.

There is one ``special case'' comment, which must be the first
non-blank line of a script file to have relevance:

\begin{quote}
{\vt \#menulabel} {\it label}
\end{quote}

The {\it label} is a word or quoted phrase, which will appear on the
button in the top level of the {\it User Menu} which executes the
script.  Otherwise, like any comment, the line is ignored.

Each line of a script generally contains one statement or clause, the
entirety of which should be contained in the same logical line. 
Physical lines can be continued with a backslash character to form a
single logical line.  If the last character on a line is the backslash
({\vt "{$\backslash$}"}) character, the line that follows will be
logically appended, replacing the backslash.

The parser will parse the opening clause of a line, and if there is
additional text, the parser will continue reading, until all text on
the line has been processed.  Thus, a single line can actually contain
multiple statements.  Each statement can be terminated with a
semicolon ({\vt ";"}) to explicitly terminate the statement.  Almost
always, this is optional, however there may be rare cases where
explicit termination is needed to force the parser into a correct
interpretation.  The end-of-line will also act as a statement
terminator, which is why a statement must appear in a single logical
line.

With a couple of exceptions, an entire script can be given on a single line. 
This is not recommended, as line-numbered error messages would not
mean much, and the debugger would be useless, however this facilitates
creating complicated macros with the ``{\vt \#define}'' preprocessor
directive, which must always expand to a single line.

The two exceptions are:
\begin{enumerate}
\item{Comments and preprocessor directives start with `\#' and
continue to the end of the current line.  Preprocessor directives must
be given at the start of a line, though comments can appear in a line
where a new statement could appear.  It is not possible to include
(unrelated) command text after a comment or preprocessor directive in
a line.}

\item{The declaration lists that follow the {\vt static} and {\vt
global} keywords must be terminated with a semicolon if a different
construct (including a comment) is to appear on the same line
following the {\vt static} or {\vt global} construct.}
\end{enumerate}

Scripts can interact with forms in HTML documents so that the form can
be used as input for {\Xic} scripts.  This is often more convenient
than issuing a sequence of prompts to the user for input.  The forms
interface makes use of the HTML viewer used with the help system.

There is an expanding library of internal functions which can be
called from scripts, described in \ref{iffuncs}.  The parser also
supports user-defined functions.

Identifiers (function and variable names) must start with an
alphabetic character or underscore, and can contain digits. 
Characters other than alphanumerics and underscore are generally not
accepted in identifiers and will cause syntax errors.  Identifiers are
case-sensitive. 

\section{Error Reporting}
\index{scripts!error reporting}

Compile and run-time error messages go to the standard error channel. 
That means, in interactive graphical mode, that the messages will
appear in the terminal window from which {\Xic} was launched.  Under
Microsoft Windows, if {\Xic} is started from an icon or the {\cb
Start} menu, a terminal window will be created.  This window is
usually hidden behind the main graphics window, so one should make
this window visible when developing scripts.  The same applies to the
terminal window under Unix/Linux.

\section{Data Types}
\index{scripts!data types}

\index{scripts!variable types}
Variables may be one of several different types.  The types that
are currently implemented are listed below.
\begin{description}
\item{no type}\\
Before a variable receives an assignment, it has no type, but behaves
in all respects as a string with a value of the variable name.
\item{string}\\
The string type contains text data.
\item{scalar}\\
Scalars are real numbers that are stored internally in
double-precision floating point format.  Conversion to integer values,
such as for array subscription, is performed automatically where
needed.
\item{array}\\
The array type contains a 1--3 dimensional array of numerical values.
\item{complex}\\
The complex type contains real and imaginary double precision
floating-point scalar values.  Most math functions and operators
accept complex values, and return complex values if passed a complex
value.
\item{handle}\\
The handle type contains a reference to a complex data object.  There
are a number of different object types that can be referenced by
handles.
\item{zoidlist}\\
Zoidlists contain a list of trapezoids that define spatial regions.
\item{layer\_expr}\\
This variable type contains a parse tree for a ``layer expression''
(see \ref{layerexp}).  A layer expression is a logical expression
involving layer names.
\end{description}

The type of a variable is determined by its assignment, or in the case
of arrays, by declaration.  Once a type is assigned, it is generally
an error to assign a different type.  Exceptions are the undefining of
array pointers (to be discussed), the promotion of scalars to handles
when a handle is assigned to a scalar, and use of the {\vt delete}
operator to unassign a variable and free its contents.

Variables that are referenced before assignment, or after being
operated on by {\vt delete}, behave as strings with a string value
set to the variable name.  For example, if an unassigned variable is
passed to one of the print functions the name of that variable will be
printed.

Type identification of a literal is by context.  A quoted quantity is
always taken to be a string, e.g., {\vt "2.345"} is a string.  Quote
marks can be included in strings by preceding them with a backslash. 
A number in integer, floating, or exponential format is always taken
as a scalar.

\subsection{Scalars}
\index{scripts!scalars}

\index{scripts!hex constants}
\index{scripts!char constants}
Scalar variables do not need to be declared, and are type assigned
when an assignment is first made.  Any unquoted number representation
in integer, floating point, or exponential notation is taken as a
scalar constant.  Character constants enclosed in single quotes (as in
C) are accepted, with the value being the ASCII character code.  There
is a {\vt ToChar} function which converts ASCII codes to a string
representation for printing.  Also accepted are hexadecimal integer
constants in the form
\begin{quote}
{\vt 0x}{\it hex\_number}
\end{quote}
 For example, {\vt 0x0}, {\vt 0x2a}, and {\vt 0xffff003b} are all
valid constants.

In addition to the standard floating-point formats, numbers can be
represented using SPICE multiplier suffixes.  These are alphabetic
characters and sequences shown in the table below, which appear
immediately following a fixed-point number or integer.  The suffix is
case-insensitive.  For example, the following tokens all represent the
same number:  {\vt 1000}, {\vt 1e3}, {\vt 1k}.  Likewise:  {\vt
0.0001234}, {\vt 1.234e-4}, {\vt 123.4u}.

\begin{tabular}{|l|l|l|} \hline
\bf suffix & \bf multiplier & \bf name\\ \hline\hline
\vt a & \vt 1e-18 & \vt atto\\ \hline
\vt f & \vt 1e-15 & \vt femto\\ \hline
\vt p & \vt 1e-12 & \vt pico\\ \hline
\vt n & \vt 1e-9 & \vt nano\\ \hline
\vt u & \vt 1e-6 & \vt micro\\ \hline
\vt m & \vt 1e-3 & \vt milli\\ \hline
\vt mil & \vt 25.4 & \vt mil\\ \hline
\vt k & \vt 1e3 & \vt kilo\\ \hline
\vt meg & \vt 1e6 & \vt mega\\ \hline
\vt g & \vt 1e9 & \vt giga\\ \hline
\vt t & \vt 1e12 & \vt tera\\ \hline
\end{tabular}

\subsection{Strings}
\index{scripts!strings}

String variables do not need to be declared, and are type assigned
when an assignment is first made.  Double quote marks are used to
delimit literal strings, and are strictly necessary if the string
contains spaces or other non-alphanumeric characters.

Whenever a string is defined as a literal in a script or from the {\cb
Monitor} panel in the {\cb Script Debugger}, it is filtered through a
function which converts the following escape codes into the actual
character value.  The escape codes recognized, from ANSI C Standard
X3J11, are

\begin{tabular}{ll}\\
$\backslash${\vt a} & bell\\
$\backslash${\vt b} & backspace\\
$\backslash${\vt f} & form-feed\\
$\backslash${\vt n} & new-line\\
$\backslash${\vt r} & carriage return\\
$\backslash${\vt t} & tab\\
$\backslash${\vt v} & vertical tab\\
$\backslash${\vt '} & single quote\\
$\backslash${\vt "} & double quote\\
$\backslash\backslash$ & backslash\\
\end{tabular}

In addition, forms like ``$\backslash${\it num}'' are interpreted as
an 8-bit character with ASCII value the 1, 2, or 3--digit octal
number {\it num}.

When a subscript is applied to a string, the index applies to the
string with escapes substituted, e.g., ``$\backslash${\vt n}'' counts
as one character.  When a string is printed to the {\cb Monitor}
panel, the reverse filtering is performed.

A special case is the null string, which can be produced by many of
the interface functions, usually to signal end-of-input or an error.  A
null string has no storage.  Null strings are not accepted by some
functions, so return values from these functions should be tested.

For example:

\begin{quote}
\begin{verbatim}
retstr = Get(blather)
if (retstr == NULL)     # NULL is an alias for 0
#    the string is null
end
if (retstr == "")
#    the string is empty
end
\end{verbatim}
\end{quote}

This example above also illustrates the overloading of ``{\vt ==}''
for strings.

The notation can be even simpler:
\begin{quote}
\begin{verbatim}
if (retstr)
#    the string is not null (but may be empty)
else
#    the string is null
end
\end{verbatim}
\end{quote}

\index{scripts!string subscripts}
The {\vt []} notation can be used to address individual characters in
strings.  Also, {\vt {\it string1\/} = {\it string2\/} + {\it
number\/}} is accepted, yielding a string pointing at the {\it
number\/}'th character of {\it string2\/}.  However, it is a fatal
error if {\it number} is negative, so it is not possible to point
backwards into a string.  Also, if the {\it number} exceeds the text
length of the string, a fatal error is generated.  A fatal error is an
error which will terminate script execution.

Strings are not copied in assignment, so if multiple variables
point to the same string, they will all see any modifications to
the string.  For example:
\begin{quote}
\begin{verbatim}
s1 = "a string"
s2 = s1 + 2
Print(s2)        # prints "string"
s1[4] = ' '
Print(s2)        # prints "st ing"
Print(s1)        # prints "a st ing"
\end{verbatim}
\end{quote}

The {\vt Strdup} function can be used to make an independent copy of
an existing string.

\subsection{Arrays}
\label{arrayvars}
\index{scripts!arrays}

{\Xic} provides arrays with up to three dimensions.  The indices are
specified as comma-separated expressions enclosed in square brackets
which follow the variable name, as in {\vt x[c,d]} for a two
dimensional array.  The higher dimensions appear to the right, so that
{\vt c} in the example is the ``inner'' index.

\subsubsection{Declaring and Defining Arrays}

Arrays must be declared either by initial assignment, or by a line
consisting of the array name followed by square-bracketed indices
representing the maximum index in each dimension.  In each case, the
number of comma-separated indices sets the dimensionality of the
array.  In the initial declaration, the indices must be integers and
not expressions.  Indices are 0-based.

Examples
\begin{quote}\vt
x[2, 4]\\
\# This defines an array x: five blocks of three values\\
\\
x[2, 4] = {\it some\_expression}\\
\# This likewise defines the array, and additionally sets\\
\# the highest index to the result of an expression
\end{quote}

Note that the numbers in the declaration are {\it not} sizes, but
maximum values.  This is different than C.  Once an array has been
defined, subsequent use allows expressions as the index values.

\subsubsection{Initialization}

We have seen that array elements can be initialized individually by
assignment.  It is also possible to initialize all or part of an array
as a block, using the syntax below: 

\begin{quote}
{\it array\/}{\vt [}{\it index\/}[,...]{\vt ]} = {\vt [}{\it a}, {\it b},
 {\it c}, ... {\vt ]}
\end{quote}

The left side represents a starting address, in the format of an array
element reference.  The outer square brackets are explicit, the inner
square brackets represent optional higher dimension indices and are
not explicit.  The square brackets on the right side are explicit, and
entries are separated by commas and optional white space.  One can use
backslash-continuation to break a long initializer into multiple
physical (but not logical) lines.  The values from the right side are
placed in the array starting at the indicated address, in the natural
order of array scalar access.  The array size is expanded when
necessary.  The line also serves to declare the array.

The {\it a}, {\it b}, {\it c}, ...  can be expressions, or most
commonly simple numbers.

Example:    
\begin{quote}\vt
ary[0] = [1, 2, 3, 4]
\end{quote}

This declares and creates a size 4 array named {\vt ary}, with
components 1, 2, 3, 4.  This is equivalent to the lines
\begin{quote}\vt
ary[0] = 1\\
ary[1] = 2\\
ary[2] = 3\\
ary[3] = 4
\end{quote}

\subsubsection{Dynamic Resizing}

In an assignment, if an index is given that is ``too large'', the
array will be reconfigured so that the new data point will be
included.  The existing data in the array will remain.

Example
\begin{quote}\vt
x[2, 4]\\
x[3, 0] = 2\\
\# The array is now sized as if declared with "x[3,4]"
\end{quote}

After the assignment, the maximum index for each dimension will be the
larger of the previous index and the assigning index.

When assigning values to an array, dimensional indices that are
omitted are taken as zero, though at least one value must be supplied.

Example
\begin{quote}\vt
x[2, 4]\\
x[1] = 3\\
\# This is equivalent to x[1,0] = 3
\end{quote}

This treatment of missing indices only applies in assignment, and {\it
not} in general references, as will be seen below.

There is one important restriction on dynamic resizing:  arrays that
have pointer variables pointing at them can not be resized, and arrays
can not be resized through a pointer.  Pointers are described below.

The {\vt GetDims} function can be used to obtain the current
dimensions of an array.

\subsubsection{Pointers}

A pointer to an array is a variable which points to the data of an
array, and behaves as an array itself but does not contain its own
data.  Pointers can point to the array itself, or to a sub-array of an
array with multiple dimensions, or to an offset into the data of a
single dimensional array. 
 
The simplest case is a direct assignment to an array.
\begin{quote}\vt
x[2, 4]\\
y = x
\end{quote}

In this case, the data (held in {\vt x}) can be accessed through {\vt
y} or {\vt x} equivalently.  In this special case, {\vt y} is an
alias, and the array can be dynamically resized through {\vt y} or
{\vt x}.

A more interesting case is provided through use of the overloaded
`{\vt +}' operator.  For example
\begin{quote}\vt
x[2, 4]\\
y = x + 1
\end{quote}

In this construct, the offset is into the highest dimension of {\vt
x}, and the return value is the sub-array found at this offset.  In
the example, {\vt y} is a ``[2]'' which is located at the address of
{\vt x[0,1]}, i.e., {\vt y[0] = x[0, 1]}, {\vt y[1] = x[1, 1]}, {\vt
y[2] = x[2, 1]}.

If {\vt x} is a single dimensional array, {\vt y} would also be a
single dimensional array, but accessing the data through the offset. 
For example
\begin{quote}\vt
x[32]\\
y = x + 10
\end{quote}

Then {\vt y[0] = x[10]}, {\vt y[1] = x[11]}, etc.

In general references, but {\it not} assignments, supplying a smaller
number of dimensions to an array will return a sub-array.  For
example,
\begin{quote}\vt
x[2, 4]\\
y = x[1]
\end{quote}

This is equivalent to ``{\vt y = x + 1}'', and {\vt y} will point to a
``[2]'' at the location of {\vt x[0,1]}.

\begin{quote}\vt
x[2,4,5]\\
y = x[2]\\
z = x[3,4]
\end{quote}

The variable {\vt y} is a ``[2,4]'' located at {\vt x[0,0,2]}.  The
variable {\vt z} is a ``[2]'' located at {\vt x[0,3,4]}.

When a pointer is defined, a reference count is incremented in the
pointed-to array.  When this reference count is nonzero, the array can
not be resized through the dynamic resizing mechanism.  The pointers
to an array must be reassigned or undefined to allow resizing of the
array.  Pointers can be reassigned simply by changing them to point to
a different array.  This can be done arbitrarily.
\begin{quote}\vt
x[2, 4]\\
y[32]\\
z = x + 1\\
\# can't resize x here\\
z = y\\
\# now ok to resize x
\end{quote}

One can undefine a pointer by setting it to 0.  Once this is done, the
pointer variable has no type, and can actually be reused as another
type of variable.  It is {\it not} an integer unless it is assigned to
an integer.  The same effect may be obtained by applying the {\vt
delete} operator.
\begin{quote}\vt
x[2, 4]\\
y = x + 1\\
\# can't resize x here\\
y = 0\\
\# now ok to resize x\\
Print(y)\\
\# will give "y", y has no type and acts like a string\\
y = 0\\
Print(y)\\
\# will give "0", y is now an integer
\end{quote}

In our initial case,
\begin{quote}\vt
x[2, 4]\\
y = x
\end{quote}
where the pointer is simply a reference to the array, {\vt y} is not
strictly speaking a pointer, but rather an alias.  In particular, this
has no limitation on resizing.  The array data can be resized through
{\vt y} or {\vt x}.  Thus, arrays can be resized from within function
calls if the reference to the array itself is passed to the function,
and not a pointer (with an offset).

\subsection{Complex}

Support for complex numbers is provided via the complex data type. 
The basic math operators and functions accept complex numbers,
possibly intermixed with scalar values, and will produce a complex
result when given a complex operand when appropriate.  Generally, a
complex number can be passed to a function expecting a real number,
and the real part of the complex number will be used.  Similarly, a
scalar passed to a function expecting a complex number will be
accepted as a complex value with zero imaginary part.

Presently, functions will not produce a complex result unless passed a
complex argument.  For example, the {\vt sqrt} function, if passed a
negative scalar, will return a scalar zero.  If passed a complex
number with negative real part and zero imaginary part, the return
will be the complex square root value as one would expect.

Complex numbers can be created with the {\vt cmplx} initializer
function, which takes as arguments two scalar values that initialize
the real and imaginary part.  There are special functions that return
as scalars the real and imaginary values, magnitude, and phase of a
complex operand.  The {\vt Print} function and similar will print a
complex value as a comma-separated pair of numbers enclosed in
parentheses.

\subsection{Handles}

Several of the interface functions return ``handles'', which are
variables which contain a reference to a complex data object.  The
handles are in turn passed to other functions which operate on the
referenced data object.  If an active handle is passed to the {\vt
Print} family of functions, a string giving the type of handle will be
printed.

When done with a handle, it should be closed (with the {\vt Close}
function) to free the memory used by the data object.  The same effect
is obtained by applying the {\vt delete} operator to the handle.  When
iterating over a list-type of handle, the handle will be closed
automatically when iteration is complete.

There are many different types of data object that can be accessed
with a handle, some examples being:
\begin{quote}
    string lists\\
    database objects\\
    file descriptors\\
    properties
\end{quote}
With a few exceptions, notably the file descriptor, a handle generally
points to a list of objects, such as the currently selected objects,
that can be iterated through.  Once the iteration is complete, the
handle is automatically closed, and further references will not reference an
object.

See the section on math operators (\ref{mathops}) for a discussion of
the operations available on handles.

The {\vt HandleContent} function can be called on any handle, and will
return the number of objects that can be referenced through the
handle.  Zero is returned when the handle has iterated to completion. 
This function is useful in loops which contain iterations over
handles.

If a handle still contains references but it is no longer needed, the
{\vt Close} function should be called on the handle, or the {\vt
delete} operator applied to the handle, to free internal resources.

\subsection{Zoidlists}

A ``zoidlist'' is a list of trapezoids, which represents a set of
spatial regions.  Like handles, zoidlists are created by certain
functions, for use in other functions.

As in layer expressions, the logical operators can be applied to
zoidlists, with the result being a new zoidlist representing the
geometric result of the operation.  Available operations include
intersection (and), union (or), inversion, and clipping.  See the
section on math operators (\ref{mathops}) for a discussion of the
operations available on zoidlists.

There is a current ``reference'' zoidlist which represents the
``background''.  If not explicitly set (with the {\vt SetZref}
function), this is taken as the boundary of the current cell.  The
reference is used in operations such as inversion and exclusive-or
where the size of the background must be assumed.  Note that this
background can be an arbitrary shape.

In binary operators with zoidlists, if one of the operands is an
integer, 0 represents an empty list, and nonzero represents the
reference list.

If a zoidlist is given to one of the {\vt Print} family of functions,
the coordinates are printed, one trapezoid per line, in order
x-lower-left, x-lower-right, y-lower, x-upper-left, x-upper-right,
y-upper.

Zoidlists can be assigned from other zoidlists, in which case a copy
is made internally.  If the assigned-to zoidlist already contained a
list, that list is freed from memory.

\subsection{Lexpers}

The layer\_expr variable contains a parsed layer expression.  A layer
expression is an expression consisting of layer names and logical
operators.  A layer expression is evaluated within a certain region,
representing part of a physical layout, and returns the regions where
the layer expression is ``true''.

A layer\_expr is a piece of compiled code that can execute very
quickly.  Functions that accept a layer\_expr argument will generally
also accept a string containing the layer expression, and will compile
the string before use.  If an expression is to be used multiple times,
if is far more efficient to pass a layer\_expr variable.

These variables can not be assigned, and no operators can be applied. 
They can be passed to functions only.

If passed to the {\vt Print} family of functions, the layer
expression string will be printed.


% -----------------------------------------------------------------------------
% scr:mathops 110115
\section{Math Operators}
\label{mathops}

\index{scripts!operators}
The following mathematical operations are supported:

\begin{tabular}{|l|l|p{9.5cm}|} \hline
Symbol & Arity & Description\\ \hline\hline
$+$ & binary & addition\\ \hline
$-$ & unary & negation\\ \hline
$-$ & binary & subtraction\\ \hline
$++$ & unary & pre- and post-increment\\ \hline
$--$ & unary & pre- and post-decrement\\ \hline
$*$ & binary & multiplication\\ \hline
$/$ & binary & division\\ \hline
$\%$ & binary & remainder, e.g., $5 \% 3 = 2$\\ \hline
$\symbol{94}$ & binary & power, x $\symbol{94}$ y $=$ x to power y\\ \hline
\vt $\&$ , and & binary & and, value is 1 if both operands
                           are nonzero\\ \hline
\vt $|$ , or & binary & or, value is 1 if either operand is
                           nonzero\\ \hline
\vt $!$ , \symbol{126} , not & unary & not, value is 1/0 if operand is
                           zero/nonzero\\ \hline
\vt $>$ , gt & binary & greater than, value is 1 if left
                           operand is greater than the right\\ \hline
\vt $>=$ , ge & binary & greater or equal, value is 1 if left
                           operand is greater or equal to the right\\ \hline
\vt $<$ , lt & binary & less than, value is 1 if the left
                           operand is less than the right\\ \hline
\vt $<=$ , le & binary & less or equal, value is 1 if the left
                           operand is less than or equal to the right\\ \hline
\vt $!=$ , ne , $<>$ , $><$ &
                 binary & not equal, value is 1 if the left
                           operand is not equal to the right\\  \hline
\et $==$ , eq & binary & equal, value is 1 if the left operand
                           is equal to the right\\ \hline
\vt $=$ & binary & the left operand takes the value of the
                           right, and the value is that of the right
                           operand.  The type of the left operand
                           becomes that of the right.\\ \hline
\end{tabular}

The operator-equivalent keywords ({\vt gt}, {\vt lt}, {\vt ge}, {\vt
le}, {\vt ne}, {\vt eq}, {\vt or}, {\vt and}, {\vt not}) are
recognized without case sensitivitty.

A variable type is determined by its first assignment, of by
declaration for arrays.  It is generally an error to attempt to
redefine a variable to a different type, though if a scalar is
assigned from a handle, the scalar type is promoted to handle type.

Note that all operators, including assignment, return a value.  Thus,
expressions like {\vt $3*(x > y)$} make sense (the value is 0 or 3). 
Binary truth is indicated by a nonzero value.

The increment/decrement operators ({\vt ++/--}) behave as in the C
language.  That is
\begin{quote}
$y = x++$  is equivalent to $y = x; x = x + 1$\\
$y = x--$  is equivalent to $y = x; x = x - 1$\\
$y = ++x$  is equivalent to $x = x + 1; y = x$\\
$y = x--$  is equivalent to $x = x - 1; y = x$
\end{quote}

All of these operations apply to scalar or complex values.  If complex
and scalar values are mixed, scalar operands are promoted to complex
with zero imaginary value.  If a complex operand is given, the result
is also complex, except for comparison and logical operators, and
modulus, which always return scalar values.  Comparison operators are
applied to both real and imaginary parts, and both must separately
satisfy the relation.  Increment and decrement operations apply only
to the real part of a complex value.  In logical operations, a
complex value is ``false'' if both the real and imaginary parts are 0.

\subsection{Operator Overloading}

In general, the operators apply only to numerical variables.  However,
some of these operators can be used with particular variable types, in
which case a function, relevant to that variable, is invoked.  In most
cases, this is equivalent to invoking an actual function call from the
user interface.  If a non-numeric variable is supplied to an operator
for which no overload exists, the script will generally abort with an
error.

\subsubsection{String Overloads}

The operators $==$, $!=$, $>$, $>=$, $<$, $<=$ have been overloaded
for strings.  If the two operands are strings, the C {\vt strcmp}
function is invoked to compare the two strings.  If either string is
null, it is treated as if it has a lexically minimal value.  Either
operand can be a scalar 0, which is treated as a null string.  Thus,
forms like {\vt if (string == 0)} can be used to test for a null
string.  Null strings, which have no storage, are produced be some
script functions.  These are different from empty strings, produced
for example by {\vt string = ""}, which contain an invisible string
termination character.

The $+$ operator has been overloaded for strings to perform
concatenation, similar to the {\vt Strcat} library function.  The
expression {\vt s3 = s1 + s2} is equivalent to {\vt s3 = Strcat(s1,
s2)}.

The $+$ and $-$ operators can be applied where the first argument is a
string and the second argument is a scalar, and vice-versa in the case
of $+$.  The result of the operation is a pointer into the string,
which behaves as a string with the first character at the offset given
by the scalar.  An error is generated if the offset is negative, or is
beyond the end of the string.

The $-$ operator can be applied where both operands are strings.  The
result is a scalar variable representing the difference between the
memory addresses of the two strings.  This is only useful if both
operands are references to the same string.

The {\vt !} operator can be applied to strings.  The construct is
true only if the string variable contains a null string.

\subsubsection{Array Overloads}

Pointer arithmetic is discussed in the section describing
array variables (\ref{arrayvars}).

\subsubsection{Handle Overloads}

Handles can be used in conditional and logical expressions using the
and ($\&$), or ($\|$), and not ($!$) operators.  If the handle is
non-empty, it is ``true'', otherwise it is ``false''.  This can be
used as a far more efficient loop termination test than a call to {\vt
HandleContent}.

The relational operators have been overloaded for handles.  The
behavior for handles is the same as for scalars, with the handle index
being used in the comparison.  This is not expected to be useful,
except perhaps for file descriptor handles.

The $+$ operator is overloaded to perform concatenation, equivalent to
a call to the {\vt HandleCat} function.  The syntax is
\begin{quote}
{\vt [h1 =] h2 + h3}
\end{quote}

This applies only to handles that contain a list of data items.  Both
{\vt h2} and {\vt h3} must contain lists of the same type of data. 
The list in {\vt h3} is copied and pasted on the end of {\vt h2}.  If
a left hand side is given, it will be assigned the {\vt h2} handle
value and be equivalent to {\vt h2}.  Most of the time, this is not
needed.

The increment operator $++$ is overloaded to perform iteration,
equivalent to a call to {\vt HandleNext} or similar functions.  The
postfix and prefix forms are equivalent.  The return value is simply a
copy of the handle, so again use in an assignment is unlikely to be
needed often.

Without overloading, code to iterate over a list handle would appear
as

\begin{verbatim}
h = func_returning_list_handle()
while (HandleContent(h) != 0)
    (do something)
    HandleNext(h)
done
\end{verbatim}

Making use of overloading, the same loop could take the following
form:

\begin{verbatim}
h = func_returning_list_handle()
while (h)
    (do something)
    h++
done
\end{verbatim}

\subsubsection{Zoidlist Overloads}

The math and logical operators are overloaded for zoidlists as follows:

\begin{tabular}{|l|l|} \hline
$+$, {\vt |}  & union\\ \hline
$-$           & and-not\\ \hline
$*$, {\vt \&} & intersection\\ \hline
$\symbol{94}$ & exclusive or\\ \hline
{\vt !}       & inverse\\ \hline
\end{tabular}

The result of the operation is a new zoidlist, with neither of the
operands affected.

To test for an empty zoidlist, the {\vt ==} and {\vt !=} comparisons
to the value 0 can be applied.  Note that ``{\vt if (!zlist)}'' is an
incorrect test for an empty zoidlist; it will invert the list and
return true if the inverted list is not empty.

There is a current ``reference'' zoidlist which represents the
``background''.  If not explicitly set (with the {\vt SetZref}
function), this is taken as the boundary of the current cell.  The
reference is used in operations such as inversion and exclusive-or
where the size of the background must be assumed.  Note that this
background can be an arbitrary shape.  In binary operators with
zoidlists, if one of the operands is a scalar, 0 represents an empty
list, and nonzero represents the reference list.


% -----------------------------------------------------------------------------
% scr:control 032612
\section{Control Structures}

As in C, logical ``true'' is indicated by a nonzero value.  The
following control statements are accepted.

\subsection{\vt delete}

\begin{quote}
{\vt delete} $[${\it variable\/}$]$
\end{quote}

Although not strictly a ``control'' keyword, the {\vt delete} operator
is handled at the control-block level.  The operator will return the
variable to its undefined state, as if before any assignment, and free
the contents.  After the {\vt delete} operator is applied, the
variable can be assigned to any data type.

Using the {\vt delete} operator on an array will remove the array
characteristics, so in general the variable can not subsequently be
used as an array, except by assigning the variable to another array. 
The {\vt FreeArray} function can be used to clear the data while still
preserving the variable as an array, so that values can still be
directly assigned at indices.

The delete operator applied to a handle will close the handle, as if
the {\vt Close} function was called.  However, the handle will become
an undefined variable after {\vt delete}, rather than a scalar 0.

There are two reasons why this operator exists.  The user may wish to
delete unused variables that contain large data blocks to conserve
memory.  Also, the {\vt ConvertReply} function can return a variable
of any type, thus we must have an undefined variable to take the
return, which is impossible in a loop without use of the {\vt delete}
operator.

The {\vt delete} operator will generally delete the contents, however
for arrays and strings, if the variable has an alias, the content will
be retained in the alias, and all pointers or substrings remain valid. 
If the array or string variable has no alias, any associated pointers
or substrings will also be reinitialized, and the underlying data will
be freed from memory.  Deleting a pointer or substring variable causes
that variable to be undefined, but does not affect the pointed-to
data.

\subsection{\vt return}

\begin{quote}
{\vt return} $[${\it expression\/}$]$
\end{quote}

If the {\vt return} keyword is encountered in the main part of a
script, execution of the script terminates at that point, and the
value returned from any following {\it expression} is saved.  This
return value is available as a return from the {\vt Exec} function, if
that command was used to execute the script.  In general, the return
value is ignored.

If used in a function (see \ref{functions}), the function returns
immediately with the value of the {\it expression}, if given.

\subsection{\vt if, elif, else}

\index{scripts!if elif else end}
\index{scripts!elif}
\begin{quote}
{\vt if} {\it expression1}\\
\hspace*{2em}{\it statements1}\\
{\vt elif} {\it expression2}\\
\hspace*{2em}{\it statements2}\\
...\\
{\vt elif} {\it expressionN}\\
\hspace*{2em}{\it statementsN}\\
{\vt else}\\
\hspace*{2em}{\it statements}\\
{\vt end}
\end{quote}

If {\it expression1\/} evaluates nonzero, {\it statements1\/} will be
executed, otherwise if {\it expression2} evaluates nonzero, {\it
statements2} will be executed, and so on.  If none of the expressions
evaluate nonzero, {\it statements} following {\vt else} will be
executed.  The only parts that are mandatory are {\vt if}, {\it
expression1}, and {\vt end}, all other clauses are optional.

Note that {\vt elif} is {\it not} the same as ``{\vt else if}''.  The
following two blocks are equivalent:

\begin{quote}
\begin{verbatim}
# example using "elif"
if (a == 1)
    Print(1)
elif (a == 2)
    Print(2)
else
    Print("?");
end

# example using "else if"
if (a == 1)
    Print(1)
else
    if (a == 2)
        Print(2)
    else
        Print("?")
    end
end
\end{verbatim}
\end{quote}

In particular, a common error is the following:

\begin{quote}
\begin{verbatim}
if (a == 1)
    Print(1)
else if (a == 2)
    Print(2)
else
    Print("?")
end
\end{verbatim}
\end{quote}

This is missing an ``{\vt end}'' statement (see the second form
above).

\subsection{ternary conditional}
\index{ternary conditional operator}

\begin{quote}
{\it a} {\vt ?} {\it b} {\vt :} {\it c}
\end{quote}
 
The ternary conditional operation, familiar from the C programming
language, is supported.  In this construct, the `{\vt ?}' and `{\vt
:}' are literal, the {\it a\/}, {\it b\/}, and {\it c} are
expressions.  If {\it a} evaluates as {\vt true}, then {\it b} is
evaluated and the construct returns its result.  Otherwise, {\it c} is
evaluated and the construct returns that result.  Hence, the form
 
\begin{quote}\vt
x = a ? b : c
\end{quote}
 
is equivalent to
 
\begin{quote}\vt
\begin{verbatim}
if (a)
    x = b
else
    x = c
end
\end{verbatim}
\end{quote}
 
The ``{\vt true}'' condition depends on the type of variable
represented by {\it a}, as for the {\vt if} operator.  For example,
the following are {\vt true}:

\begin{itemize}
\item{A nonzero numeric value.  This includes the result of a
conditional expression when the condition is satisfied.}

\item{An active handle.}

\item{A non-empty zoidlist (a layer expression is evaluated to obtain
a zoidlist).}

\item{A non-null string.}
\end{itemize}

\subsection{\vt repeat}

\index{scripts!repeat}
\begin{quote}
{\vt repeat} {\it expression}\\
\hspace*{2em}{\it statements}\\
{\vt end}
\end{quote}

Execute {\it statements\/} {\it n} times, where {\it n\/} is the
integer result of evaluating {\it expression}.  The {\it expression}
is evaluated once only when the block is entered, and the integer
value computed is used as the loop counter.  The value is tested for
zero, which will terminate the loop, and is decremented after each
pass.  A negative value will produce an error and the script will
terminate.

\subsection{\vt while}

\index{scripts!while}
\begin{quote}
{\vt while} {\it expression}\\
\hspace*{2em}{\it statements}\\
{\vt end}
\end{quote}

On each pass through the loop, if {\it expression\/} evaluates nonzero,
execute {\it statements\/}, otherwise exit the loop.

\subsection{\vt dowhile}

\index{scripts!dowhile}
\begin{quote}
{\vt dowhile} {\it  expression}\\
\hspace*{2em}{\it statements}\\
{\vt end}
\end{quote}
  
On each pass through the loop, execute {\it statements\/}, then
evaluate {\it expression\/}.  If {\it expression\/} evaluates to zero,
exit the loop.

\subsection{\vt break}

\index{scripts!break}
\begin{quote}
\noindent
{\vt break} $[${\it n\/}$]$
\end{quote}

In a loop, the {\vt break} statement will exit the loop.  If an
integer {\it n\/} is given, control reaches the bottom of the {\it n\/}'th
enclosing loop.

\begin{quote}
Example:
\begin{verbatim}
while x <= 100
  while y <= 50
    while z <= 20
       statements
       if (x + y + z == 10)
           break 2
       end
    end
  end
# break will jump here
end
\end{verbatim}
\end{quote}

\subsection{\vt continue}

\index{scripts!continue}
\begin{quote}
{\vt continue} $[${\it n\/}$]$
\end{quote}

In a loop, {\vt continue} causes the loop to be reentered from the top.  If
an integer {\it n\/} is given, the {\it n\/}'th enclosing loop is
reentered. 

\begin{quote}
Example:
\begin{verbatim}
while x <= 100
  # continue will jump here
  while y <= 50
    while z <= 20
       statements
       if (x + y + z == 10)
           continue 2
       end
    end
  end
end
\end{verbatim}
\end{quote}

\subsection{\vt goto, label}

\index{scripts!goto label}
\begin{quote}
{\vt goto} {\it name}\\
{\vt label} {\it name}
\end{quote}

Execution can jump to an arbitrary location in a routine with the {\vt goto}
statement.  Execution resumes at the statement following the
associated {\vt label}.

\begin{quote}
Example:\\
\\
{\it statements}\\
{\vt if (z != 0)}\\
\hspace*{2em}{\vt goto error}\\
{\vt end}\\
{\it statements\/} ...\\
{\vt label error}\\
{\vt Print("error occurred")}
\end{quote}


% -----------------------------------------------------------------------------
% scr:preproc 020511
\section{``Preprocessor'' Directives}
\label{preproc}

The script parser interprets C-like ``preprocessor'' keywords.   
Unlike C, there is only a single pass through the text, so
``preprocessor'' is a misnomer.

The script preprocessor utilizes the generic macro preprocessor (see
\ref{macro}) used in various places within {\Xic}.  In the present
context, the keywords start with the comment `{\vt \#}' character.

In addition to the predefined macros of the generic macro
preprocessor, the following predefined macro is used in scripts.

\begin{description}
\item{\vt THIS\_SCRIPT}\\
For any script which is read from a file (not counting the technology
file) the token {\vt THIS\_SCRIPT} is effectively defined to be the
name of the script (for scripts launched from the {\cb User Menu}) or
a path to the file.  Thus, in the script, the token {\vt THIS\_SCRIPT}
is replaced by the file or script name.  For example, to print the
script name in the console window, one could add a line
\begin{quote} \vt
Print("The name of this script is THIS\_SCRIPT")
\end{quote}
\end{description}

The following ``preprocessor'' keywords are understood in scripts. 
These pretty much follow the C/C++ standards and behave similarly, and
correspond to the generalized keywords described for the macro
preprocessor.  These are:

\begin{tabular}{ll}
\bf Keyword & \bf Function\\
{\vt \#define} & Define a macro.\\
{\vt \#if} & Conditional evaluated test.\\
{\vt \#ifdef} & Conditional definition test.\\
{\vt \#ifndef} & Conditional non-definition test.\\
{\vt \#else} & Conditional else clause.\\
{\vt \#endif} & Conditional end clause.\\
\end{tabular}

In addition, the following keyword, which has no counterpart in the
generic macro preprocessor, is recognized in scripts:

\begin{description}
\item{\vt \#macro}\\
\index{scripts!\#macro}
The {\vt \#macro} directive, which has no counterpart in C, is assumed
to be followed by macro statements in the format used in the {\vt
.xicmacros} file, followed by {\vt \#end} or {\vt \#endif}.  If the
{\vt \#macro} sequence appears in a script file, the macro is defined
at that point.
\end{description}

Throughout the script, each line is macro expanded.  The actual
arguments replace the formal arguments (if any) in the substitution
text, which replaces the macro reference.  The macro is recognized as
a text token, i.e., it must be surrounded by punctuation or white
space.


% -----------------------------------------------------------------------------
% scr:mathfuncs 110115
\section{Math Functions}
\index{scripts!math functions}

\index{cmplx function}
\index{real function}
\index{imag function}
\index{mag function}
\index{ang function}

The following functions apply to complex numbers.

\begin{tabular}{|l|l|} \hline
Name                   &    Returns\\ \hline\hline
\vt cmplx({\it x\/}, {\it i\/})    &    Return a complex number\\ \hline
\vt real({\it c\/})    &    Return real part of a complex number\\ \hline
\vt imag({\it c\/})    &    Return imaginary part of a complex number\\ \hline
\vt mag({\it c\/})     &    Return the magnitude of a complex number\\ \hline
\vt ang({\it c\/})     &    Return the phase of a complex number\\ \hline
\end{tabular}

The {\vt cmplx} function is used to initialize a complex number
through assignment, for example
\begin{quote}\vt
cx = cmplx(1.0, 0.5)
\end{quote}
creates a complex number {\vt cx} with value {\vt (1.0 + j0.5)}.  The
other functions take a complex number as an argument, and return a
scalar result.

\index{abs function}
\index{acos function}
\index{acosh function}
\index{asin function}
\index{asinh function}
\index{atan function}
\index{atan2 function}
\index{atanh function}
\index{cbrt function}
\index{ceil function}
\index{cos function}
\index{cosh function}
\index{erf function}
\index{erfc function}
\index{exp function}
\index{floor function}
\index{gauss function}
\index{int function}
\index{j0 function}
\index{j1 function}
\index{jn function}
\index{ln function}
\index{log function}
\index{log10 function}
\index{max function}
\index{min function}
\index{pow function}
\index{random function}
\index{rint function}
\index{seed function}
\index{sgn function}
\index{sin function}
\index{sinh function}
\index{sqrt function}
\index{tan function}
\index{tanh function}
\index{y0 function}
\index{y1 function}
\index{yn function}

The following math functions are defined internally, and all take
scalar or complex arguments.

\begin{tabular}{|l|l|} \hline
Name                  &    Returns\\ \hline\hline
\vt abs({\it x\/})    &    absolute value or magnitude of {\it x}\\ \hline
\vt acos({\it x\/})   &    arc-cosine of {\it x}\\ \hline
\vt acosh({\it x\/})  &    arc-hyperbolic cosine of {\it x}\\ \hline
\vt asin({\it x\/})   &    arc-sine of {\it x}\\ \hline
\vt asinh({\it x\/})  &    arc-hyperbolic sine of {\it x}\\ \hline
\vt atan({\it x\/})   &    arc-tangent of {\it x}\\ \hline
\vt atan2({\it x\/}, {\vt y\/}) & arc tangent of {\it x\/}, {\it y}\\ \hline
\vt atanh({\it x\/})  &    arc-hyperbolic tangent of {\it x}\\ \hline
\vt cbrt({\it x\/})   &    cube root of {\it x}\\ \hline
\vt ceil({\it x\/})   &    smallest integer $>=$ {\it x}\\ \hline
\vt cos({\it x\/})    &    cosine of {\it x}\\ \hline
\vt cosh({\it x\/})   &    hyperbolic cosine of {\it x}\\ \hline
\vt erf({\it x\/})    &    error function of {\it x}\\ \hline
\vt erfc({\it x\/})   &    complementary error function of {\it x}\\ \hline
\vt exp({\it x\/})    &    e to the {\it x} power\\ \hline
\vt floor({\it x\/})  &    largest integer $<=$ {\it x}\\ \hline
\vt gauss()           &    gaussian random number\\ \hline
\vt int({\it x\/})    &    truncated integer value of {\it x}\\ \hline
\vt j0({\it x\/})     &    Bessel function order 0 of {\it x}\\ \hline
\vt j1({\it x\/})     &    Bessel function order 1 of {\it x}\\ \hline
\vt jn({\it x\/}, {\it n\/}) & Bessel function order n of {\it x}\\ \hline
\vt ln({\it x\/})     &    natural logarithm of {\it x}\\ \hline
\vt log({\it x\/})    &    natural logarithm of {\it x}, see below\\ \hline
\vt log10({\it x\/})  &    base 10 logarithm of {\it x}\\ \hline
\vt max({\it x\/}, {\it y\/}) & largest of {\it x\/}, {\it y}\\ \hline
\vt min({\it x\/}, {\it y\/}) & smallest of {\it x\/}, {\it y}\\ \hline
\vt pow({\it x\/}, {\it y\/}) & {\it x\/} to the {\it y} power\\ \hline
\vt random()          &    random value in $[0,1)$\\ \hline
\vt rint({\it x\/})   &    integer nearest to {\it x}\\ \hline
\vt seed({\it x\/})   &    seed random number generator\\ \hline
\vt sgn({\it x\/})    &    $+1, 0, -1$ if $x > 0, x = 0, x < 0$\\ \hline
\vt sin({\it x\/})    &    sine of {\it x}\\ \hline
\vt sinh({\it x\/})   &    hyperbolic sine of {\it x}\\ \hline
\vt sqrt({\it x\/})   &    square root of {\it x}\\ \hline
\vt tan({\it x\/})    &    tangent of {\it x}\\ \hline
\vt tanh({\it x\/})   &    hyperbolic tangent of {\it x}\\ \hline
\vt y0({\it x\/})     &    Neumann function order 0 of {\it x}\\ \hline
\vt y1({\it x\/})     &    Neumann function order 1 of {\it x}\\ \hline
\vt yn({\it x\/}, {\it n\/}) & Neumann function order n of {\it x}\\ \hline
\end{tabular}

Most of these functions, when given complex arguments, will compute a
complex result.  The {\vt atan2}, {\vt seed}, Bessel, Neuman, and
error functions ignore any imaginary parts and compute a real value
only.  The {\vt ceil}, {\vt floor}, {\vt int}, {\vt rint}, and {\vt
sgn} functions apply the operation to real and imaginary parts of
complex arguments.  The {\vt min} and {\vt max} functions generate a
complex result containing the operations applied to the real and
imaginary parts of the arguments, if at least one argument is complex. 
The functions listed that take no arguments return scalar values.

With scalar arguments, these functions behave as the corresponding
functions in the C library, though the random number functions are
specialized to {\Xic}.  The {\vt seed} function applies a seed value
to the random number generators.  This can be used to ensure that
successive runs using random numbers choose different values.  The
seed value given is converted to an integer before use.  The {\vt
random} function returns a random value in the range $[0--1)$.  The
numbers generated have a uniform distribution.  The {\vt gauss}
function returns Gaussian random numbers with zero mean and unit
deviation.

\index{LogIsLog10 variable}
{\bf Note regarding the log function}\\
In {\Xic} releases prior to 3.2.23, the {\vt log} function returned
the base-10 logarithm.  This definition was changed in 3.2.23, and
the {\vt log10} function added, for consistency with programming
languages, {\WRspice}, and most other software.  This will require
users to update legacy scripts that use the {\vt log} function to
call {\vt log10} instead.  However, there is a {\et LogIsLog10}
variable that can be set to revert {\vt log} to base-10.  This can be
used temporarily, but is not recommended for the long-term.

% -----------------------------------------------------------------------------
% scr:functions 062908
\section{User-Defined Functions}
\label{functions}
In scripts, user-defined functions are supported.  The function must
be defined before it is called.

A function definition starts with the keyword {\vt function}, followed
by the function name and argument list.  The keyword {\vt endfunc}
terminates the definition.  These blocks can appear anywhere between
statements in a script file, however they must appear before any calls
to the function.  Once a function has been parsed, it is added to an
internal database, where a compiled representation is retained.  If
the same function is parsed again, the in-memory representation is
updated.  There is a mechanism for automatically loading libraries
(see \ref{scriptintro}) of script functions at program startup.  Use
of this mechanism avoids the overhead of repeatedly parsing function
definitions that are found in script files.

The function is called just like a built-in function.  Scalar
variables are passed by value, other types are passed to the function
by reference.  Variables defined within a function are automatic by
default.

Functions can return a value.  In a function, the construct
\begin{quote}\vt
return [{\it expression\/}]
\end{quote}
can be used to terminate execution, and the value of the {\it
expression} is returned by the function.  The value returned can be
of any type.  If the return value is a local string, the string will
be copied.  If the return value is a pointer to an array, the array
must have been passed as an argument or have been declared {\vt
static} or {\vt global} (see \ref{static}).

The example below illustrates how variables are passed, and the
scope for changes.  Strings and arrays can not be redefined in a
function, but elements can be changed.  Arrays are used to pass
results back to the calling function.

Example:

\begin{quote}
\begin{verbatim}
function myfunc(a, b, c)
Print(a, b, c)
endfunc

function examp(a, b, c)
# a is a constant, can be redefined within the scope of examp
a = 2
# b is a string, it is an error to redefine, but can be altered, in
# which case the string is altered in the calling function as well
#b = "b string"   (this produces an error)
b[2] = 'x'
# c is an array, it is an error to redefine, but elements can be
# changed, in which case this is reflected to all users of the array
# c[3]  (redefinition, this is an error)
c[1] = 1.234
myfunc(a, b, c)
endfunc

Print("this is a test")
x = 1
y = "a string"
z[2]
myfunc(x, y, z)
examp(x, y, z)
myfunc(x, y, z)
\end{verbatim}
\end{quote}

It is presently not possible to single-step through a function in
the {\cb Script Debugger}.


% -----------------------------------------------------------------------------
% scr:exec 062908
\section{The {\vt exec} Keyword --- Immediate Execution}
\label{screxec}
\index{exec keyword in scripts}
\index{immediate execution}
Script execution is a two-step process:  first, the text of the script
is parsed, and executable data structures are created internally, and
second, the execution is performed.  Consider the following script:

\begin{quote}
{\vt Set("ScriptPath", "}{\it /path/to/library\_dir\/}{\vt ")}\\
{\it some\_library\_function\/}{\vt ()}
\end{quote}

Naively, the first line will set the script path to the directory
containing the {\vt library} (see \ref{scriptintro}) file, and the
second line will execute a function from the library.  However, this
will not run, since the library function must be resolved before the
parser can process the function call.  Somehow, we must ensure that
the {\vt Set} line is executed before the following line is parsed.

The {\vt exec} keyword will perform this trick.  When an {\vt exec}
keyword is encountered, the remainder of the line (or to the next
semicolon) is parsed and executed immediately, and is {\it not} added
to the parse tree for scheduled execution with the other lines.  Thus,
the example above should be

\begin{quote}
{\vt exec Set("ScriptPath", "}{\it /path/to/library\_dir\/}{\vt ")}\\
{\it some\_library\_function\/}{\vt ()}
\end{quote}

Multiple {\vt exec} lines are executed in order of appearance. 
Variables can be used and set, but remember that this will be done
before any manipulation from the normal script lines.  For example,
the {\et ScriptPath} switch can be hidden:

\begin{quote}
{\vt exec tmppath = GetPath("ScriptPath")}\\
{\vt exec Set("ScriptPath", "}{\it /path/to/library\_dir\/}{\vt ")}\\
{\it some\_library\_function\/}{\vt ()}\\
{\vt Set("ScriptPath", tmppath)}
\end{quote}

The {\vt tmppath} variable will be set first, and is used to reset the
{\et ScriptPath} as a final operation.


% -----------------------------------------------------------------------------
% scr:global 062908
\section{Static and Global Variables}
\label{static}
\index{static variables}
\index{static keyword}

Variables defined in script functions are automatic by default.  The
term ``automatic'' means that every call of the function provides a
fresh set of variables.  A static variable, on the other hand, retains
its contents between calls, and the same variable storage is used in
all calls to the function.  One can explicitly assign a variable in a
function to be static using the {\vt static} keyword.  This construct
should appear only in functions (not the main procedure), and {\it
must} appear ahead of all other executable statements.  The syntax is

\begin{quote}
{\vt static {\it var1} [= {\it val\/}] {\it var2}} ...
\end{quote}

The terms can be separated by white space and/or commas.  The {\it
var1}, etc., are variables used in the function that are to have
static storage.  They can optionally be initialized by including an
assignment.  If an assignment is used, the right hand side should
consist of constants and variables that have already been assigned,
meaning that they appear to the left in the present line or in a
previous {\vt static} line (there can be more than one).  Array
variables should have an initial dimensionality/size specification
consisting of comma-separated integers enclosed in square brackets. 
Each such integer represents the maximum index for the dimension, with
the lowest dimension listed to the left.  This is the standard syntax
for array declaration.

Example:
\begin{quote}\vt
function myfunc(a, b, c)\\
static callcnt = 0\\
...\\
callcnt = callcnt + 1\\
Print("myfunc has been called", callcnt, "times")\\
endfunc\\
\end{quote}

\index{global variables}
\index{global keyword}
There is also provision for global variables.  Global variables are
variables whose scope extends to all functions where the variables
have been declared, including the main procedure.  These are useful
for data items that are accessed frequently throughout a script
application.

The {\vt global} keyword is used to declare global variables.  The
syntax is identical to that for the {\vt static} keyword, and
similarly the declaration must appear at the top of a function and the
main procedure.  There can be more than one {\vt global} line.
\begin{quote}
{\vt global} {\it var1} $[=$ {\it value\/}$]$ {\it var2} ...
\end{quote}

In functions, the list following the keyword can not contain
assignments or array subscripting.  As with {\vt static} declarations,
{\vt global} declarations must appear at the top of the function body. 
There can be multiple {\vt global} lines, and these can be freely
mixed with {\vt static} lines.  Global variables are not accessible
unless declared.

A global variable must be declared in each function where it is to be
accessed, and in the main procedure.  Assignments and array
initialization can be applied in the declarations in the main
procedure only.  It is an error to declare a global with assignment
more than once, or to declare with an assignment in a function.  Like
other variables, if a global variable is not initialized in a
declaration, the first assignment will define the variable type. 
Global array variables must be initialized with the maximum initial
indices in each dimension, comma separated, enclosed in square
brackets in the main function, but indices should {\it not} appear in
the declarations in functions.

Example:

\begin{verbatim}
function myfunc()
    global gvar
    Print(gvar)
    gvar = gvar + 1
endfunc

global gvar = 1
myfunc()
Print(gvar)
# output is:
# 1
# 2
\end{verbatim}

Global variables declared in functions create links to the global
variable of the same name declared in the main procedure.  If the
function is defined in a separate file from the main procedure, such
as a {\vt library} file, and a global variable is declared and used in
the function that is not also declared in the main procedure, an error
results.


% -----------------------------------------------------------------------------
% scr:constants 032613
\section{Predefined Constants}

\index{scripts!predefined constants}
\index{NULL}
\index{INFINITY}
The following constants are recognized by name:\\
\begin{tabular}{ll}\\
\vt e        &  Natural log base\\
\vt pi       &  3.14159...\\
\vt PI       &  3.14159...\\
\vt NULL     &  0\\
\vt INFINITY &  Maximum extent of object coordinate field\\
\vt TRUE     &  1\\
\vt FALSE    &  0\\
\vt EOF      &  -1\\
\vt CHARGE   &  1.60217646e-19\\
\vt CTOK     &  273.15\\
\vt BOLTZ    &  1.3806226e-23\\
\vt ROOT2    &  square root of 2\\
\end{tabular}

The value of {\vt INFINITY} is the internally assumed absolute limit
for valid coordinates, in microns.  This is 1e9 divided by the
database resolution, which is the value of the {\et
DatabaseResolution} variable if set, or the default of 1000.


% -----------------------------------------------------------------------------
% scr:forms 062908
\section{HTML Forms and Scripts}
\label{htmlforms}

\index{HTML forms}
HTML forms can be used as input devices for scripts.  A form may
provide a more convenient interface than a sequence of {\vt Ask}{\it
XXX\/} functions, and arbitrary text and links into the help system
can also be provided in the form text.

\subsection{Introduction to HTML Forms}

Those interested in learning about forms in HTML should obtain a
book on the subject.  A decent book on writing HTML documents is
\begin{quote}
\underline{HTML for the World Wide Web}, Elizabeth Castro,
Peachpit Press, Berkeley CA 1989, isbn 0-201-69696-7.
\end{quote}
Below is a quick summary of the form-related tags.

An HTML form is a collection of input objects such as toggle buttons,
text areas, and menus which allow the user to provide input.  Within
the form is a {\cb submit} button, which when pressed causes a
predefined action to occur.  In HTML, the form is usually processed by
the web page server (through a cgi script).  In {\Xic}, the form may
instead be processed by an {\Xic} script.

A form starts with a tag in the format
\begin{quote}
{\vt <form method="post" action="{\it some text\/}">}
\end{quote}

All but ``{\it some text}'' should be copied verbatim.  In {\Xic}, the
``{\it some text}'' is of the form
\begin{quote}
``{\vt action\_local\_xic} {\it script\_path}''.
\end{quote}
The quotes are required, and {\vt action\_local\_xic} should be copied
verbatim.  The second word is the name of an {\Xic} script file.  The
``{\vt .scr}'' extension is optional, and if a directory path is not
given, the script should exist in the script search path.  Often, {\it
script\_path} is the predefined macro {\vt THIS\_SCRIPT}, which is
replaced by the name of or path to the present script.

The opening {\vt <form ...>} tag is followed by the contents of
the form itself, which can consist of formatted text, and references
to the following objects.

Every object is given a unique name.  This name is used to access the
data in the script.  Each button object will also have a value, which
is a string token passed to identify a choice, i.e., which button of a
group is selected.  This may be different than the label on the
button.  In the tags, constructs like {\vt name="{\it name}"} indicate
the keyword {\vt name}, followed by an `{\vt =}' with no surrounding
space, which is followed by a quoted text string.

\begin{description}
\item{Text Boxes}\\
These are one-line entry areas.  The tag format is
\begin{quote}
{\vt <input type="text" name="{\it name\/}" {\it options\/}>}
\end{quote}
The {\it options} (which are not required), can be:
\begin{description}
\item{\tt size="{\it n\/}"}\\
The {\it n} is an integer that sets the field width in characters.
\item{\vt maxlength="{\it n\/}"}\\
The {\it n} is an integer which limits the length of input text.
\item{\vt value="{\it some text\/}"}\\
This indicates the text that will initially appear.
\end{description}

Example:
\begin{quote}
{\vt Enter username: <input type="text" name="usertext">}
\end{quote}
\end{description}

In this and other similar elements that take a ``{\vt value="}{\it
string\/}{\vt "}'' clause, note that this will fail if {\it string}
contains quote (`{\vt "}') characters.  However, HTML `{\vt \&}'
escapes are expanded in the string, so quote characters can be
replaced with ``{\vt \&quot;}'' to include quotation in the string.

\begin{description}
\item{Password Boxes}\\
These are text boxes, except that characters are printed as
`*' for security.  The format is similar to text boxes:
\begin{quote}
{\tt <input type="password" name="{\it name\/}" {\it options\/}>}
\end{quote}
The {\it options} are the same as for text boxes.

Example:
\begin{quote}
{\vt Enter password: <input type="password" name="passwd">}
\end{quote}

\item{Radio Buttons}\\
These are groups of buttons, one and only one of which is always
selected.  The tag format is
\begin{quote}
{\vt <input type="radio" name="{\it name\/}"
    value="{\it value1\/}" {\it option\/}>{\it text}}\\
{\vt <input type="radio" name="{\it name\/}"
    value="{\it value2\/}" {\it option\/}>{\it text}}\\
...\\
\end{quote}

Each radio button in the group has a line of the form above.  The only
{\it option} is ``{\vt checked}'' which can appear in only one line,
and indicates which button is initially pressed (default is the first
button listed).  Each button should have the same {\vt name}, and a
different {\vt value}.  The text that follows the tag appears next to
the button, and is usually but not necessarily the same as the value.

Example:
\begin{quote}
{\vt <input type="radio" name="radioset" value="1">1}\\
{\vt <input type="radio" name="radioset" value="2" checked>2}\\
{\vt <input type="radio" name="radioset" value="3">3}
\end{quote}


\item{Check Boxes}\\
These are toggle buttons.  The tag format is
\begin{quote}
{\vt <input type="checkbox" name="{\it name\/}"
    value="{\it value\/}" {\it option\/}>{\it text}}
\end{quote}

The only option is ``{\vt checked}'' which indicates that the button
is initially pressed.  The text following the tag appears next to the
button, and is usually but not necessarily the same as the value.

Example:
\begin{quote}
{\vt <input type="checkbox" name="check1" value="check1">check1}\\
{\vt <input type="checkbox" name="check2" value="check2" checked>check2}
\end{quote}

\item{Text Blocks}\\
These are multi-line text input areas.  The tag format is
\begin{quote}
{\vt <textarea name="{\it name\/}" {\it options\/}>{\it default
    text\/}</textarea>}
\end{quote}

The options are
\begin{description}
\item{\vt rows="{\it n\/}"}\\
The {\it n} is an integer that sets the height to n characters.
\item{\vt cols="{\it n\/}"}\\
The {\it n} is an integer that sets the width to n characters.
\end{description}

The {\it default text\/}, if any, will appear in the text area
initially.

Example:
\begin{quote}
{\vt Type in your message:<br>}\\
{\vt <textarea name="message" rows="12" cols="40">Dear sirs,}\\
{\vt </textarea>}
\end{quote}

\item{Option Menu}\\
This is a menu of selections, shown as a button containing the
current selection.  Pressing the button produces a drop-down menu
of choices.  the tag format is
\begin{quote}
{\vt <select name="{\it name\/}" size="1">}\\
{\vt <option value="{\it value1\/}" {\it option\/}>{\it text}}\\
{\vt <option value="{\it value2\/}"
  {\it option\/}>{\it text}}\\
...\\
{\vt </select>}
\end{quote}

There is one {\vt <option ...>} tag per menu entry.  The text
following the {\vt <option ...>} tag will appear in the menu.  The
only option is ``{\vt selected}'' which can be given on only one line
and indicates which item is initially selected (the default is the
first item listed).

Example:
\begin{quote}
{\vt <select name="opmenu" size="1">}\\
{\vt <option value="choose1">choose1}\\
{\vt <option value="choose2">choose2}\\
{\vt <option value="choose3">choose3}\\
{\vt <option value="choose4" selected>choose4}\\
{\vt </select>}
\end{quote}

\item{Selection Menu}\\
This type of menu has multiple lines, which can be selected by
clicking.  The menu may be scrollable.  The tag format is
\begin{quote}
{\vt <select name="{\it name\/}" size="{\it n\/}" {\it option\/}>}\\
{\vt <option value="{\it value1\/}" {\it option\/}>{\it text}}\\
{\vt <option value="{\it value2\/}" {\it option\/}>{\it text}}\\
...\\
{\vt </select>}
\end{quote}

The size in the {\vt <select ...>} tag is an integer greater than
1, which indicates the number of lines visible.  If this is less than
the number of {\vt <option ...>} lines that follow, the menu will
be scrollable.  The option that can appear in the {\vt <select
...>} tag is ``{\vt multiple}'' which if given allows multiple lines
to be selected, otherwise only a single entry can be selected.

Example:
\begin{quote}
{\vt <select name="menu" size="2">}\\
{\vt <option value="choose1">choose1}\\
{\vt <option value="choose2">choose2}\\
{\vt <option value="choose3">choose3}\\
{\vt <option value="choose4">choose4}\\
{\vt </select>}
\end{quote}

\item{File Selection}\\
This is a text area with a {\cb browse} button.  When the {\cb browse}
button is pressed, the {\cb File Selection} panel appears, and the
{\cb Ok} button of the {\cb File Selection} panel will transfer the
selected file name to the form text area.  The format of the tag is
\begin{quote}
{\vt <input type="file" name="{\it name\/}" {\it option\/}>}
\end{quote}
The only option is {\vt size="{\it n\/}"} to set the width in
characters of the text area.
\end{description}

Example:
\begin{quote}
{\vt <input type="file" name="filesel" size="64">}
\end{quote}

Each form must have a {\cb submit} button.  A {\cb reset} button,
which resets all objects to their initial state, is generally useful.

\begin{description}
\item{Submit Button}\\
This is a button which initiates action on the form.  This button is
required if any action is to be taken on the form data.  The tag
format is
\begin{quote}
{\vt <input type="submit" {\it option\/}>}
\end{quote}
The only option is {\vt value="{\it message\/}"}, where the {\it
message} is the text that actually appears on the button, which is
``Submit'' if no value is specified.

Example:
\begin{quote}
{\vt <input type="submit" value="Done">}
\end{quote}

\item{Reset Button}\\
This button resets each component of the form to the initial
state.  The tag format is
\begin{quote}
{\vt <input type="reset" {\it option\/}>}
\end{quote}
The only option is {\vt value="{\it message\/}"}, where the
{\it message} is the text that actually appears in the button,
and is ``Reset'' if no value is specified.

Example:
\begin{quote}
{\vt <input type="reset">}
\end{quote}
\end{description}

The form items can be intermixed with text, images, or other HTML
formatting and objects.  To terminate a form definition, one must
supply the tag
\begin{quote}
{\vt </form>}
\end{quote}

Below is the ``{\vt spform.html}'' file which is used with the {\cb
spiralform} demonstration script, as an example.

\begin{verbatim}
<h2>Forms Demo -- Generate a Spiral</h2>
This page demonstrates the use of HTML forms as input devices for
<i>Xic</i> <a href="xicscript">scripts</a>.  Press the <b>Submit</b>
button when ready.  The spiral will be attached to the pointer, and
can be placed by clicking in a drawing window.

<p>
<form method="post" action="action_local_xic spiralform">

Choose the number of turns in the spiral
<select name="opmenu" size="1">
<option value="1">1
<option value="2">2
<option value="3">3
<option value="4">4
<option value="5">5
<option value="6">6
<option value="7">7
<option value="8">8
<option value="9">9
</select>

<p>
Enter the path width: <input type="text" name="pwidth" value="4"><br>
Enter the starting radius: <input type="text" name="rad1" value="20"><br>
Enter the pitch: <input type="text" name="pitch" value="10"><br>

<p>
Select the number of edges per turn:
<input type="radio" name="radioset" value="10" checked>10
<input type="radio" name="radioset" value="20">20
<input type="radio" name="radioset" value="40">40

<p>
<input type="submit">
<input type="reset">
</form>
\end{verbatim}

\index{scripts!forms interface}
\subsection{Interfacing Forms to {\Xic} Scripts}

If a form has an action which is in the format ``{\vt
action\_local\_xic} {\it scriptname}'' then, when the {\cb submit}
button is pressed, the script in {\it scriptname} will be called, with
the following:

\begin{itemize}
\item The preprocessor variable SUBMIT is defined.
\item A string-type variable is created for each active form
element.  The variable name is that of the {\vt name} field in
the form element tag (so these must be unique).  The value of the
variable is from the {\vt value} tag of the selected button, or
the text of a text-entry object.  The variable is defined only if
the text object had text, or if a check button was pressed.
\end{itemize}

Within the script, one must supply the following logic:

\begin{itemize}
\item Determine if the SUBMIT preprocessor variable is defined.  If
yes, than the script was called by a form, otherwise the script was
called by a button in the {\cb User Menu}.  Note that this enables
the script to initiate showing the form, as will be seen in the
example below.
\item For each variable, the script must identify if the variable
was set, i.e., a text entry had text, and possibly convert the
text to a numeric value.  The input should also be sanity checked
at this point.
\end{itemize}

Below are the first few lines of an example script which could
interface to the example form given above.  When the script is
selected in the {\cb User Menu}, it will display the help window
containing its input form.  When the {\cb submit} button of the form
is pressed, the script will be called again, and the data processed.

\begin{verbatim}
#ifndef SUBMIT
# SUBMIT is not defined, so we are being called from the User Menu
# pop-up our input form in the HTML viewer and exit
TextCmd("help spform.html")
Exit()
#endif

# We are being called from the form (SUBMIT is defined)
# First check the option menu return.  The entries are digits, which must
# be converted from text strings to real values
#
if Defined(opmenu)
    num = ToReal(opmenu)
else
# This should never fail, since the option menu always has a selection
    ShowPrompt("number of turns unknown")
    Exit()
end

# Next check the return from a text entry object.  Exit if the variable
# is undefined (text input empty), or the result is a bad numeric value
#
if Defined(pwidth)
    width = ToReal(pwidth)
    if (width < 0)
        ShowPrompt("Bad input (< 0) for width: ", width)
        Exit()
    end
 
    ShowPrompt("width unknown")
    Exit()
end

(check other input variables)
(perform calculations/operations)
Exit
\end{verbatim}

This is typical boilerplate for a form-entry script.


% -----------------------------------------------------------------------------
% not in help
\section{Example Script}
Below is the text of a script which will generate a spiral object, provided
as an example.

\begin{verbatim}
# Example script to produce a spiral on the current layer
#
# solicit geometrical info from user
num = AskReal("Number of turns? ", "1")
if (num < 0 | num > 100)
    ShowPrompt("Bad input (< 0 or > 100) for turns: ", num)
    Exit()
end
width = AskReal("Width of spiral path? ", "4")
if (width < 0)
    ShowPrompt("Bad input (< 0) for width: ", width)
    Exit()
end
rmin = AskReal("Starting radius? ", "20")
if (rmin < width/2)
    ShowPrompt("Bad input (< width/2) for min radius: ", rmin)
    Exit()
end
spa = AskReal("Pitch? ", "10")
if (spa < width)
    ShowPrompt("Bad input (< width) for pitch: ", pitch)
    Exit()
end
nums = AskReal("Edges per 360 degrees? ", "50")
if (nums < 3 | nums > 90)
    ShowPrompt("Bad input (< 3 or > 90) for edge count: ", nums)
    Exit()
end

# initialize
width = width/2
dth = 2*pi/nums
n = nums*num + 1
i = 0
theta = 0

# there is an internal limit of 2000 polygon vertices
nverts = 2*n + 1
if (nverts > 2000)
    ShowPrompt("Sorry, resulting polygon would have too many vertices.")
    Exit()
end

# allocate array, size 2*nverts
array[4000] = 0

l = 4*n
j = 0

# fill in the array
while (i < n)
    r = rmin + theta*spa/(2*pi)
    x = (r-width)*cos(theta)
    y = (r-width)*sin(theta)
    array[j] = x
    array[j+1] = y
    x = (r+width)*cos(theta)
    y = (r+width)*sin(theta)
    array[l-j-2] = x
    array[l-j-1] = y
    j = j + 2
    i = i + 1
    theta = theta + dth
end

# close the path, necessary for polygon
array[l] = array[0]
array[l+1] = array[1]

# get the location for the spiral and transform array
ShowPrompt("Point to locate center of spiral")
xy[2]
PushGhost(array, nverts)
ShowGhost(8)
if !Point(xy)
    Exit()
end
ShowGhost(0)
PopGhost()

i = 0
j = 0
while (i < nverts)
    array[j] = array[j] + xy[0]
    array[j+1] = array[j+1] + xy[1]
    i = i + 1
    j = j + 2
end

# create the polygon
drc = DRCstate(0)
Polygon(nverts, array)
Commit()
DRCstate(drc)
ShowPrompt("Info: spiral not drc'ed.  Drc takes a long time for these objects.")

#done
\end{verbatim}


