% -----------------------------------------------------------------------------
% Xic Manual
% (C) Copyright 2009, Whiteley Research Inc., Sunnyvale CA
% $Id: filetool.tex,v 1.9 2014/09/09 16:58:23 stevew Exp $
% -----------------------------------------------------------------------------

% filetool 090814
\chapter{The FileTool Utility}
\index{filetool utility}
\label{filetool}
\section{Introduction}

The {\FileTool} is a command-line program for analysis and
manipulation of layout files.  Although the {\FileTool} originated as
a separate stand-alone application that made use of {\Xic} technology,
the current version is a polymorphism of the {\it Xic} executable. 
There are two ways to access the {\FileTool}:

\begin{enumerate}
\item{Copy or symbolically link one the {\vt xic} executable file (or
the {\vt xic.exe} file under Windows) to a new link or file named
``{\vt filetool}'' (or ``{\vt filetool.exe}'' under Windows).  You now
have a {\FileTool} program that behaves in all respects as described
in this documentation.

Under Unix/Linux/OS X, the best way is to use a symbolic link.  For
example, in the same directory as the {\vt xic} executable, become
root and type (for example)
\begin{quote}
\vt ln -s xic filetool
\end{quote}
This will symbolically link the {\vt xic} binary executable to the
{\vt filetool} name, without actually copying the file.  If the {\vt
xic} file is replaced for an update, the link will automatically
access the new executable.

This is {\bf not} automatically done when the programs are installed. 
The user must intervene to obtain a {\vt filetool} executable target.

Under Windows, there are no symbolic links, so the file must actually
be copied.  Thus, after an update, the copy operation should be
repeated, to obtain any updates that relate to the {\FileTool}.}

\item{One can also effectively run the {\FileTool} directly from
{\Xic} with, for example,
\begin{quote}
{\vt xic -F} {\it filetool\_args...}
\end{quote}
The {\vt -F} must be the first argument, and all arguments that
follow are interpreted as {\FileTool} arguments.  The program will
behave in all respects as if started under the name ``{\vt
filetool}''.}
\end{enumerate}

The {\FileTool} can be incorporated in the user's automation scripts
to implement perhaps complicated manipulations on layout files, or as
an aid to understanding content and diagnosing problems with layout
files, or as a general purpose utility.  Here are some of the tasks
that the {\FileTool} can perform:

\begin{itemize}
\item{Print information about a layout file: statistics, layers used,
 top-level cell, etc.}
\item{Translate layout files, or parts of layout files, to a
 different format (CIF, CGX, GDSII, OASIS are supported), or to an
 ASCII text representation.}
\item{When writing, many different translation modes are available:
 layer filtering and aliasing, cell name mapping operations, windowing
 with or without clipping, flattening, scaling, empty cell removal.}
\item{Compare two layout files, listing the differences.}
\item{Split a layout file into multiple files, each representing 
 a portion of the original layout.}
\item{Combine cells from multiple layout files into a single file.}
\item{Generate or process assemble scripts as used by the {\Xic}
 {\cb !assemble} command.}
\end{itemize}

\index{.filetoolrc file}
\index{filetool startup file}
When started, none of the {\Xic} startup or technology files are
read.  Instead, a file named ``{\vt .filetoolrc}'' will be read, if
it can be found in the current directory of the user's home
directory.  This is a script file, like the {\vt .xicstart} file,
however the only function likely to be useful is the {\vt Set}
function, which sets variables.  Variables can also be set from the
{\FileTool} command line, but the {\vt .filetoolrc} file can be used
to set variables that are almost always needed, such as favorite
OASIS flags when working with OASIS files.

The file formats supported by the {\FileTool} are:

\begin{description}
\item{\bf GDSII}\\
The industry standard stream format.  Any release level is supported
for input.  For output, the default release level is 7, but this can
be set to earlier levels.  Compressed (gzipped) GDSII files can be
read or written.

\item{\bf OASIS}\\
The emerging standard, which provides more compact data files than
GDSII.  Any conforming OASIS file can be read as input.  A number of
options affect OASIS output.

\item{\bf CGX}\\
A compact data representation developed by Whiteley Research Inc. 
Compressed (gzipped) CGX files can be read or written.

\item{CIF}\\
The obsolete but still used CIF format.  Any known dialect should
work as input.  The output dialect can be selected via options.
\end{description}

Input files can be any of these file types, the format is recognized
automatically.  Output files can also be any of these file types, but
the format is specified by the extension of the file name.

The operations can be saved to a script file, or read from a script
file.  The script file format is the same as used by the {\cb
!assemble} command in {\Xic}, thus scripts generated by the
{\FileTool} can be executed in {\Xic}.

%------------------
% filetool 120110
\section{Command Line Options}
\index{filetool arguments}

If the {\FileTool} is executed without arguments, a synopsis of
available command line options is printed.  Otherwise, the arguments
are given in one of the following forms.

\begin{quote}
{\vt filetool} [{\vt -set} {\it var\/}[={\it value\/}] {\vt ...}]\\
    \hspace*{1em}{\vt -eval} {\it script\_file\_to\_read} {\vt |}\\
    \hspace*{1em}{\vt -info} {\it layout\_file} [{\it flags\/}] {\vt |}\\
    \hspace*{1em}{\vt -text} {\it layout\_file} [{\it text\_opts\/}] {\vt |}\\
    \hspace*{1em}{\vt -comp} {\it comp\_args\/} {\vt |}\\
    \hspace*{1em}{\vt -split} {\it split\_args} {\vt |}\\
    \hspace*{1em}{\vt -cfile} {\it cfile\_args} {\vt |}\\
    \hspace*{1em}{\it translate\_args}
\end{quote}

The {\vt -set} option is used to set internal variables, which have
relevance in the modes indicated by the other main options.

The {\vt -eval} option is used to execute an assemble script.

The {\vt -info} option is used to obtain information and statistics
about a layout file.

The {\vt -text} option will translate all or part of a layout file to
an ASCII text representation.

The {\vt -comp} option will set up a comparison of two layout files,
recording differences.

The {\vt -split} option is used to write multiple layout files
corresponding to regions in a large layout.

The {\vt -cfile} option is used to write a Cell Hierarchy Digest (CHD)
file from a layout file, similar to the {\cb Save} button in the {\cb
Cell Hierarchy Digests} panel.

Otherwise, the given options are expected to provide directives
similar in logic to that of an assembly script.

%------------------
% ft:set 051310
\section{FileTool:  Setting Variables}
\index{filetool variables}

There are a number of internal variables which control various
properties of the file readers/writers, translation modes, etc. 
These are the same variables as used in {\Xic}.  In some cases, these
variables are overridden by command line options, but in cases where
no applicable option exists, these variables can be set to provide
the desired effect.  Variables can also be set in the {\vt
.filetoolrc} file.  Variables set from the command line will override
settings in the {\vt .filetoolrc} file.

The {\vt -set} options must appear first on the command line, and
unlike the other main directives, can appear ahead of the other
directives.  These are optional.

The format can take two forms:  either a single {\vt -set} option
followed by a quoted list of {\it name\/}={\it value} pairs:
\begin{quote}
{\vt -set} {\vt "}{\it name1\/}={\it value1} {\it name2}
  {\it name3\/}={\it value3} ...{\vt "}
\end{quote}
or, each {\it name\/}={\it value} pair can have its own ``{\vt -set}'':
\begin{quote}
{\vt -set} {\it name1\/}={\it value1} {\vt -set} {\it name2}
  {\vt -set} {\it name3\/}={\it value3}
\end{quote}

Note that the value part is optional, for boolean variables.  The
token following each ``{\vt -set}'' must not contain white space, or
be quoted if it contains white space, e.g.,
\begin{quote}
{\vt -set} {\vt "}{\it name} = {\it value}{\vt "}
\end{quote}
is legitimate.

The following variables have relevance to operations that are
available through the {\FileTool}.

In addition to the variables listed in the table, which are {\Xic}
variables, there is one special boolean variable recognized: 
\begin{quote}
{\vt timedbg}[={\it filename\/}]
\end{quote}
If set, run times for various operations are printed, similar to
enabling the {\cb !timedbg} feature in {\Xic}.  If set to a value, the
value is taken as a path to a file for the timing messages. 

\newpage

\begin{tabular}{|p{3in}|p{3in}|}\hline
\multicolumn{2}{|c|}{\bf Database Setup}\\ \hline
\et DatabaseResolution & \\ \hline

\multicolumn{2}{|c|}{\bf Symbol Path}\\ \hline
\et Path\newline
NoReadExclusive
&
\et AddToBack\\ \hline

\multicolumn{2}{|c|}{\bf Conversion - General}\\ \hline
\et ChdFailOnUnresolved\newline
MultiMapOk\newline
UnknownGdsLayerBase
&
\et UnknownGdsDatatype\newline
NoStrictCellnames\\ \hline

\multicolumn{2}{|c|}{\bf Conversion - Import and Conversion Commands}\\ \hline
\et AutoRename\newline
NoOverwritePhys\newline
NoOverwriteElec\newline
NoOverwriteLibCells\newline
NoCheckEmpties\newline
NoReadLabels\newline
MergeInput\newline
NoPolyCheck\newline
DupCheckMode\newline
LayerList\newline
UseLayerList\newline
LayerAlias
&
\et UseLayerAlias\newline
InToLower\newline
InToUpper\newline
InUseAlias\newline
InCellNamePrefix\newline
InCellNameSuffix\newline
NoMapDatatypes\newline
CifLayerMode\newline
OasReadNoChecksum\newline
OasPrintNoWrap\newline
OasPrintOffset\\ \hline

\multicolumn{2}{|c|}{\bf Conversion - Export Commands}\\ \hline
\et StripForExport\newline
KeepLibMasters\newline
SkipInvisible\newline
KeepBadArchive\newline
NoCompressContext\newline
RefCellAutoRename\newline
UseCellTab\newline
SkipOverrideCells\newline
OutToLower\newline
OutToUpper\newline
OutUseAlias\newline
OutCellNamePrefix\newline
OutCellNameSuffix\newline
CifOutStyle\newline
CifOutExtensions
&
\et CifAddBBox\newline
GdsOutLevel\newline
GdsMunit\newline
NoGdsMapOk\newline
OasWriteCompressed\newline
OasWriteNameTab\newline
OasWriteRep\newline
OasWriteChecksum\newline
OasWriteNoTrapezoids\newline
OasWriteWireToBox\newline
OasWriteRndWireToPoly\newline
OasWriteNoGCDcheck\newline
OasWriteUseFastSort\newline
OasWritePrptyMask\\ \hline

\multicolumn{2}{|c|}{\bf Geometry}\\ \hline
\et JoinMaxPolyVerts\newline
JoinMaxPolyGroup\newline
JoinMaxPolyQueue
&
\et JoinBreakClean\newline
PartitionSize\\ \hline
\end{tabular}

% ft:eval 102109
%------------------
\section{FileTool:  Assemble Script File Evaluation}
\index{filetool scripts}

Assemble script files can be produced by {\Xic}, and contain a
specification for complicated operations on layout files, such as
merging several files into a single output file, while creating a new
top-level cell to contain instances of the cells read from input. 
These files can be evaluated with the {\FileTool}.

The command is of the form
\begin{quote}
{\vt filetool} [{\vt -set} {\it variables\/}] {\vt -eval}
  {\it script\_file}
\end{quote}

The {\ FileTool} will read and execute the script, reading input and
generating output as per the directives in the script file.

The script file format is described in \ref{assemble}.

% ft:info 102109
%------------------
\section{FileTool:  Obtaining File Information}
\index{filetool file info}

In this mode, the {\FileTool} will read a layout file, and print
useful information about the file.  The command line for this mode is
\begin{quote}
{\vt filetool} [{\vt -set} {\it variables\/}] {\vt -info}
  {\it filename} [{\it flags\/}]
\end{quote}

It is unlikely that the {\vt -set} variables will be used with this
option, though the layer filtering options may apply on occasion.

The output format and {\it flags} are identical to those of the
{\Xic} {\cb !fileinfo} command (\ref{fileinfo}).

% ft:text 102109
%------------------
\section{FileTool:  ASCII Text Representation of Layout Files}
\index{filetool text conversion}

The supported file formats other than CIF are binary, and thus the
content is not easy to decipher.  This mode of the {\it FileTool}
will convert records from a layout file into an ASCII representation. 
This may be valuable for identifying problems in the file or
understanding file organization and content.

For this mode, the command takes the form:
\begin{quote}
{\vt filetool} [{\vt -set} {\it variables\/}] {\vt -text}
    {\it layout\_file} [{\vt -o} {\it output\_file\/}]
    [{\it start\/}[-{\it end\/}]] [{\vt -c} {\it cells\/}]
    [{\vt -r} {\it recs\/}]
\end{quote}

Following the layout file path, there are optional arguments.

\begin{description}
\item{\vt -o} {\it output\_file}\\
If this is given, the text output will be placed in the supplied file
name.  Without this option given, text output is to the standard
output.

The remaining arguments control the range of text conversion. 
Without these options, the entire file will be written as ASCII text. 
For all but tiny layout files, the user will probably want to limit
the size of the output.

\item{[{\it start\/}[-{\it end\/}]]}\\
The {\it start} and {\it end} are file offsets, which can be given in
decimal or ``0x'' hex form.  Printing will start with the first
record with offset greater than or equal to {\it start\/}.  If {\it
end} is given, the last record printed will be at most the record
containing this offset.  If both numbers are given, they must be
separated by a `{\vt -}' with no white space.

\item{\vt -c} {\it cells}\\
This options supplies a count, indicating the number of cell
definitions that will be printed.  If the count is 0, and {\it start}
is also given, the records from {\it start} to the end of the cell
definition will be printed.

\item{\vt -r} {\it recs}\\
This provides a count of the number of records to print.  Printing
will stop after the indicated number of records have been output.
\end{description}

Printing will start at the beginning of the file or the {\it start}
record if given, and will end at the end of file or the point at
which the first end condition is satisfied.

There are two variables which may be of interest when using this
mode.  These can be set with {\vt -set} options ahead of the
{\vt -text} argument.

\begin{description}
\item{\et OasPrintNoWrap}\\
Value: boolean\\
This applies when converting OASIS input to ASCII text.  When set,
the text output for a single record will occupy one (arbitrarily
long) line.  When not set, lines are broken and continued with
indentation.

\item{\et OasPrintOffset}\\
Value: boolean\\
This applies when converting OASIS input to ASCII text.  When set,
the first token for each record output gives the offset in the file
or containing CBLOCK.  When not set, file offsets are not printed.
\end{description}

% ft:comp 102109
%------------------
\section{FileTool:  Layout File Comparison}
\index{filetool layout comparison}

This mode compares the geometry and instance placements in cells from
two cell hierarchies, usually from different files.  The results are
written to a log file.

The command line format for this mode is
\begin{quote}
{\vt filetool} [{\vt -set} {\it variables\/}] {\vt -comp}
 {\it comp\_args}
\end{quote}

The operations and arguments are identical to those of the {\Xic} {\cb
!compare} command (\ref{cellcomp}).  This includes the operations
involving Cell Hierarchy Digests (CHDs) and in-memory hierarchies,
provided that those have been created by script functions in the {\vt
.filetoolrc} file.  However, it is most likely that from the
{\FileTool}, the sources will always be on-disk layout files.

% ft:split 102109
%------------------
\section{FileTool:  Layout File Splitting}
\index{filetool file splitting}

The {\FileTool} can be used to split a large layout file into a
collection of smaller layout files.

For splitting, the command line takes the form:
\begin{quote}
{\vt filetool} [{\vt -set} {\it variables\/}] {\vt -split}
  {\it split\_args}
\end{quote}

This mode will write output files corresponding to the partitions of
a square grid logically covering all or part of a specified cell in a
given layout file.  The output files contain physical data only. 
These files can be flat or hierarchical.

The operations and {\it split\_args} are identical to those of the
{\Xic} {\cb !splwrite} command (\ref{splwrite}).

% ft:cfile 120110
%------------------
\section{FileTool:  CHD File Generation}
\index{filetool CHD file generation}

The {\FileTool} can be used to generate a Cell Hierarchy Digest (CHD)
file.  The file format is the same as produced with the {\cb Save}
button in the {\cb Cell Hierarchy Digests} panel.  The CHD file can
subsequently be read into the {\cb Cell Hierarchy Digests} panel with
the {\cb Add} button.

The command line takes the form:
\begin{quote}
{\vt filetool} [{\vt -set} {\it variables\/}] {\vt -cfile}
  {\vt -i} {\it layout\_file} {\vt -o} {\it chd\_file} [{\vt -g}] [{\vt -c}]
\end{quote}

If the {\vt -g} option is given, geometry records will be included in
the file.  These records are effectively a concatenation of a Cell
Geometry Digest file representation.  Layer filtering can be employed
to specify layers to include.

The resulting file is a highly compact but easily random-accessible
representation of the layout file.

Future releases of {\Xic} will make use of these files in creative
ways, stay tuned.

The {\vt -c} option will skip use of compression when creating the
file.  Files produced with this option (and without geometry) should
be compatible with {\Xic} release 3.2.16 and earlier, which did not
support compression.  If backward compatibility is not needed, this
option should not be used.

% ft:assem 071010
%------------------
\section{FileTool:  Layout File Merging and Translation}
\index{filetool file merging}

The {\FileTool} can take a list of arguments which correspond
logically to the keywords of an assembly specification script.  The
argument list begins after any {\vt -set} variables present.

This automates reading of cells from archives, subsequent processing,
and writing to a new archive file.  It provides the capabilities of
the {\cb Conversion} panel in the {\cb Convert Menu} in {\Xic}, such
as format translation, windowing, and flattening.  Additionally,
multiple input files and cells can be processed and merged into a
larger archive, on-the-fly or by using a Cell Hierarchy Digest (CHD)
so as to avoid memory limitations.  Cell definitions for the read and
possibly modified cells are streamed into the output file, and the
output file can contain a new top-level cell in which the cells read
are instantiated.  The input and output can be any of the supported
archive formats (CGX, CIF, GDSII, OASIS), in any combination.

The same operations can be controlled by a specification script file,
the path to which is given as the argument following ``{\vt -eval}''. 
The script uses a language which will be described.  This supplies
the output file name and the description of the top-level cell (if
any), the files to be used as input, the cells to extract from these
files, and the operations to perform.  It is a simple text file,
prepared by the user, containing a number of keywords with values. 
The specification script can also be obtained from the {\cb
Assemble} command in the {\cb Convert Menu}, which is a graphical
front-end to the {\cb !assemble} command in {\Xic}.

Alternatively, the argument list can consist of a series of option
tokens and values.  These are logically almost equivalent to the
language of the specification file.  This gives the user the option
to enter job descriptions entirely from the command line.  These
command-line options start with a `{\vt -}' character.

Only physical data are read, electrical data will be stripped in
output.  A log file is produced when the command is run.  If not
specified with a {\vt LogFile}/{\vt -log} directive, ``{\vt
filetool.log}'' and is written in the current directory.  The log file
contains warning and error messages emitted by the readers during file
processing, and should be consulted if a problem occurs.

The details of the file format and corresponding command line options
are provided in the description of the {\cb !assemble} command.

