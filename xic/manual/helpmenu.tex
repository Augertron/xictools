% -----------------------------------------------------------------------------
% Xic Manual
% (C) Copyright 2009, Whiteley Research Inc., Sunnyvale CA
% $Id: helpmenu.tex,v 1.23 2015/03/07 21:18:34 stevew Exp $
% -----------------------------------------------------------------------------

% -----------------------------------------------------------------------------
% xic:helpmenu 062313
\chapter{The Help Menu: Obtain Program Documentation}
\index{Help Menu}
The commands in the {\cb Help Menu} provide documentation and help to
{\Xic} users.

The commands found in the {\cb Help Menu} are summarized in the table
below.  The table provides the internal name for the command, and
a brief description.

\begin{tabular}{|l|l|l|l|} \hline
\multicolumn{4}{|c|}{\kb Help Menu}\\ \hline
\kb Label & \kb Name & \kb Pop-up & \kb Function\\ \hline\hline
\et Help & \vt help & \cb Help Viewer & Show help, enter help mode\\ \hline
\et Multi-Window & \vt multw & none & Set multi-window help mode\\ \hline
\et About & \vt about & \cb About Panel & Show version info\\ \hline
\et Release Notes & \vt notes & \cb Text Editor & Show release notes\\ \hline
\et Log Files & \vt logs & \cb File Selection & Provide access to log files\\
\et Logging & \vt dblog & \cb Logging Options & Set logging and debugging
  options\\
 \hline
\end{tabular}


% -----------------------------------------------------------------------------
% helpsys 031413
\section{The {\cb Help} Button: Obtain Help}
\label{helpsys}
\index{Help button}
\index{help mode}
\index{help viewer}
{\Xic} provides on-line context-sensitive help through activation of
the {\cb Help} button in the {\cb Help Menu}.  When this button is
pressed, {\Xic} enters help mode, and (unless suppressed) the help
window appears with the default top-level topic.  While help mode is
active, information about commands and screen objects can be obtained
by clicking with the left mouse button (button 1) on menu buttons or
other screen objects.  While in help mode, menu buttons will perform
their normal functions rather than bringing up help text if the {\kb
Shift} key is held while the menu entry is activated.  Help mode can
be exited by pressing the {\kb Esc} key while the pointer is in a
drawing window, or by pressing the {\cb Help} button a second time,
but these will not remove the help window from the screen.  Help mode
is also exited when all help windows have been deleted, either with
the {\cb Quit} button in the help window {\cb File} menu, or with
window manager functions.  If a help window is brought up with the
keyboard {\cb !help} command, {\Xic} is not in help mode, thus menu
buttons will have their normal functions.

\index{HelpDefaultTopic variable}
If the variable {\et HelpDefaultTopic} is set (with the {\cb !set}
command or otherwise) to an empty string, pressing the {\cb Help}
button will not bring up the default top-level window.  However,
clicking on objects and buttons will bring up help topics as usual. 
One can also set this variable to a URL or database keyword, the
content from which will appear in the initial window as the default
topic.

Clicking on a colored HTML reference will bring up the text of the
selected topic.  If button 1 is used to click, the text will appear in
the same window.  If button 2 is used to click, a new help window
containing the selected topic will appear.

\index{HelpMultiWin variable}
The help system operates in one of two modes.  The default mode is to
use a single window for each new topic generated by pressing a command
or menu button.  In the multi-window implementation, which can be
selected in {\Xic} by selecting the {\cb Multi-Window Mode} button in
the {\cb Help Menu}, or by setting the boolean variable {\et
HelpMultiWin} with the {\cb !set} command, a separate window is
brought up for each press of a command button or menu item while in
help mode.  In either case, clicking on a link may or may not produce
a new window, depending upon whether button 1 or button 2 was clicked.

Text shown in the viewer that is not part of an image can be selected
by dragging with button 1, and can be pasted into other windows in the
usual way.

The viewer can be used to display any text file or URL.  In {\Xic} and
its derivatives, pressing the question mark key (``{\kb ?}'') will
prompt the user for text to display.  The {\cb !help} command has the
same effect.  In {\WRspice}, the text to display can follow the ``{\cb
help}'' command keyword on the command line.  The name given to the
command, or to to the {\cb Open} command in the viewer's {\cb File}
menu, can be
\begin{itemize}
\item{A keyword for an entry in the help database.}
\item{A path to a file on the local machine.  The file can be an image
in any standard format, or HTML or plain text.}
\item{An arbitrary URL accessible through the internet.}
\end{itemize}

If the given name can be resolved, the resulting page will be
displayed in the viewer.  Also, the HTML viewer is sensitive as a drop
receiver.  If a file name or URL is dragged into the viewer and
dropped, that file or URL is read into the viewer, after confirmation.

The ability to access general URLs should be convenient for accessing
information from the internet while using {\Xic}.  The prefix ``{\tt
http://}'' {\it must} be provided with the URL.  Thus,
for example,
\begin{quote}\vt
? http://wrcad.com
\end{quote}
will bring up the Whiteley Research web page in {\Xic} or {\WRspice}. 
The links can be followed by clicking in the usual way.  Of course,
the computer must have internet access for web pages to be accessible.

Be advised, however, that the ``{\vt mozy}'' HTML viewer used in
Unix/Linux releases is HTML-3.2 compliant with only a few HTML-4.0
features implemented, and has no JavaScript, Java or Flash
capabilities.  A few years ago, this was sufficient for viewing most
web sites, but this is no longer true.  Most sites now rely on css
styles, JavaScript, and other features not available in {\vt mozy}. 
Most sites are still readable, to varying degrees, but without correct
formatting.

The given URL is not relative to the current page, however if a `+' is
given before the URL, it will be treated as relative.  For example, if
the viewer is currently displaying {\vt http://www.foo.bar}, if one
enters ``{\vt /dir/file.html}'', the display will be updated to {\vt
/dir/file.html} on the local machine.  If instead one enters ``{\vt
+/dir/file.html}'', the display will be loaded with {\vt
http://www.foo.bar/dir/file.html}.

The HTTP capability imposes some obvious limitations on the string
tokens which can be used in the help database.  These keywords should
not use the `/' character, or begin with a protocol specifier such as
``{\vt http:}''.

HTML files on a local machine can be loaded by giving the full path
name to the file.  Relative references will be found.  HTML files will
also be found if they are located in the help path, however relative
references will be found only if the referenced file is also in the
help path.  If a directory is referenced rather than a file, a
formatted list of the files in the directory is shown.

If a filename passed to the viewer has one of the following
extensions, the text is shown verbatim.  The (case insensitive)
extensions for plain-text files are ``{\vt .txt}'', ``{\vt .log}'',
``{\vt .scr}'', ``{\vt .sh}'', ``{\vt .csh}'', ``{\vt .c}'', ``{\vt
.cc}'', ``{\vt .cpp}'', ``{\vt .h}'', ``{\vt .py}'', ``{\vt .tcl}'',
and ``{\vt .tk}''

Holding {\kb Shift} while clicking on an anchor that points to a URL
which specifies a file on a remote system will download the file. 
References to files with extensions ``{\vt .rpm}'', ``{\vt .gz}'', and
other common binary file suffixes will automatically cause downloading
rather than viewing.  When downloading, the file selection pop-up will
appear, pre-loaded with the file name (or ``{\vt http\_return}'' if
the name is not known) in the current directory.  One can change the
saved name and the directory of the file to be downloaded.  Pressing
the {\cb Download} button will start downloading.  A pop-up will
appear that monitors the transfer, which can be aborted with the {\cb
Cancel} button.

% helpview 072014
\subsection{The HTML Viewer}
\label{helpview}

\index{help viewer!back}
\index{help viewer!forward}
\index{help viewer!stop}
The help viewer windows provide access to the help system topics, and
can display general HTML and image files.

There are three colored buttons in the menu bar of the viewer.  The
left-facing arrow button (back) will return to the previous topic
shown in the window.  The right-facing arrow button (forward) will
advance to the next topic, if the back button has been used.  The {\cb
Stop} button will stop HTTP transfers in progress.

There are four drop-down menus in the menu bar:  {\cb File}, which
contains basic commands for loading and printing, {\cb Options}, which
contains commands for setting display attributes, {\cb Bookmarks},
which allows saving frequently used references, and {\cb Help} which
provides documentation.

The {\cb File} menu contains the following command buttons.
\begin{description}
\index{help viewer!Open}
\index{help viewer!Open File}
\item{\cb Open}\\
The {\cb Open} button in the {\cb File} menu pops up a dialog into
which a new keyword, URL, or file name can be entered.

\item{\cb Open File}\\
The {\cb Open File} button brings up the {\cb File Selection} panel. 
The {\cb Ok} button (green octagon) on the {\cb File Selection} panel
will load the selected file into the viewer (the file should be a
viewable file).  The file can also be dragged into the viewer from the
{\cb File Selection} panel.

\index{help viewer!Save}
\index{save help text}
\item{\cb Save}\\
The {\cb Save} button in the {\cb File} menu allows the text of the
current window to be saved in a file.  This functionality is also
provided by the {\cb Print} button.  The saved text is pure ASCII.

\index{help viewer!Print}
\index{print help text}
\item{\cb Print}\\
The {\cb Print} button brings up a pop-up which allows the user to
send the help text to a printer, or to a file.  The format of the text
is set by the drop-down menu, with the current setting indicated on
the menu button.  The choices are PostScript in four fonts (Times,
Helvetica, New Century Schoolbook, and Lucida Bright), HTML, or plain
text.  If the {\cb To File} button is active, output goes to that
file, otherwise the command string is executed to send output to a
printer.  If the characters ``{\vt \%s}'' appear in the command
string, they are replaced with the temporary print file name,
otherwise the temporary file name is appended to the string, separated
by a space character.

\index{help viewer!Reload}
\item{\cb Reload}\\
The {\cb Reload} button in the {\cb File} menu will re-read the input
file and redisplay the contents.  This can be useful when writing new
help text or HTML files, as it will show changes made to the input
file.  However, if you edit a ``{\vt .hlp}'' file, the internally
cached offsets for the topics below the editing point will be wrong,
and will not display correctly.  When developing a help text topic,
placing it in a separate file will avoid this problem.  One can also
use the {\cb !helpreset} command to update the file offset table.  If
the displayed object is a web page, the page will be redisplayed from
the disk cache if it is enabled, rather than being downloaded again.

\index{help viewer!Old Charset}
\item{\cb Old Charset}\\
The help viewer uses the UTF-8 character set, which is the current
standard international character set.  However, older input sources
may assume another character set, such as ISO-8859, that will display
some characters incorrectly.  If the user observes that some
characters are missing or wrong in the display, setting this mode
might help.

\index{help viewer!Make FIFO}
\item{\cb Make FIFO}\\
This controls an obscure but unique feature.  When the button is
pressed, a named pipe, or FIFO, is created in the user's home
directory.  The name is ``{\vt mozyfifo}'', or if this name is in use,
an integer suffix is added to make the name unique.  This is a special
type of file, that has the property in this case that text written to
this ``file'' will be parsed and displayed on the viewer screen.

The feature was developed for use in the stand-alone {\vt mozy}
program, for use as a HTML viewer for the {\vt mutt} mail client.  If
an HTML MIME attachment is ``saved'' to the FIFO file, it will be
displayed in the viewer.

The FIFO will be destroyed if this toggle button is pressed a second
time, or when the help window exist normally.  If the program crashes,
the FIFO may be left behind and require manual removal.

\index{help viewer!Quit}
\item{\cb Quit}\\
The {\cb Quit} button in the {\cb File} menu removes the help window. 
This will exit help mode (where clicking on a command button brings up
help) if there are no other help windows visible.  Pressing the {\cb
Help} button in the {\cb Help Menu} a second time or pressing the {\kb
Esc} key also exits help mode, though the help windows remain visible.
\end{description}

The {\cb Options} menu presents a number of configuration and visual
attribute choices to the user.  These are described below.
\begin{description}

\index{help viewer!Save Config}
\index{help viewer!.mozyrc file}
\item{\cb Save Config}\\
The {\cb Save Config} button in the {\cb Options} will save a
configuration file in the user's home directory, named ``{\vt
.mozyrc}''.  This file is read whenever a new help window appears, and
sets various parameters, defaults, etc.  This provides persistence of
the options selected in the {\cb Options} menu.  Without an existing 
{\vt .mozyrc} file, changes are discarded.  If the file exists, it
will be updated whenever a help window is dismissed.

\index{help viewer!Search Database}
\index{search help database}
\item{\cb Search Database}\\
The {\cb Search Database} button in the {\cb Options} menu brings up a
dialog which solicits a regular expression to use as a search key into
the help database.  The regular expression syntax follows POSIX 1003.2
extended format (roughly that used by the Unix {\vt egrep} command). 
The search is case-insensitive.  When the search is complete, a new
display appears, with the database entries which contained a match
listed in the ``References'' field.  The library functions which
implement the regular expression evaluation differ slightly between
systems.  Further information can be found in the Unix manual pages
for ``regex''.

\index{help viewer!Find Text}
\item{\cb Find Text}\\
The {\cb Find Text} command enables searching for text in the window. 
A dialog window appears, into which a regular expression is entered. 
Text matching the regular expression, if any, is selected and scrolled
into view, on pressing one of the blue up/down arrow buttons.  The
down arrow searches from the text shown at the top of the window to
the end of the document, and will highlight the first match found, and
bring it into view if necessary.  The up button will search the text
starting with that shown at the bottom of the window to the start of
the document, in reverse order.  Similarly, it will highlight and
possibly scroll to the first match found.  The buttons can be pressed
repeatedly to visit all matches.

\index{help viewer!Default Colors}
\item{\cb Default Colors}\\
The {\cb Default Colors} button in the {\cb Options} menu brings up
the {\cb Default Colors} panel, from which the default colors used in
the display may be set.  The entries provide defaults which are used
when the document being displayed does not provide alternative values
(in a {\vt <body>} tag).  The defaults apply in general to help text.

The color entries can take a color name, as listed in the listing
brought up with the {\cb Colors} button, or a numerical RGB entry in
any common format.  The entries are the following:

\begin{description}
\item{\cb Background color}\\
Set the default background color used.

\item{\cb Background image}\\
If set to a path to an image file in any standard image format, the
image is used to tile the background.

\item{\cb Text color}\\
The default color to use for text.

\item{\cb Link color}\\
The default color to use for un-visited links.

\item{\cb Visited link color}\\
The default color to use for visited links.

\item{\cb Activated link color}\\
The default color to use for a link over which the user presses a
mouse button.

\item{\cb Select color}\\
The color to use as the background of selected text.  This color can
not be set from the document.

\item{\cb Imagemap border color}\\
The color to use for the border drawn around imagemaps.  This color
can not be set from the document.
\end{description}

The {\cb Colors} button brings up a panel which lists available named
colors.  Clicking on a name in this panel selects it, and enters the
name into the system clipboard.  The ``paste'' operation can then be
used to enter the color name into an entry area.  This may vary
between systems, typically clicking on an entry area with the middle
mouse button will paste text from the clipboard.

Pressing the {\cb Apply} button will apply the new colors to the
viewer window.  Pressing {\cb Dismiss} or otherwise retiring the panel
without pressing {\cb Apply} will discard changes.  Changes made will
{\bf not} be persistent unless the {\cb Save Config} button has been
used to create a {\vt .mozyrc} file, as mentioned above.

\index{help viewer!Set Font}
\item{\cb Set Font}\\
The {\cb Set Font} button in the {\cb Options} menu will bring up a
font selection pop-up.  One can choose a typeface from among those
listed in the left panel.  The base size can be selected in the right
panel.  There are two separate font families used by the viewer:  the
normal, proportional-spaced font, and a fixed-pitch font for
preformatted and ``typewriter'' text.  Pressing {\cb Apply} will set
the currently selected font.  The display will be redrawn using the
new font.

In {\Xic}, there are commands to set the font families:
\begin{quote}\vt
!helpfixed [{\it family-size\/}]\\
!helpfont [{\it family-size}]
\end{quote}
The format of the {\it family-size} argument depends upon the version
of the GTK toolkit employed.

\index{help viewer!disk cache}
\item{\cb Cache} group\\
A disk cache of downloaded pages and images is maintained.  The cache
is located in the user's home directory under a subdirectory named
``{\vt .wr\_cache}''.  The cache files are named ``{\vt wr\_cache}{\it
N}''" where {\it N} is an integer.  A file named ``{\vt directory}''
in this directory contains a human-readable listing of the cache files
and the original URLs.  The listing consists of a line with internal
data, followed by data for the cache files.  Each such line has three
columns.  The first column indicates the file number {\it N}.  The
second column is 0 if the {\vt wr\_cache}{\it N} file exists and is
complete, 1 otherwise.  The third column is the source URL for the
file.  The number of files saved is limited, defaulting to 64.  The
cache only pertains to files obtained through HTTP transfer.  This
directory may also contain a file named ``{\vt cookies}'' which
contains a list of cookies received from web sites. 

A page will not be downloaded if it exists in the cache, unless the
modification time of the page is newer than the modification time of
the cache file.

\index{help viewer!Don't Cache}
The {\cb Don't Cache} button in the {\cb Options} menu will disable
caching of downloaded pages and images.

\index{help viewer!Clear Cache}
The {\cb Clear Cache} button in the {\cb Options} menu will clear the
internal references to the cache.  The files, however, are not cleared.

\index{help viewer!Reload Cache}
The {\cb Reload Cache} button in the {\cb Options} menu will clear and
reload the internal cache references from the files that presently
exist in the cache directory.

\index{help viewer!Show Cache}
The {\cb Show Cache} button in the {\cb Options} menu brings up a
listing of the URLs in the internal cache.  Clicking on one of the
URLs in the listing will load that page or image into the viewer. 
This is particularly useful on a system that is not continuously
on-line.  One can access the pages while on-line, then read them
later, from cache, without being on-line.

\index{help viewer!cookies}
\index{help viewer!No Cookies}
\item{\cb No Cookies}\\
Support is provided for Netscape-style cookies.  Cookies are small
fragments if information stored by the browser and transmitted to or
received from the web site.  The {\cb No Cookies} button in the {\cb
Options} menu will disable sending and receiving cookies.  With
cookies, it is possible to view certain web sites that require
registration (for example).  It is also possible to view some commerce
sites that require cookies.  There is no encryption, so it is not a
good idea to send sensitive information such as credit card numbers.

\index{help viewer!image formats}
\index{help viewer!No Images}
\index{help viewer!Sync Images}
\index{help viewer!Delayed Images}
\index{help viewer!Progressive Images}
\item{\cb Images} group\\
Image support is provided for gif, jpeg, png, tiff, xbm, and xpm. 
Animated gifs are supported as well.  Images found on the local file
system are always displayed immediately (unless debugging options are
set in the startup file).  The treatment of images that must be
downloaded is set by this button group in the {\cb Options} menu.  One
and only one of these choices is active.  If {\cb No Images} is
chosen, images that aren't local will not be displayed at all.  If
{\cb Sync Images} is chosen, images are downloaded as they are
encountered.  All downloading will be complete before the page is
displayed.  If {\cb Delayed Images} is chosen, images are downloaded
after the page is displayed.  The display will be updated as the
images are received.  If {\cb Progressive Images} is chosen, images
are downloaded after the page is displayed, and images are displayed
in sections as downloading progresses.

\index{help viewer!anchor styles}
\index{help viewer!Anchor Plain}
\index{help viewer!Anchor Buttons}
\index{help viewer!Anchor Underline}
\index{help viewer!Anchor Underline}
\index{help viewer!Anchor Highlight}
\item{\cb Anchor} group\\
There are choices as to how anchors (the clickable references) are
displayed.  If the {\cb Anchor Plain} button in the {\cb Options} menu
is selected, anchors will be displayed with standard blue text.  If
{\cb Anchor Buttons} is selected, a button metaphor will be used to
display the anchors.  If {\cb Anchor Underline} is selected, the
anchor will consist of underlined blue text.  The underlining style
can be changed in the ``{\vt mozyrc}'' startup file.  One and only
one of these three choices is active.  In addition, if {\cb Anchor
Highlight} is selected, the anchors are highlighted when the pointer
passes over them.

\index{help viewer!Bad HTML Warnings}
\item{\cb Bad HTML Warnings}\\
If the {\cb Bad HTML Warnings} button in the {\cb Options} menu is
active, messages about incorrect HTML format are emitted to standard
output.

\index{help viewer!Freeze Animations}
\item{\cb Freeze Animations}\\
If the {\cb Freeze Animations} button in the {\cb Options} menu is
active, active animations are frozen at the current frame.  New
animations will stop after the first frame is shown.  This is for
users who find animations distracting.

\index{help viewer!Log Transactions}
\item{\cb Log Transactions}\\
If the {\cb Log Transactions} button in the {\cb Options} menu is
active, the header text emitted and received during HTTP transactions
is printed on the terminal screen.  This is for debugging and hacking.
\end{description}

The {\cb Bookmarks} menu contains entries to add and delete entries,
plus a list of entries.  The entries, previously added by the user,
are help keywords, file names, or URLs that can be accessed by
selecting the entry.  Thus, frequently accessed pages can be saved for
convenient access.  Pressing the {\cb Add} button will add the page
currently displayed in the viewer to the list.  The next time the {\cb
Bookmarks} menu is displayed, the topic should appear in the menu.  To
remove a topic, the {\cb Delete} button is pressed.  Then, the menu is
brought up again, and the item to delete is selected.  This will
remove the item from the menu.  Selecting any of the other items in
the menu will display the item in the viewer.  The bookmark entries
are saved in a file named ``{\vt bookmarks}'' which is located in the
same directory containing the cache files.

% helpdb 031413
\subsection{The Help Database}

\index{help database}
The help system uses a fast hashed lookup table containing cached file
offsets to the entry text.  A modular database provides flexibility
and portability.  The files are located by default in the directories
named ``{\vt help}'' under the library tree, which is usually rooted
at {\vt /usr/local/xictools}.  {\Xic} and {\WRspice} allow the
user to specify the help search path through environment variables
and/or startup files.  All of the files with suffix ``{\vt .hlp}'' in
the directories along the help search path are parsed, and reference
pointers added to the internal list, the first time the help command
is issued in the application.  In addition, other types of files, such
as image files, which are referenced in the HTML help text may be
present as well.

The help search path can be set in the environment with the variable
{\et XIC\_HLP\_PATH}, and/or may be set in the technology file.  The
information on a given keyword can be accessed at any time using the
``shell escape'' command ``{\cb !help} {\it keyword}'' in the prompt
window.

The ``{\vt .hlp}'' files have a simple format allowing users to create
and modify them.  Each help item is indexed by a keyword which should
be unique in the database.  The help text may be in HTML or plain text
format.  The file format is described in \ref{helpfiles}.

% helpforms 031413
\subsection{Help System Forms Processing}

There exists basic support for HTML forms.  In {\Xic}, HTML
forms can be used as input sources for scripts.  More information
is available in \ref{htmlforms}.

When the form ``Submit'' button is pressed, a temporary file is
created which contains the form output data.  The file consists of
key/value pairs in the following formats:
\begin{quote}
{\it name\/}={\it single\_token}\\
{\it name\/}="{\it any text\/}''
\end{quote}
There is no white space around `=', and text containing white space is
double-quoted.  Each assignment is on a separate line.

The action string from the ``{\vt <form ...>}'' tag determines how
this file is used.  The file is a temporary file, and is deleted
immediately after use.  If the action string is in the form ``{\vt
action\_local\_}{\it xxxx\/}'', then the form data are processed
internally.

If the full path for the action string begins with ``{\vt http://}''
or ``{\vt ftp://}'', then the form data are encoded into a query
string and sent to the location (though it is likely an error for
ftp).  Otherwise, the file will processed locally.  This enables the
output from the form to be processed by a local shell script or
program, which can be very useful.  The command given as the action
string is given the file contents as standard input.  The command
standard output will appear in the HTML viewer window.  Thus, one can
create HTML form front-ends for favorite shell commands and programs.

% helpinit 031413
\subsection{Help System Initialization File}
\label{mozyrcfile}

When a help window pops up, an initialization file is read, if it
exists.  This file is named ``{\vt .mozyrc}'' and is sought in the
user's home directory.  This file is not created automatically, but is
created or overwritten with the {\cb Save Config} button in the {\cb
Options} menu of a help window.  This need be done once only.  It
should be done if a {\vt .mozyrc} file exists, but it is from a
release branch earlier than 3.3.  Once a {\vt .mozyrc} file exists,
it will be updated when leaving help, reflecting any setting changes.

Incidently ``mozy'' is the name of the stand-alone version of the HTML
viewer/web browser available on the Whiteley Research web site.


% -----------------------------------------------------------------------------
% xic:multw 062313
\section{The {\cb Multi-Window} Button: Set Multi-Window Help Mode}
\index{Multi-Window Mode button}
When the {\cb Multi-Window Mode} button in the {\cb Help Menu} is set,
in help mode, clicking on a menu item or screen object will pop up a
new help window, rather than reusing a single existing window.

This menu item tracks the state of the {\et HelpMultiWin} variable.


% -----------------------------------------------------------------------------
% xic:about 062008
\section{The {\cb About} Button: Program and Legal Info}
\index{About button}
The {\cb About} button in the {\cb Help Menu} brings up a text window
which provides the {\Xic} revision number and legal information.  This
window also appears when the key sequence {\kb Ctrl-v} is entered,
with the pointer in a drawing window.


% -----------------------------------------------------------------------------
% xic:notes 062008
\section{The {\cb Release Notes} Button: View Release Notes}
\index{Release Notes button}
The {\cb Release Notes} button in the {\cb Help Menu} brings up a text
browser window loaded with the release notes for the current {\Xic}
release.

The release notes are installed by default in the directory {\vt
/usr/local/xictools/xic/docs}, and {\Xic} searches this
directory for the notes.  {\Xic} can be directed to look in a
different directory in two ways.  First, the environment variable {\et
XIC\_DOCS\_DIR} can be set to the directory to search.  Second, the
variable {\et DocsDir} can be set (with the {\cb !set} command) to the
directory to search.  The release notes describe bugs fixed and new
features added to {\Xic}, and should be read after a new release is
installed.  Also, they serve as supplements to the manual between
printings.  By policy, all updated information contained in the
release note is incorporated into the help database for a given
release.


% -----------------------------------------------------------------------------
% xic:logs 110413
\section{The {\cb Log Files} Button: Access Log Files}
\index{Log Files button}
The {\cb Log Files} button in the {\cb Help Menu} brings up the {\cb
File Selection} panel pointing at the directory containing the log
files.  "Opening" one of the entries will bring up the {\cb File
Browser} loaded with the selected file.

The log files are kept in a temporary directory which is created when
{\Xic} is started.  On normal exit, this directory is deleted, so if
the user wishes to retain one or more of the log files, the files must
be copied to a safe place.  If {\Xic} terminates unexpectedly, the
directory is retained, and therefor the files are available for
post-mortem debugging.


% -----------------------------------------------------------------------------
% xic:dblog 110613
\section{The {\cb Logging} Button: Set Logging and Debugging Options}
\index{Logging button}
This {\cb Logging} button in the {\cb Help Menu} brings up the {\cb
Logging Options} panel, from which various logging and debugging
options can be set.  Probably, there is not much here that would be of
interest to most users.  Some users may find this useful for
diagnosing problems, however.

The top half of the panel contains a number of check boxes, each with
a description.  Checking these boxes enables a debugging mode for the
described subsystem or feature.  This may involve additional
consistency testing and messages.  By default, these messages will go
to the console window, unless a path to a file is entered into the
{\cb Message file} entry area, in which case messages will be saved in
that file.
 
The bottom half of the panel enables logging output from the indicated
subsystems, into the file whose name is given.  These files will be
created in the log files area, which is a temporary directory that is
removed on normal program exit.  The files in the log files area can
be accessed with the {\cb Log Files} button in the {\cb Help Menu}.
 
This panel can also be brought up with the (undocumented) {\cb !debug}
command.

